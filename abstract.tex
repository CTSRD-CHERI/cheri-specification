\section*{Abstract}

This technical report describes CHERI ISAv9, the ninth version of the
CHERI architecture
being developed by SRI International and the University of Cambridge.
This design captures ten years of research, development, experimentation,
refinement, formal analysis, and validation through hardware and software
implementation.

CHERI introduces an architecture-neutral capability-based protection
model, which has been instantiated in various commodity base architectures
to give CHERI-MIPS,
CHERI-RISC-V, Arm's prototype Morello architecture, and (sketched)
CHERI-x86-64.
It enables software
to efficiently implement fine-grained memory protection and scalable software
compartmentalization, by providing strong, non-probabilistic, efficient
mechanisms to support the principles of least privilege and intentional use
in the execution of software at multiple levels of abstraction, preventing and
mitigating vulnerabilities.
Design goals include incremental adoptability from current ISAs and
software stacks, low performance overhead for memory protection, significant
performance improvements for software compartmentalization, formal grounding,
and programmer-friendly underpinnings.

CHERI blends traditional paged virtual memory with an
in-address-space capability model that includes capability values in registers,
capability instructions, and tagged memory to enforce capability integrity.
This hybrid approach, inspired by the Capsicum security model, addresses the performance and robustness issues that arise
when trying to express more secure programming models, minimising
privilege, above conventional architectures that provide only
MMU-based protection.
CHERI builds on the C-language fat-pointer literature: its capabilities can
describe fine-grained regions of memory, and can be substituted for data or
code pointers in generated code, protecting data and improving
control-flow robustness.
Strong capability integrity and monotonicity properties allow CHERI
to express a variety of protection idioms, from enforcing valid C-language
pointer provenance and bounds checking to implementing the isolation and
controlled communication structures required for software
compartmentalization.

CHERI's hybrid approach allows incremental adoption of capability-oriented design:
critical components can be ported and recompiled to use capabilities throughout,
providing fine-grain memory protection, or be largely unmodified but encapsulated in ways that permit
only controlled interaction. 
Potential early deployment scenarios include low-level software Trusted Computing
Bases (TCBs) such as separation kernels, hypervisors, and operating-system
kernels,  userspace TCBs such as language runtimes and web browsers, 
and particularly high-risk
software libraries such as data compression, protocol parsing, and image
processing (which are concentrations of both complex and historically
vulnerability-prone code exposed to untrustworthy data sources).

CHERI ISAv9 is a substantial enhancement to prior ISA versions.
%Capability compression is now part of the abstract model.
%Both 32-bit and 64-bit architectural address sizes are supported.
%Various previously experimental features, such as sentry capabilities and
%CHERI-RISC-V, are now considered mature.
%We have defined a number of new temporal memory-safety acceleration features
%including MMU assistance for a load-side-barrier revocation model.
%We have added a chapter on practical CHERI microarchitecture.
%CHERI ISAv8 is synchronized with Arm Morello.
