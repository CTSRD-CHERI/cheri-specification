\chapter{Experimental Features and Instructions}
\label{app:experimental}

This appendix describes experimental features and instructions proposed for
possible inclusion in later versions of the CHERI ISA.
These items for consideration include optimizations, new permissions, new
compression formats, and overhauls of existing CHERI mechanisms.
Some are relatively mature, and we anticipate their achieving a
non-experimental status in the next version of the CHERI ISA specification
(e.g., capability flags and temporal memory safety).
Others arose as part of our more general design-space exploration, and we
document these alternative approaches (e.g., indirect capabilities) or
potential future avenues of investigation (e.g., linear capabilities).
We present them here in roughly increasing order of complexity.
The body of the appendix describes the rationale and approach for each
experimental feature.
% ; specific instruction encodings and semantics may be
% found in Section~\ref{app:exp:insns}.
\pgnnote{roughly increasing order of complexity?
  might they instead be grouped
  more structurally as major sections, and then roughly increasing
  order within those major sections?}

% >>>
\section{Additional Architectural Assistance For Revocation} % <<<

\subsection{CClearTags} % <<<
\insnriscvlabel{ccleartags} % silence undefined hyperref warning until this instruction is defined.

Typically, allocators allow for the return of ``uninitialized'' memory
(e.g., \texttt{malloc} vs.\@ \texttt{calloc}).  In the context of temporal
safety, this proves to be problematic unless the allocator is type- and
use-specicialized: data and pointers may unintentionally flow into the
possession of the holder of a new allocation, violating both confidentality
and, in a non-architectural sense, provenance integrity.  Most treatments of
security-conscious allocators therefore always zero memory.  CHERI can
directly probe at the difference between confidentality of data and
provenance integrity of pointers.  Towards this end, we introduce a
\insnref{CClearTags} instruction, which permits bulk zeroing of
capabilities within a cacheline (i.e., at the same granularity as
\insnref{CLoadTags}).

While \insnref{CClearTags} may be a useful alternative to
\texttt{bzero}-ing memory when confidentiality is not required, it has at
least one additional use in the context of revocation specifically.
\insnref{CClearTags} should accelerate sweeping by allowing allocators
to optimistically, even if not perfectly, remove capabilities within freed
regions, removing comparatively expensive look-aside checks of validity
during sweeps, at a lower cost than zeroing.

\insnref{CClearTags} is intended to be \emph{non-allocating} in the caches,
as with its counterpart \insnref{CLoadTags}.  If a cacheline is not present
in the cache fabric, the induced store should always transit to the tag cache
rather than pulling the line data into the caches.

\nwfnote{We could also introduce a \insnnoref{CAndTags} instruction
that took a bitmask of which entries in the line to zero; this would make it
useful for small allocations, but it's not clear that that's actually
worthwhile.  Small allocations seem like they might be likely to be in the
cache already.}

% >>>
\subsection{Non-Temporal (Streaming) CLC} % <<<
\label{app:exp:clcnt}
\insnriscvlabel{clcnt} % silence undefined hyperref warning until this instruction is defined.

During revocation, whenever a capability is identified via
\insnref{CLoadTags}, it is fetched from memory for analysis.  A large
fraction of these capabilities are expected to be valid, and so will not
cause additional activity within their cache line.%
%
\footnote{Those capabilities that are found to be revoked are then subjected
to a \insnnoref{CLR.C}/\insnnoref{CSC.C} sequence for atomic replacement
with their revoked image.}
%
Being able to hint to the cache that the revoker is \emph{streaming} through
memory, and so should not pave over the caches, seems like a worth-while
objective.  We therefore introduce a non-temporal, streaming
\insnref{CLC} analogue, \insnref{CLCNT} with architectural semantics
exactly matching those of \insnref{CLC}, but as a separate opcode to
hint to the microarchitecture.

While one implementation of such a streaming \insnref{CLC} would be to never promote
lines in the cache hierarchy (i.e., leave them in place, in DRAM or the LLC
when loading), analysis done as part of the Efficient Tagged Memory \cite[\S
VI.B]{joannou2017:tagged-memory} suggests that if a cacheline has one
capability, it is relatively likely to have another.%
%
\footnote{Specifically, the ratio of the probability of a capability being
found in memory with a ``grouping factor 8'' (corresponding to our FPGA's
caches for CHERI Concentrate capabilities) to an independent sampling of
eight ``grouping factor 1'' binomials (analysing on a word-by-word basis) is
between $0.97$ and $0.32$, implying clustering of capabilities.}
%
This suggests that \insnref{CLCNT} should be caching, but restrictedly
so.  Na\"ively, we suggest that ensuring misses usually allocate below some
top $k$ lines of the MRU queue for the cache ``way'' activated will result
in it being evicted relatively soon and without introducing too much
contention with the application.  When a \insnref{CLCNT} hits in the
cache, it should not trigger promotion to the front of the MRU queue.
Whether this policy is a good one, and, if so, what the correct values of
$k$ and ``usually'' are, remain open questions.

% >>>

\nwfnote{We may also be able to offer stronger guarantees to lock-free data
structures by adding a ``conditional CLC'' instruction which gates its operation
on the LL/SC link flag.  In some toy examples, this seems to make full SMR
hazards unnecessary.  Unclear that it is worth pursuing.}

% >>>
\section{Recursive Mutable Load Permission} % <<<
\label{app:exp:recmutload}

\makecapperm{RML}{Recursive\_Mutable\_Load}

Several software capability systems have exploited the use of immutable data
structured to facilitate safe sharing (e.g., Joe-E~\cite{mettler:joee}).
CHERI capabilities can provide references through which stores are not
permitted; however, because they can be refined and distributed throughout the
system, simply holding a read-only reference is not sufficient to allow a
consumer to ensure that no simultaneous access can occur to the same memory
via another capability.
Further, passing a read-only reference to memory does not ensure that further
loads of capabilities from within that memory provide only read-only access to
`deep' data structures -- e.g., linked lists.

Various software-level invariants could be used to improve confidence for both
callers and callees.
For example, the software runtime might make use of read-only MMU mappings for
immutable data, and provide capabilities that clearly provide an indication
that they refer to those read-only mappings -- e.g., via use of a
software-defined permission bit set only for such references, via use of
reserved portions of the address space, sealed via a certain type, or
checkable via a dynamic service operating in a trustworthy protection domain.
In addition, memory could be allocated as mutable and its MMU mapping later
modified to `freeze' the contents, or by performing a revocation-like sweep
to convert any extant store-enabled capabilities into load-only capabilities.

However, providing strong architectural invariants to software offers
significant value.  One idea we have considered is a new permission,
\cappermRML, which if not present, clears store
permissions and the recursive mutable load permission, on any capability
loaded via a capability with this permission present.%
%
\footnote{The concept of such \emph{transitively} read-only capabilities
appears to have been first developed in KeyKOS, where such capabilities were
termed `sensory keys'~\cite{hardy:keykos}.  While sensory keys were
necessarily read-only, the descendent notion of the `weak' access modifier
in EROS could be applied to both read and write operations.  When modifying
reads, it behaves as described so far; attempts to store some input
capability through a weak write-permitting capability resulted in a
weakened version of the input capability being stored~\cite{shapiro:eros}.
In the successor system Coyotos, `weak' was once again made to imply
read-only access~\cite{doerrie2015:confinement,shapiro:coyotosspec}.}
%
A module may clear the store permissions and also clear
\cappermRML on a capability before passing it to another
module.  Having done so, the originator is guaranteed that this passed
capability could not then be used to mutate memory it directly describes
(lacking store permissions) or memory transitively referenced therefrom, even
if the latter capabilities, authorizing transitive access, bear some store
permissions.
%
This would not prevent temporal vulnerabilities
associated with reallocation of the memory; subject to other invariants
and safety properties, it might make it easier to construct safe references.
In particular, this mechanism is likely to be of great utility to systems
wishing to enforce the `*-property' (`no write down') of the model of
Bell and La Padula~\cite{B+LP76}.%
%
\footnote{Readers may be familiar with the infamous proof of
Boebert~\cite{boebert:inabilitystar} that ``an unmodified capability machine''
is unable to enforce this property.  As CHERI distinguishes between
capabilities and data, the proof is not directly
applicable~\cite{miller:capmyths}, and, indeed, one could imagine using trusted
intermediate software to emulate the effects of
\cappermRML, as proposed by
Miller~\cite{miller:paradigmregained}.  Despite that,
\cappermRML is still of practical utility, as it is a
light-weight, architecturally enforceable mechanism that avoids indirection.}

\paragraph{Interaction With Sealed Capabilities}
%
A question arises about loads of \emph{sealed} capabilities bearing (for
example) \cappermS through a capability lacking \cappermRML: in some sense,
this sealed capability is authorizing mutation; are we to clear its \cappermS,
despite the seal?  We view the immutability of sealed capabilities as taking
precence, and so preserve \cappermS under seal even in this scenario.  Beyond
aesthetics, we conjecture that this interpretation is convenient for software:
\cappermRML can be cleared to create read-only collections of sealed handles to
software objects.%
%
\footnote{By way of example, software can create an immutable collection $c$ of
\emph{arbitrary} unsealed capabilities by sealing each capability logically
held in $c$ to a type with an \emph{ambiently available} unsealing right and
referencing $c$ itself through \cappermRML-clear capabilities.  A layer of
indirection suffices to permit arbitrary \emph{sealed} capabilities within $c$
as well.  Software can restrict this model by using compartmentalized unsealers
rather than ambient authority.}
%
Nevertheless, software must be aware that mutation authority under seal is not
stripped by \cappermRML.

\let\cappermRML\undefined

% >>>
\section{Hierarchical Revocation From Ephemeral Capabilities} % <<<
\label{app:exp:hierarchal-evocation}

The ``revocation'' work atop CHERI to date has been about revoking all
(non-TCB) access to resources (usually, virtual address space).  However,
``Capability Revocation'' more typically means the ability for any agent,
which has delegated access to some resource, to revoke some (or all) of its
delegations while retaining access and the ability to further delegate
access \cite{Redell74}.
%
Efforts to develop such capacity atop CHERI would likely rely upon sealed
capabilities or domain transitions (with or without exceptions) to mediate
delegation, as these are the most apparent mechanisms available to us to
prevent unchecked duplication of usable authority.  However, these come with
somewhat large costs: domain transitions impose cycle overheads, and either
strategy would seem to require that the set of \emph{operations} on the
delegated resource be fixed by its constructor.  For example, the original
source of a revokably delegated data resource would likely have to provide
specialized \texttt{memcpy} implementations for moving data out of or in to
the resource's memory.
In this section, we explore the implementation of directly-accessible
revokable delegation assuming the existence of \emph{ephemeral}
capabilities, which cannot be stored to memory once loaded into a register
file.

\subsection{Ephemeral Capabilities} % <<<

The basic primitive of an ephemeral capability is a degenerate
generalization of the \cappermG* / \cappermSLC* mechanism of CHERI (recall
\cref{sect:capability-permission-bits}) and the ``capability coloring''
proposal of \cref{sec:compactcolors}.  Ephemeral capabilities are
constructed via a new \emph{store} instruction,
\insnnoref{CStoreEphemeral} which, given a non-ephemeral capability
in a register, places an ephemeral version in memory, which may thereafter
be loaded with \insnref{CLC}.%
%
\footnote{We propose the use of a reserved \cotype{} value, rather than a new
permission bit, to designate capabilities of this form (but without
interpreting this value as \emph{sealed}).  As we intend these non-storable
capabilities to be ephemeral, there will not be enough of them present in
the system at any moment to justify the use of a permission bit.  Moreover,
while the use of an \cotype{} means that we cannot seal these non-storable
capabilities, even in registers, this does not seem to be a loss.}
%
If an attempt is made to store, via \insnref{CSC}, an ephemeral
capability to memory, an un-tagged version is stored instead.  Thus, once
loaded, an ephemeral capability is confined to the register file and even a
context switch will destroy it.  Similarly, a (transitive) callee's attempt
to spill such a capability to the stack will instead detag it, and so these
ephemeral capabilities must be considered lost across general procedure
calls.  (While ephemeral capabilities can still be passed in registers as
arguments, in general, we suggest passing a non-ephemeral capability to the
ephemeral one instead.)

Software consuming these ephemeral capabilities must be prepared to deal
with revocation and the \emph{appearance} of revocation, by attempting to
reload a tagged ephemeral capability from memory.  Software must be careful
to preserve access to the memory locus of the ephemeral capability loaded to
the register file.  Operations done against such revokably delegated
resources should be idempotent (which may just mean that they are precisely
resumable).%
%
\footnote{It is possible to imagine architectural assistance beyond trapping
on untagged capabilities, should the ``trap-and-reload'' approach sketched
here be unduely onerous.}

% >>>
\subsection{Revocation} % <<<

Armed with such a primitive capability form, resource revocation still needs
to remove (a hierarchy of) ephemeral capabilities from memory, but the
issuing authority has ensured that access to the delegated resource has not
spread unchecked in memory.  Having removed all the targeted ephemeral
capabilities from memory, a single context switch on all cores suffices to
ensure that there is no retained access.  These context switches may be
actively driven (with IPIs) or passively observed (as with epoch-revocation
schemes).  We propose that a \emph{compartment} within the TCB oversee the
construction and (hierarchical) delegation of revokable delegate
intermediaries.

In order to efficiently revoke a subtree of the delegation relationship, we
will need to construct that relationship explicitly in memory in a way that
its subtrees are easily enumerated.  The design we propose herein makes
heavy use of sealed capabilities to small regions of memory, directly
storing the delegation relationship metadata with the delegated, ephemeral
capabilities themselves.  These small regions of memory are used once and so
may be reclaimed by the existing \emph{global} revocation mechanisms after
their purpose has been served.%
%
\footnote{Users of this mechanism must therefore be prepared for faults
while using the ephemeral capability as well as when attempting to reload
it.  Fortunately, this seems straightforward.}
%
Such a \emph{delegation box} contains:
%
\begin{itemize}
%
  \item An ephemeral capability to the delegated resource
%
  \item A capability to the progenitor delegation box, if any.
%
  \item A pair of capabilities forming a doubly-linked list of this box's
delegation \emph{siblings}.
%
  \item A capability to one of its child delegation boxes, if any.
%
\end{itemize}
%
Straightforward rose tree \cite{skillicorn:partreeskel} operations suffice
to maintain the hierarchical delegation structure using these boxes, and
sealing ensures that we can safely give out the rights to manipulate a box
(constructing a new child or revoking the subtree of which it is the root).
A separate grant of access directly to the ephemeral capability allows the
direct use of the delegated resource until the delegation box's revocation.

% >>>
\section{Compressed Permission Representations} % <<<
\label{app:exp:compressperm}

The model of Section~\ref{sect:capability-permission-bits} describes each
permission as a separate bit.  This has certain advantages, including the
ability to describe {\em the} all-powerful capability, a uniform
presentation, wherein the monotonic non-increase of rights is directly
encoded by the monotonic operation of bitwise \emph{and}, and a fast operational
test for a given permission.  However, in use and interpretation, the
permission bits are not orthogonal, so one could aim for a compressed
representation, freeing up bits for use as user permissions, or reserving
them for future expansion of the ISA.  We do not fully develop this story;
instead, we merely indicate examples of redundancy in the abstract model,
which may be useful to architects wishing to squeeze every last bit out of
any particular representation.

The \cappermG attribute, despite being enumerated as a permission, does not
describe permissions to the memory or objects designated by a capability.
Instead, it interacts with data storage permissions of other capabilities
(via \cappermSLC).  As such, it truly is orthogonal to
the rest of the permission bits (though it remains `monotonic' in the
sense that clearing the \cappermG permission results in a capability capable of
participating in fewer operations).

Broadly speaking, there are three spaces of identifiers described within the
CHERI capability system: virtual addresses, object types, and compartment
identifiers.  Rights concerning executability, loads, and stores apply only
to capabilities describing virtual addresses, while the rights to (un)seal
an object apply only to capabilities describing object types.  The
\cappermCid permission applies only to capabilities describing
compartment identifiers.  This permits some reduction of encoding space.

Similar reduction in encoding space may be realized if one mandates that
certain {\em user} permission bits are similarly applicable only to novel
non-architectural spaces of identifiers (e.g., UNIX file descriptors).
However, at
present we consider the sealing mechanism more useful and flexible
for the construction of such spaces of identifiers, as typically such
identifiers are ultimately given meaning by some bytes in virtual memory, to
which one may gain access by unsealing an object capability used as a
reference.%
%
\footnote{Sadly, while sealed capabilities are almost exactly what one wants
for file descriptors, because UNIX chose to type file descriptors as
\texttt{int}, the conversion to use sealed capabilities will be broadly
invasive, even if most of the changes will simply be to change the types.}
%
However, the notion of other spaces is not entirely out of the question; {\em
physical} addresses may prove to be a compelling example on some systems.

While \cappermInvoke* is {\em checked} only as part of \insnref{CInvoke}'s
operation on sealed (i.e., object) capabilities, it is inherited from these
sealed capabilities' precursors.  That is, the present CHERI architecture
permits the creation of regions of virtual address space that can be
(subdivided and) sealed, but for which these derived object capabilities are
not useful with \insnref{CInvoke} (just with \insnref{CUnseal}).
The utility of such regions is perhaps not readily apparent, but any shift
to make \cappermInvoke* apply only to object capabilities would require
modification of the \insnref{CSeal} instruction and would slightly
change the capability ontology.

Within the virtual-address-specific permissions, one finds several
opportunities for compressing representations.  First, many architectures
consider writable-and-executable to be too dangerous to permit; applying
this to CHERI's taxonomy would mean that the presence of \cappermX* implied
the absence of \cappermS*, \cappermSC*, and \cappermSLC* (see
\cref{app:exp:compressperm:wxorx}).  Further, granting \cappermLC*
effectively implies granting \cappermL*:  \insnref{CLC} and \insnnoref{CLR.C} would trap without
the latter, but more substantially, a capability load of an un-tagged (in
memory or via the paging hardware) `should' result in a load of data
transferred in to a capability register, albeit with the tag cleared.  On
the store side, \cappermSLC* implies \cappermSC*, which, in turn, implies
\cappermS*.  Taking all of these implications into consideration, one finds
that there are $15$ consistent states of the six virtual-address-space
rights (\cappermX*, \cappermL*, \cappermLC*, \cappermS*, \cappermSC*,
\cappermSLC*) considered, enabling a four-bit compressed representation.%
%
\footnote{If one restricts consideration to just the five bits of
\cappermL*, \cappermLC*, \cappermS*, \cappermSC*, and \cappermSLC*, one finds 12
valid states, requiring four bits.  A straightforward reduced encoding then
leaves \cappermL* and \cappermLC* unaltered but can use two bits to indicate
which of $\emptyset$, \{\cappermS*{}\}, \{\cappermSC*, \cappermS*{}\}, or
\{\cappermSLC*, \cappermSC*, \cappermS*{}\} is present.}

Consider the powerful \cappermASR permission.  Because this
bit is meaningful only on capabilities used as a program counter, at the
very least its presence rather directly implies \cappermX.  Moreover,
because this bit gates access to other architectural protection mechanisms,
including those, such as the paging hardware, involved in {\em interpreting}
(other) capabilities, it seems likely that this bit implies the ability to
at least read, and likely mutate (or cause the mutation of), any other
capability present in the system.  (Admittedly, perhaps the ability to
synthesize new capabilities from whole cloth would remain beyond the reach
of code executing with \cappermASR*, but given the far-reaching
powers potentially conveyed, this hardly seems worth nitpicking.)  As such,
one may be justified in considering \cappermASR* to be a single
value in one's encoding of capability permissions, rather than an orthogonal
bit.

\subsection{A Worked Example of Type Segregation} % <<<

Pushing a bit further on the `spaces of identifiers' concept above, we can
describe an alternative use of the 15 bits of {\cmuperms} available in the
128-bit encoding scheme of Section~\ref{subsec:cheri-128-implementation}.  We
continue to leave the 18-bit \cotype{} field where it stands, and we claim no
new use of any reserved bits.  Diagrams of the bit representations may be found
in Figure~\ref{fig:app:comprperm:typeseg}.

In all capabilities, we reserve three bits for uninterpreted user
permissions, and four bits for the flow control detailed in
Section~\ref{sec:compactcolors}.%
%
\footnote{Absent the use of this experimental coloring scheme, these
reserved bits can instead be used to carry the \cappermG* and \cappermSLC*
bits, with two bits remaining reserved.}
%
One more bit distinguishes between virtual-address capabilities and all
other types.  We have thus far consumed 8 of the 15 permission bits.

For virtual-address capabilities
(subsequently to be abbreviated as `VA capabilities'),
the remaining seven bits correspond
one-to-one with memory-specific permissions.  Specifically, they are:
\cappermX* (Ex), \cappermL* (L), \cappermS* (St),
\cappermLC* (LC), \cappermSC* (SC), \cappermInvoke* (I),%
%
\footnote{While any capability type can, in principle, be sealed and could be
unsealed at \insnref{CInvoke} time, \insnref{CInvoke} unseals only two
capabilities, installing them as PCC and IDC.  As such, it seems sensible to
restrict \insnref{CInvoke} to operating only on VA capabilities, and so \cappermInvoke is
defined only therein.}
%
and \cappermASR* (ASR).  We have made no effort to
eliminate redundancy in this particular segment of the encoding, but all the
observations made above about these bits continue to hold.

For non-virtual-address capabilities, we take one bit to distinguish
\emph{architectural control} capabilities from \emph{guarded-word}
capabilities.  The latter are as might be expected: they are simply bounded (as
per usual with CHERI capabilities) \emph{integers}, protected by architectural
provenance, monotonicity, and nonforgeability.  Guarded-word capabilities confer
no architectural authority, but may be of use to system software (e.g., for
describing file descriptors).  The remaining six bits are all permission-like
(and are subject to manipulation via \insnref{CAndPerm}), but are
otherwise uninterpreted by the hardware.%
%
\footnote{It may seem odd to deliberately create architecturally `useless'
tagged integers; it may seem as though they could simply be VA capabilities
with all permission bits cleared.  However, just because an agent has some
rights to memory address 0x1234 does not imply that they have rights to the
\emph{integer} 0x1234, but monotonic action on a capability authorizing the
former could result in one authorizing the latter in this hypothetical
`all-permission-bits-zero' encoding.  The \emph{separate provenance tree}
of guarded-word capabilities distinguishes these: there is no monotonic
mechanism to transmute one into the other.}

Architectural control capabilities include the ability to seal and unseal
particular object types, set the compartment identifier, and manipulate colors
(again, as detailed in Section~\ref{sec:compactcolors}).  The remaining
six bits are, again, all permission-like.  Three are reserved for future use
(not currently interpreted), while the other three correspond to the current
\cappermUnseal (U), \cappermSeal (Se), and \cappermCid (CID).  No attempt
has been made to further refine the type space, so we continue to
architecturally conflate object types and compartment identifiers and rely on
system software to maintain proper partitioning.

In this scheme, three primordial architectural roots should be created at
system reset: one for virtual addresses, one for architectural control, and
one for guarded words.  All primordial capabilities should be unsealed, have
all defined and user permission bits asserted, and cover the full space of
their respective identifiers.


\begin{figure}
\small\centering\begin{tabular}{cl}

\textbf{Type} & \textbf{Bit layout} \\

\raiseforbf{Virtual Address} &
{\begin{bytefield}[bitwidth=21pt]{15}
  \bitbox{1}{1} & \bitbox{1}{ASR} & \bitbox{1}{CC} & \bitbox{1}{SC} & \bitbox{1}{LC} & \bitbox{1}{St} & \bitbox{1}{L} & \bitbox{1}{Ex} & \bitbox{3}{user perms'3}
 \end{bytefield}} \\

\raiseforbf{Architectural Control} &
{\begin{bytefield}[bitwidth=21pt]{15}
  \bitbox{1}{0} & \bitbox{1}{1} & \bitbox{3}{\color{lightgray}\rule{\width}{\height}} & \bitbox{1}{CID} & \bitbox{1}{Se} & \bitbox{1}{U} & \bitbox{3}{user perms'3}
 \end{bytefield}} \\

\raiseforbf{Guarded word} &
{\begin{bytefield}[bitwidth=21pt]{15}
  \bitbox{1}{0} & \bitbox{1}{0} & \bitbox{9}{user perms'9}
 \end{bytefield}} \\

\end{tabular}

\caption{Bit-level representations of a type-segregated metadata-bit-packing scheme.}
\label{fig:app:comprperm:typeseg}

\end{figure}

% >>>
\subsection{Type-segregation and Multiple Sealed Forms} % <<<

\begin{figure}
\small\centering\begin{tabular}{cl}

\textbf{Type} & \textbf{Bit layout} \\

\raiseforbf{Unsealed VA} &
{\begin{bytefield}[bitwidth=21pt]{15}
  \bitbox{1}{1} & \bitbox{1}{0} & \bitbox{1}{0} & \bitbox{1}{ASR} & \bitbox{1}{CC} & \bitbox{1}{SC} & \bitbox{1}{LC} & \bitbox{1}{St} & \bitbox{1}{L} & \bitbox{1}{Ex} & \bitbox{3}{user perms'3}
 \end{bytefield}} \\

\raiseforbf{Sealed VA} &
{\begin{bytefield}[bitwidth=21pt]{15}
  \bitbox{1}{1} & \bitbox{1}{1} & \bitbox{1}{SV} & \bitbox{1}{ASR} & \bitbox{1}{CC} & \bitbox{1}{SC} & \bitbox{1}{LC} & \bitbox{1}{St} & \bitbox{1}{L} & \bitbox{1}{Ex} & \bitbox{3}{user perms'3}
 \end{bytefield}} \\

\raiseforbf{Architectural Control} &
{\begin{bytefield}[bitwidth=21pt]{15}
  \bitbox{1}{1} & \bitbox{1}{0} & \bitbox{1}{1} & \bitbox{4}{\color{lightgray}\rule{\width}{\height}} & \bitbox{1}{CID} & \bitbox{1}{Se} & \bitbox{1}{U} & \bitbox{3}{user perms'3}
 \end{bytefield}} \\

\raiseforbf{Unsealed guarded word} &
{\begin{bytefield}[bitwidth=21pt]{15}
  \bitbox{1}{0} & \bitbox{1}{0} & \bitbox{1}{0} & \bitbox{10}{user perms'10}
 \end{bytefield}} \\

\raiseforbf{Sealed guarded word} &
{\begin{bytefield}[bitwidth=21pt]{15}
  \bitbox{1}{0} & \bitbox{1}{1} & \bitbox{1}{0} & \bitbox{10}{user perms'10}
 \end{bytefield}} \\

\raiseforbf{Reserved} &
{\begin{bytefield}[bitwidth=21pt]{15}
  \bitbox{1}{0} & \bitbox{1}{\color{lightgray}\rule{\width}{\height}} & \bitbox{1}{1} & \bitbox{10}{\color{lightgray}\rule{\width}{\height}}
 \end{bytefield}} \\

\end{tabular}

\caption{A variant of packed metadata including multiple sealed forms.}
\label{fig:app:comprperm:typeseg2}

\end{figure}

\pgnnote{I would like to declare war on the gratuitous comma
`adjective, adjective noun' as in `small, sealed memory objects' --
  `small' can refer only to `memory objects', not to `sealed'}
Experiments with CheriOS have found that the increased alignment requirements
for sealed capabilities induced by the original 128-bit compressed format are
awkward (recall Section~\ref{subsec:cheri-128-implementation}).  In particular,
there is a desire to pass small sealed memory objects, with size (and so,
ideal alignment) well below the requisite alignment size for sealing.
Subsequent work has defined a different CHERI Concentrate form with a dedicated
\cotype{} field, no need of a sealed bit, and no increased alignment
requirements to make room or the \cotype{} bits.  And so, the remainder of this
subsection is largely mooted: all capabilities may be sealed in the new CHERI
Concentrate format.  We retain it in this document for interest and its
possible applicability to implementers considering different capability
encoding options.

The small objects passed by CheriOS are never sealed as interior pointers.
That is, the sealed forms are guaranteed to have offset zero (i.e., equal
cursor and base addresses).  This permits 10 bits of the B field to be
transferred to the T field, offering much smaller alignment requirements.
(Byte alignment remains possible until objects approach 1 mibibyte in length.
Offsets need not be zero, but must be small, in the sense that they must be
below $2^{\mathbf{e}}$.)  The experimental architectural encoding presently
requires stealing one of the two bits described in this document as reserved
within a capability representation.  Given the possible utility of this
additional sealed form to the other provenance trees discussed above, it seems
worthwhile to present a possible unified story.

For this example, we drop the ability to seal architectural control
capabilities, as we do not think these will be passed as tokens; instead, we
believe, should system programmers desire similar policies, they are free to
indirect, i.e., to place architectural control capabilities into small
regions of memory, seal the rights thereto, and pass that sealed capability
instead of a sealed architectural control capability.
This further removes concerns around the encoding of \cotype{}s and capability
color changing permissions (to be discussed).

This illustrative
encoding uses 17 bits: 15 from the former \cmuperms{}, 1 from the
former sealed flag, and 1 formerly reserved.  Bit-field representations are
shown in Figure~\ref{fig:app:comprperm:typeseg2}.  For VA capabilities, the new
`Sealed Variant' (SV) flag, which is not a permission bit (and so not subject
to manipulation by \insnref{CAndPerm}), distinguishes between the form with
both T and B specified and the form with only T specified.  We expect an
architecture using this form to have two \insnref{CSeal}-like instructions,
each generating one of the variants.  For sealed guarded-word capabilities, we
permit only the latter form, as we believe sealed guarded words are more likely
to be used as tokens than as regions of integers.  One-fourth of our type
encoding values are reserved for future expansion.

% >>>
\subsection{\texttt{W\textasciicircum{}X} Saves A Bit} % <<<
\label{app:exp:compressperm:wxorx}

\texttt{W\textasciicircum{}X} (`W xor X') is a shorthand for the notion
that no block of memory should be, at the same time, both writable and
executable.  Most implementations in hardware work within the MMU, and rely
on the operating system to enforce the exclusivity of write and execute
permissions.  From the view of application software, this means that a given
pointer value has additional hidden state beyond its being mapped or
unmapped.  Applications on CHERI could, instead, structure the permissions
within capabilities to enforce exclusivity of write and execute permissions,
trading the stateful MMU protection for having multiple capabilities
representing the two different rights.

Were we to push \texttt{W\textasciicircum{}X} on CHERI to an extreme,
it could become a property of the capability encoding itself and, thereby,
allow for more compact encoding of permissions.  The existing eight-bit
architectural permission field,

\begin{center}
%
{\begin{bytefield}[bitwidth=25pt]{8}
  \bitbox{1}{ASR} & \bitbox{1}{CC} & \bitbox{1}{SLC} & \bitbox{1}{SC} & \bitbox{1}{LC} & \bitbox{1}{St} & \bitbox{1}{L} & \bitbox{1}{Ex}
 \end{bytefield}}
%
\end{center}

\noindent could instead be re-coded as a 7-bit field, making the
\texttt{W\textasciicircum{}X} explicit:

\begin{center}\begin{tabular}{rl}
%
\raiseforbf{RX capability:} &
{\begin{bytefield}[bitwidth=25pt]{8}
  \bitbox{1}{0} & \bitbox{1}{CC} & \bitbox{1}{\color{lightgray}\rule{\width}{\height}} & \bitbox{1}{ASR} & \bitbox{1}{LC} & \bitbox{1}{Ex} & \bitbox{1}{L} \\
 \end{bytefield}} \\
%
\raiseforbf{RW capability:} &
{\begin{bytefield}[bitwidth=25pt]{8}
  \bitbox{1}{1} & \bitbox{1}{CC} & \bitbox{1}{SLC} & \bitbox{1}{SC} & \bitbox{1}{LC} & \bitbox{1}{St} & \bitbox{1}{L}
 \end{bytefield}}
%
\end{tabular}\end{center}

As in the type-segregation proposals, this design creates yet another split
of architectural provenance roots: there must be two capabilities present at
system startup, granting separate read-write and read-execute regions.
Similarly, a single capability then could not express the total set of
permissions that may be granted by, e.g., the *nix \texttt{mmap()} call; the
API and consumers must be revised.  (One hopes that relatively few consumers
initially request (or later transition, via \texttt{mprotect()}, to having)
both write and execute permissions.)  It is not yet clear what additional
challenges this split imposes on our goal of C compatibility.

There is some redundancy yet in this encoding, in that either RX or RW
capabilities can be monotonically turned into read-only capabilities.  One
could imagine further segregation into a
\texttt{R\textasciicircum{}W\textasciicircum{}X} taxonomy, but this seems
especially likely to complicate C compatibility.  Moreover, the obvious
utility of RW capabilities and popularity of data constants adjacent to
executable code (and thereby reachable using relative offsets from the
instruction pointer) argue for permitting read permissions in both write and
execute forms.

When and if combined with the compact coloring proposal below, the
\cappermSLC (SLC) bit and its unused slot in the RX form
would vanish.

% >>>
% >>>
\section{Memory-Capability Versioning} % <<<
\label{app:exp:versioning}

Several existing architectures have responded to temporal safety issues in
software by proposing to `version' memory, embed versions into pointers,
and require that the versions of the pointer and target match on each
dereference.  Two prominent examples are Oracle's SPARC's ADI/SSM
\cite{sparc-m7-adi} and Arm's MTE~\cite{arm-a64-v8-a-beta}.  We conjecture
that the combination of these ideas with CHERI would enhance both and continue
to have reasonable performance overheads.  Between these mechanisms, we can
offer an attractive secure mitigation of temporal safety violations in
untrusted code.

Specifically, we propose to embed a four-bit version field%
%
\footnote{There is nothing special about the value four; even a one-bit
versioning scheme has practical utility, while more bits reduce likelihood of
collision in stochastic schemes and delay revocation in deterministic schemes
(see \cref{app:exp:versioning:syssoft}).  Four simply seems to be a popularly
acceptable value.}
%
in every memory-authorizing capability, either using reserved bits or by
shrinking the address field from 64 to 60 bits.%
%
\footnote{Practically, most modern systems do not make use of their entire
64-bit virtual address space and require that all such addresses be
sign-extended values derived from (typically) 40-bit to 57-bit values,
depending on the architecture.  We can therefore repurpose some of these bits
with only modest, localized changes to system software.}
%
Further, we pair the same number of bits of version with each `granule' of
physical memory, which we suggest to be roughly 64 bytes.  (The proposed values
give a spatial overhead equivalent to CHERI's capability tags: one bit per 16
bytes.)  To ensure that untrusted code cannot inappropriately re-version memory
granules, we provide a simple model of authorization that does not require the
intervention of supervisor software.

We divide memory-authorizing capabilities into two classes, versioned and
unversioned, and introduce an instruction that derives a versioned capability
from an unversioned one.  The core of this protection mechanism is this: if a
versioned capability is used to access a granule, the access succeeds only if
(in addition to passing the existing CHERI permissions and bounds checks as
well as any MMU permissions checks) the granule and intra-capability versions
are equal.  In the case of mismatch, an implementation must, at a minimum,
cause data fetches to return $0$, capability fetches to return untagged NULLs,
stores to fail silently, and instruction fetches to trap.  To improve the
debugging experience, implementations may provide optional or mandatory traps
on these fetch and stores as well.

Only unversioned capabilities can authorize the re-versioning of memory
granules.  Additionally, unversioned capabilities authorize access regardless
of the version of the granule being accessed.  We expect that these will become
closely held within subsystems that then exchange derived versioned
capabilities with other subsystems; the canonical example is, of course, memory
allocators, which will hold unversioned capabilities internally and give out
(and take back) versioned capabilities.

Versions are `sticky,' in that any capability monotonically derived from a
versioned progenitor will have the same version.  Dually, derivations from
unversioned capabilities are unversioned, unless the version is explicitly
branded into the progeny.

\subsection{Legacy Memory Versioning Behaviors} % <<<

When adding CHERI to an architecture that already has memory versioning support
(e.g., SPARC or Arm), it may be desirable to retain compatibility with existing
mechanism in hybrid or legacy code.  That is, we may wish, assuming the system
has enabled memory versioning and has provided a non-NULL DDC, for legacy
(i.e., integer pointer using) load and store instructions to continue to
specify the intended memory version and legacy version manipulation
instructions to continue to function.  (Recall that all such legacy
instructions have their integer addresses interposed by DDC.)

We therefore propose that the interposed integer offsets arising from legacy
instructions be interpreted subject to existing architectural address handling
rules.  Arm's MTE, for example, requires the use of Top Byte Ignore (TBI),
which partitions the 64-bit address into an 8-bit metadata field and a 56-bit
address; we propose that Arm processors with CHERI and MTE continue to claim
the top 8 bits of any integer offset within a capability as a metadata field.%
%
\footnote{Because integer offsets often come about through arithmetic, which
may not be aware of the 8+56 partition in the semantics of the bits being
manipulated, it may be useful to slightly tweak the encoding of versions.
Instead of directly taking the top 8 bits as the source of the version value,
it may be useful to XOR them with the top bit of the remaining 56-bit offset.
Thus, the 64-bit 2's-complement values of 1 and -1 would be interpreted as
56-bit 1 and -1, respectively, but both with a version field of zero.}
%
When combined with an \emph{unversioned} capability, the integer offset
specifies the memory version used for a memory transaction; a \emph{versioned}
capability instead overrides the requested version from the offset.%
%
\footnote{As memory transactions are already opportunities for traps in most
architectures, it may be worth trapping if the integer offset calls for a
non-zero version field in combination with a versioned capability.  On the
other hand, it is likely acceptable from a security policy perspective if the
discrepancy is ignored.}
%
This policy may also be applicable to \emph{capability-authorized} instructions
with integer offset register operands, which may simplify capability-aware
supervisory software that must operate on versioned integer addresses.  (It
seems unlikely that there is utility to permitting offset \emph{immediate}
operands to influence memory version fields.)

% >>>
\subsection{Instructions} % <<<

\begin{itemize}

	\item \insnnoref{CStoreVersion} sets the version bits of a memory
granule to the value given in a register operand; the authorizing capability
must be unversioned and must authorize stores of both data and capabilities
to the entire target granule.  Setting the granule's version to $0$ will
cause it to be accessible only to unversioned capabilities.

	\item \insnnoref{CFetchVersion} fetches the version bits of a memory
granule; the authorizing capability must be unversioned, and must authorize
data fetches from the entire target granule.  A return of $0$ indicates that
the granule is accessible only via unversioned capabilities.

	\item \insnnoref{CGetVersion} copies the version field of a
capability into a register.  It is useful mostly for debugging and for
maintaining an abstract interface to capabilities despite the encoded form
bits' being accessible to software.

	\item \insnnoref{CSetVersion} derives a versioned capability from an
unversioned capability and a version value from a register operand.
Attempting to set the version to $0$ will trap.  No other fields are
modified in the derived copy.  Attempting to make a versioned capability
from a versioned one may succeed only if the desired and existing versions are
equal, otherwise the result will have its tag cleared.%
%
\footnote{It may be sensible to always clear the tag or always trap, as
well.  We do not have a use case for the tagged result when-equal case.}

	\item \insnnoref{CLoadVersions} loads version fields for an
entire cache-line of memory granules into an integer register, akin to
\insnref{CLoadTags}.  It is intended as an optimization for system software
paging virtual memory.

\end{itemize}

\subsubsection{Atomics} % <<<

In addition to the above, we desire a means for \emph{atomic} update of the
version of a memory granule (as well as up to a capability-sized word within
it).  Unfortunately, our desires brush up against (micro)architectural limits.
A version-manipulating, capability-sized (and -aligned) store-conditional
instruction, for example, should take four operands:
%
\begin{inenum}

  \item an unversioned capability authorizing access to the target,

  \item the data/capability to store to memory,

  \item the desired new version, and

  \item the destination register indicating success or failure of the store.

\end{inenum}
%
However, it is challenging to fit so many register indices into a single
instruction and this may also exceed the port availability of the processor's
register file.  (A general compare-and-swap instruction is even worse, adding
both the expected value of memory and the expected memory version.)
%
With these constraints in mind, we propose two possibly feasible subsets:
%
\begin{itemize}

  \item \insnnoref{CSCAndUnversion} takes an unversioned memory
capability authorizing the store, a capability register to store, and the
output register.  (It is, therefore, rather like an ordinary
\insnref{CSC}.)  It fixes the desired new version to the unversioned value.
Thus, on successful store, the memory version granule is inaccessible to any
versioned capability, and the same authority used with this instruction can be
used with \insnnoref{CStoreVersion} to subsequently update the target
granule's version.

  \item \insnnoref{CSCWithVersion} is similar, but reads the desired
version \emph{from the output register} before storing back the success
indication.  This works around the encoding space problem, but may still
require an excess of access to the register file.
%
\nwfnote{Cross-reference the behavior of \insnnoref{SC.C.CAP} and
friends, which store back to one of their inputs.  I didn't find a convenient
chunk of prose to point at.}

\end{itemize}

% >>>
% >>>
\subsection{Use With System Software} % <<<
\label{app:exp:versioning:syssoft}

We envision that software will make use of memory versions
\emph{monotonically}.  That is, versions of memory granules will be altered to
revoke \emph{all} access by any existing versioned capability inclusive of that
granule rather than to \emph{restore} access at some earlier version.  Thus, we
believe that \insnnoref{CSCAndUnversion} is sufficiently atomic for
software's needs.  Despite the observable transition of the granule to an
unversioned state before any subsequent transition to a version not yet held
anywhere in the system, the net authority in the system remains the same.

Because there are only finitely many versions available, we further envision
that the \emph{system software} will provide a \emph{revocation} mechanism (in
the style of Cornucopia \cite{cornucopia}) to de-tag or otherwise remove
authority from all capabilities with mismatching versions.  To minimize the
testing required by this facility, it will test only the granule containing the
\emph{base} of each versioned capability it encounters; software engaging in
version-based revocation should, nevertheless, re-version all (partially)
contained granules so that derived capabilities with offset bases are also
revoked.  In a sense, granules exist because they are a sufficient and
straightforward mechanism to capture spans of version information, not because
we expect individual granules within a single segment authorized by a
capability to be changed.  Dually, objects with different lifetimes should not
share granules; this results in much stronger alignment requirements for
allocators, but the practical impact remains to be measured.

We do not specify the shape of the interface exposed for this facility; a
traditional system call to the (privileged) kernel is one possibility for
implementation, but more `autonomic' approaches are feasible as well.  We
envision a global `epoch' counter maintained by the kernel, stepping after
every revocation pass.  If software remembered the counter's value at the time
each allocation came to have its current version, that software would know when
all capabilities with their base in that allocation and of the wrong version
had necessarily been destroyed: in the second epoch after re-versioning.  Such
a scheme would permit sharing work across many allocators desiring revocation
within the same address space.

Because revocation may be done in the background, versions are intended to
be used once between revocations. That is, software should not assume that
it can restore an earlier version to re-authorize an existing capability,
because at any moment the mismatched capability may have become de-tagged.

Whereas we conjecture that the minimum requirements given above for mismatched
versions for loads and stores are sufficient to eliminate temporal safety
issues, there remains the possibility of apparently \emph{inducing} bugs in
programs running under our new semantics.  For example, if software attempts
to (re)initialize an object using a stale capability, the memory will not be
updated and may be reused in inconsistent state.  Trapping on version
mismatch would better expose such issues.

% >>>
\subsection{Microarchitectural Impact} % <<<

The cache fabric must now store the version of each granule in each cache line
(which, in the proposal above, is one, given 64-byte cache lines).
Dereference operations must forward the capability's version field down to
the cache fabric as well.  The minimum requirements for version mismatch
are, however, intended to remove the need to track store requests through
the memory hierarchy.  While precise traps on stores would require
essentially a full read-modify-write cycle, the cache fabric may be able to
raise \emph{imprecise} traps well after accepting a store by tracking the
tentative version bits until they can be checked against the authoritative version
table.

% >>>
% >>>
\section{Linear Capabilities} % <<<
\label{section:linear-capabilities}

Linear capabilities are intended to support the implementation of
operating-system and language-level linearity features, which ensure that at
most one reference to an object is held at a time.
This feature might be used to help support efficient memory reuse -- e.g., by
requiring that a reference to stack memory be `returned' before a caller is
able to reuse the memory.
Architectural linearity does not prevent destruction of the reference, which
may require slow-path behavior such as garbage collection, but can support
strong invariants that would help avoid that behavior in the presence of
compliant software.
This architectural proposal has not yet been validated through
implementation in architecture, microarchitecture, or software.

\subsection{Capability Linearity in Architecture}

We propose to add a new bit to the capability format marking a capability as
\textit{linear}.
%
\note{Because the ISA permits overwrites of linear capabilities (in both
registers and memory), the more appropriate moniker
from substructural logic would be `affine', rather than `linear'.  It
may be worth calling this out in the prose, but not worth renaming
everything.}{nwf}
%
It could be that this is a permission (e.g., Permit\_Non\_Linear).
However, as this
feature changes a number of other aspects of capability behavior, we recommend
not conflating this behavior with the permission mechanism, instead adding a
new field.

Two new \textit{linear move} instructions would be added:

\begin{description}
\item[Linear Load Capability Register (\insnnoref{LLCR})]
This instruction loads a capability from memory into a register, atomically
clearing the memory location [regardless of whether it loaded a linear
capability?].
\note{Clear the whole word or just the tag?}{nwf}

\item[Linear Store Capability Register (\insnnoref{LSCR})]
This instruction stores a capability from a register into memory, atomically
clearing the register when a successful store takes place (e.g., if it does
not trigger a page fault) [regardless of whether it stored a linear
capability?].
\end{description}

The reason to introduce an explicit
linear
load is to avoid taking the cost of an
atomic operation for every capability load dependent on whether the loaded
capability is linear.
%
\note{A non-linear load of a linear capability results in an untagged
register?}{nwf}
%
A separate linear store instruction is not motivated by this concern, but
would add
symmetry, avoiding the need for store instructions to vary their behavior
based on capability type.

A new Permit\_Linear\_Override permission is added, which controls how
existing capability load and store instructions (e.g., \insnref{CLC} and
\insnref{CSC}) interact with linear capabilities.
If the permission is not present, then loaded linear capabilities will have
their tag cleared when written into a register, and stored linear
capabilities will have their tag cleared when written to memory.
This behavior maintains linearity without changing the register or memory
write-back behaviors of these instructions.

If Permit\_Linear\_Override is present on the capability being used to load or
store non-linear capabilities, then linearity is violated, allowing both the
in-register and in-memory capabilities to continue to be valid and marked as
linear.
This permission allows for privileged system software to violate linearity
when, for example, implementing mechanisms such as Copy-on-Write (COW) in the
the OS virtual-memory subsystem or debugging features.

To save instruction encoding space, we might limit these memory access
instructions to be R-type with only a register-specified offset.  This
may be adequate if the instructions are infrequently used.

For register-to-register instructions, there are several options -- in
particular, when implementing capability-manipulation instructions such as
\insnref{CIncOffset} and \insnref{CSetOffset}:

\begin{itemize}
\item We might make existing instructions remove the tag in register write
  back for linear capabilities, enforcing linearity by preventing duplication
  of linear capabilities.

\item We might require that, when existing instructions operate on linear
  capabilities, they write back to their source register, enforcing linearity
  by avoiding duplication to a second register.  This might be
  simplest microarchitecturally.

\item We might add new explicitly linear variants of some existing
  instructions, which would enforce linearity by clearing the source register,
  preventing duplication.
\end{itemize}

In general, ensuring write-back to the same register is easy and cheap to
check dynamically; it avoids the need to introduce a large number of new
instructions offering near-identical behavior.
It also avoids increasing the number of registers that must be written back
by instructions.

Additional concerns exist around the implementation of \PCC{} as relates to
\insnref{AUIPCC}, which normally duplicates a capability.
Although undesirable, the natural design choice is to strip the tag when
writing to the target register, if \PCC{} is linear.

\subsection{Capability Linearity in Software}

The above architectural behavior means that, on the whole, software must
be aware when handling linear capabilities; code must be generated
specifically to use new linear load and store instructions, and to utilize
other register-to-register instructions in a manner consistent with linearity.
There are several specific implications that must be taken into account
when writing system software or compilers:

\begin{itemize}
\item Linear capabilities must be explicitly identified via the source
  language -- e.g., via types or qualifiers -- so as to guide code generation.
  It might be desirable to utilize techniques such as symbol mangling to
  prevent accidents.

\item Linear values cannot be properly preserved by ordinary stack loads and
  spills, so the compiler must take explicit action to prevent this from being
  necessary.
  This might also require static limitations on use of capabilities in the
  language.

\item When linear capabilities are used and manipulated as pointers, it may be
  necessary to generate code quite differently, or to limit expressiveness.
  For example, implied pointer arithmetic when iterating using a pointer
  requires that the original pointer be destroyed, or that the pointer be
  left unmodified but accessed using an integer-register index.
  It is not yet clear to what extent this would interact with common C-language
  idioms.

\item Some systems code must be linearity-oblivious, such as context-switching
  or VM code, and can employ Permit\_Linear\_Override to load and store
  ordinary and linear capabilities using non-linear loads and stores.
  However, it must assuredly not violate invariants of affected software, or
  else linearity may not be enforced.

\item Many current C-language OS and library APIs may be linearity-unfriendly,
  as they frequently accept an existing pointer as an argument, but do not
  `return' it to the caller.
  It may be desirable to have a specific set of extended APIs that are
  linearity-friendly -- e.g., variants of \ccode{memcpy} that copy data into
  and out of linearly referenced memory.
  It is unclear
  whether this would extend to a broader suite of APIs, such as OS \ccode{read}
  and \ccode{write} system calls --  and perhaps would imply polyinstantiation.

\item Debugging tools would need to become aware of linearity so as to
  accurately display information about linear capabilities found in registers
  or memory.
  They might use Permit\_Linear\_Override to gain access to the full contents
  of the register with tag, but must still inspect capability fields suitably,
  and avoid the need to spill values.
  It is not clear how this would interact with current debugger internals.
\end{itemize}

In general, when linearity is violated, it will lead to loss of tags,
preventing dereferences that violate invariants.
It is not clear to what extent this would be easily debuggable.
We can imagine having non-linear sequences generate an exception,
but in some cases this may be microarchitecturally awkward.

Overall, it is not clear to what extent this proposal can interact well with
real-world software designs, or to what extent it usefully supports new
software behaviors.
Key use cases motivating this design typically involve garbage collection
avoidance: e.g., passing an stack pointer across protection-domain boundaries
and checking that it is `returned' before continuing, avoiding the need for
a GC to sweep the recipient domain.
But this does not necessarily alleviate the need to implement more complex
behaviors such as GC in the event that the invariant is violated.

\subsection{Related Work in Linear Capabilities}

Skorstengaard et al.~have concurrently developed ideas about linear
capabilities~\cite{Skorstengaard:2019:stktokens}, which focus on how to
produce a memory-safe execution substrate over a CHERI-derived abstract
capability instruction set. They are able to use linear capabilities to
construct a temporally safe stack calling convention against the model.
This allows formal proof of well-bracketed control flow and stack-frame
encapsulation.  However, their approach also relies on two further
instructions not present in our current sketch: capability split and splice
instructions allowing linear capabilities for stack subsets to be separated,
delegated, returned, and rejoined.  It is not yet clear to us whether these
additional instructions are microarchitecturally realistic, especially in
the presence of compressed capabilities.

\nwfnote{The `splice' instruction of~\cite{Skorstengaard:2019:stktokens} has
potential interactions with sweeping revocation.  The latter depends on a
strict hierarchical partitioning of memory, in which the allocator/revoker
can be certain that there are no capabilities that partially overlap the
regions it revokes.  Permitting splicing of arbitrary linear capabilities
violates this assumption: splice a linear capability at the upper end of one
allocation
together with
one at the lower end of the adjacent, successive allocation.
Allocators
could prevent this by ensuring gaps between
allocations,
or the system can enforce restricted joins of linear capabilities via a
single architectural bit to indicate that a capability is the lowest result
of a series of splits.  When splitting, if this bit is clear in the input,
then leave it clear only in the lower result; if it is set in the input,
leave it set in both outputs.  When splicing, enforce that either both bits
are set (and leave it set in the result) or that the lower is clear and the
upper is set (and leave it clear in the result).  When initially allocating
a segment, clear the bit.  The effect is to enforce the tracking of the
lowermost capability arising from a series of splits and preventing it from
being spliced with a yet-lower linear capability, which will either be an
initial allocation (bit clear) or the upper result of a split (bit
set).}
\pgnnote{This seems REALLY PIGGY.}
\nwfnote{I am not sure what that means.}

The creators of the SAFE architecture~\cite{chiricescu2013safe} also propose
that \textit{linear pointers} could contribute to reasoning about concurrent
memory use.

% >>>
\section{Indirect Capabilities} % <<<
\label{section:indirect-capabilities}

Indirect capabilities could support revocable or relocatable objects without
modification of application executables.
An indirect capability would be identified by the hardware as a pointer to the pointer
to the data.
That is, a load that takes as an address a capability that is marked as an
indirect capability would load a capability from the base address of the
indirect capability, and then would apply any offset to the loaded capability
before dereferencing and placing the returned data in the destination register.
Therefore, a single load that finds an indirect capability as its address would
perform two loads, a pointer access, and then a data access.

\subsection{Indirect Capabilities in Architecture}

We propose to add a new bit to the capability format, marking a capability as
\textit{indirect}.
We recommend
not conflating this behavior with the permission mechanism, instead adding a
new field.

One new instruction would be added:

\begin{description}
\item[Make Indirect (\insnnoref{CMI})]
This instruction makes an ordinary capability into an indirect capability
such that any future dereference will effectively dereference the capability
pointed to by this indirect capability.
The bounds of the capability must be at least the size of one capability,
and will be effectively truncated to this length by \insnnoref{CMI},
though the original bounds will be preserved and applied to the
pointer on data access.

\end{description}

The \insnnoref{CMI} instruction makes a capability indirect, but no
instruction can make an indirect capability direct again.
As a result, delegating
an indirect capability does not
delegate access to the pointer that is dereferenced, but only to the data being
pointed to.

Capability-manipulation instructions such as
\insnref{CIncOffset} and \insnref{CSetOffset}
would transform the offset of the indirect capability,
but this offset would be applied to the pointer on data access.
The pointer access will always use the base of the indirect capability.
In addition, \insnref{CSetBounds} will transform the bounds of the indirect
capability, but these bounds will be applied to the pointer on data access.
The final access must be both within the length of the indirect capability,
which may contain program-narrowed bounds, and the bounds of the object pointer.
The bounds of the indirect capability would be implicitly the size of one
capability, and would not need to be stored.
This behavior allows pointer arithmetic to work as expected on indirect
capabilities, to allow programs expecting standard capabilities to work
unmodified.

\subsection{Indirect Capabilities in Software}

The above architectural behavior means that, on the whole, that software need
not be aware when handling indirect capabilities, but only code that performs
allocation or delegation would construct indirect capabilities, maintaining
pointer tables.

Indirect capabilities might be used for general revocation between compartments.
A buffer passed to another compartment could be passed as an indirect
capability,
with a word allocated by the caller to hold the pointer.
On return, this pointer capability will be invalidated, and no further use
of the indirect capability will succeed.

Indirect capabilities might be used to achieve memory safety for the heap in C.
Every allocation could return an indirect capability, and generate a new entry
in a pointer table.
A call to free would invalidate the entry in the pointer table, and memory could
be reused immediately with a new allocation in the pointer table.
Sweeping revocation may eventually be necessary to free virtual memory space
consumed by freed segments of the pointer table.

Indirect capabilities might be used for a copying garbage collector.
Relocation of allocated objects would be facilitated by all references being
indirected through a single pointer.
When an object is moved, a single pointer could be updated.
While an object is being moved, the pointer could be made invalid, with any
use causing a trap that could be caught and handled appropriately.

% >>>
\section{Indirect Sentry Capabilities} % <<<
\label{app:exp:indsentry}

While sentry capabilities facilitate the construction of capabilities
that grant the right to run code from a fixed entry point, if that code is
intended to run in a particular (register) context, software must use
trampolines (e.g., the PLT stubs) to ensure that this context is constructed
correctly.  These trampolines must intermingle data and code, as the trampoline
has amplified access, relative to its caller, only to the region of its
instruction pointer.  The trampolines must, as well, be \emph{per-context}
(e.g., library instance), which necessitates duplication of the trampoline code
sequence for each context.

\subsection{Points-to-PCC} % <<<

Herein, we propose yet another architecturally-understood form of sealed
capability, the \emph{indirect} sentry capability, which is a curious
hybrid of a sentry capability (of \cref{sec:arch-sentry}) and a special
case of an indirect capability (recall \cref{section:indirect-capabilities}).
Where sentry capabilities point directly at the code to be run (and
expose the entire region bounded by PCC to the callee), these indirect
capabilities point at a capability to be installed into PCC (which, in turn,
points to the code to be run).  Upon invoking such a capability, it is unsealed
and installed into the IDC (capability) register and the pointed-to capability
is installed into PCC; thereby, the callee is granted access to both regions of
memory.%
%
\footnote{If this pointed-to capability is, itself, a sentry capability,
it should be unsealed as part of the load into PCC.  We do not currently
believe that \emph{requiring} this capability to be a sentry capability
has any meaningful impact on the security properties of the system, and so we
do not.}
%
The unsealing and IDC register writeback is not separable from the load from
memory and change of PCC: either both registers are updated and the instruction
completes, or neither are updated and the instruction traps.  We propose a
\insnref{CInvoke}-like, single-operand instruction for such invocations,
\insnnoref{CInvokeInd}; we intend this to be a separate instruction from
\insnref{CJR} so that there is no need for a conditional load in the
microarchitecture.  We do not envision a version of \insnnoref{CInvokeInd}
that writes a link address, but see below for discussion of making function
calls and returns with \insnnoref{CInvokeInd}.

Any capability authorizing capability load may be made into a sealed indirect
entry capability, for which we propose reserving the \cotype{} $2^{64} - 4$.%
%
\footnote{While we do not anticipate comingling code and data within the
authorized region, we do not see much benefit in enforcing a lack of
\cappermX on the original capability nor in shedding it as part of
sealing.
%
\nwfnote{If we get really tight on encoding space, we could use \cappermX
to distinguish between sentry and indirect sentry capabilities both with
\cotype{} $2^{64} - 2$.}
%
}
%
A new, two-operand instruction is required for this action, which we call
\insnnoref{CSealIndEntry}.

It is straightforward to adapt the designs of \cref{sec:arch-sentry} to this
instruction so that, for example, the PLT stub \emph{code} can be relocated to
the common, read-only section, leaving a kind of data-only trampoline which
contains capabilities to the (also shared) code to be run and the per-instance
RW data.  Each entrypoint requires one capability, rather than a full PLT stub.
This enables unifying the per-instance PLT stub and per-instance data regions
into a single per-instance region which continues to not need execute
permission.

Additionally, this mechanism could be suitable for decreasing the information
exposure between caller and callee functions.  If, rather than exposing a return
(sentry) capability to the callee, the caller were to spill its return
capability to the stack and expose a sealed indirect entry capability derived
from the stack, the callee can have its access to the caller's stack completely
removed.  Upon return, the caller's original stack capability would be
available in IDC.  Spilling the return address will involve storing a
capability derived from PCC but pointing past the \insnnoref{CInvokeInd}
instruction.  All told, we expect this kind of function call to require ten
instructions on call rather than the one \insnref{CJALR}:
%
\begin{itemize}

	\item three (\insnref{AUIPCC}, \insnref{CIncOffsetImm},
		\insnref{CSC}) to compute and spill the return address,

	\item two to move the stack pointer
		(\insnref{CRepresentableAlignmentMask},
		\insnnoref{CAndAddr}),

	\item four to bound the stack pointer (\insnref{CGetOffset},
		\insnref{CSetOffset} (to zero), \insnref{CSetBounds},
		\insnref{CSetOffset} (back)),

	\item one to seal indirect sentry capability into the link register
		(\insnnoref{CSealIndEntry}), and

	\item one to transfer control (\insnref{CJR} or
		\insnnoref{CInvokeInd}).

\end{itemize}
%
There is likely opportunity for additional, specialized instructions here;
some plausible examples include:
%
\begin{itemize}
%
  \item An instruction which set the \emph{limit} (i.e., $\cbase +
  \clength$) of a capability to the cursor and left the base alone could
  replace the four instruction sequence bounding the stack pointer.
%
  \item \insnref{CRepresentableAlignmentMask} and \insnnoref{CAndAddr}
  could be fused into a dedicated instruction for aligning the capability's
  offset appropriately.
%
\end{itemize}

% >>>
\subsection{Points to Pair} % <<<

Another option, for architectures open to multi-word transactions in their
memory subsystems, is an indirect sentry capability which points to the pair of
PCC and IDC.  Invocation of such a sentry performs two capability loads through
an ephemeral (architecturally invisible), unsealed copy of the given sentry and
then, with both capabilities in hand, installs both into the register file
atomically.  There is no requirement that the two capabilities pointed at be
sealed.  Because these capabilities reside in memory, the instruction
constructing these ``points to pair'' indirect sentries likely cannot perform
any validation on their contents.

% >>>
% >>>
\section{Anti-tamper Seals}
\label{sec:anti-tamper}

When implementing allocators such as the C language's malloc and free,
it is common to require that the caller only pass values to free that
were previously returned to malloc (according to ISO C, doing otherwise
is undefined behavior.)
As is typical of C, run-time programmers exploit this and do not perform
checks that the passed pointer is in fact an allocated pointer and the
implementation may not retain sufficent information to confirm this.
We could greatly reduce the number of check in the free path if we could
be certain that the passed capability was exactly the one we handed out.

To address this need, we propose a new variant of sealing: anti-tamper seals.
A portion of the otype space would be reserved for anti-tamper seals
and capabilities sealed with an anti-tamper otype would have the following
properties:
\begin{itemize}
  \item The capability can be dereferenced or jumped to as though it were
  unsealed.
%
  \item Address modifying instructions (e.g. CSetOffset) work as though
  the capability where unsealed.
%
  \item CAndPerm, CSetBounds, and CSetBoundsExact unseal the capability
  (setting its otype to -1).
  \bdnote{Other capability derivation instructions require evaluation...}
\end{itemize}

The justification for allowing address adjustments is similar to that
for allowing capabilities to stray out of bounds.
We want to allow for the case that a programmer alters the address
of a capability before restoring its address using some seperate state
(e.g. buffer length) and freeing it.
It's unclear how common such code is, but intuitively such patterns
will be difficult to detect statically.

% >>>
\section{Compact Capability Coloring} % <<<
\label{sec:compactcolors}

\note{CInvoke presents an interesting challenge, generally, when we desire to
restrict capability flow: the colors of the sealed capabilities authorizing
CInvoke, the current PCC, and any argument capabilities to be passed must be
sensible to the calling domain (which holds them, of course, prior to the
CInvoke), but will become visible in the callee domain.  This suggests that
callees must have some private, suitably colored region of memory whither to
spill whatever colors come their way.  This further suggests, perhaps, that
the assignment of semantics to individual colors may not be as unconstrained
as desired, as the decisions are not confined to particular security
domains.}{nwf}

As noted above, the \cappermG permission described in the model of
Section~\ref{sect:capability-permission-bits} is semantically not parallel
to the other permissions.  It is a one-bit attribute of the capability
itself, a concept we term a \emph{color}, borrowing from the
information-flow
analysis community~\cite{Popek79}.  Capabilities without the \cappermG
color (called Local) have their \emph{flow} constrained, in that they can
be stored only through a capability (of any color) bearing the
\cappermSLC permission (as well as
\cappermSC and \cappermS).  These two bits, one color and
one permission, are leveraged by the existing runtime system to ensure that
pointers to the stack can be stored only to the stack (and not the heap).
That is, excepting capabilities within the TCB, all capabilities authorizing
access to stack memory are colored Local, and all capabilities bearing the
\cappermSLC permission authorize access only to stack
memory.  While the model permits a capability to stack memory (which must,
per the above restriction, be Local) to be without the
\cappermSLC permission, such capabilities are not
deliberately constructed (unless they lack \cappermSC and/or
\cappermS as well, i.e., as part of a read-only view).

To recapitulate, then, we have the following four states of being for
capabilities:

\begin{center}\begin{tabular}{ccc}
{\bf Color} & {\bf \cappermSLC} & {\bf Use} \\
\hline
Global & Yes & TCB only \\
Global & No  & Heap memory \\
Local  & Yes & Stack memory \\
Local  & No  & Unused
\end{tabular}\end{center}

The last configuration may be created (even outside its read-only utility)
by monotonic action from any of the other configurations.  These colorings
and permissions capture the following intended flow policy:

\begin{center}\begin{tabular}{ccc}
{\bf Capability type...} & {\bf Stored through type...} & {\bf Permitted} \\
\hline
Stack & Stack & Yes \\
Heap  & Stack & Yes \\
Stack & Heap  & No  \\
Heap  & Heap  & Yes \\
\end{tabular}\end{center}

In this policy, stack-type capabilities are universal authorizers of
stores (`universal recipients', if you will) and heap-type capabilities
are universally authorized to be stored (`universal donors').  (The
TCB-only, Global capabilities with \cappermSLC may be
stored to and may authorize any capability store; the unused state can be
stored only to TCB- or stack-state capabilities, and may authorize storage
only of TCB- or heap-state capabilities.)

Neglecting the TCB state for a moment, we see that a single bit should be
sufficient to encode our desired policy, using a material conditional:
\emph{if} the capability being stored is stack-type, then the capability
authorizing this store must also be stack-stated (or, equivalently, phrased
as the contrapositive, \emph{if} the capability authorizing the store is
heap-stated, the capability being stored must also be heap-stated). Similar
flow policies also exist for flows across permission rings (the kernel may
hold its own and user capabilities, but user programs may hold only user
capabilities) and for flows through garbage-collector-managed memory regions
(capabilities to managed memory may be stored only in managed memory,
so that the collector must be notified of roots escaping).  This suggests that we are justified in
carving out several bits for orthogonal colorations; we suggest at least
three, for the cases just considered, and perhaps no more than six, for
reasons we will discuss below.

To abstract over the several colors, we adopt the terms `positively colored' and
`negatively colored' to refer to the two possible states of a color.  The flow
policy is the logical \emph{and}
of the conditional for each color: ``if the
capability being stored is positively colored, then the capability authorizing the
store must also be positively colored'' or, equivalently, ``if the capability
authorizing the store is negatively colored, the capability being stored must be
negatively colored.''  Positively colored capabilities are the `universal
recipients', and
negatively colored capabilities are the `universal donors'.%
%
\footnote{Another dimension of generalization would be to have
\emph{load}-side color checking.  That is, we could imagine enforcing
policies of the form ``if the capability authorizing a load is positively colored, then
the capability loaded must also be positively colored (and if not, the result is not a
capability).''  We have no immediate use for such policies, but for somewhat
related considerations, see Section~\ref{app:exp:recmutload}.}

The two-bit color-and-permission scheme described at the start of the
section has a simple answer to the `primordial' coloring of capabilities,
and to the recoloring of capabilities into target states: the
maximally permissive TCB state may be monotonically transformed with
\insnref{CAndPerm} into any other state.  Subsequent (monotonic)
actions will never convert a heap-type capability into a stack-type one, or
vice-versa.  Given only a single bit for our color, any primordial
capability must have \emph{some} color, not a dedicated TCB-only
`colorless' choice.  Further, our one-bit scheme must not ambiently permit
conversion, in either direction, between the two states.  We therefore
propose that color bits are separate from permissions, immune to the action
of the ambiently available \insnref{CAndPerm} instruction.  We suggest
that, primordially, capabilities be positively colored in all colors, so that, having
explicitly changed the color of some memory capabilities, the software may
not accidentally store into these now negatively colored regions.

What remains to be spelled out, then, is the \emph{selective} authority to
alter colors.  Towards this end, we conceptually introduce yet another
`space' of identifiers guarded by capabilities and introduce a
`color-change authority' capability, which moves about the system as any
other (and itself bears colors).  The primordial capability authorizes
any change to any color of any capability anywhere in memory.  Such
authority may be monotonically shed, coming to authorize only some changes
(e.g., creating stacks from heap memory, but not the reverse) to some colors
(e.g., changing only the stack/heap color but not the kernel/user color).%
%
\footnote{In principle, one could also monotonically confine color changes
to capabilities located in particular parts of memory or, perhaps more
usefully, to memory capabilities \emph{referencing} particular parts of
memory.  Encoding a restricted notion of change authority for non-memory
capabilities such as sealing, compartment, or color-change capabilities is
less obvious.  We are not yet sure how to proceed in this dimension of
monotonicity, and do not so here.  Our color-change capabilities will
always authorize changes to any capability anywhere, but, of course, the
would-be authorized agent needs access to the source capability in the first
place.}

\paragraph{Variant 1}
%
We introduce a new instruction, \insnnoref{CChangeColor}, which takes a
capability register containing the source capability, another for the
destination, and a third for the authority capability.  This instruction
carries out \emph{all authorized transitions} to produce a target that differs from the source only
in its colors.  We might have preferred a four-parameter instruction, which
additionally specified \emph{which} color to change from the authorized set,
but this would likely require too many bits; in practice, we believe that
color-change-authorizing capabilities would be few and relatively static, so
the cost of tailoring to uses would be small.

An initial encoding of such color-change authority capabilities,
backwards-compatible with the existing capability encoding described in this
document, is to use a capability that
%
\begin{itemize}
%
  \item Bears no permissions other than a new Permit\_Change\_Color permission.
(Ideally, this would be encoded as the \emph{type} of the capability, and not
consume an entire permission bit.)
%
  \item Has a base of zero and a limit of the top of the address space.
%
  \item Stores in its offset a bitmask authorizing color changes as follows:
color $n$ may be transitioned from its current value $c_n$ to its negation
if bit $2n + c_n$ is set.
%
\end{itemize}
%
It is immaterial which of `0' or `1' one assigns to the different color
choices.  However, the system must pick one; we suggest using `1',
commonly read as `true', for the `positively colored' choice, in keeping with the
presentation above.
In this encoding, the offset-adjusting instructions must be modified to
permit only bitwise \emph{and} operations on the offsets of these capabilities.
(If one is conflating capability types, as we do at present, the appropriate
guard is that \emph{only} Permit\_Change\_Color is set.) This is perhaps the
most awkward feature of this design, though we believe the checks can be
added without impacting timing.  (In a world where capability types were
explicit and separate from permission bits, we could reuse the permission
bits, already subject to manipulation only by \insnref{CAndPerm} to
carry our permission bitmask, assuming there are at most half as many colors
as permission bits.)

\paragraph{Variant 2}
%
Perhaps a more natural encoding
would instead have capabilities that enact
exactly one color change when cited (but may \emph{authorize} more than one).
Here, we propose that the space of integers from $0$ to $2C$, with $C$ being
the number of color bits available in the system, be another `identifier
space' for capabilities.  A color-change capability holding value $2n+c_n$
requests toggling color $n < C$ from $c_n$ to its negation when used as the
authorizing capability with the \insnnoref{CChangeColor} instruction.  In
this scheme, there would be no need for any fiddly bit manipulations of
capability offsets, but at the cost of more capabilities held by agents
authorized to perform some, but not all, color changes.

\paragraph{Variant 3}
%
In fact, there is no need to introduce an entirely new capability type,
permission bit, or instruction.  Because sealing object types
(\cotype{}), in practice, are only at most 24
bits wide, and there are very few colors, we could reuse invalid encoding space
for sealing capabilities to also authorize color changes: values $x$ in the
range of $2^{24}$ to $2^{24}+C$ could be defined as colors rather than invalid
\cotype{}s and the existing use of \cappermSeal and \cappermUnseal bits could
control setting the target capability's color number $x - 2^{24}$ to become
positively or negatively colored.  The existing \insnref{CSeal} and
\insnref{CUnseal}
instructions could be used in lieu of any new
\insnnoref{CChangeColor}.  This shares with variant 2 the need to have many capabilities
held by agents authorized to change multiple colors if they are not contiguous
or authorize different transition directions.

\note{In light of the increased alignment requirements imposed on sealed
capabilities and to facilitate sealing of capabilities authorizing color
change, one may wish to shift the color index up by 12 bits, using $b$ to $b
+ 2^{12}C$ and ignoring the bits below 12.  One might be tempted by $b =
2^{24}$, so as to be `just above' the \cotype{} space, as
it
was above, but
as sealing alignment requirements apply to them as well, perhaps $b =
2^{36}$ is a better choice.}{nwf}

% >>>
\section{Sealing With In-Memory Tokens} % <<<
\label{app:exp:typetoken}

Deciding on the number of \cotype{} bits within a sealed capability has
been challenging, because the bits come at the expense of bits for precision
of bounds, permissions, and colors.  In this section, we propose that
\emph{virtual addresses} can play double-duty as \emph{type identifiers},
either supplanting or reducing the need for in-capability \cotype{}
bits.  The design of this section is a somewhat invasive change to CHERI,
but appears promising.

\subsection{Mechanism Overview} % <<<

We propose that sealed objects have their type not in the referring
capability, but rather in a tagged capability-sized structure at the
\emph{base} of the object in memory.  This structure is termed a `type
token' and it contains a virtual address (and metadata) but does not confer
any permissions, to its contained address or otherwise, to its bearer; in
fact, as a defensive posture, we do not permit tagged type tokens to be
loaded into registers unless PCC has \cappermASR.%
%
\footnote{This means that a sealed object cannot simply be copied via
\texttt{memmove}; a copy or move constructor must be invoked to reconstruct
the type tag on the target memory.  This does not seem to be an especially
high burden.  In fact, even the \cappermASR caveat can
be removed if an alternative mechanism for tag reconstruction is made
available to the kernel; for example, capability reconstruction
could gain the ability to
reconstruct tags given the sealing authority.}
%
In addition to creating a sealed reference capability, sealing an object
would \emph{store} a suitable type token to memory, derived from the
capability used to authorize the seal.  Unsealing \emph{fetches} and
verifies this type token against the capability authorizing the unsealing.

% >>>
\subsection{Shared VTables with Sentry Capabilities and Type Tokens} % <<<
\label{app:exp:typetoken:vt}

\begin{figure}[htb] % fig:app:exp:typetoken:vt <<<
  \centering
  \includegraphics{fig-type-token.pdf}

  \caption{Schematic representation of a shared VTable design for a base class.
  The user directly holds a sentry capability to the object constructor guard,
  which uses the adjacent Permit\_Create\_Type\_Token-bearing capability to
  stamp object instances.  Each object instance is held by the user through a
  \cappermLC-bearing capability and has a two-capability header,
  consisting of a \cappermLC-bearing capability to the VTable and
  a sealed capability bearing load and store permissions to the object instance
  data.  The VTable itself is an array of sentry capabilities pointing at
  method guards, which in turn verify the object instance's type token against
  their unsealing right before invoking the actual class method handler.}
  %
  \label{fig:app:exp:typetoken:vt}

\end{figure} % >>>

Sentry capabilities (recall \cref{sec:arch-sentry}) give software the ability
to ensure that control flow can enter a given region at a particular
address: the bearer of a sentry capability can jump to it but cannot adjust
its offset.  However, unlike the existing
\insnref{CInvoke} mechanism, sentry capabilities when invoked transition only the PCC register.
To transition other registers as a function of the instance, we propose a
PLT-like scheme using dedicated trampolines to load \emph{unsealed}
capabilities that were nevertheless beyond the reach of the caller, due to
the sealed nature of the sentry capability held.

In-memory type tokens allow software the ability to mimic the existing
CHERI sealing mechanism, trading one capability in memory to not need the
\cotype{} bits in referring capabilities.  (This does come with the
additional cost that sealing a region of memory under multiple seals will
require the use of several tokens in memory with successively larger bounds
in the referring capabilities.) In \cref{fig:app:exp:typetoken:vt} we show a
schematic representation of using in-memory type tokens to guard method
invocation of a multiply instantiated (C++) object.

Combined with sentry capabilities, an object's shared code can now securely
verify that its first argument is indeed a sealed capability to a data
region resulting from this object's constructor.  The constructor is made
available as a sentery capability to a region containing a capability bearing
\cappermSeal.  The non-constructor capabilities in the VTable are sentry
capabilities pointing within a region bearing corresponding \cappermUnseal
rights.  These three regions (the constructor guard code, the method guard
code, and the VTable) are created once, when the object class is loaded, and
will never be written to thereafter.  Conveniently, the object-class code
location can be used as its own type token value, there is no need for a
separate pool of virtual addresses for type token values.  The separation of unsealing
rights is not essential and is another defense in depth: the non-constructor
methods will not necessarily come to hold, even transitively, a capability
bearing \cappermSeal for this object type.

% >>>
\subsection{The Mechanism in More Detail} % <<<

Type tokens are created directly into memory with a new \insnnoref{CSealTyT}
instruction, stored at the \emph{base address} of the capability
being sealed, which must be capability-aligned (and the to-be-sealed
capability must authorize an at-least-one-capability-sized segment of
memory).  \insnnoref{CSealTyT} requires that the capability to be sealed
bear \cappermL and \cappermS and that the invocation reference an
in-bounds \cappermSeal-bearing%
%
\footnote{For compatibility with CHERI-MIPS, we exclude from
\insnnoref{CSealTyT}'s domain sealing capabilities referencing the
bottom of memory, from $0$ and to the maximum \cotype{} value, interpreted
as an unsigned integer, available to the implementation, inclusive. These are
reserved for use with the existing \insnref{CSeal} instruction.}
%
capability whose cursor will form the type tag.%
%
\footnote{It is not clear whether \insnnoref{CSealTyT} should permit the
clearing of \cappermL and/or \cappermS in the resulting sealed
capability, despite requiring them on input.}
%
Software must ensure that the store done as part of sealing is visible to
other processors before publishing the sealed capability anywhere it may be
read by another core.  Immediate fencing is not always required, and so we
suggest it not be intrinsic to the \insnnoref{CSealTyT} instruction.
%
The sealed capability resulting from \insnnoref{CSealTyT} will have its
\cotype{} set to $2^{64} - 3$, truncated as required by the implementation.

Attempting to load a type token via \insnref{CLC} will succeed, but will
strip the tag.  The resulting register contents need not be particularly
well specified; in particular, we should no more expect sensible results
from the capability-observing instructions here than if we had loaded an
arbitrary untagged region of memory.

Token-mediated unsealing is done by a new \insnnoref{CUnsealTyT} that
takes a sealed capability (with \cotype{} of $2^{64} - 3$)
and an in-bounds authorizing capability bearing
\cappermUnseal.  If the cursor of the authorizing capability matches the
virtual address stored in the type token at the base of the sealed object,%
%
\footnote{This load is why \insnnoref{CSealTyT} required \cappermL of
its to-be-sealed capability.}
%
then \insnnoref{CUnsealTyT} produces an unsealed version of the sealed
capability.  Microarchitecturally, \insnnoref{CUnsealTyT} is somewhat akin to a
compare-and-swap whose store-back is into the register file rather than
memory.

It might be helpful to software to add a \insnnoref{CGetTypeTyT}
instruction that
somewhat mirrors the \insnref{CGetType} instruction.
\insnnoref{CGetTypeTyT} would fetch from the base address of a
sealed capability (of the right \cotype{}) and store
the virtual address from the type token back to a general-purpose
integer register.
We propose that, if an exception is not desirable, that the value $2^{64} - 1$ be
used if the memory at the base is not a type token.

% >>>
\subsection{Unseal-Once Type Tokens} % <<<

It is likely useful to have a version of unsealing that atomically prevents
any future attempts.  Rather than merely \emph{fetch} the type token, this
instruction would carry out a CAS-like update of the type token in memory.

% >>>
\subsection{User Permissions For Type-Sealed VA Capabilities} % <<<

Because type tokens are capability-sized structures used only for their
contained virtual addresses, there are many spare bits in the structure (in
fact, a few type-tagging bits shy of an entire machine word's worth).  One
especially attractive possibility, if it can be demonstrated to be
sufficiently secure, is to push the architecturally defined permission bits
within the sealed capability into the type token.  This would permit the use
of the intra-capability permission bits as user permissions, subject to the
action of \texttt{CAndPerm} despite the sealed nature of the capability.
We would then be able to use capability permission bits to help arbitrate
permissions to methods within an object, as is typical of other capability
systems, rather than, as suggested by the design in
\cref{app:exp:typetoken:vt} above, having one sentry capability per procedure
and gating permission by possession of the procedure's guard's sentry
capability.  \insnnoref{CUnsealTyT} would use the bits from the type token in
its output capability, and software would be able to inspect the permission
bits of the input object reference (i.e., there would be no need for a second
register storeback in \insnnoref{CUnsealTyT}).

In this scheme, should an object wish to be able to grant sealed references
with one of several sets of architectural permissions, it suffices to place
an array of type tokens at the beginning of instance memory and adjust the
base of the (to be sealed) capability, while leaving the cursor to point at
the start of the object's data.  Any type tokens within reach confer no
authority, even after we have moved architectural permission bits into them.
Further, because type tokens cannot be created in memory except by
\insnnoref{CSealTyT} or highly privileged software, aliasing of the
memory containing the type token cannot \emph{de novo} amplify architectural
access (but may be vulnerable to confusion within suitably authorized
control flow).

% >>>
\subsection{Token-mediated CInvoke} % <<<

\insnref{CInvoke} poses something of a challenge for in-memory type tags: a
single instruction must, seemingly, perform \emph{two} fetches from memory
and then do a comparison on the loaded values.  However, because the
instruction cares only about the equality, it seems that we can turn this
into a fetch from one capability's base and then a CAS-style
\emph{comparison} against the other's.  In fact, this combines nicely with
unseal-once type tokens: if \insnref{CInvoke} fetches from the sealed code
capability first, it is then in a position to issue the appropriate CAS
against the sealed data capability.  In CHERI-MIPS, \insnref{CInvoke} is
already a two-cycle instruction, occupying two successive stages of the
pipeline, and so we conjecture that the changes requisite to support
token-mediation are small.
\jhbnote{We should update this last sentence to be relevant to
  CHERI-RISC-V designs}

% >>>
\subsection{Hybridization} % <<<

This scheme uses one \cotype{} value for its sealed capabilities; the remaining
values are still available for the rest of the system's use.  It is our hope
that most users of \cotype{} values can be rearchitected to use this in-memory
scheme and that the \cotype{} field can be reduced in size.  However, the
\cotype{} field should not be entirely eliminated: its existence allows us avoid
some of the overhead of this design in the innermost ring of the system.%
%
\footnote{Because the innermost ring is presumably the kernel's TCB, a
hypervisor, or `nanokernel'-- effectively microcode -- the resulting system
has some similarities to the Intel 432 / BiiN / i960MX lineage, which had a
few architecturally understood special types of capabilities -- but relied on
software interpretation for the rest.}
%
Such \cotype{} bits would also let software create sealed objects other than
enter capabilities without memory footprint.

% >>>
% >>>
\section{Windowed Short Capabilities} % <<<
\label{sec:windowedshortcaps}

An frequent initial objection to CHERI is that even the 128-bit compressed
form of capabilities occupies too much space, especially for pointer-heavy
workloads.  However, when discussing a 64-bit virtual address space, it
seems plausible that 128 bits is the best we can do: the metadata CHERI
requires vastly outstrips any `spare' bits in the address, and any size
that was not a power of two bits would be awkward, at best.  One way out
would be to imagine that one could mix 128-bit and 64-bit capabilities
within an address space, with the caveat that the 64-bit capabilities could
address only a 32-bit address space (i.e., they would have a 4 GiB reach)
and would have a smaller set of permission bits, fewer flag bits, and fewer
bits for object types.  While we could limit all 64-bit capabilities to
referencing a particular, fixed 4 GiB region of the larger addess space
(e.g., the first 4 GiB), a better design, if we could get it, would be to
allow the 4 GiB window to be chosen by a 128-bit capability.

The design we detail here treats these 64-bit capabilities as specialized
representations of 128-bit capabilities.  Importantly, this design does not
modify the representation or semantics of capabilities within the register
file: the bulk of the system's operation is not impacted.  We introduce new,
purpose-made instructions for loading and storing these short
representations of capabilities; stores especially may fail if translation
is not possible.

\subsection{Restricting Capabilities to 32-bit Windows} % <<<

Because 64-bit capabilities operate only within a 4 GiB window of the
address space, when fetching a 64-bit capability from memory, we fill in the
implied upper 32 bits of the full 64-bit address from the \emph{cursor} of
the \emph{capability authorizing the fetch}.  This straightforward operation
is provided by the \insnnoref{CLShC} instruction.

When attempting to (encode and) store a capability to a short form in memory,
the store will fail%
%
\footnote{It would be sufficient to store a de-tagged word, but trapping is
more likely programmer friendly.  While this is a data-dependent action, as it
requires a comparison between the (untranslated, virtual) target address and
the capability from the register file, this is not the only data dependence in
the short capability store instruction.}
%
unless all three of the following addresses agree on their top 32 bits: the
computed destination address of the store and the base and limit of the
capability being stored; the cursor of the capability to be stored is permitted
to be within either adjacent 4 GiB window (but must still be representable).%
%
\footnote{Alternatively, it would suffice to ensure that, on decoding, any
access beyond the limits of the 4-GiB-aligned region had been shed.  Because
short capabilities are never used directly, there is some flexibility in
enforcement here.}
%
All of this is provided by the \insnnoref{CSShC} instruction.

A consequence of this design is that short capabilities (transitively
reached through short capabilities) are always interpreted within the 4 GiB
window specified by the initial reference through a full capability.  These
capabilities may be stored as short capabilities anywhere within this window
(or as full capabilities anywhere in the address space).  Because
capabilities in registers always have their full 64-bit virtual address
cursor and bounds, it is impossible to use a short capability in one 4 GiB
window to derive a capability to any part of a different window: the
dereferencable region is always contained within the original window whence
the capability was loaded, and so attempted stores to another window will
fail.%
%
\footnote{If ever direct memory-to-memory capability copies become possible,
it would be necessary to explicitly check that copied short capabilities are
not being replicated in ways that would change their decoding.}

\nwfnote{The proposed instruction encodings have quite sizable footprint in
the encoding space.  Moreover, we probably want \insnnoref{CLLShC}
and \insnnoref{CSCShC} opcodes, too.}

% >>>
\subsection{Restrictions Within Short Capabilities} % <<<

In order to reduce the space required for metadata within short
capabilities, we suggest several restrictions.

Within the permissions field, we suggest that short capabilities be limited
to expressing virtual address space, so that \cappermSeal, \cappermUnseal,
and \cappermCid are implicitly false for any short capability.  This
seems reasonable, as these gate fundamentally new facilities offered by CHERI
and seem like they will be relatively rare even in fully CHERI-fied software
stacks, so the requirement to use a 128-bit capability should not be
onerous.
%
Further, because we intend short capabilities to be used mostly for
sandboxes within a larger ecosystem, we think it reasonable to imply that
\cappermASR is also false.
%
Similarly, we do not foresee the utility of the Local/Global distinction for
short capabilities, and so propose implying
\cappermSLC to be false.%
%
\footnote{We could also imply the Global permission bit to be \emph{true},
but then we would need to fail attempts to encode local capabilities into short
forms.  While we do not anticipate the use of capabilities bearing
\cappermSLC outside trusted
software, it nevertheless seems simpler to leave Global within the short
capability encoding.}
%
All told, these implications eliminate five existing permission bits from
short capabilities' representations.

We suggest a reduced object type range for short capabilities, as well.
This will have implications in the software stack: `small' object types
will be somewhat precious, and so may need to have special handling in the
allocator(s) thereof.  The utility of sealed short capabilities, and
especially of architecturally defined sealing object types to short
capabilities, remains an open question.

Bound metadata may also be subject to pressure, and so short capabilities
may face stricter alignment requirements for large objects than full,
128-bit capabilities.  While this would not be great, it may be that
references to large objects are relatively sparse, and so software may find
it easier to fall back to full capabilities rather than insist that all
capabilities should be short whenever possible.

% >>>
\subsection{Tag Bits and Representation for Shared Memory} % <<<

Short ``capabilities" could plausably be left untagged in the architecture
and used only as forgeable fat pointers which are lifted into the capability
space on coversion.
If we were to tag short capabilities, we require more bits for
distinguishing mixed capability widths from data.
In a 128-bit-sized and -aligned
region of memory, there are five possible options, assuming that 128-bit
capabilities must remain 128-bit-aligned:
%
\begin{inenum}
%
  \item One 128-bit capability.
%
  \item Two 64-bit capabilities.
%
  \item One 64-bit capability, followed by data.
%
  \item One 64-bit capability, preceded by data.
%
  \item Only data.
%
\end{inenum}
%
There are several ways that we could arrange to distinguish these
possibilities, but two seem especially attractive.  Perhaps the simplest
approach is to use three out-of-band tag bits rather than the one per
128-bit granule of memory that CHERI now imposes; this would leave us with
three values reserved for future expansion.  One could slightly tamp down on
the need for tag bits by tagging entire \emph{cache lines} instead: eight
sets of 5-way discrimination, corresponding to 128-byte cache lines, requires
only 19 bits rather than the more straightforward 24, at the cost of more
complex decoding logic (likely in the LLC).

However, we may be better served by the use of two out-of-band tags and one
bit in the capability encodings themselves, effectively giving us somewhere
between two and four bits of metadata, depending on the scenario.  One
possible encoding is shown in \cref{tab:shorttags}.  Forbidden states should
trigger machine check exceptions or something similarly indicative of
catastrophe.
%
This scheme is relatively straight foward to operate, but requires a little
awkward handling of the inherent asymmetry between the upper and lower 64
bits within a 128-bit granule.  A load of a full capability must verify that
both out of band tag bits and $t_\text{hi}$ are all asserted.  A load of a
short capability from the upper position must verify that $T_\text{hi}$ is
asserted and $t_\text{hi}$ is clear.  A load of a short capability from the
lower position must verify that $T_\text{low}$ is asserted, that
$t_\text{low}$ is clear, and that either $T_\text{hi}$ or $t_\text{hi}$ is
clear.  Data stores always clear the corresponding out-of-band bit; stores
to the lower half of a capability granule must additionally access
$T_\text{hi}$ and, if $T_\text{hi}$ is asserted, then access $t_\text{hi}$
to determine whether $T_\text{hi}$ should be cleared as well (to avoid the
forbidden states marked with $\dagger$).  Fortunately, all of this state
machine logic is localized within a cache line and its tag bits.

\begin{table}
\begin{center}
\begin{tabular}{cccc|l}

$T_\text{hi}$ & $T_\text{low}$ & $t_\text{hi}$ & $t_\text{low}$ & Meaning \\
\hline\hline

0 & 0 & $X$ & $X$ & Two data words \\
0 & 1 & $X$ & 0   & 64 bits of data above a 64-bit capability \\
0 & 1 & $X$ & 1   & Forbidden \\
1 & 0 & 0   & $X$ & A 64-bit capability above 64 bits of data \\
1 & 0 & 1   & $X$ & Forbidden\textsuperscript{$\dagger$} \\
1 & 1 & 0   & 0   & Two 64-bit capabilities \\
1 & 1 & 0   & 1   & Forbidden \\
1 & 1 & 1   & $X$ & A 128-bit capability \\

\end{tabular}
\end{center}

\caption{A possible hybrid out-of-band and in-band tagging scheme for mixing
128-bit and 64-bit capabilities.  $t_\text{hi}$ and $t_\text{low}$ are the
intra-capability tag bits for the upper and lower 64-bit regions,
respectively, while $T_\text{hi}$ and $T_\text{low}$ denote the
corresponding two out-of-band tag bits.  $X$ indicates `don't care' and
stands for either bit value.}

\label{tab:shorttags}
\end{table}

Similar considerations hold should we wish to mix all of 64-, 128-, and
256-bit capability forms.  In such a system, there are 26 states for every
256-bit granule of memory: each 128-bit granule may be in each of the 5
states given above, or an adjacent pair may hold a 256-bit capability.

\subsubsection{With Relaxed Alignment Requirements} % <<<

It may be more natural to permit \emph{all} capabilities, both 64-bit and
128-bit, to be stored at 64-bit alignment.  In such a case, within a
128-bit-sized and -aligned region, there are now these 10 possibilities:
%
\begin{inenum}
%
  \item One 128-bit capability, spanning the whole region.
%
  \item The tail of a 128-bit capability, followed by the head of a 128-bit capability.
%
  \item The tail of a 128-bit capability, followed by a 64-bit capability.
%
  \item The tail of a 128-bit capability, followed by data.
%
  \item The head of a 128-bit capability, preceeded by a 64-bit capability.
%
  \item The head of a 128-bit capability, preceeded by data.
%
  \item Two 64-bit capabilities.
%
  \item One 64-bit capability, followed by data.
%
  \item One 64-bit capability, preceded by data.
%
  \item Only data.
%
\end{inenum}


% >>>
% >>>
\subsection{SoCs With Mixed-Size Capabilities} % <<<

It is frequently the case that Systems on Chip (SoCs) contain 64-bit
application cores and also 32-bit microcontrollers.
One potential further use for this approach is to allow bridging between those
two worlds: 64-bit cores with 128-bit capabilites that are able to load and
store 64-bit capabilities used by 32-bit cores connected to the same memory
fabric.
Care would be required to ensure that capabilities originating on one core
were derefenced only with a suitable address space on a second core able to
access them.

\nwfnote{In such circumstances, one could imagine that the shared memory
block(s) have a capability granularity and then restrict larger cores to
using the short-cap operations and requiring that software preserve the
alignment requirements for these regions when they are virtually mapped.}

% >>>
% >>>
\section{Capabilities For Physical Addresses} % <<<
\label{app:exp:physcap}

\subsection{Motivation}

CHERI capabilities that authorize access to memory are typically interpreted
in combination with an ambient virtual address translation configuration.
That is, the addresses authorized by a CHERI memory capability are taken to
be virtual addresses, which are then translated to physical addresses by the
core's MMU.  The MMU configuration defines a virtual address space; it is,
ultimately, in all modern, mainstream architectures, described by
\emph{integers} (PTEs).
The use of provenance-free integers to describe such configurations carries
risks, just as with pointers.  Necessarily, the ability to configure the MMU
must be confined to privileged, and necessarily trusted, software; this
software must enforce its intended policies concerning permitted access to
the core's view of physical memory and it must do so with no architectural
safeguards.

Moreover, a (software) system may, as part of timesharing the CPU core,
reprogram the MMU to achieve isolation (and, possibly, controlled
non-isolation) between different `process contexts'.  Further, these
contexts may be dynamic, reshaping their associated MMU configurations
across time.  CHERI capabilities are not explicitly associated with a
particular context and/or time.  As a result, software must ensure that
capabilities are not transmissible improperly%
%
\footnote{The simplest and most restrictive policy is to entirely prevent
transmission of capabilities between contexts.  However, if contexts have
common identically interpreted regions of their address spaces, one could
imagine utility in passing capabilities referencing only these spaces.  Such
passing would, in CHERI's design, necessarily have to go via a software
intermediate rather than more direct passing through the shared region
itself.}
%
from one context to another, nor retained improperly as context mappings
evolve.  Thus, the direct mechanisms available for capability passing within
a single context (including between CHERI compartments therein) are likely
not available for cross-context communication.

A similar story plays out in hardware: `physical' addresses are meaningful
only when paired with a \emph{location}, as bus bridges may remap addresses
in transit from one port to another.  When devices or cores wish to
communicate, they must model the action of the intermediate fabric and
generate (integer) addresses that may not be meaningful locally but will be
at the remote endpoint, across the bus fabric.  Again, all the problems with
integer addresses resurface and are exacerbated by the relatively minimal
protection mechanisms available at the physical bus layer.

For this section, we focus on two cases: software on a CHERI core seeking to
escalate its privilege, and peripheral devices wishing to attack the core
(possibly in cooperation with software).  In both cases, the intended victim
of the attack(s) will be taken to be the CHERI core's trusted computing base
(e.g., a hypervisor).  We restrict our attention to steady-state operation
rather than attacks against the initial bootstrap; that is, we assume that
any would-be attacker was not present during the load of said TCB and that
the \emph{core} itself is trusted to faithfully execute instructions.
Note that these extensions may be added to a system which supports
CHERI for virtual address pointers with no impact to most user mode
software.  These extensions most affect the interfaces between
firmware, hypervisors, and operating system kernels.

\subsection{Capability-Mediated CPU Physical Memory Protection} % <<<

RISC-V has a notion of a Physical Memory Protection (PMP) unit that
validates every (post-virtual-address-translation) memory request issued by
a processor core.  Roughly, for each request, an $n$-way associative lookup
against a table of (region, permissions) pairs is performed, and the request
is authorized only if the table contains a region containing the requested
address and the request is of a type permitted by that region.  For details,
see the RISC-V Privileged Architecture specification~\cite[\S
3.6]{RISCV:Privileged:1.10}.

The control interface to the PMP is, as might be imagined, based on
integers: coarsely speaking, machine-mode code is able to write arbitrary
bits to the PMP table through the core's CSR interface.  Supervisor and user
mode code are not permitted access to the table.  Thus, any code in machine
mode can alter restrictions imposed on supervisor or user memory access, and
so a confused deputy attack on the machine mode could result in privilege
escalation for the supervisor or user programs.  We would prefer to have a
more `least authority'-friendly option.

We propose a `capability-mediated PMP' (CPMP).  Its control interface will
permit table entries to be populated only from valid (tagged) capabilities.
We imagine using a pair of a CSR and a special capability register to
provide row-by-row access to the augmented table.

Because machine-mode code on RISC-V has explicit control over whether
address translation is enabled, a baseline capability-mediated PMP
implementation could repurpose the existing CHERI capability mechanisms and
rely on software to track the distinction between capabilities intended for
use as physical addresses and those intended for use as virtual addresses.
Such an approach runs slightly against the grain of our design principles,
and has limitations; for example, sealed forms must be used if these
capabilities are to be given to supervisor (or user) code.

For these reasons, and to enable a wider series of uses, we envision creating
a new capability provenance \emph{root}.  Capabilities derived from this
root are distinct from existing CHERI capabilities (by, say, having a bit
immutably set that the existing capabilities maintain cleared) and denote
ranges of physical addresses, even in the presence of paging.  Accesses via
these capabilities bypass any paging mechanism and, dually, we can now make
accesses via the existing CHERI capabilities
that
 \emph{always} go via address translation, even in machine mode.%
%
\footnote{This obviates the RISC-V \texttt{mstatus} MPRV mechanism for
toggling address translation.}
%
These capabilities may have their born authority decreased as with any other
CHERI capability, and may flow to non-machine-mode code to enable (for
example) light-weight partitioning of physical resources between multiple
supervisors.

% >>>
\subsection{Capability-Mediated DMA Physical Memory Protection} % <<<

Whereas RISC-V considers PMPs only in the context of a CPU core, nearly
identical hardware can be used to gate peripheral DMA requests.  Here, the
PMP's control interface is exposed to the CPU, most likely as a
memory-mapped region, and the direction of requests is backwards, but the
operation of the device is fundamentally the same.  When presented with a
memory request \emph{by the peripheral}, such a gate performs an associative
scan of the configured table and either permits the request to enter the bus
or rejects the request.  We tentatively call such a gate an IOPMP.

Whereas IOPMPs could be programmed using integers (as in the RISC-V PMPs), or
using existing CHERI capabilities transported over the memory bus, the story
is much more credible if they can require physical-address capabilities.  So
equipped, we reduce the risk of confusion or misbehavior of machine-mode
code but, more excitingly, we gain the possibility of directly exposing
peripheral IOPMPs to non-machine-mode code for efficient device
pass-through.

This story is fairly satisfying for the control of the IOPMP itself;
however, there remains a challenge of translating the authority carried by
the CHERI CPU core into an address suitable for comprehension by the
peripheral.  That is, because the peripheral continues to speak in
\emph{integer} addresses in its control messages, software on the core could
easily treat the peripheral as a confused deputy, causing it to DMA to
regions authorized by, for example, other (software) compartments.  It may
be necessary to limit sharing of peripherals this way, or more directly
involve the IOPMPs in device control.  One could imagine, for example, that
the IOPMP could `back-translate' core-originated capabilities in control
messages into integers for the peripheral's consumption, perhaps with a tag.

% >>>
\subsection{Capability-Based Page Tables} % <<<
\label{app:exp:physcap:ptw}

Traditionally, hypervisors must deny the supervisors they oversee the
ability to directly control the memory translation tables.  Towards the
`paravirtualization' end of the spectrum, the hypervisors require that the
guests make hypercalls to manipulate the page tables.  Towards the
`hardware-assisted' end, the CPU's MMU will use `nested translation':
the `guest physical' addresses manipulated by the guest are subject to
re-translation, through tables controlled by the hypervisor, before becoming
`host physical' addresses and exiting the CPU core.  Both approaches have
substantial costs.

A more radical approach would have us change the traditional memory
management unit (MMU) page tables.  Instead of mapping virtual
addresses to \emph{integer} physical addresses, the page tables would yield
a \emph{physical capability} for a virtual address.  We envision repurposing
the capability permission bits for the PTE permission bits, and extending the
flags field of \cref{sec:model-flags} to encompass non-authority flags
of PTEs, notably including accessed, dirty, and global flags.

To simplify the system, we may require that physical capabilities installed
in page tables have offset zero and length at least a full page (of the
appropriate level of the tree).  This allows us to skip a capability bounds
check when translating a virtual address but retains proof of
\emph{provenance} of the authority to access a given region of physical
addresses.

% >>>
\subsection{Capability-Based Page Tables in IOMMUs} % <<<

As with the PMPs, this new facility also finds use in guarding peripherals.
Rather than the associative table scans of the IOPMPs above, we could have
capability-mediated IOMMUs whose page-table entires, again, contain
physical-address capabilities.  Of course, there is no reason that an IOPMP
expose a 64-bit address space to the peripheral, nor that it use
hierarchical pages.  For many peripherals, a \emph{single} page-sized
aperature (or even smaller) may suffice.  The concern of integer addresses
in peripheral control messages continues to apply.

% >>>
\subsection{Exposing Capabilities Directly To Peripherals} % <<<

Both IOPMPs and IOMMUs, mediated by capabilities or not, continue to expose
an \emph{integer} address space to the peripheral.  While the peripheral may
be using CHERI for its internal computations, its interface with the host
remains capability-less.  In some cases of mutually distrusting peers, this
may suffice, and each side may have capability-mediating devices under its
control to guard the interconnect.

However, in other cases the host may wish to extend the \emph{tagged}
memory bus all the way to the peripheral, and then grant capabilities
directly to the device as though it were a software process.  In such cases,
we expect that an IOPMP- or IOMMU-like guarding device will still be useful,
to prevent a malicious or errant device from synthesizing or retaining (and
subsequently using) capabilities that the host does not intend.  All
capabilities transiting the guard would be checked to be a \emph{subset} of
a capability in the guard's table.  We note, in passing, that such guard
devices are also useful for the case of direct peripheral-to-peripheral
access, not merely the case of peripheral-to-memory as we have generally
focused upon here.  The details of the control interface to such a device,
as well as its internal operation, are left to future work.

% >>>
% >>>
\section{Distributed Capabilities For Peripherals And Accelerators} % <<<
%\tmnote{This section overlaps with the previous and is intentionally being kept separate for now.
%It will need merging in due course.  Commented out until it's a bit more stable}
CHERI's design focuses on the `main' CPU core(s), in which there is a single operating system, and capabilities are used within virtual address spaces, mediated via an MMU and a memory-coherency system.

Many systems are composed of distributed compute elements that share memory.  In various contexts these are termed `peripherals', `DMA engines', `accelerators' or `remote DMA network cards'.  These may be on a single system-on-chip using fabrics such as AXI, or across interconnect such as PCI Express, Thunderbolt or Infiniband.

When capabilities are used in such a system, there is a requirement to
protect them from inappropriate modification by cores that might be outside the purview of the primary operating system.  Additionally, it would be advantageous for such cores to use capabilities for their own code and data, without having to mediate them from centralized authority.  Furthermore, such systems frequently use multiple levels of address translation -- not just a virtual address space (as capabilities in this document primarily refer to), but a patchwork of multiple physical address spaces (including the guest physical address spaces used by hypervisors), as well as virtual address spaces used by accelerators and other cores.

There are two challenges: first, preventing a core from modifying a
capability it does not own, and second, handling the case that capabilities
can alias if they refer to an incorrect address space.

To achieve these goals, we propose several architectural features.

\subsection{Scope and threat model}
This feature assumes that peripherals are capability-aware, in that they are able to load, store and manipulate capabilities and their tags.  A number of scenarios with trustworthy hardware and software, untrustworthy software on trustworthy hardware, or untrustworthy hardware and software may be envisaged.  Hardware that is not capability-aware and uses integers as pointers is out of scope for this extension, although it may be constrained or otherwise capability-wrapped by some other structure.

\subsection{Address-space coloring}
\label{app:exp:coloring}
We may deconstruct systems into regions of address-space colors (ASCs).  A region with a common color has addresses with a single unambiguous meaning.  Generalizing, a color could apply to an application's virtual address space, the system's hardware physical address space, the guest physical address space of a virtual machine, or a piece of memory on a peripheral.
A Processing Element (PE -- processor, DMA engine or other core) is assigned a color based on its physical location in the system topography.

Colors also represent single regions of authority.  Within a colored region, it is assumed that every device that can synthesize a capability has rights to do so.  If a device is untrustworthy, it should be segmented into a different colored region.

An address space may be connected to a different address space via an Address Translation and Protection Unit (ATPU).  Examples of ATPUs might be MMUs, IOMMUs, and hypervisor page translation, but also more limited cases such as PCI BAR mapping or driving upper address bits from a page register.  An ATPU may provide no translation between mutually distrusting hardware that happens to share an address space, but still apply protection between them.

We generalize an ATPU as a bridge by which requests come in from one address space and are dispatched into another.  The ATPU may itself make memory requests to determine the translation, such as when walking page tables.  These would potentially occur in an address space of a third color.

\subsection{Capability coloring}
Capabilities refer to addresses in particular address spaces, hence capabilities are given a color that is stored within the capability.  It is now possible to disambiguate the address within a capability with the address space to which it refers.

\subsubsection{Representation}
We describe architecturally the notion of address space color without specifying the specific representation.
However microarchitecturally we expect that the \cotype{} field in a capability would be reused, based on a tagging scheme to distinguish them from a software-defined \cotype{}.  Given the limited number of bits available in the \cotype{} for the otype-color, it may be impossible to represent all the colors within these bits.  It is not necessary for the otype-color field to be unique, only that it is possible to disambiguate which address space region is referred by a capability.  For example, the upper bits of the address may be used to distinguish two regions with the same otype-color field which are each smaller than 64-bit addressing.  Architecturally such regions would be thought of as having different colors.

\paragraph{otype reuse}
\label{app:exp:otype-reuse}
Since 128-bit capabilities are constrained by size, we propose using the \cotype{} field to represent some or all of the ASC.  To disambiguate from the softwaredefined \cotype{}, the data structure should be tagged.

To avoid reducing the bits for the \cotype{}, we propose a variable-length tag.  This also allows embedding one instance of a variety of other metadata in the \cotype{} field.  For example:

\begin{verbatim}
0x_xxxx_xxxx_xxxx_xxxx: Software-defined otype (17 bits)
10_xxxx_xxxx_xxxx_xxxx: Metadata type A (16 bits)
11_0xxx_xxxx_xxxx_xxxx: Metadata type B (15 bits)
...
11_1110_cccx_xxxx_xxxx: Metadata type E (8 alternatives of 9 bits each)
\end{verbatim}


\subsubsection{Operations on colored capabilities}
Colors are used to enforce policy by processing elements and ATPUs.  A processing element has, in its hardware, an awareness of the color of its local address space.

In this area exist a number of possibilities, subject to further research.

Most conservatively, a PE could deal only with capabilities of its own color.  A capability with another color is treated as if the tag is cleared.  This would allow PEs to use capabilities internally, without sharing between them.  A privileged process (boot loader, management processor, hypervisor, operating system on an application core) is used to generate initial colored capabilities for each PE, from where they are used internally.

In this scenario, capability conversion between colors would be minimal or require a call to the privileged process.
Conversion would require translation between address spaces, with a chance that a capability could not be directly represented in the target address space (if it represents disparate physical pages, for instance).

Other approaches are possible.  For instance, colored capabilities could be treated as sealed.  This would enable devices to be given `handles' to memory in another address space that they cannot access, but can pass around to other data structures.  For instance, networking data structures might contain linked list pointers in network stack address space -- the NIC can build its own linked list, without the ability to access the data being pointed to.  Much care would be required here to avoid confused deputy attacks.

%We consider the following operations by PEs:

%\begin{description}
%\item[Read of a capability, local color.] Allowed.  A PE can make free use of capabilities of its local color.  All the %regular operations on capabilities have their usual effect.
%\item[Write of a capability, local color.] Allowed.  The capability is stored in memory with its color (potentially %transformed by ATPUs as described below).
%\item[Read/write of a capability, non-local color.] Non-local capabilities are treated as sealed.  They can be stored in %memory and passed to other devices without modification.
%\item[Unseal capability, non-local color.] The unseal operation transforms a capability from a foreign address space into %the current address space, converting it to the local color.  This may require a call-out to ATPU(s) to receive the %translation, potentially passing through several ATPUs.
%\item[Seal capability, local color.]  Local use of sealing is still permitted, generating a sealed capability of a local %color.  Given microarchitectural constraints, a smaller \cotype{} space may be available.
%\item[Seal capability, non-local color.] May change to the representation of a sealed capability with the target color.
%\item[Change color on capability.] A local capability may be converted to a capability of a different color by passing it %through ATPU(s), which return the appropriate non-local capability.  The rights on the returned capability may be reduced %compared to the input.
%\end{description}

\subsubsection{Enforcement}
Both PEs and system bridges are tasked with enforcing the capability model:

\paragraph{Bridges} enforce operations on colored capabilities.  For example, a bridge may disallow capabilities of other colors to pass through it.  Bridges are viewed as more trustworthy than devices they connect, although a hierarchy exists -- bridges closer to DRAM are able to disallow capabilities that are accepted by bridges further away.  Bridges have an awareness of whether hardware might be untrustworthy (for instance, plugged in to a motherboard slot or external port) and apply external enforcement of properties where the hardware might be untrustworthy.

ATPUs could also transform capabilities that pass through them according to their address space remapping -- e.g., allowing a PE to store capabilities with its local address space, but remap them to physical addresses when storing to DRAM.\tmnote{Does this make sense? Reversibility? Representability? Need to consider what happens when something looks at the integer address of such a transformed cap.}

%\paragraph{PEs} are tasked with enforcing sealing of non-local capabilities.
%\tmnote{What happens when malicious PE hardware violates the sealing of a non-local cap?  Can the bridge efficiently enforce that?}
\paragraph{PEs} are tasked with enforcing the capability model within their local software.  Bridges enforce colors, but PEs enforce the remainder of the capability model (monotonicity, tagging, etc).  An untrustworthy PE may corrupt its own capabilities, but since the coloring is enforced by the bridge it will only have detrimental effects on its own software.

\subsubsection{Implementation outline}
For a minimalist implementation, the following might be done:

\begin{enumerate}
\item Choose a representation for the bits that indicate a color within a capability
\item Implement a CSR that configures the current color of a processor core.  The
permissions on this register are up for debate but access control might be
similar to that of the IOMMU page table base register - potentially
set externally rather than internal to a PE
\item In user mode, loading or dereferencing capabilities checks the color
matches the current one, and causes an exception if mismatched
\item A privileged instruction takes two capabilities, an input capability
and an authorization capability.  The output of the instruction is the input
capability with the color changed to match the authorization capability.
\tmnote{Is this something that should be restricted by a permission? Or by
a mode?}
\item Checks may be necessary to verify the input capability is a subset of the
authorization capability, bearing in mind they may be in different address
spaces without a contiguous mapping.
\item It may be necessary for a PE that wishes to change the color of its
capability to call out to a more trustworthy component, such as a bridge or
another processor with more authority.  This could be modeled by the PE not
possessing a suitable authorization capability, and thus making an API call to
another PE or ATPU.  ATPUs may implement such translation in hardware to
make it efficient.
\end{enumerate}


\subsection{Shared Host and Peripheral Virtual Address Spaces} % <<<
An alternative to tagging capabilities with an address space color would be
to arrange for all capabilities to represent the same address space, preventing
ambiguity.

Unlike in section \ref{app:exp:physcap}, when an IOMMU is present capabilities
may hold virtual addresses.  The question is how should I/O virtual address
spaces and capabilities interact.

Parallels can be drawn to with the implementation of
capabilities in Unix processes.  There, capabilities are used for fine-grained
protection while the MMU may be used for address translation.
In the I/O space, we can similarly use capabilities for fine-grained
protection of memory objects while using an IOMMU to acheive address
translation between the address spaces of different initiators.  Relegating the
IOMMU to a secondary role gives the opportunity to
simplify its hardware, potentially allowing fewer levels of page translation
or even a non-paged IOMMU.

As with virtually addressed capabilities in operating systems, there comes
the challenge that capabilities without an address-space color (as sketched in
section \ref{app:exp:coloring}) must be distinguished as to which
address space they relate to.  In software the OS is in charge of process
switching, maintains an awareness of the current address space, and prevents
capabilities from leaking between address spaces.  In I/O this is not true:
devices are making transactions independently and without coordination,
hence transactions in different address spaces can and do occur
concurrently.  Unlike traditional IOMMU usage, we are concerned not
just with the address on the memory interconnect (where we can tell the address
space based on the transaction's origin in the system topology) but also
with addresses in memory (which have no clear provenance).

An alternative approach is to use a simplified IOMMU to retain
topological address provenance, but establish a \emph{shared I/O virtual
address space}.
In this way all devices share the same address mappings, but
capabilities are used to restrict those accesses (either by device hardware
being trustworthy and playing by the rules, or externally checking the outputs
of untrusted hardware).

One way to implement this is to ringfence a single range of virtual address
space
in the OS that is set up so that both OS and all peripherals' virtual address
spaces align -- i.e.  an address in this region refers to the same piece of
memory whether it comes from any software on the host or peripheral device.
An analogy may be drawn with operating systems' \emph{direct mapped region},
a chunk of virtual address space where physical memory is mapped contiguously
-- this mapping exists in every process and is never switched out.

This means that any capability that refers to this address space is
unambiguous to what it refers.  Hence there is never a disambiguation
problem and hence no need to color the capabilities.  The downside is that
we cannot use such capabilities with physical addresses (unless the CPU lets
software use physical addresses, which may be true on a microcontroller but
less in an application-class system) and hence translation is required.
Such translation could be relatively lightweight however -- eg simply
adding a fixed offset to a physical address to turn it into a virtual
address in this region.

In this scenario access would require a capability to access a certain range of
shared I/O virtual address space, and a way to securely revoke capabilities
from devices once they were no longer required.  The IOMMU would be limited to
ensuring that devices could not access memory outside the shared address space.

A simplified IOMMU would map this region to physical address space.
Alternatively a more complex page-based IOMMU could instead
dynamically provision this space with physical pages to allow sparse mappings
and contiguous blocks in fragmented physical memory -- but it could do so at a
larger granularity than that of software objects (eg megabyte-sized pages
rather than 4~KiB).  Such design choices would depend on the kinds of accesses
expected by the system designer, but some tradeoffs could be made at runtime --
for example, a traditional paged IOMMU could be populated with 'huge' pages
(saving translation costs) since it is no longer needed to use
the IOMMU for protection of individual objects.  Additionally, by handling
revocation via capabilities rather than in the IOMMU, it reduces IOTLB churn
and invalidation delays.


% >>>
% \section{Details of Proposed Instructions} % <<<
% \label{app:exp:insns}

% The following instructions are described using the same syntax and approach as
% those in Chapter~\ref{chap:isaref-riscv}.

% \clearpage
\phantomsection
\addcontentsline{toc}{subsection}{CAndAddr}
\insnmipslabel{candaddr}
\subsection*{CAndAddr: Mask Address of Capability}

\subsubsection*{Format}

CAndAddr cd, cb, rs

\begin{center}
\cherithreeop[header]{0x24}{cd}{cb}{rs}
\end{center}

\subsubsection*{Description}

\emph{cd} is set to \emph{cb} with \emph{cb}.\caddr{} logically ANDed with \emph{rs}.
If the new address causes the capability to become unrepresentable then an untagged capability with address set to the masked address is returned.

\subsubsection*{Semantics}

\sailMIPScode{CAndAddr}

\subsubsection*{Exceptions}

A coprocessor 2 exception is raised if:

\begin{itemize}
\item
\emph{cb}.\ctag{} is set and \emph{cb} is sealed.
\end{itemize}

\subsubsection*{Notes}

\begin{itemize}
\item This instruction may be appropriate for use cases in which C is masking pointers
before use, for example when the lower bits are used to store additional metadata in a pointer with known alignment.
\item This instructions also allows more efficient aligning down (and up) of capabilities. In C source code this will
almost always be a compile time constant so we might discover that we need a version of this instruction that
takes an immediate operand. Using a 5-bit immediate field we could encode align-down/align-up masks from
$2$ to $2^{32}$.
\item This instruction is also useful in order to create precisely representable bounds for heap and stack allocations.
\end{itemize}

% \clearpage
\phantomsection
\addcontentsline{toc}{subsection}{CBuildCap}
\insnmipslabel{cbuildcap}
\subsection*{CBuildCap: Import a Capability}

\subsubsection*{Format}

CBuildCap cd, cb, ct

\begin{center}
\begin{bytefield}{32}
\bitheader[endianness=big]{0,5,6,10,11,15,16,20,21,25,26,31}\\
\bitbox{6}{0x12}
\bitbox{5}{0x0}
\bitbox{5}{cd}
\bitbox{5}{cb}
\bitbox{5}{ct}
\bitbox{6}{0x1d}
\end{bytefield}
\end{center}

\usesDDCinsteadofNULL{cb}

\subsubsection*{Description}

\insnmipsref{CBuildCap} attempts to interpret the contents of \emph{ct}
as if it were a valid capability (even though \emph{ct}.\ctag{} is not
required to be set and so \emph{ct} might contain any bit pattern) and
extracts its \cbase{}, \clength{}, \coffset{}, \cperms{} and \cuperms{}
fields.
If the bounds of \emph{ct} cannot be extracted because the bit
pattern in \emph{ct} does not correspond to a permitted value of the capability
type (e.g. \clength{} is negative), then an exception is raised.

If the extracted bounds of \emph{ct} are within the bounds of \emph{cb}, and
the permissions of \emph{ct} are within the permissions of \emph{cb}, then
\emph{cd} is set equal to \emph{cb} with the \cbase{}, \clength{},
\coffset{}, \cperms{} and \cuperms{} of \emph{ct}.

If \emph{ct} is sealed, this instruction does not copy its \cotype{} into \emph{cd}. With
compressed capabilities, a different representation may be used for the bounds
of sealed and unsealed capabilities. If \emph{ct} is sealed,
\insnmipsref{CBuildCap} will change the representation of the bounds so
that their values are preserved.

Because \emph{ct}.\ctag{} is not required to be set, there is no guarantee
that the bounds of \emph{ct} will be in canonical form.
\insnmipsref{CBuildCap} may convert the bounds into canonical form rather than
simply copying their bit representation.

\insnmipsref{CBuildCap} does not copy the fields of \emph{ct} that are
reserved for future use.

\insnmipsref{CBuildCap} can be used to set the tag bit on a capability (e.g.,
one whose non-tag contents has previously been swapped to disk and than
reloaded into memory, or during dynamic linking as untagged capability values
are relocated and tagged after being loaded from a file).
This provides both improved efficiency relative to manual rederivation of the
tagged capability via a series of instructions, and also provides improved
architectural abstraction by avoiding embedding the rederivation sequence in
code.

\subsubsection*{Semantics}
\sailMIPScode{CBuildCap}

\subsubsection*{Exceptions}

A coprocessor 2 exception is raised if:

\begin{itemize}
\item
\emph{cd}, \emph{cb} or \emph{ct} is a reserved register and \PCC.\cperms{} does
not grant \emph{Permit\_Access\_System\_Registers}.
\item
\emph{cb}.\ctag{} is not set.
\item
\emph{cb} is sealed.
\item
The bounds of \emph{ct} are outside the bounds of \emph{cb}.
\item
The values of \cbase{} and \clength{} found in \emph{ct} are not within the
range permitted for a capability with its \ctag{} bit set.
\item
\emph{ct}.\cperms{} grants a permission that is not granted by
\emph{cb}.\cperms{}.
\item
\emph{ct}.\cuperms{} grants a permission that is not granted by
\emph{cb}.\cuperms{}.
\end{itemize}

\subsubsection*{Notes}

\begin{itemize}
\item
This instruction acts both as an optimization, and to provide architectural
abstraction in the face of future change to the capability model.
A similar effect, albeit with reduced abstraction, could be achieved by
using \insnmipsref{CGetBase}, \insnmipsref{CGetLen} and
\insnmipsref{CGetPerm} to query \emph{ct}, and then using
\insnmipsref{CSetBounds} and \insnmipsref{CAndPerm} to set the bounds
and \cperms{} of \emph{cd}.
\item
Despite the description of its intended use above, \insnmipsref{CBuildCap}
does not actually require that \emph{ct} have an unset tag.
\item
\emph{ct} might be a sealed capability that has had its \ctag{} bit cleared.
In this case (assuming an exception is not raised for another reason),
\emph{cd} will be unsealed and the bit representation of
the \cbase{} and \clength{} fields might be changed to take account of
the differing compressed representations for sealed and unsealed capabilities.
\item
This instruction can not be used to break security properties of the capability
mechanism (such as monotonicity) because \emph{cb} must be a valid capability
and the instruction cannot be used to create a capability that grants rights
that were not granted by \emph{cb}.
\item
As the tag bit on \emph{ct} does not need to be set, there is no guarantee
that the bit pattern in \emph{ct} was created by clearing the tag bit on a
valid capability. It might be an arbitrary bit pattern that was created by
other means. As a result, there is no guarantee that the bit pattern in
\emph{ct} corresponds to the encoding of a valid value of the capability type,
especially when capability compression is in use. Fields might have values outside
of their defined range, and invariants such as \cbase{} $\ge 0$, \cbase{} $+$
\clength{} $\le 2^{64}$ or \clength{} $\ge 0$ might not be true. In addition,
fields might not be in a canonical (normalized) form.
\insnmipsref{CBuildCap} checks that the \cbase{} and \clength{} fields are
within the permitted range for the type and satisfy the above invariants,
raising a length exception if they are not. If the fields are not in normalized
form, \insnmipsref{CBuildCap} may renormalize them rather than simply
copying the bit pattern from \emph{ct} into \emph{cd}.
\item
The type constraint \emph{cd}.\ctag{} $\implies$ \emph{cd}.\cbase{} $\ge 0$ is
guaranteed to be satisfied
because \emph{cb}.\cbase{} $\ge 0$ and an exception would be raised if
\emph{ct}.\cbase{} $\le$ \emph{cb}.\cbase{}.
\item
The type constraint \emph{cd}.\ctag{} $\implies$ \emph{cd}.\cbase{} $+$
\emph{cd}.\clength{} $\le 2^{64}$ is guaranteed to be satisfied because this
constraint is true for \emph{cb}, and an exception would be raised if
\emph{ct}.\cbase $+$ \emph{ct}.\clength{} $>$ \emph{cb}.\cbase{} $+$
\emph{cb}.\clength{}.
\item
Is the value of \emph{cd} guaranteed to be representable?
If \emph{ct} was created by clearing the tag bit on a capability, then its
bounds can be represented exactly and there will be no loss of precision.
If \emph{ct} is sealed, then there is a potential issue that the values
of the bounds that are representable in a sealed capability are not the same as
the range of bounds that are representable in an unsealed capability. We rely
on a property of the existing capability formats that if a value of the bounds
is representable in a sealed capability, then it is also representable in an
unsealed capability.
\item As \insnmipsref{CBuildCap} is not able to restore the seal on a
re-tagged capability, it is intended to be used alongside
\insnmipsref{CCSeal}, which will conditionally seal a capability based on a \cotype{} value
extracted with \insnmipsref{CCopyType}.
These instructions will normally be used in sequence to \textit{(i)} re-tag a
capability with CBuildCap, \textit{(ii)} extract a possible object type from
the untagged value with \insnmipsref{CCopyType}, and \textit{(iii)}
conditionally seal the resulting capability with \insnmipsref{CCSeal}.
\item
The typical use of \insnmipsref{CBuildCap} assumes that there is a single
capability \emph{cb} whose bounds include every capability value that is
expected to be encountered in \emph{ct} (with out of range values being an
error). The following are two examples of situations where this is not the
case, and the sequence of instructions to recreate a capability might need
to decide which capability to use as \emph{cb}:
(a) The operating system has enforced a ``write xor execute'' policy, and
the program attempting to recreate \emph{ct} has a capability with
\emph{Permit\_Write} permission and a capability with \emph{Permit\_Execute}
permission, but does not have a capability with both permissions.
(b) The capability in \emph{ct} might be a capability that authorizes sealing
with the \emph{Permit\_Seal} permission, and the program attempting to
recreate it has a capability for a range of memory addresses and a capability
for a range of \cotype{} values, but does not have a single capability that
includes both ranges.
\end{itemize}

\mrnote{TO DO: We can solve the problem of multiple authorizing capabilities
if CBuildCap does nothing if the source already has the tag set, or if the
authorizing capability does not grant sufficient rights. Then, you can just
try all of the authorizing capabilties in turn and see if any of them worked.}

% \clearpage
\phantomsection
\addcontentsline{toc}{subsection}{CClearTags}
\insnmipslabel{ccleartags}
\subsection*{CClearTags: Zero Multiple Capability Tags in Memory}

\subsubsection*{Format}

CClearTags rd, cb \\

\begin{center}
\begin{bytefield}{32}
\bitheader[endianness=big]{0,5,6,10,11,15,16,20,21,25,26,31}\\
\bitbox{6}{0x12}
\bitbox{5}{0x00}
\bitbox{5}{cb}
\bitbox{5}{0x18}
\bitbox{5}{0x1F}
\bitbox{5}{0x3F}
\end{bytefield}
\end{center}

\usesDDCinsteadofNULL{cb}

\subsubsection*{Description}

The \emph{tags} occupying the \emph{\insnmipsref{CLoadTags} stride} within
memory with base \emph{cb} are zeroed.  The result of this instruction must be
coherent with other processors, \emph{as if} the corresponding data memory
words had been stored, untagged, with their current contents.  The number of
tags zeroed is an implementation-defined constant but is equal to the number of
tags returned by \insnmipsref{CLoadTags}.

Capability register \emph{cb} must contain a capability that grants permission
to store data; capability storage is not required as the bits could just as
well be cleared by \insnmipsref[csbhwd]{CSD}.  The virtual address \emph{cb.\cbase{}}
$+$ \emph{cb.\coffset{}} must be suitably aligned, as with
\insnmipsref{CLoadTags}.  If any tag to be cleared corresponds to memory out of
bounds of \emph{cb}, a length violation is indicated.

\subsubsection*{Semantics}

\sailMIPScode{CClearTags}

\subsubsection*{Exceptions}

A coprocessor 2 exception is raised if:

\begin{itemize}
\item
\cchecktag{}
\item
\emph{cb.\ctag{}} is clear.
\item
\emph{cb} is sealed.
\item
\emph{cb}.\cperms.\emph{Permit\_Store} is not set.
\item
\emph{cb}.\cbase{} $+$ \emph{cb}.\coffset{} + $n$ $*$ \emph{capability\_size} $>$ \emph{cb}.\cbase{} $+$ \emph{cb}.\clength{},
		where $n$ is the number of capabilities to be cleared (\isail{caps_per_cacheline} in the Sail code).
\item
\emph{cb}.\cbase{} $+$ \emph{cb}.\coffset{} $<$ \emph{cb}.\cbase{}.
\end{itemize}

An address error during load (AdEL) exception is raised if:

\begin{itemize}
\item
The virtual address \emph{addr} is not $n$ $*$ \emph{capability\_size} aligned.
\end{itemize}

% \clearpage
\phantomsection
\addcontentsline{toc}{subsection}{CCopyType}
\insnmipslabel{ccopytype}
\subsection*{CCopyType: Import the otype field of a Capability}

\subsubsection*{Format}

CCopyType cd, cb, ct

\begin{center}
\begin{bytefield}{32}
\bitheader[endianness=big]{0,5,6,10,11,15,16,20,21,25,26,31}\\
\bitbox{6}{0x12}
\bitbox{5}{0x0}
\bitbox{5}{cd}
\bitbox{5}{cb}
\bitbox{5}{ct}
\bitbox{6}{0x1e}
\end{bytefield}
\end{center}

\subsubsection*{Description}


\insnmipsref{CCopyType} attempts to interpret the contents of \emph{ct} as
if it were a valid capability (even though \emph{ct}.\ctag{} is not required
to be set, and so might contain any bit pattern),
and extracts its \cotype{} field.
If \emph{ct} is sealed, \emph{cd} is set to \emph{cb} with its \coffset{} field set to
\emph{ct}.\cotype{} $-$ \emph{cb}.\cbase{}. If \emph{ct} is not sealed,
\emph{cd} is set to the NULL capability with its \cbase{} $+$ \coffset{} fields
set to $-1$.

\subsubsection*{Semantics}
\sailMIPScode{CCopyType}

\subsubsection*{Exceptions}

A coprocessor 2 exception is raised if:

\begin{itemize}
\item
\emph{cd}, \emph{cb} or \emph{ct} are reserved registers and \PCC{} does not
grant \emph{Access\_System\_Registers} permission.
\item
\emph{cb}.\ctag{} is not set.
\item
\emph{cb} is sealed.
\item
\emph{ct}.\cotype{} is outside the bounds of \emph{cb}.
\end{itemize}

\subsubsection*{Notes}

\begin{itemize}
\item
The intended use case for this instruction is as part of a routine for
resetting the tag bit on a capability that has had its tag bit cleared
(e.g. by being swapped out to disk and then back into memory).

It is a requirement of this specification that if a capability has its
tag bit cleared (either with \insnmipsref{CClearTag} or by copying it as
data), and \insnmipsref{CCopyType} is used to extract the \cotype{} field
of the result, then \emph{cd}.\cbase{} $+$ \emph{cd}.\coffset{} will be
equal to the \cotype{} of the original capability if it was sealed,
and \emph{cd}.\coffset{} will be -1 if the original capability was not
sealed.
\item
Typical usage of this instruction will be to use \insnmipsref{CBuildCap} to
extract the bounds and permissions of a capability, \insnmipsref{CCopyType}
to extract the \cotype{}, and then use \insnmipsref{CCSeal} to seal the
result of the first step with the correct \cotype{}.
\item
This instruction is an optimization. A similar effect could be achieved by
using \insnmipsref{CGetType} to get \emph{ct}.\cotype{} and then
\insnmipsref{CSetOffset} to set \emph{cd}.\coffset{}.
\item
-1 is not a valid value for the \cotype{} field, so the result distinguishes
between the case when \emph{ct} was sealed and the case when it was not
sealed.
\item
If \emph{ct} is sealed and an exception is not raised, then the result is
guaranteed to be representable, because the bounds checks ensure that
\emph{cd}'s cursor is within its bounds.
\item
If \emph{ct}.\cotype{} is outside of the bounds of \emph{ct}, this is an
error condition (attempting to reconstruct a capability that \emph{cb} does
not give you permission to create). In order to catch this error condition
near to where the problem occurred, we raise an exception. This also has
the benefit of avoiding the case where changing \emph{cb}'s \coffset{}
results in a value that is not representable, as explained in the previous
note.
\end{itemize}

\mrnote{What if \emph{cb} covers only a portion of the \cotype{} space,
and \emph{ct}'s \cotype{} doesn't fall within it? As currently defined,
we raise an exception. But what if we want to detect this error and handle
it somehow? What if we have several capabilities covering different portions
of the type space, and want to try them in turn until one works?}

% \clearpage
\phantomsection
\addcontentsline{toc}{subsection}{CCSeal}
\insnmipslabel{ccseal}
\subsection*{CCSeal: Conditionally Seal a Capability}

\subsubsection*{Format}

CCSeal cd, cs, ct

\begin{center}
\begin{bytefield}{32}
\bitheader[endianness=big]{0,5,6,10,11,15,16,20,21,25,26,31}\\
\bitbox{6}{0x12}
\bitbox{5}{0x0}
\bitbox{5}{cd}
\bitbox{5}{cs}
\bitbox{5}{ct}
\bitbox{6}{0x1f}
\end{bytefield}
\end{center}

\subsubsection*{Description}

If \emph{ct}.\ctag{} is false or \emph{ct}.\cbase{} $+$ \emph{ct}.\coffset{}
$= -1$, \emph{cs} is copied into \emph{cd}.
Otherwise, capability register \emph{cs} is sealed with an \cotype{} of
\emph{ct}.\cbase{} $+$ \emph{ct}.\coffset{}
and the result is placed in \emph{cd} as follows:

\begin{itemize}
\item
\emph{cd} is sealed with \emph{cd}.\cotype{} set to \emph{ct}.\cbase{} + \emph{ct}.\coffset{};
\item
and the other fields of \emph{cd} are copied from \emph{cs}.
\end{itemize}

\emph{ct} must grant \emph{Permit\_Seal} permission, and the new \cotype{}
of \emph{cd} must be between \emph{ct}.\cbase{} and \emph{ct}.\cbase{} $+$
\emph{ct}.\clength{} $-$ 1.

\subsubsection*{Semantics}
\sailMIPScode{CCSeal}

\subsubsection*{Exceptions}

A coprocessor 2 exception is raised if:

\begin{itemize}
\item
\emph{cs}.\ctag{} is not set.
\item
\emph{cs} is sealed.
\item
\emph{ct}.\ctag{} is set and \emph{ct} is sealed.
\item
\emph{ct}.\cperms.\emph{Permit\_Seal} is not set.
% \item
% \emph{ct}.\cperms.\emph{Permit\_Execute} is not set.
\item
\emph{ct}.\ctag{} and \emph{ct}.\coffset{} $\ge$ \emph{ct}.length{}
\item
\emph{ct}.\ctag{} and \emph{ct}.\cbase{} $+$ \emph{ct}.\coffset{} $> \emph{max\_otype}$
\item
The bounds of \emph{cb} cannot be represented exactly in a sealed capability.
\end{itemize}

\subsubsection*{Notes}

\begin{itemize}
\item
If capability compression is in use, the range of possible (\cbase{},
\clength{}, \coffset{}) values might be smaller for sealed capabilities
than for unsealed capabilities. This means that \insnmipsref{CCSeal}
can fail with an exception in the case where the bounds are no longer
precisely representable.
\item
This instruction provides two means of indicating that the capability should
not be sealed: either clearing the tag bit on \emph{ct} or setting \emph{ct}'s
cursor to $-1$. A potential problem with just using a cursor of $-1$ (rather than
clearing the tag bit) to disable sealing is that, depending on \emph{ct}'s
\cbase{} and \coffset{}, setting \emph{ct}'s cursor to $-1$ might have a result
that is not representable. However, the NULL capability has \ctag{} clear
and can always have its cursor set to $-1$. (We implement casts from
\verb+int+ to \verb+int_cap_t+ by setting the cursor of NULL to the value of the
integer, and so this can hold a value of $-1$.) Directly clearing \emph{ct}'s
tag to indicate that sealing should not be performed will work, because it
is always possible to clear the tag bit. Setting \emph{ct}'s cursor to $-1$ with
\insnmipsref{CSetOffset} to indicate that sealing should not be performed will
also work, because this will either set the cursor to $-1$ or (if the result would
not be representable) both clear the tag bit and set the cursor to $-1$. The
latter method may be preferred in a code sequence that extracts the \cotype{}
of a capability with \insnmipsref{CGetType}, getting a value of $-1$ if the
capability is not sealed, setting the cursor of \emph{ct} to the result, and
then using \insnmipsref{CCSeal} to seal a new capability to the same
\cotype{} as the original.
\end{itemize}

% \clearpage
\phantomsection
\addcontentsline{toc}{subsection}{CGetAndAddr}
\insnmipslabel{cgetandaddr}
\subsection*{CGetAndAddr: Get Address of Capability with mask}

\subsubsection*{Format}

CGetAndAddr rd, cb, rs

\begin{center}
\cherithreeop[header]{0x23}{rd}{cb}{rs}
\end{center}

\subsubsection*{Description}

\emph{rd} is set to \emph{cb}.\caddr{}  logically ANDed with \emph{rs}.

\subsubsection*{Semantics}

\sailMIPScode{CGetAndAddr}

\subsubsection*{Notes}

\begin{itemize}
\item This instruction may be useful when C is using the lower bits of a pointer with known alignment to store additional metadata.
\item This instruction is also useful when using a capability type as a union of pointers, integers and floating pointer numbers using a NaN-boxing representation.
\item In a merged register file this instruction would not be necessary since the integer bitwise and could be used instead.
\end{itemize}

% \clearpage
\phantomsection
\addcontentsline{toc}{subsection}{CLCNT}
\insnmipslabel{clcnt}
\subsection*{CLCNT: Non-Temporal (Streaming) Load Capability via Capability}

\subsubsection*{Format}

CLCNT cd, rt(cb) \\

\begin{center}
\begin{bytefield}{32}
\bitheader[endianness=big]{0,5,6,10,11,15,16,20,21,25,26,31}\\
\bitbox{6}{0x12}
\bitbox{5}{0x00}
\bitbox{5}{cd}
\bitbox{5}{cb}
\bitbox{5}{rt}
\bitbox{6}{0x38}
\end{bytefield}
\end{center}

\usesDDCinsteadofNULL{cb}

\subsubsection*{Description}

Architecturally equivalent to \insnmipsref{CLC} (q.v.), this instruction is
intended as a hint to the cache fabric that the loaded line is expected to be
used only a few times.  See \cref{app:exp:clcnt} for rationale and more
discussion.

\subsubsection*{Semantics}
\sailMIPScode{CLCNT}

% \clearpage
\phantomsection
\addcontentsline{toc}{subsection}{CLShC}
\insnmipslabel{clshc}
\insnmipslabel{clshci}
\insnmipslabel{clshcr}
\subsection*{CLShC: Load Short Capability via Capability}

\subsubsection*{Format}

CLShC cd, rt, offset(cb) \\
CLShCR cd, rt(cb) \\
CLShCI cd, offset(cb)

\usesDDCinsteadofNULL{cb}

\subsubsection*{Description}

Capability register \emph{cd} is loaded with the decoded form of the short
capability located at the memory location specified by
%
\emph{addr} $\stackrel{\Delta}{=}$ \emph{cb.\cbase{}} $+$
\emph{cb.\coffset{}} $+$ \emph{rt} $+$ 8 $*$ \emph{offset},
%
provided this address is suitably aligned for short capabilities.
%
Capability register \emph{cb} must contain a capability that grants
permission to load capabilities.

The tag bit associated with \emph{cd} is set if the tag bits associated with
\emph{addr} indicate that this address contain a short capability and
\emph{cb} bears Permit\_Load\_Capability.  See
\cref{sec:windowedshortcaps} for possible tag representations.

% \subsubsection*{Semantics}

\subsubsection*{Exceptions}

A coprocessor 2 exception is raised if:

\begin{itemize}
\item
\cchecktag{}
\item
\emph{cb} is sealed.
\item
\emph{cb}.\cperms.\emph{Permit\_Load} is not set.
\item
\emph{addr} + \emph{short\_capability\_size} $>$ \emph{cb}.\cbase{} $+$ \emph{cb}.\clength{}.
\item
\emph{addr} $<$ \emph{cb}.\cbase{}.
\end{itemize}

An address error during load (AdEL) exception is raised if:

\begin{itemize}
\item
The virtual address \emph{addr} is not \emph{capability\_size} aligned.
\end{itemize}

\subsubsection*{Notes}

\begin{itemize}
\item
\emph{offset} is interpreted as a signed integer.
\item
The \insnmipsref{CLShCI} mnemonic is equivalent to \insnmipsref{CLShC} with
\emph{cb} being the zero register (\$zero). The \insnmipsref{CLShCR} mnemonic
is equivalent to \insnmipsref{CLShC} with \emph{offset} set to
zero.
\item
Although the \emph{short\_capability\_size} may vary, the offset is always in
multiples of 8 bytes (64 bits).
\item
See \cref{sec:windowedshortcaps} for details of short capabilities.  In
particular, note that several permissions are not transported in short
capabilities.
\end{itemize}

% \clearpage
\phantomsection
\addcontentsline{toc}{subsection}{CSShC}
\insnmipslabel{csshc}
\subsection*{CSShC: Store Short Capability via Capability}

\subsubsection*{Format}

CSShC cs, rt, offset(cb) \\
CSShCR cs, rt(cb) \\
CSShCI cs, offset(cb)

\usesDDCinsteadofNULL{cb}

\subsubsection*{Description}

Capability register \emph{cs} is encoded to a short capability form and
stored at the memory location specified by
%
\emph{addr} $\stackrel{\Delta}{=}$ \emph{cb.\cbase{}} $+$ \emph{cb.\coffset{}} $+$ \emph{rt} $+$ 16 $*$
\emph{offset},
%
provided \emph{addr} is suitably aligned for short capabilities.
%
Capability register \emph{cb} must contain a capability that grants
permission to store capabilities.
%
The bits in the tag memory associated with \emph{addr} are updated to
indicate a short capability resides at this location if \emph{cs.\ctag{}} is
set and encoding was successful.  See \cref{sec:windowedshortcaps} for
possible tag representations.

\subsubsection*{Encoding}

As per \cref{sec:windowedshortcaps}, several properties must hold of the
capabilities given to \insnmipsref{CSShC} for a successful store to take
place:

\begin{itemize}
  \item \emph{cs} may be out of bounds by strictly less than 4 GiB.
  \item The bounds of \emph{cs} must be representable in short capabilities.
  \item If \emph{cs} is sealed, its \cotype{} must be representable in short
        capabilities.
\end{itemize}

If any of the above tests fail, the store will take place but will update
the tags corresponding to \emph{addr} to indicate that the memory contains
data.

Further, the following permission bits may not be stored in the short
capability format, and may read back (via \insnmipsref{CLShC}) as false:
Permit\_Seal, Permit\_Unseal, Permit\_SetCID,
Permit\_Access\_System\_Registers, and Permit\_Store\_Local\_Capability.
Software must, however, not depend on \insnmipsref{CSShC} to clear these
bits; an \insnmipsref{CAndPerm} must be used to ensure that these rights
are discarded.

% \subsubsection*{Semantics}

\subsubsection*{Exceptions}

A coprocessor 2 exception is raised if:

\begin{itemize}
\item
\cchecktag{}
\item
\emph{cb} is sealed.
\item
\emph{cb}.\cperms.\emph{Permit\_Store} is not set.
\item
\emph{cb.\cperms.Permit\_Store\_Capability} is not set.
\item
\emph{cb.\cperms{}.Permit\_Store\_Local} is not set and
\emph{cs.tag} is set and \emph{cs.\cperms{}.Global} is not set.
\item
\emph{addr} $+$ \emph{short\_capability\_size} $>$ \emph{cb}.\cbase{} $+$ \emph{cb.\clength{}}.
\item
\emph{addr} $<$ \emph{cb}.\cbase{}.
\item
\emph{cs.\cbase{}}, \emph{cs.\cbase{}} $+$ \emph{cs.\clength{}}, and
the store's target memory location computed above differ in their top 32 bits.
\end{itemize}

\noindent
A TLB Store exception is raised if:

\begin{itemize}
\item
\emph{cs.\ctag{}} is set and the \emph{S} bit in the TLB entry for the page
containing \emph{addr} is not set.
\end{itemize}

\noindent
An address error during store (AdES) exception is raised if:

\begin{itemize}
\item
The virtual address \emph{addr} is not \emph{short\_capability\_size} aligned.
\end{itemize}

\subsubsection*{Notes}

\begin{itemize}
\item
If the address alignment check fails and one of the security checks fails,
a coprocessor 2 exception (and not an address error exception) is raised.
The priority of the exceptions is security-critical, because otherwise a
malicious program could use the type of the exception that is raised to
test the bottom bits of a register that it is not permitted to access.
\item
\emph{offset} is interpreted as a signed integer.
\item
The \insnnoref{CSShCI} mnemonic is equivalent to \insnmipsref{CSShC} with
\emph{cb} being the zero register (\$zero). The \insnnoref{CSShCR} mnemonic
is equivalent to \insnmipsref{CSShC} with \emph{offset} set to zero.
\item
Although the \emph{short\_capability\_size} can vary, the offset is always in
multiples of 8 bytes (64 bits).
\end{itemize}

% \clearpage
\phantomsection
\addcontentsline{toc}{subsection}{CTestSubset}
\insnmipslabel{ctestsubset}
\subsection*{CTestSubset: Test that Capability is a Subset of Another}

\subsubsection*{Format}

CTestSubset rd, cb, ct

\begin{center}
\begin{bytefield}{32}
\bitheader[endianness=big]{0,5,6,10,11,15,16,20,21,25,26,31}\\
\bitbox{6}{0x12}
\bitbox{5}{0x0}
\bitbox{5}{rd}
\bitbox{5}{cb}
\bitbox{5}{ct}
\bitbox{6}{0x20}
\end{bytefield}
\end{center}

\usesDDCinsteadofNULL{cb}

\subsubsection*{Description}

\insnmipsref{CTestSubset} tests if the bounds of \emph{ct} are within the
bounds of \emph{cb}, and the permissions of \emph{ct} are within the permissions
of \emph{cb}, setting \emph{rd} to \emph{1} if so, and \emph{0} if not.

\note{mr101}{What should happen if one of the capabilities is NULL? Is the
NULL capability a subset of a valid capability? Comparing the \cbase{} and
\clength{} fields when the \ctag{} bit is clear seems the wrong thing to do}

\note{mr101}{Is a zero length capability a subset of a valid capability,
even if it's \cbase{} is not within the range. If you view the operation as
a subset of the memory bytes, then it is a subset}

\note{mr101}{The motivating use case is \insnmipsref{CToPtr},
\insnmipsref{CTestSubset}, \insnnoref{MOVZ reg, zero}. In error cases
where \insnmipsref{CToPtr} has returned zero, we don't really care what
\insnmipsref{CTestSubset} returns, because we're going to get the NULL
pointer anyway.}

\subsubsection*{Semantics}
\sailMIPScode{CTestSubset}

\subsubsection*{Exceptions}

A coprocessor 2 exception is raised if:

\begin{itemize}
\item
\emph{cd}, \emph{cb} or \emph{ct} is a reserved register and \PCC.\cperms{} does
not grant \emph{Permit\_Access\_System\_Registers}.
\end{itemize}

\subsubsection*{Notes}

\begin{itemize}
\item
This instruction was originally motivated as an additional check for
\insnmipsref{CToPtr}.
A conversion of a capability to a pointer with respect to a default capability
would normally expect that the entire capability is accessible within the
default capability with (at least) the original permissions.
\insnmipsref{CTestSubset} can perform this assertion, and a \insnmipsref{CMove}
instruction can replace the result of the \insnmipsref{CToPtr} with NULL upon
failure.
\item
Another use case for this instruction is in garbage collection. For this
application, we want to be able to test if one capability is a subset of
the other even if one is sealed and the other is not. (For the purposes of
garbage collection, a sealed reference to a region of memory is still a
reference to that region of memory). With compressed capabilities, the bounds
are represented differently for sealed and unsealed capabilities, but
\insnmipsref{CTestSubset} is still able to perform the subset check.
\end{itemize}


% >>>

% vim: foldmethod=marker:foldmarker=<<<,>>>

% >>>
% >>>
\section{Compartment ID Sealing} % <<<

\subsection{Motivation}

Compartment identification is essential. Code and data associated with a compartment must be somehow identified and validated when they are being accessed.
The most natural way seems to be to give them identification numbers.
This has traditionally been done in software, e.g., processes are assigned process IDs (PIDs) in operating systems. In some cases, a software ID can correspond to a hardware ID, e.g., PIDs can correspond to address space identifiers (ASIDs).
However, that is not necessarily the case.
In this proposal, we suggest an architectural ID for compartments.
We call these numbers compartment IDs (CIDs). That means that one CID is mapped to one compartment at a time.
An architecturally defined CID needs to be implemented by hardware.

\subsection{Storage}
\label{subsec:storage}

The first question to tackle is where to store a CID. We envision multiple options for that.
First, one can use a dedicated register for compartment identification as done by the Morello architecture.
Second, one can import CIDs into the capability format, which is what we propose in this document.
Having this solution leads to an atomic change of CID and code capability -- a property we envision to be useful for secure compartmentalisation.

The CID will substitute the otype bits (18 bits) in the 128-bit capability format.
The otype bits are currently not well-used. The otype field indicates whether a capability is sealed and if so which ``type'' it has.
There exists currently two fixed values that are used: one for unsealed capabilities and one for sealed entry (sentry) capabilities.
There are 14 more reserved fixed values, which are not currently used. All other values are used as values for sealed capabilities.

However, there is also the possibility to combine otypes and CIDs.
One potential approach is to subdivide the 18 otype bits such that there is space for a few otypes and the remainder of the bits is dedicated to CIDs.
This would allow sentries and CID sealing to be combined.

In the following text, we propose to completely substitute otypes.
All values but the zero CID (CID==0) are valid IDs.
Furthermore, every CID sealed code capability already is a sealed entry capability.
Please note that every capability within a compartment can be manipulated by the compartment itself.
This also includes CID sealed code capabilities.
In the CHERI world up to now, sentries cannot be manipulated.

An extension would be to add an additional bit indicating whether a capability is a sentry.
This will make capabilites immutable even within a compartment.
The only way to unseal a sentry is to jump to it.

At the moment, this CID sealing proposal is limited to the 128-bit capability format.
However, the underlying mechanism to add CID bits to the meta information bits of the capability works for every format.
In the 128-bit capability format, we propose to allocate 10 bits to the compartment ID, but we envision any number of bits fitting in the meta information bits to be a valid implementation of CID Sealing.
This length is also known as the ACID\_LENGTH, and keep the remaining 8 bits reserved, which allows for 1024 CIDs to be encoded.
These CIDs are referred to as the Architectural CIDs (ACIDs).
Software may choose to virtualise these and can create Software CIDs (SCIDs), which are a concept similar to PIDs.
A SCID may consist of the corresponding ACID added to other bits in order to create a virtual identifier or can be fully independent of the ACID it is mapped to.

Compartment IDs come with two new instructions.
One instruction for reading the CID into a general purpose register and one instruction for setting the CID of a capability.
Like capabilities themselves, CIDs are not considered secret.
Therefore, the CID reading instruction is not privileged, but the writing instruction is restricted by the PERMIT\_SET\_CID permission.
The PERMIT\_SET\_CID bit is a hardware permission bit that, if set in a PCC, allows manipulating the ACID.
Like all hardware permission bits in CHERI, PERMIT\_SET\_CID is constrained by monotonicity.
A code capability with this bit set should only be available to supervisor code.
We suggest encodings for the instructions in order to demonstrate the adaptability of CID to the currently existing CHERI-RISC-V ISA.

\texttt{CSetCID cd, rs0}:\\
set the CID in cd to the value in rs0.
This instruction needs the PERMIT\_SET\_CID bit set, otherwise it throws an exception.
This instruction uses the ACID\_LENGTH lower bits of rs0 and ignores the upper bits.

Possible encoding (31:0): 0x7f, 0x19, rs0, 0x0, cd, 0x5b (assign a random free funct5)

\texttt{CGetCID rd, cs1}:\\
extract the CID of cs1 and store it in rd

Preliminary encoding (31:0): 0x7f, 0x1, cs1, 0x0, rd, 0x5b (this is the same encoding as CGetType, which will be obsolete with this proposal)

An alternative way of enabling sealing not just constrained by the PERMIT\_SET\_CID bit is to use non-memory capabilities.
Comparable to the mechanism already present for otypes in the CHERI-RISC-V ISA, we envision a more fine-grained mechanism.
We propose to create capabilities that authorise for another capability to be sealed with a CID being in a certain range of CIDs.
This authorising capability has the same fields as a conventional CHERI memory capability, but uses its fields differently.
The address field is used as the CID another capability is sealed with and the bounds allow for the range of CIDs this authorising capabilities allows.

\subsection{Sealing}

The current CID (the CID of the current compartment) is determined by the CID in the PCC, which is also a capability.
If the PCC changes and the new PCC has a different CID, this constitutes a compartment change (see Section~\ref{subsec:comp_change}).

With an ACID being in all capabilities, we can establish a concept referred to as CID Sealing.
All capabilities are implicitly sealed by their CID.
Capabilities are considered unsealed if and only if: their CID matches \texttt{PCC.CID} or their CID is 0. All other capabilities are implicitly sealed.
An implicitly sealed capability can be inspected (its fields can be read), but it cannot be manipulated nor can it be used to reference memory.

CID sealing allows a register file to hold capabilities from different compartments without allowing capability leaks.
The currently executing compartment can only make use of its own capabilities or explicitly unsealed ones.

In this proposal, CID sealing completely replaces the sealing mechanism currently present in CHERI.
A CID sealed capability can only be manipulated by the compartment it is owned by.
Therefore, it can be securely handed out to other compartments, e.g., as a code pointer back.

However, CID sealing can also co-exist with current sealing mechanisms in place, e.g., sentries, as discussed in Section~\ref{subsec:storage}.
In comparison to conventional CHERI-RISC-V sealing, CID sealing cannot produce code pointers that are immutable within a compartment, but only across compartments.

\subsection{Compartment Change}
\label{subsec:comp_change}

A compartment change is done when the CID is changed.
The CID can be changed by installing a code capability with a different CID into the PCC register.
Installing a new PCC can be facilitated in many ways, e.g., by jumping to a capability using the CJALR instruction.
We envision there to exist multiple ways of changing the PCC and the CID sealing mechanism is independent of the concrete way.
Once the new PCC is installed, the new compartment needs to bootstrap.
We envision two ways for that, which only differ in the way they retrieve their initial data capability.

First, the new PCC can have further capabilities in its global space, which can be loaded into the register file.
This can be facilitated by the AUIPCC instruction.
The following instruction sequence loads a capability at an example fixed offset:

\texttt{auipcc ct0, 2}\\
\texttt{clc cs0, 0x100(ct0)}

Second, a new data capability can be installed alongside the new PCC.
This gives the compartment a capability ready to use in the register file.
One option for this kind of compartment transition would be indirect sealed capability pairs.
The following code example shows the brevity of this approach.
The instruction sequence was shrunk to just one instruction reading from the new data capability(ct6).

\texttt{clc cs0, 0(ct6)}

\subsection{Sharing}

Sealing explicitly forbids sharing capabilities between compartments.
However, sometimes this behaviour is desired by software. We have designed two mechanisms for sharing capabilities between compartments:

\begin{itemize}
\item Explicit unsealing: This will explicitly unseal a capability and it can be shared either via the register file or via memory.
Another compartment can pick up this capability and seal it again and use it.
This does not protect against transitive capability leaks - an unsealed capability can be passed on to a third compartment and thus be leaked.
\item CID spaces: CIDs can be separated into CID spaces (see Section~\ref{subsec:cid_spaces}).
One approach is to allow unsealing within a CID space.
This allows all compartments in that space to use that capability, but no other compartment can unseal the capability.
This protects from transitive capability leaks outside of the CID space.
As an addition, we envision a bit that -- if set -- allows sharing within the CID space.
\end{itemize}

Furthermore, we also envision the possibility of \textit{resealing} capabilities.
A capability belonging to one compartment can be resealed to another compartment.
This effectively means that code can seal one or more of its own capabilities for another compartment.
The main advantages are that this avoids capabilities to be unsealed in the open as done with explicit unsealing as well as potential performance improvements avoiding the need to unseal and seal capabilites.
However, resealing brings the disavantage for the receiving compartment of not knowing whether a capability was sealed by itself or another compartment.
Therefore, we would expect the need for additional validation checks.

\subsection{Explicit Unsealing}

We need to reserve a CID value that represents an unsealed capability instead of a valid compartment ID.
We can choose any value to represent an unsealed capability.
In this proposal, a capability is unsealed if and only if  its ACID is set to 0.
This ACID is also known as the zero ACID. A capability with the zero ACID can be CID sealed by any compartment.
This introduces two new instructions (in comparison to the three-operand operations currently present in the CHERI-RISC-V, our proposed operations have two operands. The third implicit operand is PCC):

\texttt{CCIDUnseal cd, cs1}\\
If cs1.CID and PCC.CID match, then cs1 will be assigned to cd with cd.CID=0. Otherwise, clear the tag of cs1 and assign it to cd.

Preliminary encoding(31:0): 0x7f, 0x1a, rs0, 0x0, cd, 0x5b (assign a random free funct5)

\texttt{CCIDSeal cd, cs1}\\
If cs1.CID==0, then cs1 will be assigned cd with cd.CID=PCC.CID. Otherwise, clear the tag of cs1 and assign it to cd.

Preliminary encoding(31:0): 0x7f, 0x1b, rs0, 0x0, cd, 0x5b (assign a random free funct5)

A capability with CID==0 can be used by any compartment.
An unsealed capability can be misused and the system can become victim to transitive capability leaks, e.g., by code that carelessly stores capabilities in shared memory.

Having the zero CID represent an unsealed capability seems intuitive and allows code unaware of CID sealing to operate correctly.
In contrast, with otypes, the unsealed capability is represented by $-1$, which expands to all 1s in the otype bits.

\subsection{CID Spaces}
\label{subsec:cid_spaces}

We envision that CIDs can build subspaces in order to model relationships between compartments.
One possibility is to put compartments with identical upper bits into the same group, e.g., CID\_Group = 0b10101010xx.
This would lead to a CID group of size 4.

CID spaces can express trust relationships between compartments, e.g., capabilities within a CID space are implicitly unsealed in any compartment of that CID space.

A further refinement to this is to make the mask configurable with which the CID spaces are defined.
This would use another field of size log\_2(ACID\_LENGTH) in order to specify how many of the most significant ACID bits are the mask.

Another improvement might be to add a bit that determines whether a capability is allowed to be used within CID spaces.
This would add fine-grained control over which capabilities can be shared within CID spaces and which are private to the owning compartment.

\subsection{Code and Data Compartments}

A common use case for compartments is to have one code base that operates on multiple data sets as found often in web browsers.
We envision this to be modelled by CID sealing and present our preliminary model for code and data compartments in the following paragraph:

The capabilities for different data sets are separated into compartments.
There will be one code compartment for each data, but each code capability maps to the same code.
This means that all code capabilities are identical in all fields, except for the CID field.
Depending on whether the different compartments trust each other, the compartments can be placed into the same CID space.

It is also possible to use conventional CHERI compartmentalisation where each data compartment has the same CID, but the data capabilities are non-overlapping sets between the compartments.
In this case, the supervisor code needs to be careful that two data capabilities from different compartments are never accessible to one capability at the same time.

\subsection{Revocation}

When employing more compartments than storable in the ACID space, software will need to virtualise CIDs.
This will lead to one or more ACIDs needing to be reused.
In order to maintain safety and security, every capability from the old compartment needs to be made inaccessible.
This prevents leaking capabilities from the old to the new compartment where both of them have the same ACID, but different SCIDs.

We have come up with a preliminary revocation mechanism, which we will sketch in the following paragraphs.
Please note that this mechanism likely incurs a substantial performance penalty.

When the supervisor code, e.g., the operating system, has run out of available ACIDs, it needs to revoke a ACID currently in use.
We currently propose to pick this ACID by random.
However, one could imagine keeping information in the OS that would enable different strategies, e.g., implementing a least recently used (LRU) policy.
The OS saves all capabilities of the old compartment in its own space as they are.
During this sweep, the OS will make all of these capabilities unusable, e.g., by marking it with an \textit{unusable} bit.
If code tries to legitimately use this capability, the OS will need to jump in and assign a new ACID to this SCID and update all of its pointers to make it usable again.
For example, a legitimate use case would be waking up the compartment after a longer phase of not invoking it.

\subsection{Performance Implications}

Using CID sealing can lead to performance improvements.
When changing compartments, the calling compartment no longer has to invalidate its own capabilities, but can rely on the sealing mechanism to prevent another compartment from using its capabilities.
This saves the calling compartment from using multiple instructions to zero out the register file (with CHERI, a compartment can also use cclear instructions, which can zero out a quarter of the register file on CHERI-RISC-V).
CID sealing does not help with confidentiality.
Sealed capabilities are still readable and therefore can leak secrets, e.g., keys.

One potential improvement is that short compartment calls with only a few instructions do not poison many registers.
After returning from a short-instruction callee compartment, many registers will still hold the original value of the calling compartment.
This can potentially be used to enhance performance even more because the calling compartment does not have to completely re-instantiate its register state.
