\errorcontextlines 10000
\usepackage{xparse}
\usepackage{xspace}
\usepackage{environ}
\usepackage{suffix}
\usepackage{xpatch}

%\renewcommand{\baselinestretch}{2} % double space for editors
\usepackage[headings]{fullpage}
\usepackage{bitset}
\usepackage{comment}
\usepackage{graphicx}
\usepackage{tabularx}
\usepackage{marginnote}
\usepackage{booktabs}
\usepackage{ifthen}
\usepackage{bytefield}
\usepackage{rotating}
\input{binhex}
\makeatletter\@ifclassloaded{standalone}{%
% The svgnames option conflicts with \documentclass[tikz]{standalone}
\usepackage{xcolor}
}{% else
\usepackage[svgnames]{xcolor}
}\makeatother % end of \@ifclassloaded{standalone}
\definecolor{lightgray}{gray}{0.8}
\usepackage{times}
\usepackage{algpseudocode}
\newcommand{\note}[2]{{\color{blue}[ Note: #1 - #2]}}
%%%%
%%%% For releases, uncomment to cause notes to disappear:
%%%%
\renewcommand{\note}[2]{\relax\ifhmode\unskip\fi}
\newcommand{\deprecated}[2]{{\color{grey}[ Note: #1 - #2]}}
\newcommand{\ajnote}[1]{\note{#1}{Alexandre J.}}
\newcommand{\arnote}[1]{\note{#1}{Alex R.}}
\newcommand{\bdnote}[1]{\note{#1}{Brooks D.}}
\newcommand{\dcnote}[1]{\note{#1}{David C.}}
\newcommand{\hmnote}[1]{\note{#1}{Hesham A.}}
\newcommand{\jhbnote}[1]{\note{#1}{John B.}}
\newcommand{\jrtcnote}[1]{\note{#1}{Jess C.}}
\newcommand{\jwnote}[1]{\note{#1}{Jon W.}}
\newcommand{\knnote}[1]{\note{#1}{Kyndylan N.}}
\newcommand{\mmnote}[1]{\note{#1}{Marno vdM}}
\newcommand{\mrnote}[1]{\note{#1}{Michael R.}}
\newcommand{\nwfnote}[1]{\note{#1}{nwf}}
\newcommand{\pdrnote}[1]{\note{#1}{Peter R.}}
\newcommand{\pgnnote}[1]{\note{#1}{Peter N.}}
\newcommand{\pmnote}[1]{\note{#1}{Prashanth M.}}
\newcommand{\psnote}[1]{\note{#1}{Peter S.}}
\newcommand{\rmnnote}[1]{\note{#1}{Robert N.}}
\newcommand{\rwnote}[1]{\note{#1}{Robert W.}}
\newcommand{\smnote}[1]{\note{#1}{Simon M.}}
\newcommand{\tmnote}[1]{\note{#1}{Theo M.}}

\usepackage{listings}
\usepackage{rotating}
\usepackage{setspace}
\usepackage{enumitem}
\usepackage{amsmath}
\usepackage{amssymb}
\usepackage{makecell}
\usepackage{hyphenat}
\usepackage{placeins}

\usepackage[utf8]{inputenc}
\usepackage[T1]{fontenc}

\usepackage{tikz}
 \usetikzlibrary{calc}
 \usetikzlibrary{decorations.pathreplacing}
 \usetikzlibrary{fit}
 \usetikzlibrary{matrix}
 \usetikzlibrary{positioning}
 \usetikzlibrary{shapes}
 \usetikzlibrary{patterns}

\newcommand*{\circnum}[2][gray!25]{%
  \protect\tikz[baseline={([yshift=-1.5pt]n.base)}]%
  \protect\node[fill=#1,shape=circle,inner sep=1pt,draw](n){\tiny #2};}
\newlist{inenum}{enumerate*}{1}
\setlist[inenum]{label={\circnum{\arabic*}}}

% Makes complex expansions slightly more readable
\let\ea\expandafter

\usepackage[scaled=0.82]{beramono}

\makeatletter
\newcommand{\makesailcmds@core}[2]{%
  \providecommand\saildoclabelled[2]{\phantomsection\label{#1}#2}
\providecommand\saildocval[2]{#1 #2}
\providecommand\saildocoutcome[2]{#1 #2}
\providecommand\saildocfcl[2]{#1 #2}
\providecommand\saildoctype[2]{#1 #2}
\providecommand\saildocfn[2]{#1 #2}
\providecommand\saildocoverload[2]{#1 #2}
\providecommand\saildocabbrev[1]{#1\@}
\providecommand\saildoclet[2]{#1 #2}
\providecommand\saildocregister[2]{#1 #2}

\newcommand{\sailRISCVvaledivInt}{\saildoclabelled{sailRISCVzedivzyint}{\saildocval{Euclidean division

}{\lstinputlisting[language=sail]{sail_latex_riscv/valzediv_int5aaf4d3d5a3d15a7aebaf90d3bfb6650.tex}}}}

\newcommand{\sailRISCVvalemodInt}{\saildoclabelled{sailRISCVzemodzyint}{\saildocval{}{\lstinputlisting[language=sail]{sail_latex_riscv/valzemod_int8e3d74b3b6a72e24e6bd03570d8e21ba.tex}}}}

\newcommand{\sailRISCVvalabsIntAtom}{\saildoclabelled{sailRISCVzabszyintzyatom}{\saildocval{}{\lstinputlisting[language=sail]{sail_latex_riscv/valzabs_int_atom414063313cc5ac5d9a742f9c8a111704.tex}}}}

\newcommand{\sailRISCVoverloadBabsInt}{\saildoclabelled{sailRISCVoverloadBzabszyint}{\saildocoverload{}{\lstinputlisting[language=sail]{sail_latex_riscv/overloadBzabs_intef5fbb521189282054dc80dc7173013d.tex}}}}

\newcommand{\sailRISCVtypeoption}{\saildoclabelled{sailRISCVtypezoption}{\saildoctype{}{\lstinputlisting[language=sail]{sail_latex_riscv/typezoptiona3271ef8b6a63c78e6db36dac0ee6547.tex}}}}

\newcommand{\sailRISCVvalisNone}{\saildoclabelled{sailRISCVziszynone}{\saildocval{}{\lstinputlisting[language=sail]{sail_latex_riscv/valzis_nonebebf4558161c4d567fb50f7df9e82374.tex}}}}

\newcommand{\sailRISCVfnisNone}{\saildoclabelled{sailRISCVfnziszynone}{\saildocfn{}{\lstinputlisting[language=sail]{sail_latex_riscv/fnzis_nonebebf4558161c4d567fb50f7df9e82374.tex}}}}

\newcommand{\sailRISCVvalisSome}{\saildoclabelled{sailRISCVziszysome}{\saildocval{}{\lstinputlisting[language=sail]{sail_latex_riscv/valzis_some1c925a3fbbb4ddc7f552b6fd691664ee.tex}}}}

\newcommand{\sailRISCVfnisSome}{\saildoclabelled{sailRISCVfnziszysome}{\saildocfn{}{\lstinputlisting[language=sail]{sail_latex_riscv/fnzis_some1c925a3fbbb4ddc7f552b6fd691664ee.tex}}}}

\newcommand{\sailRISCVvaleqUnit}{\saildoclabelled{sailRISCVzeqzyunit}{\saildocval{}{\lstinputlisting[language=sail]{sail_latex_riscv/valzeq_unit996f84433ac0995f4aadfca5b68cd358.tex}}}}

\newcommand{\sailRISCVvaleqBit}{\saildoclabelled{sailRISCVzeqzybit}{\saildocval{}{\lstinputlisting[language=sail]{sail_latex_riscv/valzeq_bit7182cc37406e2c0d4c1e739a98e248ea.tex}}}}

\newcommand{\sailRISCVfneqUnit}{\saildoclabelled{sailRISCVfnzeqzyunit}{\saildocfn{}{\lstinputlisting[language=sail]{sail_latex_riscv/fnzeq_unit996f84433ac0995f4aadfca5b68cd358.tex}}}}

\newcommand{\sailRISCVvalnotBool}{\saildoclabelled{sailRISCVznotzybool}{\saildocval{}{\lstinputlisting[language=sail]{sail_latex_riscv/valznot_boole1dd3e44bc87a2a10d8e257004c2d36a.tex}}}}

\newcommand{\sailRISCVvalandBool}{\saildoclabelled{sailRISCVzandzybool}{\saildocval{}{\lstinputlisting[language=sail]{sail_latex_riscv/valzand_boola4a2cf9ccaa44106300961b15ab20e79.tex}}}}

\newcommand{\sailRISCVvalandBoolNoFlow}{\saildoclabelled{sailRISCVzandzyboolzynozyflow}{\saildocval{}{\lstinputlisting[language=sail]{sail_latex_riscv/valzand_bool_no_flow5d5041fa8ff689136cdc03e3a11eda3a.tex}}}}

\newcommand{\sailRISCVvalorBool}{\saildoclabelled{sailRISCVzorzybool}{\saildocval{}{\lstinputlisting[language=sail]{sail_latex_riscv/valzor_bool5f07f9d72d4d1495c45a3531c787546a.tex}}}}

\newcommand{\sailRISCVvaleqInt}{\saildoclabelled{sailRISCVzeqzyint}{\saildocval{}{\lstinputlisting[language=sail]{sail_latex_riscv/valzeq_int364a98dbf8a9faa70e666cce41d8c1aa.tex}}}}

\newcommand{\sailRISCVvaleqBool}{\saildoclabelled{sailRISCVzeqzybool}{\saildocval{}{\lstinputlisting[language=sail]{sail_latex_riscv/valzeq_bool0e93587306381c3f984dc7cea6ae190d.tex}}}}

\newcommand{\sailRISCVvalneqInt}{\saildoclabelled{sailRISCVzneqzyint}{\saildocval{}{\lstinputlisting[language=sail]{sail_latex_riscv/valzneq_int4fd2be7a83f27bec736b67bdbab1d8c6.tex}}}}

\newcommand{\sailRISCVfnneqInt}{\saildoclabelled{sailRISCVfnzneqzyint}{\saildocfn{}{\lstinputlisting[language=sail]{sail_latex_riscv/fnzneq_int4fd2be7a83f27bec736b67bdbab1d8c6.tex}}}}

\newcommand{\sailRISCVvalneqBool}{\saildoclabelled{sailRISCVzneqzybool}{\saildocval{}{\lstinputlisting[language=sail]{sail_latex_riscv/valzneq_bool40d90a9f3b3bd9e0f1966f198535e779.tex}}}}

\newcommand{\sailRISCVfnneqBool}{\saildoclabelled{sailRISCVfnzneqzybool}{\saildocfn{}{\lstinputlisting[language=sail]{sail_latex_riscv/fnzneq_bool40d90a9f3b3bd9e0f1966f198535e779.tex}}}}

\newcommand{\sailRISCVvallteqInt}{\saildoclabelled{sailRISCVzlteqzyint}{\saildocval{}{\lstinputlisting[language=sail]{sail_latex_riscv/valzlteq_intc80d1082e443aa434e39355e493ece1e.tex}}}}

\newcommand{\sailRISCVvalgteqInt}{\saildoclabelled{sailRISCVzgteqzyint}{\saildocval{}{\lstinputlisting[language=sail]{sail_latex_riscv/valzgteq_inte32033a8d137f46d187455cff7dbe40e.tex}}}}

\newcommand{\sailRISCVvalltInt}{\saildoclabelled{sailRISCVzltzyint}{\saildocval{}{\lstinputlisting[language=sail]{sail_latex_riscv/valzlt_int996a8b8c361a31bed6b5509ca6686e1a.tex}}}}

\newcommand{\sailRISCVvalgtInt}{\saildoclabelled{sailRISCVzgtzyint}{\saildocval{}{\lstinputlisting[language=sail]{sail_latex_riscv/valzgt_intef94a8c66f39b1f715cb72941ed95921.tex}}}}

\newcommand{\sailRISCVoverloadCzEightoperatorzZerozJzJzNine}{\saildoclabelled{sailRISCVoverloadCzz8operatorz0zJzJz9}{\saildocoverload{}{\lstinputlisting[language=sail]{sail_latex_riscv/overloadCzz8operatorz0zjzjz9c650f45e06411dd4e97578ff2bad6338.tex}}}}

\newcommand{\sailRISCVoverloadDzEightoperatorzZerozOnezJzNine}{\saildoclabelled{sailRISCVoverloadDzz8operatorz0z1zJz9}{\saildocoverload{}{\lstinputlisting[language=sail]{sail_latex_riscv/overloadDzz8operatorz0z1zjz981ebe433e26f9e2dfa2a9d2c7f4fe1f4.tex}}}}

\newcommand{\sailRISCVoverloadEzEightoperatorzZerozUzNine}{\saildoclabelled{sailRISCVoverloadEzz8operatorz0zUz9}{\saildocoverload{}{\lstinputlisting[language=sail]{sail_latex_riscv/overloadEzz8operatorz0zuz99af95b281314726fa91893b57fc290dc.tex}}}}

\newcommand{\sailRISCVoverloadFzEightoperatorzZerozSixzNine}{\saildoclabelled{sailRISCVoverloadFzz8operatorz0z6z9}{\saildocoverload{}{\lstinputlisting[language=sail]{sail_latex_riscv/overloadFzz8operatorz0z6z9d3731bb9b1c9d765858778ad48ba6b3a.tex}}}}

\newcommand{\sailRISCVoverloadGzEightoperatorzZerozIzJzNine}{\saildoclabelled{sailRISCVoverloadGzz8operatorz0zIzJz9}{\saildocoverload{}{\lstinputlisting[language=sail]{sail_latex_riscv/overloadGzz8operatorz0zizjz95c366628fed7d8b7c251f1acd527ee3b.tex}}}}

\newcommand{\sailRISCVoverloadHzEightoperatorzZerozIzNine}{\saildoclabelled{sailRISCVoverloadHzz8operatorz0zIz9}{\saildocoverload{}{\lstinputlisting[language=sail]{sail_latex_riscv/overloadHzz8operatorz0ziz9714b8c400aed24ebd80eac39b4f9d751.tex}}}}

\newcommand{\sailRISCVoverloadIzEightoperatorzZerozKzJzNine}{\saildoclabelled{sailRISCVoverloadIzz8operatorz0zKzJz9}{\saildocoverload{}{\lstinputlisting[language=sail]{sail_latex_riscv/overloadIzz8operatorz0zkzjz94161e4bfad2d20e5d25bc774612b6588.tex}}}}

\newcommand{\sailRISCVoverloadJzEightoperatorzZerozKzNine}{\saildoclabelled{sailRISCVoverloadJzz8operatorz0zKz9}{\saildocoverload{}{\lstinputlisting[language=sail]{sail_latex_riscv/overloadJzz8operatorz0zkz93747e4d4a6f99eb3fca0b477d2437ed5.tex}}}}

\newcommand{\sailRISCVvalId}{\saildoclabelled{sailRISCVzzyzyid}{\saildocval{}{\lstinputlisting[language=sail]{sail_latex_riscv/valz__ided888b8991a27578d5dd72f84db80bce.tex}}}}

\newcommand{\sailRISCVfnId}{\saildoclabelled{sailRISCVfnzzyzyid}{\saildocfn{}{\lstinputlisting[language=sail]{sail_latex_riscv/fnz__ided888b8991a27578d5dd72f84db80bce.tex}}}}

\newcommand{\sailRISCVoverloadKSizze}{\saildoclabelled{sailRISCVoverloadKzzyzysizze}{\saildocoverload{}{\lstinputlisting[language=sail]{sail_latex_riscv/overloadKz__sizze5b2e36a5dbb42eaba80b4d164e45d3ae.tex}}}}

\newcommand{\sailRISCVvalDeref}{\saildoclabelled{sailRISCVzzyzyderef}{\saildocval{}{\lstinputlisting[language=sail]{sail_latex_riscv/valz__deref1dbc379e24bd1b182e1db755dea8c453.tex}}}}

\newcommand{\sailRISCVvaladdAtom}{\saildoclabelled{sailRISCVzaddzyatom}{\saildocval{}{\lstinputlisting[language=sail]{sail_latex_riscv/valzadd_atomd34efc9e611b6d3b6757e17f4932b12b.tex}}}}

\newcommand{\sailRISCVvaladdInt}{\saildoclabelled{sailRISCVzaddzyint}{\saildocval{}{\lstinputlisting[language=sail]{sail_latex_riscv/valzadd_intb17710be4fd02ace68d83b9dba907034.tex}}}}

\newcommand{\sailRISCVoverloadLzEightoperatorzZerozBzNine}{\saildoclabelled{sailRISCVoverloadLzz8operatorz0zBz9}{\saildocoverload{}{\lstinputlisting[language=sail]{sail_latex_riscv/overloadLzz8operatorz0zbz9a2d0168f574b152e5f31357e86602c16.tex}}}}

\newcommand{\sailRISCVvalsubAtom}{\saildoclabelled{sailRISCVzsubzyatom}{\saildocval{}{\lstinputlisting[language=sail]{sail_latex_riscv/valzsub_atom328a68dfbab1a07c42d4e7b98eac766f.tex}}}}

\newcommand{\sailRISCVvalsubInt}{\saildoclabelled{sailRISCVzsubzyint}{\saildocval{}{\lstinputlisting[language=sail]{sail_latex_riscv/valzsub_intf17f348f33594e77fdc3ef8b6a46b569.tex}}}}

\newcommand{\sailRISCVoverloadMzEightoperatorzZerozDzNine}{\saildoclabelled{sailRISCVoverloadMzz8operatorz0zDz9}{\saildocoverload{}{\lstinputlisting[language=sail]{sail_latex_riscv/overloadMzz8operatorz0zdz9aaaae29f381509679e21c2555127a5dd.tex}}}}

\newcommand{\sailRISCVvalsubNat}{\saildoclabelled{sailRISCVzsubzynat}{\saildocval{}{\lstinputlisting[language=sail]{sail_latex_riscv/valzsub_nat1e51a6ef44b288dd12f7f69af44dfd3e.tex}}}}

\newcommand{\sailRISCVvalnegateAtom}{\saildoclabelled{sailRISCVznegatezyatom}{\saildocval{}{\lstinputlisting[language=sail]{sail_latex_riscv/valznegate_atomfefdbde89b468d9df54837e864426d70.tex}}}}

\newcommand{\sailRISCVvalnegateInt}{\saildoclabelled{sailRISCVznegatezyint}{\saildocval{}{\lstinputlisting[language=sail]{sail_latex_riscv/valznegate_int42f776f84c124d77c3e367500082e43f.tex}}}}

\newcommand{\sailRISCVoverloadNnegate}{\saildoclabelled{sailRISCVoverloadNznegate}{\saildocoverload{}{\lstinputlisting[language=sail]{sail_latex_riscv/overloadNznegatef5714e2e9cd970a9cb8b7c6fdf3732b8.tex}}}}

\newcommand{\sailRISCVvalmultAtom}{\saildoclabelled{sailRISCVzmultzyatom}{\saildocval{}{\lstinputlisting[language=sail]{sail_latex_riscv/valzmult_atomdbad478b99777b7676dde1f5a7900711.tex}}}}

\newcommand{\sailRISCVvalmultInt}{\saildoclabelled{sailRISCVzmultzyint}{\saildocval{}{\lstinputlisting[language=sail]{sail_latex_riscv/valzmult_inte25d1b032a27b461f0eaf0c84be37a2b.tex}}}}

\newcommand{\sailRISCVoverloadOzEightoperatorzZerozAzNine}{\saildoclabelled{sailRISCVoverloadOzz8operatorz0zAz9}{\saildocoverload{}{\lstinputlisting[language=sail]{sail_latex_riscv/overloadOzz8operatorz0zaz94d99df7698c53c990108e8f028c06211.tex}}}}

\newcommand{\sailRISCVvalprintInt}{\saildoclabelled{sailRISCVzprintzyint}{\saildocval{}{\lstinputlisting[language=sail]{sail_latex_riscv/valzprint_intfb625bfb7a4021903513aeb4396bd878.tex}}}}

\newcommand{\sailRISCVvalprerrInt}{\saildoclabelled{sailRISCVzprerrzyint}{\saildocval{}{\lstinputlisting[language=sail]{sail_latex_riscv/valzprerr_int00b48f715fbb32df5901801dff63b643.tex}}}}

\newcommand{\sailRISCVvalShlEight}{\saildoclabelled{sailRISCVzzyshl8}{\saildocval{A common idiom in asl is to take two bits of an opcode and convert in into a variable like

\lstinputlisting[language=sail]{sail_latex_riscv/blocka2cd1c63e1ba9c2d625830f7e4de8f31.sail}\lstinline{_shl8} ensures that in this case the typechecker knows that the end result will be a value in the set \lstinline`{8, 16, 32, 64}`

Similarly, we define shifts of 32 and 1 (i.e., powers of two).

The most general shift operations also allow negative shifts which go in the opposite direction, for compatibility with ASL.

}{\lstinputlisting[language=sail]{sail_latex_riscv/valz_shl8e01c74b934d4c323501a597baa8e6f73.tex}}}}

\newcommand{\sailRISCVvalShlThreeTwo}{\saildoclabelled{sailRISCVzzyshl32}{\saildocval{}{\lstinputlisting[language=sail]{sail_latex_riscv/valz_shl32469ae968a52f81e1a28aeacf7e2d496b.tex}}}}

\newcommand{\sailRISCVvalShlOne}{\saildoclabelled{sailRISCVzzyshl1}{\saildocval{}{\lstinputlisting[language=sail]{sail_latex_riscv/valz_shl1b261f5995acb90d475c10ee0cdbc12ce.tex}}}}

\newcommand{\sailRISCVvalShlInt}{\saildoclabelled{sailRISCVzzyshlzyint}{\saildocval{}{\lstinputlisting[language=sail]{sail_latex_riscv/valz_shl_int86f4e1bc3609625860bc16734d7f2614.tex}}}}

\newcommand{\sailRISCVvalShrThreeTwo}{\saildoclabelled{sailRISCVzzyshr32}{\saildocval{}{\lstinputlisting[language=sail]{sail_latex_riscv/valz_shr328ec48e4bcaebfdbf5c374b77ca7b535b.tex}}}}

\newcommand{\sailRISCVvalShrInt}{\saildoclabelled{sailRISCVzzyshrzyint}{\saildocval{}{\lstinputlisting[language=sail]{sail_latex_riscv/valz_shr_int34025c843d841a08930cb64bf99a1693.tex}}}}

\newcommand{\sailRISCVvalShlIntGeneral}{\saildoclabelled{sailRISCVzzyshlzyintzygeneral}{\saildocval{}{\lstinputlisting[language=sail]{sail_latex_riscv/valz_shl_int_generalab86cb298d6e60b48e78627d598f6165.tex}}}}

\newcommand{\sailRISCVfnShlIntGeneral}{\saildoclabelled{sailRISCVfnzzyshlzyintzygeneral}{\saildocfn{}{\lstinputlisting[language=sail]{sail_latex_riscv/fnz_shl_int_generalab86cb298d6e60b48e78627d598f6165.tex}}}}

\newcommand{\sailRISCVvalShrIntGeneral}{\saildoclabelled{sailRISCVzzyshrzyintzygeneral}{\saildocval{}{\lstinputlisting[language=sail]{sail_latex_riscv/valz_shr_int_generalf06a573ed81aec273a6397188519fd34.tex}}}}

\newcommand{\sailRISCVfnShrIntGeneral}{\saildoclabelled{sailRISCVfnzzyshrzyintzygeneral}{\saildocfn{}{\lstinputlisting[language=sail]{sail_latex_riscv/fnz_shr_int_generalf06a573ed81aec273a6397188519fd34.tex}}}}

\newcommand{\sailRISCVoverloadPshlInt}{\saildoclabelled{sailRISCVoverloadPzshlzyint}{\saildocoverload{}{\lstinputlisting[language=sail]{sail_latex_riscv/overloadPzshl_int4772030e3fc0913189e795ec25e86dc5.tex}}}}

\newcommand{\sailRISCVoverloadQshrInt}{\saildoclabelled{sailRISCVoverloadQzshrzyint}{\saildocoverload{}{\lstinputlisting[language=sail]{sail_latex_riscv/overloadQzshr_int5f4032eb21b9c850a9e2a8de5872a2a2.tex}}}}

\newcommand{\sailRISCVvaltdivInt}{\saildoclabelled{sailRISCVztdivzyint}{\saildocval{Truncating division (rounds towards zero)

}{\lstinputlisting[language=sail]{sail_latex_riscv/valztdiv_int5e119ac7ab9ff04c8877846f345d1159.tex}}}}

\newcommand{\sailRISCVvalTmodInt}{\saildoclabelled{sailRISCVzzytmodzyint}{\saildocval{Remainder for truncating division (has sign of dividend)

}{\lstinputlisting[language=sail]{sail_latex_riscv/valz_tmod_inta2984ba6dbfa10758476d9b3b7f62560.tex}}}}

\newcommand{\sailRISCVvalTmodIntPositive}{\saildoclabelled{sailRISCVzzytmodzyintzypositive}{\saildocval{If we know the second argument is positive, we know the result is positive

}{\lstinputlisting[language=sail]{sail_latex_riscv/valz_tmod_int_positive6f0621d972182279e90a43c082e50c10.tex}}}}

\newcommand{\sailRISCVoverloadRtmodInt}{\saildoclabelled{sailRISCVoverloadRztmodzyint}{\saildocoverload{}{\lstinputlisting[language=sail]{sail_latex_riscv/overloadRztmod_int76b131b53b88df8b201279295eacebbe.tex}}}}

\newcommand{\sailRISCVvalfdivInt}{\saildoclabelled{sailRISCVzfdivzyint}{\saildocval{}{\lstinputlisting[language=sail]{sail_latex_riscv/valzfdiv_intd3535e930b3252acc5f18a9e4b34e63a.tex}}}}

\newcommand{\sailRISCVfnfdivInt}{\saildoclabelled{sailRISCVfnzfdivzyint}{\saildocfn{}{\lstinputlisting[language=sail]{sail_latex_riscv/fnzfdiv_intd3535e930b3252acc5f18a9e4b34e63a.tex}}}}

\newcommand{\sailRISCVvalfmodInt}{\saildoclabelled{sailRISCVzfmodzyint}{\saildocval{}{\lstinputlisting[language=sail]{sail_latex_riscv/valzfmod_int7e215ca2b888f4e92201959fd40958a5.tex}}}}

\newcommand{\sailRISCVfnfmodInt}{\saildoclabelled{sailRISCVfnzfmodzyint}{\saildocfn{}{\lstinputlisting[language=sail]{sail_latex_riscv/fnzfmod_int7e215ca2b888f4e92201959fd40958a5.tex}}}}

\newcommand{\sailRISCVvalabsIntPlain}{\saildoclabelled{sailRISCVzabszyintzyplain}{\saildocval{}{\lstinputlisting[language=sail]{sail_latex_riscv/valzabs_int_plainb54aa4afeed2c86b519a464eb2e4c77c.tex}}}}

\newcommand{\sailRISCVoverloadSabsInt}{\saildoclabelled{sailRISCVoverloadSzabszyint}{\saildocoverload{}{\lstinputlisting[language=sail]{sail_latex_riscv/overloadSzabs_intef5fbb521189282054dc80dc7173013d.tex}}}}

\newcommand{\sailRISCVvaleqString}{\saildoclabelled{sailRISCVzeqzystring}{\saildocval{}{\lstinputlisting[language=sail]{sail_latex_riscv/valzeq_string75dfa57c0476ae3f43f8e55ffe51a116.tex}}}}

\newcommand{\sailRISCVoverloadTzEightoperatorzZerozJzJzNine}{\saildoclabelled{sailRISCVoverloadTzz8operatorz0zJzJz9}{\saildocoverload{}{\lstinputlisting[language=sail]{sail_latex_riscv/overloadTzz8operatorz0zjzjz9c650f45e06411dd4e97578ff2bad6338.tex}}}}

\newcommand{\sailRISCVvalconcatStr}{\saildoclabelled{sailRISCVzconcatzystr}{\saildocval{}{\lstinputlisting[language=sail]{sail_latex_riscv/valzconcat_str366019c233188ef65ab3d1f977f04112.tex}}}}

\newcommand{\sailRISCVvaldecStr}{\saildoclabelled{sailRISCVzdeczystr}{\saildocval{}{\lstinputlisting[language=sail]{sail_latex_riscv/valzdec_str7582ccea1482759c248b1f1ac9f6ae63.tex}}}}

\newcommand{\sailRISCVvalhexStr}{\saildoclabelled{sailRISCVzhexzystr}{\saildocval{}{\lstinputlisting[language=sail]{sail_latex_riscv/valzhex_str47c735e2941ef5c87d4f7502a5e92a2a.tex}}}}

\newcommand{\sailRISCVvalbitsStr}{\saildoclabelled{sailRISCVzbitszystr}{\saildocval{}{\lstinputlisting[language=sail]{sail_latex_riscv/valzbits_strae053d842c21f0867dea1e830d1773cc.tex}}}}

\newcommand{\sailRISCVvalconcatStrBits}{\saildoclabelled{sailRISCVzconcatzystrzybits}{\saildocval{}{\lstinputlisting[language=sail]{sail_latex_riscv/valzconcat_str_bitsd8fc2224310ed49d394cba090cf60741.tex}}}}

\newcommand{\sailRISCVfnconcatStrBits}{\saildoclabelled{sailRISCVfnzconcatzystrzybits}{\saildocfn{}{\lstinputlisting[language=sail]{sail_latex_riscv/fnzconcat_str_bitsd8fc2224310ed49d394cba090cf60741.tex}}}}

\newcommand{\sailRISCVvalconcatStrDec}{\saildoclabelled{sailRISCVzconcatzystrzydec}{\saildocval{}{\lstinputlisting[language=sail]{sail_latex_riscv/valzconcat_str_dec4a6431591803433e2668ed9b4afaadd0.tex}}}}

\newcommand{\sailRISCVfnconcatStrDec}{\saildoclabelled{sailRISCVfnzconcatzystrzydec}{\saildocfn{}{\lstinputlisting[language=sail]{sail_latex_riscv/fnzconcat_str_dec4a6431591803433e2668ed9b4afaadd0.tex}}}}

\newcommand{\sailRISCVvalprintEndline}{\saildoclabelled{sailRISCVzprintzyendline}{\saildocval{}{\lstinputlisting[language=sail]{sail_latex_riscv/valzprint_endline03a43e2779561cb054d0761733c27e9b.tex}}}}

\newcommand{\sailRISCVvalprerrEndline}{\saildoclabelled{sailRISCVzprerrzyendline}{\saildocval{}{\lstinputlisting[language=sail]{sail_latex_riscv/valzprerr_endline73ce57fcf6e847727670556577cb2de0.tex}}}}

\newcommand{\sailRISCVtypebits}{\saildoclabelled{sailRISCVtypezbits}{\saildoctype{}{\lstinputlisting[language=sail]{sail_latex_riscv/typezbitsa4b31f9b3dc11c921007b665e0d0fce6.tex}}}}

\newcommand{\sailRISCVvaleqBits}{\saildoclabelled{sailRISCVzeqzybits}{\saildocval{}{\lstinputlisting[language=sail]{sail_latex_riscv/valzeq_bits886ce7cf3ec93a28308e8d4e9d63f4be.tex}}}}

\newcommand{\sailRISCVoverloadUzEightoperatorzZerozJzJzNine}{\saildoclabelled{sailRISCVoverloadUzz8operatorz0zJzJz9}{\saildocoverload{}{\lstinputlisting[language=sail]{sail_latex_riscv/overloadUzz8operatorz0zjzjz9c650f45e06411dd4e97578ff2bad6338.tex}}}}

\newcommand{\sailRISCVvalneqBits}{\saildoclabelled{sailRISCVzneqzybits}{\saildocval{}{\lstinputlisting[language=sail]{sail_latex_riscv/valzneq_bits167748c906c068e62596c88540a84f42.tex}}}}

\newcommand{\sailRISCVfnneqBits}{\saildoclabelled{sailRISCVfnzneqzybits}{\saildocfn{}{\lstinputlisting[language=sail]{sail_latex_riscv/fnzneq_bits167748c906c068e62596c88540a84f42.tex}}}}

\newcommand{\sailRISCVoverloadVzEightoperatorzZerozOnezJzNine}{\saildoclabelled{sailRISCVoverloadVzz8operatorz0z1zJz9}{\saildocoverload{}{\lstinputlisting[language=sail]{sail_latex_riscv/overloadVzz8operatorz0z1zjz981ebe433e26f9e2dfa2a9d2c7f4fe1f4.tex}}}}

\newcommand{\sailRISCVvalbitvectorLength}{\saildoclabelled{sailRISCVzbitvectorzylength}{\saildocval{}{\lstinputlisting[language=sail]{sail_latex_riscv/valzbitvector_lengthcd74a5cced7567d19500671e4b6e1031.tex}}}}

\newcommand{\sailRISCVvalvectorLength}{\saildoclabelled{sailRISCVzvectorzylength}{\saildocval{}{\lstinputlisting[language=sail]{sail_latex_riscv/valzvector_length9ee541b308cdfd9738d44bfb3dff4b46.tex}}}}

\newcommand{\sailRISCVoverloadWlength}{\saildoclabelled{sailRISCVoverloadWzlength}{\saildocoverload{}{\lstinputlisting[language=sail]{sail_latex_riscv/overloadWzlength469e3f917f7b24f4691faf3caf842eba.tex}}}}

\newcommand{\sailRISCVvalcountLeadingZeros}{\saildoclabelled{sailRISCVzcountzyleadingzyzzeros}{\saildocval{}{\lstinputlisting[language=sail]{sail_latex_riscv/valzcount_leading_zzeros315ae28f559df1d42a7d2ca4cfff2905.tex}}}}

\newcommand{\sailRISCVvalprintBits}{\saildoclabelled{sailRISCVzprintzybits}{\saildocval{}{\lstinputlisting[language=sail]{sail_latex_riscv/valzprint_bits30cf225474fbf3e575d7aa83aa309559.tex}}}}

\newcommand{\sailRISCVvalprerrBits}{\saildoclabelled{sailRISCVzprerrzybits}{\saildocval{}{\lstinputlisting[language=sail]{sail_latex_riscv/valzprerr_bits932899725108ebe483d3226f250f2b92.tex}}}}

\newcommand{\sailRISCVvalsailSignExtend}{\saildoclabelled{sailRISCVzsailzysignzyextend}{\saildocval{}{\lstinputlisting[language=sail]{sail_latex_riscv/valzsail_sign_extendb66ac7c1aaedb0cb21bdf07e4518af5e.tex}}}}

\newcommand{\sailRISCVvalsailZeroExtend}{\saildoclabelled{sailRISCVzsailzyzzerozyextend}{\saildocval{}{\lstinputlisting[language=sail]{sail_latex_riscv/valzsail_zzero_extend411875c18d3b332113845d17151890c2.tex}}}}

\newcommand{\sailRISCVvaltruncate}{\saildoclabelled{sailRISCVztruncate}{\saildocval{\lstinline{truncate}\lstinline`(v, n)` truncates \lstinline`v`, keeping only the \emph{least} significant \lstinline`n` bits.

}{\lstinputlisting[language=sail]{sail_latex_riscv/valztruncatea666e28ae0c8ca7327a2b3fcd1ed4ec7.tex}}}}

\newcommand{\sailRISCVvaltruncateLSB}{\saildoclabelled{sailRISCVztruncateLSB}{\saildocval{\lstinline{truncateLSB}\lstinline`(v, n)` truncates \lstinline`v`, keeping only the \emph{most} significant \lstinline`n` bits.

}{\lstinputlisting[language=sail]{sail_latex_riscv/valztruncatelsb4d124c6ec672453343dc0b20d295e82d.tex}}}}

\newcommand{\sailRISCVvalsailMask}{\saildoclabelled{sailRISCVzsailzymask}{\saildocval{}{\lstinputlisting[language=sail]{sail_latex_riscv/valzsail_maske146b73afc824e90813dd8234bfa3053.tex}}}}

\newcommand{\sailRISCVfnsailMask}{\saildoclabelled{sailRISCVfnzsailzymask}{\saildocfn{}{\lstinputlisting[language=sail]{sail_latex_riscv/fnzsail_maske146b73afc824e90813dd8234bfa3053.tex}}}}

\newcommand{\sailRISCVoverloadXzEightoperatorzZerozQzNine}{\saildoclabelled{sailRISCVoverloadXzz8operatorz0zQz9}{\saildocoverload{}{\lstinputlisting[language=sail]{sail_latex_riscv/overloadXzz8operatorz0zqz9ccbd65071d8f0fbb9677c7f6e86d3527.tex}}}}

\newcommand{\sailRISCVvalbitvectorConcat}{\saildoclabelled{sailRISCVzbitvectorzyconcat}{\saildocval{}{\lstinputlisting[language=sail]{sail_latex_riscv/valzbitvector_concat6176f8be1468d8779ee8370fd3b4a6e0.tex}}}}

\newcommand{\sailRISCVoverloadYappend}{\saildoclabelled{sailRISCVoverloadYzappend}{\saildocoverload{}{\lstinputlisting[language=sail]{sail_latex_riscv/overloadYzappend88575169e0ec1639b6ae3851df999710.tex}}}}

\newcommand{\sailRISCVvalappendSixFour}{\saildoclabelled{sailRISCVzappendzy64}{\saildocval{}{\lstinputlisting[language=sail]{sail_latex_riscv/valzappend_6433ef192058d4bf5f092d6f8b6d97f4c4.tex}}}}

\newcommand{\sailRISCVvalbitvectorAccess}{\saildoclabelled{sailRISCVzbitvectorzyaccess}{\saildocval{}{\lstinputlisting[language=sail]{sail_latex_riscv/valzbitvector_access8b584ca86770abb6b0da5ef059a02ed9.tex}}}}

\newcommand{\sailRISCVvalplainVectorAccess}{\saildoclabelled{sailRISCVzplainzyvectorzyaccess}{\saildocval{}{\lstinputlisting[language=sail]{sail_latex_riscv/valzplain_vector_access792547dd734d4ff2e6078cbb88967469.tex}}}}

\newcommand{\sailRISCVoverloadZvectorAccess}{\saildoclabelled{sailRISCVoverloadZzvectorzyaccess}{\saildocoverload{}{\lstinputlisting[language=sail]{sail_latex_riscv/overloadZzvector_accessbe81ec250d2df2ebadde393ea37a85a4.tex}}}}

\newcommand{\sailRISCVvalbitvectorUpdate}{\saildoclabelled{sailRISCVzbitvectorzyupdate}{\saildocval{}{\lstinputlisting[language=sail]{sail_latex_riscv/valzbitvector_update20826799a1ff3ff40895206db0df14bb.tex}}}}

\newcommand{\sailRISCVvalplainVectorUpdate}{\saildoclabelled{sailRISCVzplainzyvectorzyupdate}{\saildocval{}{\lstinputlisting[language=sail]{sail_latex_riscv/valzplain_vector_updateb31d67bfe51b1a6f79983347dfc57da0.tex}}}}

\newcommand{\sailRISCVoverloadAAvectorUpdate}{\saildoclabelled{sailRISCVoverloadAAzvectorzyupdate}{\saildocoverload{}{\lstinputlisting[language=sail]{sail_latex_riscv/overloadAAzvector_updateb14d5207ae01ed7fc9d9882c9cc3ebef.tex}}}}

\newcommand{\sailRISCVvaladdBits}{\saildoclabelled{sailRISCVzaddzybits}{\saildocval{}{\lstinputlisting[language=sail]{sail_latex_riscv/valzadd_bits24373ffc11f289d5bb648df2f4f41b25.tex}}}}

\newcommand{\sailRISCVvaladdBitsInt}{\saildoclabelled{sailRISCVzaddzybitszyint}{\saildocval{}{\lstinputlisting[language=sail]{sail_latex_riscv/valzadd_bits_inta5424052402522ff4653275c899f7543.tex}}}}

\newcommand{\sailRISCVoverloadBBzEightoperatorzZerozBzNine}{\saildoclabelled{sailRISCVoverloadBBzz8operatorz0zBz9}{\saildocoverload{}{\lstinputlisting[language=sail]{sail_latex_riscv/overloadBBzz8operatorz0zbz9a2d0168f574b152e5f31357e86602c16.tex}}}}

\newcommand{\sailRISCVvalsubBits}{\saildoclabelled{sailRISCVzsubzybits}{\saildocval{}{\lstinputlisting[language=sail]{sail_latex_riscv/valzsub_bitsf0dc4fc3429d45517c523db21af72127.tex}}}}

\newcommand{\sailRISCVvalnotVec}{\saildoclabelled{sailRISCVznotzyvec}{\saildocval{}{\lstinputlisting[language=sail]{sail_latex_riscv/valznot_vecfb45897f737be88160f5363827ef4a4b.tex}}}}

\newcommand{\sailRISCVvalandVec}{\saildoclabelled{sailRISCVzandzyvec}{\saildocval{}{\lstinputlisting[language=sail]{sail_latex_riscv/valzand_vec99be3fe45d23194b597520c9e407ad35.tex}}}}

\newcommand{\sailRISCVoverloadCCzEightoperatorzZerozSixzNine}{\saildoclabelled{sailRISCVoverloadCCzz8operatorz0z6z9}{\saildocoverload{}{\lstinputlisting[language=sail]{sail_latex_riscv/overloadCCzz8operatorz0z6z9d3731bb9b1c9d765858778ad48ba6b3a.tex}}}}

\newcommand{\sailRISCVvalorVec}{\saildoclabelled{sailRISCVzorzyvec}{\saildocval{}{\lstinputlisting[language=sail]{sail_latex_riscv/valzor_vec467c7a3f74be27085fe1b2aa3651ffe7.tex}}}}

\newcommand{\sailRISCVoverloadDDzEightoperatorzZerozUzNine}{\saildoclabelled{sailRISCVoverloadDDzz8operatorz0zUz9}{\saildocoverload{}{\lstinputlisting[language=sail]{sail_latex_riscv/overloadDDzz8operatorz0zuz99af95b281314726fa91893b57fc290dc.tex}}}}

\newcommand{\sailRISCVvalxorVec}{\saildoclabelled{sailRISCVzxorzyvec}{\saildocval{}{\lstinputlisting[language=sail]{sail_latex_riscv/valzxor_vecdacd54acc32f073fb01d1c188177bc8c.tex}}}}

\newcommand{\sailRISCVvalsubrangeBits}{\saildoclabelled{sailRISCVzsubrangezybits}{\saildocval{}{\lstinputlisting[language=sail]{sail_latex_riscv/valzsubrange_bits6c497c14df4f4754bd345a6c56ca2aad.tex}}}}

\newcommand{\sailRISCVoverloadEEvectorSubrange}{\saildoclabelled{sailRISCVoverloadEEzvectorzysubrange}{\saildocoverload{}{\lstinputlisting[language=sail]{sail_latex_riscv/overloadEEzvector_subrange270c799ffa6c20b5244f22c64fba0367.tex}}}}

\newcommand{\sailRISCVvalupdateSubrangeBits}{\saildoclabelled{sailRISCVzupdatezysubrangezybits}{\saildocval{}{\lstinputlisting[language=sail]{sail_latex_riscv/valzupdate_subrange_bitsb5ffe862b26310b45a779cd45bbf041e.tex}}}}

\newcommand{\sailRISCVoverloadFFvectorUpdateSubrange}{\saildoclabelled{sailRISCVoverloadFFzvectorzyupdatezysubrange}{\saildocoverload{}{\lstinputlisting[language=sail]{sail_latex_riscv/overloadFFzvector_update_subrangeb77be803268d55f5f112399f9d0dfbc2.tex}}}}

\newcommand{\sailRISCVvalsailShiftleft}{\saildoclabelled{sailRISCVzsailzyshiftleft}{\saildocval{}{\lstinputlisting[language=sail]{sail_latex_riscv/valzsail_shiftlefta7bc10407d10355c4e981688c9926084.tex}}}}

\newcommand{\sailRISCVvalsailShiftright}{\saildoclabelled{sailRISCVzsailzyshiftright}{\saildocval{}{\lstinputlisting[language=sail]{sail_latex_riscv/valzsail_shiftrighte403ac5c2740b7767c2bdfe689082562.tex}}}}

\newcommand{\sailRISCVvalsailArithShiftright}{\saildoclabelled{sailRISCVzsailzyarithzyshiftright}{\saildocval{}{\lstinputlisting[language=sail]{sail_latex_riscv/valzsail_arith_shiftrighta24f06e92ffcd84e26ed61085c833371.tex}}}}

\newcommand{\sailRISCVvalsailZeros}{\saildoclabelled{sailRISCVzsailzyzzeros}{\saildocval{}{\lstinputlisting[language=sail]{sail_latex_riscv/valzsail_zzeros174d4d4928427d9df9fa9749f1df5f96.tex}}}}

\newcommand{\sailRISCVvalsailOnes}{\saildoclabelled{sailRISCVzsailzyones}{\saildocval{}{\lstinputlisting[language=sail]{sail_latex_riscv/valzsail_ones0510f34656bd3d7b905b0ff315bf81d7.tex}}}}

\newcommand{\sailRISCVfnsailOnes}{\saildoclabelled{sailRISCVfnzsailzyones}{\saildocfn{}{\lstinputlisting[language=sail]{sail_latex_riscv/fnzsail_ones0510f34656bd3d7b905b0ff315bf81d7.tex}}}}

\newcommand{\sailRISCVvalslice}{\saildoclabelled{sailRISCVzslice}{\saildocval{}{\lstinputlisting[language=sail]{sail_latex_riscv/valzslice9979e992fd48f77a2c3fef7fbcce068e.tex}}}}

\newcommand{\sailRISCVvalreplicateBits}{\saildoclabelled{sailRISCVzreplicatezybits}{\saildocval{}{\lstinputlisting[language=sail]{sail_latex_riscv/valzreplicate_bitsb29bdab6bb9437712accf2dc81ea3d3e.tex}}}}

\newcommand{\sailRISCVvalsliceMask}{\saildoclabelled{sailRISCVzslicezymask}{\saildocval{}{\lstinputlisting[language=sail]{sail_latex_riscv/valzslice_maske01cafc7448fbf1583dc5dd96b06c854.tex}}}}

\newcommand{\sailRISCVfnsliceMask}{\saildoclabelled{sailRISCVfnzslicezymask}{\saildocfn{}{\lstinputlisting[language=sail]{sail_latex_riscv/fnzslice_maske01cafc7448fbf1583dc5dd96b06c854.tex}}}}

\newcommand{\sailRISCVvalgetSliceInt}{\saildoclabelled{sailRISCVzgetzyslicezyint}{\saildocval{}{\lstinputlisting[language=sail]{sail_latex_riscv/valzget_slice_int3c313e973dc436aff309f66096377164.tex}}}}

\newcommand{\sailRISCVvalsetSliceInt}{\saildoclabelled{sailRISCVzsetzyslicezyint}{\saildocval{}{\lstinputlisting[language=sail]{sail_latex_riscv/valzset_slice_intf4b6b0ed3d8b3bb2f2e0d7a492959629.tex}}}}

\newcommand{\sailRISCVvalsetSliceBits}{\saildoclabelled{sailRISCVzsetzyslicezybits}{\saildocval{}{\lstinputlisting[language=sail]{sail_latex_riscv/valzset_slice_bits5956200094c551f35973411fcc90a521.tex}}}}

\newcommand{\sailRISCVvalunsigned}{\saildoclabelled{sailRISCVzunsigned}{\saildocval{converts a bit vector of length $n$ to an integer in the range $0$ to $2^n - 1$.

}{\lstinputlisting[language=sail]{sail_latex_riscv/valzunsigned1010eda2cdd2666cd8fd0ddf82ac526f.tex}}}}

\newcommand{\sailRISCVvalsigned}{\saildoclabelled{sailRISCVzsigned}{\saildocval{converts a bit vector of length $n$ to an integer in the range $-2^{n-1}$ to $2^{n-1} - 1$ using twos-complement.

}{\lstinputlisting[language=sail]{sail_latex_riscv/valzsigned36d2317f34f1dacb4e465e6e56b185e6.tex}}}}

\newcommand{\sailRISCVoverloadGGSizze}{\saildoclabelled{sailRISCVoverloadGGzzyzysizze}{\saildocoverload{}{\lstinputlisting[language=sail]{sail_latex_riscv/overloadGGz__sizze5b2e36a5dbb42eaba80b4d164e45d3ae.tex}}}}

\newcommand{\sailRISCVtyperegfp}{\saildoclabelled{sailRISCVtypezregfp}{\saildoctype{}{\lstinputlisting[language=sail]{sail_latex_riscv/typezregfpedcf3a6440b11288a4e07504f1ebdfae.tex}}}}

\newcommand{\sailRISCVtyperegfps}{\saildoclabelled{sailRISCVtypezregfps}{\saildoctype{}{\lstinputlisting[language=sail]{sail_latex_riscv/typezregfps6fc0ab735834848cecec1fbd72e56328.tex}}}}

\newcommand{\sailRISCVtypeniafp}{\saildoclabelled{sailRISCVtypezniafp}{\saildoctype{}{\lstinputlisting[language=sail]{sail_latex_riscv/typezniafpdcbcae7343979e4fb47e41a0909b121f.tex}}}}

\newcommand{\sailRISCVtypeniafps}{\saildoclabelled{sailRISCVtypezniafps}{\saildoctype{}{\lstinputlisting[language=sail]{sail_latex_riscv/typezniafps1c85f5c2a0d9da30d236aad9e6b48b40.tex}}}}

\newcommand{\sailRISCVtypediafp}{\saildoclabelled{sailRISCVtypezdiafp}{\saildoctype{}{\lstinputlisting[language=sail]{sail_latex_riscv/typezdiafp900a9a3c892b92c007276686dcd307f6.tex}}}}

\newcommand{\sailRISCVtypereadKind}{\saildoclabelled{sailRISCVtypezreadzykind}{\saildoctype{}{\lstinputlisting[language=sail]{sail_latex_riscv/typezread_kindc722f7d2aff68c2bd16feb054ed367f8.tex}}}}

\newcommand{\sailRISCVvalreadKindOfNum}{\saildoclabelled{sailRISCVzreadzykindzyofzynum}{\saildocval{}{\lstinputlisting[language=sail]{sail_latex_riscv/valzread_kind_of_numd8fea9b1331732e205bdd70279e0ba47.tex}}}}

\newcommand{\sailRISCVfnreadKindOfNum}{\saildoclabelled{sailRISCVfnzreadzykindzyofzynum}{\saildocfn{}{\lstinputlisting[language=sail]{sail_latex_riscv/fnzread_kind_of_numd8fea9b1331732e205bdd70279e0ba47.tex}}}}

\newcommand{\sailRISCVvalnumOfReadKind}{\saildoclabelled{sailRISCVznumzyofzyreadzykind}{\saildocval{}{\lstinputlisting[language=sail]{sail_latex_riscv/valznum_of_read_kind9f1d12d5627d7618c1e31c888906fc68.tex}}}}

\newcommand{\sailRISCVfnnumOfReadKind}{\saildoclabelled{sailRISCVfnznumzyofzyreadzykind}{\saildocfn{}{\lstinputlisting[language=sail]{sail_latex_riscv/fnznum_of_read_kind9f1d12d5627d7618c1e31c888906fc68.tex}}}}

\newcommand{\sailRISCVtypewriteKind}{\saildoclabelled{sailRISCVtypezwritezykind}{\saildoctype{}{\lstinputlisting[language=sail]{sail_latex_riscv/typezwrite_kindd407cee84c148660ae0b73dee4f0ddc7.tex}}}}

\newcommand{\sailRISCVvalwriteKindOfNum}{\saildoclabelled{sailRISCVzwritezykindzyofzynum}{\saildocval{}{\lstinputlisting[language=sail]{sail_latex_riscv/valzwrite_kind_of_num3c6c37285ad605eea3332f170d5b12d9.tex}}}}

\newcommand{\sailRISCVfnwriteKindOfNum}{\saildoclabelled{sailRISCVfnzwritezykindzyofzynum}{\saildocfn{}{\lstinputlisting[language=sail]{sail_latex_riscv/fnzwrite_kind_of_num3c6c37285ad605eea3332f170d5b12d9.tex}}}}

\newcommand{\sailRISCVvalnumOfWriteKind}{\saildoclabelled{sailRISCVznumzyofzywritezykind}{\saildocval{}{\lstinputlisting[language=sail]{sail_latex_riscv/valznum_of_write_kind056951dbaa3b47d3c25ba586d7093c91.tex}}}}

\newcommand{\sailRISCVfnnumOfWriteKind}{\saildoclabelled{sailRISCVfnznumzyofzywritezykind}{\saildocfn{}{\lstinputlisting[language=sail]{sail_latex_riscv/fnznum_of_write_kind056951dbaa3b47d3c25ba586d7093c91.tex}}}}

\newcommand{\sailRISCVtypeaSixFourBarrierDomain}{\saildoclabelled{sailRISCVtypeza64zybarrierzydomain}{\saildoctype{}{\lstinputlisting[language=sail]{sail_latex_riscv/typeza64_barrier_domaind28b62adeb08a7dce5456d5fc74d0c80.tex}}}}

\newcommand{\sailRISCVvalaSixFourBarrierDomainOfNum}{\saildoclabelled{sailRISCVza64zybarrierzydomainzyofzynum}{\saildocval{}{\lstinputlisting[language=sail]{sail_latex_riscv/valza64_barrier_domain_of_num6e122924ff562010f42f288ecc2cdbe3.tex}}}}

\newcommand{\sailRISCVfnaSixFourBarrierDomainOfNum}{\saildoclabelled{sailRISCVfnza64zybarrierzydomainzyofzynum}{\saildocfn{}{\lstinputlisting[language=sail]{sail_latex_riscv/fnza64_barrier_domain_of_num6e122924ff562010f42f288ecc2cdbe3.tex}}}}

\newcommand{\sailRISCVvalnumOfASixFourBarrierDomain}{\saildoclabelled{sailRISCVznumzyofzya64zybarrierzydomain}{\saildocval{}{\lstinputlisting[language=sail]{sail_latex_riscv/valznum_of_a64_barrier_domainfd9b4ecf6f4c38bf5c7e299b7fb7b219.tex}}}}

\newcommand{\sailRISCVfnnumOfASixFourBarrierDomain}{\saildoclabelled{sailRISCVfnznumzyofzya64zybarrierzydomain}{\saildocfn{}{\lstinputlisting[language=sail]{sail_latex_riscv/fnznum_of_a64_barrier_domainfd9b4ecf6f4c38bf5c7e299b7fb7b219.tex}}}}

\newcommand{\sailRISCVtypeaSixFourBarrierType}{\saildoclabelled{sailRISCVtypeza64zybarrierzytype}{\saildoctype{}{\lstinputlisting[language=sail]{sail_latex_riscv/typeza64_barrier_typec915d041169289864dae061f42a2c131.tex}}}}

\newcommand{\sailRISCVvalaSixFourBarrierTypeOfNum}{\saildoclabelled{sailRISCVza64zybarrierzytypezyofzynum}{\saildocval{}{\lstinputlisting[language=sail]{sail_latex_riscv/valza64_barrier_type_of_numc06c55fe3b04f35ecb4741ea01acc85e.tex}}}}

\newcommand{\sailRISCVfnaSixFourBarrierTypeOfNum}{\saildoclabelled{sailRISCVfnza64zybarrierzytypezyofzynum}{\saildocfn{}{\lstinputlisting[language=sail]{sail_latex_riscv/fnza64_barrier_type_of_numc06c55fe3b04f35ecb4741ea01acc85e.tex}}}}

\newcommand{\sailRISCVvalnumOfASixFourBarrierType}{\saildoclabelled{sailRISCVznumzyofzya64zybarrierzytype}{\saildocval{}{\lstinputlisting[language=sail]{sail_latex_riscv/valznum_of_a64_barrier_typef15d849f5523574b740454d956b74505.tex}}}}

\newcommand{\sailRISCVfnnumOfASixFourBarrierType}{\saildoclabelled{sailRISCVfnznumzyofzya64zybarrierzytype}{\saildocfn{}{\lstinputlisting[language=sail]{sail_latex_riscv/fnznum_of_a64_barrier_typef15d849f5523574b740454d956b74505.tex}}}}

\newcommand{\sailRISCVtypebarrierKind}{\saildoclabelled{sailRISCVtypezbarrierzykind}{\saildoctype{}{\lstinputlisting[language=sail]{sail_latex_riscv/typezbarrier_kind0e1536e14e65f0b7937be9cba7867981.tex}}}}

\newcommand{\sailRISCVtypetransKind}{\saildoclabelled{sailRISCVtypeztranszykind}{\saildoctype{}{\lstinputlisting[language=sail]{sail_latex_riscv/typeztrans_kind8352eb8eb0b209f0054bd957c33bf07d.tex}}}}

\newcommand{\sailRISCVvaltransKindOfNum}{\saildoclabelled{sailRISCVztranszykindzyofzynum}{\saildocval{}{\lstinputlisting[language=sail]{sail_latex_riscv/valztrans_kind_of_num89fdff5348b6925bdad7af7bbcc092d6.tex}}}}

\newcommand{\sailRISCVfntransKindOfNum}{\saildoclabelled{sailRISCVfnztranszykindzyofzynum}{\saildocfn{}{\lstinputlisting[language=sail]{sail_latex_riscv/fnztrans_kind_of_num89fdff5348b6925bdad7af7bbcc092d6.tex}}}}

\newcommand{\sailRISCVvalnumOfTransKind}{\saildoclabelled{sailRISCVznumzyofzytranszykind}{\saildocval{}{\lstinputlisting[language=sail]{sail_latex_riscv/valznum_of_trans_kind7086883ee37c97f3e6858f19cebb2163.tex}}}}

\newcommand{\sailRISCVfnnumOfTransKind}{\saildoclabelled{sailRISCVfnznumzyofzytranszykind}{\saildocfn{}{\lstinputlisting[language=sail]{sail_latex_riscv/fnznum_of_trans_kind7086883ee37c97f3e6858f19cebb2163.tex}}}}

\newcommand{\sailRISCVtypecacheOpKind}{\saildoclabelled{sailRISCVtypezcachezyopzykind}{\saildoctype{}{\lstinputlisting[language=sail]{sail_latex_riscv/typezcache_op_kind900c2bd360568e5562e384cfad1cfbd4.tex}}}}

\newcommand{\sailRISCVvalcacheOpKindOfNum}{\saildoclabelled{sailRISCVzcachezyopzykindzyofzynum}{\saildocval{}{\lstinputlisting[language=sail]{sail_latex_riscv/valzcache_op_kind_of_num612a346f1c1edf9d5bae987ac9d9912c.tex}}}}

\newcommand{\sailRISCVfncacheOpKindOfNum}{\saildoclabelled{sailRISCVfnzcachezyopzykindzyofzynum}{\saildocfn{}{\lstinputlisting[language=sail]{sail_latex_riscv/fnzcache_op_kind_of_num612a346f1c1edf9d5bae987ac9d9912c.tex}}}}

\newcommand{\sailRISCVvalnumOfCacheOpKind}{\saildoclabelled{sailRISCVznumzyofzycachezyopzykind}{\saildocval{}{\lstinputlisting[language=sail]{sail_latex_riscv/valznum_of_cache_op_kindbd96e6225a906fea23a868ff35718006.tex}}}}

\newcommand{\sailRISCVfnnumOfCacheOpKind}{\saildoclabelled{sailRISCVfnznumzyofzycachezyopzykind}{\saildocfn{}{\lstinputlisting[language=sail]{sail_latex_riscv/fnznum_of_cache_op_kindbd96e6225a906fea23a868ff35718006.tex}}}}

\newcommand{\sailRISCVtypeinstructionKind}{\saildoclabelled{sailRISCVtypezinstructionzykind}{\saildoctype{}{\lstinputlisting[language=sail]{sail_latex_riscv/typezinstruction_kinda0a17f6dfb4c893282fe838bdd846354.tex}}}}

\newcommand{\sailRISCVvalReadMem}{\saildoclabelled{sailRISCVzzyzyreadzymem}{\saildocval{}{\lstinputlisting[language=sail]{sail_latex_riscv/valz__read_mem5b50614e040054739d7452238393251d.tex}}}}

\newcommand{\sailRISCVvalReadMemt}{\saildoclabelled{sailRISCVzzyzyreadzymemt}{\saildocval{}{\lstinputlisting[language=sail]{sail_latex_riscv/valz__read_memt00147dd8cf6dc9809e14fc1395c45ce6.tex}}}}

\newcommand{\sailRISCVvalWriteMemEa}{\saildoclabelled{sailRISCVzzyzywritezymemzyea}{\saildocval{}{\lstinputlisting[language=sail]{sail_latex_riscv/valz__write_mem_ea084b77c6ab56479698cd76a013fd7cad.tex}}}}

\newcommand{\sailRISCVvalWriteMem}{\saildoclabelled{sailRISCVzzyzywritezymem}{\saildocval{}{\lstinputlisting[language=sail]{sail_latex_riscv/valz__write_mem7fed12b7fc053a5ef3b5be1c753041b9.tex}}}}

\newcommand{\sailRISCVvalWriteMemt}{\saildoclabelled{sailRISCVzzyzywritezymemt}{\saildocval{}{\lstinputlisting[language=sail]{sail_latex_riscv/valz__write_memte6e713c02b822271b225111a241edb5f.tex}}}}

\newcommand{\sailRISCVvalWriteTag}{\saildoclabelled{sailRISCVzzyzywritezytag}{\saildocval{}{\lstinputlisting[language=sail]{sail_latex_riscv/valz__write_taga3db5671522692a95035f749d1a6fc7c.tex}}}}

\newcommand{\sailRISCVvalExclRes}{\saildoclabelled{sailRISCVzzyzyexclzyres}{\saildocval{}{\lstinputlisting[language=sail]{sail_latex_riscv/valz__excl_res213a4f8bb9ba5c1a34b50a170a41bba0.tex}}}}

\newcommand{\sailRISCVvalBarrier}{\saildoclabelled{sailRISCVzzyzybarrier}{\saildocval{}{\lstinputlisting[language=sail]{sail_latex_riscv/valz__barrier9c91ff87b358aa40ed8f2b1e1d97f44c.tex}}}}

\newcommand{\sailRISCVvalBranchAnnounce}{\saildoclabelled{sailRISCVzzyzybranchzyannounce}{\saildocval{}{\lstinputlisting[language=sail]{sail_latex_riscv/valz__branch_announce3f5ec48a7e84580ebc85c9d355048c29.tex}}}}

\newcommand{\sailRISCVvalCacheMaintenance}{\saildoclabelled{sailRISCVzzyzycachezymaintenance}{\saildocval{}{\lstinputlisting[language=sail]{sail_latex_riscv/valz__cache_maintenance664ff31aad5ce99f3549048fee01a578.tex}}}}

\newcommand{\sailRISCVvalInstrAnnounce}{\saildoclabelled{sailRISCVzzyzyinstrzyannounce}{\saildocval{}{\lstinputlisting[language=sail]{sail_latex_riscv/valz__instr_announce247eaf1a7feec56ee067d896e6f0ee3e.tex}}}}

\newcommand{\sailRISCVvalstringStartswith}{\saildoclabelled{sailRISCVzstringzystartswith}{\saildocval{}{\lstinputlisting[language=sail]{sail_latex_riscv/valzstring_startswith150dcce016e36e283e03e57c7aa7f479.tex}}}}

\newcommand{\sailRISCVvalstringDrop}{\saildoclabelled{sailRISCVzstringzydrop}{\saildocval{}{\lstinputlisting[language=sail]{sail_latex_riscv/valzstring_drop09d7231db4951d3343ee0356a8f98d4e.tex}}}}

\newcommand{\sailRISCVvalstringTake}{\saildoclabelled{sailRISCVzstringzytake}{\saildocval{}{\lstinputlisting[language=sail]{sail_latex_riscv/valzstring_takef5fd0689e1a11681ccf5471f34195539.tex}}}}

\newcommand{\sailRISCVvalstringLength}{\saildoclabelled{sailRISCVzstringzylength}{\saildocval{}{\lstinputlisting[language=sail]{sail_latex_riscv/valzstring_length138975cd51f6e879bf061905da0059f9.tex}}}}

\newcommand{\sailRISCVvalstringAppend}{\saildoclabelled{sailRISCVzstringzyappend}{\saildocval{}{\lstinputlisting[language=sail]{sail_latex_riscv/valzstring_appendca8a3ce55684edee6af875f28a1550b0.tex}}}}

\newcommand{\sailRISCVvaleqAnything}{\saildoclabelled{sailRISCVzeqzyanything}{\saildocval{}{\lstinputlisting[language=sail]{sail_latex_riscv/valzeq_anything99dff1d931070d33dac5c755eae24439.tex}}}}

\newcommand{\sailRISCVoverloadHHzEightoperatorzZerozJzJzNine}{\saildoclabelled{sailRISCVoverloadHHzz8operatorz0zJzJz9}{\saildocoverload{}{\lstinputlisting[language=sail]{sail_latex_riscv/overloadHHzz8operatorz0zjzjz9c650f45e06411dd4e97578ff2bad6338.tex}}}}

\newcommand{\sailRISCVvalregDeref}{\saildoclabelled{sailRISCVzregzyderef}{\saildocval{}{\lstinputlisting[language=sail]{sail_latex_riscv/valzreg_deref8bd1d78c61978d7074c7e0e5195e4bf7.tex}}}}

\newcommand{\sailRISCVvalRegDeref}{\saildoclabelled{sailRISCVzzyregzyderef}{\saildocval{}{\lstinputlisting[language=sail]{sail_latex_riscv/valz_reg_deref95099334a598a80d1fa0f47df99c8b42.tex}}}}

\newcommand{\sailRISCVvalanyVectorUpdate}{\saildoclabelled{sailRISCVzanyzyvectorzyupdate}{\saildocval{}{\lstinputlisting[language=sail]{sail_latex_riscv/valzany_vector_updatef0077a3dd1846db0e7c8f84fa1a2eed5.tex}}}}

\newcommand{\sailRISCVoverloadIIvectorUpdate}{\saildoclabelled{sailRISCVoverloadIIzvectorzyupdate}{\saildocoverload{}{\lstinputlisting[language=sail]{sail_latex_riscv/overloadIIzvector_updateb14d5207ae01ed7fc9d9882c9cc3ebef.tex}}}}

\newcommand{\sailRISCVvalupdateSubrange}{\saildoclabelled{sailRISCVzupdatezysubrange}{\saildocval{}{\lstinputlisting[language=sail]{sail_latex_riscv/valzupdate_subrangea3cf2a13bfd32a2a89bc44a498800493.tex}}}}

\newcommand{\sailRISCVvalvectorConcat}{\saildoclabelled{sailRISCVzvectorzyconcat}{\saildocval{}{\lstinputlisting[language=sail]{sail_latex_riscv/valzvector_concate0e61f7c9864d8d335d1f5c434546f7c.tex}}}}

\newcommand{\sailRISCVoverloadJJappend}{\saildoclabelled{sailRISCVoverloadJJzappend}{\saildocoverload{}{\lstinputlisting[language=sail]{sail_latex_riscv/overloadJJzappend88575169e0ec1639b6ae3851df999710.tex}}}}

\newcommand{\sailRISCVvalnotBit}{\saildoclabelled{sailRISCVznotzybit}{\saildocval{}{\lstinputlisting[language=sail]{sail_latex_riscv/valznot_bit3b618f3ab6887bbe967eaa12bf52b297.tex}}}}

\newcommand{\sailRISCVfnnotBit}{\saildoclabelled{sailRISCVfnznotzybit}{\saildocfn{}{\lstinputlisting[language=sail]{sail_latex_riscv/fnznot_bit3b618f3ab6887bbe967eaa12bf52b297.tex}}}}

\newcommand{\sailRISCVoverloadKKzW}{\saildoclabelled{sailRISCVoverloadKKzzW}{\saildocoverload{}{\lstinputlisting[language=sail]{sail_latex_riscv/overloadKKzzw805a9067649c7cfeedcb41b57a7e2c86.tex}}}}

\newcommand{\sailRISCVvalneqVec}{\saildoclabelled{sailRISCVzneqzyvec}{\saildocval{}{\lstinputlisting[language=sail]{sail_latex_riscv/valzneq_vecefa97ba9877d7fde3cd929d8ec7a401a.tex}}}}

\newcommand{\sailRISCVfnneqVec}{\saildoclabelled{sailRISCVfnzneqzyvec}{\saildocfn{}{\lstinputlisting[language=sail]{sail_latex_riscv/fnzneq_vecefa97ba9877d7fde3cd929d8ec7a401a.tex}}}}

\newcommand{\sailRISCVvalneqAnything}{\saildoclabelled{sailRISCVzneqzyanything}{\saildocval{}{\lstinputlisting[language=sail]{sail_latex_riscv/valzneq_anythingf220233154ca93d75c0323f604bb8d16.tex}}}}

\newcommand{\sailRISCVfnneqAnything}{\saildoclabelled{sailRISCVfnzneqzyanything}{\saildocfn{}{\lstinputlisting[language=sail]{sail_latex_riscv/fnzneq_anythingf220233154ca93d75c0323f604bb8d16.tex}}}}

\newcommand{\sailRISCVoverloadLLzEightoperatorzZerozOnezJzNine}{\saildoclabelled{sailRISCVoverloadLLzz8operatorz0z1zJz9}{\saildocoverload{}{\lstinputlisting[language=sail]{sail_latex_riscv/overloadLLzz8operatorz0z1zjz981ebe433e26f9e2dfa2a9d2c7f4fe1f4.tex}}}}

\newcommand{\sailRISCVoverloadMMzEightoperatorzZerozSixzNine}{\saildoclabelled{sailRISCVoverloadMMzz8operatorz0z6z9}{\saildocoverload{}{\lstinputlisting[language=sail]{sail_latex_riscv/overloadMMzz8operatorz0z6z9d3731bb9b1c9d765858778ad48ba6b3a.tex}}}}

\newcommand{\sailRISCVoverloadNNzEightoperatorzZerozUzNine}{\saildoclabelled{sailRISCVoverloadNNzz8operatorz0zUz9}{\saildocoverload{}{\lstinputlisting[language=sail]{sail_latex_riscv/overloadNNzz8operatorz0zuz99af95b281314726fa91893b57fc290dc.tex}}}}

\newcommand{\sailRISCVvalstringOfInt}{\saildoclabelled{sailRISCVzstringzyofzyint}{\saildocval{}{\lstinputlisting[language=sail]{sail_latex_riscv/valzstring_of_int03988e4e2d2976513793427ac823afbe.tex}}}}

\newcommand{\sailRISCVvalstringOfBits}{\saildoclabelled{sailRISCVzstringzyofzybits}{\saildocval{}{\lstinputlisting[language=sail]{sail_latex_riscv/valzstring_of_bits43debe172d2009dbb056cb5252821d62.tex}}}}

\newcommand{\sailRISCVvalstringOfBit}{\saildoclabelled{sailRISCVzstringzyofzybit}{\saildocval{}{\lstinputlisting[language=sail]{sail_latex_riscv/valzstring_of_bit7313cdbf7b05129d4977581f9bb14794.tex}}}}

\newcommand{\sailRISCVfnstringOfBit}{\saildoclabelled{sailRISCVfnzstringzyofzybit}{\saildocfn{}{\lstinputlisting[language=sail]{sail_latex_riscv/fnzstring_of_bit7313cdbf7b05129d4977581f9bb14794.tex}}}}

\newcommand{\sailRISCVoverloadOOBitStr}{\saildoclabelled{sailRISCVoverloadOOzBitStr}{\saildocoverload{}{\lstinputlisting[language=sail]{sail_latex_riscv/overloadOOzbitstr0d04da018975c4776e05a9c59c2e380e.tex}}}}



\newcommand{\sailRISCVvalintPower}{\saildoclabelled{sailRISCVzintzypower}{\saildocval{}{\lstinputlisting[language=sail]{sail_latex_riscv/valzint_powerb0c5fc1a9fb0852260414607a93aeae6.tex}}}}

\newcommand{\sailRISCVoverloadPPzEightoperatorzZerozQzNine}{\saildoclabelled{sailRISCVoverloadPPzz8operatorz0zQz9}{\saildocoverload{}{\lstinputlisting[language=sail]{sail_latex_riscv/overloadPPzz8operatorz0zqz9ccbd65071d8f0fbb9677c7f6e86d3527.tex}}}}

\newcommand{\sailRISCVvalsubVec}{\saildoclabelled{sailRISCVzsubzyvec}{\saildocval{}{\lstinputlisting[language=sail]{sail_latex_riscv/valzsub_vec326e0ba0bb00229be26645e2d44dbd83.tex}}}}

\newcommand{\sailRISCVvalsubVecInt}{\saildoclabelled{sailRISCVzsubzyveczyint}{\saildocval{}{\lstinputlisting[language=sail]{sail_latex_riscv/valzsub_vec_int5e6c04459782b1b8cc706ba2e4c8a435.tex}}}}

\newcommand{\sailRISCVoverloadQQzEightoperatorzZerozDzNine}{\saildoclabelled{sailRISCVoverloadQQzz8operatorz0zDz9}{\saildocoverload{}{\lstinputlisting[language=sail]{sail_latex_riscv/overloadQQzz8operatorz0zdz9aaaae29f381509679e21c2555127a5dd.tex}}}}

\newcommand{\sailRISCVvalquotRoundZero}{\saildoclabelled{sailRISCVzquotzyroundzyzzero}{\saildocval{}{\lstinputlisting[language=sail]{sail_latex_riscv/valzquot_round_zzeroa8d9d278dc91a14956dfe19d01766403.tex}}}}

\newcommand{\sailRISCVvalremRoundZero}{\saildoclabelled{sailRISCVzremzyroundzyzzero}{\saildocval{}{\lstinputlisting[language=sail]{sail_latex_riscv/valzrem_round_zzero90d115d6c3e756b94f7766d1b76fbb83.tex}}}}

\newcommand{\sailRISCVoverloadRRzEightoperatorzZerozFivezNine}{\saildoclabelled{sailRISCVoverloadRRzz8operatorz0z5z9}{\saildocoverload{}{\lstinputlisting[language=sail]{sail_latex_riscv/overloadRRzz8operatorz0z5z9194a289f0ceb02e29c9b6febc5146071.tex}}}}

\newcommand{\sailRISCVvalminInt}{\saildoclabelled{sailRISCVzminzyint}{\saildocval{}{\lstinputlisting[language=sail]{sail_latex_riscv/valzmin_intaf4626ab3b9c2d0b9494d7e8d265dd26.tex}}}}

\newcommand{\sailRISCVvalmaxInt}{\saildoclabelled{sailRISCVzmaxzyint}{\saildocval{}{\lstinputlisting[language=sail]{sail_latex_riscv/valzmax_inta8f95a0baf723be8373221a893afa8f3.tex}}}}

\newcommand{\sailRISCVoverloadSSmin}{\saildoclabelled{sailRISCVoverloadSSzmin}{\saildocoverload{}{\lstinputlisting[language=sail]{sail_latex_riscv/overloadSSzmin95ae3c0ebde1421750e6db87bdf74801.tex}}}}

\newcommand{\sailRISCVoverloadTTmax}{\saildoclabelled{sailRISCVoverloadTTzmax}{\saildocoverload{}{\lstinputlisting[language=sail]{sail_latex_riscv/overloadTTzmax91b641c464c0dc87660499321a356d93.tex}}}}

\newcommand{\sailRISCVvalpowTwo}{\saildoclabelled{sailRISCVzpow2}{\saildocval{}{\lstinputlisting[language=sail]{sail_latex_riscv/valzpow2e971ce2f9ebb899590551317286dfd1b.tex}}}}

\newcommand{\sailRISCVvalprint}{\saildoclabelled{sailRISCVzprint}{\saildocval{}{\lstinputlisting[language=sail]{sail_latex_riscv/valzprintc9b8c9c569def1934362480628956c85.tex}}}}

\newcommand{\sailRISCVvalprintString}{\saildoclabelled{sailRISCVzprintzystring}{\saildocval{}{\lstinputlisting[language=sail]{sail_latex_riscv/valzprint_string4b7e44eb021c25878c749374ea5657f7.tex}}}}

\newcommand{\sailRISCVvalprintInstr}{\saildoclabelled{sailRISCVzprintzyinstr}{\saildocval{}{\lstinputlisting[language=sail]{sail_latex_riscv/valzprint_instr59f46726e427ed18b9d5d81c8247a576.tex}}}}

\newcommand{\sailRISCVvalprintReg}{\saildoclabelled{sailRISCVzprintzyreg}{\saildocval{}{\lstinputlisting[language=sail]{sail_latex_riscv/valzprint_reg1e86a2863e701cf481babd6538033417.tex}}}}

\newcommand{\sailRISCVvalprintMem}{\saildoclabelled{sailRISCVzprintzymem}{\saildocval{}{\lstinputlisting[language=sail]{sail_latex_riscv/valzprint_memb6cd54f21a6e2c7a036f86473d516264.tex}}}}

\newcommand{\sailRISCVvalprintPlatform}{\saildoclabelled{sailRISCVzprintzyplatform}{\saildocval{}{\lstinputlisting[language=sail]{sail_latex_riscv/valzprint_platform894b0df15559ad78744140d27df61719.tex}}}}

\newcommand{\sailRISCVvalgetConfigPrintInstr}{\saildoclabelled{sailRISCVzgetzyconfigzyprintzyinstr}{\saildocval{}{\lstinputlisting[language=sail]{sail_latex_riscv/valzget_config_print_instrcd725d09d3941c391aadf6b945a364c4.tex}}}}

\newcommand{\sailRISCVvalgetConfigPrintReg}{\saildoclabelled{sailRISCVzgetzyconfigzyprintzyreg}{\saildocval{}{\lstinputlisting[language=sail]{sail_latex_riscv/valzget_config_print_rega14ad214dae5f2d46538a3770abc93e2.tex}}}}

\newcommand{\sailRISCVvalgetConfigPrintMem}{\saildoclabelled{sailRISCVzgetzyconfigzyprintzymem}{\saildocval{}{\lstinputlisting[language=sail]{sail_latex_riscv/valzget_config_print_memae95e5785b79c4a3f0105a772bf99dca.tex}}}}

\newcommand{\sailRISCVvalgetConfigPrintPlatform}{\saildoclabelled{sailRISCVzgetzyconfigzyprintzyplatform}{\saildocval{}{\lstinputlisting[language=sail]{sail_latex_riscv/valzget_config_print_platform65eee2fe1a7d52174acea1de4e724e03.tex}}}}

\newcommand{\sailRISCVfngetConfigPrintInstr}{\saildoclabelled{sailRISCVfnzgetzyconfigzyprintzyinstr}{\saildocfn{}{\lstinputlisting[language=sail]{sail_latex_riscv/fnzget_config_print_instrcd725d09d3941c391aadf6b945a364c4.tex}}}}

\newcommand{\sailRISCVfngetConfigPrintReg}{\saildoclabelled{sailRISCVfnzgetzyconfigzyprintzyreg}{\saildocfn{}{\lstinputlisting[language=sail]{sail_latex_riscv/fnzget_config_print_rega14ad214dae5f2d46538a3770abc93e2.tex}}}}

\newcommand{\sailRISCVfngetConfigPrintMem}{\saildoclabelled{sailRISCVfnzgetzyconfigzyprintzymem}{\saildocfn{}{\lstinputlisting[language=sail]{sail_latex_riscv/fnzget_config_print_memae95e5785b79c4a3f0105a772bf99dca.tex}}}}

\newcommand{\sailRISCVfngetConfigPrintPlatform}{\saildoclabelled{sailRISCVfnzgetzyconfigzyprintzyplatform}{\saildocfn{}{\lstinputlisting[language=sail]{sail_latex_riscv/fnzget_config_print_platform65eee2fe1a7d52174acea1de4e724e03.tex}}}}

\newcommand{\sailRISCVvalEXTS}{\saildoclabelled{sailRISCVzEXTS}{\saildocval{}{\lstinputlisting[language=sail]{sail_latex_riscv/valzexts8a10d418fac6a2072ef1dfede4580873.tex}}}}

\newcommand{\sailRISCVvalEXTZ}{\saildoclabelled{sailRISCVzEXTZ}{\saildocval{}{\lstinputlisting[language=sail]{sail_latex_riscv/valzextzdb77018947d632a113deb15d298290d4.tex}}}}

\newcommand{\sailRISCVfnEXTS}{\saildoclabelled{sailRISCVfnzEXTS}{\saildocfn{}{\lstinputlisting[language=sail]{sail_latex_riscv/fnzexts8a10d418fac6a2072ef1dfede4580873.tex}}}}

\newcommand{\sailRISCVfnEXTZ}{\saildoclabelled{sailRISCVfnzEXTZ}{\saildocfn{}{\lstinputlisting[language=sail]{sail_latex_riscv/fnzextzdb77018947d632a113deb15d298290d4.tex}}}}

\newcommand{\sailRISCVvalzzerosImplicit}{\saildoclabelled{sailRISCVzzzeroszyimplicit}{\saildocval{}{\lstinputlisting[language=sail]{sail_latex_riscv/valzzzeros_implicitce1dd4153c9a1823a9697c4472c43ebf.tex}}}}

\newcommand{\sailRISCVfnzzerosImplicit}{\saildoclabelled{sailRISCVfnzzzeroszyimplicit}{\saildocfn{}{\lstinputlisting[language=sail]{sail_latex_riscv/fnzzzeros_implicitce1dd4153c9a1823a9697c4472c43ebf.tex}}}}

\newcommand{\sailRISCVoverloadUUzzeros}{\saildoclabelled{sailRISCVoverloadUUzzzeros}{\saildocoverload{}{\lstinputlisting[language=sail]{sail_latex_riscv/overloadUUzzzerosc530711942e216cef3921733c1c5d101.tex}}}}

\newcommand{\sailRISCVvalones}{\saildoclabelled{sailRISCVzones}{\saildocval{}{\lstinputlisting[language=sail]{sail_latex_riscv/valzones26f94136f5db8afd4e9df1e512f7fdc5.tex}}}}

\newcommand{\sailRISCVfnones}{\saildoclabelled{sailRISCVfnzones}{\saildocfn{}{\lstinputlisting[language=sail]{sail_latex_riscv/fnzones26f94136f5db8afd4e9df1e512f7fdc5.tex}}}}

\newcommand{\sailRISCVvalboolToBits}{\saildoclabelled{sailRISCVzboolzytozybits}{\saildocval{}{\lstinputlisting[language=sail]{sail_latex_riscv/valzbool_to_bits827ded794caf4c773562dc8baff6a29a.tex}}}}

\newcommand{\sailRISCVfnboolToBits}{\saildoclabelled{sailRISCVfnzboolzytozybits}{\saildocfn{}{\lstinputlisting[language=sail]{sail_latex_riscv/fnzbool_to_bits827ded794caf4c773562dc8baff6a29a.tex}}}}

\newcommand{\sailRISCVvalbitToBool}{\saildoclabelled{sailRISCVzbitzytozybool}{\saildocval{}{\lstinputlisting[language=sail]{sail_latex_riscv/valzbit_to_bool238fffa8d41cb3108fd20322f5500ff3.tex}}}}

\newcommand{\sailRISCVfnbitToBool}{\saildoclabelled{sailRISCVfnzbitzytozybool}{\saildocfn{}{\lstinputlisting[language=sail]{sail_latex_riscv/fnzbit_to_bool238fffa8d41cb3108fd20322f5500ff3.tex}}}}

\newcommand{\sailRISCVvaltoBits}{\saildoclabelled{sailRISCVztozybits}{\saildocval{}{\lstinputlisting[language=sail]{sail_latex_riscv/valzto_bits9fb7c0bf64c9bfa589ae4882a09f2a40.tex}}}}

\newcommand{\sailRISCVfntoBits}{\saildoclabelled{sailRISCVfnztozybits}{\saildocfn{}{\lstinputlisting[language=sail]{sail_latex_riscv/fnzto_bits9fb7c0bf64c9bfa589ae4882a09f2a40.tex}}}}

\newcommand{\sailRISCVvalzEightoperatorzZerozISzNine}{\saildoclabelled{sailRISCVzz8operatorz0zIzysz9}{\saildocval{}{\lstinputlisting[language=sail]{sail_latex_riscv/valzz8operatorz0zi_sz956bf0eb8f384ccc952f43b53c00f14d1.tex}}}}

\newcommand{\sailRISCVvalzEightoperatorzZerozKzJSzNine}{\saildoclabelled{sailRISCVzz8operatorz0zKzJzysz9}{\saildocval{}{\lstinputlisting[language=sail]{sail_latex_riscv/valzz8operatorz0zkzj_sz904d1eed458afb5704c50166298da928d.tex}}}}

\newcommand{\sailRISCVvalzEightoperatorzZerozIUzNine}{\saildoclabelled{sailRISCVzz8operatorz0zIzyuz9}{\saildocval{}{\lstinputlisting[language=sail]{sail_latex_riscv/valzz8operatorz0zi_uz975e6e2563e418725e99f2d020a6e269f.tex}}}}

\newcommand{\sailRISCVvalzEightoperatorzZerozKzJUzNine}{\saildoclabelled{sailRISCVzz8operatorz0zKzJzyuz9}{\saildocval{}{\lstinputlisting[language=sail]{sail_latex_riscv/valzz8operatorz0zkzj_uz932ccbf178c78f699a55ad5e4e3db033c.tex}}}}

\newcommand{\sailRISCVvalzEightoperatorzZerozIzJUzNine}{\saildoclabelled{sailRISCVzz8operatorz0zIzJzyuz9}{\saildocval{}{\lstinputlisting[language=sail]{sail_latex_riscv/valzz8operatorz0zizj_uz99c310fa9a514922f781c01ba7354f99f.tex}}}}

\newcommand{\sailRISCVfnzEightoperatorzZerozISzNine}{\saildoclabelled{sailRISCVfnzz8operatorz0zIzysz9}{\saildocfn{}{\lstinputlisting[language=sail]{sail_latex_riscv/fnzz8operatorz0zi_sz956bf0eb8f384ccc952f43b53c00f14d1.tex}}}}

\newcommand{\sailRISCVfnzEightoperatorzZerozKzJSzNine}{\saildoclabelled{sailRISCVfnzz8operatorz0zKzJzysz9}{\saildocfn{}{\lstinputlisting[language=sail]{sail_latex_riscv/fnzz8operatorz0zkzj_sz904d1eed458afb5704c50166298da928d.tex}}}}

\newcommand{\sailRISCVfnzEightoperatorzZerozIUzNine}{\saildoclabelled{sailRISCVfnzz8operatorz0zIzyuz9}{\saildocfn{}{\lstinputlisting[language=sail]{sail_latex_riscv/fnzz8operatorz0zi_uz975e6e2563e418725e99f2d020a6e269f.tex}}}}

\newcommand{\sailRISCVfnzEightoperatorzZerozKzJUzNine}{\saildoclabelled{sailRISCVfnzz8operatorz0zKzJzyuz9}{\saildocfn{}{\lstinputlisting[language=sail]{sail_latex_riscv/fnzz8operatorz0zkzj_uz932ccbf178c78f699a55ad5e4e3db033c.tex}}}}

\newcommand{\sailRISCVfnzEightoperatorzZerozIzJUzNine}{\saildoclabelled{sailRISCVfnzz8operatorz0zIzJzyuz9}{\saildocfn{}{\lstinputlisting[language=sail]{sail_latex_riscv/fnzz8operatorz0zizj_uz99c310fa9a514922f781c01ba7354f99f.tex}}}}

\newcommand{\sailRISCVvalshiftBitsRight}{\saildoclabelled{sailRISCVzshiftzybitszyright}{\saildocval{}{\lstinputlisting[language=sail]{sail_latex_riscv/valzshift_bits_right281f5e6a28fe3c92d35fe5c78a0deb41.tex}}}}

\newcommand{\sailRISCVvalshiftBitsLeft}{\saildoclabelled{sailRISCVzshiftzybitszyleft}{\saildocval{}{\lstinputlisting[language=sail]{sail_latex_riscv/valzshift_bits_left0754e8b870e2a3ba46646c35dac7af10.tex}}}}

\newcommand{\sailRISCVvalshiftl}{\saildoclabelled{sailRISCVzshiftl}{\saildocval{}{\lstinputlisting[language=sail]{sail_latex_riscv/valzshiftl7827d0dcac29bd8258f158e7c1e77658.tex}}}}

\newcommand{\sailRISCVvalshiftr}{\saildoclabelled{sailRISCVzshiftr}{\saildocval{}{\lstinputlisting[language=sail]{sail_latex_riscv/valzshiftr173b7dba7206ed1b61a12344bdf9182a.tex}}}}

\newcommand{\sailRISCVoverloadVVzEightoperatorzZerozKzKzNine}{\saildoclabelled{sailRISCVoverloadVVzz8operatorz0zKzKz9}{\saildocoverload{}{\lstinputlisting[language=sail]{sail_latex_riscv/overloadVVzz8operatorz0zkzkz9e772b5e121d0113826739b52dbbce0f8.tex}}}}

\newcommand{\sailRISCVoverloadWWzEightoperatorzZerozIzIzNine}{\saildoclabelled{sailRISCVoverloadWWzz8operatorz0zIzIz9}{\saildocoverload{}{\lstinputlisting[language=sail]{sail_latex_riscv/overloadWWzz8operatorz0ziziz90068ca3610cb726b2dddda4048ca7686.tex}}}}

\newcommand{\sailRISCVvalshiftRightArithSixFour}{\saildoclabelled{sailRISCVzshiftzyrightzyarith64}{\saildocval{}{\lstinputlisting[language=sail]{sail_latex_riscv/valzshift_right_arith642d6a56971daae2b1fdb862ebbbaf6a46.tex}}}}

\newcommand{\sailRISCVfnshiftRightArithSixFour}{\saildoclabelled{sailRISCVfnzshiftzyrightzyarith64}{\saildocfn{}{\lstinputlisting[language=sail]{sail_latex_riscv/fnzshift_right_arith642d6a56971daae2b1fdb862ebbbaf6a46.tex}}}}

\newcommand{\sailRISCVvalshiftRightArithThreeTwo}{\saildoclabelled{sailRISCVzshiftzyrightzyarith32}{\saildocval{}{\lstinputlisting[language=sail]{sail_latex_riscv/valzshift_right_arith32247e0e7505241d38fca8e6a3bcdfea9e.tex}}}}

\newcommand{\sailRISCVfnshiftRightArithThreeTwo}{\saildoclabelled{sailRISCVfnzshiftzyrightzyarith32}{\saildocfn{}{\lstinputlisting[language=sail]{sail_latex_riscv/fnzshift_right_arith32247e0e7505241d38fca8e6a3bcdfea9e.tex}}}}

\newcommand{\sailRISCVvalrotateBitsRight}{\saildoclabelled{sailRISCVzrotatezybitszyright}{\saildocval{}{\lstinputlisting[language=sail]{sail_latex_riscv/valzrotate_bits_right483383a7bc5976025102ba9af61ab11d.tex}}}}

\newcommand{\sailRISCVfnrotateBitsRight}{\saildoclabelled{sailRISCVfnzrotatezybitszyright}{\saildocfn{}{\lstinputlisting[language=sail]{sail_latex_riscv/fnzrotate_bits_right483383a7bc5976025102ba9af61ab11d.tex}}}}

\newcommand{\sailRISCVvalrotateBitsLeft}{\saildoclabelled{sailRISCVzrotatezybitszyleft}{\saildocval{}{\lstinputlisting[language=sail]{sail_latex_riscv/valzrotate_bits_left4e03f7aab76acaf8206bfa04b360ceec.tex}}}}

\newcommand{\sailRISCVfnrotateBitsLeft}{\saildoclabelled{sailRISCVfnzrotatezybitszyleft}{\saildocfn{}{\lstinputlisting[language=sail]{sail_latex_riscv/fnzrotate_bits_left4e03f7aab76acaf8206bfa04b360ceec.tex}}}}

\newcommand{\sailRISCVvalrotater}{\saildoclabelled{sailRISCVzrotater}{\saildocval{}{\lstinputlisting[language=sail]{sail_latex_riscv/valzrotater926375a3d21b09b3881f991747cd9b14.tex}}}}

\newcommand{\sailRISCVfnrotater}{\saildoclabelled{sailRISCVfnzrotater}{\saildocfn{}{\lstinputlisting[language=sail]{sail_latex_riscv/fnzrotater926375a3d21b09b3881f991747cd9b14.tex}}}}

\newcommand{\sailRISCVvalrotatel}{\saildoclabelled{sailRISCVzrotatel}{\saildocval{}{\lstinputlisting[language=sail]{sail_latex_riscv/valzrotatelf071331b6fd023da4d264842a39a016a.tex}}}}

\newcommand{\sailRISCVfnrotatel}{\saildoclabelled{sailRISCVfnzrotatel}{\saildocfn{}{\lstinputlisting[language=sail]{sail_latex_riscv/fnzrotatelf071331b6fd023da4d264842a39a016a.tex}}}}

\newcommand{\sailRISCVoverloadXXzEightoperatorzZerozKzKzKzNine}{\saildoclabelled{sailRISCVoverloadXXzz8operatorz0zKzKzKz9}{\saildocoverload{}{\lstinputlisting[language=sail]{sail_latex_riscv/overloadXXzz8operatorz0zkzkzkz957fa2f300078df169cd8f1b6939b710d.tex}}}}

\newcommand{\sailRISCVoverloadYYzEightoperatorzZerozIzIzIzNine}{\saildoclabelled{sailRISCVoverloadYYzz8operatorz0zIzIzIz9}{\saildocoverload{}{\lstinputlisting[language=sail]{sail_latex_riscv/overloadYYzz8operatorz0ziziziz9daf9b9d2ffdbec7af40418527044fa38.tex}}}}

\newcommand{\sailRISCVvalreverseBitsInByte}{\saildoclabelled{sailRISCVzreversezybitszyinzybyte}{\saildocval{}{\lstinputlisting[language=sail]{sail_latex_riscv/valzreverse_bits_in_byte37f50ef4b8566a812c20f22ef8bba201.tex}}}}

\newcommand{\sailRISCVfnreverseBitsInByte}{\saildoclabelled{sailRISCVfnzreversezybitszyinzybyte}{\saildocfn{}{\lstinputlisting[language=sail]{sail_latex_riscv/fnzreverse_bits_in_byte37f50ef4b8566a812c20f22ef8bba201.tex}}}}

\newcommand{\sailRISCVoverloadZZreverse}{\saildoclabelled{sailRISCVoverloadZZzreverse}{\saildocoverload{}{\lstinputlisting[language=sail]{sail_latex_riscv/overloadZZzreversed9f0df85c2b57ca70b0103d18920d22f.tex}}}}

\newcommand{\sailRISCVvalspc}{\saildoclabelled{sailRISCVzspc}{\saildocval{}{\lstinputlisting[language=sail]{sail_latex_riscv/valzspca574d99b4c3d28e08386a1f673633994.tex}}}}

\newcommand{\sailRISCVvaloptSpc}{\saildoclabelled{sailRISCVzoptzyspc}{\saildocval{}{\lstinputlisting[language=sail]{sail_latex_riscv/valzopt_spc4aab1150dfed90f36fea1776963edbf0.tex}}}}

\newcommand{\sailRISCVvaldefSpc}{\saildoclabelled{sailRISCVzdefzyspc}{\saildocval{}{\lstinputlisting[language=sail]{sail_latex_riscv/valzdef_spce04ebdaa1e0acd4aa4dd3326642e673e.tex}}}}

\newcommand{\sailRISCVvaldecimalStringOfBits}{\saildoclabelled{sailRISCVzdecimalzystringzyofzybits}{\saildocval{}{\lstinputlisting[language=sail]{sail_latex_riscv/valzdecimal_string_of_bits7da73b6b29137ed7163460292c5440b0.tex}}}}

\newcommand{\sailRISCVvalhexBits}{\saildoclabelled{sailRISCVzhexzybits}{\saildocval{}{\lstinputlisting[language=sail]{sail_latex_riscv/valzhex_bits6151f0f3396959dd9a279f1e74f7d7ec.tex}}}}

\newcommand{\sailRISCVvalnLeadingSpaces}{\saildoclabelled{sailRISCVznzyleadingzyspaces}{\saildocval{}{\lstinputlisting[language=sail]{sail_latex_riscv/valzn_leading_spaces05ea6c2f03435a60412f4bef062a912a.tex}}}}

\newcommand{\sailRISCVfnnLeadingSpaces}{\saildoclabelled{sailRISCVfnznzyleadingzyspaces}{\saildocfn{}{\lstinputlisting[language=sail]{sail_latex_riscv/fnzn_leading_spaces05ea6c2f03435a60412f4bef062a912a.tex}}}}

\newcommand{\sailRISCVvalspcForwards}{\saildoclabelled{sailRISCVzspczyforwards}{\saildocval{}{\lstinputlisting[language=sail]{sail_latex_riscv/valzspc_forwardsabfa1efbce2d58c6d3e26c86435d3af4.tex}}}}

\newcommand{\sailRISCVfnspcForwards}{\saildoclabelled{sailRISCVfnzspczyforwards}{\saildocfn{}{\lstinputlisting[language=sail]{sail_latex_riscv/fnzspc_forwardsabfa1efbce2d58c6d3e26c86435d3af4.tex}}}}

\newcommand{\sailRISCVvalspcBackwards}{\saildoclabelled{sailRISCVzspczybackwards}{\saildocval{}{\lstinputlisting[language=sail]{sail_latex_riscv/valzspc_backwardsa712e20ab4070963924d2974cc8aa941.tex}}}}

\newcommand{\sailRISCVfnspcBackwards}{\saildoclabelled{sailRISCVfnzspczybackwards}{\saildocfn{}{\lstinputlisting[language=sail]{sail_latex_riscv/fnzspc_backwardsa712e20ab4070963924d2974cc8aa941.tex}}}}

\newcommand{\sailRISCVvalspcMatchesPrefix}{\saildoclabelled{sailRISCVzspczymatcheszyprefix}{\saildocval{}{\lstinputlisting[language=sail]{sail_latex_riscv/valzspc_matches_prefix38c7965c7edeefb5fb2ccd6915f5bdbb.tex}}}}

\newcommand{\sailRISCVfnspcMatchesPrefix}{\saildoclabelled{sailRISCVfnzspczymatcheszyprefix}{\saildocfn{}{\lstinputlisting[language=sail]{sail_latex_riscv/fnzspc_matches_prefix38c7965c7edeefb5fb2ccd6915f5bdbb.tex}}}}

\newcommand{\sailRISCVvaloptSpcForwards}{\saildoclabelled{sailRISCVzoptzyspczyforwards}{\saildocval{}{\lstinputlisting[language=sail]{sail_latex_riscv/valzopt_spc_forwards395c7cf20c474712cbbb7c80edd24bda.tex}}}}

\newcommand{\sailRISCVfnoptSpcForwards}{\saildoclabelled{sailRISCVfnzoptzyspczyforwards}{\saildocfn{}{\lstinputlisting[language=sail]{sail_latex_riscv/fnzopt_spc_forwards395c7cf20c474712cbbb7c80edd24bda.tex}}}}

\newcommand{\sailRISCVvaloptSpcBackwards}{\saildoclabelled{sailRISCVzoptzyspczybackwards}{\saildocval{}{\lstinputlisting[language=sail]{sail_latex_riscv/valzopt_spc_backwards68e297450ccdf6f2339325379c27029f.tex}}}}

\newcommand{\sailRISCVfnoptSpcBackwards}{\saildoclabelled{sailRISCVfnzoptzyspczybackwards}{\saildocfn{}{\lstinputlisting[language=sail]{sail_latex_riscv/fnzopt_spc_backwards68e297450ccdf6f2339325379c27029f.tex}}}}

\newcommand{\sailRISCVvaloptSpcMatchesPrefix}{\saildoclabelled{sailRISCVzoptzyspczymatcheszyprefix}{\saildocval{}{\lstinputlisting[language=sail]{sail_latex_riscv/valzopt_spc_matches_prefix495f7798e6650e2ff628a5b7715c161c.tex}}}}

\newcommand{\sailRISCVfnoptSpcMatchesPrefix}{\saildoclabelled{sailRISCVfnzoptzyspczymatcheszyprefix}{\saildocfn{}{\lstinputlisting[language=sail]{sail_latex_riscv/fnzopt_spc_matches_prefix495f7798e6650e2ff628a5b7715c161c.tex}}}}

\newcommand{\sailRISCVvaldefSpcForwards}{\saildoclabelled{sailRISCVzdefzyspczyforwards}{\saildocval{}{\lstinputlisting[language=sail]{sail_latex_riscv/valzdef_spc_forwards4eafa854d5b706686aca12e499e738fa.tex}}}}

\newcommand{\sailRISCVfndefSpcForwards}{\saildoclabelled{sailRISCVfnzdefzyspczyforwards}{\saildocfn{}{\lstinputlisting[language=sail]{sail_latex_riscv/fnzdef_spc_forwards4eafa854d5b706686aca12e499e738fa.tex}}}}

\newcommand{\sailRISCVvaldefSpcBackwards}{\saildoclabelled{sailRISCVzdefzyspczybackwards}{\saildocval{}{\lstinputlisting[language=sail]{sail_latex_riscv/valzdef_spc_backwardseab568d1f8e592642bb1655eb934a620.tex}}}}

\newcommand{\sailRISCVfndefSpcBackwards}{\saildoclabelled{sailRISCVfnzdefzyspczybackwards}{\saildocfn{}{\lstinputlisting[language=sail]{sail_latex_riscv/fnzdef_spc_backwardseab568d1f8e592642bb1655eb934a620.tex}}}}

\newcommand{\sailRISCVvaldefSpcMatchesPrefix}{\saildoclabelled{sailRISCVzdefzyspczymatcheszyprefix}{\saildocval{}{\lstinputlisting[language=sail]{sail_latex_riscv/valzdef_spc_matches_prefix0e41afd1e8fe11919b3e9439288e00c3.tex}}}}

\newcommand{\sailRISCVfndefSpcMatchesPrefix}{\saildoclabelled{sailRISCVfnzdefzyspczymatcheszyprefix}{\saildocfn{}{\lstinputlisting[language=sail]{sail_latex_riscv/fnzdef_spc_matches_prefix0e41afd1e8fe11919b3e9439288e00c3.tex}}}}

\newcommand{\sailRISCVvalhexBitsOne}{\saildoclabelled{sailRISCVzhexzybitszy1}{\saildocval{}{\lstinputlisting[language=sail]{sail_latex_riscv/valzhex_bits_1478f9a2bfac5b1844d4822c98afbb89d.tex}}}}

\newcommand{\sailRISCVvalhexBitsOneForwards}{\saildoclabelled{sailRISCVzhexzybitszy1zyforwards}{\saildocval{}{\lstinputlisting[language=sail]{sail_latex_riscv/valzhex_bits_1_forwardsf283099d0fc824606ed0d57afc78de8a.tex}}}}

\newcommand{\sailRISCVvalhexBitsOneForwardsMatches}{\saildoclabelled{sailRISCVzhexzybitszy1zyforwardszymatches}{\saildocval{}{\lstinputlisting[language=sail]{sail_latex_riscv/valzhex_bits_1_forwards_matches95bbf636d6fecebea6adab4fb1b42ac0.tex}}}}

\newcommand{\sailRISCVfnhexBitsOneForwardsMatches}{\saildoclabelled{sailRISCVfnzhexzybitszy1zyforwardszymatches}{\saildocfn{}{\lstinputlisting[language=sail]{sail_latex_riscv/fnzhex_bits_1_forwards_matches95bbf636d6fecebea6adab4fb1b42ac0.tex}}}}

\newcommand{\sailRISCVvalhexBitsOneMatchesPrefix}{\saildoclabelled{sailRISCVzhexzybitszy1zymatcheszyprefix}{\saildocval{}{\lstinputlisting[language=sail]{sail_latex_riscv/valzhex_bits_1_matches_prefixd812208f2aa5a7b3cf37c8de5c575662.tex}}}}

\newcommand{\sailRISCVvalhexBitsOneBackwardsMatches}{\saildoclabelled{sailRISCVzhexzybitszy1zybackwardszymatches}{\saildocval{}{\lstinputlisting[language=sail]{sail_latex_riscv/valzhex_bits_1_backwards_matches3d78f15da2768104290b952c286f2f91.tex}}}}

\newcommand{\sailRISCVfnhexBitsOneBackwardsMatches}{\saildoclabelled{sailRISCVfnzhexzybitszy1zybackwardszymatches}{\saildocfn{}{\lstinputlisting[language=sail]{sail_latex_riscv/fnzhex_bits_1_backwards_matches3d78f15da2768104290b952c286f2f91.tex}}}}

\newcommand{\sailRISCVvalhexBitsOneBackwards}{\saildoclabelled{sailRISCVzhexzybitszy1zybackwards}{\saildocval{}{\lstinputlisting[language=sail]{sail_latex_riscv/valzhex_bits_1_backwards90e5440bb0c40b9363498a8d972dba41.tex}}}}

\newcommand{\sailRISCVfnhexBitsOneBackwards}{\saildoclabelled{sailRISCVfnzhexzybitszy1zybackwards}{\saildocfn{}{\lstinputlisting[language=sail]{sail_latex_riscv/fnzhex_bits_1_backwards90e5440bb0c40b9363498a8d972dba41.tex}}}}

\newcommand{\sailRISCVvalhexBitsTwo}{\saildoclabelled{sailRISCVzhexzybitszy2}{\saildocval{}{\lstinputlisting[language=sail]{sail_latex_riscv/valzhex_bits_270bd1024c2f2d5860abcddb1e5cdd513.tex}}}}

\newcommand{\sailRISCVvalhexBitsTwoForwards}{\saildoclabelled{sailRISCVzhexzybitszy2zyforwards}{\saildocval{}{\lstinputlisting[language=sail]{sail_latex_riscv/valzhex_bits_2_forwards72b197d3d92a10d7d05fb57812f597ab.tex}}}}

\newcommand{\sailRISCVvalhexBitsTwoForwardsMatches}{\saildoclabelled{sailRISCVzhexzybitszy2zyforwardszymatches}{\saildocval{}{\lstinputlisting[language=sail]{sail_latex_riscv/valzhex_bits_2_forwards_matches7ab140c97f69659d3ab1d910b07425fb.tex}}}}

\newcommand{\sailRISCVfnhexBitsTwoForwardsMatches}{\saildoclabelled{sailRISCVfnzhexzybitszy2zyforwardszymatches}{\saildocfn{}{\lstinputlisting[language=sail]{sail_latex_riscv/fnzhex_bits_2_forwards_matches7ab140c97f69659d3ab1d910b07425fb.tex}}}}

\newcommand{\sailRISCVvalhexBitsTwoMatchesPrefix}{\saildoclabelled{sailRISCVzhexzybitszy2zymatcheszyprefix}{\saildocval{}{\lstinputlisting[language=sail]{sail_latex_riscv/valzhex_bits_2_matches_prefix948a15ea9e341681f6b07cb073ac1114.tex}}}}

\newcommand{\sailRISCVvalhexBitsTwoBackwardsMatches}{\saildoclabelled{sailRISCVzhexzybitszy2zybackwardszymatches}{\saildocval{}{\lstinputlisting[language=sail]{sail_latex_riscv/valzhex_bits_2_backwards_matches823cb41992ed3a26886d992e75e86734.tex}}}}

\newcommand{\sailRISCVfnhexBitsTwoBackwardsMatches}{\saildoclabelled{sailRISCVfnzhexzybitszy2zybackwardszymatches}{\saildocfn{}{\lstinputlisting[language=sail]{sail_latex_riscv/fnzhex_bits_2_backwards_matches823cb41992ed3a26886d992e75e86734.tex}}}}

\newcommand{\sailRISCVvalhexBitsTwoBackwards}{\saildoclabelled{sailRISCVzhexzybitszy2zybackwards}{\saildocval{}{\lstinputlisting[language=sail]{sail_latex_riscv/valzhex_bits_2_backwards18d79d6a578203ebce0993a1703a4245.tex}}}}

\newcommand{\sailRISCVfnhexBitsTwoBackwards}{\saildoclabelled{sailRISCVfnzhexzybitszy2zybackwards}{\saildocfn{}{\lstinputlisting[language=sail]{sail_latex_riscv/fnzhex_bits_2_backwards18d79d6a578203ebce0993a1703a4245.tex}}}}

\newcommand{\sailRISCVvalhexBitsThree}{\saildoclabelled{sailRISCVzhexzybitszy3}{\saildocval{}{\lstinputlisting[language=sail]{sail_latex_riscv/valzhex_bits_3aeb691402af5232353b21f4b29231f77.tex}}}}

\newcommand{\sailRISCVvalhexBitsThreeForwards}{\saildoclabelled{sailRISCVzhexzybitszy3zyforwards}{\saildocval{}{\lstinputlisting[language=sail]{sail_latex_riscv/valzhex_bits_3_forwards7bd80f2805155aa4c3a7525bb0bfd448.tex}}}}

\newcommand{\sailRISCVvalhexBitsThreeForwardsMatches}{\saildoclabelled{sailRISCVzhexzybitszy3zyforwardszymatches}{\saildocval{}{\lstinputlisting[language=sail]{sail_latex_riscv/valzhex_bits_3_forwards_matchese79d3588a48711c3aab539462e1e83ad.tex}}}}

\newcommand{\sailRISCVfnhexBitsThreeForwardsMatches}{\saildoclabelled{sailRISCVfnzhexzybitszy3zyforwardszymatches}{\saildocfn{}{\lstinputlisting[language=sail]{sail_latex_riscv/fnzhex_bits_3_forwards_matchese79d3588a48711c3aab539462e1e83ad.tex}}}}

\newcommand{\sailRISCVvalhexBitsThreeMatchesPrefix}{\saildoclabelled{sailRISCVzhexzybitszy3zymatcheszyprefix}{\saildocval{}{\lstinputlisting[language=sail]{sail_latex_riscv/valzhex_bits_3_matches_prefix0b670e4346265cc88b71be93bef14dae.tex}}}}

\newcommand{\sailRISCVvalhexBitsThreeBackwardsMatches}{\saildoclabelled{sailRISCVzhexzybitszy3zybackwardszymatches}{\saildocval{}{\lstinputlisting[language=sail]{sail_latex_riscv/valzhex_bits_3_backwards_matches952f17dffac18d98619e00bc97a700fc.tex}}}}

\newcommand{\sailRISCVfnhexBitsThreeBackwardsMatches}{\saildoclabelled{sailRISCVfnzhexzybitszy3zybackwardszymatches}{\saildocfn{}{\lstinputlisting[language=sail]{sail_latex_riscv/fnzhex_bits_3_backwards_matches952f17dffac18d98619e00bc97a700fc.tex}}}}

\newcommand{\sailRISCVvalhexBitsThreeBackwards}{\saildoclabelled{sailRISCVzhexzybitszy3zybackwards}{\saildocval{}{\lstinputlisting[language=sail]{sail_latex_riscv/valzhex_bits_3_backwardsf853f7ad9bfed642772b10dafd6910d1.tex}}}}

\newcommand{\sailRISCVfnhexBitsThreeBackwards}{\saildoclabelled{sailRISCVfnzhexzybitszy3zybackwards}{\saildocfn{}{\lstinputlisting[language=sail]{sail_latex_riscv/fnzhex_bits_3_backwardsf853f7ad9bfed642772b10dafd6910d1.tex}}}}

\newcommand{\sailRISCVvalhexBitsFour}{\saildoclabelled{sailRISCVzhexzybitszy4}{\saildocval{}{\lstinputlisting[language=sail]{sail_latex_riscv/valzhex_bits_4cc2b0d5d62eeaf81d463b3beb7b16ede.tex}}}}

\newcommand{\sailRISCVvalhexBitsFourForwards}{\saildoclabelled{sailRISCVzhexzybitszy4zyforwards}{\saildocval{}{\lstinputlisting[language=sail]{sail_latex_riscv/valzhex_bits_4_forwards17651a29b8d01053a94a764e92e93964.tex}}}}

\newcommand{\sailRISCVvalhexBitsFourForwardsMatches}{\saildoclabelled{sailRISCVzhexzybitszy4zyforwardszymatches}{\saildocval{}{\lstinputlisting[language=sail]{sail_latex_riscv/valzhex_bits_4_forwards_matchesc9b2af172c13a489fa870acdd673758a.tex}}}}

\newcommand{\sailRISCVfnhexBitsFourForwardsMatches}{\saildoclabelled{sailRISCVfnzhexzybitszy4zyforwardszymatches}{\saildocfn{}{\lstinputlisting[language=sail]{sail_latex_riscv/fnzhex_bits_4_forwards_matchesc9b2af172c13a489fa870acdd673758a.tex}}}}

\newcommand{\sailRISCVvalhexBitsFourMatchesPrefix}{\saildoclabelled{sailRISCVzhexzybitszy4zymatcheszyprefix}{\saildocval{}{\lstinputlisting[language=sail]{sail_latex_riscv/valzhex_bits_4_matches_prefix7449c0ebcb161f55de6260595c0db150.tex}}}}

\newcommand{\sailRISCVvalhexBitsFourBackwardsMatches}{\saildoclabelled{sailRISCVzhexzybitszy4zybackwardszymatches}{\saildocval{}{\lstinputlisting[language=sail]{sail_latex_riscv/valzhex_bits_4_backwards_matches798cb282812c7bc9b083a091bb0b23ca.tex}}}}

\newcommand{\sailRISCVfnhexBitsFourBackwardsMatches}{\saildoclabelled{sailRISCVfnzhexzybitszy4zybackwardszymatches}{\saildocfn{}{\lstinputlisting[language=sail]{sail_latex_riscv/fnzhex_bits_4_backwards_matches798cb282812c7bc9b083a091bb0b23ca.tex}}}}

\newcommand{\sailRISCVvalhexBitsFourBackwards}{\saildoclabelled{sailRISCVzhexzybitszy4zybackwards}{\saildocval{}{\lstinputlisting[language=sail]{sail_latex_riscv/valzhex_bits_4_backwardsf22b435ac96c309eec82519c5c010323.tex}}}}

\newcommand{\sailRISCVfnhexBitsFourBackwards}{\saildoclabelled{sailRISCVfnzhexzybitszy4zybackwards}{\saildocfn{}{\lstinputlisting[language=sail]{sail_latex_riscv/fnzhex_bits_4_backwardsf22b435ac96c309eec82519c5c010323.tex}}}}

\newcommand{\sailRISCVvalhexBitsFive}{\saildoclabelled{sailRISCVzhexzybitszy5}{\saildocval{}{\lstinputlisting[language=sail]{sail_latex_riscv/valzhex_bits_5daa8fabe8bf4a20ba57e93cf91addd75.tex}}}}

\newcommand{\sailRISCVvalhexBitsFiveForwards}{\saildoclabelled{sailRISCVzhexzybitszy5zyforwards}{\saildocval{}{\lstinputlisting[language=sail]{sail_latex_riscv/valzhex_bits_5_forwards79911e0485d859b00a2fdcba8cff5318.tex}}}}

\newcommand{\sailRISCVvalhexBitsFiveForwardsMatches}{\saildoclabelled{sailRISCVzhexzybitszy5zyforwardszymatches}{\saildocval{}{\lstinputlisting[language=sail]{sail_latex_riscv/valzhex_bits_5_forwards_matches3e31f036e77b0ee1a3e8312b0067959b.tex}}}}

\newcommand{\sailRISCVfnhexBitsFiveForwardsMatches}{\saildoclabelled{sailRISCVfnzhexzybitszy5zyforwardszymatches}{\saildocfn{}{\lstinputlisting[language=sail]{sail_latex_riscv/fnzhex_bits_5_forwards_matches3e31f036e77b0ee1a3e8312b0067959b.tex}}}}

\newcommand{\sailRISCVvalhexBitsFiveMatchesPrefix}{\saildoclabelled{sailRISCVzhexzybitszy5zymatcheszyprefix}{\saildocval{}{\lstinputlisting[language=sail]{sail_latex_riscv/valzhex_bits_5_matches_prefix998186306b8236e178fc5d664bda45c0.tex}}}}

\newcommand{\sailRISCVvalhexBitsFiveBackwardsMatches}{\saildoclabelled{sailRISCVzhexzybitszy5zybackwardszymatches}{\saildocval{}{\lstinputlisting[language=sail]{sail_latex_riscv/valzhex_bits_5_backwards_matchesf675096dc18dd61104842d90f51800a8.tex}}}}

\newcommand{\sailRISCVfnhexBitsFiveBackwardsMatches}{\saildoclabelled{sailRISCVfnzhexzybitszy5zybackwardszymatches}{\saildocfn{}{\lstinputlisting[language=sail]{sail_latex_riscv/fnzhex_bits_5_backwards_matchesf675096dc18dd61104842d90f51800a8.tex}}}}

\newcommand{\sailRISCVvalhexBitsFiveBackwards}{\saildoclabelled{sailRISCVzhexzybitszy5zybackwards}{\saildocval{}{\lstinputlisting[language=sail]{sail_latex_riscv/valzhex_bits_5_backwards79c47a53cbcc5ca893c439296fe2bd3f.tex}}}}

\newcommand{\sailRISCVfnhexBitsFiveBackwards}{\saildoclabelled{sailRISCVfnzhexzybitszy5zybackwards}{\saildocfn{}{\lstinputlisting[language=sail]{sail_latex_riscv/fnzhex_bits_5_backwards79c47a53cbcc5ca893c439296fe2bd3f.tex}}}}

\newcommand{\sailRISCVvalhexBitsSix}{\saildoclabelled{sailRISCVzhexzybitszy6}{\saildocval{}{\lstinputlisting[language=sail]{sail_latex_riscv/valzhex_bits_61cff7c3b87dd820982f77ae4876bbe56.tex}}}}

\newcommand{\sailRISCVvalhexBitsSixForwards}{\saildoclabelled{sailRISCVzhexzybitszy6zyforwards}{\saildocval{}{\lstinputlisting[language=sail]{sail_latex_riscv/valzhex_bits_6_forwards3455f13508c617afbe64b39dede36554.tex}}}}

\newcommand{\sailRISCVvalhexBitsSixForwardsMatches}{\saildoclabelled{sailRISCVzhexzybitszy6zyforwardszymatches}{\saildocval{}{\lstinputlisting[language=sail]{sail_latex_riscv/valzhex_bits_6_forwards_matchescaeea7989719fec9093c81797a7f4155.tex}}}}

\newcommand{\sailRISCVfnhexBitsSixForwardsMatches}{\saildoclabelled{sailRISCVfnzhexzybitszy6zyforwardszymatches}{\saildocfn{}{\lstinputlisting[language=sail]{sail_latex_riscv/fnzhex_bits_6_forwards_matchescaeea7989719fec9093c81797a7f4155.tex}}}}

\newcommand{\sailRISCVvalhexBitsSixMatchesPrefix}{\saildoclabelled{sailRISCVzhexzybitszy6zymatcheszyprefix}{\saildocval{}{\lstinputlisting[language=sail]{sail_latex_riscv/valzhex_bits_6_matches_prefixb8878485899ba2293890312e921ac156.tex}}}}

\newcommand{\sailRISCVvalhexBitsSixBackwardsMatches}{\saildoclabelled{sailRISCVzhexzybitszy6zybackwardszymatches}{\saildocval{}{\lstinputlisting[language=sail]{sail_latex_riscv/valzhex_bits_6_backwards_matches4f038823a669140ed233992d9e5ddb69.tex}}}}

\newcommand{\sailRISCVfnhexBitsSixBackwardsMatches}{\saildoclabelled{sailRISCVfnzhexzybitszy6zybackwardszymatches}{\saildocfn{}{\lstinputlisting[language=sail]{sail_latex_riscv/fnzhex_bits_6_backwards_matches4f038823a669140ed233992d9e5ddb69.tex}}}}

\newcommand{\sailRISCVvalhexBitsSixBackwards}{\saildoclabelled{sailRISCVzhexzybitszy6zybackwards}{\saildocval{}{\lstinputlisting[language=sail]{sail_latex_riscv/valzhex_bits_6_backwardsabaf13af56590b164adcda6b4fd1b52f.tex}}}}

\newcommand{\sailRISCVfnhexBitsSixBackwards}{\saildoclabelled{sailRISCVfnzhexzybitszy6zybackwards}{\saildocfn{}{\lstinputlisting[language=sail]{sail_latex_riscv/fnzhex_bits_6_backwardsabaf13af56590b164adcda6b4fd1b52f.tex}}}}

\newcommand{\sailRISCVvalhexBitsSeven}{\saildoclabelled{sailRISCVzhexzybitszy7}{\saildocval{}{\lstinputlisting[language=sail]{sail_latex_riscv/valzhex_bits_7b8d1ab69e279cfbd7a9d7c59b1171b24.tex}}}}

\newcommand{\sailRISCVvalhexBitsSevenForwards}{\saildoclabelled{sailRISCVzhexzybitszy7zyforwards}{\saildocval{}{\lstinputlisting[language=sail]{sail_latex_riscv/valzhex_bits_7_forwards9bd105e290f5b2e7d8bb2e10f49c067b.tex}}}}

\newcommand{\sailRISCVvalhexBitsSevenForwardsMatches}{\saildoclabelled{sailRISCVzhexzybitszy7zyforwardszymatches}{\saildocval{}{\lstinputlisting[language=sail]{sail_latex_riscv/valzhex_bits_7_forwards_matches9d7aa1993e9d141c0950fe9e67734abc.tex}}}}

\newcommand{\sailRISCVfnhexBitsSevenForwardsMatches}{\saildoclabelled{sailRISCVfnzhexzybitszy7zyforwardszymatches}{\saildocfn{}{\lstinputlisting[language=sail]{sail_latex_riscv/fnzhex_bits_7_forwards_matches9d7aa1993e9d141c0950fe9e67734abc.tex}}}}

\newcommand{\sailRISCVvalhexBitsSevenMatchesPrefix}{\saildoclabelled{sailRISCVzhexzybitszy7zymatcheszyprefix}{\saildocval{}{\lstinputlisting[language=sail]{sail_latex_riscv/valzhex_bits_7_matches_prefixfe0ccd9dca07ce0b4f7d3176ad3fc10c.tex}}}}

\newcommand{\sailRISCVvalhexBitsSevenBackwardsMatches}{\saildoclabelled{sailRISCVzhexzybitszy7zybackwardszymatches}{\saildocval{}{\lstinputlisting[language=sail]{sail_latex_riscv/valzhex_bits_7_backwards_matchescc985ebeb5f5adedc7e184499a8978db.tex}}}}

\newcommand{\sailRISCVfnhexBitsSevenBackwardsMatches}{\saildoclabelled{sailRISCVfnzhexzybitszy7zybackwardszymatches}{\saildocfn{}{\lstinputlisting[language=sail]{sail_latex_riscv/fnzhex_bits_7_backwards_matchescc985ebeb5f5adedc7e184499a8978db.tex}}}}

\newcommand{\sailRISCVvalhexBitsSevenBackwards}{\saildoclabelled{sailRISCVzhexzybitszy7zybackwards}{\saildocval{}{\lstinputlisting[language=sail]{sail_latex_riscv/valzhex_bits_7_backwards33c217f9145eec45b693f09bca0d89ea.tex}}}}

\newcommand{\sailRISCVfnhexBitsSevenBackwards}{\saildoclabelled{sailRISCVfnzhexzybitszy7zybackwards}{\saildocfn{}{\lstinputlisting[language=sail]{sail_latex_riscv/fnzhex_bits_7_backwards33c217f9145eec45b693f09bca0d89ea.tex}}}}

\newcommand{\sailRISCVvalhexBitsEight}{\saildoclabelled{sailRISCVzhexzybitszy8}{\saildocval{}{\lstinputlisting[language=sail]{sail_latex_riscv/valzhex_bits_8e02cd45b50bbc73ca04e08e3df697f5d.tex}}}}

\newcommand{\sailRISCVvalhexBitsEightForwards}{\saildoclabelled{sailRISCVzhexzybitszy8zyforwards}{\saildocval{}{\lstinputlisting[language=sail]{sail_latex_riscv/valzhex_bits_8_forwards168d310ff118fd3c7d7bc80f8de29d67.tex}}}}

\newcommand{\sailRISCVvalhexBitsEightForwardsMatches}{\saildoclabelled{sailRISCVzhexzybitszy8zyforwardszymatches}{\saildocval{}{\lstinputlisting[language=sail]{sail_latex_riscv/valzhex_bits_8_forwards_matches0746becf5ca996196bfae6048bab9bb1.tex}}}}

\newcommand{\sailRISCVfnhexBitsEightForwardsMatches}{\saildoclabelled{sailRISCVfnzhexzybitszy8zyforwardszymatches}{\saildocfn{}{\lstinputlisting[language=sail]{sail_latex_riscv/fnzhex_bits_8_forwards_matches0746becf5ca996196bfae6048bab9bb1.tex}}}}

\newcommand{\sailRISCVvalhexBitsEightMatchesPrefix}{\saildoclabelled{sailRISCVzhexzybitszy8zymatcheszyprefix}{\saildocval{}{\lstinputlisting[language=sail]{sail_latex_riscv/valzhex_bits_8_matches_prefixf713d7df51d74309ab21b6aa029dbeed.tex}}}}

\newcommand{\sailRISCVvalhexBitsEightBackwardsMatches}{\saildoclabelled{sailRISCVzhexzybitszy8zybackwardszymatches}{\saildocval{}{\lstinputlisting[language=sail]{sail_latex_riscv/valzhex_bits_8_backwards_matchesa183109b599ae5e880adc653bde047ba.tex}}}}

\newcommand{\sailRISCVfnhexBitsEightBackwardsMatches}{\saildoclabelled{sailRISCVfnzhexzybitszy8zybackwardszymatches}{\saildocfn{}{\lstinputlisting[language=sail]{sail_latex_riscv/fnzhex_bits_8_backwards_matchesa183109b599ae5e880adc653bde047ba.tex}}}}

\newcommand{\sailRISCVvalhexBitsEightBackwards}{\saildoclabelled{sailRISCVzhexzybitszy8zybackwards}{\saildocval{}{\lstinputlisting[language=sail]{sail_latex_riscv/valzhex_bits_8_backwards8502ba5c5fdfe5404c185ff925d26a02.tex}}}}

\newcommand{\sailRISCVfnhexBitsEightBackwards}{\saildoclabelled{sailRISCVfnzhexzybitszy8zybackwards}{\saildocfn{}{\lstinputlisting[language=sail]{sail_latex_riscv/fnzhex_bits_8_backwards8502ba5c5fdfe5404c185ff925d26a02.tex}}}}

\newcommand{\sailRISCVvalhexBitsNine}{\saildoclabelled{sailRISCVzhexzybitszy9}{\saildocval{}{\lstinputlisting[language=sail]{sail_latex_riscv/valzhex_bits_9eeceb6bd9aab72270435e4d9e8c5dc4c.tex}}}}

\newcommand{\sailRISCVvalhexBitsNineForwards}{\saildoclabelled{sailRISCVzhexzybitszy9zyforwards}{\saildocval{}{\lstinputlisting[language=sail]{sail_latex_riscv/valzhex_bits_9_forwards5bc4abb19984f8188c037a8554a3ce15.tex}}}}

\newcommand{\sailRISCVvalhexBitsNineForwardsMatches}{\saildoclabelled{sailRISCVzhexzybitszy9zyforwardszymatches}{\saildocval{}{\lstinputlisting[language=sail]{sail_latex_riscv/valzhex_bits_9_forwards_matches73ffbf54790f1b1c63b0dc63773cd0c0.tex}}}}

\newcommand{\sailRISCVfnhexBitsNineForwardsMatches}{\saildoclabelled{sailRISCVfnzhexzybitszy9zyforwardszymatches}{\saildocfn{}{\lstinputlisting[language=sail]{sail_latex_riscv/fnzhex_bits_9_forwards_matches73ffbf54790f1b1c63b0dc63773cd0c0.tex}}}}

\newcommand{\sailRISCVvalhexBitsNineMatchesPrefix}{\saildoclabelled{sailRISCVzhexzybitszy9zymatcheszyprefix}{\saildocval{}{\lstinputlisting[language=sail]{sail_latex_riscv/valzhex_bits_9_matches_prefix693d173094578fcf50e2fb151d0c156e.tex}}}}

\newcommand{\sailRISCVvalhexBitsNineBackwardsMatches}{\saildoclabelled{sailRISCVzhexzybitszy9zybackwardszymatches}{\saildocval{}{\lstinputlisting[language=sail]{sail_latex_riscv/valzhex_bits_9_backwards_matches34e146fa5f101fa17110f57d272dec69.tex}}}}

\newcommand{\sailRISCVfnhexBitsNineBackwardsMatches}{\saildoclabelled{sailRISCVfnzhexzybitszy9zybackwardszymatches}{\saildocfn{}{\lstinputlisting[language=sail]{sail_latex_riscv/fnzhex_bits_9_backwards_matches34e146fa5f101fa17110f57d272dec69.tex}}}}

\newcommand{\sailRISCVvalhexBitsNineBackwards}{\saildoclabelled{sailRISCVzhexzybitszy9zybackwards}{\saildocval{}{\lstinputlisting[language=sail]{sail_latex_riscv/valzhex_bits_9_backwards90cddf73654805833800c2414c31ab0f.tex}}}}

\newcommand{\sailRISCVfnhexBitsNineBackwards}{\saildoclabelled{sailRISCVfnzhexzybitszy9zybackwards}{\saildocfn{}{\lstinputlisting[language=sail]{sail_latex_riscv/fnzhex_bits_9_backwards90cddf73654805833800c2414c31ab0f.tex}}}}

\newcommand{\sailRISCVvalhexBitsOneZero}{\saildoclabelled{sailRISCVzhexzybitszy10}{\saildocval{}{\lstinputlisting[language=sail]{sail_latex_riscv/valzhex_bits_1001007fc28a3cc2512c7863ca8f700d8b.tex}}}}

\newcommand{\sailRISCVvalhexBitsOneZeroForwards}{\saildoclabelled{sailRISCVzhexzybitszy10zyforwards}{\saildocval{}{\lstinputlisting[language=sail]{sail_latex_riscv/valzhex_bits_10_forwardseca444da7d319d1932fce8e708381d1e.tex}}}}

\newcommand{\sailRISCVvalhexBitsOneZeroForwardsMatches}{\saildoclabelled{sailRISCVzhexzybitszy10zyforwardszymatches}{\saildocval{}{\lstinputlisting[language=sail]{sail_latex_riscv/valzhex_bits_10_forwards_matches5805c03590a8c87ee15c304ac5e14038.tex}}}}

\newcommand{\sailRISCVfnhexBitsOneZeroForwardsMatches}{\saildoclabelled{sailRISCVfnzhexzybitszy10zyforwardszymatches}{\saildocfn{}{\lstinputlisting[language=sail]{sail_latex_riscv/fnzhex_bits_10_forwards_matches5805c03590a8c87ee15c304ac5e14038.tex}}}}

\newcommand{\sailRISCVvalhexBitsOneZeroMatchesPrefix}{\saildoclabelled{sailRISCVzhexzybitszy10zymatcheszyprefix}{\saildocval{}{\lstinputlisting[language=sail]{sail_latex_riscv/valzhex_bits_10_matches_prefix1bc10569a97d5b943c4d0e1bee005436.tex}}}}

\newcommand{\sailRISCVvalhexBitsOneZeroBackwardsMatches}{\saildoclabelled{sailRISCVzhexzybitszy10zybackwardszymatches}{\saildocval{}{\lstinputlisting[language=sail]{sail_latex_riscv/valzhex_bits_10_backwards_matches40e05acb0a138e49841d114ad4d36956.tex}}}}

\newcommand{\sailRISCVfnhexBitsOneZeroBackwardsMatches}{\saildoclabelled{sailRISCVfnzhexzybitszy10zybackwardszymatches}{\saildocfn{}{\lstinputlisting[language=sail]{sail_latex_riscv/fnzhex_bits_10_backwards_matches40e05acb0a138e49841d114ad4d36956.tex}}}}

\newcommand{\sailRISCVvalhexBitsOneZeroBackwards}{\saildoclabelled{sailRISCVzhexzybitszy10zybackwards}{\saildocval{}{\lstinputlisting[language=sail]{sail_latex_riscv/valzhex_bits_10_backwards3ea67a7e3b03c19fc3f62bce5d70adcf.tex}}}}

\newcommand{\sailRISCVfnhexBitsOneZeroBackwards}{\saildoclabelled{sailRISCVfnzhexzybitszy10zybackwards}{\saildocfn{}{\lstinputlisting[language=sail]{sail_latex_riscv/fnzhex_bits_10_backwards3ea67a7e3b03c19fc3f62bce5d70adcf.tex}}}}

\newcommand{\sailRISCVvalhexBitsOneOne}{\saildoclabelled{sailRISCVzhexzybitszy11}{\saildocval{}{\lstinputlisting[language=sail]{sail_latex_riscv/valzhex_bits_1196b03eaf6acc67335c0bb71c802e6a22.tex}}}}

\newcommand{\sailRISCVvalhexBitsOneOneForwards}{\saildoclabelled{sailRISCVzhexzybitszy11zyforwards}{\saildocval{}{\lstinputlisting[language=sail]{sail_latex_riscv/valzhex_bits_11_forwardsa103ffbf0ab1e78130e8eca0e504aba5.tex}}}}

\newcommand{\sailRISCVvalhexBitsOneOneForwardsMatches}{\saildoclabelled{sailRISCVzhexzybitszy11zyforwardszymatches}{\saildocval{}{\lstinputlisting[language=sail]{sail_latex_riscv/valzhex_bits_11_forwards_matchesc72e36f9e9259ad52cfd038ae9c0251f.tex}}}}

\newcommand{\sailRISCVfnhexBitsOneOneForwardsMatches}{\saildoclabelled{sailRISCVfnzhexzybitszy11zyforwardszymatches}{\saildocfn{}{\lstinputlisting[language=sail]{sail_latex_riscv/fnzhex_bits_11_forwards_matchesc72e36f9e9259ad52cfd038ae9c0251f.tex}}}}

\newcommand{\sailRISCVvalhexBitsOneOneMatchesPrefix}{\saildoclabelled{sailRISCVzhexzybitszy11zymatcheszyprefix}{\saildocval{}{\lstinputlisting[language=sail]{sail_latex_riscv/valzhex_bits_11_matches_prefix5542021dd19d9d49c8135e0408dda5bf.tex}}}}

\newcommand{\sailRISCVvalhexBitsOneOneBackwardsMatches}{\saildoclabelled{sailRISCVzhexzybitszy11zybackwardszymatches}{\saildocval{}{\lstinputlisting[language=sail]{sail_latex_riscv/valzhex_bits_11_backwards_matchesd1ccc1a72470b9d5d21bdc4ac13ff26b.tex}}}}

\newcommand{\sailRISCVfnhexBitsOneOneBackwardsMatches}{\saildoclabelled{sailRISCVfnzhexzybitszy11zybackwardszymatches}{\saildocfn{}{\lstinputlisting[language=sail]{sail_latex_riscv/fnzhex_bits_11_backwards_matchesd1ccc1a72470b9d5d21bdc4ac13ff26b.tex}}}}

\newcommand{\sailRISCVvalhexBitsOneOneBackwards}{\saildoclabelled{sailRISCVzhexzybitszy11zybackwards}{\saildocval{}{\lstinputlisting[language=sail]{sail_latex_riscv/valzhex_bits_11_backwards2baa8535235545b1d36570cd9bfa54d3.tex}}}}

\newcommand{\sailRISCVfnhexBitsOneOneBackwards}{\saildoclabelled{sailRISCVfnzhexzybitszy11zybackwards}{\saildocfn{}{\lstinputlisting[language=sail]{sail_latex_riscv/fnzhex_bits_11_backwards2baa8535235545b1d36570cd9bfa54d3.tex}}}}

\newcommand{\sailRISCVvalhexBitsOneTwo}{\saildoclabelled{sailRISCVzhexzybitszy12}{\saildocval{}{\lstinputlisting[language=sail]{sail_latex_riscv/valzhex_bits_125a4772f29c6286c50b66c991563e61c3.tex}}}}

\newcommand{\sailRISCVvalhexBitsOneTwoForwards}{\saildoclabelled{sailRISCVzhexzybitszy12zyforwards}{\saildocval{}{\lstinputlisting[language=sail]{sail_latex_riscv/valzhex_bits_12_forwardsfead99eab466bc9417dd261cbed56176.tex}}}}

\newcommand{\sailRISCVvalhexBitsOneTwoForwardsMatches}{\saildoclabelled{sailRISCVzhexzybitszy12zyforwardszymatches}{\saildocval{}{\lstinputlisting[language=sail]{sail_latex_riscv/valzhex_bits_12_forwards_matches23afc09f193a32dea040a72b95ec91b2.tex}}}}

\newcommand{\sailRISCVfnhexBitsOneTwoForwardsMatches}{\saildoclabelled{sailRISCVfnzhexzybitszy12zyforwardszymatches}{\saildocfn{}{\lstinputlisting[language=sail]{sail_latex_riscv/fnzhex_bits_12_forwards_matches23afc09f193a32dea040a72b95ec91b2.tex}}}}

\newcommand{\sailRISCVfnhexBitsOneTwoMatchesPrefix}{\saildoclabelled{sailRISCVfnzhexzybitszy12zymatcheszyprefix}{\saildocfn{}{\lstinputlisting[language=sail]{sail_latex_riscv/fnzhex_bits_12_matches_prefixc31c0737edfea85e61e2e4a0e4afa0dc.tex}}}}

\newcommand{\sailRISCVvalhexBitsOneTwoBackwardsMatches}{\saildoclabelled{sailRISCVzhexzybitszy12zybackwardszymatches}{\saildocval{}{\lstinputlisting[language=sail]{sail_latex_riscv/valzhex_bits_12_backwards_matches6a7e459d92157f0ba9dee2e4c7899300.tex}}}}

\newcommand{\sailRISCVfnhexBitsOneTwoBackwardsMatches}{\saildoclabelled{sailRISCVfnzhexzybitszy12zybackwardszymatches}{\saildocfn{}{\lstinputlisting[language=sail]{sail_latex_riscv/fnzhex_bits_12_backwards_matches6a7e459d92157f0ba9dee2e4c7899300.tex}}}}

\newcommand{\sailRISCVvalhexBitsOneTwoBackwards}{\saildoclabelled{sailRISCVzhexzybitszy12zybackwards}{\saildocval{}{\lstinputlisting[language=sail]{sail_latex_riscv/valzhex_bits_12_backwards3d80f810c3a0bfaf0d29151a18d72567.tex}}}}

\newcommand{\sailRISCVfnhexBitsOneTwoBackwards}{\saildoclabelled{sailRISCVfnzhexzybitszy12zybackwards}{\saildocfn{}{\lstinputlisting[language=sail]{sail_latex_riscv/fnzhex_bits_12_backwards3d80f810c3a0bfaf0d29151a18d72567.tex}}}}

\newcommand{\sailRISCVvalhexBitsOneThree}{\saildoclabelled{sailRISCVzhexzybitszy13}{\saildocval{}{\lstinputlisting[language=sail]{sail_latex_riscv/valzhex_bits_13f2e19b07147e1bc75b3a752515303583.tex}}}}

\newcommand{\sailRISCVvalhexBitsOneThreeForwards}{\saildoclabelled{sailRISCVzhexzybitszy13zyforwards}{\saildocval{}{\lstinputlisting[language=sail]{sail_latex_riscv/valzhex_bits_13_forwards4a6f3d80543dcfc7821c2bf16e534c80.tex}}}}

\newcommand{\sailRISCVvalhexBitsOneThreeForwardsMatches}{\saildoclabelled{sailRISCVzhexzybitszy13zyforwardszymatches}{\saildocval{}{\lstinputlisting[language=sail]{sail_latex_riscv/valzhex_bits_13_forwards_matches24cf47063279787f52ede006100876ff.tex}}}}

\newcommand{\sailRISCVfnhexBitsOneThreeForwardsMatches}{\saildoclabelled{sailRISCVfnzhexzybitszy13zyforwardszymatches}{\saildocfn{}{\lstinputlisting[language=sail]{sail_latex_riscv/fnzhex_bits_13_forwards_matches24cf47063279787f52ede006100876ff.tex}}}}

\newcommand{\sailRISCVvalhexBitsOneThreeMatchesPrefix}{\saildoclabelled{sailRISCVzhexzybitszy13zymatcheszyprefix}{\saildocval{}{\lstinputlisting[language=sail]{sail_latex_riscv/valzhex_bits_13_matches_prefix808dd1e0db4d0e2cbb2964dee566f987.tex}}}}

\newcommand{\sailRISCVvalhexBitsOneThreeBackwardsMatches}{\saildoclabelled{sailRISCVzhexzybitszy13zybackwardszymatches}{\saildocval{}{\lstinputlisting[language=sail]{sail_latex_riscv/valzhex_bits_13_backwards_matches3380960c8e44f21f0eeafd0244600ee6.tex}}}}

\newcommand{\sailRISCVfnhexBitsOneThreeBackwardsMatches}{\saildoclabelled{sailRISCVfnzhexzybitszy13zybackwardszymatches}{\saildocfn{}{\lstinputlisting[language=sail]{sail_latex_riscv/fnzhex_bits_13_backwards_matches3380960c8e44f21f0eeafd0244600ee6.tex}}}}

\newcommand{\sailRISCVvalhexBitsOneThreeBackwards}{\saildoclabelled{sailRISCVzhexzybitszy13zybackwards}{\saildocval{}{\lstinputlisting[language=sail]{sail_latex_riscv/valzhex_bits_13_backwards75db4a8adaa37e62b3ea51d810cbbf3b.tex}}}}

\newcommand{\sailRISCVfnhexBitsOneThreeBackwards}{\saildoclabelled{sailRISCVfnzhexzybitszy13zybackwards}{\saildocfn{}{\lstinputlisting[language=sail]{sail_latex_riscv/fnzhex_bits_13_backwards75db4a8adaa37e62b3ea51d810cbbf3b.tex}}}}

\newcommand{\sailRISCVvalhexBitsOneFour}{\saildoclabelled{sailRISCVzhexzybitszy14}{\saildocval{}{\lstinputlisting[language=sail]{sail_latex_riscv/valzhex_bits_147ca431b3e814f7be996cd1b6ec3c0902.tex}}}}

\newcommand{\sailRISCVvalhexBitsOneFourForwards}{\saildoclabelled{sailRISCVzhexzybitszy14zyforwards}{\saildocval{}{\lstinputlisting[language=sail]{sail_latex_riscv/valzhex_bits_14_forwards8f53488826326c38c678631a29ef042b.tex}}}}

\newcommand{\sailRISCVvalhexBitsOneFourForwardsMatches}{\saildoclabelled{sailRISCVzhexzybitszy14zyforwardszymatches}{\saildocval{}{\lstinputlisting[language=sail]{sail_latex_riscv/valzhex_bits_14_forwards_matches229689dc85657ccebc6a0143fb21144e.tex}}}}

\newcommand{\sailRISCVfnhexBitsOneFourForwardsMatches}{\saildoclabelled{sailRISCVfnzhexzybitszy14zyforwardszymatches}{\saildocfn{}{\lstinputlisting[language=sail]{sail_latex_riscv/fnzhex_bits_14_forwards_matches229689dc85657ccebc6a0143fb21144e.tex}}}}

\newcommand{\sailRISCVvalhexBitsOneFourMatchesPrefix}{\saildoclabelled{sailRISCVzhexzybitszy14zymatcheszyprefix}{\saildocval{}{\lstinputlisting[language=sail]{sail_latex_riscv/valzhex_bits_14_matches_prefix43b0d69de97191b01d4d4ad6d99d2952.tex}}}}

\newcommand{\sailRISCVvalhexBitsOneFourBackwardsMatches}{\saildoclabelled{sailRISCVzhexzybitszy14zybackwardszymatches}{\saildocval{}{\lstinputlisting[language=sail]{sail_latex_riscv/valzhex_bits_14_backwards_matches7a9acb89282cc5c0ad18d89d22bf3cfa.tex}}}}

\newcommand{\sailRISCVfnhexBitsOneFourBackwardsMatches}{\saildoclabelled{sailRISCVfnzhexzybitszy14zybackwardszymatches}{\saildocfn{}{\lstinputlisting[language=sail]{sail_latex_riscv/fnzhex_bits_14_backwards_matches7a9acb89282cc5c0ad18d89d22bf3cfa.tex}}}}

\newcommand{\sailRISCVvalhexBitsOneFourBackwards}{\saildoclabelled{sailRISCVzhexzybitszy14zybackwards}{\saildocval{}{\lstinputlisting[language=sail]{sail_latex_riscv/valzhex_bits_14_backwards825f2ad69791fa9a1bb2792ac85efc54.tex}}}}

\newcommand{\sailRISCVfnhexBitsOneFourBackwards}{\saildoclabelled{sailRISCVfnzhexzybitszy14zybackwards}{\saildocfn{}{\lstinputlisting[language=sail]{sail_latex_riscv/fnzhex_bits_14_backwards825f2ad69791fa9a1bb2792ac85efc54.tex}}}}

\newcommand{\sailRISCVvalhexBitsOneFive}{\saildoclabelled{sailRISCVzhexzybitszy15}{\saildocval{}{\lstinputlisting[language=sail]{sail_latex_riscv/valzhex_bits_15073380baa5f552c999850321b1ecde51.tex}}}}

\newcommand{\sailRISCVvalhexBitsOneFiveForwards}{\saildoclabelled{sailRISCVzhexzybitszy15zyforwards}{\saildocval{}{\lstinputlisting[language=sail]{sail_latex_riscv/valzhex_bits_15_forwards6c990ff9d54de24c898e16e2e488b179.tex}}}}

\newcommand{\sailRISCVvalhexBitsOneFiveForwardsMatches}{\saildoclabelled{sailRISCVzhexzybitszy15zyforwardszymatches}{\saildocval{}{\lstinputlisting[language=sail]{sail_latex_riscv/valzhex_bits_15_forwards_matches923e93be7d0b466ae08582f59a3c9529.tex}}}}

\newcommand{\sailRISCVfnhexBitsOneFiveForwardsMatches}{\saildoclabelled{sailRISCVfnzhexzybitszy15zyforwardszymatches}{\saildocfn{}{\lstinputlisting[language=sail]{sail_latex_riscv/fnzhex_bits_15_forwards_matches923e93be7d0b466ae08582f59a3c9529.tex}}}}

\newcommand{\sailRISCVvalhexBitsOneFiveMatchesPrefix}{\saildoclabelled{sailRISCVzhexzybitszy15zymatcheszyprefix}{\saildocval{}{\lstinputlisting[language=sail]{sail_latex_riscv/valzhex_bits_15_matches_prefixa2a5a94154c4b4c6efd07c71b5146350.tex}}}}

\newcommand{\sailRISCVvalhexBitsOneFiveBackwardsMatches}{\saildoclabelled{sailRISCVzhexzybitszy15zybackwardszymatches}{\saildocval{}{\lstinputlisting[language=sail]{sail_latex_riscv/valzhex_bits_15_backwards_matchesc6cd9235f40d6808371474280c964ff1.tex}}}}

\newcommand{\sailRISCVfnhexBitsOneFiveBackwardsMatches}{\saildoclabelled{sailRISCVfnzhexzybitszy15zybackwardszymatches}{\saildocfn{}{\lstinputlisting[language=sail]{sail_latex_riscv/fnzhex_bits_15_backwards_matchesc6cd9235f40d6808371474280c964ff1.tex}}}}

\newcommand{\sailRISCVvalhexBitsOneFiveBackwards}{\saildoclabelled{sailRISCVzhexzybitszy15zybackwards}{\saildocval{}{\lstinputlisting[language=sail]{sail_latex_riscv/valzhex_bits_15_backwardscb4ce484e79dc8d005cf2bda729e7867.tex}}}}

\newcommand{\sailRISCVfnhexBitsOneFiveBackwards}{\saildoclabelled{sailRISCVfnzhexzybitszy15zybackwards}{\saildocfn{}{\lstinputlisting[language=sail]{sail_latex_riscv/fnzhex_bits_15_backwardscb4ce484e79dc8d005cf2bda729e7867.tex}}}}

\newcommand{\sailRISCVvalhexBitsOneSix}{\saildoclabelled{sailRISCVzhexzybitszy16}{\saildocval{}{\lstinputlisting[language=sail]{sail_latex_riscv/valzhex_bits_167d95ad16536f8aabd412012ac1a3e03f.tex}}}}

\newcommand{\sailRISCVvalhexBitsOneSixForwards}{\saildoclabelled{sailRISCVzhexzybitszy16zyforwards}{\saildocval{}{\lstinputlisting[language=sail]{sail_latex_riscv/valzhex_bits_16_forwards7f61f6684c0758bacebc97ae288552c8.tex}}}}

\newcommand{\sailRISCVvalhexBitsOneSixForwardsMatches}{\saildoclabelled{sailRISCVzhexzybitszy16zyforwardszymatches}{\saildocval{}{\lstinputlisting[language=sail]{sail_latex_riscv/valzhex_bits_16_forwards_matches99127000f54090aaa9f0aeb2c20f4685.tex}}}}

\newcommand{\sailRISCVfnhexBitsOneSixForwardsMatches}{\saildoclabelled{sailRISCVfnzhexzybitszy16zyforwardszymatches}{\saildocfn{}{\lstinputlisting[language=sail]{sail_latex_riscv/fnzhex_bits_16_forwards_matches99127000f54090aaa9f0aeb2c20f4685.tex}}}}

\newcommand{\sailRISCVvalhexBitsOneSixMatchesPrefix}{\saildoclabelled{sailRISCVzhexzybitszy16zymatcheszyprefix}{\saildocval{}{\lstinputlisting[language=sail]{sail_latex_riscv/valzhex_bits_16_matches_prefix38bcd18ff44c77953c7f4f991dfa37b8.tex}}}}

\newcommand{\sailRISCVvalhexBitsOneSixBackwardsMatches}{\saildoclabelled{sailRISCVzhexzybitszy16zybackwardszymatches}{\saildocval{}{\lstinputlisting[language=sail]{sail_latex_riscv/valzhex_bits_16_backwards_matches8ff37bbe35e588c9d66f041b5b32d8e0.tex}}}}

\newcommand{\sailRISCVfnhexBitsOneSixBackwardsMatches}{\saildoclabelled{sailRISCVfnzhexzybitszy16zybackwardszymatches}{\saildocfn{}{\lstinputlisting[language=sail]{sail_latex_riscv/fnzhex_bits_16_backwards_matches8ff37bbe35e588c9d66f041b5b32d8e0.tex}}}}

\newcommand{\sailRISCVvalhexBitsOneSixBackwards}{\saildoclabelled{sailRISCVzhexzybitszy16zybackwards}{\saildocval{}{\lstinputlisting[language=sail]{sail_latex_riscv/valzhex_bits_16_backwards8afef1c1c4af4f381f60fc4dc7267c9c.tex}}}}

\newcommand{\sailRISCVfnhexBitsOneSixBackwards}{\saildoclabelled{sailRISCVfnzhexzybitszy16zybackwards}{\saildocfn{}{\lstinputlisting[language=sail]{sail_latex_riscv/fnzhex_bits_16_backwards8afef1c1c4af4f381f60fc4dc7267c9c.tex}}}}

\newcommand{\sailRISCVvalhexBitsOneSeven}{\saildoclabelled{sailRISCVzhexzybitszy17}{\saildocval{}{\lstinputlisting[language=sail]{sail_latex_riscv/valzhex_bits_17d357434a9e9aecc580dae54ef5402fcf.tex}}}}

\newcommand{\sailRISCVvalhexBitsOneSevenForwards}{\saildoclabelled{sailRISCVzhexzybitszy17zyforwards}{\saildocval{}{\lstinputlisting[language=sail]{sail_latex_riscv/valzhex_bits_17_forwards1d1b1aacf0ce8d2f042cb60029feef84.tex}}}}

\newcommand{\sailRISCVvalhexBitsOneSevenForwardsMatches}{\saildoclabelled{sailRISCVzhexzybitszy17zyforwardszymatches}{\saildocval{}{\lstinputlisting[language=sail]{sail_latex_riscv/valzhex_bits_17_forwards_matches45df8bfd52709ac4d9906f538498742e.tex}}}}

\newcommand{\sailRISCVfnhexBitsOneSevenForwardsMatches}{\saildoclabelled{sailRISCVfnzhexzybitszy17zyforwardszymatches}{\saildocfn{}{\lstinputlisting[language=sail]{sail_latex_riscv/fnzhex_bits_17_forwards_matches45df8bfd52709ac4d9906f538498742e.tex}}}}

\newcommand{\sailRISCVvalhexBitsOneSevenMatchesPrefix}{\saildoclabelled{sailRISCVzhexzybitszy17zymatcheszyprefix}{\saildocval{}{\lstinputlisting[language=sail]{sail_latex_riscv/valzhex_bits_17_matches_prefix02c803ee497f2adf656b9a546dc9d148.tex}}}}

\newcommand{\sailRISCVvalhexBitsOneSevenBackwardsMatches}{\saildoclabelled{sailRISCVzhexzybitszy17zybackwardszymatches}{\saildocval{}{\lstinputlisting[language=sail]{sail_latex_riscv/valzhex_bits_17_backwards_matchesd594e6fb2cf338d85aa8e3a859821359.tex}}}}

\newcommand{\sailRISCVfnhexBitsOneSevenBackwardsMatches}{\saildoclabelled{sailRISCVfnzhexzybitszy17zybackwardszymatches}{\saildocfn{}{\lstinputlisting[language=sail]{sail_latex_riscv/fnzhex_bits_17_backwards_matchesd594e6fb2cf338d85aa8e3a859821359.tex}}}}

\newcommand{\sailRISCVvalhexBitsOneSevenBackwards}{\saildoclabelled{sailRISCVzhexzybitszy17zybackwards}{\saildocval{}{\lstinputlisting[language=sail]{sail_latex_riscv/valzhex_bits_17_backwards96076e6b0ee1d8c887acfae20a696145.tex}}}}

\newcommand{\sailRISCVfnhexBitsOneSevenBackwards}{\saildoclabelled{sailRISCVfnzhexzybitszy17zybackwards}{\saildocfn{}{\lstinputlisting[language=sail]{sail_latex_riscv/fnzhex_bits_17_backwards96076e6b0ee1d8c887acfae20a696145.tex}}}}

\newcommand{\sailRISCVvalhexBitsOneEight}{\saildoclabelled{sailRISCVzhexzybitszy18}{\saildocval{}{\lstinputlisting[language=sail]{sail_latex_riscv/valzhex_bits_1819473bd89c251dd41ec313d06e46da0e.tex}}}}

\newcommand{\sailRISCVvalhexBitsOneEightForwards}{\saildoclabelled{sailRISCVzhexzybitszy18zyforwards}{\saildocval{}{\lstinputlisting[language=sail]{sail_latex_riscv/valzhex_bits_18_forwards47e61ad2851c0c0e789494a5e30c9e1c.tex}}}}

\newcommand{\sailRISCVvalhexBitsOneEightForwardsMatches}{\saildoclabelled{sailRISCVzhexzybitszy18zyforwardszymatches}{\saildocval{}{\lstinputlisting[language=sail]{sail_latex_riscv/valzhex_bits_18_forwards_matches6e035e739a1d5414f87357bc391d08b3.tex}}}}

\newcommand{\sailRISCVfnhexBitsOneEightForwardsMatches}{\saildoclabelled{sailRISCVfnzhexzybitszy18zyforwardszymatches}{\saildocfn{}{\lstinputlisting[language=sail]{sail_latex_riscv/fnzhex_bits_18_forwards_matches6e035e739a1d5414f87357bc391d08b3.tex}}}}

\newcommand{\sailRISCVvalhexBitsOneEightMatchesPrefix}{\saildoclabelled{sailRISCVzhexzybitszy18zymatcheszyprefix}{\saildocval{}{\lstinputlisting[language=sail]{sail_latex_riscv/valzhex_bits_18_matches_prefix0123403473ed4dd9dbffc716d963c568.tex}}}}

\newcommand{\sailRISCVvalhexBitsOneEightBackwardsMatches}{\saildoclabelled{sailRISCVzhexzybitszy18zybackwardszymatches}{\saildocval{}{\lstinputlisting[language=sail]{sail_latex_riscv/valzhex_bits_18_backwards_matches72b528c9dd8e982607b88bd8ab46a37d.tex}}}}

\newcommand{\sailRISCVfnhexBitsOneEightBackwardsMatches}{\saildoclabelled{sailRISCVfnzhexzybitszy18zybackwardszymatches}{\saildocfn{}{\lstinputlisting[language=sail]{sail_latex_riscv/fnzhex_bits_18_backwards_matches72b528c9dd8e982607b88bd8ab46a37d.tex}}}}

\newcommand{\sailRISCVvalhexBitsOneEightBackwards}{\saildoclabelled{sailRISCVzhexzybitszy18zybackwards}{\saildocval{}{\lstinputlisting[language=sail]{sail_latex_riscv/valzhex_bits_18_backwards0500c1a0187a21072c3255f60243909f.tex}}}}

\newcommand{\sailRISCVfnhexBitsOneEightBackwards}{\saildoclabelled{sailRISCVfnzhexzybitszy18zybackwards}{\saildocfn{}{\lstinputlisting[language=sail]{sail_latex_riscv/fnzhex_bits_18_backwards0500c1a0187a21072c3255f60243909f.tex}}}}

\newcommand{\sailRISCVvalhexBitsOneNine}{\saildoclabelled{sailRISCVzhexzybitszy19}{\saildocval{}{\lstinputlisting[language=sail]{sail_latex_riscv/valzhex_bits_1938ed2064ef7ce8152e6dfd8213383d7e.tex}}}}

\newcommand{\sailRISCVvalhexBitsOneNineForwards}{\saildoclabelled{sailRISCVzhexzybitszy19zyforwards}{\saildocval{}{\lstinputlisting[language=sail]{sail_latex_riscv/valzhex_bits_19_forwardsc23db4c1e0ead6b3252ea4464c668ee1.tex}}}}

\newcommand{\sailRISCVvalhexBitsOneNineForwardsMatches}{\saildoclabelled{sailRISCVzhexzybitszy19zyforwardszymatches}{\saildocval{}{\lstinputlisting[language=sail]{sail_latex_riscv/valzhex_bits_19_forwards_matches07c7d804aca651680480475ad9d45b66.tex}}}}

\newcommand{\sailRISCVfnhexBitsOneNineForwardsMatches}{\saildoclabelled{sailRISCVfnzhexzybitszy19zyforwardszymatches}{\saildocfn{}{\lstinputlisting[language=sail]{sail_latex_riscv/fnzhex_bits_19_forwards_matches07c7d804aca651680480475ad9d45b66.tex}}}}

\newcommand{\sailRISCVvalhexBitsOneNineMatchesPrefix}{\saildoclabelled{sailRISCVzhexzybitszy19zymatcheszyprefix}{\saildocval{}{\lstinputlisting[language=sail]{sail_latex_riscv/valzhex_bits_19_matches_prefix28d0e3106abb68242ac4e35ed75b7eea.tex}}}}

\newcommand{\sailRISCVvalhexBitsOneNineBackwardsMatches}{\saildoclabelled{sailRISCVzhexzybitszy19zybackwardszymatches}{\saildocval{}{\lstinputlisting[language=sail]{sail_latex_riscv/valzhex_bits_19_backwards_matchesd823f57016d205b9df109966ee967db9.tex}}}}

\newcommand{\sailRISCVfnhexBitsOneNineBackwardsMatches}{\saildoclabelled{sailRISCVfnzhexzybitszy19zybackwardszymatches}{\saildocfn{}{\lstinputlisting[language=sail]{sail_latex_riscv/fnzhex_bits_19_backwards_matchesd823f57016d205b9df109966ee967db9.tex}}}}

\newcommand{\sailRISCVvalhexBitsOneNineBackwards}{\saildoclabelled{sailRISCVzhexzybitszy19zybackwards}{\saildocval{}{\lstinputlisting[language=sail]{sail_latex_riscv/valzhex_bits_19_backwards8a1ab4ffc6b6e5e3c0cf6702ec74b710.tex}}}}

\newcommand{\sailRISCVfnhexBitsOneNineBackwards}{\saildoclabelled{sailRISCVfnzhexzybitszy19zybackwards}{\saildocfn{}{\lstinputlisting[language=sail]{sail_latex_riscv/fnzhex_bits_19_backwards8a1ab4ffc6b6e5e3c0cf6702ec74b710.tex}}}}

\newcommand{\sailRISCVvalhexBitsTwoZero}{\saildoclabelled{sailRISCVzhexzybitszy20}{\saildocval{}{\lstinputlisting[language=sail]{sail_latex_riscv/valzhex_bits_20073767bf3c4e755df8668de0f192df95.tex}}}}

\newcommand{\sailRISCVvalhexBitsTwoZeroForwards}{\saildoclabelled{sailRISCVzhexzybitszy20zyforwards}{\saildocval{}{\lstinputlisting[language=sail]{sail_latex_riscv/valzhex_bits_20_forwards00e9d80df23932a0cd55fe8e4cea5260.tex}}}}

\newcommand{\sailRISCVvalhexBitsTwoZeroForwardsMatches}{\saildoclabelled{sailRISCVzhexzybitszy20zyforwardszymatches}{\saildocval{}{\lstinputlisting[language=sail]{sail_latex_riscv/valzhex_bits_20_forwards_matchesde8955f96663b48788095ef74c50f0aa.tex}}}}

\newcommand{\sailRISCVfnhexBitsTwoZeroForwardsMatches}{\saildoclabelled{sailRISCVfnzhexzybitszy20zyforwardszymatches}{\saildocfn{}{\lstinputlisting[language=sail]{sail_latex_riscv/fnzhex_bits_20_forwards_matchesde8955f96663b48788095ef74c50f0aa.tex}}}}

\newcommand{\sailRISCVvalhexBitsTwoZeroMatchesPrefix}{\saildoclabelled{sailRISCVzhexzybitszy20zymatcheszyprefix}{\saildocval{}{\lstinputlisting[language=sail]{sail_latex_riscv/valzhex_bits_20_matches_prefixdf9ef269745a9eb5e6b479b1935b64ab.tex}}}}

\newcommand{\sailRISCVvalhexBitsTwoZeroBackwardsMatches}{\saildoclabelled{sailRISCVzhexzybitszy20zybackwardszymatches}{\saildocval{}{\lstinputlisting[language=sail]{sail_latex_riscv/valzhex_bits_20_backwards_matches8c546ae28d487f510dca7fe30ea629c5.tex}}}}

\newcommand{\sailRISCVfnhexBitsTwoZeroBackwardsMatches}{\saildoclabelled{sailRISCVfnzhexzybitszy20zybackwardszymatches}{\saildocfn{}{\lstinputlisting[language=sail]{sail_latex_riscv/fnzhex_bits_20_backwards_matches8c546ae28d487f510dca7fe30ea629c5.tex}}}}

\newcommand{\sailRISCVvalhexBitsTwoZeroBackwards}{\saildoclabelled{sailRISCVzhexzybitszy20zybackwards}{\saildocval{}{\lstinputlisting[language=sail]{sail_latex_riscv/valzhex_bits_20_backwardsed333b625b69d435d8e8df1b980d11ef.tex}}}}

\newcommand{\sailRISCVfnhexBitsTwoZeroBackwards}{\saildoclabelled{sailRISCVfnzhexzybitszy20zybackwards}{\saildocfn{}{\lstinputlisting[language=sail]{sail_latex_riscv/fnzhex_bits_20_backwardsed333b625b69d435d8e8df1b980d11ef.tex}}}}

\newcommand{\sailRISCVvalhexBitsTwoOne}{\saildoclabelled{sailRISCVzhexzybitszy21}{\saildocval{}{\lstinputlisting[language=sail]{sail_latex_riscv/valzhex_bits_215a399be0fd96b92fed46a34ed6d97e44.tex}}}}

\newcommand{\sailRISCVvalhexBitsTwoOneForwards}{\saildoclabelled{sailRISCVzhexzybitszy21zyforwards}{\saildocval{}{\lstinputlisting[language=sail]{sail_latex_riscv/valzhex_bits_21_forwards5d6372cc66bb45095acb2dbe03a6acfc.tex}}}}

\newcommand{\sailRISCVvalhexBitsTwoOneForwardsMatches}{\saildoclabelled{sailRISCVzhexzybitszy21zyforwardszymatches}{\saildocval{}{\lstinputlisting[language=sail]{sail_latex_riscv/valzhex_bits_21_forwards_matches05f5fa78d5ad49b7efae27025b546cb7.tex}}}}

\newcommand{\sailRISCVfnhexBitsTwoOneForwardsMatches}{\saildoclabelled{sailRISCVfnzhexzybitszy21zyforwardszymatches}{\saildocfn{}{\lstinputlisting[language=sail]{sail_latex_riscv/fnzhex_bits_21_forwards_matches05f5fa78d5ad49b7efae27025b546cb7.tex}}}}

\newcommand{\sailRISCVvalhexBitsTwoOneMatchesPrefix}{\saildoclabelled{sailRISCVzhexzybitszy21zymatcheszyprefix}{\saildocval{}{\lstinputlisting[language=sail]{sail_latex_riscv/valzhex_bits_21_matches_prefix159b58fe3bdaba7ade8041d986b2ecf6.tex}}}}

\newcommand{\sailRISCVvalhexBitsTwoOneBackwardsMatches}{\saildoclabelled{sailRISCVzhexzybitszy21zybackwardszymatches}{\saildocval{}{\lstinputlisting[language=sail]{sail_latex_riscv/valzhex_bits_21_backwards_matches477a84c0bff49ab6511438197e40d9c3.tex}}}}

\newcommand{\sailRISCVfnhexBitsTwoOneBackwardsMatches}{\saildoclabelled{sailRISCVfnzhexzybitszy21zybackwardszymatches}{\saildocfn{}{\lstinputlisting[language=sail]{sail_latex_riscv/fnzhex_bits_21_backwards_matches477a84c0bff49ab6511438197e40d9c3.tex}}}}

\newcommand{\sailRISCVvalhexBitsTwoOneBackwards}{\saildoclabelled{sailRISCVzhexzybitszy21zybackwards}{\saildocval{}{\lstinputlisting[language=sail]{sail_latex_riscv/valzhex_bits_21_backwardsc44f7bb35fd47e4506558a2d7d995f8e.tex}}}}

\newcommand{\sailRISCVfnhexBitsTwoOneBackwards}{\saildoclabelled{sailRISCVfnzhexzybitszy21zybackwards}{\saildocfn{}{\lstinputlisting[language=sail]{sail_latex_riscv/fnzhex_bits_21_backwardsc44f7bb35fd47e4506558a2d7d995f8e.tex}}}}

\newcommand{\sailRISCVvalhexBitsTwoTwo}{\saildoclabelled{sailRISCVzhexzybitszy22}{\saildocval{}{\lstinputlisting[language=sail]{sail_latex_riscv/valzhex_bits_228df71d429e9677759c4b9421597b9624.tex}}}}

\newcommand{\sailRISCVvalhexBitsTwoTwoForwards}{\saildoclabelled{sailRISCVzhexzybitszy22zyforwards}{\saildocval{}{\lstinputlisting[language=sail]{sail_latex_riscv/valzhex_bits_22_forwards5f20df76cf594392e558c43103ffc6e4.tex}}}}

\newcommand{\sailRISCVvalhexBitsTwoTwoForwardsMatches}{\saildoclabelled{sailRISCVzhexzybitszy22zyforwardszymatches}{\saildocval{}{\lstinputlisting[language=sail]{sail_latex_riscv/valzhex_bits_22_forwards_matches4fe97899c8181992923cd1daaa36d526.tex}}}}

\newcommand{\sailRISCVfnhexBitsTwoTwoForwardsMatches}{\saildoclabelled{sailRISCVfnzhexzybitszy22zyforwardszymatches}{\saildocfn{}{\lstinputlisting[language=sail]{sail_latex_riscv/fnzhex_bits_22_forwards_matches4fe97899c8181992923cd1daaa36d526.tex}}}}

\newcommand{\sailRISCVvalhexBitsTwoTwoMatchesPrefix}{\saildoclabelled{sailRISCVzhexzybitszy22zymatcheszyprefix}{\saildocval{}{\lstinputlisting[language=sail]{sail_latex_riscv/valzhex_bits_22_matches_prefix1a790a92ddf1d11e5f9d0ff8b4ca8b1c.tex}}}}

\newcommand{\sailRISCVvalhexBitsTwoTwoBackwardsMatches}{\saildoclabelled{sailRISCVzhexzybitszy22zybackwardszymatches}{\saildocval{}{\lstinputlisting[language=sail]{sail_latex_riscv/valzhex_bits_22_backwards_matchesfd56758d4cd0fd6150ee51afa372c83b.tex}}}}

\newcommand{\sailRISCVfnhexBitsTwoTwoBackwardsMatches}{\saildoclabelled{sailRISCVfnzhexzybitszy22zybackwardszymatches}{\saildocfn{}{\lstinputlisting[language=sail]{sail_latex_riscv/fnzhex_bits_22_backwards_matchesfd56758d4cd0fd6150ee51afa372c83b.tex}}}}

\newcommand{\sailRISCVvalhexBitsTwoTwoBackwards}{\saildoclabelled{sailRISCVzhexzybitszy22zybackwards}{\saildocval{}{\lstinputlisting[language=sail]{sail_latex_riscv/valzhex_bits_22_backwardsfbed1dd3d215529ff44a9e8165218968.tex}}}}

\newcommand{\sailRISCVfnhexBitsTwoTwoBackwards}{\saildoclabelled{sailRISCVfnzhexzybitszy22zybackwards}{\saildocfn{}{\lstinputlisting[language=sail]{sail_latex_riscv/fnzhex_bits_22_backwardsfbed1dd3d215529ff44a9e8165218968.tex}}}}

\newcommand{\sailRISCVvalhexBitsTwoThree}{\saildoclabelled{sailRISCVzhexzybitszy23}{\saildocval{}{\lstinputlisting[language=sail]{sail_latex_riscv/valzhex_bits_23418f7302c0b7161a092d463e4d8af7dc.tex}}}}

\newcommand{\sailRISCVvalhexBitsTwoThreeForwards}{\saildoclabelled{sailRISCVzhexzybitszy23zyforwards}{\saildocval{}{\lstinputlisting[language=sail]{sail_latex_riscv/valzhex_bits_23_forwardsdd8792f3a53735041dd06c758d6e9f7e.tex}}}}

\newcommand{\sailRISCVvalhexBitsTwoThreeForwardsMatches}{\saildoclabelled{sailRISCVzhexzybitszy23zyforwardszymatches}{\saildocval{}{\lstinputlisting[language=sail]{sail_latex_riscv/valzhex_bits_23_forwards_matchesb58b33da7b8d1a86392eabd6cc87b2c3.tex}}}}

\newcommand{\sailRISCVfnhexBitsTwoThreeForwardsMatches}{\saildoclabelled{sailRISCVfnzhexzybitszy23zyforwardszymatches}{\saildocfn{}{\lstinputlisting[language=sail]{sail_latex_riscv/fnzhex_bits_23_forwards_matchesb58b33da7b8d1a86392eabd6cc87b2c3.tex}}}}

\newcommand{\sailRISCVvalhexBitsTwoThreeMatchesPrefix}{\saildoclabelled{sailRISCVzhexzybitszy23zymatcheszyprefix}{\saildocval{}{\lstinputlisting[language=sail]{sail_latex_riscv/valzhex_bits_23_matches_prefix48403fc7c51b83fad5e40b0f6370b8fa.tex}}}}

\newcommand{\sailRISCVvalhexBitsTwoThreeBackwardsMatches}{\saildoclabelled{sailRISCVzhexzybitszy23zybackwardszymatches}{\saildocval{}{\lstinputlisting[language=sail]{sail_latex_riscv/valzhex_bits_23_backwards_matchesa97370178cb502dd696426e4a347c318.tex}}}}

\newcommand{\sailRISCVfnhexBitsTwoThreeBackwardsMatches}{\saildoclabelled{sailRISCVfnzhexzybitszy23zybackwardszymatches}{\saildocfn{}{\lstinputlisting[language=sail]{sail_latex_riscv/fnzhex_bits_23_backwards_matchesa97370178cb502dd696426e4a347c318.tex}}}}

\newcommand{\sailRISCVvalhexBitsTwoThreeBackwards}{\saildoclabelled{sailRISCVzhexzybitszy23zybackwards}{\saildocval{}{\lstinputlisting[language=sail]{sail_latex_riscv/valzhex_bits_23_backwards2183795dfc2ff2f588be2709c2cd7639.tex}}}}

\newcommand{\sailRISCVfnhexBitsTwoThreeBackwards}{\saildoclabelled{sailRISCVfnzhexzybitszy23zybackwards}{\saildocfn{}{\lstinputlisting[language=sail]{sail_latex_riscv/fnzhex_bits_23_backwards2183795dfc2ff2f588be2709c2cd7639.tex}}}}

\newcommand{\sailRISCVvalhexBitsTwoFour}{\saildoclabelled{sailRISCVzhexzybitszy24}{\saildocval{}{\lstinputlisting[language=sail]{sail_latex_riscv/valzhex_bits_246fe29a7e07e79124e13aa086639a725a.tex}}}}

\newcommand{\sailRISCVvalhexBitsTwoFourForwards}{\saildoclabelled{sailRISCVzhexzybitszy24zyforwards}{\saildocval{}{\lstinputlisting[language=sail]{sail_latex_riscv/valzhex_bits_24_forwardsdeaf00c9fe92ade0c2e92648a2b58039.tex}}}}

\newcommand{\sailRISCVvalhexBitsTwoFourForwardsMatches}{\saildoclabelled{sailRISCVzhexzybitszy24zyforwardszymatches}{\saildocval{}{\lstinputlisting[language=sail]{sail_latex_riscv/valzhex_bits_24_forwards_matchesf4dd0da48edd6a31433b0bae1bedc1f5.tex}}}}

\newcommand{\sailRISCVfnhexBitsTwoFourForwardsMatches}{\saildoclabelled{sailRISCVfnzhexzybitszy24zyforwardszymatches}{\saildocfn{}{\lstinputlisting[language=sail]{sail_latex_riscv/fnzhex_bits_24_forwards_matchesf4dd0da48edd6a31433b0bae1bedc1f5.tex}}}}

\newcommand{\sailRISCVvalhexBitsTwoFourMatchesPrefix}{\saildoclabelled{sailRISCVzhexzybitszy24zymatcheszyprefix}{\saildocval{}{\lstinputlisting[language=sail]{sail_latex_riscv/valzhex_bits_24_matches_prefix6ba2f72e46135bdb8c80c62ad4b58e8f.tex}}}}

\newcommand{\sailRISCVvalhexBitsTwoFourBackwardsMatches}{\saildoclabelled{sailRISCVzhexzybitszy24zybackwardszymatches}{\saildocval{}{\lstinputlisting[language=sail]{sail_latex_riscv/valzhex_bits_24_backwards_matchesb2a483aa8b7cae87c37d98bc8b2a7345.tex}}}}

\newcommand{\sailRISCVfnhexBitsTwoFourBackwardsMatches}{\saildoclabelled{sailRISCVfnzhexzybitszy24zybackwardszymatches}{\saildocfn{}{\lstinputlisting[language=sail]{sail_latex_riscv/fnzhex_bits_24_backwards_matchesb2a483aa8b7cae87c37d98bc8b2a7345.tex}}}}

\newcommand{\sailRISCVvalhexBitsTwoFourBackwards}{\saildoclabelled{sailRISCVzhexzybitszy24zybackwards}{\saildocval{}{\lstinputlisting[language=sail]{sail_latex_riscv/valzhex_bits_24_backwardsa48c754c7049ebcb72d1fbba8bcfaadf.tex}}}}

\newcommand{\sailRISCVfnhexBitsTwoFourBackwards}{\saildoclabelled{sailRISCVfnzhexzybitszy24zybackwards}{\saildocfn{}{\lstinputlisting[language=sail]{sail_latex_riscv/fnzhex_bits_24_backwardsa48c754c7049ebcb72d1fbba8bcfaadf.tex}}}}

\newcommand{\sailRISCVvalhexBitsTwoFive}{\saildoclabelled{sailRISCVzhexzybitszy25}{\saildocval{}{\lstinputlisting[language=sail]{sail_latex_riscv/valzhex_bits_25b2fb96ee1c2b78dcefb54132cd9af05a.tex}}}}

\newcommand{\sailRISCVvalhexBitsTwoFiveForwards}{\saildoclabelled{sailRISCVzhexzybitszy25zyforwards}{\saildocval{}{\lstinputlisting[language=sail]{sail_latex_riscv/valzhex_bits_25_forwards72bc0d414786f2c38ca3e1ba78cab739.tex}}}}

\newcommand{\sailRISCVvalhexBitsTwoFiveForwardsMatches}{\saildoclabelled{sailRISCVzhexzybitszy25zyforwardszymatches}{\saildocval{}{\lstinputlisting[language=sail]{sail_latex_riscv/valzhex_bits_25_forwards_matches6d8831afa1ed150c2dca6afcd8d46fd1.tex}}}}

\newcommand{\sailRISCVfnhexBitsTwoFiveForwardsMatches}{\saildoclabelled{sailRISCVfnzhexzybitszy25zyforwardszymatches}{\saildocfn{}{\lstinputlisting[language=sail]{sail_latex_riscv/fnzhex_bits_25_forwards_matches6d8831afa1ed150c2dca6afcd8d46fd1.tex}}}}

\newcommand{\sailRISCVvalhexBitsTwoFiveMatchesPrefix}{\saildoclabelled{sailRISCVzhexzybitszy25zymatcheszyprefix}{\saildocval{}{\lstinputlisting[language=sail]{sail_latex_riscv/valzhex_bits_25_matches_prefixa3fd9e9a9ab3f00e09575979dd9a94e8.tex}}}}

\newcommand{\sailRISCVvalhexBitsTwoFiveBackwardsMatches}{\saildoclabelled{sailRISCVzhexzybitszy25zybackwardszymatches}{\saildocval{}{\lstinputlisting[language=sail]{sail_latex_riscv/valzhex_bits_25_backwards_matchese97e136616a70048191383b2a45471f6.tex}}}}

\newcommand{\sailRISCVfnhexBitsTwoFiveBackwardsMatches}{\saildoclabelled{sailRISCVfnzhexzybitszy25zybackwardszymatches}{\saildocfn{}{\lstinputlisting[language=sail]{sail_latex_riscv/fnzhex_bits_25_backwards_matchese97e136616a70048191383b2a45471f6.tex}}}}

\newcommand{\sailRISCVvalhexBitsTwoFiveBackwards}{\saildoclabelled{sailRISCVzhexzybitszy25zybackwards}{\saildocval{}{\lstinputlisting[language=sail]{sail_latex_riscv/valzhex_bits_25_backwardsb4f6ac6cc03e034620b7a4a11a89c5fc.tex}}}}

\newcommand{\sailRISCVfnhexBitsTwoFiveBackwards}{\saildoclabelled{sailRISCVfnzhexzybitszy25zybackwards}{\saildocfn{}{\lstinputlisting[language=sail]{sail_latex_riscv/fnzhex_bits_25_backwardsb4f6ac6cc03e034620b7a4a11a89c5fc.tex}}}}

\newcommand{\sailRISCVvalhexBitsTwoSix}{\saildoclabelled{sailRISCVzhexzybitszy26}{\saildocval{}{\lstinputlisting[language=sail]{sail_latex_riscv/valzhex_bits_26663364c85fd29e21280b999b60cb2a55.tex}}}}

\newcommand{\sailRISCVvalhexBitsTwoSixForwards}{\saildoclabelled{sailRISCVzhexzybitszy26zyforwards}{\saildocval{}{\lstinputlisting[language=sail]{sail_latex_riscv/valzhex_bits_26_forwards5a6114b9bf479d01fa3a9e2beec5d91b.tex}}}}

\newcommand{\sailRISCVvalhexBitsTwoSixForwardsMatches}{\saildoclabelled{sailRISCVzhexzybitszy26zyforwardszymatches}{\saildocval{}{\lstinputlisting[language=sail]{sail_latex_riscv/valzhex_bits_26_forwards_matchesec5940b6c813530a4c8c509af6c000b6.tex}}}}

\newcommand{\sailRISCVfnhexBitsTwoSixForwardsMatches}{\saildoclabelled{sailRISCVfnzhexzybitszy26zyforwardszymatches}{\saildocfn{}{\lstinputlisting[language=sail]{sail_latex_riscv/fnzhex_bits_26_forwards_matchesec5940b6c813530a4c8c509af6c000b6.tex}}}}

\newcommand{\sailRISCVvalhexBitsTwoSixMatchesPrefix}{\saildoclabelled{sailRISCVzhexzybitszy26zymatcheszyprefix}{\saildocval{}{\lstinputlisting[language=sail]{sail_latex_riscv/valzhex_bits_26_matches_prefix7b5176f43e479ddf2dec5f4a2011b98f.tex}}}}

\newcommand{\sailRISCVvalhexBitsTwoSixBackwardsMatches}{\saildoclabelled{sailRISCVzhexzybitszy26zybackwardszymatches}{\saildocval{}{\lstinputlisting[language=sail]{sail_latex_riscv/valzhex_bits_26_backwards_matches0ea73e31736f776e20958f45ec3e81bd.tex}}}}

\newcommand{\sailRISCVfnhexBitsTwoSixBackwardsMatches}{\saildoclabelled{sailRISCVfnzhexzybitszy26zybackwardszymatches}{\saildocfn{}{\lstinputlisting[language=sail]{sail_latex_riscv/fnzhex_bits_26_backwards_matches0ea73e31736f776e20958f45ec3e81bd.tex}}}}

\newcommand{\sailRISCVvalhexBitsTwoSixBackwards}{\saildoclabelled{sailRISCVzhexzybitszy26zybackwards}{\saildocval{}{\lstinputlisting[language=sail]{sail_latex_riscv/valzhex_bits_26_backwardsca9980b4fb1beb5e20216094b1b6b352.tex}}}}

\newcommand{\sailRISCVfnhexBitsTwoSixBackwards}{\saildoclabelled{sailRISCVfnzhexzybitszy26zybackwards}{\saildocfn{}{\lstinputlisting[language=sail]{sail_latex_riscv/fnzhex_bits_26_backwardsca9980b4fb1beb5e20216094b1b6b352.tex}}}}

\newcommand{\sailRISCVvalhexBitsTwoSeven}{\saildoclabelled{sailRISCVzhexzybitszy27}{\saildocval{}{\lstinputlisting[language=sail]{sail_latex_riscv/valzhex_bits_27b316c424414b7868c1946d3aa5df4022.tex}}}}

\newcommand{\sailRISCVvalhexBitsTwoSevenForwards}{\saildoclabelled{sailRISCVzhexzybitszy27zyforwards}{\saildocval{}{\lstinputlisting[language=sail]{sail_latex_riscv/valzhex_bits_27_forwardsd1cadb067b94f8db9ac21a09b0a9c465.tex}}}}

\newcommand{\sailRISCVvalhexBitsTwoSevenForwardsMatches}{\saildoclabelled{sailRISCVzhexzybitszy27zyforwardszymatches}{\saildocval{}{\lstinputlisting[language=sail]{sail_latex_riscv/valzhex_bits_27_forwards_matches6867770d92d07d048cf22050458e9a6f.tex}}}}

\newcommand{\sailRISCVfnhexBitsTwoSevenForwardsMatches}{\saildoclabelled{sailRISCVfnzhexzybitszy27zyforwardszymatches}{\saildocfn{}{\lstinputlisting[language=sail]{sail_latex_riscv/fnzhex_bits_27_forwards_matches6867770d92d07d048cf22050458e9a6f.tex}}}}

\newcommand{\sailRISCVvalhexBitsTwoSevenMatchesPrefix}{\saildoclabelled{sailRISCVzhexzybitszy27zymatcheszyprefix}{\saildocval{}{\lstinputlisting[language=sail]{sail_latex_riscv/valzhex_bits_27_matches_prefixdc708eca229898ef6aae7c4b60f3b806.tex}}}}

\newcommand{\sailRISCVvalhexBitsTwoSevenBackwardsMatches}{\saildoclabelled{sailRISCVzhexzybitszy27zybackwardszymatches}{\saildocval{}{\lstinputlisting[language=sail]{sail_latex_riscv/valzhex_bits_27_backwards_matchesc36973ea671b22434d73ad552c3b36f1.tex}}}}

\newcommand{\sailRISCVfnhexBitsTwoSevenBackwardsMatches}{\saildoclabelled{sailRISCVfnzhexzybitszy27zybackwardszymatches}{\saildocfn{}{\lstinputlisting[language=sail]{sail_latex_riscv/fnzhex_bits_27_backwards_matchesc36973ea671b22434d73ad552c3b36f1.tex}}}}

\newcommand{\sailRISCVvalhexBitsTwoSevenBackwards}{\saildoclabelled{sailRISCVzhexzybitszy27zybackwards}{\saildocval{}{\lstinputlisting[language=sail]{sail_latex_riscv/valzhex_bits_27_backwards4d49d22a17608696fe779e708d0d5091.tex}}}}

\newcommand{\sailRISCVfnhexBitsTwoSevenBackwards}{\saildoclabelled{sailRISCVfnzhexzybitszy27zybackwards}{\saildocfn{}{\lstinputlisting[language=sail]{sail_latex_riscv/fnzhex_bits_27_backwards4d49d22a17608696fe779e708d0d5091.tex}}}}

\newcommand{\sailRISCVvalhexBitsTwoEight}{\saildoclabelled{sailRISCVzhexzybitszy28}{\saildocval{}{\lstinputlisting[language=sail]{sail_latex_riscv/valzhex_bits_288d4243d5b924b9df9b04ff6f5779762e.tex}}}}

\newcommand{\sailRISCVvalhexBitsTwoEightForwards}{\saildoclabelled{sailRISCVzhexzybitszy28zyforwards}{\saildocval{}{\lstinputlisting[language=sail]{sail_latex_riscv/valzhex_bits_28_forwards1cd2299af6ec5f9078eb9063e9f85a94.tex}}}}

\newcommand{\sailRISCVvalhexBitsTwoEightForwardsMatches}{\saildoclabelled{sailRISCVzhexzybitszy28zyforwardszymatches}{\saildocval{}{\lstinputlisting[language=sail]{sail_latex_riscv/valzhex_bits_28_forwards_matchesc4d71e8f4caf976479bf0fd50de2a1f7.tex}}}}

\newcommand{\sailRISCVfnhexBitsTwoEightForwardsMatches}{\saildoclabelled{sailRISCVfnzhexzybitszy28zyforwardszymatches}{\saildocfn{}{\lstinputlisting[language=sail]{sail_latex_riscv/fnzhex_bits_28_forwards_matchesc4d71e8f4caf976479bf0fd50de2a1f7.tex}}}}

\newcommand{\sailRISCVvalhexBitsTwoEightMatchesPrefix}{\saildoclabelled{sailRISCVzhexzybitszy28zymatcheszyprefix}{\saildocval{}{\lstinputlisting[language=sail]{sail_latex_riscv/valzhex_bits_28_matches_prefix70f347867a8aba96b2c10f78f8f7a372.tex}}}}

\newcommand{\sailRISCVvalhexBitsTwoEightBackwardsMatches}{\saildoclabelled{sailRISCVzhexzybitszy28zybackwardszymatches}{\saildocval{}{\lstinputlisting[language=sail]{sail_latex_riscv/valzhex_bits_28_backwards_matches0baec25715c212015f55f72f54702da4.tex}}}}

\newcommand{\sailRISCVfnhexBitsTwoEightBackwardsMatches}{\saildoclabelled{sailRISCVfnzhexzybitszy28zybackwardszymatches}{\saildocfn{}{\lstinputlisting[language=sail]{sail_latex_riscv/fnzhex_bits_28_backwards_matches0baec25715c212015f55f72f54702da4.tex}}}}

\newcommand{\sailRISCVvalhexBitsTwoEightBackwards}{\saildoclabelled{sailRISCVzhexzybitszy28zybackwards}{\saildocval{}{\lstinputlisting[language=sail]{sail_latex_riscv/valzhex_bits_28_backwards2c2d2daf166e3a6a7935d895391a113e.tex}}}}

\newcommand{\sailRISCVfnhexBitsTwoEightBackwards}{\saildoclabelled{sailRISCVfnzhexzybitszy28zybackwards}{\saildocfn{}{\lstinputlisting[language=sail]{sail_latex_riscv/fnzhex_bits_28_backwards2c2d2daf166e3a6a7935d895391a113e.tex}}}}

\newcommand{\sailRISCVvalhexBitsTwoNine}{\saildoclabelled{sailRISCVzhexzybitszy29}{\saildocval{}{\lstinputlisting[language=sail]{sail_latex_riscv/valzhex_bits_29efa82370fcd77919b51fb03d70b237d5.tex}}}}

\newcommand{\sailRISCVvalhexBitsTwoNineForwards}{\saildoclabelled{sailRISCVzhexzybitszy29zyforwards}{\saildocval{}{\lstinputlisting[language=sail]{sail_latex_riscv/valzhex_bits_29_forwards25570c63210d57bb228476e7b744cbc0.tex}}}}

\newcommand{\sailRISCVvalhexBitsTwoNineForwardsMatches}{\saildoclabelled{sailRISCVzhexzybitszy29zyforwardszymatches}{\saildocval{}{\lstinputlisting[language=sail]{sail_latex_riscv/valzhex_bits_29_forwards_matchesed84d3ffca269bb316dc98b23b11184a.tex}}}}

\newcommand{\sailRISCVfnhexBitsTwoNineForwardsMatches}{\saildoclabelled{sailRISCVfnzhexzybitszy29zyforwardszymatches}{\saildocfn{}{\lstinputlisting[language=sail]{sail_latex_riscv/fnzhex_bits_29_forwards_matchesed84d3ffca269bb316dc98b23b11184a.tex}}}}

\newcommand{\sailRISCVvalhexBitsTwoNineMatchesPrefix}{\saildoclabelled{sailRISCVzhexzybitszy29zymatcheszyprefix}{\saildocval{}{\lstinputlisting[language=sail]{sail_latex_riscv/valzhex_bits_29_matches_prefix203fecad348616b1e04e6c54168376e2.tex}}}}

\newcommand{\sailRISCVvalhexBitsTwoNineBackwardsMatches}{\saildoclabelled{sailRISCVzhexzybitszy29zybackwardszymatches}{\saildocval{}{\lstinputlisting[language=sail]{sail_latex_riscv/valzhex_bits_29_backwards_matchese61c62311ca330ee022b0ce689f2bf7a.tex}}}}

\newcommand{\sailRISCVfnhexBitsTwoNineBackwardsMatches}{\saildoclabelled{sailRISCVfnzhexzybitszy29zybackwardszymatches}{\saildocfn{}{\lstinputlisting[language=sail]{sail_latex_riscv/fnzhex_bits_29_backwards_matchese61c62311ca330ee022b0ce689f2bf7a.tex}}}}

\newcommand{\sailRISCVvalhexBitsTwoNineBackwards}{\saildoclabelled{sailRISCVzhexzybitszy29zybackwards}{\saildocval{}{\lstinputlisting[language=sail]{sail_latex_riscv/valzhex_bits_29_backwards388cf5f0a97b7aadfda6550d7d52cc35.tex}}}}

\newcommand{\sailRISCVfnhexBitsTwoNineBackwards}{\saildoclabelled{sailRISCVfnzhexzybitszy29zybackwards}{\saildocfn{}{\lstinputlisting[language=sail]{sail_latex_riscv/fnzhex_bits_29_backwards388cf5f0a97b7aadfda6550d7d52cc35.tex}}}}

\newcommand{\sailRISCVvalhexBitsThreeZero}{\saildoclabelled{sailRISCVzhexzybitszy30}{\saildocval{}{\lstinputlisting[language=sail]{sail_latex_riscv/valzhex_bits_30cc6f2a56c4decf190e17ab7b80196c34.tex}}}}

\newcommand{\sailRISCVvalhexBitsThreeZeroForwards}{\saildoclabelled{sailRISCVzhexzybitszy30zyforwards}{\saildocval{}{\lstinputlisting[language=sail]{sail_latex_riscv/valzhex_bits_30_forwardscea5ca246547b759f92507a3787c7a96.tex}}}}

\newcommand{\sailRISCVvalhexBitsThreeZeroForwardsMatches}{\saildoclabelled{sailRISCVzhexzybitszy30zyforwardszymatches}{\saildocval{}{\lstinputlisting[language=sail]{sail_latex_riscv/valzhex_bits_30_forwards_matches3630c22b1e1e03c4f0e0de4c42a765db.tex}}}}

\newcommand{\sailRISCVfnhexBitsThreeZeroForwardsMatches}{\saildoclabelled{sailRISCVfnzhexzybitszy30zyforwardszymatches}{\saildocfn{}{\lstinputlisting[language=sail]{sail_latex_riscv/fnzhex_bits_30_forwards_matches3630c22b1e1e03c4f0e0de4c42a765db.tex}}}}

\newcommand{\sailRISCVvalhexBitsThreeZeroMatchesPrefix}{\saildoclabelled{sailRISCVzhexzybitszy30zymatcheszyprefix}{\saildocval{}{\lstinputlisting[language=sail]{sail_latex_riscv/valzhex_bits_30_matches_prefixbf5a6ef1a55049d8f6c26bfca65564f4.tex}}}}

\newcommand{\sailRISCVvalhexBitsThreeZeroBackwardsMatches}{\saildoclabelled{sailRISCVzhexzybitszy30zybackwardszymatches}{\saildocval{}{\lstinputlisting[language=sail]{sail_latex_riscv/valzhex_bits_30_backwards_matches55c635ce0a1a3c012b3119662b49178c.tex}}}}

\newcommand{\sailRISCVfnhexBitsThreeZeroBackwardsMatches}{\saildoclabelled{sailRISCVfnzhexzybitszy30zybackwardszymatches}{\saildocfn{}{\lstinputlisting[language=sail]{sail_latex_riscv/fnzhex_bits_30_backwards_matches55c635ce0a1a3c012b3119662b49178c.tex}}}}

\newcommand{\sailRISCVvalhexBitsThreeZeroBackwards}{\saildoclabelled{sailRISCVzhexzybitszy30zybackwards}{\saildocval{}{\lstinputlisting[language=sail]{sail_latex_riscv/valzhex_bits_30_backwards3391751afa83c025b48e297fc4113f6e.tex}}}}

\newcommand{\sailRISCVfnhexBitsThreeZeroBackwards}{\saildoclabelled{sailRISCVfnzhexzybitszy30zybackwards}{\saildocfn{}{\lstinputlisting[language=sail]{sail_latex_riscv/fnzhex_bits_30_backwards3391751afa83c025b48e297fc4113f6e.tex}}}}

\newcommand{\sailRISCVvalhexBitsThreeOne}{\saildoclabelled{sailRISCVzhexzybitszy31}{\saildocval{}{\lstinputlisting[language=sail]{sail_latex_riscv/valzhex_bits_314eac40254c178573e1a2e3e8c866b4b9.tex}}}}

\newcommand{\sailRISCVvalhexBitsThreeOneForwards}{\saildoclabelled{sailRISCVzhexzybitszy31zyforwards}{\saildocval{}{\lstinputlisting[language=sail]{sail_latex_riscv/valzhex_bits_31_forwards4963ea17f27efca42c2c3b70e3678a1f.tex}}}}

\newcommand{\sailRISCVvalhexBitsThreeOneForwardsMatches}{\saildoclabelled{sailRISCVzhexzybitszy31zyforwardszymatches}{\saildocval{}{\lstinputlisting[language=sail]{sail_latex_riscv/valzhex_bits_31_forwards_matches830a8b6e6851a2528701adaae04355bf.tex}}}}

\newcommand{\sailRISCVfnhexBitsThreeOneForwardsMatches}{\saildoclabelled{sailRISCVfnzhexzybitszy31zyforwardszymatches}{\saildocfn{}{\lstinputlisting[language=sail]{sail_latex_riscv/fnzhex_bits_31_forwards_matches830a8b6e6851a2528701adaae04355bf.tex}}}}

\newcommand{\sailRISCVvalhexBitsThreeOneMatchesPrefix}{\saildoclabelled{sailRISCVzhexzybitszy31zymatcheszyprefix}{\saildocval{}{\lstinputlisting[language=sail]{sail_latex_riscv/valzhex_bits_31_matches_prefixb56aff200a836c2c62dd85dd75e2e301.tex}}}}

\newcommand{\sailRISCVvalhexBitsThreeOneBackwardsMatches}{\saildoclabelled{sailRISCVzhexzybitszy31zybackwardszymatches}{\saildocval{}{\lstinputlisting[language=sail]{sail_latex_riscv/valzhex_bits_31_backwards_matches68cc0d3844c92f7e0ebe6bcaa48a03ca.tex}}}}

\newcommand{\sailRISCVfnhexBitsThreeOneBackwardsMatches}{\saildoclabelled{sailRISCVfnzhexzybitszy31zybackwardszymatches}{\saildocfn{}{\lstinputlisting[language=sail]{sail_latex_riscv/fnzhex_bits_31_backwards_matches68cc0d3844c92f7e0ebe6bcaa48a03ca.tex}}}}

\newcommand{\sailRISCVvalhexBitsThreeOneBackwards}{\saildoclabelled{sailRISCVzhexzybitszy31zybackwards}{\saildocval{}{\lstinputlisting[language=sail]{sail_latex_riscv/valzhex_bits_31_backwards36aadfc7c167e0b19e4147eb55e51cba.tex}}}}

\newcommand{\sailRISCVfnhexBitsThreeOneBackwards}{\saildoclabelled{sailRISCVfnzhexzybitszy31zybackwards}{\saildocfn{}{\lstinputlisting[language=sail]{sail_latex_riscv/fnzhex_bits_31_backwards36aadfc7c167e0b19e4147eb55e51cba.tex}}}}

\newcommand{\sailRISCVvalhexBitsThreeTwo}{\saildoclabelled{sailRISCVzhexzybitszy32}{\saildocval{}{\lstinputlisting[language=sail]{sail_latex_riscv/valzhex_bits_3297bfc3853730082de31dbfe87d09e9ff.tex}}}}

\newcommand{\sailRISCVvalhexBitsThreeTwoForwards}{\saildoclabelled{sailRISCVzhexzybitszy32zyforwards}{\saildocval{}{\lstinputlisting[language=sail]{sail_latex_riscv/valzhex_bits_32_forwards0b8436c3566927b7a7b6bca8d7dd5251.tex}}}}

\newcommand{\sailRISCVvalhexBitsThreeTwoForwardsMatches}{\saildoclabelled{sailRISCVzhexzybitszy32zyforwardszymatches}{\saildocval{}{\lstinputlisting[language=sail]{sail_latex_riscv/valzhex_bits_32_forwards_matches6b2dec12f42cfb341d5a812b57201577.tex}}}}

\newcommand{\sailRISCVfnhexBitsThreeTwoForwardsMatches}{\saildoclabelled{sailRISCVfnzhexzybitszy32zyforwardszymatches}{\saildocfn{}{\lstinputlisting[language=sail]{sail_latex_riscv/fnzhex_bits_32_forwards_matches6b2dec12f42cfb341d5a812b57201577.tex}}}}

\newcommand{\sailRISCVvalhexBitsThreeTwoMatchesPrefix}{\saildoclabelled{sailRISCVzhexzybitszy32zymatcheszyprefix}{\saildocval{}{\lstinputlisting[language=sail]{sail_latex_riscv/valzhex_bits_32_matches_prefix164ead7d22300ab7799ffb9205a2b7a2.tex}}}}

\newcommand{\sailRISCVvalhexBitsThreeTwoBackwardsMatches}{\saildoclabelled{sailRISCVzhexzybitszy32zybackwardszymatches}{\saildocval{}{\lstinputlisting[language=sail]{sail_latex_riscv/valzhex_bits_32_backwards_matchesc5862ab9ce54a89fb3d05c1a1e742ba8.tex}}}}

\newcommand{\sailRISCVfnhexBitsThreeTwoBackwardsMatches}{\saildoclabelled{sailRISCVfnzhexzybitszy32zybackwardszymatches}{\saildocfn{}{\lstinputlisting[language=sail]{sail_latex_riscv/fnzhex_bits_32_backwards_matchesc5862ab9ce54a89fb3d05c1a1e742ba8.tex}}}}

\newcommand{\sailRISCVvalhexBitsThreeTwoBackwards}{\saildoclabelled{sailRISCVzhexzybitszy32zybackwards}{\saildocval{}{\lstinputlisting[language=sail]{sail_latex_riscv/valzhex_bits_32_backwardse98f8968c64c820f7d6ab0adad242eea.tex}}}}

\newcommand{\sailRISCVfnhexBitsThreeTwoBackwards}{\saildoclabelled{sailRISCVfnzhexzybitszy32zybackwards}{\saildocfn{}{\lstinputlisting[language=sail]{sail_latex_riscv/fnzhex_bits_32_backwardse98f8968c64c820f7d6ab0adad242eea.tex}}}}

\newcommand{\sailRISCVvalhexBitsThreeThree}{\saildoclabelled{sailRISCVzhexzybitszy33}{\saildocval{}{\lstinputlisting[language=sail]{sail_latex_riscv/valzhex_bits_33bf5a1dfcb81a2b69ce2e63f2b8ac8ba4.tex}}}}

\newcommand{\sailRISCVvalhexBitsThreeThreeForwards}{\saildoclabelled{sailRISCVzhexzybitszy33zyforwards}{\saildocval{}{\lstinputlisting[language=sail]{sail_latex_riscv/valzhex_bits_33_forwards30b0334f0b6652dcc7d0b51e3609d45f.tex}}}}

\newcommand{\sailRISCVvalhexBitsThreeThreeForwardsMatches}{\saildoclabelled{sailRISCVzhexzybitszy33zyforwardszymatches}{\saildocval{}{\lstinputlisting[language=sail]{sail_latex_riscv/valzhex_bits_33_forwards_matches4b4b2c7ba9f6a4dd54548314428e59f6.tex}}}}

\newcommand{\sailRISCVfnhexBitsThreeThreeForwardsMatches}{\saildoclabelled{sailRISCVfnzhexzybitszy33zyforwardszymatches}{\saildocfn{}{\lstinputlisting[language=sail]{sail_latex_riscv/fnzhex_bits_33_forwards_matches4b4b2c7ba9f6a4dd54548314428e59f6.tex}}}}

\newcommand{\sailRISCVvalhexBitsThreeThreeMatchesPrefix}{\saildoclabelled{sailRISCVzhexzybitszy33zymatcheszyprefix}{\saildocval{}{\lstinputlisting[language=sail]{sail_latex_riscv/valzhex_bits_33_matches_prefixfe32236f637b1e2c06fb0442b85daff6.tex}}}}

\newcommand{\sailRISCVvalhexBitsThreeThreeBackwardsMatches}{\saildoclabelled{sailRISCVzhexzybitszy33zybackwardszymatches}{\saildocval{}{\lstinputlisting[language=sail]{sail_latex_riscv/valzhex_bits_33_backwards_matchesa0efa80a7912e2204c7a08b11236c3ea.tex}}}}

\newcommand{\sailRISCVfnhexBitsThreeThreeBackwardsMatches}{\saildoclabelled{sailRISCVfnzhexzybitszy33zybackwardszymatches}{\saildocfn{}{\lstinputlisting[language=sail]{sail_latex_riscv/fnzhex_bits_33_backwards_matchesa0efa80a7912e2204c7a08b11236c3ea.tex}}}}

\newcommand{\sailRISCVvalhexBitsThreeThreeBackwards}{\saildoclabelled{sailRISCVzhexzybitszy33zybackwards}{\saildocval{}{\lstinputlisting[language=sail]{sail_latex_riscv/valzhex_bits_33_backwardscc814f008a05b1129f62004b8fa5c8e5.tex}}}}

\newcommand{\sailRISCVfnhexBitsThreeThreeBackwards}{\saildoclabelled{sailRISCVfnzhexzybitszy33zybackwards}{\saildocfn{}{\lstinputlisting[language=sail]{sail_latex_riscv/fnzhex_bits_33_backwardscc814f008a05b1129f62004b8fa5c8e5.tex}}}}

\newcommand{\sailRISCVvalhexBitsFourEight}{\saildoclabelled{sailRISCVzhexzybitszy48}{\saildocval{}{\lstinputlisting[language=sail]{sail_latex_riscv/valzhex_bits_48bcd68487a0855653b3f1bd0b77874f47.tex}}}}

\newcommand{\sailRISCVvalhexBitsFourEightForwards}{\saildoclabelled{sailRISCVzhexzybitszy48zyforwards}{\saildocval{}{\lstinputlisting[language=sail]{sail_latex_riscv/valzhex_bits_48_forwards00cd6bb42ad747a155b252e7b9981df1.tex}}}}

\newcommand{\sailRISCVvalhexBitsFourEightForwardsMatches}{\saildoclabelled{sailRISCVzhexzybitszy48zyforwardszymatches}{\saildocval{}{\lstinputlisting[language=sail]{sail_latex_riscv/valzhex_bits_48_forwards_matchesf2b53432d5d61a0c6dbc5e249b95d790.tex}}}}

\newcommand{\sailRISCVfnhexBitsFourEightForwardsMatches}{\saildoclabelled{sailRISCVfnzhexzybitszy48zyforwardszymatches}{\saildocfn{}{\lstinputlisting[language=sail]{sail_latex_riscv/fnzhex_bits_48_forwards_matchesf2b53432d5d61a0c6dbc5e249b95d790.tex}}}}

\newcommand{\sailRISCVvalhexBitsFourEightMatchesPrefix}{\saildoclabelled{sailRISCVzhexzybitszy48zymatcheszyprefix}{\saildocval{}{\lstinputlisting[language=sail]{sail_latex_riscv/valzhex_bits_48_matches_prefixcaea63593a204f4e52c03e2824ea453e.tex}}}}

\newcommand{\sailRISCVvalhexBitsFourEightBackwardsMatches}{\saildoclabelled{sailRISCVzhexzybitszy48zybackwardszymatches}{\saildocval{}{\lstinputlisting[language=sail]{sail_latex_riscv/valzhex_bits_48_backwards_matches3765c7a672ed4a4188e6adcad2e1769c.tex}}}}

\newcommand{\sailRISCVfnhexBitsFourEightBackwardsMatches}{\saildoclabelled{sailRISCVfnzhexzybitszy48zybackwardszymatches}{\saildocfn{}{\lstinputlisting[language=sail]{sail_latex_riscv/fnzhex_bits_48_backwards_matches3765c7a672ed4a4188e6adcad2e1769c.tex}}}}

\newcommand{\sailRISCVvalhexBitsFourEightBackwards}{\saildoclabelled{sailRISCVzhexzybitszy48zybackwards}{\saildocval{}{\lstinputlisting[language=sail]{sail_latex_riscv/valzhex_bits_48_backwards57285976d132280592935deb245ad737.tex}}}}

\newcommand{\sailRISCVfnhexBitsFourEightBackwards}{\saildoclabelled{sailRISCVfnzhexzybitszy48zybackwards}{\saildocfn{}{\lstinputlisting[language=sail]{sail_latex_riscv/fnzhex_bits_48_backwards57285976d132280592935deb245ad737.tex}}}}

\newcommand{\sailRISCVvalhexBitsSixFour}{\saildoclabelled{sailRISCVzhexzybitszy64}{\saildocval{}{\lstinputlisting[language=sail]{sail_latex_riscv/valzhex_bits_6481d941ae17c8b68343a58c840ecca1cd.tex}}}}

\newcommand{\sailRISCVvalhexBitsSixFourForwards}{\saildoclabelled{sailRISCVzhexzybitszy64zyforwards}{\saildocval{}{\lstinputlisting[language=sail]{sail_latex_riscv/valzhex_bits_64_forwards7510d79394d351421610c028bea6e7a5.tex}}}}

\newcommand{\sailRISCVvalhexBitsSixFourForwardsMatches}{\saildoclabelled{sailRISCVzhexzybitszy64zyforwardszymatches}{\saildocval{}{\lstinputlisting[language=sail]{sail_latex_riscv/valzhex_bits_64_forwards_matches9b4aa7858f48288bd5ee715f31bb0cdb.tex}}}}

\newcommand{\sailRISCVfnhexBitsSixFourForwardsMatches}{\saildoclabelled{sailRISCVfnzhexzybitszy64zyforwardszymatches}{\saildocfn{}{\lstinputlisting[language=sail]{sail_latex_riscv/fnzhex_bits_64_forwards_matches9b4aa7858f48288bd5ee715f31bb0cdb.tex}}}}

\newcommand{\sailRISCVvalhexBitsSixFourMatchesPrefix}{\saildoclabelled{sailRISCVzhexzybitszy64zymatcheszyprefix}{\saildocval{}{\lstinputlisting[language=sail]{sail_latex_riscv/valzhex_bits_64_matches_prefix72a39a7f0e633af539b7e5e9e8e13575.tex}}}}

\newcommand{\sailRISCVvalhexBitsSixFourBackwardsMatches}{\saildoclabelled{sailRISCVzhexzybitszy64zybackwardszymatches}{\saildocval{}{\lstinputlisting[language=sail]{sail_latex_riscv/valzhex_bits_64_backwards_matches2ac7f320ee670e1403b866f7b48f9b95.tex}}}}

\newcommand{\sailRISCVfnhexBitsSixFourBackwardsMatches}{\saildoclabelled{sailRISCVfnzhexzybitszy64zybackwardszymatches}{\saildocfn{}{\lstinputlisting[language=sail]{sail_latex_riscv/fnzhex_bits_64_backwards_matches2ac7f320ee670e1403b866f7b48f9b95.tex}}}}

\newcommand{\sailRISCVvalhexBitsSixFourBackwards}{\saildoclabelled{sailRISCVzhexzybitszy64zybackwards}{\saildocval{}{\lstinputlisting[language=sail]{sail_latex_riscv/valzhex_bits_64_backwardsef9778c97dfbfdc25f03d089d5788d24.tex}}}}

\newcommand{\sailRISCVfnhexBitsSixFourBackwards}{\saildoclabelled{sailRISCVfnzhexzybitszy64zybackwards}{\saildocfn{}{\lstinputlisting[language=sail]{sail_latex_riscv/fnzhex_bits_64_backwardsef9778c97dfbfdc25f03d089d5788d24.tex}}}}

\newcommand{\sailRISCVtypexlen}{\saildoclabelled{sailRISCVtypezxlen}{\saildoctype{}{\lstinputlisting[language=sail]{sail_latex_riscv/typezxlen69b5c0c212b6eef630aa879251a43faf.tex}}}}

\newcommand{\sailRISCVtypexlenBytes}{\saildoclabelled{sailRISCVtypezxlenzybytes}{\saildoctype{}{\lstinputlisting[language=sail]{sail_latex_riscv/typezxlen_bytesee19340a9dc2d174c74e7e66789a8ce5.tex}}}}

\newcommand{\sailRISCVtypexlenbits}{\saildoclabelled{sailRISCVtypezxlenbits}{\saildoctype{}{\lstinputlisting[language=sail]{sail_latex_riscv/typezxlenbitsf5ff3be63c10995c090bf7946ccc6ec1.tex}}}}

\newcommand{\sailRISCVtypeflen}{\saildoclabelled{sailRISCVtypezflen}{\saildoctype{}{\lstinputlisting[language=sail]{sail_latex_riscv/typezflen27ebfaa38e1d50ee3669425f05ac8951.tex}}}}

\newcommand{\sailRISCVtypeflenBytes}{\saildoclabelled{sailRISCVtypezflenzybytes}{\saildoctype{}{\lstinputlisting[language=sail]{sail_latex_riscv/typezflen_bytesd7ba48b222b9ef1a7ea6335686de6c81.tex}}}}

\newcommand{\sailRISCVtypeflenbits}{\saildoclabelled{sailRISCVtypezflenbits}{\saildoctype{}{\lstinputlisting[language=sail]{sail_latex_riscv/typezflenbits34a10f66673a8dd853aa9dcae8af6905.tex}}}}

\newcommand{\sailRISCVvalMEMrTag}{\saildoclabelled{sailRISCVzMEMrzytag}{\saildocval{}{\lstinputlisting[language=sail]{sail_latex_riscv/valzmemr_tag8062b3657c5bfba8d529ea5c413abc36.tex}}}}

\newcommand{\sailRISCVvalMEMwTag}{\saildoclabelled{sailRISCVzMEMwzytag}{\saildocval{}{\lstinputlisting[language=sail]{sail_latex_riscv/valzmemw_tag0a1745a5ee5b60705e9e4051c5099882.tex}}}}

\newcommand{\sailRISCVvalMAX}{\saildoclabelled{sailRISCVzMAX}{\saildocval{}{\lstinputlisting[language=sail]{sail_latex_riscv/valzmax84a1c708b7c8789c33f72b5bb9ee31e8.tex}}}}

\newcommand{\sailRISCVfnMAX}{\saildoclabelled{sailRISCVfnzMAX}{\saildocfn{}{\lstinputlisting[language=sail]{sail_latex_riscv/fnzmax84a1c708b7c8789c33f72b5bb9ee31e8.tex}}}}

\newcommand{\sailRISCVvalnot}{\saildoclabelled{sailRISCVznot}{\saildocval{}{\lstinputlisting[language=sail]{sail_latex_riscv/valznotcbe861867f25b28c34f5ae99957794ed.tex}}}}

\newcommand{\sailRISCVvalboolToBit}{\saildoclabelled{sailRISCVzboolzytozybit}{\saildocval{}{\lstinputlisting[language=sail]{sail_latex_riscv/valzbool_to_bit5cc99dc0718457cc8a182fa8507f045a.tex}}}}

\newcommand{\sailRISCVfnboolToBit}{\saildoclabelled{sailRISCVfnzboolzytozybit}{\saildocfn{}{\lstinputlisting[language=sail]{sail_latex_riscv/fnzbool_to_bit5cc99dc0718457cc8a182fa8507f045a.tex}}}}

\newcommand{\sailRISCVletreservedOtypes}{\saildoclabelled{sailRISCVletzreservedzyotypes}{\saildoclet{}{\lstinputlisting[language=sail]{sail_latex_riscv/letzreserved_otypesfc67c8a176aa3efad89f5a340aec39ac.tex}}}}

\newcommand{\sailRISCVletotypeUnsealed}{\saildoclabelled{sailRISCVletzotypezyunsealed}{\saildoclet{}{\lstinputlisting[language=sail]{sail_latex_riscv/letzotype_unsealed27aafb55f546c5283a4542e7c4697e4f.tex}}}}

\newcommand{\sailRISCVletotypeSentry}{\saildoclabelled{sailRISCVletzotypezysentry}{\saildoclet{}{\lstinputlisting[language=sail]{sail_latex_riscv/letzotype_sentry3b0a7de15855ff7324219a7bedb11aca.tex}}}}

\newcommand{\sailRISCVtypeCPtrCmpOp}{\saildoclabelled{sailRISCVtypezCPtrCmpOp}{\saildoctype{}{\lstinputlisting[language=sail]{sail_latex_riscv/typezcptrcmpopb3b1dde403387930c5431415f0c993e5.tex}}}}

\newcommand{\sailRISCVvalCPtrCmpOpOfNum}{\saildoclabelled{sailRISCVzCPtrCmpOpzyofzynum}{\saildocval{}{\lstinputlisting[language=sail]{sail_latex_riscv/valzcptrcmpop_of_num73ef06bb0c979dffcf7e6619077debb0.tex}}}}

\newcommand{\sailRISCVfnCPtrCmpOpOfNum}{\saildoclabelled{sailRISCVfnzCPtrCmpOpzyofzynum}{\saildocfn{}{\lstinputlisting[language=sail]{sail_latex_riscv/fnzcptrcmpop_of_num73ef06bb0c979dffcf7e6619077debb0.tex}}}}

\newcommand{\sailRISCVvalnumOfCPtrCmpOp}{\saildoclabelled{sailRISCVznumzyofzyCPtrCmpOp}{\saildocval{}{\lstinputlisting[language=sail]{sail_latex_riscv/valznum_of_cptrcmpop261df9a3b627d5fc110f91fa10e6b254.tex}}}}

\newcommand{\sailRISCVfnnumOfCPtrCmpOp}{\saildoclabelled{sailRISCVfnznumzyofzyCPtrCmpOp}{\saildocfn{}{\lstinputlisting[language=sail]{sail_latex_riscv/fnznum_of_cptrcmpop261df9a3b627d5fc110f91fa10e6b254.tex}}}}

\newcommand{\sailRISCVtypeClearRegSet}{\saildoclabelled{sailRISCVtypezClearRegSet}{\saildoctype{}{\lstinputlisting[language=sail]{sail_latex_riscv/typezclearregsete5eab3e282a2a338ddb8311b5edbb561.tex}}}}

\newcommand{\sailRISCVvalClearRegSetOfNum}{\saildoclabelled{sailRISCVzClearRegSetzyofzynum}{\saildocval{}{\lstinputlisting[language=sail]{sail_latex_riscv/valzclearregset_of_numcd5fbceac9f286632a9dd1aa0eafe241.tex}}}}

\newcommand{\sailRISCVfnClearRegSetOfNum}{\saildoclabelled{sailRISCVfnzClearRegSetzyofzynum}{\saildocfn{}{\lstinputlisting[language=sail]{sail_latex_riscv/fnzclearregset_of_numcd5fbceac9f286632a9dd1aa0eafe241.tex}}}}

\newcommand{\sailRISCVvalnumOfClearRegSet}{\saildoclabelled{sailRISCVznumzyofzyClearRegSet}{\saildocval{}{\lstinputlisting[language=sail]{sail_latex_riscv/valznum_of_clearregset49e10f200544574f819f7f660071e10b.tex}}}}

\newcommand{\sailRISCVfnnumOfClearRegSet}{\saildoclabelled{sailRISCVfnznumzyofzyClearRegSet}{\saildocfn{}{\lstinputlisting[language=sail]{sail_latex_riscv/fnznum_of_clearregset49e10f200544574f819f7f660071e10b.tex}}}}

\newcommand{\sailRISCVtypeCapEx}{\saildoclabelled{sailRISCVtypezCapEx}{\saildoctype{}{\lstinputlisting[language=sail]{sail_latex_riscv/typezcapexbc9797bf8ddff0359677ca6657edba93.tex}}}}

\newcommand{\sailRISCVvalCapExOfNum}{\saildoclabelled{sailRISCVzCapExzyofzynum}{\saildocval{}{\lstinputlisting[language=sail]{sail_latex_riscv/valzcapex_of_num5060d93d1da28509784feb6c153b90e3.tex}}}}

\newcommand{\sailRISCVfnCapExOfNum}{\saildoclabelled{sailRISCVfnzCapExzyofzynum}{\saildocfn{}{\lstinputlisting[language=sail]{sail_latex_riscv/fnzcapex_of_num5060d93d1da28509784feb6c153b90e3.tex}}}}

\newcommand{\sailRISCVvalnumOfCapEx}{\saildoclabelled{sailRISCVznumzyofzyCapEx}{\saildocval{}{\lstinputlisting[language=sail]{sail_latex_riscv/valznum_of_capexa430f3db535161473e26bac337cc3ffe.tex}}}}

\newcommand{\sailRISCVfnnumOfCapEx}{\saildoclabelled{sailRISCVfnznumzyofzyCapEx}{\saildocfn{}{\lstinputlisting[language=sail]{sail_latex_riscv/fnznum_of_capexa430f3db535161473e26bac337cc3ffe.tex}}}}

\newcommand{\sailRISCVvalCapExCode}{\saildoclabelled{sailRISCVzCapExCode}{\saildocval{}{\lstinputlisting[language=sail]{sail_latex_riscv/valzcapexcodee065d40e92bb99703db21c8c18bedf11.tex}}}}

\newcommand{\sailRISCVfnCapExCode}{\saildoclabelled{sailRISCVfnzCapExCode}{\saildocfn{}{\lstinputlisting[language=sail]{sail_latex_riscv/fnzcapexcodee065d40e92bb99703db21c8c18bedf11.tex}}}}

\newcommand{\sailRISCVvalstringOfCapex}{\saildoclabelled{sailRISCVzstringzyofzycapex}{\saildocval{}{\lstinputlisting[language=sail]{sail_latex_riscv/valzstring_of_capexa149bb71f2b82372115d021ece4e6416.tex}}}}

\newcommand{\sailRISCVfnstringOfCapex}{\saildoclabelled{sailRISCVfnzstringzyofzycapex}{\saildocfn{}{\lstinputlisting[language=sail]{sail_latex_riscv/fnzstring_of_capexa149bb71f2b82372115d021ece4e6416.tex}}}}

\newcommand{\sailRISCVtypecapregIdx}{\saildoclabelled{sailRISCVtypezcapregzyidx}{\saildoctype{}{\lstinputlisting[language=sail]{sail_latex_riscv/typezcapreg_idx46cb8ec24174ac7dafd6635c5040002d.tex}}}}

\newcommand{\sailRISCVletPCCIDX}{\saildoclabelled{sailRISCVletzPCCzyIDX}{\saildoclet{}{\lstinputlisting[language=sail]{sail_latex_riscv/letzpcc_idx8d31365769ac040611d33688d2935b81.tex}}}}

\newcommand{\sailRISCVletDDCIDX}{\saildoclabelled{sailRISCVletzDDCzyIDX}{\saildoclet{}{\lstinputlisting[language=sail]{sail_latex_riscv/letzddc_idx84de9edbdd981a44b8619e09ee340e49.tex}}}}

\newcommand{\sailRISCVtypescreg}{\saildoclabelled{sailRISCVtypezscreg}{\saildoctype{}{\lstinputlisting[language=sail]{sail_latex_riscv/typezscreg7666c10c12f9f7ff3cd4397f32a77413.tex}}}}

\newcommand{\sailRISCVtypecapSizze}{\saildoclabelled{sailRISCVtypezcapzysizze}{\saildoctype{}{\lstinputlisting[language=sail]{sail_latex_riscv/typezcap_sizze1bdaf9cd8fe2936d527bdd4bc7dd72ca.tex}}}}

\newcommand{\sailRISCVletcapSizze}{\saildoclabelled{sailRISCVletzcapzysizze}{\saildoclet{}{\lstinputlisting[language=sail]{sail_latex_riscv/letzcap_sizze1bdaf9cd8fe2936d527bdd4bc7dd72ca.tex}}}}

\newcommand{\sailRISCVtypelogTwoCapSizze}{\saildoclabelled{sailRISCVtypezlog2zycapzysizze}{\saildoctype{}{\lstinputlisting[language=sail]{sail_latex_riscv/typezlog2_cap_sizzea1e54166750a789fdd63562b89a8f0d4.tex}}}}

\newcommand{\sailRISCVletlogTwoCapSizze}{\saildoclabelled{sailRISCVletzlog2zycapzysizze}{\saildoclet{}{\lstinputlisting[language=sail]{sail_latex_riscv/letzlog2_cap_sizzea1e54166750a789fdd63562b89a8f0d4.tex}}}}

\newcommand{\sailRISCVtypeCapBits}{\saildoclabelled{sailRISCVtypezCapBits}{\saildoctype{}{\lstinputlisting[language=sail]{sail_latex_riscv/typezcapbits32830b87cefa69a0cdb78ef00d25b781.tex}}}}

\newcommand{\sailRISCVtypecapHpermsWidth}{\saildoclabelled{sailRISCVtypezcapzyhpermszywidth}{\saildoctype{}{\lstinputlisting[language=sail]{sail_latex_riscv/typezcap_hperms_width7a71e6f5d2bdf15a0098752dcdf93274.tex}}}}

\newcommand{\sailRISCVletcapHpermsWidth}{\saildoclabelled{sailRISCVletzcapzyhpermszywidth}{\saildoclet{}{\lstinputlisting[language=sail]{sail_latex_riscv/letzcap_hperms_width7a71e6f5d2bdf15a0098752dcdf93274.tex}}}}

\newcommand{\sailRISCVtypecapUpermsWidth}{\saildoclabelled{sailRISCVtypezcapzyupermszywidth}{\saildoctype{}{\lstinputlisting[language=sail]{sail_latex_riscv/typezcap_uperms_widthf6dfed0942499b0c2d58b90971faca40.tex}}}}

\newcommand{\sailRISCVletcapUpermsWidth}{\saildoclabelled{sailRISCVletzcapzyupermszywidth}{\saildoclet{}{\lstinputlisting[language=sail]{sail_latex_riscv/letzcap_uperms_widthf6dfed0942499b0c2d58b90971faca40.tex}}}}

\newcommand{\sailRISCVtypecapOtypeWidth}{\saildoclabelled{sailRISCVtypezcapzyotypezywidth}{\saildoctype{}{\lstinputlisting[language=sail]{sail_latex_riscv/typezcap_otype_widthcbba33421b0d173c367e3bd232817e06.tex}}}}

\newcommand{\sailRISCVletcapOtypeWidth}{\saildoclabelled{sailRISCVletzcapzyotypezywidth}{\saildoclet{}{\lstinputlisting[language=sail]{sail_latex_riscv/letzcap_otype_widthcbba33421b0d173c367e3bd232817e06.tex}}}}

\newcommand{\sailRISCVtypecapReservedWidth}{\saildoclabelled{sailRISCVtypezcapzyreservedzywidth}{\saildoctype{}{\lstinputlisting[language=sail]{sail_latex_riscv/typezcap_reserved_width162aad08df94c6f762b5521864cec58f.tex}}}}

\newcommand{\sailRISCVletcapReservedWidth}{\saildoclabelled{sailRISCVletzcapzyreservedzywidth}{\saildoclet{}{\lstinputlisting[language=sail]{sail_latex_riscv/letzcap_reserved_width162aad08df94c6f762b5521864cec58f.tex}}}}

\newcommand{\sailRISCVtypecapFlagsWidth}{\saildoclabelled{sailRISCVtypezcapzyflagszywidth}{\saildoctype{}{\lstinputlisting[language=sail]{sail_latex_riscv/typezcap_flags_width335bf4f42772a3e75f4dbee10e086942.tex}}}}

\newcommand{\sailRISCVletcapFlagsWidth}{\saildoclabelled{sailRISCVletzcapzyflagszywidth}{\saildoclet{}{\lstinputlisting[language=sail]{sail_latex_riscv/letzcap_flags_width335bf4f42772a3e75f4dbee10e086942.tex}}}}

\newcommand{\sailRISCVtypecapMantissaWidth}{\saildoclabelled{sailRISCVtypezcapzymantissazywidth}{\saildoctype{}{\lstinputlisting[language=sail]{sail_latex_riscv/typezcap_mantissa_width9664b91a58eac86639a6d61f4c90f63a.tex}}}}

\newcommand{\sailRISCVletcapMantissaWidth}{\saildoclabelled{sailRISCVletzcapzymantissazywidth}{\saildoclet{}{\lstinputlisting[language=sail]{sail_latex_riscv/letzcap_mantissa_width9664b91a58eac86639a6d61f4c90f63a.tex}}}}

\newcommand{\sailRISCVtypecapEWidth}{\saildoclabelled{sailRISCVtypezcapzyEzywidth}{\saildoctype{}{\lstinputlisting[language=sail]{sail_latex_riscv/typezcap_e_width973f9a73b0496551f021cfb45bc16fba.tex}}}}

\newcommand{\sailRISCVletcapEWidth}{\saildoclabelled{sailRISCVletzcapzyEzywidth}{\saildoclet{}{\lstinputlisting[language=sail]{sail_latex_riscv/letzcap_e_width973f9a73b0496551f021cfb45bc16fba.tex}}}}

\newcommand{\sailRISCVtypecapAddrWidth}{\saildoclabelled{sailRISCVtypezcapzyaddrzywidth}{\saildoctype{}{\lstinputlisting[language=sail]{sail_latex_riscv/typezcap_addr_width7d4b25117798a50c5addfe865237309d.tex}}}}

\newcommand{\sailRISCVletcapAddrWidth}{\saildoclabelled{sailRISCVletzcapzyaddrzywidth}{\saildoclet{}{\lstinputlisting[language=sail]{sail_latex_riscv/letzcap_addr_width7d4b25117798a50c5addfe865237309d.tex}}}}

\newcommand{\sailRISCVtypecapLenWidth}{\saildoclabelled{sailRISCVtypezcapzylenzywidth}{\saildoctype{}{\lstinputlisting[language=sail]{sail_latex_riscv/typezcap_len_width9583637a11dff282b382f15c2de94120.tex}}}}

\newcommand{\sailRISCVletcapLenWidth}{\saildoclabelled{sailRISCVletzcapzylenzywidth}{\saildoclet{}{\lstinputlisting[language=sail]{sail_latex_riscv/letzcap_len_width9583637a11dff282b382f15c2de94120.tex}}}}

\newcommand{\sailRISCVtypecapsPerCacheLine}{\saildoclabelled{sailRISCVtypezcapszyperzycachezyline}{\saildoctype{}{\lstinputlisting[language=sail]{sail_latex_riscv/typezcaps_per_cache_line7254e281fd7ea6b3c919ca9a34d729ad.tex}}}}

\newcommand{\sailRISCVletcapsPerCacheLine}{\saildoclabelled{sailRISCVletzcapszyperzycachezyline}{\saildoclet{}{\lstinputlisting[language=sail]{sail_latex_riscv/letzcaps_per_cache_line7254e281fd7ea6b3c919ca9a34d729ad.tex}}}}

\newcommand{\sailRISCVtypeinternalETakeBits}{\saildoclabelled{sailRISCVtypezinternalzyEzytakezybits}{\saildoctype{}{\lstinputlisting[language=sail]{sail_latex_riscv/typezinternal_e_take_bits2053c0fdb4543de9de260a4bcf9ff42e.tex}}}}

\newcommand{\sailRISCVletinternalETakeBits}{\saildoclabelled{sailRISCVletzinternalzyEzytakezybits}{\saildoclet{}{\lstinputlisting[language=sail]{sail_latex_riscv/letzinternal_e_take_bits2053c0fdb4543de9de260a4bcf9ff42e.tex}}}}

\newcommand{\sailRISCVtypeEncCapability}{\saildoclabelled{sailRISCVtypezEncCapability}{\saildoctype{}{\lstinputlisting[language=sail]{sail_latex_riscv/typezenccapability8703f01ba56a7670e501750457ffebed.tex}}}}

\newcommand{\sailRISCVvalcapBitsToEncCapability}{\saildoclabelled{sailRISCVzcapBitsToEncCapability}{\saildocval{}{\lstinputlisting[language=sail]{sail_latex_riscv/valzcapbitstoenccapability376eace6c92fa3a226106f41f4481487.tex}}}}

\newcommand{\sailRISCVfncapBitsToEncCapability}{\saildoclabelled{sailRISCVfnzcapBitsToEncCapability}{\saildocfn{}{\lstinputlisting[language=sail]{sail_latex_riscv/fnzcapbitstoenccapability376eace6c92fa3a226106f41f4481487.tex}}}}

\newcommand{\sailRISCVvalencCapToBits}{\saildoclabelled{sailRISCVzencCapToBits}{\saildocval{}{\lstinputlisting[language=sail]{sail_latex_riscv/valzenccaptobits948ae8b1faab2106f5564feb2065ac65.tex}}}}

\newcommand{\sailRISCVfnencCapToBits}{\saildoclabelled{sailRISCVfnzencCapToBits}{\saildocfn{}{\lstinputlisting[language=sail]{sail_latex_riscv/fnzenccaptobits948ae8b1faab2106f5564feb2065ac65.tex}}}}

\newcommand{\sailRISCVtypememMeta}{\saildoclabelled{sailRISCVtypezmemzymeta}{\saildoctype{}{\lstinputlisting[language=sail]{sail_latex_riscv/typezmem_metaf5b381cf38597ad4a56a53301483f3c2.tex}}}}

\newcommand{\sailRISCVletdefaultMeta}{\saildoclabelled{sailRISCVletzdefaultzymeta}{\saildoclet{}{\lstinputlisting[language=sail]{sail_latex_riscv/letzdefault_metafd00fef0d81d77ebcd9967cf47fd9879.tex}}}}

\newcommand{\sailRISCVtypetagaddrbits}{\saildoclabelled{sailRISCVtypeztagaddrbits}{\saildoctype{}{\lstinputlisting[language=sail]{sail_latex_riscv/typeztagaddrbitsc26f5e6e74dd7ca6d16ebef3163d1a53.tex}}}}

\newcommand{\sailRISCVvaladdrToTagAddr}{\saildoclabelled{sailRISCVzaddrzytozytagzyaddr}{\saildocval{}{\lstinputlisting[language=sail]{sail_latex_riscv/valzaddr_to_tag_addr21cfb55575f4fc24f9ec71484b7d4eb8.tex}}}}

\newcommand{\sailRISCVfnaddrToTagAddr}{\saildoclabelled{sailRISCVfnzaddrzytozytagzyaddr}{\saildocfn{}{\lstinputlisting[language=sail]{sail_latex_riscv/fnzaddr_to_tag_addr21cfb55575f4fc24f9ec71484b7d4eb8.tex}}}}

\newcommand{\sailRISCVvaltagAddrToAddr}{\saildoclabelled{sailRISCVztagzyaddrzytozyaddr}{\saildocval{}{\lstinputlisting[language=sail]{sail_latex_riscv/valztag_addr_to_addrec4cd9758dc545430904849bc06af049.tex}}}}

\newcommand{\sailRISCVfntagAddrToAddr}{\saildoclabelled{sailRISCVfnztagzyaddrzytozyaddr}{\saildocfn{}{\lstinputlisting[language=sail]{sail_latex_riscv/fnztag_addr_to_addrec4cd9758dc545430904849bc06af049.tex}}}}

\newcommand{\sailRISCVvalWriteRAMMeta}{\saildoclabelled{sailRISCVzzyzyWriteRAMzyMeta}{\saildocval{}{\lstinputlisting[language=sail]{sail_latex_riscv/valz__writeram_meta071a60a48b7f4ceb27499d72826fb174.tex}}}}

\newcommand{\sailRISCVfnWriteRAMMeta}{\saildoclabelled{sailRISCVfnzzyzyWriteRAMzyMeta}{\saildocfn{}{\lstinputlisting[language=sail]{sail_latex_riscv/fnz__writeram_meta071a60a48b7f4ceb27499d72826fb174.tex}}}}

\newcommand{\sailRISCVvalReadRAMMeta}{\saildoclabelled{sailRISCVzzyzyReadRAMzyMeta}{\saildocval{}{\lstinputlisting[language=sail]{sail_latex_riscv/valz__readram_meta16c05ad578ee799cab7403aa8924f5dd.tex}}}}

\newcommand{\sailRISCVfnReadRAMMeta}{\saildoclabelled{sailRISCVfnzzyzyReadRAMzyMeta}{\saildocfn{}{\lstinputlisting[language=sail]{sail_latex_riscv/fnz__readram_meta16c05ad578ee799cab7403aa8924f5dd.tex}}}}

\newcommand{\sailRISCVtypemaxMemAccess}{\saildoclabelled{sailRISCVtypezmaxzymemzyaccess}{\saildoctype{}{\lstinputlisting[language=sail]{sail_latex_riscv/typezmax_mem_access644b1e3f30ca973c2c87d72a6996dab7.tex}}}}

\newcommand{\sailRISCVvalwriteRam}{\saildoclabelled{sailRISCVzwritezyram}{\saildocval{}{\lstinputlisting[language=sail]{sail_latex_riscv/valzwrite_ramaf59f53ca3a497b3b8d64cf319996fb8.tex}}}}

\newcommand{\sailRISCVfnwriteRam}{\saildoclabelled{sailRISCVfnzwritezyram}{\saildocfn{}{\lstinputlisting[language=sail]{sail_latex_riscv/fnzwrite_ramaf59f53ca3a497b3b8d64cf319996fb8.tex}}}}

\newcommand{\sailRISCVvalwriteRamEa}{\saildoclabelled{sailRISCVzwritezyramzyea}{\saildocval{}{\lstinputlisting[language=sail]{sail_latex_riscv/valzwrite_ram_ea38ee1d0d3a88b7ca22f44ac1921c34c8.tex}}}}

\newcommand{\sailRISCVfnwriteRamEa}{\saildoclabelled{sailRISCVfnzwritezyramzyea}{\saildocfn{}{\lstinputlisting[language=sail]{sail_latex_riscv/fnzwrite_ram_ea38ee1d0d3a88b7ca22f44ac1921c34c8.tex}}}}

\newcommand{\sailRISCVvalreadRam}{\saildoclabelled{sailRISCVzreadzyram}{\saildocval{}{\lstinputlisting[language=sail]{sail_latex_riscv/valzread_ram020d2ffaf84d982d4588177095c24b8e.tex}}}}

\newcommand{\sailRISCVfnreadRam}{\saildoclabelled{sailRISCVfnzreadzyram}{\saildocfn{}{\lstinputlisting[language=sail]{sail_latex_riscv/fnzread_ram020d2ffaf84d982d4588177095c24b8e.tex}}}}

\newcommand{\sailRISCVvalTraceMemoryWrite}{\saildoclabelled{sailRISCVzzyzyTraceMemoryWrite}{\saildocval{}{\lstinputlisting[language=sail]{sail_latex_riscv/valz__tracememorywrite59b064eac2207f0323d075cfc74a28ea.tex}}}}

\newcommand{\sailRISCVvalTraceMemoryRead}{\saildoclabelled{sailRISCVzzyzyTraceMemoryRead}{\saildocval{}{\lstinputlisting[language=sail]{sail_latex_riscv/valz__tracememoryread11a5e2cc4158cfc2c22e91249b3a83cb.tex}}}}

\newcommand{\sailRISCVletcapMaxAddr}{\saildoclabelled{sailRISCVletzcapzymaxzyaddr}{\saildoclet{}{\lstinputlisting[language=sail]{sail_latex_riscv/letzcap_max_addrb48a2b4f764eb38185432895e980eebf.tex}}}}

\newcommand{\sailRISCVletcapMaxOtype}{\saildoclabelled{sailRISCVletzcapzymaxzyotype}{\saildoclet{}{\lstinputlisting[language=sail]{sail_latex_riscv/letzcap_max_otype1df3f442380c0e0fa0c48eae5affe3e1.tex}}}}

\newcommand{\sailRISCVtypecapUpermsShift}{\saildoclabelled{sailRISCVtypezcapzyupermszyshift}{\saildoctype{}{\lstinputlisting[language=sail]{sail_latex_riscv/typezcap_uperms_shiftbe2a57515ea08a94424ab7ec6686232b.tex}}}}

\newcommand{\sailRISCVletcapUpermsShift}{\saildoclabelled{sailRISCVletzcapzyupermszyshift}{\saildoclet{}{\lstinputlisting[language=sail]{sail_latex_riscv/letzcap_uperms_shiftbe2a57515ea08a94424ab7ec6686232b.tex}}}}

\newcommand{\sailRISCVtypecapPermsWidth}{\saildoclabelled{sailRISCVtypezcapzypermszywidth}{\saildoctype{}{\lstinputlisting[language=sail]{sail_latex_riscv/typezcap_perms_widthf1a97a7ea919ad9f42706f84547f5b1a.tex}}}}

\newcommand{\sailRISCVletcapPermsWidth}{\saildoclabelled{sailRISCVletzcapzypermszywidth}{\saildoclet{}{\lstinputlisting[language=sail]{sail_latex_riscv/letzcap_perms_widthf1a97a7ea919ad9f42706f84547f5b1a.tex}}}}

\newcommand{\sailRISCVtypeCapAddrBits}{\saildoclabelled{sailRISCVtypezCapAddrBits}{\saildoctype{}{\lstinputlisting[language=sail]{sail_latex_riscv/typezcapaddrbitsdbb96675d10b9e9b205f71df6a15d202.tex}}}}

\newcommand{\sailRISCVtypeCapAddrInt}{\saildoclabelled{sailRISCVtypezCapAddrInt}{\saildoctype{}{\lstinputlisting[language=sail]{sail_latex_riscv/typezcapaddrintf3dc84d3e8f46c74b19aac7a9fd4f1f2.tex}}}}

\newcommand{\sailRISCVtypeCapLenBits}{\saildoclabelled{sailRISCVtypezCapLenBits}{\saildoctype{}{\lstinputlisting[language=sail]{sail_latex_riscv/typezcaplenbits0c88c644020f10b2ca144bfbaba3ec9e.tex}}}}

\newcommand{\sailRISCVtypeCapLen}{\saildoclabelled{sailRISCVtypezCapLen}{\saildoctype{}{\lstinputlisting[language=sail]{sail_latex_riscv/typezcaplenf6618af706b03f95ca9741cffae7687d.tex}}}}

\newcommand{\sailRISCVtypeCapPermsBits}{\saildoclabelled{sailRISCVtypezCapPermsBits}{\saildoctype{}{\lstinputlisting[language=sail]{sail_latex_riscv/typezcappermsbits6613dfcf198c86f0eab4b7594494ef02.tex}}}}

\newcommand{\sailRISCVtypeCapFlagsBits}{\saildoclabelled{sailRISCVtypezCapFlagsBits}{\saildoctype{}{\lstinputlisting[language=sail]{sail_latex_riscv/typezcapflagsbitsc0aea8b13e99ce9586aae1d820edcb88.tex}}}}

\newcommand{\sailRISCVletcapMaxE}{\saildoclabelled{sailRISCVletzcapzymaxzyE}{\saildoclet{}{\lstinputlisting[language=sail]{sail_latex_riscv/letzcap_max_e82f87bd5762ef4c31d546d60e860476b.tex}}}}

\newcommand{\sailRISCVletcapResetE}{\saildoclabelled{sailRISCVletzcapzyresetzyE}{\saildoclet{}{\lstinputlisting[language=sail]{sail_latex_riscv/letzcap_reset_e517db3dd6e3b65ffc982098f6ac671bc.tex}}}}

\newcommand{\sailRISCVletcapResetT}{\saildoclabelled{sailRISCVletzcapzyresetzyT}{\saildoclet{}{\lstinputlisting[language=sail]{sail_latex_riscv/letzcap_reset_t56aac6ecd6f934ba784056463b1e3b8b.tex}}}}

\newcommand{\sailRISCVtypeCapability}{\saildoclabelled{sailRISCVtypezCapability}{\saildoctype{A partially decompressed version of a capability -- E, B, T,
lenMSB, sealed and otype fields are not present in all formats and are
populated by capBitsToCapability.

}{\lstinputlisting[language=sail]{sail_latex_riscv/typezcapability5646515621fe4c3bb7fe8874d1909f0e.tex}}}}

\newcommand{\sailRISCVletnullCap}{\saildoclabelled{sailRISCVletznullzycap}{\saildoclet{}{\lstinputlisting[language=sail]{sail_latex_riscv/letznull_cap8d594f9cb0e5a810506a1875526bbdc9.tex}}}}

\newcommand{\sailRISCVletdefaultCap}{\saildoclabelled{sailRISCVletzdefaultzycap}{\saildoclet{}{\lstinputlisting[language=sail]{sail_latex_riscv/letzdefault_cap985ae72689b286ef08b36a5248d34d7e.tex}}}}

\newcommand{\sailRISCVvalgetCapHardPerms}{\saildoclabelled{sailRISCVzgetCapHardPerms}{\saildocval{}{\lstinputlisting[language=sail]{sail_latex_riscv/valzgetcaphardperms801568201da814b3b7b1126c01e5c34b.tex}}}}

\newcommand{\sailRISCVfngetCapHardPerms}{\saildoclabelled{sailRISCVfnzgetCapHardPerms}{\saildocfn{}{\lstinputlisting[language=sail]{sail_latex_riscv/fnzgetcaphardperms801568201da814b3b7b1126c01e5c34b.tex}}}}

\newcommand{\sailRISCVvalencCapabilityToCapability}{\saildoclabelled{sailRISCVzencCapabilityToCapability}{\saildocval{}{\lstinputlisting[language=sail]{sail_latex_riscv/valzenccapabilitytocapability7ba73fdb2e8fe4a2b512fe117732f648.tex}}}}

\newcommand{\sailRISCVfnencCapabilityToCapability}{\saildoclabelled{sailRISCVfnzencCapabilityToCapability}{\saildocfn{}{\lstinputlisting[language=sail]{sail_latex_riscv/fnzenccapabilitytocapability7ba73fdb2e8fe4a2b512fe117732f648.tex}}}}

\newcommand{\sailRISCVvalcapBitsToCapability}{\saildoclabelled{sailRISCVzcapBitsToCapability}{\saildocval{}{\lstinputlisting[language=sail]{sail_latex_riscv/valzcapbitstocapability5a9b90d5c99889a6865e9bf96c63fbdd.tex}}}}

\newcommand{\sailRISCVfncapBitsToCapability}{\saildoclabelled{sailRISCVfnzcapBitsToCapability}{\saildocfn{}{\lstinputlisting[language=sail]{sail_latex_riscv/fnzcapbitstocapability5a9b90d5c99889a6865e9bf96c63fbdd.tex}}}}

\newcommand{\sailRISCVvalcapToEncCap}{\saildoclabelled{sailRISCVzcapToEncCap}{\saildocval{}{\lstinputlisting[language=sail]{sail_latex_riscv/valzcaptoenccapc141a465054fbb8de522a42c4c66faf3.tex}}}}

\newcommand{\sailRISCVfncapToEncCap}{\saildoclabelled{sailRISCVfnzcapToEncCap}{\saildocfn{}{\lstinputlisting[language=sail]{sail_latex_riscv/fnzcaptoenccapc141a465054fbb8de522a42c4c66faf3.tex}}}}

\newcommand{\sailRISCVvalcapToBits}{\saildoclabelled{sailRISCVzcapToBits}{\saildocval{}{\lstinputlisting[language=sail]{sail_latex_riscv/valzcaptobits025010a6e8c284beecc438f72babcc70.tex}}}}

\newcommand{\sailRISCVfncapToBits}{\saildoclabelled{sailRISCVfnzcapToBits}{\saildocfn{}{\lstinputlisting[language=sail]{sail_latex_riscv/fnzcaptobits025010a6e8c284beecc438f72babcc70.tex}}}}

\newcommand{\sailRISCVletnullCapBits}{\saildoclabelled{sailRISCVletznullzycapzybits}{\saildoclet{}{\lstinputlisting[language=sail]{sail_latex_riscv/letznull_cap_bitsf03c34ea7169765fddcb54d4ab2c95c8.tex}}}}

\newcommand{\sailRISCVvalcapToMemBits}{\saildoclabelled{sailRISCVzcapToMemBits}{\saildocval{}{\lstinputlisting[language=sail]{sail_latex_riscv/valzcaptomembitsdd93cf3e1664bb5bed89aa04e4889329.tex}}}}

\newcommand{\sailRISCVfncapToMemBits}{\saildoclabelled{sailRISCVfnzcapToMemBits}{\saildocfn{}{\lstinputlisting[language=sail]{sail_latex_riscv/fnzcaptomembitsdd93cf3e1664bb5bed89aa04e4889329.tex}}}}

\newcommand{\sailRISCVvalmemBitsToCapability}{\saildoclabelled{sailRISCVzmemBitsToCapability}{\saildocval{}{\lstinputlisting[language=sail]{sail_latex_riscv/valzmembitstocapability5eb6ab79951caec58164c1aecfc2f63f.tex}}}}

\newcommand{\sailRISCVfnmemBitsToCapability}{\saildoclabelled{sailRISCVfnzmemBitsToCapability}{\saildocfn{}{\lstinputlisting[language=sail]{sail_latex_riscv/fnzmembitstocapability5eb6ab79951caec58164c1aecfc2f63f.tex}}}}

\newcommand{\sailRISCVvalgetCapBoundsBits}{\saildoclabelled{sailRISCVzgetCapBoundsBits}{\saildocval{}{\lstinputlisting[language=sail]{sail_latex_riscv/valzgetcapboundsbitscff2a996e27fb45794770bad0b82e1fe.tex}}}}

\newcommand{\sailRISCVfngetCapBoundsBits}{\saildoclabelled{sailRISCVfnzgetCapBoundsBits}{\saildocfn{}{\lstinputlisting[language=sail]{sail_latex_riscv/fnzgetcapboundsbitscff2a996e27fb45794770bad0b82e1fe.tex}}}}

\newcommand{\sailRISCVvalgetCapBounds}{\saildoclabelled{sailRISCVzgetCapBounds}{\saildocval{}{\lstinputlisting[language=sail]{sail_latex_riscv/valzgetcapboundsd43bce602e08447feaa9f5135ec44e2f.tex}}}}

\newcommand{\sailRISCVfngetCapBounds}{\saildoclabelled{sailRISCVfnzgetCapBounds}{\saildocfn{}{\lstinputlisting[language=sail]{sail_latex_riscv/fnzgetcapboundsd43bce602e08447feaa9f5135ec44e2f.tex}}}}

\newcommand{\sailRISCVvalsetCapBounds}{\saildoclabelled{sailRISCVzsetCapBounds}{\saildocval{}{\lstinputlisting[language=sail]{sail_latex_riscv/valzsetcapbounds7a50a538fe976a2bfbe0b9f81cc7642e.tex}}}}

\newcommand{\sailRISCVfnsetCapBounds}{\saildoclabelled{sailRISCVfnzsetCapBounds}{\saildocfn{}{\lstinputlisting[language=sail]{sail_latex_riscv/fnzsetcapbounds7a50a538fe976a2bfbe0b9f81cc7642e.tex}}}}

\newcommand{\sailRISCVvalgetCapPerms}{\saildoclabelled{sailRISCVzgetCapPerms}{\saildocval{}{\lstinputlisting[language=sail]{sail_latex_riscv/valzgetcapperms6aed04c1602f540bb5d604425f922d92.tex}}}}

\newcommand{\sailRISCVfngetCapPerms}{\saildoclabelled{sailRISCVfnzgetCapPerms}{\saildocfn{}{\lstinputlisting[language=sail]{sail_latex_riscv/fnzgetcapperms6aed04c1602f540bb5d604425f922d92.tex}}}}

\newcommand{\sailRISCVvalsetCapPerms}{\saildoclabelled{sailRISCVzsetCapPerms}{\saildocval{}{\lstinputlisting[language=sail]{sail_latex_riscv/valzsetcappermsbb03905a9ed7e94e44018326fd80a0d0.tex}}}}

\newcommand{\sailRISCVfnsetCapPerms}{\saildoclabelled{sailRISCVfnzsetCapPerms}{\saildocfn{}{\lstinputlisting[language=sail]{sail_latex_riscv/fnzsetcappermsbb03905a9ed7e94e44018326fd80a0d0.tex}}}}

\newcommand{\sailRISCVvalgetCapFlags}{\saildoclabelled{sailRISCVzgetCapFlags}{\saildocval{Gets the architecture specific capability flags for given capability.

}{\lstinputlisting[language=sail]{sail_latex_riscv/valzgetcapflags06024d55b7e2cd94f99830e3c12d9adf.tex}}}}

\newcommand{\sailRISCVfngetCapFlags}{\saildoclabelled{sailRISCVfnzgetCapFlags}{\saildocfn{}{\lstinputlisting[language=sail]{sail_latex_riscv/fnzgetcapflags06024d55b7e2cd94f99830e3c12d9adf.tex}}}}

\newcommand{\sailRISCVvalsetCapFlags}{\saildoclabelled{sailRISCVzsetCapFlags}{\saildocval{\lstinline{setCapFlags}\lstinline`(cap, flags)` sets the architecture specific capability flags on \lstinline`cap`
to \lstinline`flags` and returns the result as new capability.

}{\lstinputlisting[language=sail]{sail_latex_riscv/valzsetcapflags1cebd5e15eac27fc3dbd3e6dc534158a.tex}}}}

\newcommand{\sailRISCVfnsetCapFlags}{\saildoclabelled{sailRISCVfnzsetCapFlags}{\saildocfn{}{\lstinputlisting[language=sail]{sail_latex_riscv/fnzsetcapflags1cebd5e15eac27fc3dbd3e6dc534158a.tex}}}}

\newcommand{\sailRISCVvalisCapSealed}{\saildoclabelled{sailRISCVzisCapSealed}{\saildocval{}{\lstinputlisting[language=sail]{sail_latex_riscv/valziscapsealeda9077bc28a9d2efcd12e42755a4de536.tex}}}}

\newcommand{\sailRISCVfnisCapSealed}{\saildoclabelled{sailRISCVfnzisCapSealed}{\saildocfn{}{\lstinputlisting[language=sail]{sail_latex_riscv/fnziscapsealeda9077bc28a9d2efcd12e42755a4de536.tex}}}}

\newcommand{\sailRISCVvalhasReservedOType}{\saildoclabelled{sailRISCVzhasReservedOType}{\saildocval{Tests whether the capability has a reserved otype (larger than \hyperref[sailRISCVzcapzymaxzyotype]{\lstinline{cap_max_otype}}).
Note that this includes both sealed (\saildocabbrev{e.g.} sentry) and unsealed (all ones)
otypes.

}{\lstinputlisting[language=sail]{sail_latex_riscv/valzhasreservedotypee1cbb5365f130582a0df82f04b53cb52.tex}}}}

\newcommand{\sailRISCVfnhasReservedOType}{\saildoclabelled{sailRISCVfnzhasReservedOType}{\saildocfn{}{\lstinputlisting[language=sail]{sail_latex_riscv/fnzhasreservedotypee1cbb5365f130582a0df82f04b53cb52.tex}}}}

\newcommand{\sailRISCVvalsealCap}{\saildoclabelled{sailRISCVzsealCap}{\saildocval{}{\lstinputlisting[language=sail]{sail_latex_riscv/valzsealcap2d2c6ffa10772e30f9bf6dea4aba0364.tex}}}}

\newcommand{\sailRISCVfnsealCap}{\saildoclabelled{sailRISCVfnzsealCap}{\saildocfn{}{\lstinputlisting[language=sail]{sail_latex_riscv/fnzsealcap2d2c6ffa10772e30f9bf6dea4aba0364.tex}}}}

\newcommand{\sailRISCVvalunsealCap}{\saildoclabelled{sailRISCVzunsealCap}{\saildocval{}{\lstinputlisting[language=sail]{sail_latex_riscv/valzunsealcap58689ae49a7317c60147327414a678d2.tex}}}}

\newcommand{\sailRISCVfnunsealCap}{\saildoclabelled{sailRISCVfnzunsealCap}{\saildocfn{}{\lstinputlisting[language=sail]{sail_latex_riscv/fnzunsealcap58689ae49a7317c60147327414a678d2.tex}}}}

\newcommand{\sailRISCVvalgetCapBaseBits}{\saildoclabelled{sailRISCVzgetCapBaseBits}{\saildocval{}{\lstinputlisting[language=sail]{sail_latex_riscv/valzgetcapbasebits475f0d31f4d2e3d821fa3069875fc752.tex}}}}

\newcommand{\sailRISCVfngetCapBaseBits}{\saildoclabelled{sailRISCVfnzgetCapBaseBits}{\saildocfn{}{\lstinputlisting[language=sail]{sail_latex_riscv/fnzgetcapbasebits475f0d31f4d2e3d821fa3069875fc752.tex}}}}

\newcommand{\sailRISCVvalgetCapBase}{\saildoclabelled{sailRISCVzgetCapBase}{\saildocval{}{\lstinputlisting[language=sail]{sail_latex_riscv/valzgetcapbase6e35c9ba8902471f4b873925840c53a4.tex}}}}

\newcommand{\sailRISCVfngetCapBase}{\saildoclabelled{sailRISCVfnzgetCapBase}{\saildocfn{}{\lstinputlisting[language=sail]{sail_latex_riscv/fnzgetcapbase6e35c9ba8902471f4b873925840c53a4.tex}}}}

\newcommand{\sailRISCVvalgetCapTopBits}{\saildoclabelled{sailRISCVzgetCapTopBits}{\saildocval{}{\lstinputlisting[language=sail]{sail_latex_riscv/valzgetcaptopbits5ee890f973b03f4c37ec6911afe96449.tex}}}}

\newcommand{\sailRISCVfngetCapTopBits}{\saildoclabelled{sailRISCVfnzgetCapTopBits}{\saildocfn{}{\lstinputlisting[language=sail]{sail_latex_riscv/fnzgetcaptopbits5ee890f973b03f4c37ec6911afe96449.tex}}}}

\newcommand{\sailRISCVvalgetCapTop}{\saildoclabelled{sailRISCVzgetCapTop}{\saildocval{}{\lstinputlisting[language=sail]{sail_latex_riscv/valzgetcaptop94c52bdb95931df50575f8a40f8b9865.tex}}}}

\newcommand{\sailRISCVfngetCapTop}{\saildoclabelled{sailRISCVfnzgetCapTop}{\saildocfn{}{\lstinputlisting[language=sail]{sail_latex_riscv/fnzgetcaptop94c52bdb95931df50575f8a40f8b9865.tex}}}}

\newcommand{\sailRISCVvalgetCapOffsetBits}{\saildoclabelled{sailRISCVzgetCapOffsetBits}{\saildocval{}{\lstinputlisting[language=sail]{sail_latex_riscv/valzgetcapoffsetbits4ee7332151f133f0a8370e45dc0808ca.tex}}}}

\newcommand{\sailRISCVfngetCapOffsetBits}{\saildoclabelled{sailRISCVfnzgetCapOffsetBits}{\saildocfn{}{\lstinputlisting[language=sail]{sail_latex_riscv/fnzgetcapoffsetbits4ee7332151f133f0a8370e45dc0808ca.tex}}}}

\newcommand{\sailRISCVvalgetCapOffset}{\saildoclabelled{sailRISCVzgetCapOffset}{\saildocval{}{\lstinputlisting[language=sail]{sail_latex_riscv/valzgetcapoffset9584da45b9f67a0838c0334ac7a14797.tex}}}}

\newcommand{\sailRISCVfngetCapOffset}{\saildoclabelled{sailRISCVfnzgetCapOffset}{\saildocfn{}{\lstinputlisting[language=sail]{sail_latex_riscv/fnzgetcapoffset9584da45b9f67a0838c0334ac7a14797.tex}}}}

\newcommand{\sailRISCVvalgetCapLength}{\saildoclabelled{sailRISCVzgetCapLength}{\saildocval{}{\lstinputlisting[language=sail]{sail_latex_riscv/valzgetcaplengthe378e6c1b52834806f3d0d380ea18090.tex}}}}

\newcommand{\sailRISCVfngetCapLength}{\saildoclabelled{sailRISCVfnzgetCapLength}{\saildocfn{}{\lstinputlisting[language=sail]{sail_latex_riscv/fnzgetcaplengthe378e6c1b52834806f3d0d380ea18090.tex}}}}

\newcommand{\sailRISCVvalinCapBounds}{\saildoclabelled{sailRISCVzinCapBounds}{\saildocval{}{\lstinputlisting[language=sail]{sail_latex_riscv/valzincapboundsc6e70952d3c6507cf32d75b499b90335.tex}}}}

\newcommand{\sailRISCVfninCapBounds}{\saildoclabelled{sailRISCVfnzinCapBounds}{\saildocfn{}{\lstinputlisting[language=sail]{sail_latex_riscv/fnzincapboundsc6e70952d3c6507cf32d75b499b90335.tex}}}}

\newcommand{\sailRISCVvalgetCapCursor}{\saildoclabelled{sailRISCVzgetCapCursor}{\saildocval{}{\lstinputlisting[language=sail]{sail_latex_riscv/valzgetcapcursord3f4378a821667d421527b6e82da2a0f.tex}}}}

\newcommand{\sailRISCVfngetCapCursor}{\saildoclabelled{sailRISCVfnzgetCapCursor}{\saildocfn{}{\lstinputlisting[language=sail]{sail_latex_riscv/fnzgetcapcursord3f4378a821667d421527b6e82da2a0f.tex}}}}

\newcommand{\sailRISCVvalintToCap}{\saildoclabelled{sailRISCVzintzytozycap}{\saildocval{}{\lstinputlisting[language=sail]{sail_latex_riscv/valzint_to_capf8526cbe276f6bcb8e84d62c62d4f9a1.tex}}}}

\newcommand{\sailRISCVfnintToCap}{\saildoclabelled{sailRISCVfnzintzytozycap}{\saildocfn{}{\lstinputlisting[language=sail]{sail_latex_riscv/fnzint_to_capf8526cbe276f6bcb8e84d62c62d4f9a1.tex}}}}

\newcommand{\sailRISCVvalclearTagIf}{\saildoclabelled{sailRISCVzclearTagIf}{\saildocval{}{\lstinputlisting[language=sail]{sail_latex_riscv/valzcleartagif2b7dd54bd194c70766445bcf562df280.tex}}}}

\newcommand{\sailRISCVfnclearTagIf}{\saildoclabelled{sailRISCVfnzclearTagIf}{\saildocfn{}{\lstinputlisting[language=sail]{sail_latex_riscv/fnzcleartagif2b7dd54bd194c70766445bcf562df280.tex}}}}

\newcommand{\sailRISCVvalclearTagIfSealed}{\saildoclabelled{sailRISCVzclearTagIfSealed}{\saildocval{}{\lstinputlisting[language=sail]{sail_latex_riscv/valzcleartagifsealede07c4106b2ead789cbb4cef06b928bf4.tex}}}}

\newcommand{\sailRISCVfnclearTagIfSealed}{\saildoclabelled{sailRISCVfnzclearTagIfSealed}{\saildocfn{}{\lstinputlisting[language=sail]{sail_latex_riscv/fnzcleartagifsealede07c4106b2ead789cbb4cef06b928bf4.tex}}}}

\newcommand{\sailRISCVvalclearTag}{\saildoclabelled{sailRISCVzclearTag}{\saildocval{}{\lstinputlisting[language=sail]{sail_latex_riscv/valzcleartagaf08859fa57569449d714250191d435a.tex}}}}

\newcommand{\sailRISCVfnclearTag}{\saildoclabelled{sailRISCVfnzclearTag}{\saildocfn{}{\lstinputlisting[language=sail]{sail_latex_riscv/fnzcleartagaf08859fa57569449d714250191d435a.tex}}}}

\newcommand{\sailRISCVvalcapBoundsEqual}{\saildoclabelled{sailRISCVzcapBoundsEqual}{\saildocval{}{\lstinputlisting[language=sail]{sail_latex_riscv/valzcapboundsequal1d03aee0bdb04e1debef609dd2a20edc.tex}}}}

\newcommand{\sailRISCVfncapBoundsEqual}{\saildoclabelled{sailRISCVfnzcapBoundsEqual}{\saildocfn{}{\lstinputlisting[language=sail]{sail_latex_riscv/fnzcapboundsequal1d03aee0bdb04e1debef609dd2a20edc.tex}}}}

\newcommand{\sailRISCVvalsetCapAddr}{\saildoclabelled{sailRISCVzsetCapAddr}{\saildocval{}{\lstinputlisting[language=sail]{sail_latex_riscv/valzsetcapaddr35ad6dc7effb74b141243b59c9daceff.tex}}}}

\newcommand{\sailRISCVfnsetCapAddr}{\saildoclabelled{sailRISCVfnzsetCapAddr}{\saildocfn{}{\lstinputlisting[language=sail]{sail_latex_riscv/fnzsetcapaddr35ad6dc7effb74b141243b59c9daceff.tex}}}}

\newcommand{\sailRISCVoverloadAAAzEightoperatorzZerozKzKzNine}{\saildoclabelled{sailRISCVoverloadAAAzz8operatorz0zKzKz9}{\saildocoverload{}{\lstinputlisting[language=sail]{sail_latex_riscv/overloadAAAzz8operatorz0zkzkz9e772b5e121d0113826739b52dbbce0f8.tex}}}}

\newcommand{\sailRISCVoverloadBBBzEightoperatorzZerozIzIzNine}{\saildoclabelled{sailRISCVoverloadBBBzz8operatorz0zIzIz9}{\saildocoverload{}{\lstinputlisting[language=sail]{sail_latex_riscv/overloadBBBzz8operatorz0ziziz90068ca3610cb726b2dddda4048ca7686.tex}}}}

\newcommand{\sailRISCVoverloadCCCzEightoperatorzZerozKzKSzNine}{\saildoclabelled{sailRISCVoverloadCCCzz8operatorz0zKzKzysz9}{\saildocoverload{}{\lstinputlisting[language=sail]{sail_latex_riscv/overloadCCCzz8operatorz0zkzk_sz9fd336467c8d7c9163cb44b900cb10522.tex}}}}

\newcommand{\sailRISCVvalfastRepCheck}{\saildoclabelled{sailRISCVzfastRepCheck}{\saildocval{}{\lstinputlisting[language=sail]{sail_latex_riscv/valzfastrepcheck592cc92c49a4599da60647f87c331420.tex}}}}

\newcommand{\sailRISCVfnfastRepCheck}{\saildoclabelled{sailRISCVfnzfastRepCheck}{\saildocfn{}{\lstinputlisting[language=sail]{sail_latex_riscv/fnzfastrepcheck592cc92c49a4599da60647f87c331420.tex}}}}

\newcommand{\sailRISCVvalsetCapOffset}{\saildoclabelled{sailRISCVzsetCapOffset}{\saildocval{}{\lstinputlisting[language=sail]{sail_latex_riscv/valzsetcapoffset2da95070f2a3b53b97519f3b1f6a312a.tex}}}}

\newcommand{\sailRISCVfnsetCapOffset}{\saildoclabelled{sailRISCVfnzsetCapOffset}{\saildocfn{}{\lstinputlisting[language=sail]{sail_latex_riscv/fnzsetcapoffset2da95070f2a3b53b97519f3b1f6a312a.tex}}}}

\newcommand{\sailRISCVvalsetCapOffsetChecked}{\saildoclabelled{sailRISCVzsetCapOffsetChecked}{\saildocval{}{\lstinputlisting[language=sail]{sail_latex_riscv/valzsetcapoffsetcheckedd72d1fe7b9d8608e6646350e9965652b.tex}}}}

\newcommand{\sailRISCVfnsetCapOffsetChecked}{\saildoclabelled{sailRISCVfnzsetCapOffsetChecked}{\saildocfn{}{\lstinputlisting[language=sail]{sail_latex_riscv/fnzsetcapoffsetcheckedd72d1fe7b9d8608e6646350e9965652b.tex}}}}

\newcommand{\sailRISCVvalincCapOffset}{\saildoclabelled{sailRISCVzincCapOffset}{\saildocval{}{\lstinputlisting[language=sail]{sail_latex_riscv/valzinccapoffsetc4735d243650e78b90bacb6efb419260.tex}}}}

\newcommand{\sailRISCVfnincCapOffset}{\saildoclabelled{sailRISCVfnzincCapOffset}{\saildocfn{}{\lstinputlisting[language=sail]{sail_latex_riscv/fnzinccapoffsetc4735d243650e78b90bacb6efb419260.tex}}}}

\newcommand{\sailRISCVvalcapToString}{\saildoclabelled{sailRISCVzcapToString}{\saildocval{}{\lstinputlisting[language=sail]{sail_latex_riscv/valzcaptostring7067e2f1e90748309c77a5de3d661e3d.tex}}}}

\newcommand{\sailRISCVfncapToString}{\saildoclabelled{sailRISCVfnzcapToString}{\saildocfn{}{\lstinputlisting[language=sail]{sail_latex_riscv/fnzcaptostring7067e2f1e90748309c77a5de3d661e3d.tex}}}}

\newcommand{\sailRISCVvalgetRepresentableAlignmentMask}{\saildoclabelled{sailRISCVzgetRepresentableAlignmentMask}{\saildocval{}{\lstinputlisting[language=sail]{sail_latex_riscv/valzgetrepresentablealignmentmaskdc533650b5133e10468f8840d3ad2739.tex}}}}

\newcommand{\sailRISCVfngetRepresentableAlignmentMask}{\saildoclabelled{sailRISCVfnzgetRepresentableAlignmentMask}{\saildocfn{}{\lstinputlisting[language=sail]{sail_latex_riscv/fnzgetrepresentablealignmentmaskdc533650b5133e10468f8840d3ad2739.tex}}}}

\newcommand{\sailRISCVvalgetRepresentableLength}{\saildoclabelled{sailRISCVzgetRepresentableLength}{\saildocval{}{\lstinputlisting[language=sail]{sail_latex_riscv/valzgetrepresentablelengthad3bb54ef850e37183b86b40599239a6.tex}}}}

\newcommand{\sailRISCVfngetRepresentableLength}{\saildoclabelled{sailRISCVfnzgetRepresentableLength}{\saildocfn{}{\lstinputlisting[language=sail]{sail_latex_riscv/fnzgetrepresentablelengthad3bb54ef850e37183b86b40599239a6.tex}}}}

\newcommand{\sailRISCVtypeexcCode}{\saildoclabelled{sailRISCVtypezexczycode}{\saildoctype{}{\lstinputlisting[language=sail]{sail_latex_riscv/typezexc_code41fa41b8abb196633a02f07b51e90738.tex}}}}

\newcommand{\sailRISCVtypeextPtwLc}{\saildoclabelled{sailRISCVtypezextzyptwzylc}{\saildoctype{}{\lstinputlisting[language=sail]{sail_latex_riscv/typezext_ptw_lcbba545a858359524aa6ef05819a4c22a.tex}}}}

\newcommand{\sailRISCVvalextPtwLcOfNum}{\saildoclabelled{sailRISCVzextzyptwzylczyofzynum}{\saildocval{}{\lstinputlisting[language=sail]{sail_latex_riscv/valzext_ptw_lc_of_numfefbb86ad247369523da5c5114df83a8.tex}}}}

\newcommand{\sailRISCVfnextPtwLcOfNum}{\saildoclabelled{sailRISCVfnzextzyptwzylczyofzynum}{\saildocfn{}{\lstinputlisting[language=sail]{sail_latex_riscv/fnzext_ptw_lc_of_numfefbb86ad247369523da5c5114df83a8.tex}}}}

\newcommand{\sailRISCVvalnumOfExtPtwLc}{\saildoclabelled{sailRISCVznumzyofzyextzyptwzylc}{\saildocval{}{\lstinputlisting[language=sail]{sail_latex_riscv/valznum_of_ext_ptw_lcfde46f3a7b817078010b5dc3a4f25be7.tex}}}}

\newcommand{\sailRISCVfnnumOfExtPtwLc}{\saildoclabelled{sailRISCVfnznumzyofzyextzyptwzylc}{\saildocfn{}{\lstinputlisting[language=sail]{sail_latex_riscv/fnznum_of_ext_ptw_lcfde46f3a7b817078010b5dc3a4f25be7.tex}}}}

\newcommand{\sailRISCVtypeextPtwSc}{\saildoclabelled{sailRISCVtypezextzyptwzysc}{\saildoctype{}{\lstinputlisting[language=sail]{sail_latex_riscv/typezext_ptw_sccf7d1fb4d0ca02c7aa470f10faacf4b9.tex}}}}

\newcommand{\sailRISCVvalextPtwScOfNum}{\saildoclabelled{sailRISCVzextzyptwzysczyofzynum}{\saildocval{}{\lstinputlisting[language=sail]{sail_latex_riscv/valzext_ptw_sc_of_num2137f22f5407b82d12c8574662600e18.tex}}}}

\newcommand{\sailRISCVfnextPtwScOfNum}{\saildoclabelled{sailRISCVfnzextzyptwzysczyofzynum}{\saildocfn{}{\lstinputlisting[language=sail]{sail_latex_riscv/fnzext_ptw_sc_of_num2137f22f5407b82d12c8574662600e18.tex}}}}

\newcommand{\sailRISCVvalnumOfExtPtwSc}{\saildoclabelled{sailRISCVznumzyofzyextzyptwzysc}{\saildocval{}{\lstinputlisting[language=sail]{sail_latex_riscv/valznum_of_ext_ptw_sc2a22418de077e3289633414d48a30bbe.tex}}}}

\newcommand{\sailRISCVfnnumOfExtPtwSc}{\saildoclabelled{sailRISCVfnznumzyofzyextzyptwzysc}{\saildocfn{}{\lstinputlisting[language=sail]{sail_latex_riscv/fnznum_of_ext_ptw_sc2a22418de077e3289633414d48a30bbe.tex}}}}

\newcommand{\sailRISCVtypeextPtw}{\saildoclabelled{sailRISCVtypezextzyptw}{\saildoctype{}{\lstinputlisting[language=sail]{sail_latex_riscv/typezext_ptwde60d7352876746c4773d53f332b5137.tex}}}}

\newcommand{\sailRISCVvalextPtwLcJoin}{\saildoclabelled{sailRISCVzextzyptwzylczyjoin}{\saildocval{}{\lstinputlisting[language=sail]{sail_latex_riscv/valzext_ptw_lc_joindc735a27ea989374f5eabf813141d337.tex}}}}

\newcommand{\sailRISCVfnextPtwLcJoin}{\saildoclabelled{sailRISCVfnzextzyptwzylczyjoin}{\saildocfn{}{\lstinputlisting[language=sail]{sail_latex_riscv/fnzext_ptw_lc_joindc735a27ea989374f5eabf813141d337.tex}}}}

\newcommand{\sailRISCVvalextPtwScJoin}{\saildoclabelled{sailRISCVzextzyptwzysczyjoin}{\saildocval{}{\lstinputlisting[language=sail]{sail_latex_riscv/valzext_ptw_sc_join88636243b559a2aba809497742e7ce00.tex}}}}

\newcommand{\sailRISCVfnextPtwScJoin}{\saildoclabelled{sailRISCVfnzextzyptwzysczyjoin}{\saildocfn{}{\lstinputlisting[language=sail]{sail_latex_riscv/fnzext_ptw_sc_join88636243b559a2aba809497742e7ce00.tex}}}}

\newcommand{\sailRISCVletinitExtPtw}{\saildoclabelled{sailRISCVletzinitzyextzyptw}{\saildoclet{}{\lstinputlisting[language=sail]{sail_latex_riscv/letzinit_ext_ptw8bc63264e9681db25e45d391c0279a9d.tex}}}}

\newcommand{\sailRISCVtypeextPtwFail}{\saildoclabelled{sailRISCVtypezextzyptwzyfail}{\saildoctype{}{\lstinputlisting[language=sail]{sail_latex_riscv/typezext_ptw_fail2875647ee1ba483d4a452f6a98fef854.tex}}}}

\newcommand{\sailRISCVvalextPtwFailOfNum}{\saildoclabelled{sailRISCVzextzyptwzyfailzyofzynum}{\saildocval{}{\lstinputlisting[language=sail]{sail_latex_riscv/valzext_ptw_fail_of_numcc5cb6ad05de49125f246aa8627fb15b.tex}}}}

\newcommand{\sailRISCVfnextPtwFailOfNum}{\saildoclabelled{sailRISCVfnzextzyptwzyfailzyofzynum}{\saildocfn{}{\lstinputlisting[language=sail]{sail_latex_riscv/fnzext_ptw_fail_of_numcc5cb6ad05de49125f246aa8627fb15b.tex}}}}

\newcommand{\sailRISCVvalnumOfExtPtwFail}{\saildoclabelled{sailRISCVznumzyofzyextzyptwzyfail}{\saildocval{}{\lstinputlisting[language=sail]{sail_latex_riscv/valznum_of_ext_ptw_fail785ab48a0d85159cd76e2d20768c9173.tex}}}}

\newcommand{\sailRISCVfnnumOfExtPtwFail}{\saildoclabelled{sailRISCVfnznumzyofzyextzyptwzyfail}{\saildocfn{}{\lstinputlisting[language=sail]{sail_latex_riscv/fnznum_of_ext_ptw_fail785ab48a0d85159cd76e2d20768c9173.tex}}}}

\newcommand{\sailRISCVtypeextPtwError}{\saildoclabelled{sailRISCVtypezextzyptwzyerror}{\saildoctype{}{\lstinputlisting[language=sail]{sail_latex_riscv/typezext_ptw_error5964337ea2d10baef1f3c2ff9d6e6893.tex}}}}

\newcommand{\sailRISCVvalextPtwErrorOfNum}{\saildoclabelled{sailRISCVzextzyptwzyerrorzyofzynum}{\saildocval{}{\lstinputlisting[language=sail]{sail_latex_riscv/valzext_ptw_error_of_num6c265a5359168121d0772efe3f31cd45.tex}}}}

\newcommand{\sailRISCVfnextPtwErrorOfNum}{\saildoclabelled{sailRISCVfnzextzyptwzyerrorzyofzynum}{\saildocfn{}{\lstinputlisting[language=sail]{sail_latex_riscv/fnzext_ptw_error_of_num6c265a5359168121d0772efe3f31cd45.tex}}}}

\newcommand{\sailRISCVvalnumOfExtPtwError}{\saildoclabelled{sailRISCVznumzyofzyextzyptwzyerror}{\saildocval{}{\lstinputlisting[language=sail]{sail_latex_riscv/valznum_of_ext_ptw_error7daeb8fc5a17e215ccad3f2a24541927.tex}}}}

\newcommand{\sailRISCVfnnumOfExtPtwError}{\saildoclabelled{sailRISCVfnznumzyofzyextzyptwzyerror}{\saildocfn{}{\lstinputlisting[language=sail]{sail_latex_riscv/fnznum_of_ext_ptw_error7daeb8fc5a17e215ccad3f2a24541927.tex}}}}

\newcommand{\sailRISCVtypeextExcType}{\saildoclabelled{sailRISCVtypezextzyexczytype}{\saildoctype{}{\lstinputlisting[language=sail]{sail_latex_riscv/typezext_exc_type73c827d8e97122989947956ad800fcf5.tex}}}}

\newcommand{\sailRISCVvalextExcTypeOfNum}{\saildoclabelled{sailRISCVzextzyexczytypezyofzynum}{\saildocval{}{\lstinputlisting[language=sail]{sail_latex_riscv/valzext_exc_type_of_numa5807bd3284ecfd6e2edf0e304dee26e.tex}}}}

\newcommand{\sailRISCVfnextExcTypeOfNum}{\saildoclabelled{sailRISCVfnzextzyexczytypezyofzynum}{\saildocfn{}{\lstinputlisting[language=sail]{sail_latex_riscv/fnzext_exc_type_of_numa5807bd3284ecfd6e2edf0e304dee26e.tex}}}}

\newcommand{\sailRISCVvalextExcTypeToBits}{\saildoclabelled{sailRISCVzextzyexczytypezytozybits}{\saildocval{}{\lstinputlisting[language=sail]{sail_latex_riscv/valzext_exc_type_to_bitse12334413f8dedca96749d4413e3150b.tex}}}}

\newcommand{\sailRISCVfnextExcTypeToBits}{\saildoclabelled{sailRISCVfnzextzyexczytypezytozybits}{\saildocfn{}{\lstinputlisting[language=sail]{sail_latex_riscv/fnzext_exc_type_to_bitse12334413f8dedca96749d4413e3150b.tex}}}}

\newcommand{\sailRISCVvalnumOfExtExcType}{\saildoclabelled{sailRISCVznumzyofzyextzyexczytype}{\saildocval{}{\lstinputlisting[language=sail]{sail_latex_riscv/valznum_of_ext_exc_type79451fb17925bed3ec8d5058c42d301d.tex}}}}

\newcommand{\sailRISCVfnnumOfExtExcType}{\saildoclabelled{sailRISCVfnznumzyofzyextzyexczytype}{\saildocfn{}{\lstinputlisting[language=sail]{sail_latex_riscv/fnznum_of_ext_exc_type79451fb17925bed3ec8d5058c42d301d.tex}}}}

\newcommand{\sailRISCVvalextExcTypeToStr}{\saildoclabelled{sailRISCVzextzyexczytypezytozystr}{\saildocval{}{\lstinputlisting[language=sail]{sail_latex_riscv/valzext_exc_type_to_strc4b005a75959aeec9fce26f39219c666.tex}}}}

\newcommand{\sailRISCVfnextExcTypeToStr}{\saildoclabelled{sailRISCVfnzextzyexczytypezytozystr}{\saildocfn{}{\lstinputlisting[language=sail]{sail_latex_riscv/fnzext_exc_type_to_strc4b005a75959aeec9fce26f39219c666.tex}}}}

\newcommand{\sailRISCVletxlenVal}{\saildoclabelled{sailRISCVletzxlenzyval}{\saildoclet{}{\lstinputlisting[language=sail]{sail_latex_riscv/letzxlen_valab3f687c3300006f2181c3b8615db172.tex}}}}

\newcommand{\sailRISCVletxlenMaxUnsigned}{\saildoclabelled{sailRISCVletzxlenzymaxzyunsigned}{\saildoclet{}{\lstinputlisting[language=sail]{sail_latex_riscv/letzxlen_max_unsigned8c0ad5d3937664413a64f14a81eb46de.tex}}}}

\newcommand{\sailRISCVletxlenMaxSigned}{\saildoclabelled{sailRISCVletzxlenzymaxzysigned}{\saildoclet{}{\lstinputlisting[language=sail]{sail_latex_riscv/letzxlen_max_signed6131b04fb100d244ea18cfc0904214fe.tex}}}}

\newcommand{\sailRISCVletxlenMinSigned}{\saildoclabelled{sailRISCVletzxlenzyminzysigned}{\saildoclet{}{\lstinputlisting[language=sail]{sail_latex_riscv/letzxlen_min_signed255e3a3cc3b30d2069f3b6f10d475e25.tex}}}}

\newcommand{\sailRISCVtypehalf}{\saildoclabelled{sailRISCVtypezhalf}{\saildoctype{}{\lstinputlisting[language=sail]{sail_latex_riscv/typezhalf6766630885293c014a0c4687f74d88fa.tex}}}}

\newcommand{\sailRISCVtypeword}{\saildoclabelled{sailRISCVtypezword}{\saildoctype{}{\lstinputlisting[language=sail]{sail_latex_riscv/typezword9ee2a7d7876193e4985e692f6ae78f08.tex}}}}

\newcommand{\sailRISCVtyperegidx}{\saildoclabelled{sailRISCVtypezregidx}{\saildoctype{}{\lstinputlisting[language=sail]{sail_latex_riscv/typezregidxb20ed6db135b3db9440370ddc1897517.tex}}}}

\newcommand{\sailRISCVtypecregidx}{\saildoclabelled{sailRISCVtypezcregidx}{\saildoctype{}{\lstinputlisting[language=sail]{sail_latex_riscv/typezcregidx93aa5af192dfa10c34a9c84f0c2ce1a1.tex}}}}

\newcommand{\sailRISCVtypecsreg}{\saildoclabelled{sailRISCVtypezcsreg}{\saildoctype{}{\lstinputlisting[language=sail]{sail_latex_riscv/typezcsreg01b93a287ca075effde1c3cdbc173b4e.tex}}}}

\newcommand{\sailRISCVtyperegno}{\saildoclabelled{sailRISCVtypezregno}{\saildoctype{}{\lstinputlisting[language=sail]{sail_latex_riscv/typezregno1c2cd9b00ff5d991528ec5d711c3ace5.tex}}}}

\newcommand{\sailRISCVvalregidxToRegno}{\saildoclabelled{sailRISCVzregidxzytozyregno}{\saildocval{}{\lstinputlisting[language=sail]{sail_latex_riscv/valzregidx_to_regno46c7ee8e863ba355f36bf094aa680714.tex}}}}

\newcommand{\sailRISCVfnregidxToRegno}{\saildoclabelled{sailRISCVfnzregidxzytozyregno}{\saildocfn{}{\lstinputlisting[language=sail]{sail_latex_riscv/fnzregidx_to_regno46c7ee8e863ba355f36bf094aa680714.tex}}}}

\newcommand{\sailRISCVvalcregTworegIdx}{\saildoclabelled{sailRISCVzcreg2regzyidx}{\saildocval{}{\lstinputlisting[language=sail]{sail_latex_riscv/valzcreg2reg_idxe272195325d960d58ce119619a5a1c75.tex}}}}

\newcommand{\sailRISCVfncregTworegIdx}{\saildoclabelled{sailRISCVfnzcreg2regzyidx}{\saildocfn{}{\lstinputlisting[language=sail]{sail_latex_riscv/fnzcreg2reg_idxe272195325d960d58ce119619a5a1c75.tex}}}}

\newcommand{\sailRISCVletzzreg}{\saildoclabelled{sailRISCVletzzzreg}{\saildoclet{}{\lstinputlisting[language=sail]{sail_latex_riscv/letzzzrege795023b2830c72b44abe6be45e95f69.tex}}}}

\newcommand{\sailRISCVletra}{\saildoclabelled{sailRISCVletzra}{\saildoclet{}{\lstinputlisting[language=sail]{sail_latex_riscv/letzra87484e2842b4fd7656b7a15a91dfb18e.tex}}}}

\newcommand{\sailRISCVletsp}{\saildoclabelled{sailRISCVletzsp}{\saildoclet{}{\lstinputlisting[language=sail]{sail_latex_riscv/letzsp7719b40816742c1e564e7e22863119bb.tex}}}}

\newcommand{\sailRISCVtypeopcode}{\saildoclabelled{sailRISCVtypezopcode}{\saildoctype{}{\lstinputlisting[language=sail]{sail_latex_riscv/typezopcode9a1291b5342e38220790ad44491e6b07.tex}}}}

\newcommand{\sailRISCVtypeimmOneTwo}{\saildoclabelled{sailRISCVtypezimm12}{\saildoctype{}{\lstinputlisting[language=sail]{sail_latex_riscv/typezimm12deef2ca6ef7ab7db2c410afab5972c9d.tex}}}}

\newcommand{\sailRISCVtypeimmTwoZero}{\saildoclabelled{sailRISCVtypezimm20}{\saildoctype{}{\lstinputlisting[language=sail]{sail_latex_riscv/typezimm20575f3ad0e57a29fe015b13b4df3641d3.tex}}}}

\newcommand{\sailRISCVtypeamo}{\saildoclabelled{sailRISCVtypezamo}{\saildoctype{}{\lstinputlisting[language=sail]{sail_latex_riscv/typezamo83bf3a088c1fa727e8019cb088f80417.tex}}}}

\newcommand{\sailRISCVtypeArchitecture}{\saildoclabelled{sailRISCVtypezArchitecture}{\saildoctype{}{\lstinputlisting[language=sail]{sail_latex_riscv/typezarchitecture5b5b6aa873a23d1ef07eac267bb6da07.tex}}}}

\newcommand{\sailRISCVvalArchitectureOfNum}{\saildoclabelled{sailRISCVzArchitecturezyofzynum}{\saildocval{}{\lstinputlisting[language=sail]{sail_latex_riscv/valzarchitecture_of_num798cfe8625bb4fedb9343d4984208b4c.tex}}}}

\newcommand{\sailRISCVfnArchitectureOfNum}{\saildoclabelled{sailRISCVfnzArchitecturezyofzynum}{\saildocfn{}{\lstinputlisting[language=sail]{sail_latex_riscv/fnzarchitecture_of_num798cfe8625bb4fedb9343d4984208b4c.tex}}}}

\newcommand{\sailRISCVvalnumOfArchitecture}{\saildoclabelled{sailRISCVznumzyofzyArchitecture}{\saildocval{}{\lstinputlisting[language=sail]{sail_latex_riscv/valznum_of_architecture1ae1fee7c4b64fc1c08b7336a9784f4a.tex}}}}

\newcommand{\sailRISCVfnnumOfArchitecture}{\saildoclabelled{sailRISCVfnznumzyofzyArchitecture}{\saildocfn{}{\lstinputlisting[language=sail]{sail_latex_riscv/fnznum_of_architecture1ae1fee7c4b64fc1c08b7336a9784f4a.tex}}}}

\newcommand{\sailRISCVtypearchXlen}{\saildoclabelled{sailRISCVtypezarchzyxlen}{\saildoctype{}{\lstinputlisting[language=sail]{sail_latex_riscv/typezarch_xlene776da3b62a5ddb22c2789a29c4f62e0.tex}}}}

\newcommand{\sailRISCVvalarchitecture}{\saildoclabelled{sailRISCVzarchitecture}{\saildocval{}{\lstinputlisting[language=sail]{sail_latex_riscv/valzarchitecture892e99af11725086f28c1c30cefffa9c.tex}}}}

\newcommand{\sailRISCVfnarchitecture}{\saildoclabelled{sailRISCVfnzarchitecture}{\saildocfn{}{\lstinputlisting[language=sail]{sail_latex_riscv/fnzarchitecture892e99af11725086f28c1c30cefffa9c.tex}}}}

\newcommand{\sailRISCVvalarchToBits}{\saildoclabelled{sailRISCVzarchzytozybits}{\saildocval{}{\lstinputlisting[language=sail]{sail_latex_riscv/valzarch_to_bits5b82edd15605cd21ec0f9d9dc63af541.tex}}}}

\newcommand{\sailRISCVfnarchToBits}{\saildoclabelled{sailRISCVfnzarchzytozybits}{\saildocfn{}{\lstinputlisting[language=sail]{sail_latex_riscv/fnzarch_to_bits5b82edd15605cd21ec0f9d9dc63af541.tex}}}}

\newcommand{\sailRISCVtypeprivLevel}{\saildoclabelled{sailRISCVtypezprivzylevel}{\saildoctype{}{\lstinputlisting[language=sail]{sail_latex_riscv/typezpriv_leveld5cdc3b186bdc20eb05333efc3cfd01c.tex}}}}

\newcommand{\sailRISCVtypePrivilege}{\saildoclabelled{sailRISCVtypezPrivilege}{\saildoctype{}{\lstinputlisting[language=sail]{sail_latex_riscv/typezprivilege9997cec8360c7da9f3608ad36bf538cc.tex}}}}

\newcommand{\sailRISCVvalPrivilegeOfNum}{\saildoclabelled{sailRISCVzPrivilegezyofzynum}{\saildocval{}{\lstinputlisting[language=sail]{sail_latex_riscv/valzprivilege_of_num84ee3b92a1b0c896120347634e28615d.tex}}}}

\newcommand{\sailRISCVfnPrivilegeOfNum}{\saildoclabelled{sailRISCVfnzPrivilegezyofzynum}{\saildocfn{}{\lstinputlisting[language=sail]{sail_latex_riscv/fnzprivilege_of_num84ee3b92a1b0c896120347634e28615d.tex}}}}

\newcommand{\sailRISCVvalnumOfPrivilege}{\saildoclabelled{sailRISCVznumzyofzyPrivilege}{\saildocval{}{\lstinputlisting[language=sail]{sail_latex_riscv/valznum_of_privilege40c636ca569f6d4bb59a57bee3b2742d.tex}}}}

\newcommand{\sailRISCVfnnumOfPrivilege}{\saildoclabelled{sailRISCVfnznumzyofzyPrivilege}{\saildocfn{}{\lstinputlisting[language=sail]{sail_latex_riscv/fnznum_of_privilege40c636ca569f6d4bb59a57bee3b2742d.tex}}}}

\newcommand{\sailRISCVvalprivLevelToBits}{\saildoclabelled{sailRISCVzprivLevelzytozybits}{\saildocval{}{\lstinputlisting[language=sail]{sail_latex_riscv/valzprivlevel_to_bits4b6f72dec94db401093759e81957be6b.tex}}}}

\newcommand{\sailRISCVfnprivLevelToBits}{\saildoclabelled{sailRISCVfnzprivLevelzytozybits}{\saildocfn{}{\lstinputlisting[language=sail]{sail_latex_riscv/fnzprivlevel_to_bits4b6f72dec94db401093759e81957be6b.tex}}}}

\newcommand{\sailRISCVvalprivLevelOfBits}{\saildoclabelled{sailRISCVzprivLevelzyofzybits}{\saildocval{}{\lstinputlisting[language=sail]{sail_latex_riscv/valzprivlevel_of_bitsf8754d7aa9d9aeada7d193ecf64e148c.tex}}}}

\newcommand{\sailRISCVfnprivLevelOfBits}{\saildoclabelled{sailRISCVfnzprivLevelzyofzybits}{\saildocfn{}{\lstinputlisting[language=sail]{sail_latex_riscv/fnzprivlevel_of_bitsf8754d7aa9d9aeada7d193ecf64e148c.tex}}}}

\newcommand{\sailRISCVvalprivLevelToStr}{\saildoclabelled{sailRISCVzprivLevelzytozystr}{\saildocval{}{\lstinputlisting[language=sail]{sail_latex_riscv/valzprivlevel_to_str130b731da9dd60ea89c77efcbbe0d598.tex}}}}

\newcommand{\sailRISCVfnprivLevelToStr}{\saildoclabelled{sailRISCVfnzprivLevelzytozystr}{\saildocfn{}{\lstinputlisting[language=sail]{sail_latex_riscv/fnzprivlevel_to_str130b731da9dd60ea89c77efcbbe0d598.tex}}}}

\newcommand{\sailRISCVoverloadDDDtoStr}{\saildoclabelled{sailRISCVoverloadDDDztozystr}{\saildocoverload{}{\lstinputlisting[language=sail]{sail_latex_riscv/overloadDDDzto_str8b7a6895ae35945bd4740e9f790c43ee.tex}}}}

\newcommand{\sailRISCVtypeRetired}{\saildoclabelled{sailRISCVtypezRetired}{\saildoctype{}{\lstinputlisting[language=sail]{sail_latex_riscv/typezretired3cb36a9311620933468743a8b2d2f6f4.tex}}}}

\newcommand{\sailRISCVvalRetiredOfNum}{\saildoclabelled{sailRISCVzRetiredzyofzynum}{\saildocval{}{\lstinputlisting[language=sail]{sail_latex_riscv/valzretired_of_num68ab3748534f762d814246f11fcf7c77.tex}}}}

\newcommand{\sailRISCVfnRetiredOfNum}{\saildoclabelled{sailRISCVfnzRetiredzyofzynum}{\saildocfn{}{\lstinputlisting[language=sail]{sail_latex_riscv/fnzretired_of_num68ab3748534f762d814246f11fcf7c77.tex}}}}

\newcommand{\sailRISCVvalnumOfRetired}{\saildoclabelled{sailRISCVznumzyofzyRetired}{\saildocval{}{\lstinputlisting[language=sail]{sail_latex_riscv/valznum_of_retiredc5322d8e56eb574c7eb3ebf89e0586af.tex}}}}

\newcommand{\sailRISCVfnnumOfRetired}{\saildoclabelled{sailRISCVfnznumzyofzyRetired}{\saildocfn{}{\lstinputlisting[language=sail]{sail_latex_riscv/fnznum_of_retiredc5322d8e56eb574c7eb3ebf89e0586af.tex}}}}

\newcommand{\sailRISCVtypeAccessType}{\saildoclabelled{sailRISCVtypezAccessType}{\saildoctype{}{\lstinputlisting[language=sail]{sail_latex_riscv/typezaccesstype627dc4f8d60f616c352a3659c0dfbd61.tex}}}}

\newcommand{\sailRISCVtypewordWidth}{\saildoclabelled{sailRISCVtypezwordzywidth}{\saildoctype{}{\lstinputlisting[language=sail]{sail_latex_riscv/typezword_width15338b31164a0d031656f7f88e6114c8.tex}}}}

\newcommand{\sailRISCVvalwordWidthOfNum}{\saildoclabelled{sailRISCVzwordzywidthzyofzynum}{\saildocval{}{\lstinputlisting[language=sail]{sail_latex_riscv/valzword_width_of_num5022e9594f19a45eb3d8079a7a770a00.tex}}}}

\newcommand{\sailRISCVfnwordWidthOfNum}{\saildoclabelled{sailRISCVfnzwordzywidthzyofzynum}{\saildocfn{}{\lstinputlisting[language=sail]{sail_latex_riscv/fnzword_width_of_num5022e9594f19a45eb3d8079a7a770a00.tex}}}}

\newcommand{\sailRISCVvalnumOfWordWidth}{\saildoclabelled{sailRISCVznumzyofzywordzywidth}{\saildocval{}{\lstinputlisting[language=sail]{sail_latex_riscv/valznum_of_word_width80798ebf687d8b1ac16aea948967912d.tex}}}}

\newcommand{\sailRISCVfnnumOfWordWidth}{\saildoclabelled{sailRISCVfnznumzyofzywordzywidth}{\saildocfn{}{\lstinputlisting[language=sail]{sail_latex_riscv/fnznum_of_word_width80798ebf687d8b1ac16aea948967912d.tex}}}}

\newcommand{\sailRISCVtypeInterruptType}{\saildoclabelled{sailRISCVtypezInterruptType}{\saildoctype{}{\lstinputlisting[language=sail]{sail_latex_riscv/typezinterrupttypeea26f192039f815bc5d0d5b058b4fac2.tex}}}}

\newcommand{\sailRISCVvalInterruptTypeOfNum}{\saildoclabelled{sailRISCVzInterruptTypezyofzynum}{\saildocval{}{\lstinputlisting[language=sail]{sail_latex_riscv/valzinterrupttype_of_numbcca70e199dd805ea962d03dd661ceb8.tex}}}}

\newcommand{\sailRISCVfnInterruptTypeOfNum}{\saildoclabelled{sailRISCVfnzInterruptTypezyofzynum}{\saildocfn{}{\lstinputlisting[language=sail]{sail_latex_riscv/fnzinterrupttype_of_numbcca70e199dd805ea962d03dd661ceb8.tex}}}}

\newcommand{\sailRISCVvalnumOfInterruptType}{\saildoclabelled{sailRISCVznumzyofzyInterruptType}{\saildocval{}{\lstinputlisting[language=sail]{sail_latex_riscv/valznum_of_interrupttype186751debed5f5a4e5f875749623071b.tex}}}}

\newcommand{\sailRISCVfnnumOfInterruptType}{\saildoclabelled{sailRISCVfnznumzyofzyInterruptType}{\saildocfn{}{\lstinputlisting[language=sail]{sail_latex_riscv/fnznum_of_interrupttype186751debed5f5a4e5f875749623071b.tex}}}}

\newcommand{\sailRISCVvalinterruptTypeToBits}{\saildoclabelled{sailRISCVzinterruptTypezytozybits}{\saildocval{}{\lstinputlisting[language=sail]{sail_latex_riscv/valzinterrupttype_to_bits80d6193c8205e91dcda1e3a5b6126a81.tex}}}}

\newcommand{\sailRISCVfninterruptTypeToBits}{\saildoclabelled{sailRISCVfnzinterruptTypezytozybits}{\saildocfn{}{\lstinputlisting[language=sail]{sail_latex_riscv/fnzinterrupttype_to_bits80d6193c8205e91dcda1e3a5b6126a81.tex}}}}

\newcommand{\sailRISCVtypeExceptionType}{\saildoclabelled{sailRISCVtypezExceptionType}{\saildoctype{}{\lstinputlisting[language=sail]{sail_latex_riscv/typezexceptiontype8574cf8632bbc8e423ccb2d23e61bdff.tex}}}}

\newcommand{\sailRISCVvalexceptionTypeToBits}{\saildoclabelled{sailRISCVzexceptionTypezytozybits}{\saildocval{}{\lstinputlisting[language=sail]{sail_latex_riscv/valzexceptiontype_to_bits6edc76877c2677590096c351d9b431f1.tex}}}}

\newcommand{\sailRISCVfnexceptionTypeToBits}{\saildoclabelled{sailRISCVfnzexceptionTypezytozybits}{\saildocfn{}{\lstinputlisting[language=sail]{sail_latex_riscv/fnzexceptiontype_to_bits6edc76877c2677590096c351d9b431f1.tex}}}}

\newcommand{\sailRISCVvalnumOfExceptionType}{\saildoclabelled{sailRISCVznumzyofzyExceptionType}{\saildocval{}{\lstinputlisting[language=sail]{sail_latex_riscv/valznum_of_exceptiontype13e59fd83201d81140ba0f6bbcbd1a7b.tex}}}}

\newcommand{\sailRISCVfnnumOfExceptionType}{\saildoclabelled{sailRISCVfnznumzyofzyExceptionType}{\saildocfn{}{\lstinputlisting[language=sail]{sail_latex_riscv/fnznum_of_exceptiontype13e59fd83201d81140ba0f6bbcbd1a7b.tex}}}}

\newcommand{\sailRISCVvalexceptionTypeToStr}{\saildoclabelled{sailRISCVzexceptionTypezytozystr}{\saildocval{}{\lstinputlisting[language=sail]{sail_latex_riscv/valzexceptiontype_to_str566b70f16fdf6ed4d1850ec75465ec4b.tex}}}}

\newcommand{\sailRISCVfnexceptionTypeToStr}{\saildoclabelled{sailRISCVfnzexceptionTypezytozystr}{\saildocfn{}{\lstinputlisting[language=sail]{sail_latex_riscv/fnzexceptiontype_to_str566b70f16fdf6ed4d1850ec75465ec4b.tex}}}}

\newcommand{\sailRISCVoverloadEEEtoStr}{\saildoclabelled{sailRISCVoverloadEEEztozystr}{\saildocoverload{}{\lstinputlisting[language=sail]{sail_latex_riscv/overloadEEEzto_str8b7a6895ae35945bd4740e9f790c43ee.tex}}}}

\newcommand{\sailRISCVtypeexception}{\saildoclabelled{sailRISCVtypezexception}{\saildoctype{}{\lstinputlisting[language=sail]{sail_latex_riscv/typezexceptionfaa4db8fab65c538edad4222e766a71a.tex}}}}

\newcommand{\sailRISCVvalnotImplemented}{\saildoclabelled{sailRISCVznotzyimplemented}{\saildocval{}{\lstinputlisting[language=sail]{sail_latex_riscv/valznot_implementedde41164205ef21773733c511f05a2946.tex}}}}

\newcommand{\sailRISCVfnnotImplemented}{\saildoclabelled{sailRISCVfnznotzyimplemented}{\saildocfn{}{\lstinputlisting[language=sail]{sail_latex_riscv/fnznot_implementedde41164205ef21773733c511f05a2946.tex}}}}

\newcommand{\sailRISCVvalinternalError}{\saildoclabelled{sailRISCVzinternalzyerror}{\saildocval{}{\lstinputlisting[language=sail]{sail_latex_riscv/valzinternal_error92c3548c19282819d20d44565303aa89.tex}}}}

\newcommand{\sailRISCVfninternalError}{\saildoclabelled{sailRISCVfnzinternalzyerror}{\saildocfn{}{\lstinputlisting[language=sail]{sail_latex_riscv/fnzinternal_error92c3548c19282819d20d44565303aa89.tex}}}}

\newcommand{\sailRISCVtypetvMode}{\saildoclabelled{sailRISCVtypeztvzymode}{\saildoctype{}{\lstinputlisting[language=sail]{sail_latex_riscv/typeztv_modea724d5138c36f1ab6005c9051bd94ea0.tex}}}}

\newcommand{\sailRISCVtypeTrapVectorMode}{\saildoclabelled{sailRISCVtypezTrapVectorMode}{\saildoctype{}{\lstinputlisting[language=sail]{sail_latex_riscv/typeztrapvectormodecc7f1b98e2e9e9062e5f9a94be7a00dd.tex}}}}

\newcommand{\sailRISCVvalTrapVectorModeOfNum}{\saildoclabelled{sailRISCVzTrapVectorModezyofzynum}{\saildocval{}{\lstinputlisting[language=sail]{sail_latex_riscv/valztrapvectormode_of_num8f2938d16cb187b62f9cdbbb35278d48.tex}}}}

\newcommand{\sailRISCVfnTrapVectorModeOfNum}{\saildoclabelled{sailRISCVfnzTrapVectorModezyofzynum}{\saildocfn{}{\lstinputlisting[language=sail]{sail_latex_riscv/fnztrapvectormode_of_num8f2938d16cb187b62f9cdbbb35278d48.tex}}}}

\newcommand{\sailRISCVvalnumOfTrapVectorMode}{\saildoclabelled{sailRISCVznumzyofzyTrapVectorMode}{\saildocval{}{\lstinputlisting[language=sail]{sail_latex_riscv/valznum_of_trapvectormode9673c0ba1b150ae7d0c789bf1ea8f4fe.tex}}}}

\newcommand{\sailRISCVfnnumOfTrapVectorMode}{\saildoclabelled{sailRISCVfnznumzyofzyTrapVectorMode}{\saildocfn{}{\lstinputlisting[language=sail]{sail_latex_riscv/fnznum_of_trapvectormode9673c0ba1b150ae7d0c789bf1ea8f4fe.tex}}}}

\newcommand{\sailRISCVvaltrapVectorModeOfBits}{\saildoclabelled{sailRISCVztrapVectorModezyofzybits}{\saildocval{}{\lstinputlisting[language=sail]{sail_latex_riscv/valztrapvectormode_of_bits3ab887814163c96ab28aec41cb9b44f6.tex}}}}

\newcommand{\sailRISCVfntrapVectorModeOfBits}{\saildoclabelled{sailRISCVfnztrapVectorModezyofzybits}{\saildocfn{}{\lstinputlisting[language=sail]{sail_latex_riscv/fnztrapvectormode_of_bits3ab887814163c96ab28aec41cb9b44f6.tex}}}}

\newcommand{\sailRISCVtypeextStatus}{\saildoclabelled{sailRISCVtypezextzystatus}{\saildoctype{}{\lstinputlisting[language=sail]{sail_latex_riscv/typezext_statuscc754716d0653fa6ecbe317b319fb0e7.tex}}}}

\newcommand{\sailRISCVtypeExtStatus}{\saildoclabelled{sailRISCVtypezExtStatus}{\saildoctype{}{\lstinputlisting[language=sail]{sail_latex_riscv/typezextstatus48f045aa452fc429aaaaf8df3f7b4a71.tex}}}}

\newcommand{\sailRISCVvalExtStatusOfNum}{\saildoclabelled{sailRISCVzExtStatuszyofzynum}{\saildocval{}{\lstinputlisting[language=sail]{sail_latex_riscv/valzextstatus_of_num31bfbda5f90ad0f3cdbc5a0f1b63da2d.tex}}}}

\newcommand{\sailRISCVfnExtStatusOfNum}{\saildoclabelled{sailRISCVfnzExtStatuszyofzynum}{\saildocfn{}{\lstinputlisting[language=sail]{sail_latex_riscv/fnzextstatus_of_num31bfbda5f90ad0f3cdbc5a0f1b63da2d.tex}}}}

\newcommand{\sailRISCVvalnumOfExtStatus}{\saildoclabelled{sailRISCVznumzyofzyExtStatus}{\saildocval{}{\lstinputlisting[language=sail]{sail_latex_riscv/valznum_of_extstatus8d29ae3139c8c2d5d4fa7489689b6a41.tex}}}}

\newcommand{\sailRISCVfnnumOfExtStatus}{\saildoclabelled{sailRISCVfnznumzyofzyExtStatus}{\saildocfn{}{\lstinputlisting[language=sail]{sail_latex_riscv/fnznum_of_extstatus8d29ae3139c8c2d5d4fa7489689b6a41.tex}}}}

\newcommand{\sailRISCVvalextStatusToBits}{\saildoclabelled{sailRISCVzextStatuszytozybits}{\saildocval{}{\lstinputlisting[language=sail]{sail_latex_riscv/valzextstatus_to_bits95f7e744a463eb9cfccacc81efa252ae.tex}}}}

\newcommand{\sailRISCVfnextStatusToBits}{\saildoclabelled{sailRISCVfnzextStatuszytozybits}{\saildocfn{}{\lstinputlisting[language=sail]{sail_latex_riscv/fnzextstatus_to_bits95f7e744a463eb9cfccacc81efa252ae.tex}}}}

\newcommand{\sailRISCVvalextStatusOfBits}{\saildoclabelled{sailRISCVzextStatuszyofzybits}{\saildocval{}{\lstinputlisting[language=sail]{sail_latex_riscv/valzextstatus_of_bits9da79344bfec7cda3e374a5ba6b49e27.tex}}}}

\newcommand{\sailRISCVfnextStatusOfBits}{\saildoclabelled{sailRISCVfnzextStatuszyofzybits}{\saildocfn{}{\lstinputlisting[language=sail]{sail_latex_riscv/fnzextstatus_of_bits9da79344bfec7cda3e374a5ba6b49e27.tex}}}}

\newcommand{\sailRISCVtypesatpMode}{\saildoclabelled{sailRISCVtypezsatpzymode}{\saildoctype{}{\lstinputlisting[language=sail]{sail_latex_riscv/typezsatp_mode578b50fd69503e82abb4182613570ec9.tex}}}}

\newcommand{\sailRISCVtypeSATPMode}{\saildoclabelled{sailRISCVtypezSATPMode}{\saildoctype{}{\lstinputlisting[language=sail]{sail_latex_riscv/typezsatpmodedc53ce733006f254af9d4e33c8ed6401.tex}}}}

\newcommand{\sailRISCVvalSATPModeOfNum}{\saildoclabelled{sailRISCVzSATPModezyofzynum}{\saildocval{}{\lstinputlisting[language=sail]{sail_latex_riscv/valzsatpmode_of_num09ad57622dbe0d0a7b111194f1a36856.tex}}}}

\newcommand{\sailRISCVfnSATPModeOfNum}{\saildoclabelled{sailRISCVfnzSATPModezyofzynum}{\saildocfn{}{\lstinputlisting[language=sail]{sail_latex_riscv/fnzsatpmode_of_num09ad57622dbe0d0a7b111194f1a36856.tex}}}}

\newcommand{\sailRISCVvalnumOfSATPMode}{\saildoclabelled{sailRISCVznumzyofzySATPMode}{\saildocval{}{\lstinputlisting[language=sail]{sail_latex_riscv/valznum_of_satpmode714998a67cc48d1f59bc52de3d9a052f.tex}}}}

\newcommand{\sailRISCVfnnumOfSATPMode}{\saildoclabelled{sailRISCVfnznumzyofzySATPMode}{\saildocfn{}{\lstinputlisting[language=sail]{sail_latex_riscv/fnznum_of_satpmode714998a67cc48d1f59bc52de3d9a052f.tex}}}}

\newcommand{\sailRISCVvalsatpSixFourModeOfBits}{\saildoclabelled{sailRISCVzsatp64Modezyofzybits}{\saildocval{}{\lstinputlisting[language=sail]{sail_latex_riscv/valzsatp64mode_of_bits11c1e0e3eda64d7361d8ec4950c9f1ea.tex}}}}

\newcommand{\sailRISCVfnsatpSixFourModeOfBits}{\saildoclabelled{sailRISCVfnzsatp64Modezyofzybits}{\saildocfn{}{\lstinputlisting[language=sail]{sail_latex_riscv/fnzsatp64mode_of_bits11c1e0e3eda64d7361d8ec4950c9f1ea.tex}}}}

\newcommand{\sailRISCVtypecsrRW}{\saildoclabelled{sailRISCVtypezcsrRW}{\saildoctype{}{\lstinputlisting[language=sail]{sail_latex_riscv/typezcsrrwa8d2be180025e67dbbe56d3c5c45b12e.tex}}}}

\newcommand{\sailRISCVtypeuop}{\saildoclabelled{sailRISCVtypezuop}{\saildoctype{}{\lstinputlisting[language=sail]{sail_latex_riscv/typezuopd0b3f6267a24e3937a9850ce9ebaa872.tex}}}}

\newcommand{\sailRISCVvaluopOfNum}{\saildoclabelled{sailRISCVzuopzyofzynum}{\saildocval{}{\lstinputlisting[language=sail]{sail_latex_riscv/valzuop_of_num86c5f7f375d0126822beff686eb42370.tex}}}}

\newcommand{\sailRISCVfnuopOfNum}{\saildoclabelled{sailRISCVfnzuopzyofzynum}{\saildocfn{}{\lstinputlisting[language=sail]{sail_latex_riscv/fnzuop_of_num86c5f7f375d0126822beff686eb42370.tex}}}}

\newcommand{\sailRISCVvalnumOfUop}{\saildoclabelled{sailRISCVznumzyofzyuop}{\saildocval{}{\lstinputlisting[language=sail]{sail_latex_riscv/valznum_of_uop5155bc57344db1e0852d69264b07f354.tex}}}}

\newcommand{\sailRISCVfnnumOfUop}{\saildoclabelled{sailRISCVfnznumzyofzyuop}{\saildocfn{}{\lstinputlisting[language=sail]{sail_latex_riscv/fnznum_of_uop5155bc57344db1e0852d69264b07f354.tex}}}}

\newcommand{\sailRISCVtypebop}{\saildoclabelled{sailRISCVtypezbop}{\saildoctype{}{\lstinputlisting[language=sail]{sail_latex_riscv/typezbop5369d06602ae03217d614193dd3f84fa.tex}}}}

\newcommand{\sailRISCVvalbopOfNum}{\saildoclabelled{sailRISCVzbopzyofzynum}{\saildocval{}{\lstinputlisting[language=sail]{sail_latex_riscv/valzbop_of_num74cd9479863c5a21d66fa86ae45f1bd5.tex}}}}

\newcommand{\sailRISCVfnbopOfNum}{\saildoclabelled{sailRISCVfnzbopzyofzynum}{\saildocfn{}{\lstinputlisting[language=sail]{sail_latex_riscv/fnzbop_of_num74cd9479863c5a21d66fa86ae45f1bd5.tex}}}}

\newcommand{\sailRISCVvalnumOfBop}{\saildoclabelled{sailRISCVznumzyofzybop}{\saildocval{}{\lstinputlisting[language=sail]{sail_latex_riscv/valznum_of_bop465de2d51df014a233592af62fc5056d.tex}}}}

\newcommand{\sailRISCVfnnumOfBop}{\saildoclabelled{sailRISCVfnznumzyofzybop}{\saildocfn{}{\lstinputlisting[language=sail]{sail_latex_riscv/fnznum_of_bop465de2d51df014a233592af62fc5056d.tex}}}}

\newcommand{\sailRISCVtypeiop}{\saildoclabelled{sailRISCVtypeziop}{\saildoctype{}{\lstinputlisting[language=sail]{sail_latex_riscv/typeziop4f047e83f4bb42a95f028529f28b2b82.tex}}}}

\newcommand{\sailRISCVvaliopOfNum}{\saildoclabelled{sailRISCVziopzyofzynum}{\saildocval{}{\lstinputlisting[language=sail]{sail_latex_riscv/valziop_of_numd466c8622bc5d10ff829fe51ba16e9a6.tex}}}}

\newcommand{\sailRISCVfniopOfNum}{\saildoclabelled{sailRISCVfnziopzyofzynum}{\saildocfn{}{\lstinputlisting[language=sail]{sail_latex_riscv/fnziop_of_numd466c8622bc5d10ff829fe51ba16e9a6.tex}}}}

\newcommand{\sailRISCVvalnumOfIop}{\saildoclabelled{sailRISCVznumzyofzyiop}{\saildocval{}{\lstinputlisting[language=sail]{sail_latex_riscv/valznum_of_iop7e0f948724eaec1edf1ab6539e332d14.tex}}}}

\newcommand{\sailRISCVfnnumOfIop}{\saildoclabelled{sailRISCVfnznumzyofzyiop}{\saildocfn{}{\lstinputlisting[language=sail]{sail_latex_riscv/fnznum_of_iop7e0f948724eaec1edf1ab6539e332d14.tex}}}}

\newcommand{\sailRISCVtypesop}{\saildoclabelled{sailRISCVtypezsop}{\saildoctype{}{\lstinputlisting[language=sail]{sail_latex_riscv/typezsop278747973081fc221bd6ffe68d2fa910.tex}}}}

\newcommand{\sailRISCVvalsopOfNum}{\saildoclabelled{sailRISCVzsopzyofzynum}{\saildocval{}{\lstinputlisting[language=sail]{sail_latex_riscv/valzsop_of_num788240d3b5d5ef8334c5920b24c291e9.tex}}}}

\newcommand{\sailRISCVfnsopOfNum}{\saildoclabelled{sailRISCVfnzsopzyofzynum}{\saildocfn{}{\lstinputlisting[language=sail]{sail_latex_riscv/fnzsop_of_num788240d3b5d5ef8334c5920b24c291e9.tex}}}}

\newcommand{\sailRISCVvalnumOfSop}{\saildoclabelled{sailRISCVznumzyofzysop}{\saildocval{}{\lstinputlisting[language=sail]{sail_latex_riscv/valznum_of_sopfa04a24d46338146566ae6e8a80132f0.tex}}}}

\newcommand{\sailRISCVfnnumOfSop}{\saildoclabelled{sailRISCVfnznumzyofzysop}{\saildocfn{}{\lstinputlisting[language=sail]{sail_latex_riscv/fnznum_of_sopfa04a24d46338146566ae6e8a80132f0.tex}}}}

\newcommand{\sailRISCVtyperop}{\saildoclabelled{sailRISCVtypezrop}{\saildoctype{}{\lstinputlisting[language=sail]{sail_latex_riscv/typezrop291ca9b0f81265d59b554cd7976da946.tex}}}}

\newcommand{\sailRISCVvalropOfNum}{\saildoclabelled{sailRISCVzropzyofzynum}{\saildocval{}{\lstinputlisting[language=sail]{sail_latex_riscv/valzrop_of_numdb49159dd280dafb7370c6477b545c05.tex}}}}

\newcommand{\sailRISCVfnropOfNum}{\saildoclabelled{sailRISCVfnzropzyofzynum}{\saildocfn{}{\lstinputlisting[language=sail]{sail_latex_riscv/fnzrop_of_numdb49159dd280dafb7370c6477b545c05.tex}}}}

\newcommand{\sailRISCVvalnumOfRop}{\saildoclabelled{sailRISCVznumzyofzyrop}{\saildocval{}{\lstinputlisting[language=sail]{sail_latex_riscv/valznum_of_rop6b1530298b7a57e62b47f86bb5f1b15c.tex}}}}

\newcommand{\sailRISCVfnnumOfRop}{\saildoclabelled{sailRISCVfnznumzyofzyrop}{\saildocfn{}{\lstinputlisting[language=sail]{sail_latex_riscv/fnznum_of_rop6b1530298b7a57e62b47f86bb5f1b15c.tex}}}}

\newcommand{\sailRISCVtyperopw}{\saildoclabelled{sailRISCVtypezropw}{\saildoctype{}{\lstinputlisting[language=sail]{sail_latex_riscv/typezropw43e7b9e2c8f71c945acf86b7ec6e0687.tex}}}}

\newcommand{\sailRISCVvalropwOfNum}{\saildoclabelled{sailRISCVzropwzyofzynum}{\saildocval{}{\lstinputlisting[language=sail]{sail_latex_riscv/valzropw_of_numbc49d41e4663ce1e2313189dca74c7f1.tex}}}}

\newcommand{\sailRISCVfnropwOfNum}{\saildoclabelled{sailRISCVfnzropwzyofzynum}{\saildocfn{}{\lstinputlisting[language=sail]{sail_latex_riscv/fnzropw_of_numbc49d41e4663ce1e2313189dca74c7f1.tex}}}}

\newcommand{\sailRISCVvalnumOfRopw}{\saildoclabelled{sailRISCVznumzyofzyropw}{\saildocval{}{\lstinputlisting[language=sail]{sail_latex_riscv/valznum_of_ropw2d1d1b64d2060822876c1a3c1d164870.tex}}}}

\newcommand{\sailRISCVfnnumOfRopw}{\saildoclabelled{sailRISCVfnznumzyofzyropw}{\saildocfn{}{\lstinputlisting[language=sail]{sail_latex_riscv/fnznum_of_ropw2d1d1b64d2060822876c1a3c1d164870.tex}}}}

\newcommand{\sailRISCVtypesopw}{\saildoclabelled{sailRISCVtypezsopw}{\saildoctype{}{\lstinputlisting[language=sail]{sail_latex_riscv/typezsopw29bbaf9eb4401b2d853623bfd0698c6e.tex}}}}

\newcommand{\sailRISCVvalsopwOfNum}{\saildoclabelled{sailRISCVzsopwzyofzynum}{\saildocval{}{\lstinputlisting[language=sail]{sail_latex_riscv/valzsopw_of_num696c7b3c0b6fb9b9c9d699cd0a410ea3.tex}}}}

\newcommand{\sailRISCVfnsopwOfNum}{\saildoclabelled{sailRISCVfnzsopwzyofzynum}{\saildocfn{}{\lstinputlisting[language=sail]{sail_latex_riscv/fnzsopw_of_num696c7b3c0b6fb9b9c9d699cd0a410ea3.tex}}}}

\newcommand{\sailRISCVvalnumOfSopw}{\saildoclabelled{sailRISCVznumzyofzysopw}{\saildocval{}{\lstinputlisting[language=sail]{sail_latex_riscv/valznum_of_sopw352409ee6a8831f827129fc3d78cd4d6.tex}}}}

\newcommand{\sailRISCVfnnumOfSopw}{\saildoclabelled{sailRISCVfnznumzyofzysopw}{\saildocfn{}{\lstinputlisting[language=sail]{sail_latex_riscv/fnznum_of_sopw352409ee6a8831f827129fc3d78cd4d6.tex}}}}

\newcommand{\sailRISCVtypeamoop}{\saildoclabelled{sailRISCVtypezamoop}{\saildoctype{}{\lstinputlisting[language=sail]{sail_latex_riscv/typezamoopd600bc1f48968cf6c909a7527f520a98.tex}}}}

\newcommand{\sailRISCVvalamoopOfNum}{\saildoclabelled{sailRISCVzamoopzyofzynum}{\saildocval{}{\lstinputlisting[language=sail]{sail_latex_riscv/valzamoop_of_num66fc14378761bc8bd8137ac63cffe431.tex}}}}

\newcommand{\sailRISCVfnamoopOfNum}{\saildoclabelled{sailRISCVfnzamoopzyofzynum}{\saildocfn{}{\lstinputlisting[language=sail]{sail_latex_riscv/fnzamoop_of_num66fc14378761bc8bd8137ac63cffe431.tex}}}}

\newcommand{\sailRISCVvalnumOfAmoop}{\saildoclabelled{sailRISCVznumzyofzyamoop}{\saildocval{}{\lstinputlisting[language=sail]{sail_latex_riscv/valznum_of_amoop3990788e22835bf2e0af928f223c3eba.tex}}}}

\newcommand{\sailRISCVfnnumOfAmoop}{\saildoclabelled{sailRISCVfnznumzyofzyamoop}{\saildocfn{}{\lstinputlisting[language=sail]{sail_latex_riscv/fnznum_of_amoop3990788e22835bf2e0af928f223c3eba.tex}}}}

\newcommand{\sailRISCVtypecsrop}{\saildoclabelled{sailRISCVtypezcsrop}{\saildoctype{}{\lstinputlisting[language=sail]{sail_latex_riscv/typezcsrop2c5a5cf59f588ba61c955dc544ab5638.tex}}}}

\newcommand{\sailRISCVvalcsropOfNum}{\saildoclabelled{sailRISCVzcsropzyofzynum}{\saildocval{}{\lstinputlisting[language=sail]{sail_latex_riscv/valzcsrop_of_numfc0e82db24db14fec87d0613c91892f2.tex}}}}

\newcommand{\sailRISCVfncsropOfNum}{\saildoclabelled{sailRISCVfnzcsropzyofzynum}{\saildocfn{}{\lstinputlisting[language=sail]{sail_latex_riscv/fnzcsrop_of_numfc0e82db24db14fec87d0613c91892f2.tex}}}}

\newcommand{\sailRISCVvalnumOfCsrop}{\saildoclabelled{sailRISCVznumzyofzycsrop}{\saildocval{}{\lstinputlisting[language=sail]{sail_latex_riscv/valznum_of_csropc21ef48aae10c4abc2e72f7386a31ce9.tex}}}}

\newcommand{\sailRISCVfnnumOfCsrop}{\saildoclabelled{sailRISCVfnznumzyofzycsrop}{\saildocfn{}{\lstinputlisting[language=sail]{sail_latex_riscv/fnznum_of_csropc21ef48aae10c4abc2e72f7386a31ce9.tex}}}}

\newcommand{\sailRISCVtypebropZba}{\saildoclabelled{sailRISCVtypezbropzyzzba}{\saildoctype{}{\lstinputlisting[language=sail]{sail_latex_riscv/typezbrop_zzba2fef5f71fe656386180fbdb08a9c9b3c.tex}}}}

\newcommand{\sailRISCVvalbropZbaOfNum}{\saildoclabelled{sailRISCVzbropzyzzbazyofzynum}{\saildocval{}{\lstinputlisting[language=sail]{sail_latex_riscv/valzbrop_zzba_of_num70ddd1a327c1d2a55a3182bf9129ab44.tex}}}}

\newcommand{\sailRISCVfnbropZbaOfNum}{\saildoclabelled{sailRISCVfnzbropzyzzbazyofzynum}{\saildocfn{}{\lstinputlisting[language=sail]{sail_latex_riscv/fnzbrop_zzba_of_num70ddd1a327c1d2a55a3182bf9129ab44.tex}}}}

\newcommand{\sailRISCVvalnumOfBropZba}{\saildoclabelled{sailRISCVznumzyofzybropzyzzba}{\saildocval{}{\lstinputlisting[language=sail]{sail_latex_riscv/valznum_of_brop_zzbaffac43aa090bb6e9de66329a81367cf3.tex}}}}

\newcommand{\sailRISCVfnnumOfBropZba}{\saildoclabelled{sailRISCVfnznumzyofzybropzyzzba}{\saildocfn{}{\lstinputlisting[language=sail]{sail_latex_riscv/fnznum_of_brop_zzbaffac43aa090bb6e9de66329a81367cf3.tex}}}}

\newcommand{\sailRISCVtypebropZbb}{\saildoclabelled{sailRISCVtypezbropzyzzbb}{\saildoctype{}{\lstinputlisting[language=sail]{sail_latex_riscv/typezbrop_zzbb5ce55a2db97dfab9fb8794f3b57f1ef2.tex}}}}

\newcommand{\sailRISCVvalbropZbbOfNum}{\saildoclabelled{sailRISCVzbropzyzzbbzyofzynum}{\saildocval{}{\lstinputlisting[language=sail]{sail_latex_riscv/valzbrop_zzbb_of_num1ac2bd80c3ef359df2e0c1bc2821a18f.tex}}}}

\newcommand{\sailRISCVfnbropZbbOfNum}{\saildoclabelled{sailRISCVfnzbropzyzzbbzyofzynum}{\saildocfn{}{\lstinputlisting[language=sail]{sail_latex_riscv/fnzbrop_zzbb_of_num1ac2bd80c3ef359df2e0c1bc2821a18f.tex}}}}

\newcommand{\sailRISCVvalnumOfBropZbb}{\saildoclabelled{sailRISCVznumzyofzybropzyzzbb}{\saildocval{}{\lstinputlisting[language=sail]{sail_latex_riscv/valznum_of_brop_zzbbab2648ee8593eb2e89351c7678891ba0.tex}}}}

\newcommand{\sailRISCVfnnumOfBropZbb}{\saildoclabelled{sailRISCVfnznumzyofzybropzyzzbb}{\saildocfn{}{\lstinputlisting[language=sail]{sail_latex_riscv/fnznum_of_brop_zzbbab2648ee8593eb2e89351c7678891ba0.tex}}}}

\newcommand{\sailRISCVtypebropZbkb}{\saildoclabelled{sailRISCVtypezbropzyzzbkb}{\saildoctype{}{\lstinputlisting[language=sail]{sail_latex_riscv/typezbrop_zzbkb4ef7bbd8ba542542349463262e6e2ff5.tex}}}}

\newcommand{\sailRISCVvalbropZbkbOfNum}{\saildoclabelled{sailRISCVzbropzyzzbkbzyofzynum}{\saildocval{}{\lstinputlisting[language=sail]{sail_latex_riscv/valzbrop_zzbkb_of_num4491ab5f16454ed67b90a355ce7a27b6.tex}}}}

\newcommand{\sailRISCVfnbropZbkbOfNum}{\saildoclabelled{sailRISCVfnzbropzyzzbkbzyofzynum}{\saildocfn{}{\lstinputlisting[language=sail]{sail_latex_riscv/fnzbrop_zzbkb_of_num4491ab5f16454ed67b90a355ce7a27b6.tex}}}}

\newcommand{\sailRISCVvalnumOfBropZbkb}{\saildoclabelled{sailRISCVznumzyofzybropzyzzbkb}{\saildocval{}{\lstinputlisting[language=sail]{sail_latex_riscv/valznum_of_brop_zzbkb43c910f604c93cfb88068237be34a053.tex}}}}

\newcommand{\sailRISCVfnnumOfBropZbkb}{\saildoclabelled{sailRISCVfnznumzyofzybropzyzzbkb}{\saildocfn{}{\lstinputlisting[language=sail]{sail_latex_riscv/fnznum_of_brop_zzbkb43c910f604c93cfb88068237be34a053.tex}}}}

\newcommand{\sailRISCVtypebropZbs}{\saildoclabelled{sailRISCVtypezbropzyzzbs}{\saildoctype{}{\lstinputlisting[language=sail]{sail_latex_riscv/typezbrop_zzbs873da065884c0ad20f33cd855ab908f8.tex}}}}

\newcommand{\sailRISCVvalbropZbsOfNum}{\saildoclabelled{sailRISCVzbropzyzzbszyofzynum}{\saildocval{}{\lstinputlisting[language=sail]{sail_latex_riscv/valzbrop_zzbs_of_num930c22b621a73b913089b63d53f85888.tex}}}}

\newcommand{\sailRISCVfnbropZbsOfNum}{\saildoclabelled{sailRISCVfnzbropzyzzbszyofzynum}{\saildocfn{}{\lstinputlisting[language=sail]{sail_latex_riscv/fnzbrop_zzbs_of_num930c22b621a73b913089b63d53f85888.tex}}}}

\newcommand{\sailRISCVvalnumOfBropZbs}{\saildoclabelled{sailRISCVznumzyofzybropzyzzbs}{\saildocval{}{\lstinputlisting[language=sail]{sail_latex_riscv/valznum_of_brop_zzbs4e85ebdec04930f0e563b8a4bd2ac5f5.tex}}}}

\newcommand{\sailRISCVfnnumOfBropZbs}{\saildoclabelled{sailRISCVfnznumzyofzybropzyzzbs}{\saildocfn{}{\lstinputlisting[language=sail]{sail_latex_riscv/fnznum_of_brop_zzbs4e85ebdec04930f0e563b8a4bd2ac5f5.tex}}}}

\newcommand{\sailRISCVtypebropwZba}{\saildoclabelled{sailRISCVtypezbropwzyzzba}{\saildoctype{}{\lstinputlisting[language=sail]{sail_latex_riscv/typezbropw_zzbae2fff71fc26b868b177ab0d8af385f23.tex}}}}

\newcommand{\sailRISCVvalbropwZbaOfNum}{\saildoclabelled{sailRISCVzbropwzyzzbazyofzynum}{\saildocval{}{\lstinputlisting[language=sail]{sail_latex_riscv/valzbropw_zzba_of_numbb6500f5a1f03ef8ebad88dc76af7705.tex}}}}

\newcommand{\sailRISCVfnbropwZbaOfNum}{\saildoclabelled{sailRISCVfnzbropwzyzzbazyofzynum}{\saildocfn{}{\lstinputlisting[language=sail]{sail_latex_riscv/fnzbropw_zzba_of_numbb6500f5a1f03ef8ebad88dc76af7705.tex}}}}

\newcommand{\sailRISCVvalnumOfBropwZba}{\saildoclabelled{sailRISCVznumzyofzybropwzyzzba}{\saildocval{}{\lstinputlisting[language=sail]{sail_latex_riscv/valznum_of_bropw_zzbac1a4b135d20fc6432e05d15522f0b281.tex}}}}

\newcommand{\sailRISCVfnnumOfBropwZba}{\saildoclabelled{sailRISCVfnznumzyofzybropwzyzzba}{\saildocfn{}{\lstinputlisting[language=sail]{sail_latex_riscv/fnznum_of_bropw_zzbac1a4b135d20fc6432e05d15522f0b281.tex}}}}

\newcommand{\sailRISCVtypebropwZbb}{\saildoclabelled{sailRISCVtypezbropwzyzzbb}{\saildoctype{}{\lstinputlisting[language=sail]{sail_latex_riscv/typezbropw_zzbb5a14af58a47fd1e12eae4b8133aaad3a.tex}}}}

\newcommand{\sailRISCVvalbropwZbbOfNum}{\saildoclabelled{sailRISCVzbropwzyzzbbzyofzynum}{\saildocval{}{\lstinputlisting[language=sail]{sail_latex_riscv/valzbropw_zzbb_of_numba7408999d7018da58cc69911b9ce189.tex}}}}

\newcommand{\sailRISCVfnbropwZbbOfNum}{\saildoclabelled{sailRISCVfnzbropwzyzzbbzyofzynum}{\saildocfn{}{\lstinputlisting[language=sail]{sail_latex_riscv/fnzbropw_zzbb_of_numba7408999d7018da58cc69911b9ce189.tex}}}}

\newcommand{\sailRISCVvalnumOfBropwZbb}{\saildoclabelled{sailRISCVznumzyofzybropwzyzzbb}{\saildocval{}{\lstinputlisting[language=sail]{sail_latex_riscv/valznum_of_bropw_zzbb4399379f0c2dafb3c8bd162267248062.tex}}}}

\newcommand{\sailRISCVfnnumOfBropwZbb}{\saildoclabelled{sailRISCVfnznumzyofzybropwzyzzbb}{\saildocfn{}{\lstinputlisting[language=sail]{sail_latex_riscv/fnznum_of_bropw_zzbb4399379f0c2dafb3c8bd162267248062.tex}}}}

\newcommand{\sailRISCVtypebiopZbs}{\saildoclabelled{sailRISCVtypezbiopzyzzbs}{\saildoctype{}{\lstinputlisting[language=sail]{sail_latex_riscv/typezbiop_zzbs2bd346a4a02cf7d57854a032d6f339fe.tex}}}}

\newcommand{\sailRISCVvalbiopZbsOfNum}{\saildoclabelled{sailRISCVzbiopzyzzbszyofzynum}{\saildocval{}{\lstinputlisting[language=sail]{sail_latex_riscv/valzbiop_zzbs_of_num657ce5ff4fabd66cd7718d0f4801f00d.tex}}}}

\newcommand{\sailRISCVfnbiopZbsOfNum}{\saildoclabelled{sailRISCVfnzbiopzyzzbszyofzynum}{\saildocfn{}{\lstinputlisting[language=sail]{sail_latex_riscv/fnzbiop_zzbs_of_num657ce5ff4fabd66cd7718d0f4801f00d.tex}}}}

\newcommand{\sailRISCVvalnumOfBiopZbs}{\saildoclabelled{sailRISCVznumzyofzybiopzyzzbs}{\saildocval{}{\lstinputlisting[language=sail]{sail_latex_riscv/valznum_of_biop_zzbs19ab070605bc91ca70e24fdca48cdd81.tex}}}}

\newcommand{\sailRISCVfnnumOfBiopZbs}{\saildoclabelled{sailRISCVfnznumzyofzybiopzyzzbs}{\saildocfn{}{\lstinputlisting[language=sail]{sail_latex_riscv/fnznum_of_biop_zzbs19ab070605bc91ca70e24fdca48cdd81.tex}}}}

\newcommand{\sailRISCVtypeextopZbb}{\saildoclabelled{sailRISCVtypezextopzyzzbb}{\saildoctype{}{\lstinputlisting[language=sail]{sail_latex_riscv/typezextop_zzbbd6eb4e7cfa917494114fac88d6e8669a.tex}}}}

\newcommand{\sailRISCVvalextopZbbOfNum}{\saildoclabelled{sailRISCVzextopzyzzbbzyofzynum}{\saildocval{}{\lstinputlisting[language=sail]{sail_latex_riscv/valzextop_zzbb_of_num07a1c89bcee01ee6a4319e009b3130ba.tex}}}}

\newcommand{\sailRISCVfnextopZbbOfNum}{\saildoclabelled{sailRISCVfnzextopzyzzbbzyofzynum}{\saildocfn{}{\lstinputlisting[language=sail]{sail_latex_riscv/fnzextop_zzbb_of_num07a1c89bcee01ee6a4319e009b3130ba.tex}}}}

\newcommand{\sailRISCVvalnumOfExtopZbb}{\saildoclabelled{sailRISCVznumzyofzyextopzyzzbb}{\saildocval{}{\lstinputlisting[language=sail]{sail_latex_riscv/valznum_of_extop_zzbbccc649422c5281a2100c3e881e0f493a.tex}}}}

\newcommand{\sailRISCVfnnumOfExtopZbb}{\saildoclabelled{sailRISCVfnznumzyofzyextopzyzzbb}{\saildocfn{}{\lstinputlisting[language=sail]{sail_latex_riscv/fnznum_of_extop_zzbbccc649422c5281a2100c3e881e0f493a.tex}}}}

\newcommand{\sailRISCVvalsep}{\saildoclabelled{sailRISCVzsep}{\saildocval{}{\lstinputlisting[language=sail]{sail_latex_riscv/valzsep43dac566d26bd3f7fb9088b8d09ca246.tex}}}}

\newcommand{\sailRISCVvalboolBits}{\saildoclabelled{sailRISCVzboolzybits}{\saildocval{}{\lstinputlisting[language=sail]{sail_latex_riscv/valzbool_bitsc0498d89cd314b64ec44bc657a0630ec.tex}}}}

\newcommand{\sailRISCVvalboolNotBits}{\saildoclabelled{sailRISCVzboolzynotzybits}{\saildocval{}{\lstinputlisting[language=sail]{sail_latex_riscv/valzbool_not_bits8cc7dc8d8a4bcd7dc0a9851db7a322cc.tex}}}}

\newcommand{\sailRISCVvalsizzeBits}{\saildoclabelled{sailRISCVzsizzezybits}{\saildocval{}{\lstinputlisting[language=sail]{sail_latex_riscv/valzsizze_bits40630770cee662892ab899654b7e2f0d.tex}}}}

\newcommand{\sailRISCVvalsizzeMnemonic}{\saildoclabelled{sailRISCVzsizzezymnemonic}{\saildocval{}{\lstinputlisting[language=sail]{sail_latex_riscv/valzsizze_mnemonic4114d3c8877e3689368588f52debea7b.tex}}}}

\newcommand{\sailRISCVvalwordWidthBytes}{\saildoclabelled{sailRISCVzwordzywidthzybytes}{\saildocval{}{\lstinputlisting[language=sail]{sail_latex_riscv/valzword_width_bytes3499487c0f03a80d8659fa504a62261f.tex}}}}

\newcommand{\sailRISCVfnwordWidthBytes}{\saildoclabelled{sailRISCVfnzwordzywidthzybytes}{\saildocfn{}{\lstinputlisting[language=sail]{sail_latex_riscv/fnzword_width_bytes3499487c0f03a80d8659fa504a62261f.tex}}}}

\newcommand{\sailRISCVlethaveSplitRegFile}{\saildoclabelled{sailRISCVletzhaveSplitRegFile}{\saildoclet{}{\lstinputlisting[language=sail]{sail_latex_riscv/letzhavesplitregfilec09adbb1baa006c868c3305835dc10f9.tex}}}}

\newcommand{\sailRISCVtyperegtype}{\saildoclabelled{sailRISCVtypezregtype}{\saildoctype{}{\lstinputlisting[language=sail]{sail_latex_riscv/typezregtype0be936c0d9e150ad1beff99e08691cf8.tex}}}}

\newcommand{\sailRISCVletzzeroReg}{\saildoclabelled{sailRISCVletzzzerozyreg}{\saildoclet{}{\lstinputlisting[language=sail]{sail_latex_riscv/letzzzero_reg78bddf2297db5666e334328609966802.tex}}}}

\newcommand{\sailRISCVvalRegStr}{\saildoclabelled{sailRISCVzRegStr}{\saildocval{}{\lstinputlisting[language=sail]{sail_latex_riscv/valzregstrf07d744c662238e6879fb1aee407788d.tex}}}}

\newcommand{\sailRISCVfnRegStr}{\saildoclabelled{sailRISCVfnzRegStr}{\saildocfn{}{\lstinputlisting[language=sail]{sail_latex_riscv/fnzregstrf07d744c662238e6879fb1aee407788d.tex}}}}

\newcommand{\sailRISCVvalregvalFromReg}{\saildoclabelled{sailRISCVzregvalzyfromzyreg}{\saildocval{}{\lstinputlisting[language=sail]{sail_latex_riscv/valzregval_from_rega072b7983f1f9ac4a9021e76a911b9c9.tex}}}}

\newcommand{\sailRISCVfnregvalFromReg}{\saildoclabelled{sailRISCVfnzregvalzyfromzyreg}{\saildocfn{}{\lstinputlisting[language=sail]{sail_latex_riscv/fnzregval_from_rega072b7983f1f9ac4a9021e76a911b9c9.tex}}}}

\newcommand{\sailRISCVvalregvalIntoReg}{\saildoclabelled{sailRISCVzregvalzyintozyreg}{\saildocval{}{\lstinputlisting[language=sail]{sail_latex_riscv/valzregval_into_reg03a5bdeabb3e6169090e6ed21f1a84cb.tex}}}}

\newcommand{\sailRISCVfnregvalIntoReg}{\saildoclabelled{sailRISCVfnzregvalzyintozyreg}{\saildocfn{}{\lstinputlisting[language=sail]{sail_latex_riscv/fnzregval_into_reg03a5bdeabb3e6169090e6ed21f1a84cb.tex}}}}

\newcommand{\sailRISCVtypefregtype}{\saildoclabelled{sailRISCVtypezfregtype}{\saildoctype{}{\lstinputlisting[language=sail]{sail_latex_riscv/typezfregtype005522b0120ed9774e1ab7767fd5f4aa.tex}}}}

\newcommand{\sailRISCVletzzeroFreg}{\saildoclabelled{sailRISCVletzzzerozyfreg}{\saildoclet{}{\lstinputlisting[language=sail]{sail_latex_riscv/letzzzero_freg51d2ad9aad2dcf96396ffd9ba5e9a050.tex}}}}

\newcommand{\sailRISCVvalFRegStr}{\saildoclabelled{sailRISCVzFRegStr}{\saildocval{}{\lstinputlisting[language=sail]{sail_latex_riscv/valzfregstr48009e974e6089e7ac15bb0f9271a481.tex}}}}

\newcommand{\sailRISCVfnFRegStr}{\saildoclabelled{sailRISCVfnzFRegStr}{\saildocfn{}{\lstinputlisting[language=sail]{sail_latex_riscv/fnzfregstr48009e974e6089e7ac15bb0f9271a481.tex}}}}

\newcommand{\sailRISCVvalfregvalFromFreg}{\saildoclabelled{sailRISCVzfregvalzyfromzyfreg}{\saildocval{}{\lstinputlisting[language=sail]{sail_latex_riscv/valzfregval_from_freg24f4c96a7be559c758c7de5e3dfe669b.tex}}}}

\newcommand{\sailRISCVfnfregvalFromFreg}{\saildoclabelled{sailRISCVfnzfregvalzyfromzyfreg}{\saildocfn{}{\lstinputlisting[language=sail]{sail_latex_riscv/fnzfregval_from_freg24f4c96a7be559c758c7de5e3dfe669b.tex}}}}

\newcommand{\sailRISCVvalfregvalIntoFreg}{\saildoclabelled{sailRISCVzfregvalzyintozyfreg}{\saildocval{}{\lstinputlisting[language=sail]{sail_latex_riscv/valzfregval_into_fregc2bccb9e14fa0e7ca1dec382b998fa46.tex}}}}

\newcommand{\sailRISCVfnfregvalIntoFreg}{\saildoclabelled{sailRISCVfnzfregvalzyintozyfreg}{\saildocfn{}{\lstinputlisting[language=sail]{sail_latex_riscv/fnzfregval_into_fregc2bccb9e14fa0e7ca1dec382b998fa46.tex}}}}

\newcommand{\sailRISCVtypefMaddOpH}{\saildoclabelled{sailRISCVtypezfzymaddzyopzyH}{\saildoctype{}{\lstinputlisting[language=sail]{sail_latex_riscv/typezf_madd_op_h014c0f7cceeee675c14766eef28d4847.tex}}}}

\newcommand{\sailRISCVvalfMaddOpHOfNum}{\saildoclabelled{sailRISCVzfzymaddzyopzyHzyofzynum}{\saildocval{}{\lstinputlisting[language=sail]{sail_latex_riscv/valzf_madd_op_h_of_num9c6ba0afa4ca20d700feef2eab722786.tex}}}}

\newcommand{\sailRISCVfnfMaddOpHOfNum}{\saildoclabelled{sailRISCVfnzfzymaddzyopzyHzyofzynum}{\saildocfn{}{\lstinputlisting[language=sail]{sail_latex_riscv/fnzf_madd_op_h_of_num9c6ba0afa4ca20d700feef2eab722786.tex}}}}

\newcommand{\sailRISCVvalnumOfFMaddOpH}{\saildoclabelled{sailRISCVznumzyofzyfzymaddzyopzyH}{\saildocval{}{\lstinputlisting[language=sail]{sail_latex_riscv/valznum_of_f_madd_op_h06ed7344938cfd642d47f2fed02b2eee.tex}}}}

\newcommand{\sailRISCVfnnumOfFMaddOpH}{\saildoclabelled{sailRISCVfnznumzyofzyfzymaddzyopzyH}{\saildocfn{}{\lstinputlisting[language=sail]{sail_latex_riscv/fnznum_of_f_madd_op_h06ed7344938cfd642d47f2fed02b2eee.tex}}}}

\newcommand{\sailRISCVtypefBinRmOpH}{\saildoclabelled{sailRISCVtypezfzybinzyrmzyopzyH}{\saildoctype{}{\lstinputlisting[language=sail]{sail_latex_riscv/typezf_bin_rm_op_hcb654af04c01581002301a740cd2713e.tex}}}}

\newcommand{\sailRISCVvalfBinRmOpHOfNum}{\saildoclabelled{sailRISCVzfzybinzyrmzyopzyHzyofzynum}{\saildocval{}{\lstinputlisting[language=sail]{sail_latex_riscv/valzf_bin_rm_op_h_of_num1bae706bdd26d1197b71ccf7f7957468.tex}}}}

\newcommand{\sailRISCVfnfBinRmOpHOfNum}{\saildoclabelled{sailRISCVfnzfzybinzyrmzyopzyHzyofzynum}{\saildocfn{}{\lstinputlisting[language=sail]{sail_latex_riscv/fnzf_bin_rm_op_h_of_num1bae706bdd26d1197b71ccf7f7957468.tex}}}}

\newcommand{\sailRISCVvalnumOfFBinRmOpH}{\saildoclabelled{sailRISCVznumzyofzyfzybinzyrmzyopzyH}{\saildocval{}{\lstinputlisting[language=sail]{sail_latex_riscv/valznum_of_f_bin_rm_op_h531a49b863e2a5f094150a9a18e53a50.tex}}}}

\newcommand{\sailRISCVfnnumOfFBinRmOpH}{\saildoclabelled{sailRISCVfnznumzyofzyfzybinzyrmzyopzyH}{\saildocfn{}{\lstinputlisting[language=sail]{sail_latex_riscv/fnznum_of_f_bin_rm_op_h531a49b863e2a5f094150a9a18e53a50.tex}}}}

\newcommand{\sailRISCVtypefUnRmOpH}{\saildoclabelled{sailRISCVtypezfzyunzyrmzyopzyH}{\saildoctype{}{\lstinputlisting[language=sail]{sail_latex_riscv/typezf_un_rm_op_hba65e7903cc8eeb26578b6cadc4f1702.tex}}}}

\newcommand{\sailRISCVvalfUnRmOpHOfNum}{\saildoclabelled{sailRISCVzfzyunzyrmzyopzyHzyofzynum}{\saildocval{}{\lstinputlisting[language=sail]{sail_latex_riscv/valzf_un_rm_op_h_of_nume9619d76cbed957bef00f9daac49d062.tex}}}}

\newcommand{\sailRISCVfnfUnRmOpHOfNum}{\saildoclabelled{sailRISCVfnzfzyunzyrmzyopzyHzyofzynum}{\saildocfn{}{\lstinputlisting[language=sail]{sail_latex_riscv/fnzf_un_rm_op_h_of_nume9619d76cbed957bef00f9daac49d062.tex}}}}

\newcommand{\sailRISCVvalnumOfFUnRmOpH}{\saildoclabelled{sailRISCVznumzyofzyfzyunzyrmzyopzyH}{\saildocval{}{\lstinputlisting[language=sail]{sail_latex_riscv/valznum_of_f_un_rm_op_h43cbb69fbdce1fe12385a7ef5f623309.tex}}}}

\newcommand{\sailRISCVfnnumOfFUnRmOpH}{\saildoclabelled{sailRISCVfnznumzyofzyfzyunzyrmzyopzyH}{\saildocfn{}{\lstinputlisting[language=sail]{sail_latex_riscv/fnznum_of_f_un_rm_op_h43cbb69fbdce1fe12385a7ef5f623309.tex}}}}

\newcommand{\sailRISCVtypefUnOpH}{\saildoclabelled{sailRISCVtypezfzyunzyopzyH}{\saildoctype{}{\lstinputlisting[language=sail]{sail_latex_riscv/typezf_un_op_h8bda8fcb89ae06790ea28e380d4c3fdb.tex}}}}

\newcommand{\sailRISCVvalfUnOpHOfNum}{\saildoclabelled{sailRISCVzfzyunzyopzyHzyofzynum}{\saildocval{}{\lstinputlisting[language=sail]{sail_latex_riscv/valzf_un_op_h_of_numdea98b2b78368536a12adef1e0b08a5f.tex}}}}

\newcommand{\sailRISCVfnfUnOpHOfNum}{\saildoclabelled{sailRISCVfnzfzyunzyopzyHzyofzynum}{\saildocfn{}{\lstinputlisting[language=sail]{sail_latex_riscv/fnzf_un_op_h_of_numdea98b2b78368536a12adef1e0b08a5f.tex}}}}

\newcommand{\sailRISCVvalnumOfFUnOpH}{\saildoclabelled{sailRISCVznumzyofzyfzyunzyopzyH}{\saildocval{}{\lstinputlisting[language=sail]{sail_latex_riscv/valznum_of_f_un_op_h172862984783e8eaa16c22b18b5be71f.tex}}}}

\newcommand{\sailRISCVfnnumOfFUnOpH}{\saildoclabelled{sailRISCVfnznumzyofzyfzyunzyopzyH}{\saildocfn{}{\lstinputlisting[language=sail]{sail_latex_riscv/fnznum_of_f_un_op_h172862984783e8eaa16c22b18b5be71f.tex}}}}

\newcommand{\sailRISCVtypefBinOpH}{\saildoclabelled{sailRISCVtypezfzybinzyopzyH}{\saildoctype{}{\lstinputlisting[language=sail]{sail_latex_riscv/typezf_bin_op_h490f32a951a7eb29ad316e852be6ba7e.tex}}}}

\newcommand{\sailRISCVvalfBinOpHOfNum}{\saildoclabelled{sailRISCVzfzybinzyopzyHzyofzynum}{\saildocval{}{\lstinputlisting[language=sail]{sail_latex_riscv/valzf_bin_op_h_of_num353ed4450b643a44781ade64a0c820a3.tex}}}}

\newcommand{\sailRISCVfnfBinOpHOfNum}{\saildoclabelled{sailRISCVfnzfzybinzyopzyHzyofzynum}{\saildocfn{}{\lstinputlisting[language=sail]{sail_latex_riscv/fnzf_bin_op_h_of_num353ed4450b643a44781ade64a0c820a3.tex}}}}

\newcommand{\sailRISCVvalnumOfFBinOpH}{\saildoclabelled{sailRISCVznumzyofzyfzybinzyopzyH}{\saildocval{}{\lstinputlisting[language=sail]{sail_latex_riscv/valznum_of_f_bin_op_hcc69c768e714f41d5cd8bf2d4e0c9019.tex}}}}

\newcommand{\sailRISCVfnnumOfFBinOpH}{\saildoclabelled{sailRISCVfnznumzyofzyfzybinzyopzyH}{\saildocfn{}{\lstinputlisting[language=sail]{sail_latex_riscv/fnznum_of_f_bin_op_hcc69c768e714f41d5cd8bf2d4e0c9019.tex}}}}

\newcommand{\sailRISCVtyperoundingMode}{\saildoclabelled{sailRISCVtypezroundingzymode}{\saildoctype{}{\lstinputlisting[language=sail]{sail_latex_riscv/typezrounding_modea8ccd05712e3eb82022c72b8440f235b.tex}}}}

\newcommand{\sailRISCVvalroundingModeOfNum}{\saildoclabelled{sailRISCVzroundingzymodezyofzynum}{\saildocval{}{\lstinputlisting[language=sail]{sail_latex_riscv/valzrounding_mode_of_num6135245be40748c506fabd0190282238.tex}}}}

\newcommand{\sailRISCVfnroundingModeOfNum}{\saildoclabelled{sailRISCVfnzroundingzymodezyofzynum}{\saildocfn{}{\lstinputlisting[language=sail]{sail_latex_riscv/fnzrounding_mode_of_num6135245be40748c506fabd0190282238.tex}}}}

\newcommand{\sailRISCVvalnumOfRoundingMode}{\saildoclabelled{sailRISCVznumzyofzyroundingzymode}{\saildocval{}{\lstinputlisting[language=sail]{sail_latex_riscv/valznum_of_rounding_mode41a60e82308e6507de434c7dc6e17db8.tex}}}}

\newcommand{\sailRISCVfnnumOfRoundingMode}{\saildoclabelled{sailRISCVfnznumzyofzyroundingzymode}{\saildocfn{}{\lstinputlisting[language=sail]{sail_latex_riscv/fnznum_of_rounding_mode41a60e82308e6507de434c7dc6e17db8.tex}}}}

\newcommand{\sailRISCVtypefMaddOpS}{\saildoclabelled{sailRISCVtypezfzymaddzyopzyS}{\saildoctype{}{\lstinputlisting[language=sail]{sail_latex_riscv/typezf_madd_op_sb37be1865406477b64362e39a00e2afe.tex}}}}

\newcommand{\sailRISCVvalfMaddOpSOfNum}{\saildoclabelled{sailRISCVzfzymaddzyopzySzyofzynum}{\saildocval{}{\lstinputlisting[language=sail]{sail_latex_riscv/valzf_madd_op_s_of_num9bfe1beb29f1f97cf6b7643bd9febe9b.tex}}}}

\newcommand{\sailRISCVfnfMaddOpSOfNum}{\saildoclabelled{sailRISCVfnzfzymaddzyopzySzyofzynum}{\saildocfn{}{\lstinputlisting[language=sail]{sail_latex_riscv/fnzf_madd_op_s_of_num9bfe1beb29f1f97cf6b7643bd9febe9b.tex}}}}

\newcommand{\sailRISCVvalnumOfFMaddOpS}{\saildoclabelled{sailRISCVznumzyofzyfzymaddzyopzyS}{\saildocval{}{\lstinputlisting[language=sail]{sail_latex_riscv/valznum_of_f_madd_op_s775906828ed91e55055481dedc5d9da4.tex}}}}

\newcommand{\sailRISCVfnnumOfFMaddOpS}{\saildoclabelled{sailRISCVfnznumzyofzyfzymaddzyopzyS}{\saildocfn{}{\lstinputlisting[language=sail]{sail_latex_riscv/fnznum_of_f_madd_op_s775906828ed91e55055481dedc5d9da4.tex}}}}

\newcommand{\sailRISCVtypefBinRmOpS}{\saildoclabelled{sailRISCVtypezfzybinzyrmzyopzyS}{\saildoctype{}{\lstinputlisting[language=sail]{sail_latex_riscv/typezf_bin_rm_op_s0fd11305b9d3dfc93e566ad91732c086.tex}}}}

\newcommand{\sailRISCVvalfBinRmOpSOfNum}{\saildoclabelled{sailRISCVzfzybinzyrmzyopzySzyofzynum}{\saildocval{}{\lstinputlisting[language=sail]{sail_latex_riscv/valzf_bin_rm_op_s_of_num425ba7f76e47ae16326c0417fe340273.tex}}}}

\newcommand{\sailRISCVfnfBinRmOpSOfNum}{\saildoclabelled{sailRISCVfnzfzybinzyrmzyopzySzyofzynum}{\saildocfn{}{\lstinputlisting[language=sail]{sail_latex_riscv/fnzf_bin_rm_op_s_of_num425ba7f76e47ae16326c0417fe340273.tex}}}}

\newcommand{\sailRISCVvalnumOfFBinRmOpS}{\saildoclabelled{sailRISCVznumzyofzyfzybinzyrmzyopzyS}{\saildocval{}{\lstinputlisting[language=sail]{sail_latex_riscv/valznum_of_f_bin_rm_op_s893f36c63d593d34b1cf67930090a6d2.tex}}}}

\newcommand{\sailRISCVfnnumOfFBinRmOpS}{\saildoclabelled{sailRISCVfnznumzyofzyfzybinzyrmzyopzyS}{\saildocfn{}{\lstinputlisting[language=sail]{sail_latex_riscv/fnznum_of_f_bin_rm_op_s893f36c63d593d34b1cf67930090a6d2.tex}}}}

\newcommand{\sailRISCVtypefUnRmOpS}{\saildoclabelled{sailRISCVtypezfzyunzyrmzyopzyS}{\saildoctype{}{\lstinputlisting[language=sail]{sail_latex_riscv/typezf_un_rm_op_sd0ea7b4f9a759adc91555b3055b8016a.tex}}}}

\newcommand{\sailRISCVvalfUnRmOpSOfNum}{\saildoclabelled{sailRISCVzfzyunzyrmzyopzySzyofzynum}{\saildocval{}{\lstinputlisting[language=sail]{sail_latex_riscv/valzf_un_rm_op_s_of_numcf978158d6a6f45fe1893c4a9c122140.tex}}}}

\newcommand{\sailRISCVfnfUnRmOpSOfNum}{\saildoclabelled{sailRISCVfnzfzyunzyrmzyopzySzyofzynum}{\saildocfn{}{\lstinputlisting[language=sail]{sail_latex_riscv/fnzf_un_rm_op_s_of_numcf978158d6a6f45fe1893c4a9c122140.tex}}}}

\newcommand{\sailRISCVvalnumOfFUnRmOpS}{\saildoclabelled{sailRISCVznumzyofzyfzyunzyrmzyopzyS}{\saildocval{}{\lstinputlisting[language=sail]{sail_latex_riscv/valznum_of_f_un_rm_op_se915d9a4f61459b98b43ab8bc6055666.tex}}}}

\newcommand{\sailRISCVfnnumOfFUnRmOpS}{\saildoclabelled{sailRISCVfnznumzyofzyfzyunzyrmzyopzyS}{\saildocfn{}{\lstinputlisting[language=sail]{sail_latex_riscv/fnznum_of_f_un_rm_op_se915d9a4f61459b98b43ab8bc6055666.tex}}}}

\newcommand{\sailRISCVtypefUnOpS}{\saildoclabelled{sailRISCVtypezfzyunzyopzyS}{\saildoctype{}{\lstinputlisting[language=sail]{sail_latex_riscv/typezf_un_op_s286e536373f63a1fa600b8619060bb26.tex}}}}

\newcommand{\sailRISCVvalfUnOpSOfNum}{\saildoclabelled{sailRISCVzfzyunzyopzySzyofzynum}{\saildocval{}{\lstinputlisting[language=sail]{sail_latex_riscv/valzf_un_op_s_of_num428b8623c493e45bc273b3e1a0c895e0.tex}}}}

\newcommand{\sailRISCVfnfUnOpSOfNum}{\saildoclabelled{sailRISCVfnzfzyunzyopzySzyofzynum}{\saildocfn{}{\lstinputlisting[language=sail]{sail_latex_riscv/fnzf_un_op_s_of_num428b8623c493e45bc273b3e1a0c895e0.tex}}}}

\newcommand{\sailRISCVvalnumOfFUnOpS}{\saildoclabelled{sailRISCVznumzyofzyfzyunzyopzyS}{\saildocval{}{\lstinputlisting[language=sail]{sail_latex_riscv/valznum_of_f_un_op_se59a01770756c5c1ccc368187228ef32.tex}}}}

\newcommand{\sailRISCVfnnumOfFUnOpS}{\saildoclabelled{sailRISCVfnznumzyofzyfzyunzyopzyS}{\saildocfn{}{\lstinputlisting[language=sail]{sail_latex_riscv/fnznum_of_f_un_op_se59a01770756c5c1ccc368187228ef32.tex}}}}

\newcommand{\sailRISCVtypefBinOpS}{\saildoclabelled{sailRISCVtypezfzybinzyopzyS}{\saildoctype{}{\lstinputlisting[language=sail]{sail_latex_riscv/typezf_bin_op_s70d2d316fdfc1f5fd4d53696fd688829.tex}}}}

\newcommand{\sailRISCVvalfBinOpSOfNum}{\saildoclabelled{sailRISCVzfzybinzyopzySzyofzynum}{\saildocval{}{\lstinputlisting[language=sail]{sail_latex_riscv/valzf_bin_op_s_of_num66d590442d54517898ad2679db0d80bc.tex}}}}

\newcommand{\sailRISCVfnfBinOpSOfNum}{\saildoclabelled{sailRISCVfnzfzybinzyopzySzyofzynum}{\saildocfn{}{\lstinputlisting[language=sail]{sail_latex_riscv/fnzf_bin_op_s_of_num66d590442d54517898ad2679db0d80bc.tex}}}}

\newcommand{\sailRISCVvalnumOfFBinOpS}{\saildoclabelled{sailRISCVznumzyofzyfzybinzyopzyS}{\saildocval{}{\lstinputlisting[language=sail]{sail_latex_riscv/valznum_of_f_bin_op_s67646ca4512abf0a4d4b44cd4f5b13ca.tex}}}}

\newcommand{\sailRISCVfnnumOfFBinOpS}{\saildoclabelled{sailRISCVfnznumzyofzyfzybinzyopzyS}{\saildocfn{}{\lstinputlisting[language=sail]{sail_latex_riscv/fnznum_of_f_bin_op_s67646ca4512abf0a4d4b44cd4f5b13ca.tex}}}}

\newcommand{\sailRISCVtypefMaddOpD}{\saildoclabelled{sailRISCVtypezfzymaddzyopzyD}{\saildoctype{}{\lstinputlisting[language=sail]{sail_latex_riscv/typezf_madd_op_d3909a9cf5d4cc7f2d6f5dc2a8683ef42.tex}}}}

\newcommand{\sailRISCVvalfMaddOpDOfNum}{\saildoclabelled{sailRISCVzfzymaddzyopzyDzyofzynum}{\saildocval{}{\lstinputlisting[language=sail]{sail_latex_riscv/valzf_madd_op_d_of_num115462b951eadf3acf4b0a2bb11e801b.tex}}}}

\newcommand{\sailRISCVfnfMaddOpDOfNum}{\saildoclabelled{sailRISCVfnzfzymaddzyopzyDzyofzynum}{\saildocfn{}{\lstinputlisting[language=sail]{sail_latex_riscv/fnzf_madd_op_d_of_num115462b951eadf3acf4b0a2bb11e801b.tex}}}}

\newcommand{\sailRISCVvalnumOfFMaddOpD}{\saildoclabelled{sailRISCVznumzyofzyfzymaddzyopzyD}{\saildocval{}{\lstinputlisting[language=sail]{sail_latex_riscv/valznum_of_f_madd_op_d37717f7c44daba9b30789132320443ff.tex}}}}

\newcommand{\sailRISCVfnnumOfFMaddOpD}{\saildoclabelled{sailRISCVfnznumzyofzyfzymaddzyopzyD}{\saildocfn{}{\lstinputlisting[language=sail]{sail_latex_riscv/fnznum_of_f_madd_op_d37717f7c44daba9b30789132320443ff.tex}}}}

\newcommand{\sailRISCVtypefBinRmOpD}{\saildoclabelled{sailRISCVtypezfzybinzyrmzyopzyD}{\saildoctype{}{\lstinputlisting[language=sail]{sail_latex_riscv/typezf_bin_rm_op_d17db6dd54145474f7e451ccb8f074943.tex}}}}

\newcommand{\sailRISCVvalfBinRmOpDOfNum}{\saildoclabelled{sailRISCVzfzybinzyrmzyopzyDzyofzynum}{\saildocval{}{\lstinputlisting[language=sail]{sail_latex_riscv/valzf_bin_rm_op_d_of_num4f1c6a877a7fdb6ce5cba8edac489378.tex}}}}

\newcommand{\sailRISCVfnfBinRmOpDOfNum}{\saildoclabelled{sailRISCVfnzfzybinzyrmzyopzyDzyofzynum}{\saildocfn{}{\lstinputlisting[language=sail]{sail_latex_riscv/fnzf_bin_rm_op_d_of_num4f1c6a877a7fdb6ce5cba8edac489378.tex}}}}

\newcommand{\sailRISCVvalnumOfFBinRmOpD}{\saildoclabelled{sailRISCVznumzyofzyfzybinzyrmzyopzyD}{\saildocval{}{\lstinputlisting[language=sail]{sail_latex_riscv/valznum_of_f_bin_rm_op_dc153c7dd1e01a91df37cddfd46dfd9da.tex}}}}

\newcommand{\sailRISCVfnnumOfFBinRmOpD}{\saildoclabelled{sailRISCVfnznumzyofzyfzybinzyrmzyopzyD}{\saildocfn{}{\lstinputlisting[language=sail]{sail_latex_riscv/fnznum_of_f_bin_rm_op_dc153c7dd1e01a91df37cddfd46dfd9da.tex}}}}

\newcommand{\sailRISCVtypefUnRmOpD}{\saildoclabelled{sailRISCVtypezfzyunzyrmzyopzyD}{\saildoctype{}{\lstinputlisting[language=sail]{sail_latex_riscv/typezf_un_rm_op_de2fd51e1e803f0967e3388fb50d06082.tex}}}}

\newcommand{\sailRISCVvalfUnRmOpDOfNum}{\saildoclabelled{sailRISCVzfzyunzyrmzyopzyDzyofzynum}{\saildocval{}{\lstinputlisting[language=sail]{sail_latex_riscv/valzf_un_rm_op_d_of_num3c7528aaf2a777986f767617d6c66717.tex}}}}

\newcommand{\sailRISCVfnfUnRmOpDOfNum}{\saildoclabelled{sailRISCVfnzfzyunzyrmzyopzyDzyofzynum}{\saildocfn{}{\lstinputlisting[language=sail]{sail_latex_riscv/fnzf_un_rm_op_d_of_num3c7528aaf2a777986f767617d6c66717.tex}}}}

\newcommand{\sailRISCVvalnumOfFUnRmOpD}{\saildoclabelled{sailRISCVznumzyofzyfzyunzyrmzyopzyD}{\saildocval{}{\lstinputlisting[language=sail]{sail_latex_riscv/valznum_of_f_un_rm_op_d6e3378f59faf04b5d017d7c996a625e0.tex}}}}

\newcommand{\sailRISCVfnnumOfFUnRmOpD}{\saildoclabelled{sailRISCVfnznumzyofzyfzyunzyrmzyopzyD}{\saildocfn{}{\lstinputlisting[language=sail]{sail_latex_riscv/fnznum_of_f_un_rm_op_d6e3378f59faf04b5d017d7c996a625e0.tex}}}}

\newcommand{\sailRISCVtypefBinOpD}{\saildoclabelled{sailRISCVtypezfzybinzyopzyD}{\saildoctype{}{\lstinputlisting[language=sail]{sail_latex_riscv/typezf_bin_op_d997af8255aaaca7c24b987d847247564.tex}}}}

\newcommand{\sailRISCVvalfBinOpDOfNum}{\saildoclabelled{sailRISCVzfzybinzyopzyDzyofzynum}{\saildocval{}{\lstinputlisting[language=sail]{sail_latex_riscv/valzf_bin_op_d_of_num97322de52afcea7a7f630a9ec29b7900.tex}}}}

\newcommand{\sailRISCVfnfBinOpDOfNum}{\saildoclabelled{sailRISCVfnzfzybinzyopzyDzyofzynum}{\saildocfn{}{\lstinputlisting[language=sail]{sail_latex_riscv/fnzf_bin_op_d_of_num97322de52afcea7a7f630a9ec29b7900.tex}}}}

\newcommand{\sailRISCVvalnumOfFBinOpD}{\saildoclabelled{sailRISCVznumzyofzyfzybinzyopzyD}{\saildocval{}{\lstinputlisting[language=sail]{sail_latex_riscv/valznum_of_f_bin_op_dd209f84b003673fbf65e2d0267de89cd.tex}}}}

\newcommand{\sailRISCVfnnumOfFBinOpD}{\saildoclabelled{sailRISCVfnznumzyofzyfzybinzyopzyD}{\saildocfn{}{\lstinputlisting[language=sail]{sail_latex_riscv/fnznum_of_f_bin_op_dd209f84b003673fbf65e2d0267de89cd.tex}}}}

\newcommand{\sailRISCVtypefUnOpD}{\saildoclabelled{sailRISCVtypezfzyunzyopzyD}{\saildoctype{}{\lstinputlisting[language=sail]{sail_latex_riscv/typezf_un_op_d60d5205abfb9c0cba7b1e1dbdf6ad69e.tex}}}}

\newcommand{\sailRISCVvalfUnOpDOfNum}{\saildoclabelled{sailRISCVzfzyunzyopzyDzyofzynum}{\saildocval{}{\lstinputlisting[language=sail]{sail_latex_riscv/valzf_un_op_d_of_num20c93bb1c359c6ff0f65b3c91dbe8c85.tex}}}}

\newcommand{\sailRISCVfnfUnOpDOfNum}{\saildoclabelled{sailRISCVfnzfzyunzyopzyDzyofzynum}{\saildocfn{}{\lstinputlisting[language=sail]{sail_latex_riscv/fnzf_un_op_d_of_num20c93bb1c359c6ff0f65b3c91dbe8c85.tex}}}}

\newcommand{\sailRISCVvalnumOfFUnOpD}{\saildoclabelled{sailRISCVznumzyofzyfzyunzyopzyD}{\saildocval{}{\lstinputlisting[language=sail]{sail_latex_riscv/valznum_of_f_un_op_d6d388f339a377c4a985a55a3c728bb8f.tex}}}}

\newcommand{\sailRISCVfnnumOfFUnOpD}{\saildoclabelled{sailRISCVfnznumzyofzyfzyunzyopzyD}{\saildocfn{}{\lstinputlisting[language=sail]{sail_latex_riscv/fnznum_of_f_un_op_d6d388f339a377c4a985a55a3c728bb8f.tex}}}}

\newcommand{\sailRISCVvalcsrNameMap}{\saildoclabelled{sailRISCVzcsrzynamezymap}{\saildocval{}{\lstinputlisting[language=sail]{sail_latex_riscv/valzcsr_name_map043e1d5f928269d79e6253854765ef21.tex}}}}

\newcommand{\sailRISCVvalcsrName}{\saildoclabelled{sailRISCVzcsrzyname}{\saildocval{}{\lstinputlisting[language=sail]{sail_latex_riscv/valzcsr_name355619c0d72f0a56dfaf2d45f4b72967.tex}}}}

\newcommand{\sailRISCVoverloadFFFtoStr}{\saildoclabelled{sailRISCVoverloadFFFztozystr}{\saildocoverload{}{\lstinputlisting[language=sail]{sail_latex_riscv/overloadFFFzto_str8b7a6895ae35945bd4740e9f790c43ee.tex}}}}

\newcommand{\sailRISCVvalextIsCSRDefined}{\saildoclabelled{sailRISCVzextzyiszyCSRzydefined}{\saildocval{}{\lstinputlisting[language=sail]{sail_latex_riscv/valzext_is_csr_defined3e2540173eaa97b3902070bdfa6d0f6f.tex}}}}

\newcommand{\sailRISCVvalextReadCSR}{\saildoclabelled{sailRISCVzextzyreadzyCSR}{\saildocval{}{\lstinputlisting[language=sail]{sail_latex_riscv/valzext_read_csr8af202f75b7d6e7536c08d920bd54264.tex}}}}

\newcommand{\sailRISCVvalextWriteCSR}{\saildoclabelled{sailRISCVzextzywritezyCSR}{\saildocval{}{\lstinputlisting[language=sail]{sail_latex_riscv/valzext_write_csrea3e63f4d0be7079660a260c43b112cd.tex}}}}

\newcommand{\sailRISCVvalscrNameMap}{\saildoclabelled{sailRISCVzscrzynamezymap}{\saildocval{}{\lstinputlisting[language=sail]{sail_latex_riscv/valzscr_name_map4b235a67e525de15ab35ff501d369c01.tex}}}}

\newcommand{\sailRISCVvalscrName}{\saildoclabelled{sailRISCVzscrzyname}{\saildocval{}{\lstinputlisting[language=sail]{sail_latex_riscv/valzscr_name18d33d5c24513d7598c403423ce2f8e6.tex}}}}

\newcommand{\sailRISCVoverloadGGGtoStr}{\saildoclabelled{sailRISCVoverloadGGGztozystr}{\saildocoverload{}{\lstinputlisting[language=sail]{sail_latex_riscv/overloadGGGzto_str8b7a6895ae35945bd4740e9f790c43ee.tex}}}}

\newcommand{\sailRISCVtypeextAccessType}{\saildoclabelled{sailRISCVtypezextzyaccesszytype}{\saildoctype{}{\lstinputlisting[language=sail]{sail_latex_riscv/typezext_access_type8c60f923ce211fcdc0c77548b866673e.tex}}}}

\newcommand{\sailRISCVvalextAccessTypeOfNum}{\saildoclabelled{sailRISCVzextzyaccesszytypezyofzynum}{\saildocval{}{\lstinputlisting[language=sail]{sail_latex_riscv/valzext_access_type_of_num6a15c4e70d2e9820f649a1cc6760e30a.tex}}}}

\newcommand{\sailRISCVfnextAccessTypeOfNum}{\saildoclabelled{sailRISCVfnzextzyaccesszytypezyofzynum}{\saildocfn{}{\lstinputlisting[language=sail]{sail_latex_riscv/fnzext_access_type_of_num6a15c4e70d2e9820f649a1cc6760e30a.tex}}}}

\newcommand{\sailRISCVvalnumOfExtAccessType}{\saildoclabelled{sailRISCVznumzyofzyextzyaccesszytype}{\saildocval{}{\lstinputlisting[language=sail]{sail_latex_riscv/valznum_of_ext_access_typea973217ed477a3d18820058dd3b21729.tex}}}}

\newcommand{\sailRISCVfnnumOfExtAccessType}{\saildoclabelled{sailRISCVfnznumzyofzyextzyaccesszytype}{\saildocfn{}{\lstinputlisting[language=sail]{sail_latex_riscv/fnznum_of_ext_access_typea973217ed477a3d18820058dd3b21729.tex}}}}

\newcommand{\sailRISCVletdefaultWriteAcc}{\saildoclabelled{sailRISCVletzdefaultzywritezyacc}{\saildoclet{}{\lstinputlisting[language=sail]{sail_latex_riscv/letzdefault_write_acca9d9d5ea89f91f937a8fd777a92ccb89.tex}}}}

\newcommand{\sailRISCVvalaccessTypeToStr}{\saildoclabelled{sailRISCVzaccessTypezytozystr}{\saildocval{}{\lstinputlisting[language=sail]{sail_latex_riscv/valzaccesstype_to_str58f7a46d6b3e326411426e3cf0fe52cf.tex}}}}

\newcommand{\sailRISCVfnaccessTypeToStr}{\saildoclabelled{sailRISCVfnzaccessTypezytozystr}{\saildocfn{}{\lstinputlisting[language=sail]{sail_latex_riscv/fnzaccesstype_to_str58f7a46d6b3e326411426e3cf0fe52cf.tex}}}}

\newcommand{\sailRISCVoverloadHHHtoStr}{\saildoclabelled{sailRISCVoverloadHHHztozystr}{\saildocoverload{}{\lstinputlisting[language=sail]{sail_latex_riscv/overloadHHHzto_str8b7a6895ae35945bd4740e9f790c43ee.tex}}}}

\newcommand{\sailRISCVregisterPC}{\saildoclabelled{sailRISCVregisterzPC}{\saildocregister{}{\lstinputlisting[language=sail]{sail_latex_riscv/registerzpcedbec7ccca699bef927f7d0d14fdf9d7.tex}}}}

\newcommand{\sailRISCVregisternextPC}{\saildoclabelled{sailRISCVregisterznextPC}{\saildocregister{}{\lstinputlisting[language=sail]{sail_latex_riscv/registerznextpcd2e19cc4b7215aff19cb2f9ba5652a3f.tex}}}}

\newcommand{\sailRISCVregisterinstbits}{\saildoclabelled{sailRISCVregisterzinstbits}{\saildocregister{}{\lstinputlisting[language=sail]{sail_latex_riscv/registerzinstbitsf085fa3f9045f37ea35bf6b43c50aa3e.tex}}}}

\newcommand{\sailRISCVregisterxOne}{\saildoclabelled{sailRISCVregisterzx1}{\saildocregister{}{\lstinputlisting[language=sail]{sail_latex_riscv/registerzx1fe5c4f248eef676e5a146a72e236d1f8.tex}}}}

\newcommand{\sailRISCVregisterxTwo}{\saildoclabelled{sailRISCVregisterzx2}{\saildocregister{}{\lstinputlisting[language=sail]{sail_latex_riscv/registerzx278942aa883db478f0d834604d74d6de2.tex}}}}

\newcommand{\sailRISCVregisterxThree}{\saildoclabelled{sailRISCVregisterzx3}{\saildocregister{}{\lstinputlisting[language=sail]{sail_latex_riscv/registerzx34e9c0422bae3139dadbb04561846d00b.tex}}}}

\newcommand{\sailRISCVregisterxFour}{\saildoclabelled{sailRISCVregisterzx4}{\saildocregister{}{\lstinputlisting[language=sail]{sail_latex_riscv/registerzx4199c28f5956631dca2e1897880aa83e2.tex}}}}

\newcommand{\sailRISCVregisterxFive}{\saildoclabelled{sailRISCVregisterzx5}{\saildocregister{}{\lstinputlisting[language=sail]{sail_latex_riscv/registerzx5bd044f8216696c2e47f329fea874ea1e.tex}}}}

\newcommand{\sailRISCVregisterxSix}{\saildoclabelled{sailRISCVregisterzx6}{\saildocregister{}{\lstinputlisting[language=sail]{sail_latex_riscv/registerzx6ed961bc5878899dfdad665518b966929.tex}}}}

\newcommand{\sailRISCVregisterxSeven}{\saildoclabelled{sailRISCVregisterzx7}{\saildocregister{}{\lstinputlisting[language=sail]{sail_latex_riscv/registerzx7ff8d445f2307beb1e120d4503889624d.tex}}}}

\newcommand{\sailRISCVregisterxEight}{\saildoclabelled{sailRISCVregisterzx8}{\saildocregister{}{\lstinputlisting[language=sail]{sail_latex_riscv/registerzx8bee7877b92dadf4c0acdc2a5191e0141.tex}}}}

\newcommand{\sailRISCVregisterxNine}{\saildoclabelled{sailRISCVregisterzx9}{\saildocregister{}{\lstinputlisting[language=sail]{sail_latex_riscv/registerzx9486a4635a7ae5a4baad5528daaa91335.tex}}}}

\newcommand{\sailRISCVregisterxOneZero}{\saildoclabelled{sailRISCVregisterzx10}{\saildocregister{}{\lstinputlisting[language=sail]{sail_latex_riscv/registerzx10075fda2655332c94468adf6d9cfee167.tex}}}}

\newcommand{\sailRISCVregisterxOneOne}{\saildoclabelled{sailRISCVregisterzx11}{\saildocregister{}{\lstinputlisting[language=sail]{sail_latex_riscv/registerzx1117d1c4a47980afdf5cd34b96000da662.tex}}}}

\newcommand{\sailRISCVregisterxOneTwo}{\saildoclabelled{sailRISCVregisterzx12}{\saildocregister{}{\lstinputlisting[language=sail]{sail_latex_riscv/registerzx123b94f9bb839a88ea045b5dc2b77722ad.tex}}}}

\newcommand{\sailRISCVregisterxOneThree}{\saildoclabelled{sailRISCVregisterzx13}{\saildocregister{}{\lstinputlisting[language=sail]{sail_latex_riscv/registerzx1321fb69fac3bc3f8340194e8326c1fb66.tex}}}}

\newcommand{\sailRISCVregisterxOneFour}{\saildoclabelled{sailRISCVregisterzx14}{\saildocregister{}{\lstinputlisting[language=sail]{sail_latex_riscv/registerzx1453b9e3c8d0761c7981695b0be29c94a3.tex}}}}

\newcommand{\sailRISCVregisterxOneFive}{\saildoclabelled{sailRISCVregisterzx15}{\saildocregister{}{\lstinputlisting[language=sail]{sail_latex_riscv/registerzx153a193cb31d15cb47ac4766bb7e8cf537.tex}}}}

\newcommand{\sailRISCVregisterxOneSix}{\saildoclabelled{sailRISCVregisterzx16}{\saildocregister{}{\lstinputlisting[language=sail]{sail_latex_riscv/registerzx16cba34caf0395c529be230657952c7920.tex}}}}

\newcommand{\sailRISCVregisterxOneSeven}{\saildoclabelled{sailRISCVregisterzx17}{\saildocregister{}{\lstinputlisting[language=sail]{sail_latex_riscv/registerzx17f1bdcadd9ed4be368e419a76043d65f6.tex}}}}

\newcommand{\sailRISCVregisterxOneEight}{\saildoclabelled{sailRISCVregisterzx18}{\saildocregister{}{\lstinputlisting[language=sail]{sail_latex_riscv/registerzx184fb73d217701da3dc8de6bf6b277c60b.tex}}}}

\newcommand{\sailRISCVregisterxOneNine}{\saildoclabelled{sailRISCVregisterzx19}{\saildocregister{}{\lstinputlisting[language=sail]{sail_latex_riscv/registerzx19c6b1ed9fd909c3c2bba167b24f3e6c6a.tex}}}}

\newcommand{\sailRISCVregisterxTwoZero}{\saildoclabelled{sailRISCVregisterzx20}{\saildocregister{}{\lstinputlisting[language=sail]{sail_latex_riscv/registerzx20e429a2030f03d3ded5ee9f6b91f4a127.tex}}}}

\newcommand{\sailRISCVregisterxTwoOne}{\saildoclabelled{sailRISCVregisterzx21}{\saildocregister{}{\lstinputlisting[language=sail]{sail_latex_riscv/registerzx2186c12f37ab168ea482320db302753f24.tex}}}}

\newcommand{\sailRISCVregisterxTwoTwo}{\saildoclabelled{sailRISCVregisterzx22}{\saildocregister{}{\lstinputlisting[language=sail]{sail_latex_riscv/registerzx228a193307812dfaf39bcd6dc3036fa1f4.tex}}}}

\newcommand{\sailRISCVregisterxTwoThree}{\saildoclabelled{sailRISCVregisterzx23}{\saildocregister{}{\lstinputlisting[language=sail]{sail_latex_riscv/registerzx23bb679db44ed052ce5a935b2ef3f8d7be.tex}}}}

\newcommand{\sailRISCVregisterxTwoFour}{\saildoclabelled{sailRISCVregisterzx24}{\saildocregister{}{\lstinputlisting[language=sail]{sail_latex_riscv/registerzx245872cc178d98050751041559e753ba08.tex}}}}

\newcommand{\sailRISCVregisterxTwoFive}{\saildoclabelled{sailRISCVregisterzx25}{\saildocregister{}{\lstinputlisting[language=sail]{sail_latex_riscv/registerzx2534a96aa59bd5c26a28998e3e56b9d8a9.tex}}}}

\newcommand{\sailRISCVregisterxTwoSix}{\saildoclabelled{sailRISCVregisterzx26}{\saildocregister{}{\lstinputlisting[language=sail]{sail_latex_riscv/registerzx26de04751d1f5fca723d845aad3071b708.tex}}}}

\newcommand{\sailRISCVregisterxTwoSeven}{\saildoclabelled{sailRISCVregisterzx27}{\saildocregister{}{\lstinputlisting[language=sail]{sail_latex_riscv/registerzx277a706a5d52e0e13aa438642158730295.tex}}}}

\newcommand{\sailRISCVregisterxTwoEight}{\saildoclabelled{sailRISCVregisterzx28}{\saildocregister{}{\lstinputlisting[language=sail]{sail_latex_riscv/registerzx28cf9356f207301341721307c091207286.tex}}}}

\newcommand{\sailRISCVregisterxTwoNine}{\saildoclabelled{sailRISCVregisterzx29}{\saildocregister{}{\lstinputlisting[language=sail]{sail_latex_riscv/registerzx290c8746fc9b20b75925e58e405c4ec2d0.tex}}}}

\newcommand{\sailRISCVregisterxThreeZero}{\saildoclabelled{sailRISCVregisterzx30}{\saildocregister{}{\lstinputlisting[language=sail]{sail_latex_riscv/registerzx30fd294a82b0b00107c35a3e4d1cc451f5.tex}}}}

\newcommand{\sailRISCVregisterxThreeOne}{\saildoclabelled{sailRISCVregisterzx31}{\saildocregister{}{\lstinputlisting[language=sail]{sail_latex_riscv/registerzx31708623615f01cb79bd178a411a5bca81.tex}}}}

\newcommand{\sailRISCVvalrX}{\saildoclabelled{sailRISCVzrX}{\saildocval{}{\lstinputlisting[language=sail]{sail_latex_riscv/valzrxa8aad9466d0653707390b940aa9282e7.tex}}}}

\newcommand{\sailRISCVfnrX}{\saildoclabelled{sailRISCVfnzrX}{\saildocfn{}{\lstinputlisting[language=sail]{sail_latex_riscv/fnzrxa8aad9466d0653707390b940aa9282e7.tex}}}}

\newcommand{\sailRISCVvalrvfiWX}{\saildoclabelled{sailRISCVzrvfizywX}{\saildocval{}{\lstinputlisting[language=sail]{sail_latex_riscv/valzrvfi_wxed842ecfeb56ef18626194f2f22935f3.tex}}}}

\newcommand{\sailRISCVfnrvfiWX}{\saildoclabelled{sailRISCVfnzrvfizywX}{\saildocfn{}{\lstinputlisting[language=sail]{sail_latex_riscv/fnzrvfi_wxed842ecfeb56ef18626194f2f22935f3.tex}}}}

\newcommand{\sailRISCVvalwX}{\saildoclabelled{sailRISCVzwX}{\saildocval{}{\lstinputlisting[language=sail]{sail_latex_riscv/valzwx0042b1ee0bdb45d47dcb45d5a9461882.tex}}}}

\newcommand{\sailRISCVfnwX}{\saildoclabelled{sailRISCVfnzwX}{\saildocfn{}{\lstinputlisting[language=sail]{sail_latex_riscv/fnzwx0042b1ee0bdb45d47dcb45d5a9461882.tex}}}}

\newcommand{\sailRISCVvalrXBits}{\saildoclabelled{sailRISCVzrXzybits}{\saildocval{}{\lstinputlisting[language=sail]{sail_latex_riscv/valzrx_bitsba4d35e6c426ac476fdbf36efdd5d0da.tex}}}}

\newcommand{\sailRISCVfnrXBits}{\saildoclabelled{sailRISCVfnzrXzybits}{\saildocfn{}{\lstinputlisting[language=sail]{sail_latex_riscv/fnzrx_bitsba4d35e6c426ac476fdbf36efdd5d0da.tex}}}}

\newcommand{\sailRISCVvalwXBits}{\saildoclabelled{sailRISCVzwXzybits}{\saildocval{}{\lstinputlisting[language=sail]{sail_latex_riscv/valzwx_bitseb6ef5be72b31b8cb0f1595602665261.tex}}}}

\newcommand{\sailRISCVfnwXBits}{\saildoclabelled{sailRISCVfnzwXzybits}{\saildocfn{}{\lstinputlisting[language=sail]{sail_latex_riscv/fnzwx_bitseb6ef5be72b31b8cb0f1595602665261.tex}}}}

\newcommand{\sailRISCVoverloadIIIX}{\saildoclabelled{sailRISCVoverloadIIIzX}{\saildocoverload{}{\lstinputlisting[language=sail]{sail_latex_riscv/overloadIIIzx1f3c57dd04ac52fd92f3346ca51f00ec.tex}}}}

\newcommand{\sailRISCVvalregNameAbi}{\saildoclabelled{sailRISCVzregzynamezyabi}{\saildocval{}{\lstinputlisting[language=sail]{sail_latex_riscv/valzreg_name_abi8c36e923dc671675cb54fb0175878a3f.tex}}}}

\newcommand{\sailRISCVfnregNameAbi}{\saildoclabelled{sailRISCVfnzregzynamezyabi}{\saildocfn{}{\lstinputlisting[language=sail]{sail_latex_riscv/fnzreg_name_abi8c36e923dc671675cb54fb0175878a3f.tex}}}}

\newcommand{\sailRISCVoverloadJJJtoStr}{\saildoclabelled{sailRISCVoverloadJJJztozystr}{\saildocoverload{}{\lstinputlisting[language=sail]{sail_latex_riscv/overloadJJJzto_str8b7a6895ae35945bd4740e9f790c43ee.tex}}}}

\newcommand{\sailRISCVvalregName}{\saildoclabelled{sailRISCVzregzyname}{\saildocval{}{\lstinputlisting[language=sail]{sail_latex_riscv/valzreg_namea3a934e14215c11bd841e5d2fb4f53e0.tex}}}}

\newcommand{\sailRISCVvalcregName}{\saildoclabelled{sailRISCVzcregzyname}{\saildocval{}{\lstinputlisting[language=sail]{sail_latex_riscv/valzcreg_name8aa7dd2adef25f1d7c38341bff05c8a9.tex}}}}

\newcommand{\sailRISCVvalinitBaseRegs}{\saildoclabelled{sailRISCVzinitzybasezyregs}{\saildocval{}{\lstinputlisting[language=sail]{sail_latex_riscv/valzinit_base_regs92fa12c31a7794db853235a9147b1c7b.tex}}}}

\newcommand{\sailRISCVfninitBaseRegs}{\saildoclabelled{sailRISCVfnzinitzybasezyregs}{\saildocfn{}{\lstinputlisting[language=sail]{sail_latex_riscv/fnzinit_base_regs92fa12c31a7794db853235a9147b1c7b.tex}}}}

\newcommand{\sailRISCVregistercurPrivilege}{\saildoclabelled{sailRISCVregisterzcurzyprivilege}{\saildocregister{}{\lstinputlisting[language=sail]{sail_latex_riscv/registerzcur_privilegea18daa407d5b419b6c11d0ebdecf3f7a.tex}}}}

\newcommand{\sailRISCVregistercurInst}{\saildoclabelled{sailRISCVregisterzcurzyinst}{\saildocregister{}{\lstinputlisting[language=sail]{sail_latex_riscv/registerzcur_instaaa0450d0c8f1ce0d022c56fa4a175d3.tex}}}}

\newcommand{\sailRISCVtypeMisa}{\saildoclabelled{sailRISCVtypezMisa}{\saildoctype{}{\lstinputlisting[language=sail]{sail_latex_riscv/typezmisaa32c2c216e0702e03f7fa6ac32ade17d.tex}}}}

\newcommand{\sailRISCVregistermisa}{\saildoclabelled{sailRISCVregisterzmisa}{\saildocregister{}{\lstinputlisting[language=sail]{sail_latex_riscv/registerzmisa5cceeb295703ff8a973abba52d6ee067.tex}}}}

\newcommand{\sailRISCVvalsysEnableWritableMisa}{\saildoclabelled{sailRISCVzsyszyenablezywritablezymisa}{\saildocval{}{\lstinputlisting[language=sail]{sail_latex_riscv/valzsys_enable_writable_misa83863aeddc0d69035dc46d6908595d89.tex}}}}

\newcommand{\sailRISCVvalsysEnableRvc}{\saildoclabelled{sailRISCVzsyszyenablezyrvc}{\saildocval{}{\lstinputlisting[language=sail]{sail_latex_riscv/valzsys_enable_rvc836db920a63cd54b635b109fb85bcb90.tex}}}}

\newcommand{\sailRISCVvalsysEnableFdext}{\saildoclabelled{sailRISCVzsyszyenablezyfdext}{\saildocval{}{\lstinputlisting[language=sail]{sail_latex_riscv/valzsys_enable_fdext3e294bd170189b70567069837e1534c5.tex}}}}

\newcommand{\sailRISCVvalsysEnableZfinx}{\saildoclabelled{sailRISCVzsyszyenablezyzzfinx}{\saildocval{}{\lstinputlisting[language=sail]{sail_latex_riscv/valzsys_enable_zzfinx98981c2e0d4657b0b00f46779900ee17.tex}}}}

\newcommand{\sailRISCVvalsysEnableNext}{\saildoclabelled{sailRISCVzsyszyenablezynext}{\saildocval{}{\lstinputlisting[language=sail]{sail_latex_riscv/valzsys_enable_nextd747b3af13c8724684c58a321c2e2ac6.tex}}}}

\newcommand{\sailRISCVvalextVetoDisableC}{\saildoclabelled{sailRISCVzextzyvetozydisablezyC}{\saildocval{}{\lstinputlisting[language=sail]{sail_latex_riscv/valzext_veto_disable_cd10c2d1c5077060fa007c1628d7aaa8c.tex}}}}

\newcommand{\sailRISCVvallegalizzeMisa}{\saildoclabelled{sailRISCVzlegalizzezymisa}{\saildocval{}{\lstinputlisting[language=sail]{sail_latex_riscv/valzlegalizze_misad494764bfeb5d382d189645941a1bce6.tex}}}}

\newcommand{\sailRISCVfnlegalizzeMisa}{\saildoclabelled{sailRISCVfnzlegalizzezymisa}{\saildocfn{}{\lstinputlisting[language=sail]{sail_latex_riscv/fnzlegalizze_misad494764bfeb5d382d189645941a1bce6.tex}}}}

\newcommand{\sailRISCVvalhaveAtomics}{\saildoclabelled{sailRISCVzhaveAtomics}{\saildocval{}{\lstinputlisting[language=sail]{sail_latex_riscv/valzhaveatomics63cb703b1e3f955440fac8b850da53f7.tex}}}}

\newcommand{\sailRISCVfnhaveAtomics}{\saildoclabelled{sailRISCVfnzhaveAtomics}{\saildocfn{}{\lstinputlisting[language=sail]{sail_latex_riscv/fnzhaveatomics63cb703b1e3f955440fac8b850da53f7.tex}}}}

\newcommand{\sailRISCVvalhaveRVC}{\saildoclabelled{sailRISCVzhaveRVC}{\saildocval{}{\lstinputlisting[language=sail]{sail_latex_riscv/valzhavervcd1bab8584f566314057babec4a4bfcce.tex}}}}

\newcommand{\sailRISCVfnhaveRVC}{\saildoclabelled{sailRISCVfnzhaveRVC}{\saildocfn{}{\lstinputlisting[language=sail]{sail_latex_riscv/fnzhavervcd1bab8584f566314057babec4a4bfcce.tex}}}}

\newcommand{\sailRISCVvalhaveMulDiv}{\saildoclabelled{sailRISCVzhaveMulDiv}{\saildocval{}{\lstinputlisting[language=sail]{sail_latex_riscv/valzhavemuldivee13dcf875cf3974336f980d6be89b2a.tex}}}}

\newcommand{\sailRISCVfnhaveMulDiv}{\saildoclabelled{sailRISCVfnzhaveMulDiv}{\saildocfn{}{\lstinputlisting[language=sail]{sail_latex_riscv/fnzhavemuldivee13dcf875cf3974336f980d6be89b2a.tex}}}}

\newcommand{\sailRISCVvalhaveSupMode}{\saildoclabelled{sailRISCVzhaveSupMode}{\saildocval{}{\lstinputlisting[language=sail]{sail_latex_riscv/valzhavesupmode3f08edf2b0386d18c650f34868f384db.tex}}}}

\newcommand{\sailRISCVfnhaveSupMode}{\saildoclabelled{sailRISCVfnzhaveSupMode}{\saildocfn{}{\lstinputlisting[language=sail]{sail_latex_riscv/fnzhavesupmode3f08edf2b0386d18c650f34868f384db.tex}}}}

\newcommand{\sailRISCVvalhaveUsrMode}{\saildoclabelled{sailRISCVzhaveUsrMode}{\saildocval{}{\lstinputlisting[language=sail]{sail_latex_riscv/valzhaveusrmode2e520828ead805e9613cc1f88f964861.tex}}}}

\newcommand{\sailRISCVfnhaveUsrMode}{\saildoclabelled{sailRISCVfnzhaveUsrMode}{\saildocfn{}{\lstinputlisting[language=sail]{sail_latex_riscv/fnzhaveusrmode2e520828ead805e9613cc1f88f964861.tex}}}}

\newcommand{\sailRISCVvalhaveNExt}{\saildoclabelled{sailRISCVzhaveNExt}{\saildocval{}{\lstinputlisting[language=sail]{sail_latex_riscv/valzhavenext74060caf84942f39cfe17eb66bce56e4.tex}}}}

\newcommand{\sailRISCVfnhaveNExt}{\saildoclabelled{sailRISCVfnzhaveNExt}{\saildocfn{}{\lstinputlisting[language=sail]{sail_latex_riscv/fnzhavenext74060caf84942f39cfe17eb66bce56e4.tex}}}}

\newcommand{\sailRISCVvalhaveZba}{\saildoclabelled{sailRISCVzhaveZba}{\saildocval{}{\lstinputlisting[language=sail]{sail_latex_riscv/valzhavezbad3ff136d4d7db59b82bacb79e3a0b653.tex}}}}

\newcommand{\sailRISCVfnhaveZba}{\saildoclabelled{sailRISCVfnzhaveZba}{\saildocfn{}{\lstinputlisting[language=sail]{sail_latex_riscv/fnzhavezbad3ff136d4d7db59b82bacb79e3a0b653.tex}}}}

\newcommand{\sailRISCVvalhaveZbb}{\saildoclabelled{sailRISCVzhaveZbb}{\saildocval{}{\lstinputlisting[language=sail]{sail_latex_riscv/valzhavezbb2da6236879c9d7a4b4c7378e95fbfe1c.tex}}}}

\newcommand{\sailRISCVfnhaveZbb}{\saildoclabelled{sailRISCVfnzhaveZbb}{\saildocfn{}{\lstinputlisting[language=sail]{sail_latex_riscv/fnzhavezbb2da6236879c9d7a4b4c7378e95fbfe1c.tex}}}}

\newcommand{\sailRISCVvalhaveZbc}{\saildoclabelled{sailRISCVzhaveZbc}{\saildocval{}{\lstinputlisting[language=sail]{sail_latex_riscv/valzhavezbc591ecc7eef9a65932357e179dbde458f.tex}}}}

\newcommand{\sailRISCVfnhaveZbc}{\saildoclabelled{sailRISCVfnzhaveZbc}{\saildocfn{}{\lstinputlisting[language=sail]{sail_latex_riscv/fnzhavezbc591ecc7eef9a65932357e179dbde458f.tex}}}}

\newcommand{\sailRISCVvalhaveZbs}{\saildoclabelled{sailRISCVzhaveZbs}{\saildocval{}{\lstinputlisting[language=sail]{sail_latex_riscv/valzhavezbsb73d7808e3c6776f16235f4053e14f92.tex}}}}

\newcommand{\sailRISCVfnhaveZbs}{\saildoclabelled{sailRISCVfnzhaveZbs}{\saildocfn{}{\lstinputlisting[language=sail]{sail_latex_riscv/fnzhavezbsb73d7808e3c6776f16235f4053e14f92.tex}}}}

\newcommand{\sailRISCVvalhaveZfh}{\saildoclabelled{sailRISCVzhaveZfh}{\saildocval{}{\lstinputlisting[language=sail]{sail_latex_riscv/valzhavezfhe5dbd348c0b6c69910aec2c1edb56adf.tex}}}}

\newcommand{\sailRISCVfnhaveZfh}{\saildoclabelled{sailRISCVfnzhaveZfh}{\saildocfn{}{\lstinputlisting[language=sail]{sail_latex_riscv/fnzhavezfhe5dbd348c0b6c69910aec2c1edb56adf.tex}}}}

\newcommand{\sailRISCVvalhaveZbkb}{\saildoclabelled{sailRISCVzhaveZbkb}{\saildocval{}{\lstinputlisting[language=sail]{sail_latex_riscv/valzhavezbkbb2448f035d58b7d5497beaa1d54a8db4.tex}}}}

\newcommand{\sailRISCVfnhaveZbkb}{\saildoclabelled{sailRISCVfnzhaveZbkb}{\saildocfn{}{\lstinputlisting[language=sail]{sail_latex_riscv/fnzhavezbkbb2448f035d58b7d5497beaa1d54a8db4.tex}}}}

\newcommand{\sailRISCVvalhaveZbkc}{\saildoclabelled{sailRISCVzhaveZbkc}{\saildocval{}{\lstinputlisting[language=sail]{sail_latex_riscv/valzhavezbkc084f911a43cc52023dd3045a3d96c6a9.tex}}}}

\newcommand{\sailRISCVfnhaveZbkc}{\saildoclabelled{sailRISCVfnzhaveZbkc}{\saildocfn{}{\lstinputlisting[language=sail]{sail_latex_riscv/fnzhavezbkc084f911a43cc52023dd3045a3d96c6a9.tex}}}}

\newcommand{\sailRISCVvalhaveZbkx}{\saildoclabelled{sailRISCVzhaveZbkx}{\saildocval{}{\lstinputlisting[language=sail]{sail_latex_riscv/valzhavezbkxd58ea9806aca4e3e3d904d92629349d8.tex}}}}

\newcommand{\sailRISCVfnhaveZbkx}{\saildoclabelled{sailRISCVfnzhaveZbkx}{\saildocfn{}{\lstinputlisting[language=sail]{sail_latex_riscv/fnzhavezbkxd58ea9806aca4e3e3d904d92629349d8.tex}}}}

\newcommand{\sailRISCVvalhaveZkr}{\saildoclabelled{sailRISCVzhaveZkr}{\saildocval{}{\lstinputlisting[language=sail]{sail_latex_riscv/valzhavezkra388aa07dbff228a589243680c811e13.tex}}}}

\newcommand{\sailRISCVfnhaveZkr}{\saildoclabelled{sailRISCVfnzhaveZkr}{\saildocfn{}{\lstinputlisting[language=sail]{sail_latex_riscv/fnzhavezkra388aa07dbff228a589243680c811e13.tex}}}}

\newcommand{\sailRISCVvalhaveZksh}{\saildoclabelled{sailRISCVzhaveZksh}{\saildocval{}{\lstinputlisting[language=sail]{sail_latex_riscv/valzhavezkshb39c29b7d8aaee55e1c92eddd118da40.tex}}}}

\newcommand{\sailRISCVfnhaveZksh}{\saildoclabelled{sailRISCVfnzhaveZksh}{\saildocfn{}{\lstinputlisting[language=sail]{sail_latex_riscv/fnzhavezkshb39c29b7d8aaee55e1c92eddd118da40.tex}}}}

\newcommand{\sailRISCVvalhaveZksed}{\saildoclabelled{sailRISCVzhaveZksed}{\saildocval{}{\lstinputlisting[language=sail]{sail_latex_riscv/valzhavezksed7759d42d27d1f893dea635b592e6d803.tex}}}}

\newcommand{\sailRISCVfnhaveZksed}{\saildoclabelled{sailRISCVfnzhaveZksed}{\saildocfn{}{\lstinputlisting[language=sail]{sail_latex_riscv/fnzhavezksed7759d42d27d1f893dea635b592e6d803.tex}}}}

\newcommand{\sailRISCVvalhaveZknh}{\saildoclabelled{sailRISCVzhaveZknh}{\saildocval{}{\lstinputlisting[language=sail]{sail_latex_riscv/valzhavezknhae02e997dcdde1530c835ca51d53c429.tex}}}}

\newcommand{\sailRISCVfnhaveZknh}{\saildoclabelled{sailRISCVfnzhaveZknh}{\saildocfn{}{\lstinputlisting[language=sail]{sail_latex_riscv/fnzhavezknhae02e997dcdde1530c835ca51d53c429.tex}}}}

\newcommand{\sailRISCVvalhaveZkne}{\saildoclabelled{sailRISCVzhaveZkne}{\saildocval{}{\lstinputlisting[language=sail]{sail_latex_riscv/valzhavezkne417190fb252e4c3af4ce5d4f230dc054.tex}}}}

\newcommand{\sailRISCVfnhaveZkne}{\saildoclabelled{sailRISCVfnzhaveZkne}{\saildocfn{}{\lstinputlisting[language=sail]{sail_latex_riscv/fnzhavezkne417190fb252e4c3af4ce5d4f230dc054.tex}}}}

\newcommand{\sailRISCVvalhaveZknd}{\saildoclabelled{sailRISCVzhaveZknd}{\saildocval{}{\lstinputlisting[language=sail]{sail_latex_riscv/valzhavezkndc57f53df87b039d70392edec17c7c90e.tex}}}}

\newcommand{\sailRISCVfnhaveZknd}{\saildoclabelled{sailRISCVfnzhaveZknd}{\saildocfn{}{\lstinputlisting[language=sail]{sail_latex_riscv/fnzhavezkndc57f53df87b039d70392edec17c7c90e.tex}}}}

\newcommand{\sailRISCVvalhaveZmmul}{\saildoclabelled{sailRISCVzhaveZmmul}{\saildocval{}{\lstinputlisting[language=sail]{sail_latex_riscv/valzhavezmmul9f897872e00093149d077731f7a24ffb.tex}}}}

\newcommand{\sailRISCVfnhaveZmmul}{\saildoclabelled{sailRISCVfnzhaveZmmul}{\saildocfn{}{\lstinputlisting[language=sail]{sail_latex_riscv/fnzhavezmmul9f897872e00093149d077731f7a24ffb.tex}}}}

\newcommand{\sailRISCVtypeMstatush}{\saildoclabelled{sailRISCVtypezMstatush}{\saildoctype{}{\lstinputlisting[language=sail]{sail_latex_riscv/typezmstatushbe19a0fd111f7cf462e74397c744ddef.tex}}}}

\newcommand{\sailRISCVregistermstatush}{\saildoclabelled{sailRISCVregisterzmstatush}{\saildocregister{}{\lstinputlisting[language=sail]{sail_latex_riscv/registerzmstatush99e415e2be0d9f55ef4d84aa3694681f.tex}}}}

\newcommand{\sailRISCVtypeMstatus}{\saildoclabelled{sailRISCVtypezMstatus}{\saildoctype{}{\lstinputlisting[language=sail]{sail_latex_riscv/typezmstatus6adeedf6ca37e8c03608d356a04db81f.tex}}}}

\newcommand{\sailRISCVregistermstatus}{\saildoclabelled{sailRISCVregisterzmstatus}{\saildocregister{}{\lstinputlisting[language=sail]{sail_latex_riscv/registerzmstatus8a1c8132987625b260b1ff56aa2d3463.tex}}}}

\newcommand{\sailRISCVvaleffectivePrivilege}{\saildoclabelled{sailRISCVzeffectivePrivilege}{\saildocval{}{\lstinputlisting[language=sail]{sail_latex_riscv/valzeffectiveprivilegeb14beb91b9202c1aefb99c45054bcb23.tex}}}}

\newcommand{\sailRISCVfneffectivePrivilege}{\saildoclabelled{sailRISCVfnzeffectivePrivilege}{\saildocfn{}{\lstinputlisting[language=sail]{sail_latex_riscv/fnzeffectiveprivilegeb14beb91b9202c1aefb99c45054bcb23.tex}}}}

\newcommand{\sailRISCVvalgetMstatusSXL}{\saildoclabelled{sailRISCVzgetzymstatuszySXL}{\saildocval{}{\lstinputlisting[language=sail]{sail_latex_riscv/valzget_mstatus_sxl360b4b5f8d3130ad7ed7bb536d1d7019.tex}}}}

\newcommand{\sailRISCVfngetMstatusSXL}{\saildoclabelled{sailRISCVfnzgetzymstatuszySXL}{\saildocfn{}{\lstinputlisting[language=sail]{sail_latex_riscv/fnzget_mstatus_sxl360b4b5f8d3130ad7ed7bb536d1d7019.tex}}}}

\newcommand{\sailRISCVvalsetMstatusSXL}{\saildoclabelled{sailRISCVzsetzymstatuszySXL}{\saildocval{}{\lstinputlisting[language=sail]{sail_latex_riscv/valzset_mstatus_sxl3c993e2659d94bb903dbbc607aa47f8c.tex}}}}

\newcommand{\sailRISCVfnsetMstatusSXL}{\saildoclabelled{sailRISCVfnzsetzymstatuszySXL}{\saildocfn{}{\lstinputlisting[language=sail]{sail_latex_riscv/fnzset_mstatus_sxl3c993e2659d94bb903dbbc607aa47f8c.tex}}}}

\newcommand{\sailRISCVvalgetMstatusUXL}{\saildoclabelled{sailRISCVzgetzymstatuszyUXL}{\saildocval{}{\lstinputlisting[language=sail]{sail_latex_riscv/valzget_mstatus_uxl7f41d7b1028a3b1a7e11b635a0524660.tex}}}}

\newcommand{\sailRISCVfngetMstatusUXL}{\saildoclabelled{sailRISCVfnzgetzymstatuszyUXL}{\saildocfn{}{\lstinputlisting[language=sail]{sail_latex_riscv/fnzget_mstatus_uxl7f41d7b1028a3b1a7e11b635a0524660.tex}}}}

\newcommand{\sailRISCVvalsetMstatusUXL}{\saildoclabelled{sailRISCVzsetzymstatuszyUXL}{\saildocval{}{\lstinputlisting[language=sail]{sail_latex_riscv/valzset_mstatus_uxl030f4a70ceca0610271994fbfcb96d4d.tex}}}}

\newcommand{\sailRISCVfnsetMstatusUXL}{\saildoclabelled{sailRISCVfnzsetzymstatuszyUXL}{\saildocfn{}{\lstinputlisting[language=sail]{sail_latex_riscv/fnzset_mstatus_uxl030f4a70ceca0610271994fbfcb96d4d.tex}}}}

\newcommand{\sailRISCVvallegalizzeMstatus}{\saildoclabelled{sailRISCVzlegalizzezymstatus}{\saildocval{}{\lstinputlisting[language=sail]{sail_latex_riscv/valzlegalizze_mstatus99091e0733640797e5a873808232271b.tex}}}}

\newcommand{\sailRISCVfnlegalizzeMstatus}{\saildoclabelled{sailRISCVfnzlegalizzezymstatus}{\saildocfn{}{\lstinputlisting[language=sail]{sail_latex_riscv/fnzlegalizze_mstatus99091e0733640797e5a873808232271b.tex}}}}

\newcommand{\sailRISCVvalcurArchitecture}{\saildoclabelled{sailRISCVzcurzyArchitecture}{\saildocval{}{\lstinputlisting[language=sail]{sail_latex_riscv/valzcur_architecture146db9bbeb421361d412d7f5f7f40511.tex}}}}

\newcommand{\sailRISCVfncurArchitecture}{\saildoclabelled{sailRISCVfnzcurzyArchitecture}{\saildocfn{}{\lstinputlisting[language=sail]{sail_latex_riscv/fnzcur_architecture146db9bbeb421361d412d7f5f7f40511.tex}}}}

\newcommand{\sailRISCVvalinThreeTwoBitMode}{\saildoclabelled{sailRISCVzin32BitMode}{\saildocval{}{\lstinputlisting[language=sail]{sail_latex_riscv/valzin32bitmode1fa9d4b065e1807e1dff8fc7b86cca8d.tex}}}}

\newcommand{\sailRISCVfninThreeTwoBitMode}{\saildoclabelled{sailRISCVfnzin32BitMode}{\saildocfn{}{\lstinputlisting[language=sail]{sail_latex_riscv/fnzin32bitmode1fa9d4b065e1807e1dff8fc7b86cca8d.tex}}}}

\newcommand{\sailRISCVvalhaveFExt}{\saildoclabelled{sailRISCVzhaveFExt}{\saildocval{}{\lstinputlisting[language=sail]{sail_latex_riscv/valzhavefext496ba4ad5d47088bd723ab96ed6356d0.tex}}}}

\newcommand{\sailRISCVfnhaveFExt}{\saildoclabelled{sailRISCVfnzhaveFExt}{\saildocfn{}{\lstinputlisting[language=sail]{sail_latex_riscv/fnzhavefext496ba4ad5d47088bd723ab96ed6356d0.tex}}}}

\newcommand{\sailRISCVvalhaveDExt}{\saildoclabelled{sailRISCVzhaveDExt}{\saildocval{}{\lstinputlisting[language=sail]{sail_latex_riscv/valzhavedextbb6a692871d46f160484ca963a3fffd0.tex}}}}

\newcommand{\sailRISCVfnhaveDExt}{\saildoclabelled{sailRISCVfnzhaveDExt}{\saildocfn{}{\lstinputlisting[language=sail]{sail_latex_riscv/fnzhavedextbb6a692871d46f160484ca963a3fffd0.tex}}}}

\newcommand{\sailRISCVvalhaveZhinx}{\saildoclabelled{sailRISCVzhaveZhinx}{\saildocval{}{\lstinputlisting[language=sail]{sail_latex_riscv/valzhavezhinxd3b2b0b131eb0dedee007e834f875556.tex}}}}

\newcommand{\sailRISCVfnhaveZhinx}{\saildoclabelled{sailRISCVfnzhaveZhinx}{\saildocfn{}{\lstinputlisting[language=sail]{sail_latex_riscv/fnzhavezhinxd3b2b0b131eb0dedee007e834f875556.tex}}}}

\newcommand{\sailRISCVvalhaveZfinx}{\saildoclabelled{sailRISCVzhaveZfinx}{\saildocval{}{\lstinputlisting[language=sail]{sail_latex_riscv/valzhavezfinx6ddc366dddcf5d1b79ae0941c8721386.tex}}}}

\newcommand{\sailRISCVfnhaveZfinx}{\saildoclabelled{sailRISCVfnzhaveZfinx}{\saildocfn{}{\lstinputlisting[language=sail]{sail_latex_riscv/fnzhavezfinx6ddc366dddcf5d1b79ae0941c8721386.tex}}}}

\newcommand{\sailRISCVvalhaveZdinx}{\saildoclabelled{sailRISCVzhaveZdinx}{\saildocval{}{\lstinputlisting[language=sail]{sail_latex_riscv/valzhavezdinxee2a7abd936e45d8abb3544a176bd57a.tex}}}}

\newcommand{\sailRISCVfnhaveZdinx}{\saildoclabelled{sailRISCVfnzhaveZdinx}{\saildocfn{}{\lstinputlisting[language=sail]{sail_latex_riscv/fnzhavezdinxee2a7abd936e45d8abb3544a176bd57a.tex}}}}

\newcommand{\sailRISCVtypeMinterrupts}{\saildoclabelled{sailRISCVtypezMinterrupts}{\saildoctype{}{\lstinputlisting[language=sail]{sail_latex_riscv/typezminterrupts12bbecae04331c3c2bd6fea56af2709c.tex}}}}

\newcommand{\sailRISCVregistermip}{\saildoclabelled{sailRISCVregisterzmip}{\saildocregister{}{\lstinputlisting[language=sail]{sail_latex_riscv/registerzmip6e162b56990b417fd10593013464d75a.tex}}}}

\newcommand{\sailRISCVregistermie}{\saildoclabelled{sailRISCVregisterzmie}{\saildocregister{}{\lstinputlisting[language=sail]{sail_latex_riscv/registerzmie12605a9c54aed05316f649eb9da217d0.tex}}}}

\newcommand{\sailRISCVregistermideleg}{\saildoclabelled{sailRISCVregisterzmideleg}{\saildocregister{}{\lstinputlisting[language=sail]{sail_latex_riscv/registerzmideleg09d034012485cd77f6ef7016cc2ccd91.tex}}}}

\newcommand{\sailRISCVvallegalizzeMip}{\saildoclabelled{sailRISCVzlegalizzezymip}{\saildocval{}{\lstinputlisting[language=sail]{sail_latex_riscv/valzlegalizze_mip6b35097c942aa7a4220cfbae774dd473.tex}}}}

\newcommand{\sailRISCVfnlegalizzeMip}{\saildoclabelled{sailRISCVfnzlegalizzezymip}{\saildocfn{}{\lstinputlisting[language=sail]{sail_latex_riscv/fnzlegalizze_mip6b35097c942aa7a4220cfbae774dd473.tex}}}}

\newcommand{\sailRISCVvallegalizzeMie}{\saildoclabelled{sailRISCVzlegalizzezymie}{\saildocval{}{\lstinputlisting[language=sail]{sail_latex_riscv/valzlegalizze_miea3d44a6b9220a2e1cbd134912a2b63af.tex}}}}

\newcommand{\sailRISCVfnlegalizzeMie}{\saildoclabelled{sailRISCVfnzlegalizzezymie}{\saildocfn{}{\lstinputlisting[language=sail]{sail_latex_riscv/fnzlegalizze_miea3d44a6b9220a2e1cbd134912a2b63af.tex}}}}

\newcommand{\sailRISCVvallegalizzeMideleg}{\saildoclabelled{sailRISCVzlegalizzezymideleg}{\saildocval{}{\lstinputlisting[language=sail]{sail_latex_riscv/valzlegalizze_mideleg301fb79c822ffefd5b20e289875da3e4.tex}}}}

\newcommand{\sailRISCVfnlegalizzeMideleg}{\saildoclabelled{sailRISCVfnzlegalizzezymideleg}{\saildocfn{}{\lstinputlisting[language=sail]{sail_latex_riscv/fnzlegalizze_mideleg301fb79c822ffefd5b20e289875da3e4.tex}}}}

\newcommand{\sailRISCVtypeMedeleg}{\saildoclabelled{sailRISCVtypezMedeleg}{\saildoctype{}{\lstinputlisting[language=sail]{sail_latex_riscv/typezmedeleg2ea875f42b3c32731b094792df4272ce.tex}}}}

\newcommand{\sailRISCVregistermedeleg}{\saildoclabelled{sailRISCVregisterzmedeleg}{\saildocregister{}{\lstinputlisting[language=sail]{sail_latex_riscv/registerzmedeleg3c729756e7c10c6c2d5901927a42e9b4.tex}}}}

\newcommand{\sailRISCVvallegalizzeMedeleg}{\saildoclabelled{sailRISCVzlegalizzezymedeleg}{\saildocval{}{\lstinputlisting[language=sail]{sail_latex_riscv/valzlegalizze_medeleg4472d564e1553d1bbac6c6cbaeff95c9.tex}}}}

\newcommand{\sailRISCVfnlegalizzeMedeleg}{\saildoclabelled{sailRISCVfnzlegalizzezymedeleg}{\saildocfn{}{\lstinputlisting[language=sail]{sail_latex_riscv/fnzlegalizze_medeleg4472d564e1553d1bbac6c6cbaeff95c9.tex}}}}

\newcommand{\sailRISCVtypeMtvec}{\saildoclabelled{sailRISCVtypezMtvec}{\saildoctype{}{\lstinputlisting[language=sail]{sail_latex_riscv/typezmtvec31778e1eff09235941858a1d4774e45c.tex}}}}

\newcommand{\sailRISCVregistermtvec}{\saildoclabelled{sailRISCVregisterzmtvec}{\saildocregister{}{\lstinputlisting[language=sail]{sail_latex_riscv/registerzmtvec2eb8723a42702325325376f85233a2de.tex}}}}

\newcommand{\sailRISCVvallegalizzeTvec}{\saildoclabelled{sailRISCVzlegalizzezytvec}{\saildocval{}{\lstinputlisting[language=sail]{sail_latex_riscv/valzlegalizze_tvec4a108f656cfc40d55dcc9e2cd9614e5c.tex}}}}

\newcommand{\sailRISCVfnlegalizzeTvec}{\saildoclabelled{sailRISCVfnzlegalizzezytvec}{\saildocfn{}{\lstinputlisting[language=sail]{sail_latex_riscv/fnzlegalizze_tvec4a108f656cfc40d55dcc9e2cd9614e5c.tex}}}}

\newcommand{\sailRISCVtypeMcause}{\saildoclabelled{sailRISCVtypezMcause}{\saildoctype{}{\lstinputlisting[language=sail]{sail_latex_riscv/typezmcausea662f8fdb3c01a6efcbdc5abf35af73e.tex}}}}

\newcommand{\sailRISCVregistermcause}{\saildoclabelled{sailRISCVregisterzmcause}{\saildocregister{}{\lstinputlisting[language=sail]{sail_latex_riscv/registerzmcauseb2ff15d464e637064b6f9d84c5206df2.tex}}}}

\newcommand{\sailRISCVvaltvecAddr}{\saildoclabelled{sailRISCVztveczyaddr}{\saildocval{}{\lstinputlisting[language=sail]{sail_latex_riscv/valztvec_addrc6c52b287eacbc1aca798406d1b08576.tex}}}}

\newcommand{\sailRISCVfntvecAddr}{\saildoclabelled{sailRISCVfnztveczyaddr}{\saildocfn{}{\lstinputlisting[language=sail]{sail_latex_riscv/fnztvec_addrc6c52b287eacbc1aca798406d1b08576.tex}}}}

\newcommand{\sailRISCVregistermepc}{\saildoclabelled{sailRISCVregisterzmepc}{\saildocregister{}{\lstinputlisting[language=sail]{sail_latex_riscv/registerzmepcc876acb302a918827e970def4e995c8e.tex}}}}

\newcommand{\sailRISCVvallegalizzeXepc}{\saildoclabelled{sailRISCVzlegalizzezyxepc}{\saildocval{}{\lstinputlisting[language=sail]{sail_latex_riscv/valzlegalizze_xepc8416dd9133f6232df0898ca0ae2784c2.tex}}}}

\newcommand{\sailRISCVfnlegalizzeXepc}{\saildoclabelled{sailRISCVfnzlegalizzezyxepc}{\saildocfn{}{\lstinputlisting[language=sail]{sail_latex_riscv/fnzlegalizze_xepc8416dd9133f6232df0898ca0ae2784c2.tex}}}}

\newcommand{\sailRISCVvalpcAlignmentMask}{\saildoclabelled{sailRISCVzpczyalignmentzymask}{\saildocval{}{\lstinputlisting[language=sail]{sail_latex_riscv/valzpc_alignment_mask1943267c124397815476842d08d3901e.tex}}}}

\newcommand{\sailRISCVfnpcAlignmentMask}{\saildoclabelled{sailRISCVfnzpczyalignmentzymask}{\saildocfn{}{\lstinputlisting[language=sail]{sail_latex_riscv/fnzpc_alignment_mask1943267c124397815476842d08d3901e.tex}}}}

\newcommand{\sailRISCVregistermtval}{\saildoclabelled{sailRISCVregisterzmtval}{\saildocregister{}{\lstinputlisting[language=sail]{sail_latex_riscv/registerzmtvald49a142309bde2f160e87c7e394ae344.tex}}}}

\newcommand{\sailRISCVregistermscratch}{\saildoclabelled{sailRISCVregisterzmscratch}{\saildocregister{}{\lstinputlisting[language=sail]{sail_latex_riscv/registerzmscratch69719d5dfd8d4ca30c1c37d1533cd58d.tex}}}}

\newcommand{\sailRISCVtypeCounteren}{\saildoclabelled{sailRISCVtypezCounteren}{\saildoctype{}{\lstinputlisting[language=sail]{sail_latex_riscv/typezcounteren79441db7eb726d320a01493bb921db0e.tex}}}}

\newcommand{\sailRISCVregistermcounteren}{\saildoclabelled{sailRISCVregisterzmcounteren}{\saildocregister{}{\lstinputlisting[language=sail]{sail_latex_riscv/registerzmcounterenf658f473c4b578b1af4a4054b65335aa.tex}}}}

\newcommand{\sailRISCVregisterscounteren}{\saildoclabelled{sailRISCVregisterzscounteren}{\saildocregister{}{\lstinputlisting[language=sail]{sail_latex_riscv/registerzscounterena4e89645c651838f95f9ad90e403bc28.tex}}}}

\newcommand{\sailRISCVvallegalizzeMcounteren}{\saildoclabelled{sailRISCVzlegalizzezymcounteren}{\saildocval{}{\lstinputlisting[language=sail]{sail_latex_riscv/valzlegalizze_mcounteren34ab1d8bf1d636c1c12eb9e0b743229f.tex}}}}

\newcommand{\sailRISCVfnlegalizzeMcounteren}{\saildoclabelled{sailRISCVfnzlegalizzezymcounteren}{\saildocfn{}{\lstinputlisting[language=sail]{sail_latex_riscv/fnzlegalizze_mcounteren34ab1d8bf1d636c1c12eb9e0b743229f.tex}}}}

\newcommand{\sailRISCVvallegalizzeScounteren}{\saildoclabelled{sailRISCVzlegalizzezyscounteren}{\saildocval{}{\lstinputlisting[language=sail]{sail_latex_riscv/valzlegalizze_scountereneb4feb895f759dc11e62dedd5be1c232.tex}}}}

\newcommand{\sailRISCVfnlegalizzeScounteren}{\saildoclabelled{sailRISCVfnzlegalizzezyscounteren}{\saildocfn{}{\lstinputlisting[language=sail]{sail_latex_riscv/fnzlegalizze_scountereneb4feb895f759dc11e62dedd5be1c232.tex}}}}

\newcommand{\sailRISCVtypeCounterin}{\saildoclabelled{sailRISCVtypezCounterin}{\saildoctype{}{\lstinputlisting[language=sail]{sail_latex_riscv/typezcounterin25158d322f7a7f1b254bff0638245582.tex}}}}

\newcommand{\sailRISCVregistermcountinhibit}{\saildoclabelled{sailRISCVregisterzmcountinhibit}{\saildocregister{}{\lstinputlisting[language=sail]{sail_latex_riscv/registerzmcountinhibita4aca7b8264d70b90a82f74c8279a704.tex}}}}

\newcommand{\sailRISCVvallegalizzeMcountinhibit}{\saildoclabelled{sailRISCVzlegalizzezymcountinhibit}{\saildocval{}{\lstinputlisting[language=sail]{sail_latex_riscv/valzlegalizze_mcountinhibit83a4863cf81084082c5c66150150e695.tex}}}}

\newcommand{\sailRISCVfnlegalizzeMcountinhibit}{\saildoclabelled{sailRISCVfnzlegalizzezymcountinhibit}{\saildocfn{}{\lstinputlisting[language=sail]{sail_latex_riscv/fnzlegalizze_mcountinhibit83a4863cf81084082c5c66150150e695.tex}}}}

\newcommand{\sailRISCVregistermcycle}{\saildoclabelled{sailRISCVregisterzmcycle}{\saildocregister{}{\lstinputlisting[language=sail]{sail_latex_riscv/registerzmcyclee308f19fe5eee1b57c22cf8293fd5c52.tex}}}}

\newcommand{\sailRISCVregistermtime}{\saildoclabelled{sailRISCVregisterzmtime}{\saildocregister{}{\lstinputlisting[language=sail]{sail_latex_riscv/registerzmtime6eef8fbfbcae6e4477759436537cfcbc.tex}}}}

\newcommand{\sailRISCVregisterminstret}{\saildoclabelled{sailRISCVregisterzminstret}{\saildocregister{}{\lstinputlisting[language=sail]{sail_latex_riscv/registerzminstret542e7e5da16279d2bd90859b1533a6ee.tex}}}}

\newcommand{\sailRISCVregisterminstretWritten}{\saildoclabelled{sailRISCVregisterzminstretzywritten}{\saildocregister{}{\lstinputlisting[language=sail]{sail_latex_riscv/registerzminstret_writtenaed0046aa7ec44bb59b2c8c7710e05d2.tex}}}}

\newcommand{\sailRISCVvalretireInstruction}{\saildoclabelled{sailRISCVzretirezyinstruction}{\saildocval{}{\lstinputlisting[language=sail]{sail_latex_riscv/valzretire_instructionc9e2e6b25fd8f4e5a96cd82a0dd2a675.tex}}}}

\newcommand{\sailRISCVfnretireInstruction}{\saildoclabelled{sailRISCVfnzretirezyinstruction}{\saildocfn{}{\lstinputlisting[language=sail]{sail_latex_riscv/fnzretire_instructionc9e2e6b25fd8f4e5a96cd82a0dd2a675.tex}}}}

\newcommand{\sailRISCVregistermvendorid}{\saildoclabelled{sailRISCVregisterzmvendorid}{\saildocregister{}{\lstinputlisting[language=sail]{sail_latex_riscv/registerzmvendoridb839b7bbcf61dfc453ab3b30b51b54d7.tex}}}}

\newcommand{\sailRISCVregistermimpid}{\saildoclabelled{sailRISCVregisterzmimpid}{\saildocregister{}{\lstinputlisting[language=sail]{sail_latex_riscv/registerzmimpidd5c531d28776f8d602094828aa64e270.tex}}}}

\newcommand{\sailRISCVregistermarchid}{\saildoclabelled{sailRISCVregisterzmarchid}{\saildocregister{}{\lstinputlisting[language=sail]{sail_latex_riscv/registerzmarchidb9b264fe4bdc62a0b64a2bb45094f2ee.tex}}}}

\newcommand{\sailRISCVregistermhartid}{\saildoclabelled{sailRISCVregisterzmhartid}{\saildocregister{}{\lstinputlisting[language=sail]{sail_latex_riscv/registerzmhartidb8e734c7b59827098f74be8ab8dcfa74.tex}}}}

\newcommand{\sailRISCVtypeSstatus}{\saildoclabelled{sailRISCVtypezSstatus}{\saildoctype{}{\lstinputlisting[language=sail]{sail_latex_riscv/typezsstatusb3811444f066c3543008d60a23f2fb4b.tex}}}}

\newcommand{\sailRISCVvalgetSstatusUXL}{\saildoclabelled{sailRISCVzgetzysstatuszyUXL}{\saildocval{}{\lstinputlisting[language=sail]{sail_latex_riscv/valzget_sstatus_uxl1156c19b76a50f309373d09741dbfe73.tex}}}}

\newcommand{\sailRISCVfngetSstatusUXL}{\saildoclabelled{sailRISCVfnzgetzysstatuszyUXL}{\saildocfn{}{\lstinputlisting[language=sail]{sail_latex_riscv/fnzget_sstatus_uxl1156c19b76a50f309373d09741dbfe73.tex}}}}

\newcommand{\sailRISCVvalsetSstatusUXL}{\saildoclabelled{sailRISCVzsetzysstatuszyUXL}{\saildocval{}{\lstinputlisting[language=sail]{sail_latex_riscv/valzset_sstatus_uxl4d602a7109609248f9b54c8c905b4ad0.tex}}}}

\newcommand{\sailRISCVfnsetSstatusUXL}{\saildoclabelled{sailRISCVfnzsetzysstatuszyUXL}{\saildocfn{}{\lstinputlisting[language=sail]{sail_latex_riscv/fnzset_sstatus_uxl4d602a7109609248f9b54c8c905b4ad0.tex}}}}

\newcommand{\sailRISCVvallowerMstatus}{\saildoclabelled{sailRISCVzlowerzymstatus}{\saildocval{}{\lstinputlisting[language=sail]{sail_latex_riscv/valzlower_mstatuse71dfbedf80129bee76a841456768dd1.tex}}}}

\newcommand{\sailRISCVfnlowerMstatus}{\saildoclabelled{sailRISCVfnzlowerzymstatus}{\saildocfn{}{\lstinputlisting[language=sail]{sail_latex_riscv/fnzlower_mstatuse71dfbedf80129bee76a841456768dd1.tex}}}}

\newcommand{\sailRISCVvalliftSstatus}{\saildoclabelled{sailRISCVzliftzysstatus}{\saildocval{}{\lstinputlisting[language=sail]{sail_latex_riscv/valzlift_sstatus8b6865e3f513094f03fd5bfc83d2ed77.tex}}}}

\newcommand{\sailRISCVfnliftSstatus}{\saildoclabelled{sailRISCVfnzliftzysstatus}{\saildocfn{}{\lstinputlisting[language=sail]{sail_latex_riscv/fnzlift_sstatus8b6865e3f513094f03fd5bfc83d2ed77.tex}}}}

\newcommand{\sailRISCVvallegalizzeSstatus}{\saildoclabelled{sailRISCVzlegalizzezysstatus}{\saildocval{}{\lstinputlisting[language=sail]{sail_latex_riscv/valzlegalizze_sstatus6838bd21db8615a2ed1602fc6dd7f8da.tex}}}}

\newcommand{\sailRISCVfnlegalizzeSstatus}{\saildoclabelled{sailRISCVfnzlegalizzezysstatus}{\saildocfn{}{\lstinputlisting[language=sail]{sail_latex_riscv/fnzlegalizze_sstatus6838bd21db8615a2ed1602fc6dd7f8da.tex}}}}

\newcommand{\sailRISCVtypeSedeleg}{\saildoclabelled{sailRISCVtypezSedeleg}{\saildoctype{}{\lstinputlisting[language=sail]{sail_latex_riscv/typezsedeleg261f8092f9d360d308a10bcbcc83c5f6.tex}}}}

\newcommand{\sailRISCVregistersedeleg}{\saildoclabelled{sailRISCVregisterzsedeleg}{\saildocregister{}{\lstinputlisting[language=sail]{sail_latex_riscv/registerzsedeleg947c560f2f49a3b5bc91eab68408d6a1.tex}}}}

\newcommand{\sailRISCVvallegalizzeSedeleg}{\saildoclabelled{sailRISCVzlegalizzezysedeleg}{\saildocval{}{\lstinputlisting[language=sail]{sail_latex_riscv/valzlegalizze_sedelegd33e4c93e76af1c9ae6a7795974dbcc3.tex}}}}

\newcommand{\sailRISCVfnlegalizzeSedeleg}{\saildoclabelled{sailRISCVfnzlegalizzezysedeleg}{\saildocfn{}{\lstinputlisting[language=sail]{sail_latex_riscv/fnzlegalizze_sedelegd33e4c93e76af1c9ae6a7795974dbcc3.tex}}}}

\newcommand{\sailRISCVtypeSinterrupts}{\saildoclabelled{sailRISCVtypezSinterrupts}{\saildoctype{}{\lstinputlisting[language=sail]{sail_latex_riscv/typezsinterrupts37e2ae8cbcb085a9455d8333ab820af3.tex}}}}

\newcommand{\sailRISCVvallowerMip}{\saildoclabelled{sailRISCVzlowerzymip}{\saildocval{}{\lstinputlisting[language=sail]{sail_latex_riscv/valzlower_mip782f0e78d1db1ca14e49fae5b84aab3a.tex}}}}

\newcommand{\sailRISCVfnlowerMip}{\saildoclabelled{sailRISCVfnzlowerzymip}{\saildocfn{}{\lstinputlisting[language=sail]{sail_latex_riscv/fnzlower_mip782f0e78d1db1ca14e49fae5b84aab3a.tex}}}}

\newcommand{\sailRISCVvallowerMie}{\saildoclabelled{sailRISCVzlowerzymie}{\saildocval{}{\lstinputlisting[language=sail]{sail_latex_riscv/valzlower_mie31369302ff2457befa23f7f9d54a6b02.tex}}}}

\newcommand{\sailRISCVfnlowerMie}{\saildoclabelled{sailRISCVfnzlowerzymie}{\saildocfn{}{\lstinputlisting[language=sail]{sail_latex_riscv/fnzlower_mie31369302ff2457befa23f7f9d54a6b02.tex}}}}

\newcommand{\sailRISCVvalliftSip}{\saildoclabelled{sailRISCVzliftzysip}{\saildocval{}{\lstinputlisting[language=sail]{sail_latex_riscv/valzlift_sip492375a8cff775f29029156b44dfe1bf.tex}}}}

\newcommand{\sailRISCVfnliftSip}{\saildoclabelled{sailRISCVfnzliftzysip}{\saildocfn{}{\lstinputlisting[language=sail]{sail_latex_riscv/fnzlift_sip492375a8cff775f29029156b44dfe1bf.tex}}}}

\newcommand{\sailRISCVvallegalizzeSip}{\saildoclabelled{sailRISCVzlegalizzezysip}{\saildocval{}{\lstinputlisting[language=sail]{sail_latex_riscv/valzlegalizze_sip8870e10af087ca0981c61bf7bdfe8175.tex}}}}

\newcommand{\sailRISCVfnlegalizzeSip}{\saildoclabelled{sailRISCVfnzlegalizzezysip}{\saildocfn{}{\lstinputlisting[language=sail]{sail_latex_riscv/fnzlegalizze_sip8870e10af087ca0981c61bf7bdfe8175.tex}}}}

\newcommand{\sailRISCVvalliftSie}{\saildoclabelled{sailRISCVzliftzysie}{\saildocval{}{\lstinputlisting[language=sail]{sail_latex_riscv/valzlift_sie0866dcb30be948749bf6a401d4f6594e.tex}}}}

\newcommand{\sailRISCVfnliftSie}{\saildoclabelled{sailRISCVfnzliftzysie}{\saildocfn{}{\lstinputlisting[language=sail]{sail_latex_riscv/fnzlift_sie0866dcb30be948749bf6a401d4f6594e.tex}}}}

\newcommand{\sailRISCVvallegalizzeSie}{\saildoclabelled{sailRISCVzlegalizzezysie}{\saildocval{}{\lstinputlisting[language=sail]{sail_latex_riscv/valzlegalizze_sie49baa5a30e7d5365e2d6c1dc23c7686d.tex}}}}

\newcommand{\sailRISCVfnlegalizzeSie}{\saildoclabelled{sailRISCVfnzlegalizzezysie}{\saildocfn{}{\lstinputlisting[language=sail]{sail_latex_riscv/fnzlegalizze_sie49baa5a30e7d5365e2d6c1dc23c7686d.tex}}}}

\newcommand{\sailRISCVregistersideleg}{\saildoclabelled{sailRISCVregisterzsideleg}{\saildocregister{}{\lstinputlisting[language=sail]{sail_latex_riscv/registerzsideleg3e34d5b613ef95c9b55880ce2058c24d.tex}}}}

\newcommand{\sailRISCVregisterstvec}{\saildoclabelled{sailRISCVregisterzstvec}{\saildocregister{}{\lstinputlisting[language=sail]{sail_latex_riscv/registerzstvec0eb349ea0ad646d2d168306aa856414d.tex}}}}

\newcommand{\sailRISCVregistersscratch}{\saildoclabelled{sailRISCVregisterzsscratch}{\saildocregister{}{\lstinputlisting[language=sail]{sail_latex_riscv/registerzsscratch67d5cfc9d879c6b2a26e97bebe3b79b2.tex}}}}

\newcommand{\sailRISCVregistersepc}{\saildoclabelled{sailRISCVregisterzsepc}{\saildocregister{}{\lstinputlisting[language=sail]{sail_latex_riscv/registerzsepc3740eb348272eec2397a98b3d33723db.tex}}}}

\newcommand{\sailRISCVregisterscause}{\saildoclabelled{sailRISCVregisterzscause}{\saildocregister{}{\lstinputlisting[language=sail]{sail_latex_riscv/registerzscause533cd9e5245f251f5cb7a651a94decad.tex}}}}

\newcommand{\sailRISCVregisterstval}{\saildoclabelled{sailRISCVregisterzstval}{\saildocregister{}{\lstinputlisting[language=sail]{sail_latex_riscv/registerzstval75b0a14dbb04e8af8e03e5cd50636cab.tex}}}}

\newcommand{\sailRISCVtypeSatpSixFour}{\saildoclabelled{sailRISCVtypezSatp64}{\saildoctype{}{\lstinputlisting[language=sail]{sail_latex_riscv/typezsatp64bc81c51aa01ca27913d2b0a8ed23d481.tex}}}}

\newcommand{\sailRISCVvallegalizzeSatpSixFour}{\saildoclabelled{sailRISCVzlegalizzezysatp64}{\saildocval{}{\lstinputlisting[language=sail]{sail_latex_riscv/valzlegalizze_satp64c07dfcb94af8010eb535b5f77fbb8614.tex}}}}

\newcommand{\sailRISCVfnlegalizzeSatpSixFour}{\saildoclabelled{sailRISCVfnzlegalizzezysatp64}{\saildocfn{}{\lstinputlisting[language=sail]{sail_latex_riscv/fnzlegalizze_satp64c07dfcb94af8010eb535b5f77fbb8614.tex}}}}

\newcommand{\sailRISCVtypeSatpThreeTwo}{\saildoclabelled{sailRISCVtypezSatp32}{\saildoctype{}{\lstinputlisting[language=sail]{sail_latex_riscv/typezsatp321d6ed78cf9005ac8e6444a423cd6e618.tex}}}}

\newcommand{\sailRISCVvallegalizzeSatpThreeTwo}{\saildoclabelled{sailRISCVzlegalizzezysatp32}{\saildocval{}{\lstinputlisting[language=sail]{sail_latex_riscv/valzlegalizze_satp32b15f7b0d2d9380033f3e1b09fe370a61.tex}}}}

\newcommand{\sailRISCVfnlegalizzeSatpThreeTwo}{\saildoclabelled{sailRISCVfnzlegalizzezysatp32}{\saildocfn{}{\lstinputlisting[language=sail]{sail_latex_riscv/fnzlegalizze_satp32b15f7b0d2d9380033f3e1b09fe370a61.tex}}}}

\newcommand{\sailRISCVregistertselect}{\saildoclabelled{sailRISCVregisterztselect}{\saildocregister{}{\lstinputlisting[language=sail]{sail_latex_riscv/registerztselect1d29616a408737e95100ecc48eb1bbab.tex}}}}

\newcommand{\sailRISCVtypeseedOpst}{\saildoclabelled{sailRISCVtypezseedzyopst}{\saildoctype{}{\lstinputlisting[language=sail]{sail_latex_riscv/typezseed_opstfc811b39c73cc20d6913a2136e84a2bd.tex}}}}

\newcommand{\sailRISCVvalseedOpstOfNum}{\saildoclabelled{sailRISCVzseedzyopstzyofzynum}{\saildocval{}{\lstinputlisting[language=sail]{sail_latex_riscv/valzseed_opst_of_numf83d66097ccbfc3cdc7fa8b94f8969b6.tex}}}}

\newcommand{\sailRISCVfnseedOpstOfNum}{\saildoclabelled{sailRISCVfnzseedzyopstzyofzynum}{\saildocfn{}{\lstinputlisting[language=sail]{sail_latex_riscv/fnzseed_opst_of_numf83d66097ccbfc3cdc7fa8b94f8969b6.tex}}}}

\newcommand{\sailRISCVvalnumOfSeedOpst}{\saildoclabelled{sailRISCVznumzyofzyseedzyopst}{\saildocval{}{\lstinputlisting[language=sail]{sail_latex_riscv/valznum_of_seed_opstd0ba04c3622d423a3f556e7374c80315.tex}}}}

\newcommand{\sailRISCVfnnumOfSeedOpst}{\saildoclabelled{sailRISCVfnznumzyofzyseedzyopst}{\saildocfn{}{\lstinputlisting[language=sail]{sail_latex_riscv/fnznum_of_seed_opstd0ba04c3622d423a3f556e7374c80315.tex}}}}

\newcommand{\sailRISCVvalopstCode}{\saildoclabelled{sailRISCVzopstzycode}{\saildocval{}{\lstinputlisting[language=sail]{sail_latex_riscv/valzopst_code7b9aaa3672eaf5bb93f1127686a7fe99.tex}}}}

\newcommand{\sailRISCVvalgetOneSixRandomBits}{\saildoclabelled{sailRISCVzgetzy16zyrandomzybits}{\saildocval{}{\lstinputlisting[language=sail]{sail_latex_riscv/valzget_16_random_bits80ad3e39758ad499ab49bbf6571ef9b5.tex}}}}

\newcommand{\sailRISCVvalreadSeedCsr}{\saildoclabelled{sailRISCVzreadzyseedzycsr}{\saildocval{}{\lstinputlisting[language=sail]{sail_latex_riscv/valzread_seed_csra92150792aea7284c5abb98737094d1d.tex}}}}

\newcommand{\sailRISCVfnreadSeedCsr}{\saildoclabelled{sailRISCVfnzreadzyseedzycsr}{\saildocfn{}{\lstinputlisting[language=sail]{sail_latex_riscv/fnzread_seed_csra92150792aea7284c5abb98737094d1d.tex}}}}

\newcommand{\sailRISCVvalwriteSeedCsr}{\saildoclabelled{sailRISCVzwritezyseedzycsr}{\saildocval{}{\lstinputlisting[language=sail]{sail_latex_riscv/valzwrite_seed_csrfd9531acd713c94adc54460ad38524c4.tex}}}}

\newcommand{\sailRISCVfnwriteSeedCsr}{\saildoclabelled{sailRISCVfnzwritezyseedzycsr}{\saildocfn{}{\lstinputlisting[language=sail]{sail_latex_riscv/fnzwrite_seed_csrfd9531acd713c94adc54460ad38524c4.tex}}}}

\newcommand{\sailRISCVtypePmpAddrMatchType}{\saildoclabelled{sailRISCVtypezPmpAddrMatchType}{\saildoctype{}{\lstinputlisting[language=sail]{sail_latex_riscv/typezpmpaddrmatchtype51c8a34448b8f2bea86b1b88c958b458.tex}}}}

\newcommand{\sailRISCVvalPmpAddrMatchTypeOfNum}{\saildoclabelled{sailRISCVzPmpAddrMatchTypezyofzynum}{\saildocval{}{\lstinputlisting[language=sail]{sail_latex_riscv/valzpmpaddrmatchtype_of_num05d549dd7f3bf4d35f3c38ea6a015bf3.tex}}}}

\newcommand{\sailRISCVfnPmpAddrMatchTypeOfNum}{\saildoclabelled{sailRISCVfnzPmpAddrMatchTypezyofzynum}{\saildocfn{}{\lstinputlisting[language=sail]{sail_latex_riscv/fnzpmpaddrmatchtype_of_num05d549dd7f3bf4d35f3c38ea6a015bf3.tex}}}}

\newcommand{\sailRISCVvalnumOfPmpAddrMatchType}{\saildoclabelled{sailRISCVznumzyofzyPmpAddrMatchType}{\saildocval{}{\lstinputlisting[language=sail]{sail_latex_riscv/valznum_of_pmpaddrmatchtypee330d16c3db664232af948049b8edeb9.tex}}}}

\newcommand{\sailRISCVfnnumOfPmpAddrMatchType}{\saildoclabelled{sailRISCVfnznumzyofzyPmpAddrMatchType}{\saildocfn{}{\lstinputlisting[language=sail]{sail_latex_riscv/fnznum_of_pmpaddrmatchtypee330d16c3db664232af948049b8edeb9.tex}}}}

\newcommand{\sailRISCVvalpmpAddrMatchTypeOfBits}{\saildoclabelled{sailRISCVzpmpAddrMatchTypezyofzybits}{\saildocval{}{\lstinputlisting[language=sail]{sail_latex_riscv/valzpmpaddrmatchtype_of_bits8b4e245ce3d01f111a1f89404623ac98.tex}}}}

\newcommand{\sailRISCVfnpmpAddrMatchTypeOfBits}{\saildoclabelled{sailRISCVfnzpmpAddrMatchTypezyofzybits}{\saildocfn{}{\lstinputlisting[language=sail]{sail_latex_riscv/fnzpmpaddrmatchtype_of_bits8b4e245ce3d01f111a1f89404623ac98.tex}}}}

\newcommand{\sailRISCVvalpmpAddrMatchTypeToBits}{\saildoclabelled{sailRISCVzpmpAddrMatchTypezytozybits}{\saildocval{}{\lstinputlisting[language=sail]{sail_latex_riscv/valzpmpaddrmatchtype_to_bitsd16d593c276e8bc21809612105ac8913.tex}}}}

\newcommand{\sailRISCVfnpmpAddrMatchTypeToBits}{\saildoclabelled{sailRISCVfnzpmpAddrMatchTypezytozybits}{\saildocfn{}{\lstinputlisting[language=sail]{sail_latex_riscv/fnzpmpaddrmatchtype_to_bitsd16d593c276e8bc21809612105ac8913.tex}}}}

\newcommand{\sailRISCVtypePmpcfgEnt}{\saildoclabelled{sailRISCVtypezPmpcfgzyent}{\saildoctype{}{\lstinputlisting[language=sail]{sail_latex_riscv/typezpmpcfg_entf1f5c95dff6dbc5296f79013fd3eda09.tex}}}}

\newcommand{\sailRISCVregisterpmpZerocfg}{\saildoclabelled{sailRISCVregisterzpmp0cfg}{\saildocregister{}{\lstinputlisting[language=sail]{sail_latex_riscv/registerzpmp0cfgbab5c213c1a764aa4c8650fb1bf70c22.tex}}}}

\newcommand{\sailRISCVregisterpmpOnecfg}{\saildoclabelled{sailRISCVregisterzpmp1cfg}{\saildocregister{}{\lstinputlisting[language=sail]{sail_latex_riscv/registerzpmp1cfgd71559b110dd1a256c5d4e44d9e10b0a.tex}}}}

\newcommand{\sailRISCVregisterpmpTwocfg}{\saildoclabelled{sailRISCVregisterzpmp2cfg}{\saildocregister{}{\lstinputlisting[language=sail]{sail_latex_riscv/registerzpmp2cfgafa8058bad73632a1a1a9f49062bb623.tex}}}}

\newcommand{\sailRISCVregisterpmpThreecfg}{\saildoclabelled{sailRISCVregisterzpmp3cfg}{\saildocregister{}{\lstinputlisting[language=sail]{sail_latex_riscv/registerzpmp3cfg65a18d1c464bf396096e6bac52e96e66.tex}}}}

\newcommand{\sailRISCVregisterpmpFourcfg}{\saildoclabelled{sailRISCVregisterzpmp4cfg}{\saildocregister{}{\lstinputlisting[language=sail]{sail_latex_riscv/registerzpmp4cfg4e3e23aa0c8c9ea9d4bf216afbda447f.tex}}}}

\newcommand{\sailRISCVregisterpmpFivecfg}{\saildoclabelled{sailRISCVregisterzpmp5cfg}{\saildocregister{}{\lstinputlisting[language=sail]{sail_latex_riscv/registerzpmp5cfgfeba18c8a7a4711d8933b943cbf0d4e9.tex}}}}

\newcommand{\sailRISCVregisterpmpSixcfg}{\saildoclabelled{sailRISCVregisterzpmp6cfg}{\saildocregister{}{\lstinputlisting[language=sail]{sail_latex_riscv/registerzpmp6cfg341e8a5ba4271996c23c49e415ca7094.tex}}}}

\newcommand{\sailRISCVregisterpmpSevencfg}{\saildoclabelled{sailRISCVregisterzpmp7cfg}{\saildocregister{}{\lstinputlisting[language=sail]{sail_latex_riscv/registerzpmp7cfg16660c82c2f12636460279bc80ff3476.tex}}}}

\newcommand{\sailRISCVregisterpmpEightcfg}{\saildoclabelled{sailRISCVregisterzpmp8cfg}{\saildocregister{}{\lstinputlisting[language=sail]{sail_latex_riscv/registerzpmp8cfgb013823cbf258dd9bb903df2590c7db0.tex}}}}

\newcommand{\sailRISCVregisterpmpNinecfg}{\saildoclabelled{sailRISCVregisterzpmp9cfg}{\saildocregister{}{\lstinputlisting[language=sail]{sail_latex_riscv/registerzpmp9cfga1dd9e94aac9f15f871c7f2eb5b4c93b.tex}}}}

\newcommand{\sailRISCVregisterpmpOneZerocfg}{\saildoclabelled{sailRISCVregisterzpmp10cfg}{\saildocregister{}{\lstinputlisting[language=sail]{sail_latex_riscv/registerzpmp10cfge438c73c11a494d00233816b1de4e5c9.tex}}}}

\newcommand{\sailRISCVregisterpmpOneOnecfg}{\saildoclabelled{sailRISCVregisterzpmp11cfg}{\saildocregister{}{\lstinputlisting[language=sail]{sail_latex_riscv/registerzpmp11cfg3dd6c39fc00f4c26e0eb77e545ff896b.tex}}}}

\newcommand{\sailRISCVregisterpmpOneTwocfg}{\saildoclabelled{sailRISCVregisterzpmp12cfg}{\saildocregister{}{\lstinputlisting[language=sail]{sail_latex_riscv/registerzpmp12cfg9c230b0d5d95250e7f2f77fc9e158429.tex}}}}

\newcommand{\sailRISCVregisterpmpOneThreecfg}{\saildoclabelled{sailRISCVregisterzpmp13cfg}{\saildocregister{}{\lstinputlisting[language=sail]{sail_latex_riscv/registerzpmp13cfgdd21e002e41c22a8d1e6eb7951dff2f2.tex}}}}

\newcommand{\sailRISCVregisterpmpOneFourcfg}{\saildoclabelled{sailRISCVregisterzpmp14cfg}{\saildocregister{}{\lstinputlisting[language=sail]{sail_latex_riscv/registerzpmp14cfgff0f15b9869d20a65b49c67b6c1b6646.tex}}}}

\newcommand{\sailRISCVregisterpmpOneFivecfg}{\saildoclabelled{sailRISCVregisterzpmp15cfg}{\saildocregister{}{\lstinputlisting[language=sail]{sail_latex_riscv/registerzpmp15cfgcc174867cf34a9ce6e1cfaf7fbb0e524.tex}}}}

\newcommand{\sailRISCVregisterpmpaddrZero}{\saildoclabelled{sailRISCVregisterzpmpaddr0}{\saildocregister{}{\lstinputlisting[language=sail]{sail_latex_riscv/registerzpmpaddr01318ca05b5f2c6418ee72854e83ea4b5.tex}}}}

\newcommand{\sailRISCVregisterpmpaddrOne}{\saildoclabelled{sailRISCVregisterzpmpaddr1}{\saildocregister{}{\lstinputlisting[language=sail]{sail_latex_riscv/registerzpmpaddr1afa8222fa94a6e49c7cd901cad79e14d.tex}}}}

\newcommand{\sailRISCVregisterpmpaddrTwo}{\saildoclabelled{sailRISCVregisterzpmpaddr2}{\saildocregister{}{\lstinputlisting[language=sail]{sail_latex_riscv/registerzpmpaddr248ef60d815e2ce65aec76eb9f3d20c91.tex}}}}

\newcommand{\sailRISCVregisterpmpaddrThree}{\saildoclabelled{sailRISCVregisterzpmpaddr3}{\saildocregister{}{\lstinputlisting[language=sail]{sail_latex_riscv/registerzpmpaddr3ffd2d83a692e77469cbd8e51d6f89ea4.tex}}}}

\newcommand{\sailRISCVregisterpmpaddrFour}{\saildoclabelled{sailRISCVregisterzpmpaddr4}{\saildocregister{}{\lstinputlisting[language=sail]{sail_latex_riscv/registerzpmpaddr4c61ddb0dbca63ac04d43771403696afa.tex}}}}

\newcommand{\sailRISCVregisterpmpaddrFive}{\saildoclabelled{sailRISCVregisterzpmpaddr5}{\saildocregister{}{\lstinputlisting[language=sail]{sail_latex_riscv/registerzpmpaddr5b132533741cd45135e3d63bb78a982ad.tex}}}}

\newcommand{\sailRISCVregisterpmpaddrSix}{\saildoclabelled{sailRISCVregisterzpmpaddr6}{\saildocregister{}{\lstinputlisting[language=sail]{sail_latex_riscv/registerzpmpaddr615332ad7fcf0f1c67f0ab4d767b1aec8.tex}}}}

\newcommand{\sailRISCVregisterpmpaddrSeven}{\saildoclabelled{sailRISCVregisterzpmpaddr7}{\saildocregister{}{\lstinputlisting[language=sail]{sail_latex_riscv/registerzpmpaddr736954a743533f3670d8b935770308596.tex}}}}

\newcommand{\sailRISCVregisterpmpaddrEight}{\saildoclabelled{sailRISCVregisterzpmpaddr8}{\saildocregister{}{\lstinputlisting[language=sail]{sail_latex_riscv/registerzpmpaddr8655988d03d25d633f3c455eaa19f20b2.tex}}}}

\newcommand{\sailRISCVregisterpmpaddrNine}{\saildoclabelled{sailRISCVregisterzpmpaddr9}{\saildocregister{}{\lstinputlisting[language=sail]{sail_latex_riscv/registerzpmpaddr96809da0fe6da5f4642481d0ceea3e87a.tex}}}}

\newcommand{\sailRISCVregisterpmpaddrOneZero}{\saildoclabelled{sailRISCVregisterzpmpaddr10}{\saildocregister{}{\lstinputlisting[language=sail]{sail_latex_riscv/registerzpmpaddr101ae256e930340bfdc54f3552baae4101.tex}}}}

\newcommand{\sailRISCVregisterpmpaddrOneOne}{\saildoclabelled{sailRISCVregisterzpmpaddr11}{\saildocregister{}{\lstinputlisting[language=sail]{sail_latex_riscv/registerzpmpaddr1113ef52f8ad3fa9b8f1fc7d13b2a7d87a.tex}}}}

\newcommand{\sailRISCVregisterpmpaddrOneTwo}{\saildoclabelled{sailRISCVregisterzpmpaddr12}{\saildocregister{}{\lstinputlisting[language=sail]{sail_latex_riscv/registerzpmpaddr12df6491c8d84d7a8d193886212d2fd5d9.tex}}}}

\newcommand{\sailRISCVregisterpmpaddrOneThree}{\saildoclabelled{sailRISCVregisterzpmpaddr13}{\saildocregister{}{\lstinputlisting[language=sail]{sail_latex_riscv/registerzpmpaddr13dac2b94da5f464a04974c90e70ec47b1.tex}}}}

\newcommand{\sailRISCVregisterpmpaddrOneFour}{\saildoclabelled{sailRISCVregisterzpmpaddr14}{\saildocregister{}{\lstinputlisting[language=sail]{sail_latex_riscv/registerzpmpaddr144db2dcdc54c46e1de8d422a61703f605.tex}}}}

\newcommand{\sailRISCVregisterpmpaddrOneFive}{\saildoclabelled{sailRISCVregisterzpmpaddr15}{\saildocregister{}{\lstinputlisting[language=sail]{sail_latex_riscv/registerzpmpaddr155c571c64499d194872ff351c32039af3.tex}}}}

\newcommand{\sailRISCVvalpmpReadCfgReg}{\saildoclabelled{sailRISCVzpmpReadCfgReg}{\saildocval{}{\lstinputlisting[language=sail]{sail_latex_riscv/valzpmpreadcfgreg4f212865b80d1cf6286c5525466852bb.tex}}}}

\newcommand{\sailRISCVfnpmpReadCfgReg}{\saildoclabelled{sailRISCVfnzpmpReadCfgReg}{\saildocfn{}{\lstinputlisting[language=sail]{sail_latex_riscv/fnzpmpreadcfgreg4f212865b80d1cf6286c5525466852bb.tex}}}}

\newcommand{\sailRISCVvalpmpLocked}{\saildoclabelled{sailRISCVzpmpLocked}{\saildocval{}{\lstinputlisting[language=sail]{sail_latex_riscv/valzpmplocked32d6273cc49a9c7a22cf1a063f7a3d9b.tex}}}}

\newcommand{\sailRISCVfnpmpLocked}{\saildoclabelled{sailRISCVfnzpmpLocked}{\saildocfn{}{\lstinputlisting[language=sail]{sail_latex_riscv/fnzpmplocked32d6273cc49a9c7a22cf1a063f7a3d9b.tex}}}}

\newcommand{\sailRISCVvalpmpTORLocked}{\saildoclabelled{sailRISCVzpmpTORLocked}{\saildocval{}{\lstinputlisting[language=sail]{sail_latex_riscv/valzpmptorlockedc6616187f2e2905a9fb7c4ac60d52920.tex}}}}

\newcommand{\sailRISCVfnpmpTORLocked}{\saildoclabelled{sailRISCVfnzpmpTORLocked}{\saildocfn{}{\lstinputlisting[language=sail]{sail_latex_riscv/fnzpmptorlockedc6616187f2e2905a9fb7c4ac60d52920.tex}}}}

\newcommand{\sailRISCVvalpmpWriteCfg}{\saildoclabelled{sailRISCVzpmpWriteCfg}{\saildocval{}{\lstinputlisting[language=sail]{sail_latex_riscv/valzpmpwritecfg2cde3ea402426e32bb50a4bc91e0c983.tex}}}}

\newcommand{\sailRISCVfnpmpWriteCfg}{\saildoclabelled{sailRISCVfnzpmpWriteCfg}{\saildocfn{}{\lstinputlisting[language=sail]{sail_latex_riscv/fnzpmpwritecfg2cde3ea402426e32bb50a4bc91e0c983.tex}}}}

\newcommand{\sailRISCVvalpmpWriteCfgReg}{\saildoclabelled{sailRISCVzpmpWriteCfgReg}{\saildocval{}{\lstinputlisting[language=sail]{sail_latex_riscv/valzpmpwritecfgregf08572520295a822ab88e05e223953db.tex}}}}

\newcommand{\sailRISCVfnpmpWriteCfgReg}{\saildoclabelled{sailRISCVfnzpmpWriteCfgReg}{\saildocfn{}{\lstinputlisting[language=sail]{sail_latex_riscv/fnzpmpwritecfgregf08572520295a822ab88e05e223953db.tex}}}}

\newcommand{\sailRISCVvalpmpWriteAddr}{\saildoclabelled{sailRISCVzpmpWriteAddr}{\saildocval{}{\lstinputlisting[language=sail]{sail_latex_riscv/valzpmpwriteaddr35a89b2e3be238e40992648bdf6c2e4c.tex}}}}

\newcommand{\sailRISCVfnpmpWriteAddr}{\saildoclabelled{sailRISCVfnzpmpWriteAddr}{\saildocfn{}{\lstinputlisting[language=sail]{sail_latex_riscv/fnzpmpwriteaddr35a89b2e3be238e40992648bdf6c2e4c.tex}}}}

\newcommand{\sailRISCVtypepmpAddrRange}{\saildoclabelled{sailRISCVtypezpmpzyaddrzyrange}{\saildoctype{}{\lstinputlisting[language=sail]{sail_latex_riscv/typezpmp_addr_range0c14af966e97978e2e0bea3825363bf8.tex}}}}

\newcommand{\sailRISCVvalpmpAddrRangeA}{\saildoclabelled{sailRISCVzpmpAddrRange}{\saildocval{}{\lstinputlisting[language=sail]{sail_latex_riscv/valzpmpaddrranged0e482997ef7d22fddc89c097e038f0d.tex}}}}

\newcommand{\sailRISCVfnpmpAddrRangeA}{\saildoclabelled{sailRISCVfnzpmpAddrRange}{\saildocfn{}{\lstinputlisting[language=sail]{sail_latex_riscv/fnzpmpaddrranged0e482997ef7d22fddc89c097e038f0d.tex}}}}

\newcommand{\sailRISCVvalpmpCheckRWX}{\saildoclabelled{sailRISCVzpmpCheckRWX}{\saildocval{}{\lstinputlisting[language=sail]{sail_latex_riscv/valzpmpcheckrwx6a81da10e740c25fedddbe430f079b7d.tex}}}}

\newcommand{\sailRISCVfnpmpCheckRWX}{\saildoclabelled{sailRISCVfnzpmpCheckRWX}{\saildocfn{}{\lstinputlisting[language=sail]{sail_latex_riscv/fnzpmpcheckrwx6a81da10e740c25fedddbe430f079b7d.tex}}}}

\newcommand{\sailRISCVvalpmpCheckPerms}{\saildoclabelled{sailRISCVzpmpCheckPerms}{\saildocval{}{\lstinputlisting[language=sail]{sail_latex_riscv/valzpmpcheckperms43a47caab37d2bebaa37cc41235e7387.tex}}}}

\newcommand{\sailRISCVfnpmpCheckPerms}{\saildoclabelled{sailRISCVfnzpmpCheckPerms}{\saildocfn{}{\lstinputlisting[language=sail]{sail_latex_riscv/fnzpmpcheckperms43a47caab37d2bebaa37cc41235e7387.tex}}}}

\newcommand{\sailRISCVtypepmpAddrMatch}{\saildoclabelled{sailRISCVtypezpmpAddrMatch}{\saildoctype{}{\lstinputlisting[language=sail]{sail_latex_riscv/typezpmpaddrmatch1b3d520a29cdf5ebe0ae194fe06ab693.tex}}}}

\newcommand{\sailRISCVvalpmpAddrMatchOfNum}{\saildoclabelled{sailRISCVzpmpAddrMatchzyofzynum}{\saildocval{}{\lstinputlisting[language=sail]{sail_latex_riscv/valzpmpaddrmatch_of_num92a36380d4ab664cee1f4ee0143e390f.tex}}}}

\newcommand{\sailRISCVfnpmpAddrMatchOfNum}{\saildoclabelled{sailRISCVfnzpmpAddrMatchzyofzynum}{\saildocfn{}{\lstinputlisting[language=sail]{sail_latex_riscv/fnzpmpaddrmatch_of_num92a36380d4ab664cee1f4ee0143e390f.tex}}}}

\newcommand{\sailRISCVvalnumOfPmpAddrMatch}{\saildoclabelled{sailRISCVznumzyofzypmpAddrMatch}{\saildocval{}{\lstinputlisting[language=sail]{sail_latex_riscv/valznum_of_pmpaddrmatch6db470099c3f03581d51f40437610a39.tex}}}}

\newcommand{\sailRISCVfnnumOfPmpAddrMatch}{\saildoclabelled{sailRISCVfnznumzyofzypmpAddrMatch}{\saildocfn{}{\lstinputlisting[language=sail]{sail_latex_riscv/fnznum_of_pmpaddrmatch6db470099c3f03581d51f40437610a39.tex}}}}

\newcommand{\sailRISCVvalpmpMatchAddr}{\saildoclabelled{sailRISCVzpmpMatchAddr}{\saildocval{}{\lstinputlisting[language=sail]{sail_latex_riscv/valzpmpmatchaddr4db797384cb60b665d5b05ce2f54ea2f.tex}}}}

\newcommand{\sailRISCVfnpmpMatchAddr}{\saildoclabelled{sailRISCVfnzpmpMatchAddr}{\saildocfn{}{\lstinputlisting[language=sail]{sail_latex_riscv/fnzpmpmatchaddr4db797384cb60b665d5b05ce2f54ea2f.tex}}}}

\newcommand{\sailRISCVtypepmpMatch}{\saildoclabelled{sailRISCVtypezpmpMatch}{\saildoctype{}{\lstinputlisting[language=sail]{sail_latex_riscv/typezpmpmatch43c06bad055792d6d43e48c69738fa97.tex}}}}

\newcommand{\sailRISCVvalpmpMatchOfNum}{\saildoclabelled{sailRISCVzpmpMatchzyofzynum}{\saildocval{}{\lstinputlisting[language=sail]{sail_latex_riscv/valzpmpmatch_of_num271d71ea451fba2032d2e5ea441d9f49.tex}}}}

\newcommand{\sailRISCVfnpmpMatchOfNum}{\saildoclabelled{sailRISCVfnzpmpMatchzyofzynum}{\saildocfn{}{\lstinputlisting[language=sail]{sail_latex_riscv/fnzpmpmatch_of_num271d71ea451fba2032d2e5ea441d9f49.tex}}}}

\newcommand{\sailRISCVvalnumOfPmpMatch}{\saildoclabelled{sailRISCVznumzyofzypmpMatch}{\saildocval{}{\lstinputlisting[language=sail]{sail_latex_riscv/valznum_of_pmpmatch3a839a54108c809c88766e9119a0bce5.tex}}}}

\newcommand{\sailRISCVfnnumOfPmpMatch}{\saildoclabelled{sailRISCVfnznumzyofzypmpMatch}{\saildocfn{}{\lstinputlisting[language=sail]{sail_latex_riscv/fnznum_of_pmpmatch3a839a54108c809c88766e9119a0bce5.tex}}}}

\newcommand{\sailRISCVvalpmpMatchEntry}{\saildoclabelled{sailRISCVzpmpMatchEntry}{\saildocval{}{\lstinputlisting[language=sail]{sail_latex_riscv/valzpmpmatchentryb540666374b1cccee76fe8eb4dc7362f.tex}}}}

\newcommand{\sailRISCVfnpmpMatchEntry}{\saildoclabelled{sailRISCVfnzpmpMatchEntry}{\saildocfn{}{\lstinputlisting[language=sail]{sail_latex_riscv/fnzpmpmatchentryb540666374b1cccee76fe8eb4dc7362f.tex}}}}

\newcommand{\sailRISCVvalpmpCheck}{\saildoclabelled{sailRISCVzpmpCheck}{\saildocval{}{\lstinputlisting[language=sail]{sail_latex_riscv/valzpmpcheck818accaacf804d4474fe874d5c97929a.tex}}}}

\newcommand{\sailRISCVfnpmpCheck}{\saildoclabelled{sailRISCVfnzpmpCheck}{\saildocfn{}{\lstinputlisting[language=sail]{sail_latex_riscv/fnzpmpcheck818accaacf804d4474fe874d5c97929a.tex}}}}

\newcommand{\sailRISCVvalinitPmp}{\saildoclabelled{sailRISCVzinitzypmp}{\saildocval{}{\lstinputlisting[language=sail]{sail_latex_riscv/valzinit_pmp10ae24a767d242e86512a2c1a55970ef.tex}}}}

\newcommand{\sailRISCVfninitPmp}{\saildoclabelled{sailRISCVfnzinitzypmp}{\saildocfn{}{\lstinputlisting[language=sail]{sail_latex_riscv/fnzinit_pmp10ae24a767d242e86512a2c1a55970ef.tex}}}}

\newcommand{\sailRISCVtypeccsr}{\saildoclabelled{sailRISCVtypezccsr}{\saildoctype{}{\lstinputlisting[language=sail]{sail_latex_riscv/typezccsrebba1b25012128c604b97c41d5de5508.tex}}}}

\newcommand{\sailRISCVregistermccsr}{\saildoclabelled{sailRISCVregisterzmccsr}{\saildocregister{}{\lstinputlisting[language=sail]{sail_latex_riscv/registerzmccsrf49356b1c2ead25a854b81f1d70f71b2.tex}}}}

\newcommand{\sailRISCVregistersccsr}{\saildoclabelled{sailRISCVregisterzsccsr}{\saildocregister{}{\lstinputlisting[language=sail]{sail_latex_riscv/registerzsccsrd43a7241484ea3707ea62e8cce82cdf6.tex}}}}

\newcommand{\sailRISCVregisteruccsr}{\saildoclabelled{sailRISCVregisterzuccsr}{\saildocregister{}{\lstinputlisting[language=sail]{sail_latex_riscv/registerzuccsr514002b0cdc243dc86e058883babcea1.tex}}}}

\newcommand{\sailRISCVvallegalizzeCcsr}{\saildoclabelled{sailRISCVzlegalizzezyccsr}{\saildocval{}{\lstinputlisting[language=sail]{sail_latex_riscv/valzlegalizze_ccsr8248d36ab83d1808c1e3e0b64d88c1cf.tex}}}}

\newcommand{\sailRISCVfnlegalizzeCcsr}{\saildoclabelled{sailRISCVfnzlegalizzezyccsr}{\saildocfn{}{\lstinputlisting[language=sail]{sail_latex_riscv/fnzlegalizze_ccsr8248d36ab83d1808c1e3e0b64d88c1cf.tex}}}}

\newcommand{\sailRISCVregisterPCC}{\saildoclabelled{sailRISCVregisterzPCC}{\saildocregister{}{\lstinputlisting[language=sail]{sail_latex_riscv/registerzpcc7790fb8e8f8e32aa5c43aaf6230ab26c.tex}}}}

\newcommand{\sailRISCVregisternextPCC}{\saildoclabelled{sailRISCVregisterznextPCC}{\saildocregister{}{\lstinputlisting[language=sail]{sail_latex_riscv/registerznextpccd3b6ca46e34adc6888f6be7bbff7d490.tex}}}}

\newcommand{\sailRISCVregisterDDC}{\saildoclabelled{sailRISCVregisterzDDC}{\saildocregister{}{\lstinputlisting[language=sail]{sail_latex_riscv/registerzddc252700d203e07d44dd0e84b2b21e8947.tex}}}}

\newcommand{\sailRISCVregisterUTCC}{\saildoclabelled{sailRISCVregisterzUTCC}{\saildocregister{}{\lstinputlisting[language=sail]{sail_latex_riscv/registerzutccc8bbcc4262ba36f812ab280c52d4163a.tex}}}}

\newcommand{\sailRISCVregisterUTDC}{\saildoclabelled{sailRISCVregisterzUTDC}{\saildocregister{}{\lstinputlisting[language=sail]{sail_latex_riscv/registerzutdc3e3c5ab8870db2dd03e70dec4a8fa9fd.tex}}}}

\newcommand{\sailRISCVregisterUScratchC}{\saildoclabelled{sailRISCVregisterzUScratchC}{\saildocregister{}{\lstinputlisting[language=sail]{sail_latex_riscv/registerzuscratchc82f6e363e46bb627a096cebc303e74b5.tex}}}}

\newcommand{\sailRISCVregisterUEPCC}{\saildoclabelled{sailRISCVregisterzUEPCC}{\saildocregister{}{\lstinputlisting[language=sail]{sail_latex_riscv/registerzuepcc737b78f4524a38426601387d321c40d3.tex}}}}

\newcommand{\sailRISCVregisterSTCC}{\saildoclabelled{sailRISCVregisterzSTCC}{\saildocregister{}{\lstinputlisting[language=sail]{sail_latex_riscv/registerzstcc9eb5d4caf6444a6d1e4f99a8936129c0.tex}}}}

\newcommand{\sailRISCVregisterSTDC}{\saildoclabelled{sailRISCVregisterzSTDC}{\saildocregister{}{\lstinputlisting[language=sail]{sail_latex_riscv/registerzstdcd644fe1867e010969c37b12a4ddfbea0.tex}}}}

\newcommand{\sailRISCVregisterSScratchC}{\saildoclabelled{sailRISCVregisterzSScratchC}{\saildocregister{}{\lstinputlisting[language=sail]{sail_latex_riscv/registerzsscratchc4a2701b66b3e113a1892048b5c36772d.tex}}}}

\newcommand{\sailRISCVregisterSEPCC}{\saildoclabelled{sailRISCVregisterzSEPCC}{\saildocregister{}{\lstinputlisting[language=sail]{sail_latex_riscv/registerzsepcccd25fc56bbb3518e17ffeda34ac7d6a4.tex}}}}

\newcommand{\sailRISCVregisterMTCC}{\saildoclabelled{sailRISCVregisterzMTCC}{\saildocregister{}{\lstinputlisting[language=sail]{sail_latex_riscv/registerzmtcc063472ed557a42632836fda50c7b01ea.tex}}}}

\newcommand{\sailRISCVregisterMTDC}{\saildoclabelled{sailRISCVregisterzMTDC}{\saildocregister{}{\lstinputlisting[language=sail]{sail_latex_riscv/registerzmtdc5898b0ed1cb126b3556f6f0a3d0e1f13.tex}}}}

\newcommand{\sailRISCVregisterMScratchC}{\saildoclabelled{sailRISCVregisterzMScratchC}{\saildocregister{}{\lstinputlisting[language=sail]{sail_latex_riscv/registerzmscratchcae4d8198dd83f93304ec3d932a204a32.tex}}}}

\newcommand{\sailRISCVregisterMEPCC}{\saildoclabelled{sailRISCVregisterzMEPCC}{\saildocregister{}{\lstinputlisting[language=sail]{sail_latex_riscv/registerzmepcccdd01bd19e0af0b28a07b1019e607e2a.tex}}}}

\newcommand{\sailRISCVvalminInstructionBytes}{\saildoclabelled{sailRISCVzminzyinstructionzybytes}{\saildocval{}{\lstinputlisting[language=sail]{sail_latex_riscv/valzmin_instruction_bytesb10dab453fbe2e946cf3f20de7511e71.tex}}}}

\newcommand{\sailRISCVfnminInstructionBytes}{\saildoclabelled{sailRISCVfnzminzyinstructionzybytes}{\saildocfn{}{\lstinputlisting[language=sail]{sail_latex_riscv/fnzmin_instruction_bytesb10dab453fbe2e946cf3f20de7511e71.tex}}}}

\newcommand{\sailRISCVvalhaveXcheri}{\saildoclabelled{sailRISCVzhaveXcheri}{\saildocval{}{\lstinputlisting[language=sail]{sail_latex_riscv/valzhavexcheri36112408abb05cceef19a84af1e5a22a.tex}}}}

\newcommand{\sailRISCVfnhaveXcheri}{\saildoclabelled{sailRISCVfnzhaveXcheri}{\saildocfn{}{\lstinputlisting[language=sail]{sail_latex_riscv/fnzhavexcheri36112408abb05cceef19a84af1e5a22a.tex}}}}

\newcommand{\sailRISCVvallegalizzeTcc}{\saildoclabelled{sailRISCVzlegalizzezytcc}{\saildocval{}{\lstinputlisting[language=sail]{sail_latex_riscv/valzlegalizze_tccfd2d2ccb3d791b05f6d62114b5036cb9.tex}}}}

\newcommand{\sailRISCVfnlegalizzeTcc}{\saildoclabelled{sailRISCVfnzlegalizzezytcc}{\saildocfn{}{\lstinputlisting[language=sail]{sail_latex_riscv/fnzlegalizze_tccfd2d2ccb3d791b05f6d62114b5036cb9.tex}}}}

\newcommand{\sailRISCVvallegalizzeEpcc}{\saildoclabelled{sailRISCVzlegalizzezyepcc}{\saildocval{}{\lstinputlisting[language=sail]{sail_latex_riscv/valzlegalizze_epccfd09ccf016c22f97fa9a6b5dfba65e84.tex}}}}

\newcommand{\sailRISCVfnlegalizzeEpcc}{\saildoclabelled{sailRISCVfnzlegalizzezyepcc}{\saildocfn{}{\lstinputlisting[language=sail]{sail_latex_riscv/fnzlegalizze_epccfd09ccf016c22f97fa9a6b5dfba65e84.tex}}}}

\newcommand{\sailRISCVvalrC}{\saildoclabelled{sailRISCVzrC}{\saildocval{}{\lstinputlisting[language=sail]{sail_latex_riscv/valzrcda9d82cd736deb89a37d9ca089373805.tex}}}}

\newcommand{\sailRISCVfnrC}{\saildoclabelled{sailRISCVfnzrC}{\saildocfn{}{\lstinputlisting[language=sail]{sail_latex_riscv/fnzrcda9d82cd736deb89a37d9ca089373805.tex}}}}

\newcommand{\sailRISCVvalwC}{\saildoclabelled{sailRISCVzwC}{\saildocval{}{\lstinputlisting[language=sail]{sail_latex_riscv/valzwc721d52ce2fe818d0148aaf1d5b6e2bec.tex}}}}

\newcommand{\sailRISCVfnwC}{\saildoclabelled{sailRISCVfnzwC}{\saildocfn{}{\lstinputlisting[language=sail]{sail_latex_riscv/fnzwc721d52ce2fe818d0148aaf1d5b6e2bec.tex}}}}

\newcommand{\sailRISCVvalrCBits}{\saildoclabelled{sailRISCVzrCzybits}{\saildocval{}{\lstinputlisting[language=sail]{sail_latex_riscv/valzrc_bits18d7e232c147203cdc9c7cd0ffbe7ec0.tex}}}}

\newcommand{\sailRISCVfnrCBits}{\saildoclabelled{sailRISCVfnzrCzybits}{\saildocfn{}{\lstinputlisting[language=sail]{sail_latex_riscv/fnzrc_bits18d7e232c147203cdc9c7cd0ffbe7ec0.tex}}}}

\newcommand{\sailRISCVvalwCBits}{\saildoclabelled{sailRISCVzwCzybits}{\saildocval{}{\lstinputlisting[language=sail]{sail_latex_riscv/valzwc_bitsb8b2d4f53308d4bb8d154a1f6664d336.tex}}}}

\newcommand{\sailRISCVfnwCBits}{\saildoclabelled{sailRISCVfnzwCzybits}{\saildocfn{}{\lstinputlisting[language=sail]{sail_latex_riscv/fnzwc_bitsb8b2d4f53308d4bb8d154a1f6664d336.tex}}}}

\newcommand{\sailRISCVoverloadKKKC}{\saildoclabelled{sailRISCVoverloadKKKzC}{\saildocoverload{}{\lstinputlisting[language=sail]{sail_latex_riscv/overloadKKKzc15f67105dab436b80b9241d87d1f8e9d.tex}}}}

\newcommand{\sailRISCVvalextInitRegs}{\saildoclabelled{sailRISCVzextzyinitzyregs}{\saildocval{}{\lstinputlisting[language=sail]{sail_latex_riscv/valzext_init_regs1d9ff00ce58fd5712eb26190e338015a.tex}}}}

\newcommand{\sailRISCVfnextInitRegs}{\saildoclabelled{sailRISCVfnzextzyinitzyregs}{\saildocfn{}{\lstinputlisting[language=sail]{sail_latex_riscv/fnzext_init_regs1d9ff00ce58fd5712eb26190e338015a.tex}}}}

\newcommand{\sailRISCVvalextRvfiInit}{\saildoclabelled{sailRISCVzextzyrvfizyinit}{\saildocval{This function is called after above when running rvfi and allows the model
to be initialised differently. For RVFI we initialise cap regs to default
instead of null.

}{\lstinputlisting[language=sail]{sail_latex_riscv/valzext_rvfi_init925272b6b4282430111c3e89a50f1e0b.tex}}}}

\newcommand{\sailRISCVfnextRvfiInit}{\saildoclabelled{sailRISCVfnzextzyrvfizyinit}{\saildocfn{}{\lstinputlisting[language=sail]{sail_latex_riscv/fnzext_rvfi_init925272b6b4282430111c3e89a50f1e0b.tex}}}}

\newcommand{\sailRISCVvalcapRegNameAbi}{\saildoclabelled{sailRISCVzcapzyregzynamezyabi}{\saildocval{}{\lstinputlisting[language=sail]{sail_latex_riscv/valzcap_reg_name_abi719d69842e8a63e886ffe0791675d4e0.tex}}}}

\newcommand{\sailRISCVfncapRegNameAbi}{\saildoclabelled{sailRISCVfnzcapzyregzynamezyabi}{\saildocfn{}{\lstinputlisting[language=sail]{sail_latex_riscv/fnzcap_reg_name_abi719d69842e8a63e886ffe0791675d4e0.tex}}}}

\newcommand{\sailRISCVoverloadLLLtoStr}{\saildoclabelled{sailRISCVoverloadLLLztozystr}{\saildocoverload{}{\lstinputlisting[language=sail]{sail_latex_riscv/overloadLLLzto_str8b7a6895ae35945bd4740e9f790c43ee.tex}}}}

\newcommand{\sailRISCVvalcapRegName}{\saildoclabelled{sailRISCVzcapzyregzyname}{\saildocval{}{\lstinputlisting[language=sail]{sail_latex_riscv/valzcap_reg_name2541adc6dad121efe53209371c0fbc68.tex}}}}

\newcommand{\sailRISCVvalcapCregName}{\saildoclabelled{sailRISCVzcapzycregzyname}{\saildocval{}{\lstinputlisting[language=sail]{sail_latex_riscv/valzcap_creg_nameb745905c258c936ba7a6eafe4633359f.tex}}}}

\newcommand{\sailRISCVvalgetArchPc}{\saildoclabelled{sailRISCVzgetzyarchzypc}{\saildocval{}{\lstinputlisting[language=sail]{sail_latex_riscv/valzget_arch_pc874e1e50a5765cd3e317d37fa710a52d.tex}}}}

\newcommand{\sailRISCVfngetArchPc}{\saildoclabelled{sailRISCVfnzgetzyarchzypc}{\saildocfn{}{\lstinputlisting[language=sail]{sail_latex_riscv/fnzget_arch_pc874e1e50a5765cd3e317d37fa710a52d.tex}}}}

\newcommand{\sailRISCVvalgetNextPc}{\saildoclabelled{sailRISCVzgetzynextzypc}{\saildocval{}{\lstinputlisting[language=sail]{sail_latex_riscv/valzget_next_pc52098782da35f914abcd4b708363813b.tex}}}}

\newcommand{\sailRISCVfngetNextPc}{\saildoclabelled{sailRISCVfnzgetzynextzypc}{\saildocfn{}{\lstinputlisting[language=sail]{sail_latex_riscv/fnzget_next_pc52098782da35f914abcd4b708363813b.tex}}}}

\newcommand{\sailRISCVvalsetNextPc}{\saildoclabelled{sailRISCVzsetzynextzypc}{\saildocval{}{\lstinputlisting[language=sail]{sail_latex_riscv/valzset_next_pc4a8ca0d1733a6630871f9f58d4ea2e08.tex}}}}

\newcommand{\sailRISCVfnsetNextPc}{\saildoclabelled{sailRISCVfnzsetzynextzypc}{\saildocfn{}{\lstinputlisting[language=sail]{sail_latex_riscv/fnzset_next_pc4a8ca0d1733a6630871f9f58d4ea2e08.tex}}}}

\newcommand{\sailRISCVvaltickPc}{\saildoclabelled{sailRISCVztickzypc}{\saildocval{}{\lstinputlisting[language=sail]{sail_latex_riscv/valztick_pc459868e2ecb79c51cd3817471ae33696.tex}}}}

\newcommand{\sailRISCVfntickPc}{\saildoclabelled{sailRISCVfnztickzypc}{\saildocfn{}{\lstinputlisting[language=sail]{sail_latex_riscv/fnztick_pc459868e2ecb79c51cd3817471ae33696.tex}}}}

\newcommand{\sailRISCVtypeUstatus}{\saildoclabelled{sailRISCVtypezUstatus}{\saildoctype{}{\lstinputlisting[language=sail]{sail_latex_riscv/typezustatusea9ad819ad5ad1c26e0977e4f7a545d6.tex}}}}

\newcommand{\sailRISCVvallowerSstatus}{\saildoclabelled{sailRISCVzlowerzysstatus}{\saildocval{}{\lstinputlisting[language=sail]{sail_latex_riscv/valzlower_sstatusee8227db3ac1f80e02871989d3ac7837.tex}}}}

\newcommand{\sailRISCVfnlowerSstatus}{\saildoclabelled{sailRISCVfnzlowerzysstatus}{\saildocfn{}{\lstinputlisting[language=sail]{sail_latex_riscv/fnzlower_sstatusee8227db3ac1f80e02871989d3ac7837.tex}}}}

\newcommand{\sailRISCVvalliftUstatus}{\saildoclabelled{sailRISCVzliftzyustatus}{\saildocval{}{\lstinputlisting[language=sail]{sail_latex_riscv/valzlift_ustatus9f7d2e2291fdb3a746dca97290479c47.tex}}}}

\newcommand{\sailRISCVfnliftUstatus}{\saildoclabelled{sailRISCVfnzliftzyustatus}{\saildocfn{}{\lstinputlisting[language=sail]{sail_latex_riscv/fnzlift_ustatus9f7d2e2291fdb3a746dca97290479c47.tex}}}}

\newcommand{\sailRISCVvallegalizzeUstatus}{\saildoclabelled{sailRISCVzlegalizzezyustatus}{\saildocval{}{\lstinputlisting[language=sail]{sail_latex_riscv/valzlegalizze_ustatusae91980d0571269b3dd8a4a779cb06c6.tex}}}}

\newcommand{\sailRISCVfnlegalizzeUstatus}{\saildoclabelled{sailRISCVfnzlegalizzezyustatus}{\saildocfn{}{\lstinputlisting[language=sail]{sail_latex_riscv/fnzlegalizze_ustatusae91980d0571269b3dd8a4a779cb06c6.tex}}}}

\newcommand{\sailRISCVtypeUinterrupts}{\saildoclabelled{sailRISCVtypezUinterrupts}{\saildoctype{}{\lstinputlisting[language=sail]{sail_latex_riscv/typezuinterrupts1427c63818d0a5c9cb26ba06e84f503f.tex}}}}

\newcommand{\sailRISCVvallowerSip}{\saildoclabelled{sailRISCVzlowerzysip}{\saildocval{}{\lstinputlisting[language=sail]{sail_latex_riscv/valzlower_sipfb2971a9ef35ea1e2a4c00f905ec894c.tex}}}}

\newcommand{\sailRISCVfnlowerSip}{\saildoclabelled{sailRISCVfnzlowerzysip}{\saildocfn{}{\lstinputlisting[language=sail]{sail_latex_riscv/fnzlower_sipfb2971a9ef35ea1e2a4c00f905ec894c.tex}}}}

\newcommand{\sailRISCVvallowerSie}{\saildoclabelled{sailRISCVzlowerzysie}{\saildocval{}{\lstinputlisting[language=sail]{sail_latex_riscv/valzlower_siea6b1bfe3227230c0b4c5a544f98a7636.tex}}}}

\newcommand{\sailRISCVfnlowerSie}{\saildoclabelled{sailRISCVfnzlowerzysie}{\saildocfn{}{\lstinputlisting[language=sail]{sail_latex_riscv/fnzlower_siea6b1bfe3227230c0b4c5a544f98a7636.tex}}}}

\newcommand{\sailRISCVvalliftUip}{\saildoclabelled{sailRISCVzliftzyuip}{\saildocval{}{\lstinputlisting[language=sail]{sail_latex_riscv/valzlift_uipc541c535c8821c63ef55fd81254b1078.tex}}}}

\newcommand{\sailRISCVfnliftUip}{\saildoclabelled{sailRISCVfnzliftzyuip}{\saildocfn{}{\lstinputlisting[language=sail]{sail_latex_riscv/fnzlift_uipc541c535c8821c63ef55fd81254b1078.tex}}}}

\newcommand{\sailRISCVvallegalizzeUip}{\saildoclabelled{sailRISCVzlegalizzezyuip}{\saildocval{}{\lstinputlisting[language=sail]{sail_latex_riscv/valzlegalizze_uipc8133dbe8104c4fd25f4a3933fa30bbb.tex}}}}

\newcommand{\sailRISCVfnlegalizzeUip}{\saildoclabelled{sailRISCVfnzlegalizzezyuip}{\saildocfn{}{\lstinputlisting[language=sail]{sail_latex_riscv/fnzlegalizze_uipc8133dbe8104c4fd25f4a3933fa30bbb.tex}}}}

\newcommand{\sailRISCVvalliftUie}{\saildoclabelled{sailRISCVzliftzyuie}{\saildocval{}{\lstinputlisting[language=sail]{sail_latex_riscv/valzlift_uied44a5d7049b27cdb894913f74634a628.tex}}}}

\newcommand{\sailRISCVfnliftUie}{\saildoclabelled{sailRISCVfnzliftzyuie}{\saildocfn{}{\lstinputlisting[language=sail]{sail_latex_riscv/fnzlift_uied44a5d7049b27cdb894913f74634a628.tex}}}}

\newcommand{\sailRISCVvallegalizzeUie}{\saildoclabelled{sailRISCVzlegalizzezyuie}{\saildocval{}{\lstinputlisting[language=sail]{sail_latex_riscv/valzlegalizze_uie26bad63954765d654a99bf56da523d30.tex}}}}

\newcommand{\sailRISCVfnlegalizzeUie}{\saildoclabelled{sailRISCVfnzlegalizzezyuie}{\saildocfn{}{\lstinputlisting[language=sail]{sail_latex_riscv/fnzlegalizze_uie26bad63954765d654a99bf56da523d30.tex}}}}

\newcommand{\sailRISCVregisterutvec}{\saildoclabelled{sailRISCVregisterzutvec}{\saildocregister{}{\lstinputlisting[language=sail]{sail_latex_riscv/registerzutveca59db16f2b6cece74cbf225ef0d95fcc.tex}}}}

\newcommand{\sailRISCVregisteruscratch}{\saildoclabelled{sailRISCVregisterzuscratch}{\saildocregister{}{\lstinputlisting[language=sail]{sail_latex_riscv/registerzuscratch1bd6909e190e8ed77f8d57b3f21796b0.tex}}}}

\newcommand{\sailRISCVregisteruepc}{\saildoclabelled{sailRISCVregisterzuepc}{\saildocregister{}{\lstinputlisting[language=sail]{sail_latex_riscv/registerzuepc62d45b88a7b130b5c4ae9fd612e362ba.tex}}}}

\newcommand{\sailRISCVregisterucause}{\saildoclabelled{sailRISCVregisterzucause}{\saildocregister{}{\lstinputlisting[language=sail]{sail_latex_riscv/registerzucause62743e1bf4f36ae266e96c6f93295ec2.tex}}}}

\newcommand{\sailRISCVregisterutval}{\saildoclabelled{sailRISCVregisterzutval}{\saildocregister{}{\lstinputlisting[language=sail]{sail_latex_riscv/registerzutval16cc41db8a6009306fc07dc85c83a528.tex}}}}

\newcommand{\sailRISCVtypeextException}{\saildoclabelled{sailRISCVtypezextzyexception}{\saildoctype{}{\lstinputlisting[language=sail]{sail_latex_riscv/typezext_exception91bbea5dcedef746789e1dfa97cb264d.tex}}}}

\newcommand{\sailRISCVvalhandleTrapExtension}{\saildoclabelled{sailRISCVzhandlezytrapzyextension}{\saildocval{}{\lstinputlisting[language=sail]{sail_latex_riscv/valzhandle_trap_extension9480b0ad72ce05ffcdcc27185eb4525c.tex}}}}

\newcommand{\sailRISCVfnhandleTrapExtension}{\saildoclabelled{sailRISCVfnzhandlezytrapzyextension}{\saildocfn{}{\lstinputlisting[language=sail]{sail_latex_riscv/fnzhandle_trap_extension9480b0ad72ce05ffcdcc27185eb4525c.tex}}}}

\newcommand{\sailRISCVvalprepareTrapVector}{\saildoclabelled{sailRISCVzpreparezytrapzyvector}{\saildocval{}{\lstinputlisting[language=sail]{sail_latex_riscv/valzprepare_trap_vector90a104f40adfa987d74a613f4061790f.tex}}}}

\newcommand{\sailRISCVfnprepareTrapVector}{\saildoclabelled{sailRISCVfnzpreparezytrapzyvector}{\saildocfn{}{\lstinputlisting[language=sail]{sail_latex_riscv/fnzprepare_trap_vector90a104f40adfa987d74a613f4061790f.tex}}}}

\newcommand{\sailRISCVvalgetXretTarget}{\saildoclabelled{sailRISCVzgetzyxretzytarget}{\saildocval{}{\lstinputlisting[language=sail]{sail_latex_riscv/valzget_xret_target26ce66652c1cd67f2e91b685c1d44e62.tex}}}}

\newcommand{\sailRISCVfngetXretTarget}{\saildoclabelled{sailRISCVfnzgetzyxretzytarget}{\saildocfn{}{\lstinputlisting[language=sail]{sail_latex_riscv/fnzget_xret_target26ce66652c1cd67f2e91b685c1d44e62.tex}}}}

\newcommand{\sailRISCVvalsetXretTarget}{\saildoclabelled{sailRISCVzsetzyxretzytarget}{\saildocval{}{\lstinputlisting[language=sail]{sail_latex_riscv/valzset_xret_target81095db6e6bb6da6b746ed406dccd45e.tex}}}}

\newcommand{\sailRISCVfnsetXretTarget}{\saildoclabelled{sailRISCVfnzsetzyxretzytarget}{\saildocfn{}{\lstinputlisting[language=sail]{sail_latex_riscv/fnzset_xret_target81095db6e6bb6da6b746ed406dccd45e.tex}}}}

\newcommand{\sailRISCVvalprepareXretTarget}{\saildoclabelled{sailRISCVzpreparezyxretzytarget}{\saildocval{}{\lstinputlisting[language=sail]{sail_latex_riscv/valzprepare_xret_target77691a306ebd6df5d988335f19693ba0.tex}}}}

\newcommand{\sailRISCVfnprepareXretTarget}{\saildoclabelled{sailRISCVfnzpreparezyxretzytarget}{\saildocfn{}{\lstinputlisting[language=sail]{sail_latex_riscv/fnzprepare_xret_target77691a306ebd6df5d988335f19693ba0.tex}}}}

\newcommand{\sailRISCVvalgetMtvec}{\saildoclabelled{sailRISCVzgetzymtvec}{\saildocval{}{\lstinputlisting[language=sail]{sail_latex_riscv/valzget_mtvec134cdc828faab7bf9f19733fba43da2f.tex}}}}

\newcommand{\sailRISCVfngetMtvec}{\saildoclabelled{sailRISCVfnzgetzymtvec}{\saildocfn{}{\lstinputlisting[language=sail]{sail_latex_riscv/fnzget_mtvec134cdc828faab7bf9f19733fba43da2f.tex}}}}

\newcommand{\sailRISCVvalgetStvec}{\saildoclabelled{sailRISCVzgetzystvec}{\saildocval{}{\lstinputlisting[language=sail]{sail_latex_riscv/valzget_stvec8e871a236060976d6c548af2b67478db.tex}}}}

\newcommand{\sailRISCVfngetStvec}{\saildoclabelled{sailRISCVfnzgetzystvec}{\saildocfn{}{\lstinputlisting[language=sail]{sail_latex_riscv/fnzget_stvec8e871a236060976d6c548af2b67478db.tex}}}}

\newcommand{\sailRISCVvalgetUtvec}{\saildoclabelled{sailRISCVzgetzyutvec}{\saildocval{}{\lstinputlisting[language=sail]{sail_latex_riscv/valzget_utvec375b7b6fc923ff7174c10f9b93b1b2cf.tex}}}}

\newcommand{\sailRISCVfngetUtvec}{\saildoclabelled{sailRISCVfnzgetzyutvec}{\saildocfn{}{\lstinputlisting[language=sail]{sail_latex_riscv/fnzget_utvec375b7b6fc923ff7174c10f9b93b1b2cf.tex}}}}

\newcommand{\sailRISCVvalsetMtvec}{\saildoclabelled{sailRISCVzsetzymtvec}{\saildocval{}{\lstinputlisting[language=sail]{sail_latex_riscv/valzset_mtvec628ceb1975fa7b69631fcc224f1bbbc0.tex}}}}

\newcommand{\sailRISCVfnsetMtvec}{\saildoclabelled{sailRISCVfnzsetzymtvec}{\saildocfn{}{\lstinputlisting[language=sail]{sail_latex_riscv/fnzset_mtvec628ceb1975fa7b69631fcc224f1bbbc0.tex}}}}

\newcommand{\sailRISCVvalsetStvec}{\saildoclabelled{sailRISCVzsetzystvec}{\saildocval{}{\lstinputlisting[language=sail]{sail_latex_riscv/valzset_stvec0d8dcc20a4d9dd912d52b9252d2370a7.tex}}}}

\newcommand{\sailRISCVfnsetStvec}{\saildoclabelled{sailRISCVfnzsetzystvec}{\saildocfn{}{\lstinputlisting[language=sail]{sail_latex_riscv/fnzset_stvec0d8dcc20a4d9dd912d52b9252d2370a7.tex}}}}

\newcommand{\sailRISCVvalsetUtvec}{\saildoclabelled{sailRISCVzsetzyutvec}{\saildocval{}{\lstinputlisting[language=sail]{sail_latex_riscv/valzset_utvec4e028b19c2b4db0c1c1a0a48f1c4a330.tex}}}}

\newcommand{\sailRISCVfnsetUtvec}{\saildoclabelled{sailRISCVfnzsetzyutvec}{\saildocfn{}{\lstinputlisting[language=sail]{sail_latex_riscv/fnzset_utvec4e028b19c2b4db0c1c1a0a48f1c4a330.tex}}}}

\newcommand{\sailRISCVtypesyncException}{\saildoclabelled{sailRISCVtypezsynczyexception}{\saildoctype{}{\lstinputlisting[language=sail]{sail_latex_riscv/typezsync_exceptionf6f925de3b34f9256b962854b6415d97.tex}}}}

\newcommand{\sailRISCVtypebitsRm}{\saildoclabelled{sailRISCVtypezbitszyrm}{\saildoctype{}{\lstinputlisting[language=sail]{sail_latex_riscv/typezbits_rm5b1f0169c0e4579bf2dcf98ba6079158.tex}}}}

\newcommand{\sailRISCVtypebitsFflags}{\saildoclabelled{sailRISCVtypezbitszyfflags}{\saildoctype{}{\lstinputlisting[language=sail]{sail_latex_riscv/typezbits_fflagsed9b9763ebe260174539e00fd7a0a522.tex}}}}

\newcommand{\sailRISCVtypebitsH}{\saildoclabelled{sailRISCVtypezbitszyH}{\saildoctype{}{\lstinputlisting[language=sail]{sail_latex_riscv/typezbits_h76e870fea4865b90f006ce427cb2a6df.tex}}}}

\newcommand{\sailRISCVtypebitsS}{\saildoclabelled{sailRISCVtypezbitszyS}{\saildoctype{}{\lstinputlisting[language=sail]{sail_latex_riscv/typezbits_s528d05a73240052d43e6da2026f5b50f.tex}}}}

\newcommand{\sailRISCVtypebitsD}{\saildoclabelled{sailRISCVtypezbitszyD}{\saildoctype{}{\lstinputlisting[language=sail]{sail_latex_riscv/typezbits_db59c6343f76a23d300748dc232dd81da.tex}}}}

\newcommand{\sailRISCVtypebitsW}{\saildoclabelled{sailRISCVtypezbitszyW}{\saildoctype{}{\lstinputlisting[language=sail]{sail_latex_riscv/typezbits_wcd34ca70f55b1c3a8eca237a0b2e5a5b.tex}}}}

\newcommand{\sailRISCVtypebitsWU}{\saildoclabelled{sailRISCVtypezbitszyWU}{\saildoctype{}{\lstinputlisting[language=sail]{sail_latex_riscv/typezbits_wud57fbe06e562e34a1f66b7832875f761.tex}}}}

\newcommand{\sailRISCVtypebitsL}{\saildoclabelled{sailRISCVtypezbitszyL}{\saildoctype{}{\lstinputlisting[language=sail]{sail_latex_riscv/typezbits_l07cb308b5e4f91b7017ab58409d32070.tex}}}}

\newcommand{\sailRISCVtypebitsLU}{\saildoclabelled{sailRISCVtypezbitszyLU}{\saildoctype{}{\lstinputlisting[language=sail]{sail_latex_riscv/typezbits_lu3ab4327e3a12cfe46ba2eb7149190854.tex}}}}

\newcommand{\sailRISCVregisterfloatResult}{\saildoclabelled{sailRISCVregisterzfloatzyresult}{\saildocregister{}{\lstinputlisting[language=sail]{sail_latex_riscv/registerzfloat_result26986aaec234f8ae97ca9b22e362811c.tex}}}}

\newcommand{\sailRISCVregisterfloatFflags}{\saildoclabelled{sailRISCVregisterzfloatzyfflags}{\saildocregister{}{\lstinputlisting[language=sail]{sail_latex_riscv/registerzfloat_fflagsc45deaf199d5c06ba23f632f21999ee7.tex}}}}

\newcommand{\sailRISCVvalupdateSoftfloatFflags}{\saildoclabelled{sailRISCVzupdatezysoftfloatzyfflags}{\saildocval{}{\lstinputlisting[language=sail]{sail_latex_riscv/valzupdate_softfloat_fflagsc33a686e24584b0b748b2ae8f801c505.tex}}}}

\newcommand{\sailRISCVfnupdateSoftfloatFflags}{\saildoclabelled{sailRISCVfnzupdatezysoftfloatzyfflags}{\saildocfn{}{\lstinputlisting[language=sail]{sail_latex_riscv/fnzupdate_softfloat_fflagsc33a686e24584b0b748b2ae8f801c505.tex}}}}

\newcommand{\sailRISCVvalexternFOneSixAdd}{\saildoclabelled{sailRISCVzexternzyf16Add}{\saildocval{}{\lstinputlisting[language=sail]{sail_latex_riscv/valzextern_f16add690d90d5375b969bdacb9ed1cd70fe8b.tex}}}}

\newcommand{\sailRISCVvalriscvFOneSixAdd}{\saildoclabelled{sailRISCVzriscvzyf16Add}{\saildocval{}{\lstinputlisting[language=sail]{sail_latex_riscv/valzriscv_f16add5095cba144c13a238cd70aeaf90d907c.tex}}}}

\newcommand{\sailRISCVfnriscvFOneSixAdd}{\saildoclabelled{sailRISCVfnzriscvzyf16Add}{\saildocfn{}{\lstinputlisting[language=sail]{sail_latex_riscv/fnzriscv_f16add5095cba144c13a238cd70aeaf90d907c.tex}}}}

\newcommand{\sailRISCVvalexternFOneSixSub}{\saildoclabelled{sailRISCVzexternzyf16Sub}{\saildocval{}{\lstinputlisting[language=sail]{sail_latex_riscv/valzextern_f16sub87135e1febd992558ee9e02e3f2622cc.tex}}}}

\newcommand{\sailRISCVvalriscvFOneSixSub}{\saildoclabelled{sailRISCVzriscvzyf16Sub}{\saildocval{}{\lstinputlisting[language=sail]{sail_latex_riscv/valzriscv_f16sub5a2d7181440c6293faf598e33733b141.tex}}}}

\newcommand{\sailRISCVfnriscvFOneSixSub}{\saildoclabelled{sailRISCVfnzriscvzyf16Sub}{\saildocfn{}{\lstinputlisting[language=sail]{sail_latex_riscv/fnzriscv_f16sub5a2d7181440c6293faf598e33733b141.tex}}}}

\newcommand{\sailRISCVvalexternFOneSixMul}{\saildoclabelled{sailRISCVzexternzyf16Mul}{\saildocval{}{\lstinputlisting[language=sail]{sail_latex_riscv/valzextern_f16mul9699a6de501cc6d2e5e9ea991d22a7c7.tex}}}}

\newcommand{\sailRISCVvalriscvFOneSixMul}{\saildoclabelled{sailRISCVzriscvzyf16Mul}{\saildocval{}{\lstinputlisting[language=sail]{sail_latex_riscv/valzriscv_f16mul642e6e0fdef18c8c7d9f00d37fbfaabc.tex}}}}

\newcommand{\sailRISCVfnriscvFOneSixMul}{\saildoclabelled{sailRISCVfnzriscvzyf16Mul}{\saildocfn{}{\lstinputlisting[language=sail]{sail_latex_riscv/fnzriscv_f16mul642e6e0fdef18c8c7d9f00d37fbfaabc.tex}}}}

\newcommand{\sailRISCVvalexternFOneSixDiv}{\saildoclabelled{sailRISCVzexternzyf16Div}{\saildocval{}{\lstinputlisting[language=sail]{sail_latex_riscv/valzextern_f16divf047b4e11428fe07489004b1d1ad95ee.tex}}}}

\newcommand{\sailRISCVvalriscvFOneSixDiv}{\saildoclabelled{sailRISCVzriscvzyf16Div}{\saildocval{}{\lstinputlisting[language=sail]{sail_latex_riscv/valzriscv_f16div9a10ab197ea81be7aa0605ce361b30ef.tex}}}}

\newcommand{\sailRISCVfnriscvFOneSixDiv}{\saildoclabelled{sailRISCVfnzriscvzyf16Div}{\saildocfn{}{\lstinputlisting[language=sail]{sail_latex_riscv/fnzriscv_f16div9a10ab197ea81be7aa0605ce361b30ef.tex}}}}

\newcommand{\sailRISCVvalexternFThreeTwoAdd}{\saildoclabelled{sailRISCVzexternzyf32Add}{\saildocval{}{\lstinputlisting[language=sail]{sail_latex_riscv/valzextern_f32add6570d62f364513d11456621384cd41a4.tex}}}}

\newcommand{\sailRISCVvalriscvFThreeTwoAdd}{\saildoclabelled{sailRISCVzriscvzyf32Add}{\saildocval{}{\lstinputlisting[language=sail]{sail_latex_riscv/valzriscv_f32add558e9569237c3f82255e78eba6e4d963.tex}}}}

\newcommand{\sailRISCVfnriscvFThreeTwoAdd}{\saildoclabelled{sailRISCVfnzriscvzyf32Add}{\saildocfn{}{\lstinputlisting[language=sail]{sail_latex_riscv/fnzriscv_f32add558e9569237c3f82255e78eba6e4d963.tex}}}}

\newcommand{\sailRISCVvalexternFThreeTwoSub}{\saildoclabelled{sailRISCVzexternzyf32Sub}{\saildocval{}{\lstinputlisting[language=sail]{sail_latex_riscv/valzextern_f32sub46595a3cbe22b28fcde81c3635052d96.tex}}}}

\newcommand{\sailRISCVvalriscvFThreeTwoSub}{\saildoclabelled{sailRISCVzriscvzyf32Sub}{\saildocval{}{\lstinputlisting[language=sail]{sail_latex_riscv/valzriscv_f32sub247ea008998dd84d9a3c22db11127bea.tex}}}}

\newcommand{\sailRISCVfnriscvFThreeTwoSub}{\saildoclabelled{sailRISCVfnzriscvzyf32Sub}{\saildocfn{}{\lstinputlisting[language=sail]{sail_latex_riscv/fnzriscv_f32sub247ea008998dd84d9a3c22db11127bea.tex}}}}

\newcommand{\sailRISCVvalexternFThreeTwoMul}{\saildoclabelled{sailRISCVzexternzyf32Mul}{\saildocval{}{\lstinputlisting[language=sail]{sail_latex_riscv/valzextern_f32mul3fa5520c02f9481c619db6b9185fc991.tex}}}}

\newcommand{\sailRISCVvalriscvFThreeTwoMul}{\saildoclabelled{sailRISCVzriscvzyf32Mul}{\saildocval{}{\lstinputlisting[language=sail]{sail_latex_riscv/valzriscv_f32mulc9ec954a141b4f26110e9e3c2b42b73c.tex}}}}

\newcommand{\sailRISCVfnriscvFThreeTwoMul}{\saildoclabelled{sailRISCVfnzriscvzyf32Mul}{\saildocfn{}{\lstinputlisting[language=sail]{sail_latex_riscv/fnzriscv_f32mulc9ec954a141b4f26110e9e3c2b42b73c.tex}}}}

\newcommand{\sailRISCVvalexternFThreeTwoDiv}{\saildoclabelled{sailRISCVzexternzyf32Div}{\saildocval{}{\lstinputlisting[language=sail]{sail_latex_riscv/valzextern_f32diva3f93b8d1944c9eb5853444ef6275b73.tex}}}}

\newcommand{\sailRISCVvalriscvFThreeTwoDiv}{\saildoclabelled{sailRISCVzriscvzyf32Div}{\saildocval{}{\lstinputlisting[language=sail]{sail_latex_riscv/valzriscv_f32div86cc53dd9bb61e6f44380c708e0673ca.tex}}}}

\newcommand{\sailRISCVfnriscvFThreeTwoDiv}{\saildoclabelled{sailRISCVfnzriscvzyf32Div}{\saildocfn{}{\lstinputlisting[language=sail]{sail_latex_riscv/fnzriscv_f32div86cc53dd9bb61e6f44380c708e0673ca.tex}}}}

\newcommand{\sailRISCVvalexternFSixFourAdd}{\saildoclabelled{sailRISCVzexternzyf64Add}{\saildocval{}{\lstinputlisting[language=sail]{sail_latex_riscv/valzextern_f64addeb106f14f2c0905dadf390a0565da932.tex}}}}

\newcommand{\sailRISCVvalriscvFSixFourAdd}{\saildoclabelled{sailRISCVzriscvzyf64Add}{\saildocval{}{\lstinputlisting[language=sail]{sail_latex_riscv/valzriscv_f64add33e6272fa354fe4e7b1963d6545bc4f7.tex}}}}

\newcommand{\sailRISCVfnriscvFSixFourAdd}{\saildoclabelled{sailRISCVfnzriscvzyf64Add}{\saildocfn{}{\lstinputlisting[language=sail]{sail_latex_riscv/fnzriscv_f64add33e6272fa354fe4e7b1963d6545bc4f7.tex}}}}

\newcommand{\sailRISCVvalexternFSixFourSub}{\saildoclabelled{sailRISCVzexternzyf64Sub}{\saildocval{}{\lstinputlisting[language=sail]{sail_latex_riscv/valzextern_f64subd720cdc7595c5e294f7877ef74876c27.tex}}}}

\newcommand{\sailRISCVvalriscvFSixFourSub}{\saildoclabelled{sailRISCVzriscvzyf64Sub}{\saildocval{}{\lstinputlisting[language=sail]{sail_latex_riscv/valzriscv_f64sub64a17c7e3b243ed3af4bc99790d41a9f.tex}}}}

\newcommand{\sailRISCVfnriscvFSixFourSub}{\saildoclabelled{sailRISCVfnzriscvzyf64Sub}{\saildocfn{}{\lstinputlisting[language=sail]{sail_latex_riscv/fnzriscv_f64sub64a17c7e3b243ed3af4bc99790d41a9f.tex}}}}

\newcommand{\sailRISCVvalexternFSixFourMul}{\saildoclabelled{sailRISCVzexternzyf64Mul}{\saildocval{}{\lstinputlisting[language=sail]{sail_latex_riscv/valzextern_f64mulf814ecd2e3d0ed473f76dcca74bd1c73.tex}}}}

\newcommand{\sailRISCVvalriscvFSixFourMul}{\saildoclabelled{sailRISCVzriscvzyf64Mul}{\saildocval{}{\lstinputlisting[language=sail]{sail_latex_riscv/valzriscv_f64mul1bf597fee8433219c830072b7c0b998e.tex}}}}

\newcommand{\sailRISCVfnriscvFSixFourMul}{\saildoclabelled{sailRISCVfnzriscvzyf64Mul}{\saildocfn{}{\lstinputlisting[language=sail]{sail_latex_riscv/fnzriscv_f64mul1bf597fee8433219c830072b7c0b998e.tex}}}}

\newcommand{\sailRISCVvalexternFSixFourDiv}{\saildoclabelled{sailRISCVzexternzyf64Div}{\saildocval{}{\lstinputlisting[language=sail]{sail_latex_riscv/valzextern_f64div5b9c0428c30d14b1844b77539325cd9e.tex}}}}

\newcommand{\sailRISCVvalriscvFSixFourDiv}{\saildoclabelled{sailRISCVzriscvzyf64Div}{\saildocval{}{\lstinputlisting[language=sail]{sail_latex_riscv/valzriscv_f64divecc62207bb7fa437fcf66a19597ada83.tex}}}}

\newcommand{\sailRISCVfnriscvFSixFourDiv}{\saildoclabelled{sailRISCVfnzriscvzyf64Div}{\saildocfn{}{\lstinputlisting[language=sail]{sail_latex_riscv/fnzriscv_f64divecc62207bb7fa437fcf66a19597ada83.tex}}}}

\newcommand{\sailRISCVvalexternFOneSixMulAdd}{\saildoclabelled{sailRISCVzexternzyf16MulAdd}{\saildocval{}{\lstinputlisting[language=sail]{sail_latex_riscv/valzextern_f16muladd31052b4f704cfd6e41de7fc25d121176.tex}}}}

\newcommand{\sailRISCVvalriscvFOneSixMulAdd}{\saildoclabelled{sailRISCVzriscvzyf16MulAdd}{\saildocval{}{\lstinputlisting[language=sail]{sail_latex_riscv/valzriscv_f16muladd349697a9cda553476b6ed4e37edf1197.tex}}}}

\newcommand{\sailRISCVfnriscvFOneSixMulAdd}{\saildoclabelled{sailRISCVfnzriscvzyf16MulAdd}{\saildocfn{}{\lstinputlisting[language=sail]{sail_latex_riscv/fnzriscv_f16muladd349697a9cda553476b6ed4e37edf1197.tex}}}}

\newcommand{\sailRISCVvalexternFThreeTwoMulAdd}{\saildoclabelled{sailRISCVzexternzyf32MulAdd}{\saildocval{}{\lstinputlisting[language=sail]{sail_latex_riscv/valzextern_f32muladd83ecaa2576af176cb966151d8adbd0ae.tex}}}}

\newcommand{\sailRISCVvalriscvFThreeTwoMulAdd}{\saildoclabelled{sailRISCVzriscvzyf32MulAdd}{\saildocval{}{\lstinputlisting[language=sail]{sail_latex_riscv/valzriscv_f32muladdc6b7d5aa910c0cf67022d34e7a745cd8.tex}}}}

\newcommand{\sailRISCVfnriscvFThreeTwoMulAdd}{\saildoclabelled{sailRISCVfnzriscvzyf32MulAdd}{\saildocfn{}{\lstinputlisting[language=sail]{sail_latex_riscv/fnzriscv_f32muladdc6b7d5aa910c0cf67022d34e7a745cd8.tex}}}}

\newcommand{\sailRISCVvalexternFSixFourMulAdd}{\saildoclabelled{sailRISCVzexternzyf64MulAdd}{\saildocval{}{\lstinputlisting[language=sail]{sail_latex_riscv/valzextern_f64muladd15243cc8731c3e0a2ef0ef7f3ed15e68.tex}}}}

\newcommand{\sailRISCVvalriscvFSixFourMulAdd}{\saildoclabelled{sailRISCVzriscvzyf64MulAdd}{\saildocval{}{\lstinputlisting[language=sail]{sail_latex_riscv/valzriscv_f64muladdf20eb35780f2858aa3087b34aba2250a.tex}}}}

\newcommand{\sailRISCVfnriscvFSixFourMulAdd}{\saildoclabelled{sailRISCVfnzriscvzyf64MulAdd}{\saildocfn{}{\lstinputlisting[language=sail]{sail_latex_riscv/fnzriscv_f64muladdf20eb35780f2858aa3087b34aba2250a.tex}}}}

\newcommand{\sailRISCVvalexternFOneSixSqrt}{\saildoclabelled{sailRISCVzexternzyf16Sqrt}{\saildocval{}{\lstinputlisting[language=sail]{sail_latex_riscv/valzextern_f16sqrtc0b93cc490a4898f0502297ad9b1f3d9.tex}}}}

\newcommand{\sailRISCVvalriscvFOneSixSqrt}{\saildoclabelled{sailRISCVzriscvzyf16Sqrt}{\saildocval{}{\lstinputlisting[language=sail]{sail_latex_riscv/valzriscv_f16sqrt5d5f1cbd96b6420edc617405f717d841.tex}}}}

\newcommand{\sailRISCVfnriscvFOneSixSqrt}{\saildoclabelled{sailRISCVfnzriscvzyf16Sqrt}{\saildocfn{}{\lstinputlisting[language=sail]{sail_latex_riscv/fnzriscv_f16sqrt5d5f1cbd96b6420edc617405f717d841.tex}}}}

\newcommand{\sailRISCVvalexternFThreeTwoSqrt}{\saildoclabelled{sailRISCVzexternzyf32Sqrt}{\saildocval{}{\lstinputlisting[language=sail]{sail_latex_riscv/valzextern_f32sqrtf11b42be8c2478d6f003de6a9af154f6.tex}}}}

\newcommand{\sailRISCVvalriscvFThreeTwoSqrt}{\saildoclabelled{sailRISCVzriscvzyf32Sqrt}{\saildocval{}{\lstinputlisting[language=sail]{sail_latex_riscv/valzriscv_f32sqrtd0c4078227e28f52a7984b6a7f89ce8a.tex}}}}

\newcommand{\sailRISCVfnriscvFThreeTwoSqrt}{\saildoclabelled{sailRISCVfnzriscvzyf32Sqrt}{\saildocfn{}{\lstinputlisting[language=sail]{sail_latex_riscv/fnzriscv_f32sqrtd0c4078227e28f52a7984b6a7f89ce8a.tex}}}}

\newcommand{\sailRISCVvalexternFSixFourSqrt}{\saildoclabelled{sailRISCVzexternzyf64Sqrt}{\saildocval{}{\lstinputlisting[language=sail]{sail_latex_riscv/valzextern_f64sqrta3145a122395bda30fb542d678fd4539.tex}}}}

\newcommand{\sailRISCVvalriscvFSixFourSqrt}{\saildoclabelled{sailRISCVzriscvzyf64Sqrt}{\saildocval{}{\lstinputlisting[language=sail]{sail_latex_riscv/valzriscv_f64sqrta506b6089ecf799deb21b124eaf60f0c.tex}}}}

\newcommand{\sailRISCVfnriscvFSixFourSqrt}{\saildoclabelled{sailRISCVfnzriscvzyf64Sqrt}{\saildocfn{}{\lstinputlisting[language=sail]{sail_latex_riscv/fnzriscv_f64sqrta506b6089ecf799deb21b124eaf60f0c.tex}}}}

\newcommand{\sailRISCVvalexternFOneSixToIThreeTwo}{\saildoclabelled{sailRISCVzexternzyf16ToI32}{\saildocval{}{\lstinputlisting[language=sail]{sail_latex_riscv/valzextern_f16toi3256de2c14ac166a9d57dec8ff9f253046.tex}}}}

\newcommand{\sailRISCVvalriscvFOneSixToIThreeTwo}{\saildoclabelled{sailRISCVzriscvzyf16ToI32}{\saildocval{}{\lstinputlisting[language=sail]{sail_latex_riscv/valzriscv_f16toi3299a98a77aaad7c47d620ff5857f9c060.tex}}}}

\newcommand{\sailRISCVfnriscvFOneSixToIThreeTwo}{\saildoclabelled{sailRISCVfnzriscvzyf16ToI32}{\saildocfn{}{\lstinputlisting[language=sail]{sail_latex_riscv/fnzriscv_f16toi3299a98a77aaad7c47d620ff5857f9c060.tex}}}}

\newcommand{\sailRISCVvalexternFOneSixToUiThreeTwo}{\saildoclabelled{sailRISCVzexternzyf16ToUi32}{\saildocval{}{\lstinputlisting[language=sail]{sail_latex_riscv/valzextern_f16toui32c92946eea3e8d6e68dd0536680ebdb2c.tex}}}}

\newcommand{\sailRISCVvalriscvFOneSixToUiThreeTwo}{\saildoclabelled{sailRISCVzriscvzyf16ToUi32}{\saildocval{}{\lstinputlisting[language=sail]{sail_latex_riscv/valzriscv_f16toui32dc059f0785a586dfb30c79f4143ab6ec.tex}}}}

\newcommand{\sailRISCVfnriscvFOneSixToUiThreeTwo}{\saildoclabelled{sailRISCVfnzriscvzyf16ToUi32}{\saildocfn{}{\lstinputlisting[language=sail]{sail_latex_riscv/fnzriscv_f16toui32dc059f0785a586dfb30c79f4143ab6ec.tex}}}}

\newcommand{\sailRISCVvalexternIThreeTwoToFOneSix}{\saildoclabelled{sailRISCVzexternzyi32ToF16}{\saildocval{}{\lstinputlisting[language=sail]{sail_latex_riscv/valzextern_i32tof16e689b5ec15472584b18d0e538ad0f413.tex}}}}

\newcommand{\sailRISCVvalriscvIThreeTwoToFOneSix}{\saildoclabelled{sailRISCVzriscvzyi32ToF16}{\saildocval{}{\lstinputlisting[language=sail]{sail_latex_riscv/valzriscv_i32tof162c353b802a36b82f3ec8983618baa4d8.tex}}}}

\newcommand{\sailRISCVfnriscvIThreeTwoToFOneSix}{\saildoclabelled{sailRISCVfnzriscvzyi32ToF16}{\saildocfn{}{\lstinputlisting[language=sail]{sail_latex_riscv/fnzriscv_i32tof162c353b802a36b82f3ec8983618baa4d8.tex}}}}

\newcommand{\sailRISCVvalexternUiThreeTwoToFOneSix}{\saildoclabelled{sailRISCVzexternzyui32ToF16}{\saildocval{}{\lstinputlisting[language=sail]{sail_latex_riscv/valzextern_ui32tof16f343a57214d73f2d46f58dc92f4aeead.tex}}}}

\newcommand{\sailRISCVvalriscvUiThreeTwoToFOneSix}{\saildoclabelled{sailRISCVzriscvzyui32ToF16}{\saildocval{}{\lstinputlisting[language=sail]{sail_latex_riscv/valzriscv_ui32tof16ce7781a2ef9725b4b24c236380870b4a.tex}}}}

\newcommand{\sailRISCVfnriscvUiThreeTwoToFOneSix}{\saildoclabelled{sailRISCVfnzriscvzyui32ToF16}{\saildocfn{}{\lstinputlisting[language=sail]{sail_latex_riscv/fnzriscv_ui32tof16ce7781a2ef9725b4b24c236380870b4a.tex}}}}

\newcommand{\sailRISCVvalexternFOneSixToISixFour}{\saildoclabelled{sailRISCVzexternzyf16ToI64}{\saildocval{}{\lstinputlisting[language=sail]{sail_latex_riscv/valzextern_f16toi64389fc156cbab27e4d89d7c88835293c0.tex}}}}

\newcommand{\sailRISCVvalriscvFOneSixToISixFour}{\saildoclabelled{sailRISCVzriscvzyf16ToI64}{\saildocval{}{\lstinputlisting[language=sail]{sail_latex_riscv/valzriscv_f16toi64783247df0ac6c1f02f8c46bc6b4859c0.tex}}}}

\newcommand{\sailRISCVfnriscvFOneSixToISixFour}{\saildoclabelled{sailRISCVfnzriscvzyf16ToI64}{\saildocfn{}{\lstinputlisting[language=sail]{sail_latex_riscv/fnzriscv_f16toi64783247df0ac6c1f02f8c46bc6b4859c0.tex}}}}

\newcommand{\sailRISCVvalexternFOneSixToUiSixFour}{\saildoclabelled{sailRISCVzexternzyf16ToUi64}{\saildocval{}{\lstinputlisting[language=sail]{sail_latex_riscv/valzextern_f16toui643d1838e9d9063658f48e2dd5d3245bf0.tex}}}}

\newcommand{\sailRISCVvalriscvFOneSixToUiSixFour}{\saildoclabelled{sailRISCVzriscvzyf16ToUi64}{\saildocval{}{\lstinputlisting[language=sail]{sail_latex_riscv/valzriscv_f16toui64a7b8c60fb87de80d904c2fc0644b0271.tex}}}}

\newcommand{\sailRISCVfnriscvFOneSixToUiSixFour}{\saildoclabelled{sailRISCVfnzriscvzyf16ToUi64}{\saildocfn{}{\lstinputlisting[language=sail]{sail_latex_riscv/fnzriscv_f16toui64a7b8c60fb87de80d904c2fc0644b0271.tex}}}}

\newcommand{\sailRISCVvalexternISixFourToFOneSix}{\saildoclabelled{sailRISCVzexternzyi64ToF16}{\saildocval{}{\lstinputlisting[language=sail]{sail_latex_riscv/valzextern_i64tof168a4bd290e8c00e9e61d9ac21e7ed059a.tex}}}}

\newcommand{\sailRISCVvalriscvISixFourToFOneSix}{\saildoclabelled{sailRISCVzriscvzyi64ToF16}{\saildocval{}{\lstinputlisting[language=sail]{sail_latex_riscv/valzriscv_i64tof16942644c07d0c85d4861af8402539995f.tex}}}}

\newcommand{\sailRISCVfnriscvISixFourToFOneSix}{\saildoclabelled{sailRISCVfnzriscvzyi64ToF16}{\saildocfn{}{\lstinputlisting[language=sail]{sail_latex_riscv/fnzriscv_i64tof16942644c07d0c85d4861af8402539995f.tex}}}}

\newcommand{\sailRISCVvalexternUiSixFourToFOneSix}{\saildoclabelled{sailRISCVzexternzyui64ToF16}{\saildocval{}{\lstinputlisting[language=sail]{sail_latex_riscv/valzextern_ui64tof16390eeb070edaf7db0fc0cd153317a930.tex}}}}

\newcommand{\sailRISCVvalriscvUiSixFourToFOneSix}{\saildoclabelled{sailRISCVzriscvzyui64ToF16}{\saildocval{}{\lstinputlisting[language=sail]{sail_latex_riscv/valzriscv_ui64tof16ac0ab658a6409ae2cbd906db815d39b6.tex}}}}

\newcommand{\sailRISCVfnriscvUiSixFourToFOneSix}{\saildoclabelled{sailRISCVfnzriscvzyui64ToF16}{\saildocfn{}{\lstinputlisting[language=sail]{sail_latex_riscv/fnzriscv_ui64tof16ac0ab658a6409ae2cbd906db815d39b6.tex}}}}

\newcommand{\sailRISCVvalexternFThreeTwoToIThreeTwo}{\saildoclabelled{sailRISCVzexternzyf32ToI32}{\saildocval{}{\lstinputlisting[language=sail]{sail_latex_riscv/valzextern_f32toi326cf4b3bab68206b610e134ff8908ca89.tex}}}}

\newcommand{\sailRISCVvalriscvFThreeTwoToIThreeTwo}{\saildoclabelled{sailRISCVzriscvzyf32ToI32}{\saildocval{}{\lstinputlisting[language=sail]{sail_latex_riscv/valzriscv_f32toi32b5d471852e14b8b79d2ad8fd065ac832.tex}}}}

\newcommand{\sailRISCVfnriscvFThreeTwoToIThreeTwo}{\saildoclabelled{sailRISCVfnzriscvzyf32ToI32}{\saildocfn{}{\lstinputlisting[language=sail]{sail_latex_riscv/fnzriscv_f32toi32b5d471852e14b8b79d2ad8fd065ac832.tex}}}}

\newcommand{\sailRISCVvalexternFThreeTwoToUiThreeTwo}{\saildoclabelled{sailRISCVzexternzyf32ToUi32}{\saildocval{}{\lstinputlisting[language=sail]{sail_latex_riscv/valzextern_f32toui3254cab07407e1a29139940b162c1c3d28.tex}}}}

\newcommand{\sailRISCVvalriscvFThreeTwoToUiThreeTwo}{\saildoclabelled{sailRISCVzriscvzyf32ToUi32}{\saildocval{}{\lstinputlisting[language=sail]{sail_latex_riscv/valzriscv_f32toui329abf292f8a5b1458fd639f0b62a9b052.tex}}}}

\newcommand{\sailRISCVfnriscvFThreeTwoToUiThreeTwo}{\saildoclabelled{sailRISCVfnzriscvzyf32ToUi32}{\saildocfn{}{\lstinputlisting[language=sail]{sail_latex_riscv/fnzriscv_f32toui329abf292f8a5b1458fd639f0b62a9b052.tex}}}}

\newcommand{\sailRISCVvalexternIThreeTwoToFThreeTwo}{\saildoclabelled{sailRISCVzexternzyi32ToF32}{\saildocval{}{\lstinputlisting[language=sail]{sail_latex_riscv/valzextern_i32tof32928758eb5e1cef38c85f3df1f2fa8faa.tex}}}}

\newcommand{\sailRISCVvalriscvIThreeTwoToFThreeTwo}{\saildoclabelled{sailRISCVzriscvzyi32ToF32}{\saildocval{}{\lstinputlisting[language=sail]{sail_latex_riscv/valzriscv_i32tof32262d900b5276fc61cfada0a2c080dd00.tex}}}}

\newcommand{\sailRISCVfnriscvIThreeTwoToFThreeTwo}{\saildoclabelled{sailRISCVfnzriscvzyi32ToF32}{\saildocfn{}{\lstinputlisting[language=sail]{sail_latex_riscv/fnzriscv_i32tof32262d900b5276fc61cfada0a2c080dd00.tex}}}}

\newcommand{\sailRISCVvalexternUiThreeTwoToFThreeTwo}{\saildoclabelled{sailRISCVzexternzyui32ToF32}{\saildocval{}{\lstinputlisting[language=sail]{sail_latex_riscv/valzextern_ui32tof320d34a36731fc04f9878fe858410678ad.tex}}}}

\newcommand{\sailRISCVvalriscvUiThreeTwoToFThreeTwo}{\saildoclabelled{sailRISCVzriscvzyui32ToF32}{\saildocval{}{\lstinputlisting[language=sail]{sail_latex_riscv/valzriscv_ui32tof3277c5e2e314e991c265ecd083fea05bf0.tex}}}}

\newcommand{\sailRISCVfnriscvUiThreeTwoToFThreeTwo}{\saildoclabelled{sailRISCVfnzriscvzyui32ToF32}{\saildocfn{}{\lstinputlisting[language=sail]{sail_latex_riscv/fnzriscv_ui32tof3277c5e2e314e991c265ecd083fea05bf0.tex}}}}

\newcommand{\sailRISCVvalexternFThreeTwoToISixFour}{\saildoclabelled{sailRISCVzexternzyf32ToI64}{\saildocval{}{\lstinputlisting[language=sail]{sail_latex_riscv/valzextern_f32toi640b3f54fe8e434e95ad11bc87f4692a8c.tex}}}}

\newcommand{\sailRISCVvalriscvFThreeTwoToISixFour}{\saildoclabelled{sailRISCVzriscvzyf32ToI64}{\saildocval{}{\lstinputlisting[language=sail]{sail_latex_riscv/valzriscv_f32toi640208fd5f6c45fde93d3e7c63a2825d81.tex}}}}

\newcommand{\sailRISCVfnriscvFThreeTwoToISixFour}{\saildoclabelled{sailRISCVfnzriscvzyf32ToI64}{\saildocfn{}{\lstinputlisting[language=sail]{sail_latex_riscv/fnzriscv_f32toi640208fd5f6c45fde93d3e7c63a2825d81.tex}}}}

\newcommand{\sailRISCVvalexternFThreeTwoToUiSixFour}{\saildoclabelled{sailRISCVzexternzyf32ToUi64}{\saildocval{}{\lstinputlisting[language=sail]{sail_latex_riscv/valzextern_f32toui6494114a44e0de80b91ab99de5cb70ae02.tex}}}}

\newcommand{\sailRISCVvalriscvFThreeTwoToUiSixFour}{\saildoclabelled{sailRISCVzriscvzyf32ToUi64}{\saildocval{}{\lstinputlisting[language=sail]{sail_latex_riscv/valzriscv_f32toui6423479b6a7ca0f54a5c926da7d7620403.tex}}}}

\newcommand{\sailRISCVfnriscvFThreeTwoToUiSixFour}{\saildoclabelled{sailRISCVfnzriscvzyf32ToUi64}{\saildocfn{}{\lstinputlisting[language=sail]{sail_latex_riscv/fnzriscv_f32toui6423479b6a7ca0f54a5c926da7d7620403.tex}}}}

\newcommand{\sailRISCVvalexternISixFourToFThreeTwo}{\saildoclabelled{sailRISCVzexternzyi64ToF32}{\saildocval{}{\lstinputlisting[language=sail]{sail_latex_riscv/valzextern_i64tof3238824aafa28223d09bd70270bcc53a19.tex}}}}

\newcommand{\sailRISCVvalriscvISixFourToFThreeTwo}{\saildoclabelled{sailRISCVzriscvzyi64ToF32}{\saildocval{}{\lstinputlisting[language=sail]{sail_latex_riscv/valzriscv_i64tof329ea93669bfe183b3595b128f97eb0d64.tex}}}}

\newcommand{\sailRISCVfnriscvISixFourToFThreeTwo}{\saildoclabelled{sailRISCVfnzriscvzyi64ToF32}{\saildocfn{}{\lstinputlisting[language=sail]{sail_latex_riscv/fnzriscv_i64tof329ea93669bfe183b3595b128f97eb0d64.tex}}}}

\newcommand{\sailRISCVvalexternUiSixFourToFThreeTwo}{\saildoclabelled{sailRISCVzexternzyui64ToF32}{\saildocval{}{\lstinputlisting[language=sail]{sail_latex_riscv/valzextern_ui64tof3222147d78d464deaed8427dd5e9a1f97d.tex}}}}

\newcommand{\sailRISCVvalriscvUiSixFourToFThreeTwo}{\saildoclabelled{sailRISCVzriscvzyui64ToF32}{\saildocval{}{\lstinputlisting[language=sail]{sail_latex_riscv/valzriscv_ui64tof32006c37b85131ebfdf4c65a0ecf2d321b.tex}}}}

\newcommand{\sailRISCVfnriscvUiSixFourToFThreeTwo}{\saildoclabelled{sailRISCVfnzriscvzyui64ToF32}{\saildocfn{}{\lstinputlisting[language=sail]{sail_latex_riscv/fnzriscv_ui64tof32006c37b85131ebfdf4c65a0ecf2d321b.tex}}}}

\newcommand{\sailRISCVvalexternFSixFourToIThreeTwo}{\saildoclabelled{sailRISCVzexternzyf64ToI32}{\saildocval{}{\lstinputlisting[language=sail]{sail_latex_riscv/valzextern_f64toi328ccd500b2d8a0508350bd93919a71172.tex}}}}

\newcommand{\sailRISCVvalriscvFSixFourToIThreeTwo}{\saildoclabelled{sailRISCVzriscvzyf64ToI32}{\saildocval{}{\lstinputlisting[language=sail]{sail_latex_riscv/valzriscv_f64toi32390e064989835e074118b56d834e6d48.tex}}}}

\newcommand{\sailRISCVfnriscvFSixFourToIThreeTwo}{\saildoclabelled{sailRISCVfnzriscvzyf64ToI32}{\saildocfn{}{\lstinputlisting[language=sail]{sail_latex_riscv/fnzriscv_f64toi32390e064989835e074118b56d834e6d48.tex}}}}

\newcommand{\sailRISCVvalexternFSixFourToUiThreeTwo}{\saildoclabelled{sailRISCVzexternzyf64ToUi32}{\saildocval{}{\lstinputlisting[language=sail]{sail_latex_riscv/valzextern_f64toui3201aa4791323c1c065734b1b292dfc483.tex}}}}

\newcommand{\sailRISCVvalriscvFSixFourToUiThreeTwo}{\saildoclabelled{sailRISCVzriscvzyf64ToUi32}{\saildocval{}{\lstinputlisting[language=sail]{sail_latex_riscv/valzriscv_f64toui32df8698a76aef244e4d021f6db07ed8aa.tex}}}}

\newcommand{\sailRISCVfnriscvFSixFourToUiThreeTwo}{\saildoclabelled{sailRISCVfnzriscvzyf64ToUi32}{\saildocfn{}{\lstinputlisting[language=sail]{sail_latex_riscv/fnzriscv_f64toui32df8698a76aef244e4d021f6db07ed8aa.tex}}}}

\newcommand{\sailRISCVvalexternIThreeTwoToFSixFour}{\saildoclabelled{sailRISCVzexternzyi32ToF64}{\saildocval{}{\lstinputlisting[language=sail]{sail_latex_riscv/valzextern_i32tof64fd00956ce74e2f05bee6c0c09b2fe4e9.tex}}}}

\newcommand{\sailRISCVvalriscvIThreeTwoToFSixFour}{\saildoclabelled{sailRISCVzriscvzyi32ToF64}{\saildocval{}{\lstinputlisting[language=sail]{sail_latex_riscv/valzriscv_i32tof64e97e000de9456cb796f52f30b64c9715.tex}}}}

\newcommand{\sailRISCVfnriscvIThreeTwoToFSixFour}{\saildoclabelled{sailRISCVfnzriscvzyi32ToF64}{\saildocfn{}{\lstinputlisting[language=sail]{sail_latex_riscv/fnzriscv_i32tof64e97e000de9456cb796f52f30b64c9715.tex}}}}

\newcommand{\sailRISCVvalexternUiThreeTwoToFSixFour}{\saildoclabelled{sailRISCVzexternzyui32ToF64}{\saildocval{}{\lstinputlisting[language=sail]{sail_latex_riscv/valzextern_ui32tof64e2fa1dc20fc15cc4e3898f79cd2790c4.tex}}}}

\newcommand{\sailRISCVvalriscvUiThreeTwoToFSixFour}{\saildoclabelled{sailRISCVzriscvzyui32ToF64}{\saildocval{}{\lstinputlisting[language=sail]{sail_latex_riscv/valzriscv_ui32tof6485dd2d5d0818e64745a16c8bd2e6f3ad.tex}}}}

\newcommand{\sailRISCVfnriscvUiThreeTwoToFSixFour}{\saildoclabelled{sailRISCVfnzriscvzyui32ToF64}{\saildocfn{}{\lstinputlisting[language=sail]{sail_latex_riscv/fnzriscv_ui32tof6485dd2d5d0818e64745a16c8bd2e6f3ad.tex}}}}

\newcommand{\sailRISCVvalexternFSixFourToISixFour}{\saildoclabelled{sailRISCVzexternzyf64ToI64}{\saildocval{}{\lstinputlisting[language=sail]{sail_latex_riscv/valzextern_f64toi641dd890d7fbf7a24774c052a9c246250f.tex}}}}

\newcommand{\sailRISCVvalriscvFSixFourToISixFour}{\saildoclabelled{sailRISCVzriscvzyf64ToI64}{\saildocval{}{\lstinputlisting[language=sail]{sail_latex_riscv/valzriscv_f64toi64b279b582ab789b7d76bd4628cf864db9.tex}}}}

\newcommand{\sailRISCVfnriscvFSixFourToISixFour}{\saildoclabelled{sailRISCVfnzriscvzyf64ToI64}{\saildocfn{}{\lstinputlisting[language=sail]{sail_latex_riscv/fnzriscv_f64toi64b279b582ab789b7d76bd4628cf864db9.tex}}}}

\newcommand{\sailRISCVvalexternFSixFourToUiSixFour}{\saildoclabelled{sailRISCVzexternzyf64ToUi64}{\saildocval{}{\lstinputlisting[language=sail]{sail_latex_riscv/valzextern_f64toui64941713ad89e209a4feb9e7b47c7194bb.tex}}}}

\newcommand{\sailRISCVvalriscvFSixFourToUiSixFour}{\saildoclabelled{sailRISCVzriscvzyf64ToUi64}{\saildocval{}{\lstinputlisting[language=sail]{sail_latex_riscv/valzriscv_f64toui64a8c06df736f513b3e5ec73bfb9385733.tex}}}}

\newcommand{\sailRISCVfnriscvFSixFourToUiSixFour}{\saildoclabelled{sailRISCVfnzriscvzyf64ToUi64}{\saildocfn{}{\lstinputlisting[language=sail]{sail_latex_riscv/fnzriscv_f64toui64a8c06df736f513b3e5ec73bfb9385733.tex}}}}

\newcommand{\sailRISCVvalexternISixFourToFSixFour}{\saildoclabelled{sailRISCVzexternzyi64ToF64}{\saildocval{}{\lstinputlisting[language=sail]{sail_latex_riscv/valzextern_i64tof64aed9270eafe2e670b0a96825020ff419.tex}}}}

\newcommand{\sailRISCVvalriscvISixFourToFSixFour}{\saildoclabelled{sailRISCVzriscvzyi64ToF64}{\saildocval{}{\lstinputlisting[language=sail]{sail_latex_riscv/valzriscv_i64tof645552b730328b5e488a5469c4d60ba48e.tex}}}}

\newcommand{\sailRISCVfnriscvISixFourToFSixFour}{\saildoclabelled{sailRISCVfnzriscvzyi64ToF64}{\saildocfn{}{\lstinputlisting[language=sail]{sail_latex_riscv/fnzriscv_i64tof645552b730328b5e488a5469c4d60ba48e.tex}}}}

\newcommand{\sailRISCVvalexternUiSixFourToFSixFour}{\saildoclabelled{sailRISCVzexternzyui64ToF64}{\saildocval{}{\lstinputlisting[language=sail]{sail_latex_riscv/valzextern_ui64tof64b071761b350c914c21ca717b7ff70706.tex}}}}

\newcommand{\sailRISCVvalriscvUiSixFourToFSixFour}{\saildoclabelled{sailRISCVzriscvzyui64ToF64}{\saildocval{}{\lstinputlisting[language=sail]{sail_latex_riscv/valzriscv_ui64tof647831c3694f6a0849ff1899ade6b08220.tex}}}}

\newcommand{\sailRISCVfnriscvUiSixFourToFSixFour}{\saildoclabelled{sailRISCVfnzriscvzyui64ToF64}{\saildocfn{}{\lstinputlisting[language=sail]{sail_latex_riscv/fnzriscv_ui64tof647831c3694f6a0849ff1899ade6b08220.tex}}}}

\newcommand{\sailRISCVvalexternFOneSixToFThreeTwo}{\saildoclabelled{sailRISCVzexternzyf16ToF32}{\saildocval{}{\lstinputlisting[language=sail]{sail_latex_riscv/valzextern_f16tof32cf8977ca8a5f817ab8eda3a5d99c5e95.tex}}}}

\newcommand{\sailRISCVvalriscvFOneSixToFThreeTwo}{\saildoclabelled{sailRISCVzriscvzyf16ToF32}{\saildocval{}{\lstinputlisting[language=sail]{sail_latex_riscv/valzriscv_f16tof32ab41d016238113d92beff7a7746957be.tex}}}}

\newcommand{\sailRISCVfnriscvFOneSixToFThreeTwo}{\saildoclabelled{sailRISCVfnzriscvzyf16ToF32}{\saildocfn{}{\lstinputlisting[language=sail]{sail_latex_riscv/fnzriscv_f16tof32ab41d016238113d92beff7a7746957be.tex}}}}

\newcommand{\sailRISCVvalexternFOneSixToFSixFour}{\saildoclabelled{sailRISCVzexternzyf16ToF64}{\saildocval{}{\lstinputlisting[language=sail]{sail_latex_riscv/valzextern_f16tof6480e664b68ba5f60a1fe798199310f72e.tex}}}}

\newcommand{\sailRISCVvalriscvFOneSixToFSixFour}{\saildoclabelled{sailRISCVzriscvzyf16ToF64}{\saildocval{}{\lstinputlisting[language=sail]{sail_latex_riscv/valzriscv_f16tof6425ed6d31ceae83d607a03e8d319dcb17.tex}}}}

\newcommand{\sailRISCVfnriscvFOneSixToFSixFour}{\saildoclabelled{sailRISCVfnzriscvzyf16ToF64}{\saildocfn{}{\lstinputlisting[language=sail]{sail_latex_riscv/fnzriscv_f16tof6425ed6d31ceae83d607a03e8d319dcb17.tex}}}}

\newcommand{\sailRISCVvalexternFThreeTwoToFSixFour}{\saildoclabelled{sailRISCVzexternzyf32ToF64}{\saildocval{}{\lstinputlisting[language=sail]{sail_latex_riscv/valzextern_f32tof64195ee529424460f0ea430364e6969b58.tex}}}}

\newcommand{\sailRISCVvalriscvFThreeTwoToFSixFour}{\saildoclabelled{sailRISCVzriscvzyf32ToF64}{\saildocval{}{\lstinputlisting[language=sail]{sail_latex_riscv/valzriscv_f32tof6463cfa807ac39aea5d45f7cb4ee2d9248.tex}}}}

\newcommand{\sailRISCVfnriscvFThreeTwoToFSixFour}{\saildoclabelled{sailRISCVfnzriscvzyf32ToF64}{\saildocfn{}{\lstinputlisting[language=sail]{sail_latex_riscv/fnzriscv_f32tof6463cfa807ac39aea5d45f7cb4ee2d9248.tex}}}}

\newcommand{\sailRISCVvalexternFThreeTwoToFOneSix}{\saildoclabelled{sailRISCVzexternzyf32ToF16}{\saildocval{}{\lstinputlisting[language=sail]{sail_latex_riscv/valzextern_f32tof1660ef14b0a82ddb2b87ba2c6ccbb647ff.tex}}}}

\newcommand{\sailRISCVvalriscvFThreeTwoToFOneSix}{\saildoclabelled{sailRISCVzriscvzyf32ToF16}{\saildocval{}{\lstinputlisting[language=sail]{sail_latex_riscv/valzriscv_f32tof164420d7f9129486d00b6dba20c0caa0fd.tex}}}}

\newcommand{\sailRISCVfnriscvFThreeTwoToFOneSix}{\saildoclabelled{sailRISCVfnzriscvzyf32ToF16}{\saildocfn{}{\lstinputlisting[language=sail]{sail_latex_riscv/fnzriscv_f32tof164420d7f9129486d00b6dba20c0caa0fd.tex}}}}

\newcommand{\sailRISCVvalexternFSixFourToFOneSix}{\saildoclabelled{sailRISCVzexternzyf64ToF16}{\saildocval{}{\lstinputlisting[language=sail]{sail_latex_riscv/valzextern_f64tof16a8bfffa793b3f992f6359016fd43f308.tex}}}}

\newcommand{\sailRISCVvalriscvFSixFourToFOneSix}{\saildoclabelled{sailRISCVzriscvzyf64ToF16}{\saildocval{}{\lstinputlisting[language=sail]{sail_latex_riscv/valzriscv_f64tof16c801f12e89aeaa1e2641c74d852c486e.tex}}}}

\newcommand{\sailRISCVfnriscvFSixFourToFOneSix}{\saildoclabelled{sailRISCVfnzriscvzyf64ToF16}{\saildocfn{}{\lstinputlisting[language=sail]{sail_latex_riscv/fnzriscv_f64tof16c801f12e89aeaa1e2641c74d852c486e.tex}}}}

\newcommand{\sailRISCVvalexternFSixFourToFThreeTwo}{\saildoclabelled{sailRISCVzexternzyf64ToF32}{\saildocval{}{\lstinputlisting[language=sail]{sail_latex_riscv/valzextern_f64tof329e6bdb39c58bf46aad8078a255a69675.tex}}}}

\newcommand{\sailRISCVvalriscvFSixFourToFThreeTwo}{\saildoclabelled{sailRISCVzriscvzyf64ToF32}{\saildocval{}{\lstinputlisting[language=sail]{sail_latex_riscv/valzriscv_f64tof326b35917fadbed27b5d3cecd2f8762d01.tex}}}}

\newcommand{\sailRISCVfnriscvFSixFourToFThreeTwo}{\saildoclabelled{sailRISCVfnzriscvzyf64ToF32}{\saildocfn{}{\lstinputlisting[language=sail]{sail_latex_riscv/fnzriscv_f64tof326b35917fadbed27b5d3cecd2f8762d01.tex}}}}

\newcommand{\sailRISCVvalexternFOneSixLt}{\saildoclabelled{sailRISCVzexternzyf16Lt}{\saildocval{}{\lstinputlisting[language=sail]{sail_latex_riscv/valzextern_f16ltf93cebdb1734d311d99583448a73b545.tex}}}}

\newcommand{\sailRISCVvalriscvFOneSixLt}{\saildoclabelled{sailRISCVzriscvzyf16Lt}{\saildocval{}{\lstinputlisting[language=sail]{sail_latex_riscv/valzriscv_f16lta3ed2ba63f2bca9026b606ccfeb6ab64.tex}}}}

\newcommand{\sailRISCVfnriscvFOneSixLt}{\saildoclabelled{sailRISCVfnzriscvzyf16Lt}{\saildocfn{}{\lstinputlisting[language=sail]{sail_latex_riscv/fnzriscv_f16lta3ed2ba63f2bca9026b606ccfeb6ab64.tex}}}}

\newcommand{\sailRISCVvalexternFOneSixLe}{\saildoclabelled{sailRISCVzexternzyf16Le}{\saildocval{}{\lstinputlisting[language=sail]{sail_latex_riscv/valzextern_f16le82541f452cc5ef9cea2bf63910c7157d.tex}}}}

\newcommand{\sailRISCVvalriscvFOneSixLe}{\saildoclabelled{sailRISCVzriscvzyf16Le}{\saildocval{}{\lstinputlisting[language=sail]{sail_latex_riscv/valzriscv_f16le898d0c1fa658c59f86be7840a63f5e17.tex}}}}

\newcommand{\sailRISCVfnriscvFOneSixLe}{\saildoclabelled{sailRISCVfnzriscvzyf16Le}{\saildocfn{}{\lstinputlisting[language=sail]{sail_latex_riscv/fnzriscv_f16le898d0c1fa658c59f86be7840a63f5e17.tex}}}}

\newcommand{\sailRISCVvalexternFOneSixEq}{\saildoclabelled{sailRISCVzexternzyf16Eq}{\saildocval{}{\lstinputlisting[language=sail]{sail_latex_riscv/valzextern_f16eq051178e19b597007271114f150c7bf25.tex}}}}

\newcommand{\sailRISCVvalriscvFOneSixEq}{\saildoclabelled{sailRISCVzriscvzyf16Eq}{\saildocval{}{\lstinputlisting[language=sail]{sail_latex_riscv/valzriscv_f16eq3d04f3f3d3180635ecf8ff46df64bb5d.tex}}}}

\newcommand{\sailRISCVfnriscvFOneSixEq}{\saildoclabelled{sailRISCVfnzriscvzyf16Eq}{\saildocfn{}{\lstinputlisting[language=sail]{sail_latex_riscv/fnzriscv_f16eq3d04f3f3d3180635ecf8ff46df64bb5d.tex}}}}

\newcommand{\sailRISCVvalexternFThreeTwoLt}{\saildoclabelled{sailRISCVzexternzyf32Lt}{\saildocval{}{\lstinputlisting[language=sail]{sail_latex_riscv/valzextern_f32lt76345a6740b5e66e208ded2a17434c50.tex}}}}

\newcommand{\sailRISCVvalriscvFThreeTwoLt}{\saildoclabelled{sailRISCVzriscvzyf32Lt}{\saildocval{}{\lstinputlisting[language=sail]{sail_latex_riscv/valzriscv_f32lt4e30f0abd945d0950f5c75c3397e58b8.tex}}}}

\newcommand{\sailRISCVfnriscvFThreeTwoLt}{\saildoclabelled{sailRISCVfnzriscvzyf32Lt}{\saildocfn{}{\lstinputlisting[language=sail]{sail_latex_riscv/fnzriscv_f32lt4e30f0abd945d0950f5c75c3397e58b8.tex}}}}

\newcommand{\sailRISCVvalexternFThreeTwoLe}{\saildoclabelled{sailRISCVzexternzyf32Le}{\saildocval{}{\lstinputlisting[language=sail]{sail_latex_riscv/valzextern_f32le97e1d493296eff955b8dd2b74525e2ad.tex}}}}

\newcommand{\sailRISCVvalriscvFThreeTwoLe}{\saildoclabelled{sailRISCVzriscvzyf32Le}{\saildocval{}{\lstinputlisting[language=sail]{sail_latex_riscv/valzriscv_f32le5580a4448a4b6b5ef89a7a80b6f5e23a.tex}}}}

\newcommand{\sailRISCVfnriscvFThreeTwoLe}{\saildoclabelled{sailRISCVfnzriscvzyf32Le}{\saildocfn{}{\lstinputlisting[language=sail]{sail_latex_riscv/fnzriscv_f32le5580a4448a4b6b5ef89a7a80b6f5e23a.tex}}}}

\newcommand{\sailRISCVvalexternFThreeTwoEq}{\saildoclabelled{sailRISCVzexternzyf32Eq}{\saildocval{}{\lstinputlisting[language=sail]{sail_latex_riscv/valzextern_f32eq1152b8b8a8d02d885ad3651e36443f8b.tex}}}}

\newcommand{\sailRISCVvalriscvFThreeTwoEq}{\saildoclabelled{sailRISCVzriscvzyf32Eq}{\saildocval{}{\lstinputlisting[language=sail]{sail_latex_riscv/valzriscv_f32eqbb5f45f4706cd2893dcaf9d3d15d7b11.tex}}}}

\newcommand{\sailRISCVfnriscvFThreeTwoEq}{\saildoclabelled{sailRISCVfnzriscvzyf32Eq}{\saildocfn{}{\lstinputlisting[language=sail]{sail_latex_riscv/fnzriscv_f32eqbb5f45f4706cd2893dcaf9d3d15d7b11.tex}}}}

\newcommand{\sailRISCVvalexternFSixFourLt}{\saildoclabelled{sailRISCVzexternzyf64Lt}{\saildocval{}{\lstinputlisting[language=sail]{sail_latex_riscv/valzextern_f64lt9068248c1389b86f906917568fb33729.tex}}}}

\newcommand{\sailRISCVvalriscvFSixFourLt}{\saildoclabelled{sailRISCVzriscvzyf64Lt}{\saildocval{}{\lstinputlisting[language=sail]{sail_latex_riscv/valzriscv_f64lt7fe1cae2d039f2771557eafe4cff0d62.tex}}}}

\newcommand{\sailRISCVfnriscvFSixFourLt}{\saildoclabelled{sailRISCVfnzriscvzyf64Lt}{\saildocfn{}{\lstinputlisting[language=sail]{sail_latex_riscv/fnzriscv_f64lt7fe1cae2d039f2771557eafe4cff0d62.tex}}}}

\newcommand{\sailRISCVvalexternFSixFourLe}{\saildoclabelled{sailRISCVzexternzyf64Le}{\saildocval{}{\lstinputlisting[language=sail]{sail_latex_riscv/valzextern_f64le607de75fdc830000d71db4eee82f8025.tex}}}}

\newcommand{\sailRISCVvalriscvFSixFourLe}{\saildoclabelled{sailRISCVzriscvzyf64Le}{\saildocval{}{\lstinputlisting[language=sail]{sail_latex_riscv/valzriscv_f64leedff2d39af8855d31503064db9e49593.tex}}}}

\newcommand{\sailRISCVfnriscvFSixFourLe}{\saildoclabelled{sailRISCVfnzriscvzyf64Le}{\saildocfn{}{\lstinputlisting[language=sail]{sail_latex_riscv/fnzriscv_f64leedff2d39af8855d31503064db9e49593.tex}}}}

\newcommand{\sailRISCVvalexternFSixFourEq}{\saildoclabelled{sailRISCVzexternzyf64Eq}{\saildocval{}{\lstinputlisting[language=sail]{sail_latex_riscv/valzextern_f64eqb2180e00de7f0745336b1c04e9ead3f3.tex}}}}

\newcommand{\sailRISCVvalriscvFSixFourEq}{\saildoclabelled{sailRISCVzriscvzyf64Eq}{\saildocval{}{\lstinputlisting[language=sail]{sail_latex_riscv/valzriscv_f64eq911a3686fab30a7dbcecd0d21d7bc788.tex}}}}

\newcommand{\sailRISCVfnriscvFSixFourEq}{\saildoclabelled{sailRISCVfnzriscvzyf64Eq}{\saildocfn{}{\lstinputlisting[language=sail]{sail_latex_riscv/fnzriscv_f64eq911a3686fab30a7dbcecd0d21d7bc788.tex}}}}

\newcommand{\sailRISCVvalcanonicalNaNH}{\saildoclabelled{sailRISCVzcanonicalzyNaNzyH}{\saildocval{}{\lstinputlisting[language=sail]{sail_latex_riscv/valzcanonical_nan_h080e1071b7703771f0be5ac164e23688.tex}}}}

\newcommand{\sailRISCVfncanonicalNaNH}{\saildoclabelled{sailRISCVfnzcanonicalzyNaNzyH}{\saildocfn{}{\lstinputlisting[language=sail]{sail_latex_riscv/fnzcanonical_nan_h080e1071b7703771f0be5ac164e23688.tex}}}}

\newcommand{\sailRISCVvalcanonicalNaNS}{\saildoclabelled{sailRISCVzcanonicalzyNaNzyS}{\saildocval{}{\lstinputlisting[language=sail]{sail_latex_riscv/valzcanonical_nan_s21045df28a8988e296d9749590d92369.tex}}}}

\newcommand{\sailRISCVfncanonicalNaNS}{\saildoclabelled{sailRISCVfnzcanonicalzyNaNzyS}{\saildocfn{}{\lstinputlisting[language=sail]{sail_latex_riscv/fnzcanonical_nan_s21045df28a8988e296d9749590d92369.tex}}}}

\newcommand{\sailRISCVvalcanonicalNaND}{\saildoclabelled{sailRISCVzcanonicalzyNaNzyD}{\saildocval{}{\lstinputlisting[language=sail]{sail_latex_riscv/valzcanonical_nan_d7d468e933fb666b50bdf56e90272edf5.tex}}}}

\newcommand{\sailRISCVfncanonicalNaND}{\saildoclabelled{sailRISCVfnzcanonicalzyNaNzyD}{\saildocfn{}{\lstinputlisting[language=sail]{sail_latex_riscv/fnzcanonical_nan_d7d468e933fb666b50bdf56e90272edf5.tex}}}}

\newcommand{\sailRISCVvalnanBoxH}{\saildoclabelled{sailRISCVznanzyboxzyH}{\saildocval{}{\lstinputlisting[language=sail]{sail_latex_riscv/valznan_box_h0ea85cce86bd78b74e471bf048ab3442.tex}}}}

\newcommand{\sailRISCVfnnanBoxH}{\saildoclabelled{sailRISCVfnznanzyboxzyH}{\saildocfn{}{\lstinputlisting[language=sail]{sail_latex_riscv/fnznan_box_h0ea85cce86bd78b74e471bf048ab3442.tex}}}}

\newcommand{\sailRISCVvalnanUnboxH}{\saildoclabelled{sailRISCVznanzyunboxzyH}{\saildocval{}{\lstinputlisting[language=sail]{sail_latex_riscv/valznan_unbox_hfcf3a50a7d22724f6b0cf636592d56b3.tex}}}}

\newcommand{\sailRISCVfnnanUnboxH}{\saildoclabelled{sailRISCVfnznanzyunboxzyH}{\saildocfn{}{\lstinputlisting[language=sail]{sail_latex_riscv/fnznan_unbox_hfcf3a50a7d22724f6b0cf636592d56b3.tex}}}}

\newcommand{\sailRISCVvalnanBoxS}{\saildoclabelled{sailRISCVznanzyboxzyS}{\saildocval{}{\lstinputlisting[language=sail]{sail_latex_riscv/valznan_box_s0ed68ce062e5d59895be80606e2ad1e0.tex}}}}

\newcommand{\sailRISCVfnnanBoxS}{\saildoclabelled{sailRISCVfnznanzyboxzyS}{\saildocfn{}{\lstinputlisting[language=sail]{sail_latex_riscv/fnznan_box_s0ed68ce062e5d59895be80606e2ad1e0.tex}}}}

\newcommand{\sailRISCVvalnanUnboxS}{\saildoclabelled{sailRISCVznanzyunboxzyS}{\saildocval{}{\lstinputlisting[language=sail]{sail_latex_riscv/valznan_unbox_s2dd91248bdb4ecea3585b7e5b65dec23.tex}}}}

\newcommand{\sailRISCVfnnanUnboxS}{\saildoclabelled{sailRISCVfnznanzyunboxzyS}{\saildocfn{}{\lstinputlisting[language=sail]{sail_latex_riscv/fnznan_unbox_s2dd91248bdb4ecea3585b7e5b65dec23.tex}}}}

\newcommand{\sailRISCVoverloadMMMnanBox}{\saildoclabelled{sailRISCVoverloadMMMznanzybox}{\saildocoverload{}{\lstinputlisting[language=sail]{sail_latex_riscv/overloadMMMznan_boxf593e1648915be3a65ed1e1cf0dc7712.tex}}}}

\newcommand{\sailRISCVoverloadNNNnanUnbox}{\saildoclabelled{sailRISCVoverloadNNNznanzyunbox}{\saildocoverload{}{\lstinputlisting[language=sail]{sail_latex_riscv/overloadNNNznan_unbox6971c840905d637f635a4793907fe38e.tex}}}}

\newcommand{\sailRISCVregisterfZero}{\saildoclabelled{sailRISCVregisterzf0}{\saildocregister{}{\lstinputlisting[language=sail]{sail_latex_riscv/registerzf093fe33c56e357cae27506236b348348d.tex}}}}

\newcommand{\sailRISCVregisterfOne}{\saildoclabelled{sailRISCVregisterzf1}{\saildocregister{}{\lstinputlisting[language=sail]{sail_latex_riscv/registerzf176bdcd0b68d72a1df69e942a2ad6bb87.tex}}}}

\newcommand{\sailRISCVregisterfTwo}{\saildoclabelled{sailRISCVregisterzf2}{\saildocregister{}{\lstinputlisting[language=sail]{sail_latex_riscv/registerzf2e23fe6a3694e0f8c2cdbd14631995f9c.tex}}}}

\newcommand{\sailRISCVregisterfThree}{\saildoclabelled{sailRISCVregisterzf3}{\saildocregister{}{\lstinputlisting[language=sail]{sail_latex_riscv/registerzf3967f438e8c09040b75ac380736f703fa.tex}}}}

\newcommand{\sailRISCVregisterfFour}{\saildoclabelled{sailRISCVregisterzf4}{\saildocregister{}{\lstinputlisting[language=sail]{sail_latex_riscv/registerzf48c5b36ca0c526408033d51066e8cf26a.tex}}}}

\newcommand{\sailRISCVregisterfFive}{\saildoclabelled{sailRISCVregisterzf5}{\saildocregister{}{\lstinputlisting[language=sail]{sail_latex_riscv/registerzf53370844a919455db1021fc1297433ffe.tex}}}}

\newcommand{\sailRISCVregisterfSix}{\saildoclabelled{sailRISCVregisterzf6}{\saildocregister{}{\lstinputlisting[language=sail]{sail_latex_riscv/registerzf681f57ded710aaf7d6bb016091da4ab25.tex}}}}

\newcommand{\sailRISCVregisterfSeven}{\saildoclabelled{sailRISCVregisterzf7}{\saildocregister{}{\lstinputlisting[language=sail]{sail_latex_riscv/registerzf7ffc7692f2b9ce03facaa63a8eb4b0622.tex}}}}

\newcommand{\sailRISCVregisterfEight}{\saildoclabelled{sailRISCVregisterzf8}{\saildocregister{}{\lstinputlisting[language=sail]{sail_latex_riscv/registerzf8e8464397f1264a89d05d98addebac41b.tex}}}}

\newcommand{\sailRISCVregisterfNine}{\saildoclabelled{sailRISCVregisterzf9}{\saildocregister{}{\lstinputlisting[language=sail]{sail_latex_riscv/registerzf9b106492b12040bca4966dbbbdc3a277f.tex}}}}

\newcommand{\sailRISCVregisterfOneZero}{\saildoclabelled{sailRISCVregisterzf10}{\saildocregister{}{\lstinputlisting[language=sail]{sail_latex_riscv/registerzf106f04d76b5e3ed60f5268f04acac9c16c.tex}}}}

\newcommand{\sailRISCVregisterfOneOne}{\saildoclabelled{sailRISCVregisterzf11}{\saildocregister{}{\lstinputlisting[language=sail]{sail_latex_riscv/registerzf11b15fea1cf950b070aafa041418eda929.tex}}}}

\newcommand{\sailRISCVregisterfOneTwo}{\saildoclabelled{sailRISCVregisterzf12}{\saildocregister{}{\lstinputlisting[language=sail]{sail_latex_riscv/registerzf123defe181040f28d4c27ce9dea6697e5d.tex}}}}

\newcommand{\sailRISCVregisterfOneThree}{\saildoclabelled{sailRISCVregisterzf13}{\saildocregister{}{\lstinputlisting[language=sail]{sail_latex_riscv/registerzf13b4f32b8e44f83775deadbcef47ed95a2.tex}}}}

\newcommand{\sailRISCVregisterfOneFour}{\saildoclabelled{sailRISCVregisterzf14}{\saildocregister{}{\lstinputlisting[language=sail]{sail_latex_riscv/registerzf144f5e0527533a2c91da3a69b54d87c703.tex}}}}

\newcommand{\sailRISCVregisterfOneFive}{\saildoclabelled{sailRISCVregisterzf15}{\saildocregister{}{\lstinputlisting[language=sail]{sail_latex_riscv/registerzf158321621e2e95ea25665930995998e7c4.tex}}}}

\newcommand{\sailRISCVregisterfOneSix}{\saildoclabelled{sailRISCVregisterzf16}{\saildocregister{}{\lstinputlisting[language=sail]{sail_latex_riscv/registerzf1667e9efb739790e3612e923b3bcfb3a9b.tex}}}}

\newcommand{\sailRISCVregisterfOneSeven}{\saildoclabelled{sailRISCVregisterzf17}{\saildocregister{}{\lstinputlisting[language=sail]{sail_latex_riscv/registerzf170e8431c16b19fb052c4d43c0a5f4045a.tex}}}}

\newcommand{\sailRISCVregisterfOneEight}{\saildoclabelled{sailRISCVregisterzf18}{\saildocregister{}{\lstinputlisting[language=sail]{sail_latex_riscv/registerzf1883d9610b319dfe4216a30d8a16ea71db.tex}}}}

\newcommand{\sailRISCVregisterfOneNine}{\saildoclabelled{sailRISCVregisterzf19}{\saildocregister{}{\lstinputlisting[language=sail]{sail_latex_riscv/registerzf19fb4a0a27faa4cca1d4d4de8b8bd18781.tex}}}}

\newcommand{\sailRISCVregisterfTwoZero}{\saildoclabelled{sailRISCVregisterzf20}{\saildocregister{}{\lstinputlisting[language=sail]{sail_latex_riscv/registerzf201bdc6ce89202034e683f95bd2136237e.tex}}}}

\newcommand{\sailRISCVregisterfTwoOne}{\saildoclabelled{sailRISCVregisterzf21}{\saildocregister{}{\lstinputlisting[language=sail]{sail_latex_riscv/registerzf21417078a42509b6cb627d087624aacea5.tex}}}}

\newcommand{\sailRISCVregisterfTwoTwo}{\saildoclabelled{sailRISCVregisterzf22}{\saildocregister{}{\lstinputlisting[language=sail]{sail_latex_riscv/registerzf228b3a0f80b871428c0a05df9c287efcfa.tex}}}}

\newcommand{\sailRISCVregisterfTwoThree}{\saildoclabelled{sailRISCVregisterzf23}{\saildocregister{}{\lstinputlisting[language=sail]{sail_latex_riscv/registerzf23892a277f57d51edd975b7762cb1730d2.tex}}}}

\newcommand{\sailRISCVregisterfTwoFour}{\saildoclabelled{sailRISCVregisterzf24}{\saildocregister{}{\lstinputlisting[language=sail]{sail_latex_riscv/registerzf24a3370f8d590bf8c692f34b28da0a1ae0.tex}}}}

\newcommand{\sailRISCVregisterfTwoFive}{\saildoclabelled{sailRISCVregisterzf25}{\saildocregister{}{\lstinputlisting[language=sail]{sail_latex_riscv/registerzf2552c26f36bf5318fd19f6412d4e07e3e0.tex}}}}

\newcommand{\sailRISCVregisterfTwoSix}{\saildoclabelled{sailRISCVregisterzf26}{\saildocregister{}{\lstinputlisting[language=sail]{sail_latex_riscv/registerzf263995b5fa95645a8a39f3a5a8007735ac.tex}}}}

\newcommand{\sailRISCVregisterfTwoSeven}{\saildoclabelled{sailRISCVregisterzf27}{\saildocregister{}{\lstinputlisting[language=sail]{sail_latex_riscv/registerzf2720d3fee23f2512ed3406c851f1df87ac.tex}}}}

\newcommand{\sailRISCVregisterfTwoEight}{\saildoclabelled{sailRISCVregisterzf28}{\saildocregister{}{\lstinputlisting[language=sail]{sail_latex_riscv/registerzf28aa1e414131418d9e520c1ab7ef388c48.tex}}}}

\newcommand{\sailRISCVregisterfTwoNine}{\saildoclabelled{sailRISCVregisterzf29}{\saildocregister{}{\lstinputlisting[language=sail]{sail_latex_riscv/registerzf29929512a4737a4275e393f45f882b6228.tex}}}}

\newcommand{\sailRISCVregisterfThreeZero}{\saildoclabelled{sailRISCVregisterzf30}{\saildocregister{}{\lstinputlisting[language=sail]{sail_latex_riscv/registerzf3052a3e8e2b54b96c0a43bd0d6d088be8d.tex}}}}

\newcommand{\sailRISCVregisterfThreeOne}{\saildoclabelled{sailRISCVregisterzf31}{\saildocregister{}{\lstinputlisting[language=sail]{sail_latex_riscv/registerzf3173476cdc8af6166eeb9be8c8ea331515.tex}}}}

\newcommand{\sailRISCVvaldirtyFdContext}{\saildoclabelled{sailRISCVzdirtyzyfdzycontext}{\saildocval{}{\lstinputlisting[language=sail]{sail_latex_riscv/valzdirty_fd_context8f0fcc8b7745d7294388295307af4058.tex}}}}

\newcommand{\sailRISCVfndirtyFdContext}{\saildoclabelled{sailRISCVfnzdirtyzyfdzycontext}{\saildocfn{}{\lstinputlisting[language=sail]{sail_latex_riscv/fnzdirty_fd_context8f0fcc8b7745d7294388295307af4058.tex}}}}

\newcommand{\sailRISCVvaldirtyFdContextIfPresent}{\saildoclabelled{sailRISCVzdirtyzyfdzycontextzyifzypresent}{\saildocval{}{\lstinputlisting[language=sail]{sail_latex_riscv/valzdirty_fd_context_if_presentb9ad212aebb8c346cd2f5db5d3190be4.tex}}}}

\newcommand{\sailRISCVfndirtyFdContextIfPresent}{\saildoclabelled{sailRISCVfnzdirtyzyfdzycontextzyifzypresent}{\saildocfn{}{\lstinputlisting[language=sail]{sail_latex_riscv/fnzdirty_fd_context_if_presentb9ad212aebb8c346cd2f5db5d3190be4.tex}}}}

\newcommand{\sailRISCVvalrF}{\saildoclabelled{sailRISCVzrF}{\saildocval{}{\lstinputlisting[language=sail]{sail_latex_riscv/valzrff2ed1bbacd3ac737af7186c6d8884885.tex}}}}

\newcommand{\sailRISCVfnrF}{\saildoclabelled{sailRISCVfnzrF}{\saildocfn{}{\lstinputlisting[language=sail]{sail_latex_riscv/fnzrff2ed1bbacd3ac737af7186c6d8884885.tex}}}}

\newcommand{\sailRISCVvalwF}{\saildoclabelled{sailRISCVzwF}{\saildocval{}{\lstinputlisting[language=sail]{sail_latex_riscv/valzwfd3e403dd75784ce7e0cfdf609e32706e.tex}}}}

\newcommand{\sailRISCVfnwF}{\saildoclabelled{sailRISCVfnzwF}{\saildocfn{}{\lstinputlisting[language=sail]{sail_latex_riscv/fnzwfd3e403dd75784ce7e0cfdf609e32706e.tex}}}}

\newcommand{\sailRISCVvalrFBits}{\saildoclabelled{sailRISCVzrFzybits}{\saildocval{}{\lstinputlisting[language=sail]{sail_latex_riscv/valzrf_bits63ac239a90c8d4bdc4ca1b8375c08531.tex}}}}

\newcommand{\sailRISCVfnrFBits}{\saildoclabelled{sailRISCVfnzrFzybits}{\saildocfn{}{\lstinputlisting[language=sail]{sail_latex_riscv/fnzrf_bits63ac239a90c8d4bdc4ca1b8375c08531.tex}}}}

\newcommand{\sailRISCVvalwFBits}{\saildoclabelled{sailRISCVzwFzybits}{\saildocval{}{\lstinputlisting[language=sail]{sail_latex_riscv/valzwf_bits141c8aafc4ef81f118d39a00d8a5249d.tex}}}}

\newcommand{\sailRISCVfnwFBits}{\saildoclabelled{sailRISCVfnzwFzybits}{\saildocfn{}{\lstinputlisting[language=sail]{sail_latex_riscv/fnzwf_bits141c8aafc4ef81f118d39a00d8a5249d.tex}}}}

\newcommand{\sailRISCVoverloadOOOF}{\saildoclabelled{sailRISCVoverloadOOOzF}{\saildocoverload{}{\lstinputlisting[language=sail]{sail_latex_riscv/overloadOOOzf71805d10b9ea2a56437652aba0daf9bf.tex}}}}

\newcommand{\sailRISCVvalrFOrXH}{\saildoclabelled{sailRISCVzrFzyorzyXzyH}{\saildocval{}{\lstinputlisting[language=sail]{sail_latex_riscv/valzrf_or_x_hd96ee68afa00d1a1a16a94c3a6ff0e02.tex}}}}

\newcommand{\sailRISCVfnrFOrXH}{\saildoclabelled{sailRISCVfnzrFzyorzyXzyH}{\saildocfn{}{\lstinputlisting[language=sail]{sail_latex_riscv/fnzrf_or_x_hd96ee68afa00d1a1a16a94c3a6ff0e02.tex}}}}

\newcommand{\sailRISCVvalrFOrXS}{\saildoclabelled{sailRISCVzrFzyorzyXzyS}{\saildocval{}{\lstinputlisting[language=sail]{sail_latex_riscv/valzrf_or_x_s3a35ce7506a4fa509fd34496e2dbefd3.tex}}}}

\newcommand{\sailRISCVfnrFOrXS}{\saildoclabelled{sailRISCVfnzrFzyorzyXzyS}{\saildocfn{}{\lstinputlisting[language=sail]{sail_latex_riscv/fnzrf_or_x_s3a35ce7506a4fa509fd34496e2dbefd3.tex}}}}

\newcommand{\sailRISCVvalrFOrXD}{\saildoclabelled{sailRISCVzrFzyorzyXzyD}{\saildocval{}{\lstinputlisting[language=sail]{sail_latex_riscv/valzrf_or_x_dd8e1de53f90a8fbd075b7f3c46756ac0.tex}}}}

\newcommand{\sailRISCVfnrFOrXD}{\saildoclabelled{sailRISCVfnzrFzyorzyXzyD}{\saildocfn{}{\lstinputlisting[language=sail]{sail_latex_riscv/fnzrf_or_x_dd8e1de53f90a8fbd075b7f3c46756ac0.tex}}}}

\newcommand{\sailRISCVvalwFOrXH}{\saildoclabelled{sailRISCVzwFzyorzyXzyH}{\saildocval{}{\lstinputlisting[language=sail]{sail_latex_riscv/valzwf_or_x_hc2f88749fb30cde823632a5f1fd09b02.tex}}}}

\newcommand{\sailRISCVfnwFOrXH}{\saildoclabelled{sailRISCVfnzwFzyorzyXzyH}{\saildocfn{}{\lstinputlisting[language=sail]{sail_latex_riscv/fnzwf_or_x_hc2f88749fb30cde823632a5f1fd09b02.tex}}}}

\newcommand{\sailRISCVvalwFOrXS}{\saildoclabelled{sailRISCVzwFzyorzyXzyS}{\saildocval{}{\lstinputlisting[language=sail]{sail_latex_riscv/valzwf_or_x_s315d4d6563a562bd03df2f3d680343b5.tex}}}}

\newcommand{\sailRISCVfnwFOrXS}{\saildoclabelled{sailRISCVfnzwFzyorzyXzyS}{\saildocfn{}{\lstinputlisting[language=sail]{sail_latex_riscv/fnzwf_or_x_s315d4d6563a562bd03df2f3d680343b5.tex}}}}

\newcommand{\sailRISCVvalwFOrXD}{\saildoclabelled{sailRISCVzwFzyorzyXzyD}{\saildocval{}{\lstinputlisting[language=sail]{sail_latex_riscv/valzwf_or_x_d99d1bec09a43ea1c3cba945cc2ebd80e.tex}}}}

\newcommand{\sailRISCVfnwFOrXD}{\saildoclabelled{sailRISCVfnzwFzyorzyXzyD}{\saildocfn{}{\lstinputlisting[language=sail]{sail_latex_riscv/fnzwf_or_x_d99d1bec09a43ea1c3cba945cc2ebd80e.tex}}}}

\newcommand{\sailRISCVoverloadPPPFOrXH}{\saildoclabelled{sailRISCVoverloadPPPzFzyorzyXzyH}{\saildocoverload{}{\lstinputlisting[language=sail]{sail_latex_riscv/overloadPPPzf_or_x_h3ebc47a16976857a8cd1e19cfb51cb3e.tex}}}}

\newcommand{\sailRISCVoverloadQQQFOrXS}{\saildoclabelled{sailRISCVoverloadQQQzFzyorzyXzyS}{\saildocoverload{}{\lstinputlisting[language=sail]{sail_latex_riscv/overloadQQQzf_or_x_s13ae8815ba72277dad5d10001aed7f61.tex}}}}

\newcommand{\sailRISCVoverloadRRRFOrXD}{\saildoclabelled{sailRISCVoverloadRRRzFzyorzyXzyD}{\saildocoverload{}{\lstinputlisting[language=sail]{sail_latex_riscv/overloadRRRzf_or_x_d024412251a76ad9b93d0cea341dd37e8.tex}}}}

\newcommand{\sailRISCVvalfregNameAbi}{\saildoclabelled{sailRISCVzfregzynamezyabi}{\saildocval{}{\lstinputlisting[language=sail]{sail_latex_riscv/valzfreg_name_abi149ff973f4a58e634d652021d3e44de0.tex}}}}

\newcommand{\sailRISCVoverloadSSStoStr}{\saildoclabelled{sailRISCVoverloadSSSztozystr}{\saildocoverload{}{\lstinputlisting[language=sail]{sail_latex_riscv/overloadSSSzto_str8b7a6895ae35945bd4740e9f790c43ee.tex}}}}

\newcommand{\sailRISCVvalfregName}{\saildoclabelled{sailRISCVzfregzyname}{\saildocval{}{\lstinputlisting[language=sail]{sail_latex_riscv/valzfreg_name4ac4601cc383d5edd878a00fc0bde952.tex}}}}

\newcommand{\sailRISCVvalfregOrRegName}{\saildoclabelled{sailRISCVzfregzyorzyregzyname}{\saildocval{}{\lstinputlisting[language=sail]{sail_latex_riscv/valzfreg_or_reg_name41d3b34e5fca45b7e32923d7c9e2f7a4.tex}}}}

\newcommand{\sailRISCVvalinitFdextRegs}{\saildoclabelled{sailRISCVzinitzyfdextzyregs}{\saildocval{}{\lstinputlisting[language=sail]{sail_latex_riscv/valzinit_fdext_regs66a16bff324f831ac63cd022934bcc14.tex}}}}

\newcommand{\sailRISCVfninitFdextRegs}{\saildoclabelled{sailRISCVfnzinitzyfdextzyregs}{\saildocfn{}{\lstinputlisting[language=sail]{sail_latex_riscv/fnzinit_fdext_regs66a16bff324f831ac63cd022934bcc14.tex}}}}

\newcommand{\sailRISCVtypeFcsr}{\saildoclabelled{sailRISCVtypezFcsr}{\saildoctype{}{\lstinputlisting[language=sail]{sail_latex_riscv/typezfcsrb3f3a19258d9652856cfe0e745b7acf3.tex}}}}

\newcommand{\sailRISCVregisterfcsr}{\saildoclabelled{sailRISCVregisterzfcsr}{\saildocregister{}{\lstinputlisting[language=sail]{sail_latex_riscv/registerzfcsrcafe3ad613ba5c3082476ee04c975577.tex}}}}

\newcommand{\sailRISCVvalextWriteFcsr}{\saildoclabelled{sailRISCVzextzywritezyfcsr}{\saildocval{}{\lstinputlisting[language=sail]{sail_latex_riscv/valzext_write_fcsr8225c20959aaf9205d48c42ffe341a87.tex}}}}

\newcommand{\sailRISCVfnextWriteFcsr}{\saildoclabelled{sailRISCVfnzextzywritezyfcsr}{\saildocfn{}{\lstinputlisting[language=sail]{sail_latex_riscv/fnzext_write_fcsr8225c20959aaf9205d48c42ffe341a87.tex}}}}

\newcommand{\sailRISCVvalwriteFflags}{\saildoclabelled{sailRISCVzwritezyfflags}{\saildocval{}{\lstinputlisting[language=sail]{sail_latex_riscv/valzwrite_fflagsf7610eb4597d886b515ad9553608a8d7.tex}}}}

\newcommand{\sailRISCVfnwriteFflags}{\saildoclabelled{sailRISCVfnzwritezyfflags}{\saildocfn{}{\lstinputlisting[language=sail]{sail_latex_riscv/fnzwrite_fflagsf7610eb4597d886b515ad9553608a8d7.tex}}}}

\newcommand{\sailRISCVvalaccrueFflags}{\saildoclabelled{sailRISCVzaccruezyfflags}{\saildocval{}{\lstinputlisting[language=sail]{sail_latex_riscv/valzaccrue_fflags6f6e494f89bde56691f40e88b063e194.tex}}}}

\newcommand{\sailRISCVfnaccrueFflags}{\saildoclabelled{sailRISCVfnzaccruezyfflags}{\saildocfn{}{\lstinputlisting[language=sail]{sail_latex_riscv/fnzaccrue_fflags6f6e494f89bde56691f40e88b063e194.tex}}}}

\newcommand{\sailRISCVfncsrName}{\saildoclabelled{sailRISCVfnzcsrzyname}{\saildocfn{}{\lstinputlisting[language=sail]{sail_latex_riscv/fnzcsr_name355619c0d72f0a56dfaf2d45f4b72967.tex}}}}

\newcommand{\sailRISCVfclhaveUsrModeextIsCSRDefined}{\saildoclabelled{sailRISCVfclhaveUsrModezextzyiszyCSRzydefined}{\saildocfcl{}{\lstinputlisting[language=sail]{sail_latex_riscv/fclhaveUsrModezext_is_csr_defined3e2540173eaa97b3902070bdfa6d0f6f.tex}}}}

\newcommand{\sailRISCVfclandBoolextIsCSRDefined}{\saildoclabelled{sailRISCVfclandBoolzextzyiszyCSRzydefined}{\saildocfcl{}{\lstinputlisting[language=sail]{sail_latex_riscv/fclandBoolzext_is_csr_defined3e2540173eaa97b3902070bdfa6d0f6f.tex}}}}

\newcommand{\sailRISCVfclorBoolextIsCSRDefined}{\saildoclabelled{sailRISCVfclorBoolzextzyiszyCSRzydefined}{\saildocfcl{}{\lstinputlisting[language=sail]{sail_latex_riscv/fclorBoolzext_is_csr_defined3e2540173eaa97b3902070bdfa6d0f6f.tex}}}}

\newcommand{\sailRISCVfclandBoolAextIsCSRDefined}{\saildoclabelled{sailRISCVfclandBoolAzextzyiszyCSRzydefined}{\saildocfcl{}{\lstinputlisting[language=sail]{sail_latex_riscv/fclandBoolAzext_is_csr_defined3e2540173eaa97b3902070bdfa6d0f6f.tex}}}}

\newcommand{\sailRISCVfclandBoolBextIsCSRDefined}{\saildoclabelled{sailRISCVfclandBoolBzextzyiszyCSRzydefined}{\saildocfcl{}{\lstinputlisting[language=sail]{sail_latex_riscv/fclandBoolBzext_is_csr_defined3e2540173eaa97b3902070bdfa6d0f6f.tex}}}}

\newcommand{\sailRISCVfclandBoolCextIsCSRDefined}{\saildoclabelled{sailRISCVfclandBoolCzextzyiszyCSRzydefined}{\saildocfcl{}{\lstinputlisting[language=sail]{sail_latex_riscv/fclandBoolCzext_is_csr_defined3e2540173eaa97b3902070bdfa6d0f6f.tex}}}}

\newcommand{\sailRISCVfclandBoolDextIsCSRDefined}{\saildoclabelled{sailRISCVfclandBoolDzextzyiszyCSRzydefined}{\saildocfcl{}{\lstinputlisting[language=sail]{sail_latex_riscv/fclandBoolDzext_is_csr_defined3e2540173eaa97b3902070bdfa6d0f6f.tex}}}}

\newcommand{\sailRISCVfclandBoolEextIsCSRDefined}{\saildoclabelled{sailRISCVfclandBoolEzextzyiszyCSRzydefined}{\saildocfcl{}{\lstinputlisting[language=sail]{sail_latex_riscv/fclandBoolEzext_is_csr_defined3e2540173eaa97b3902070bdfa6d0f6f.tex}}}}

\newcommand{\sailRISCVfclandBoolFextIsCSRDefined}{\saildoclabelled{sailRISCVfclandBoolFzextzyiszyCSRzydefined}{\saildocfcl{}{\lstinputlisting[language=sail]{sail_latex_riscv/fclandBoolFzext_is_csr_defined3e2540173eaa97b3902070bdfa6d0f6f.tex}}}}

\newcommand{\sailRISCVfclandBoolGextIsCSRDefined}{\saildoclabelled{sailRISCVfclandBoolGzextzyiszyCSRzydefined}{\saildocfcl{}{\lstinputlisting[language=sail]{sail_latex_riscv/fclandBoolGzext_is_csr_defined3e2540173eaa97b3902070bdfa6d0f6f.tex}}}}

\newcommand{\sailRISCVfclandBoolHextIsCSRDefined}{\saildoclabelled{sailRISCVfclandBoolHzextzyiszyCSRzydefined}{\saildocfcl{}{\lstinputlisting[language=sail]{sail_latex_riscv/fclandBoolHzext_is_csr_defined3e2540173eaa97b3902070bdfa6d0f6f.tex}}}}

\newcommand{\sailRISCVfclorBoolAextIsCSRDefined}{\saildoclabelled{sailRISCVfclorBoolAzextzyiszyCSRzydefined}{\saildocfcl{}{\lstinputlisting[language=sail]{sail_latex_riscv/fclorBoolAzext_is_csr_defined3e2540173eaa97b3902070bdfa6d0f6f.tex}}}}

\newcommand{\sailRISCVfclorBoolBextIsCSRDefined}{\saildoclabelled{sailRISCVfclorBoolBzextzyiszyCSRzydefined}{\saildocfcl{}{\lstinputlisting[language=sail]{sail_latex_riscv/fclorBoolBzext_is_csr_defined3e2540173eaa97b3902070bdfa6d0f6f.tex}}}}

\newcommand{\sailRISCVfclorBoolCextIsCSRDefined}{\saildoclabelled{sailRISCVfclorBoolCzextzyiszyCSRzydefined}{\saildocfcl{}{\lstinputlisting[language=sail]{sail_latex_riscv/fclorBoolCzext_is_csr_defined3e2540173eaa97b3902070bdfa6d0f6f.tex}}}}

\newcommand{\sailRISCVfclextIsCSRDefined}{\saildoclabelled{sailRISCVfclzextzyiszyCSRzydefined}{\saildocfcl{}{\lstinputlisting[language=sail]{sail_latex_riscv/fclzext_is_csr_defined3e2540173eaa97b3902070bdfa6d0f6f.tex}}}}



\newcommand{\sailRISCVfclGetCcsrBitsuccsrextReadCSR}{\saildoclabelled{sailRISCVfclGetCcsrBitsuccsrzextzyreadzyCSR}{\saildocfcl{}{\lstinputlisting[language=sail]{sail_latex_riscv/fclGetCcsrBitsuccsrzext_read_csr8af202f75b7d6e7536c08d920bd54264.tex}}}}

\newcommand{\sailRISCVfclGetCcsrBitssccsrextReadCSR}{\saildoclabelled{sailRISCVfclGetCcsrBitssccsrzextzyreadzyCSR}{\saildocfcl{}{\lstinputlisting[language=sail]{sail_latex_riscv/fclGetCcsrBitssccsrzext_read_csr8af202f75b7d6e7536c08d920bd54264.tex}}}}

\newcommand{\sailRISCVfclGetCcsrBitsmccsrextReadCSR}{\saildoclabelled{sailRISCVfclGetCcsrBitsmccsrzextzyreadzyCSR}{\saildocfcl{}{\lstinputlisting[language=sail]{sail_latex_riscv/fclGetCcsrBitsmccsrzext_read_csr8af202f75b7d6e7536c08d920bd54264.tex}}}}

\newcommand{\sailRISCVfclGetUstatusBitslowerSstatuslowerMstatusmstatusextReadCSR}{\saildoclabelled{sailRISCVfclGetUstatusBitslowerSstatuslowerMstatusmstatuszextzyreadzyCSR}{\saildocfcl{}{\lstinputlisting[language=sail]{sail_latex_riscv/fclGetUstatusBitslowerSstatuslowerMstatusmstatuszext_read_csr8af202f75b7d6e7536c08d920bd54264.tex}}}}

\newcommand{\sailRISCVfclGetUinterruptsBitslowerSieextReadCSR}{\saildoclabelled{sailRISCVfclGetUinterruptsBitslowerSiezextzyreadzyCSR}{\saildocfcl{}{\lstinputlisting[language=sail]{sail_latex_riscv/fclGetUinterruptsBitslowerSiezext_read_csr8af202f75b7d6e7536c08d920bd54264.tex}}}}

\newcommand{\sailRISCVfclgetUtvecextReadCSR}{\saildoclabelled{sailRISCVfclgetUtveczextzyreadzyCSR}{\saildocfcl{}{\lstinputlisting[language=sail]{sail_latex_riscv/fclgetUtveczext_read_csr8af202f75b7d6e7536c08d920bd54264.tex}}}}

\newcommand{\sailRISCVfcluscratchextReadCSR}{\saildoclabelled{sailRISCVfcluscratchzextzyreadzyCSR}{\saildocfcl{}{\lstinputlisting[language=sail]{sail_latex_riscv/fcluscratchzext_read_csr8af202f75b7d6e7536c08d920bd54264.tex}}}}

\newcommand{\sailRISCVfclandVecextReadCSR}{\saildoclabelled{sailRISCVfclandVeczextzyreadzyCSR}{\saildocfcl{}{\lstinputlisting[language=sail]{sail_latex_riscv/fclandVeczext_read_csr8af202f75b7d6e7536c08d920bd54264.tex}}}}

\newcommand{\sailRISCVfclGetMcauseBitsucauseextReadCSR}{\saildoclabelled{sailRISCVfclGetMcauseBitsucausezextzyreadzyCSR}{\saildocfcl{}{\lstinputlisting[language=sail]{sail_latex_riscv/fclGetMcauseBitsucausezext_read_csr8af202f75b7d6e7536c08d920bd54264.tex}}}}

\newcommand{\sailRISCVfclutvalextReadCSR}{\saildoclabelled{sailRISCVfclutvalzextzyreadzyCSR}{\saildocfcl{}{\lstinputlisting[language=sail]{sail_latex_riscv/fclutvalzext_read_csr8af202f75b7d6e7536c08d920bd54264.tex}}}}

\newcommand{\sailRISCVfclGetUinterruptsBitslowerSipextReadCSR}{\saildoclabelled{sailRISCVfclGetUinterruptsBitslowerSipzextzyreadzyCSR}{\saildocfcl{}{\lstinputlisting[language=sail]{sail_latex_riscv/fclGetUinterruptsBitslowerSipzext_read_csr8af202f75b7d6e7536c08d920bd54264.tex}}}}

\newcommand{\sailRISCVfclEXTZextReadCSR}{\saildoclabelled{sailRISCVfclEXTZzextzyreadzyCSR}{\saildocfcl{}{\lstinputlisting[language=sail]{sail_latex_riscv/fclEXTZzext_read_csr8af202f75b7d6e7536c08d920bd54264.tex}}}}

\newcommand{\sailRISCVfclEXTZAextReadCSR}{\saildoclabelled{sailRISCVfclEXTZAzextzyreadzyCSR}{\saildocfcl{}{\lstinputlisting[language=sail]{sail_latex_riscv/fclEXTZAzext_read_csr8af202f75b7d6e7536c08d920bd54264.tex}}}}

\newcommand{\sailRISCVfclEXTZBextReadCSR}{\saildoclabelled{sailRISCVfclEXTZBzextzyreadzyCSR}{\saildocfcl{}{\lstinputlisting[language=sail]{sail_latex_riscv/fclEXTZBzext_read_csr8af202f75b7d6e7536c08d920bd54264.tex}}}}

\newcommand{\sailRISCVfclNoneextReadCSR}{\saildoclabelled{sailRISCVfclNonezextzyreadzyCSR}{\saildocfcl{}{\lstinputlisting[language=sail]{sail_latex_riscv/fclNonezext_read_csr8af202f75b7d6e7536c08d920bd54264.tex}}}}



\newcommand{\sailRISCVfclextWriteCSR}{\saildoclabelled{sailRISCVfclzextzywritezyCSR}{\saildocfcl{}{\lstinputlisting[language=sail]{sail_latex_riscv/fclzext_write_csrea3e63f4d0be7079660a260c43b112cd.tex}}}}

\newcommand{\sailRISCVfclAextWriteCSR}{\saildoclabelled{sailRISCVfclAzextzywritezyCSR}{\saildocfcl{}{\lstinputlisting[language=sail]{sail_latex_riscv/fclAzext_write_csrea3e63f4d0be7079660a260c43b112cd.tex}}}}

\newcommand{\sailRISCVfclBextWriteCSR}{\saildoclabelled{sailRISCVfclBzextzywritezyCSR}{\saildocfcl{}{\lstinputlisting[language=sail]{sail_latex_riscv/fclBzext_write_csrea3e63f4d0be7079660a260c43b112cd.tex}}}}

\newcommand{\sailRISCVfclCextWriteCSR}{\saildoclabelled{sailRISCVfclCzextzywritezyCSR}{\saildocfcl{}{\lstinputlisting[language=sail]{sail_latex_riscv/fclCzext_write_csrea3e63f4d0be7079660a260c43b112cd.tex}}}}

\newcommand{\sailRISCVfclDextWriteCSR}{\saildoclabelled{sailRISCVfclDzextzywritezyCSR}{\saildocfcl{}{\lstinputlisting[language=sail]{sail_latex_riscv/fclDzext_write_csrea3e63f4d0be7079660a260c43b112cd.tex}}}}

\newcommand{\sailRISCVfclEextWriteCSR}{\saildoclabelled{sailRISCVfclEzextzywritezyCSR}{\saildocfcl{}{\lstinputlisting[language=sail]{sail_latex_riscv/fclEzext_write_csrea3e63f4d0be7079660a260c43b112cd.tex}}}}

\newcommand{\sailRISCVfclFextWriteCSR}{\saildoclabelled{sailRISCVfclFzextzywritezyCSR}{\saildocfcl{}{\lstinputlisting[language=sail]{sail_latex_riscv/fclFzext_write_csrea3e63f4d0be7079660a260c43b112cd.tex}}}}

\newcommand{\sailRISCVfclGextWriteCSR}{\saildoclabelled{sailRISCVfclGzextzywritezyCSR}{\saildocfcl{}{\lstinputlisting[language=sail]{sail_latex_riscv/fclGzext_write_csrea3e63f4d0be7079660a260c43b112cd.tex}}}}

\newcommand{\sailRISCVfclHextWriteCSR}{\saildoclabelled{sailRISCVfclHzextzywritezyCSR}{\saildocfcl{}{\lstinputlisting[language=sail]{sail_latex_riscv/fclHzext_write_csrea3e63f4d0be7079660a260c43b112cd.tex}}}}

\newcommand{\sailRISCVfclIextWriteCSR}{\saildoclabelled{sailRISCVfclIzextzywritezyCSR}{\saildocfcl{}{\lstinputlisting[language=sail]{sail_latex_riscv/fclIzext_write_csrea3e63f4d0be7079660a260c43b112cd.tex}}}}

\newcommand{\sailRISCVfclJextWriteCSR}{\saildoclabelled{sailRISCVfclJzextzywritezyCSR}{\saildocfcl{}{\lstinputlisting[language=sail]{sail_latex_riscv/fclJzext_write_csrea3e63f4d0be7079660a260c43b112cd.tex}}}}

\newcommand{\sailRISCVfclKextWriteCSR}{\saildoclabelled{sailRISCVfclKzextzywritezyCSR}{\saildocfcl{}{\lstinputlisting[language=sail]{sail_latex_riscv/fclKzext_write_csrea3e63f4d0be7079660a260c43b112cd.tex}}}}

\newcommand{\sailRISCVfclLextWriteCSR}{\saildoclabelled{sailRISCVfclLzextzywritezyCSR}{\saildocfcl{}{\lstinputlisting[language=sail]{sail_latex_riscv/fclLzext_write_csrea3e63f4d0be7079660a260c43b112cd.tex}}}}

\newcommand{\sailRISCVfclMextWriteCSR}{\saildoclabelled{sailRISCVfclMzextzywritezyCSR}{\saildocfcl{}{\lstinputlisting[language=sail]{sail_latex_riscv/fclMzext_write_csrea3e63f4d0be7079660a260c43b112cd.tex}}}}

\newcommand{\sailRISCVfclNoneextWriteCSR}{\saildoclabelled{sailRISCVfclNonezextzywritezyCSR}{\saildocfcl{}{\lstinputlisting[language=sail]{sail_latex_riscv/fclNonezext_write_csrea3e63f4d0be7079660a260c43b112cd.tex}}}}



\newcommand{\sailRISCVvalcsrAccess}{\saildoclabelled{sailRISCVzcsrAccess}{\saildocval{}{\lstinputlisting[language=sail]{sail_latex_riscv/valzcsraccess68432d7ad570023367f89beb42b653aa.tex}}}}

\newcommand{\sailRISCVfncsrAccess}{\saildoclabelled{sailRISCVfnzcsrAccess}{\saildocfn{}{\lstinputlisting[language=sail]{sail_latex_riscv/fnzcsraccess68432d7ad570023367f89beb42b653aa.tex}}}}

\newcommand{\sailRISCVvalcsrPriv}{\saildoclabelled{sailRISCVzcsrPriv}{\saildocval{}{\lstinputlisting[language=sail]{sail_latex_riscv/valzcsrprivc196a9e4f8a034a73a295c0ac67907ed.tex}}}}

\newcommand{\sailRISCVfncsrPriv}{\saildoclabelled{sailRISCVfnzcsrPriv}{\saildocfn{}{\lstinputlisting[language=sail]{sail_latex_riscv/fnzcsrprivc196a9e4f8a034a73a295c0ac67907ed.tex}}}}

\newcommand{\sailRISCVvalisCSRDefined}{\saildoclabelled{sailRISCVziszyCSRzydefined}{\saildocval{}{\lstinputlisting[language=sail]{sail_latex_riscv/valzis_csr_definedcd68bcdc8a87dceb6e3521cc036a67d0.tex}}}}

\newcommand{\sailRISCVfnisCSRDefined}{\saildoclabelled{sailRISCVfnziszyCSRzydefined}{\saildocfn{}{\lstinputlisting[language=sail]{sail_latex_riscv/fnzis_csr_definedcd68bcdc8a87dceb6e3521cc036a67d0.tex}}}}

\newcommand{\sailRISCVvalcheckCSRAccess}{\saildoclabelled{sailRISCVzcheckzyCSRzyaccess}{\saildocval{}{\lstinputlisting[language=sail]{sail_latex_riscv/valzcheck_csr_access77b52cf2ed56bd5489c2fcc29e6efff7.tex}}}}

\newcommand{\sailRISCVfncheckCSRAccess}{\saildoclabelled{sailRISCVfnzcheckzyCSRzyaccess}{\saildocfn{}{\lstinputlisting[language=sail]{sail_latex_riscv/fnzcheck_csr_access77b52cf2ed56bd5489c2fcc29e6efff7.tex}}}}

\newcommand{\sailRISCVvalcheckTVMSATP}{\saildoclabelled{sailRISCVzcheckzyTVMzySATP}{\saildocval{}{\lstinputlisting[language=sail]{sail_latex_riscv/valzcheck_tvm_satp8e66c1c4a67c389d24c42619a3634b21.tex}}}}

\newcommand{\sailRISCVfncheckTVMSATP}{\saildoclabelled{sailRISCVfnzcheckzyTVMzySATP}{\saildocfn{}{\lstinputlisting[language=sail]{sail_latex_riscv/fnzcheck_tvm_satp8e66c1c4a67c389d24c42619a3634b21.tex}}}}

\newcommand{\sailRISCVvalcheckCounteren}{\saildoclabelled{sailRISCVzcheckzyCounteren}{\saildocval{}{\lstinputlisting[language=sail]{sail_latex_riscv/valzcheck_counteren24e3081662742c5feea4ad876dc0a51c.tex}}}}

\newcommand{\sailRISCVfncheckCounteren}{\saildoclabelled{sailRISCVfnzcheckzyCounteren}{\saildocfn{}{\lstinputlisting[language=sail]{sail_latex_riscv/fnzcheck_counteren24e3081662742c5feea4ad876dc0a51c.tex}}}}

\newcommand{\sailRISCVvalcheckSeedCSR}{\saildoclabelled{sailRISCVzcheckzyseedzyCSR}{\saildocval{}{\lstinputlisting[language=sail]{sail_latex_riscv/valzcheck_seed_csrf99497defe8015c0237b2b8cfc908f96.tex}}}}

\newcommand{\sailRISCVfncheckSeedCSR}{\saildoclabelled{sailRISCVfnzcheckzyseedzyCSR}{\saildocfn{}{\lstinputlisting[language=sail]{sail_latex_riscv/fnzcheck_seed_csrf99497defe8015c0237b2b8cfc908f96.tex}}}}

\newcommand{\sailRISCVvalcheckCSR}{\saildoclabelled{sailRISCVzcheckzyCSR}{\saildocval{}{\lstinputlisting[language=sail]{sail_latex_riscv/valzcheck_csr588ba7a7f36d1f6476f0b3640406ee0a.tex}}}}

\newcommand{\sailRISCVfncheckCSR}{\saildoclabelled{sailRISCVfnzcheckzyCSR}{\saildocfn{}{\lstinputlisting[language=sail]{sail_latex_riscv/fnzcheck_csr588ba7a7f36d1f6476f0b3640406ee0a.tex}}}}

\newcommand{\sailRISCVvalspeculateConditional}{\saildoclabelled{sailRISCVzspeculatezyconditional}{\saildocval{}{\lstinputlisting[language=sail]{sail_latex_riscv/valzspeculate_conditionalc9cfcf9f8fbedbbbb1846b896e477d5c.tex}}}}

\newcommand{\sailRISCVvalloadReservation}{\saildoclabelled{sailRISCVzloadzyreservation}{\saildocval{}{\lstinputlisting[language=sail]{sail_latex_riscv/valzload_reservationca254f1d85cbf90f1662b9f77f637106.tex}}}}

\newcommand{\sailRISCVvalmatchReservation}{\saildoclabelled{sailRISCVzmatchzyreservation}{\saildocval{}{\lstinputlisting[language=sail]{sail_latex_riscv/valzmatch_reservationedafe654d6b1fee1b351613706caf96e.tex}}}}

\newcommand{\sailRISCVvalcancelReservation}{\saildoclabelled{sailRISCVzcancelzyreservation}{\saildocval{}{\lstinputlisting[language=sail]{sail_latex_riscv/valzcancel_reservationf12c9af3ed835ad5be485ac1fad8d56f.tex}}}}

\newcommand{\sailRISCVvalexceptionDelegatee}{\saildoclabelled{sailRISCVzexceptionzydelegatee}{\saildocval{}{\lstinputlisting[language=sail]{sail_latex_riscv/valzexception_delegateefb7e9252abe9f7e50e2f06577208c695.tex}}}}

\newcommand{\sailRISCVfnexceptionDelegatee}{\saildoclabelled{sailRISCVfnzexceptionzydelegatee}{\saildocfn{}{\lstinputlisting[language=sail]{sail_latex_riscv/fnzexception_delegateefb7e9252abe9f7e50e2f06577208c695.tex}}}}

\newcommand{\sailRISCVvalfindPendingInterrupt}{\saildoclabelled{sailRISCVzfindPendingInterrupt}{\saildocval{}{\lstinputlisting[language=sail]{sail_latex_riscv/valzfindpendinginterrupt0ed4e5cc6469ca27b20724b51027ab4c.tex}}}}

\newcommand{\sailRISCVfnfindPendingInterrupt}{\saildoclabelled{sailRISCVfnzfindPendingInterrupt}{\saildocfn{}{\lstinputlisting[language=sail]{sail_latex_riscv/fnzfindpendinginterrupt0ed4e5cc6469ca27b20724b51027ab4c.tex}}}}

\newcommand{\sailRISCVtypeinterruptSet}{\saildoclabelled{sailRISCVtypezinterruptzyset}{\saildoctype{}{\lstinputlisting[language=sail]{sail_latex_riscv/typezinterrupt_set44973ef906c7020fc165399cdb4dc72c.tex}}}}

\newcommand{\sailRISCVvalprocessPending}{\saildoclabelled{sailRISCVzprocessPending}{\saildocval{}{\lstinputlisting[language=sail]{sail_latex_riscv/valzprocesspendingf31f33d163f06483630c9da88eafecaa.tex}}}}

\newcommand{\sailRISCVfnprocessPending}{\saildoclabelled{sailRISCVfnzprocessPending}{\saildocfn{}{\lstinputlisting[language=sail]{sail_latex_riscv/fnzprocesspendingf31f33d163f06483630c9da88eafecaa.tex}}}}

\newcommand{\sailRISCVvalgetPendingSet}{\saildoclabelled{sailRISCVzgetPendingSet}{\saildocval{}{\lstinputlisting[language=sail]{sail_latex_riscv/valzgetpendingsetfe7aa2453fb185b904f0c1c2ec773ed8.tex}}}}

\newcommand{\sailRISCVfngetPendingSet}{\saildoclabelled{sailRISCVfnzgetPendingSet}{\saildocfn{}{\lstinputlisting[language=sail]{sail_latex_riscv/fnzgetpendingsetfe7aa2453fb185b904f0c1c2ec773ed8.tex}}}}

\newcommand{\sailRISCVvaldispatchInterrupt}{\saildoclabelled{sailRISCVzdispatchInterrupt}{\saildocval{}{\lstinputlisting[language=sail]{sail_latex_riscv/valzdispatchinterruptaf83562ba97e1882252696fd8999b3e7.tex}}}}

\newcommand{\sailRISCVfndispatchInterrupt}{\saildoclabelled{sailRISCVfnzdispatchInterrupt}{\saildocfn{}{\lstinputlisting[language=sail]{sail_latex_riscv/fnzdispatchinterruptaf83562ba97e1882252696fd8999b3e7.tex}}}}

\newcommand{\sailRISCVtypectlResult}{\saildoclabelled{sailRISCVtypezctlzyresult}{\saildoctype{}{\lstinputlisting[language=sail]{sail_latex_riscv/typezctl_result052149e0767cf8ed055fce7a562f2c3c.tex}}}}

\newcommand{\sailRISCVvaltval}{\saildoclabelled{sailRISCVztval}{\saildocval{}{\lstinputlisting[language=sail]{sail_latex_riscv/valztval0ee7d37a987a82754891fc591aec5852.tex}}}}

\newcommand{\sailRISCVfntval}{\saildoclabelled{sailRISCVfnztval}{\saildocfn{}{\lstinputlisting[language=sail]{sail_latex_riscv/fnztval0ee7d37a987a82754891fc591aec5852.tex}}}}

\newcommand{\sailRISCVvalrvfiTrap}{\saildoclabelled{sailRISCVzrvfizytrap}{\saildocval{}{\lstinputlisting[language=sail]{sail_latex_riscv/valzrvfi_trap0d0ddd87deff120444aa8feac182c6ce.tex}}}}

\newcommand{\sailRISCVfnrvfiTrap}{\saildoclabelled{sailRISCVfnzrvfizytrap}{\saildocfn{}{\lstinputlisting[language=sail]{sail_latex_riscv/fnzrvfi_trap0d0ddd87deff120444aa8feac182c6ce.tex}}}}

\newcommand{\sailRISCVvaltrapHandler}{\saildoclabelled{sailRISCVztrapzyhandler}{\saildocval{}{\lstinputlisting[language=sail]{sail_latex_riscv/valztrap_handler0acf2ac9f6239ac64448b463d4df9cf4.tex}}}}

\newcommand{\sailRISCVfntrapHandler}{\saildoclabelled{sailRISCVfnztrapzyhandler}{\saildocfn{}{\lstinputlisting[language=sail]{sail_latex_riscv/fnztrap_handler0acf2ac9f6239ac64448b463d4df9cf4.tex}}}}

\newcommand{\sailRISCVvalexceptionHandler}{\saildoclabelled{sailRISCVzexceptionzyhandler}{\saildocval{}{\lstinputlisting[language=sail]{sail_latex_riscv/valzexception_handlerf03729146a8718cee62ce35044e16202.tex}}}}

\newcommand{\sailRISCVfnexceptionHandler}{\saildoclabelled{sailRISCVfnzexceptionzyhandler}{\saildocfn{}{\lstinputlisting[language=sail]{sail_latex_riscv/fnzexception_handlerf03729146a8718cee62ce35044e16202.tex}}}}

\newcommand{\sailRISCVvalhandleMemException}{\saildoclabelled{sailRISCVzhandlezymemzyexception}{\saildocval{}{\lstinputlisting[language=sail]{sail_latex_riscv/valzhandle_mem_exceptionec2a0168c4f7affdb0652fb9992ac72e.tex}}}}

\newcommand{\sailRISCVfnhandleMemException}{\saildoclabelled{sailRISCVfnzhandlezymemzyexception}{\saildocfn{}{\lstinputlisting[language=sail]{sail_latex_riscv/fnzhandle_mem_exceptionec2a0168c4f7affdb0652fb9992ac72e.tex}}}}

\newcommand{\sailRISCVvalhandleException}{\saildoclabelled{sailRISCVzhandlezyexception}{\saildocval{}{\lstinputlisting[language=sail]{sail_latex_riscv/valzhandle_exception5b7a94266182a8a2ef0291c83b256387.tex}}}}

\newcommand{\sailRISCVfnhandleException}{\saildoclabelled{sailRISCVfnzhandlezyexception}{\saildocfn{}{\lstinputlisting[language=sail]{sail_latex_riscv/fnzhandle_exception5b7a94266182a8a2ef0291c83b256387.tex}}}}

\newcommand{\sailRISCVvalhandleInterrupt}{\saildoclabelled{sailRISCVzhandlezyinterrupt}{\saildocval{}{\lstinputlisting[language=sail]{sail_latex_riscv/valzhandle_interrupt9048a02caa891e49449f7a1f3f4e9ee4.tex}}}}

\newcommand{\sailRISCVfnhandleInterrupt}{\saildoclabelled{sailRISCVfnzhandlezyinterrupt}{\saildocfn{}{\lstinputlisting[language=sail]{sail_latex_riscv/fnzhandle_interrupt9048a02caa891e49449f7a1f3f4e9ee4.tex}}}}

\newcommand{\sailRISCVvalinitSys}{\saildoclabelled{sailRISCVzinitzysys}{\saildocval{}{\lstinputlisting[language=sail]{sail_latex_riscv/valzinit_sysc92898d1b2b72595dd36bce10e6a67fb.tex}}}}

\newcommand{\sailRISCVfninitSys}{\saildoclabelled{sailRISCVfnzinitzysys}{\saildocfn{}{\lstinputlisting[language=sail]{sail_latex_riscv/fnzinit_sysc92898d1b2b72595dd36bce10e6a67fb.tex}}}}

\newcommand{\sailRISCVtypeMemoryOpResult}{\saildoclabelled{sailRISCVtypezMemoryOpResult}{\saildoctype{}{\lstinputlisting[language=sail]{sail_latex_riscv/typezmemoryopresult3416532231724086920f553567ecf72c.tex}}}}

\newcommand{\sailRISCVvalMemoryOpResultAddMeta}{\saildoclabelled{sailRISCVzMemoryOpResultzyaddzymeta}{\saildocval{}{\lstinputlisting[language=sail]{sail_latex_riscv/valzmemoryopresult_add_meta6f4337fb08e42f593b3375a2d1083593.tex}}}}

\newcommand{\sailRISCVfnMemoryOpResultAddMeta}{\saildoclabelled{sailRISCVfnzMemoryOpResultzyaddzymeta}{\saildocfn{}{\lstinputlisting[language=sail]{sail_latex_riscv/fnzmemoryopresult_add_meta6f4337fb08e42f593b3375a2d1083593.tex}}}}

\newcommand{\sailRISCVvalMemoryOpResultDropMeta}{\saildoclabelled{sailRISCVzMemoryOpResultzydropzymeta}{\saildocval{}{\lstinputlisting[language=sail]{sail_latex_riscv/valzmemoryopresult_drop_metafa9388b5fdd8fbe880aa4e686c59bab1.tex}}}}

\newcommand{\sailRISCVfnMemoryOpResultDropMeta}{\saildoclabelled{sailRISCVfnzMemoryOpResultzydropzymeta}{\saildocfn{}{\lstinputlisting[language=sail]{sail_latex_riscv/fnzmemoryopresult_drop_metafa9388b5fdd8fbe880aa4e686c59bab1.tex}}}}

\newcommand{\sailRISCVtypeExtFetchAddrCheck}{\saildoclabelled{sailRISCVtypezExtzyFetchAddrzyCheck}{\saildoctype{}{\lstinputlisting[language=sail]{sail_latex_riscv/typezext_fetchaddr_checkea47f0986744ae6c29753d54694485fe.tex}}}}

\newcommand{\sailRISCVtypeExtControlAddrCheck}{\saildoclabelled{sailRISCVtypezExtzyControlAddrzyCheck}{\saildoctype{}{\lstinputlisting[language=sail]{sail_latex_riscv/typezext_controladdr_checka9d6f41abcba03f32cc75e65e808a059.tex}}}}

\newcommand{\sailRISCVtypeExtDataAddrCheck}{\saildoclabelled{sailRISCVtypezExtzyDataAddrzyCheck}{\saildoctype{}{\lstinputlisting[language=sail]{sail_latex_riscv/typezext_dataaddr_checkae2f55c86bbc0d35297825ba44134f12.tex}}}}

\newcommand{\sailRISCVtypeExtPhysAddrCheck}{\saildoclabelled{sailRISCVtypezExtzyPhysAddrzyCheck}{\saildoctype{}{\lstinputlisting[language=sail]{sail_latex_riscv/typezext_physaddr_checkc3760f9b4fda909a0f48ead90fcc3bf3.tex}}}}

\newcommand{\sailRISCVvalextCheckPhysMemRead}{\saildoclabelled{sailRISCVzextzycheckzyphyszymemzyread}{\saildocval{Validate a read from physical memory.
\lstinline{ext_check_phys_mem_read}(access\_type, paddr, size, aquire, release, reserved, read\_meta) should
return Some(exception) to abort the read or None to allow it to proceed. The
check is performed after PMP checks and does not apply to MMIO memory.

}{\lstinputlisting[language=sail]{sail_latex_riscv/valzext_check_phys_mem_readf999b4bc39f7274e42391ca653d8762d.tex}}}}

\newcommand{\sailRISCVvalextCheckPhysMemWrite}{\saildoclabelled{sailRISCVzextzycheckzyphyszymemzywrite}{\saildocval{Validate a write to physical memory.
\lstinline{ext_check_phys_mem_write}(write\_kind, paddr, size, data, metadata) should return Some(exception)
to abort the write or None to allow it to proceed. The check is performed 
after PMP checks and does not apply to MMIO memory.

}{\lstinputlisting[language=sail]{sail_latex_riscv/valzext_check_phys_mem_writeeec4b73c21b79225d7ccf5fb8b67ba61.tex}}}}

\newcommand{\sailRISCVvalhandleCheriCapException}{\saildoclabelled{sailRISCVzhandlezycherizycapzyexception}{\saildocval{}{\lstinputlisting[language=sail]{sail_latex_riscv/valzhandle_cheri_cap_exceptionc1ff083ca6d0a739fb48243e22ff4898.tex}}}}

\newcommand{\sailRISCVfnhandleCheriCapException}{\saildoclabelled{sailRISCVfnzhandlezycherizycapzyexception}{\saildocfn{}{\lstinputlisting[language=sail]{sail_latex_riscv/fnzhandle_cheri_cap_exceptionc1ff083ca6d0a739fb48243e22ff4898.tex}}}}

\newcommand{\sailRISCVvalhandleCheriRegException}{\saildoclabelled{sailRISCVzhandlezycherizyregzyexception}{\saildocval{Causes the processor to raise a capability exception by writing the given
capability exception cause and register number to the xtval register then
signalling an exception.

}{\lstinputlisting[language=sail]{sail_latex_riscv/valzhandle_cheri_reg_exceptionfad1b48ae08f4eb90d02a5d75771c894.tex}}}}

\newcommand{\sailRISCVfnhandleCheriRegException}{\saildoclabelled{sailRISCVfnzhandlezycherizyregzyexception}{\saildocfn{}{\lstinputlisting[language=sail]{sail_latex_riscv/fnzhandle_cheri_reg_exceptionfad1b48ae08f4eb90d02a5d75771c894.tex}}}}

\newcommand{\sailRISCVvalhandleCheriPccException}{\saildoclabelled{sailRISCVzhandlezycherizypcczyexception}{\saildocval{Is as \hyperref[sailRISCVzhandlezycherizycapzyexception]{\lstinline{handle_cheri_cap_exception}} except that the capability register
number uses the special value 0x10 indicating the PCC register.

}{\lstinputlisting[language=sail]{sail_latex_riscv/valzhandle_cheri_pcc_exception3ca0178a61c5394ac2c49197cedda1c1.tex}}}}

\newcommand{\sailRISCVfnhandleCheriPccException}{\saildoclabelled{sailRISCVfnzhandlezycherizypcczyexception}{\saildocfn{}{\lstinputlisting[language=sail]{sail_latex_riscv/fnzhandle_cheri_pcc_exception3ca0178a61c5394ac2c49197cedda1c1.tex}}}}

\newcommand{\sailRISCVvalpccAccessSystemRegs}{\saildoclabelled{sailRISCVzpcczyaccesszysystemzyregs}{\saildocval{}{\lstinputlisting[language=sail]{sail_latex_riscv/valzpcc_access_system_regsc75c9194580770304d0d456839785b75.tex}}}}

\newcommand{\sailRISCVfnpccAccessSystemRegs}{\saildoclabelled{sailRISCVfnzpcczyaccesszysystemzyregs}{\saildocfn{}{\lstinputlisting[language=sail]{sail_latex_riscv/fnzpcc_access_system_regsc75c9194580770304d0d456839785b75.tex}}}}

\newcommand{\sailRISCVtypeextFetchAddrError}{\saildoclabelled{sailRISCVtypezextzyfetchzyaddrzyerror}{\saildoctype{}{\lstinputlisting[language=sail]{sail_latex_riscv/typezext_fetch_addr_error77c89145df888a86f60f8c4b402dc719.tex}}}}

\newcommand{\sailRISCVvalextFetchCheckPc}{\saildoclabelled{sailRISCVzextzyfetchzycheckzypc}{\saildocval{}{\lstinputlisting[language=sail]{sail_latex_riscv/valzext_fetch_check_pc2e82f09c4f4da5465b70e5f9e6f48b77.tex}}}}

\newcommand{\sailRISCVfnextFetchCheckPc}{\saildoclabelled{sailRISCVfnzextzyfetchzycheckzypc}{\saildocfn{}{\lstinputlisting[language=sail]{sail_latex_riscv/fnzext_fetch_check_pc2e82f09c4f4da5465b70e5f9e6f48b77.tex}}}}

\newcommand{\sailRISCVvalextHandleFetchCheckError}{\saildoclabelled{sailRISCVzextzyhandlezyfetchzycheckzyerror}{\saildocval{}{\lstinputlisting[language=sail]{sail_latex_riscv/valzext_handle_fetch_check_error1c773c1438a4c5dcc256df216f0a1aa9.tex}}}}

\newcommand{\sailRISCVfnextHandleFetchCheckError}{\saildoclabelled{sailRISCVfnzextzyhandlezyfetchzycheckzyerror}{\saildocfn{}{\lstinputlisting[language=sail]{sail_latex_riscv/fnzext_handle_fetch_check_error1c773c1438a4c5dcc256df216f0a1aa9.tex}}}}

\newcommand{\sailRISCVtypeextControlAddrError}{\saildoclabelled{sailRISCVtypezextzycontrolzyaddrzyerror}{\saildoctype{}{\lstinputlisting[language=sail]{sail_latex_riscv/typezext_control_addr_error30fe954fdccd9382175241e7b6137d82.tex}}}}

\newcommand{\sailRISCVvalextControlCheckAddr}{\saildoclabelled{sailRISCVzextzycontrolzycheckzyaddr}{\saildocval{}{\lstinputlisting[language=sail]{sail_latex_riscv/valzext_control_check_addr2d404fc3390578d569e3f547f0d18fce.tex}}}}

\newcommand{\sailRISCVfnextControlCheckAddr}{\saildoclabelled{sailRISCVfnzextzycontrolzycheckzyaddr}{\saildocfn{}{\lstinputlisting[language=sail]{sail_latex_riscv/fnzext_control_check_addr2d404fc3390578d569e3f547f0d18fce.tex}}}}

\newcommand{\sailRISCVvalextControlCheckPc}{\saildoclabelled{sailRISCVzextzycontrolzycheckzypc}{\saildocval{}{\lstinputlisting[language=sail]{sail_latex_riscv/valzext_control_check_pc92c2579f955b827738ac1e5c79b85839.tex}}}}

\newcommand{\sailRISCVfnextControlCheckPc}{\saildoclabelled{sailRISCVfnzextzycontrolzycheckzypc}{\saildocfn{}{\lstinputlisting[language=sail]{sail_latex_riscv/fnzext_control_check_pc92c2579f955b827738ac1e5c79b85839.tex}}}}

\newcommand{\sailRISCVvalextHandleControlCheckError}{\saildoclabelled{sailRISCVzextzyhandlezycontrolzycheckzyerror}{\saildocval{}{\lstinputlisting[language=sail]{sail_latex_riscv/valzext_handle_control_check_error7b80ca54e4133f98238aa5f1371bfc1f.tex}}}}

\newcommand{\sailRISCVfnextHandleControlCheckError}{\saildoclabelled{sailRISCVfnzextzyhandlezycontrolzycheckzyerror}{\saildocfn{}{\lstinputlisting[language=sail]{sail_latex_riscv/fnzext_handle_control_check_error7b80ca54e4133f98238aa5f1371bfc1f.tex}}}}

\newcommand{\sailRISCVtypeextDataAddrError}{\saildoclabelled{sailRISCVtypezextzydatazyaddrzyerror}{\saildoctype{}{\lstinputlisting[language=sail]{sail_latex_riscv/typezext_data_addr_errora4f74a5b44e1f0d7e46bc0f33c466dea.tex}}}}

\newcommand{\sailRISCVvalgetCheriModeCapAddr}{\saildoclabelled{sailRISCVzgetzycherizymodezycapzyaddr}{\saildocval{For given base register and offset returns, depending on current capability
mode flag, a bounding capability, effective address, and capreg\_idx (for use
in cap cause).

}{\lstinputlisting[language=sail]{sail_latex_riscv/valzget_cheri_mode_cap_addr267a231c94a9ae3cf08d67cb43590a2e.tex}}}}

\newcommand{\sailRISCVfngetCheriModeCapAddr}{\saildoclabelled{sailRISCVfnzgetzycherizymodezycapzyaddr}{\saildocfn{}{\lstinputlisting[language=sail]{sail_latex_riscv/fnzget_cheri_mode_cap_addr267a231c94a9ae3cf08d67cb43590a2e.tex}}}}

\newcommand{\sailRISCVvalextDataGetAddr}{\saildoclabelled{sailRISCVzextzydatazygetzyaddr}{\saildocval{}{\lstinputlisting[language=sail]{sail_latex_riscv/valzext_data_get_addra719d6978c6003ef3b6e2e57ccbf64b8.tex}}}}

\newcommand{\sailRISCVfnextDataGetAddr}{\saildoclabelled{sailRISCVfnzextzydatazygetzyaddr}{\saildocfn{}{\lstinputlisting[language=sail]{sail_latex_riscv/fnzext_data_get_addra719d6978c6003ef3b6e2e57ccbf64b8.tex}}}}

\newcommand{\sailRISCVvalextHandleDataCheckError}{\saildoclabelled{sailRISCVzextzyhandlezydatazycheckzyerror}{\saildocval{}{\lstinputlisting[language=sail]{sail_latex_riscv/valzext_handle_data_check_errorad2507fc7050fbd24451608767d75b73.tex}}}}

\newcommand{\sailRISCVfnextHandleDataCheckError}{\saildoclabelled{sailRISCVfnzextzyhandlezydatazycheckzyerror}{\saildocfn{}{\lstinputlisting[language=sail]{sail_latex_riscv/fnzext_handle_data_check_errorad2507fc7050fbd24451608767d75b73.tex}}}}

\newcommand{\sailRISCVfnextCheckPhysMemRead}{\saildoclabelled{sailRISCVfnzextzycheckzyphyszymemzyread}{\saildocfn{}{\lstinputlisting[language=sail]{sail_latex_riscv/fnzext_check_phys_mem_readf999b4bc39f7274e42391ca653d8762d.tex}}}}

\newcommand{\sailRISCVfnextCheckPhysMemWrite}{\saildoclabelled{sailRISCVfnzextzycheckzyphyszymemzywrite}{\saildocfn{}{\lstinputlisting[language=sail]{sail_latex_riscv/fnzext_check_phys_mem_writeeec4b73c21b79225d7ccf5fb8b67ba61.tex}}}}

\newcommand{\sailRISCVvalextCheckXretPriv}{\saildoclabelled{sailRISCVzextzycheckzyxretzypriv}{\saildocval{}{\lstinputlisting[language=sail]{sail_latex_riscv/valzext_check_xret_priv92677d070503a361c4ac308adba53957.tex}}}}

\newcommand{\sailRISCVfnextCheckXretPriv}{\saildoclabelled{sailRISCVfnzextzycheckzyxretzypriv}{\saildocfn{}{\lstinputlisting[language=sail]{sail_latex_riscv/fnzext_check_xret_priv92677d070503a361c4ac308adba53957.tex}}}}

\newcommand{\sailRISCVvalextFailXretPriv}{\saildoclabelled{sailRISCVzextzyfailzyxretzypriv}{\saildocval{}{\lstinputlisting[language=sail]{sail_latex_riscv/valzext_fail_xret_priva071b88d32f48ad8e720a4cae916da8b.tex}}}}

\newcommand{\sailRISCVfnextFailXretPriv}{\saildoclabelled{sailRISCVfnzextzyfailzyxretzypriv}{\saildocfn{}{\lstinputlisting[language=sail]{sail_latex_riscv/fnzext_fail_xret_priva071b88d32f48ad8e720a4cae916da8b.tex}}}}

\newcommand{\sailRISCVvalextCheckCSR}{\saildoclabelled{sailRISCVzextzycheckzyCSR}{\saildocval{}{\lstinputlisting[language=sail]{sail_latex_riscv/valzext_check_csreef82c82c124fe04d74fc6fd8219bb68.tex}}}}

\newcommand{\sailRISCVfnextCheckCSR}{\saildoclabelled{sailRISCVfnzextzycheckzyCSR}{\saildocfn{}{\lstinputlisting[language=sail]{sail_latex_riscv/fnzext_check_csreef82c82c124fe04d74fc6fd8219bb68.tex}}}}

\newcommand{\sailRISCVvalextCheckCSRFail}{\saildoclabelled{sailRISCVzextzycheckzyCSRzyfail}{\saildocval{}{\lstinputlisting[language=sail]{sail_latex_riscv/valzext_check_csr_fail991cc4645a62a48eb645391d3988288a.tex}}}}

\newcommand{\sailRISCVfnextCheckCSRFail}{\saildoclabelled{sailRISCVfnzextzycheckzyCSRzyfail}{\saildocfn{}{\lstinputlisting[language=sail]{sail_latex_riscv/fnzext_check_csr_fail991cc4645a62a48eb645391d3988288a.tex}}}}

\newcommand{\sailRISCVfnextVetoDisableC}{\saildoclabelled{sailRISCVfnzextzyvetozydisablezyC}{\saildocfn{}{\lstinputlisting[language=sail]{sail_latex_riscv/fnzext_veto_disable_cd10c2d1c5077060fa007c1628d7aaa8c.tex}}}}

\newcommand{\sailRISCVvalelfTohost}{\saildoclabelled{sailRISCVzelfzytohost}{\saildocval{}{\lstinputlisting[language=sail]{sail_latex_riscv/valzelf_tohost30cafcd41bc201ebfe5bbd0510aa0b7c.tex}}}}

\newcommand{\sailRISCVvalelfEntry}{\saildoclabelled{sailRISCVzelfzyentry}{\saildocval{}{\lstinputlisting[language=sail]{sail_latex_riscv/valzelf_entry2a366da786fa7f56d47732b2fddb2821.tex}}}}

\newcommand{\sailRISCVvalplatRamBase}{\saildoclabelled{sailRISCVzplatzyramzybase}{\saildocval{}{\lstinputlisting[language=sail]{sail_latex_riscv/valzplat_ram_base0d7614688445a7408614bad359843641.tex}}}}

\newcommand{\sailRISCVvalplatRamSizze}{\saildoclabelled{sailRISCVzplatzyramzysizze}{\saildocval{}{\lstinputlisting[language=sail]{sail_latex_riscv/valzplat_ram_sizze3e389712a74fd397e22555e7f548f008.tex}}}}

\newcommand{\sailRISCVvalplatEnablePmp}{\saildoclabelled{sailRISCVzplatzyenablezypmp}{\saildocval{}{\lstinputlisting[language=sail]{sail_latex_riscv/valzplat_enable_pmpe3a19dd2b08d4a89664077b8a1cf0844.tex}}}}

\newcommand{\sailRISCVvalplatEnableDirtyUpdate}{\saildoclabelled{sailRISCVzplatzyenablezydirtyzyupdate}{\saildocval{}{\lstinputlisting[language=sail]{sail_latex_riscv/valzplat_enable_dirty_update0b255805eb9610b556493bfe506416f2.tex}}}}

\newcommand{\sailRISCVvalplatEnableMisalignedAccess}{\saildoclabelled{sailRISCVzplatzyenablezymisalignedzyaccess}{\saildocval{}{\lstinputlisting[language=sail]{sail_latex_riscv/valzplat_enable_misaligned_accesscb4191568185cb3901b1944164f65f6e.tex}}}}

\newcommand{\sailRISCVvalplatMtvalHasIllegalInstBits}{\saildoclabelled{sailRISCVzplatzymtvalzyhaszyillegalzyinstzybits}{\saildocval{}{\lstinputlisting[language=sail]{sail_latex_riscv/valzplat_mtval_has_illegal_inst_bits49cda37cfdfffb5c464c6333bb83f0b0.tex}}}}

\newcommand{\sailRISCVvalplatRomBase}{\saildoclabelled{sailRISCVzplatzyromzybase}{\saildocval{}{\lstinputlisting[language=sail]{sail_latex_riscv/valzplat_rom_basebdbb354c7f3bb238fa08300023cbf0b5.tex}}}}

\newcommand{\sailRISCVvalplatRomSizze}{\saildoclabelled{sailRISCVzplatzyromzysizze}{\saildocval{}{\lstinputlisting[language=sail]{sail_latex_riscv/valzplat_rom_sizze563dbbc4d715886f76fbde684ac8a500.tex}}}}

\newcommand{\sailRISCVvalplatClintBase}{\saildoclabelled{sailRISCVzplatzyclintzybase}{\saildocval{}{\lstinputlisting[language=sail]{sail_latex_riscv/valzplat_clint_base0196a3035036838f7fe4f78d61a1f9d8.tex}}}}

\newcommand{\sailRISCVvalplatClintSizze}{\saildoclabelled{sailRISCVzplatzyclintzysizze}{\saildocval{}{\lstinputlisting[language=sail]{sail_latex_riscv/valzplat_clint_sizze8fb2ec94310f08d19d4c7c0476f57919.tex}}}}

\newcommand{\sailRISCVvalplatHtifTohost}{\saildoclabelled{sailRISCVzplatzyhtifzytohost}{\saildocval{}{\lstinputlisting[language=sail]{sail_latex_riscv/valzplat_htif_tohostf9ba2e7ae67de81976fbefc0304b89d1.tex}}}}

\newcommand{\sailRISCVfnplatHtifTohost}{\saildoclabelled{sailRISCVfnzplatzyhtifzytohost}{\saildocfn{}{\lstinputlisting[language=sail]{sail_latex_riscv/fnzplat_htif_tohostf9ba2e7ae67de81976fbefc0304b89d1.tex}}}}

\newcommand{\sailRISCVvalphysMemSegments}{\saildoclabelled{sailRISCVzphyszymemzysegments}{\saildocval{}{\lstinputlisting[language=sail]{sail_latex_riscv/valzphys_mem_segments83ff72aa5aa998a329fa30de106aa0a7.tex}}}}

\newcommand{\sailRISCVfnphysMemSegments}{\saildoclabelled{sailRISCVfnzphyszymemzysegments}{\saildocfn{}{\lstinputlisting[language=sail]{sail_latex_riscv/fnzphys_mem_segments83ff72aa5aa998a329fa30de106aa0a7.tex}}}}

\newcommand{\sailRISCVvalwithinPhysMem}{\saildoclabelled{sailRISCVzwithinzyphyszymem}{\saildocval{}{\lstinputlisting[language=sail]{sail_latex_riscv/valzwithin_phys_mem5b6233a64f93394cb594812a945dcbb2.tex}}}}

\newcommand{\sailRISCVfnwithinPhysMem}{\saildoclabelled{sailRISCVfnzwithinzyphyszymem}{\saildocfn{}{\lstinputlisting[language=sail]{sail_latex_riscv/fnzwithin_phys_mem5b6233a64f93394cb594812a945dcbb2.tex}}}}

\newcommand{\sailRISCVvalwithinClint}{\saildoclabelled{sailRISCVzwithinzyclint}{\saildocval{}{\lstinputlisting[language=sail]{sail_latex_riscv/valzwithin_clintc139e042afc9910b6edf55c2c70f2e80.tex}}}}

\newcommand{\sailRISCVfnwithinClint}{\saildoclabelled{sailRISCVfnzwithinzyclint}{\saildocfn{}{\lstinputlisting[language=sail]{sail_latex_riscv/fnzwithin_clintc139e042afc9910b6edf55c2c70f2e80.tex}}}}

\newcommand{\sailRISCVvalwithinHtifWritable}{\saildoclabelled{sailRISCVzwithinzyhtifzywritable}{\saildocval{}{\lstinputlisting[language=sail]{sail_latex_riscv/valzwithin_htif_writablec356d0ea372a60437fdf28745e5e9ae3.tex}}}}

\newcommand{\sailRISCVfnwithinHtifWritable}{\saildoclabelled{sailRISCVfnzwithinzyhtifzywritable}{\saildocfn{}{\lstinputlisting[language=sail]{sail_latex_riscv/fnzwithin_htif_writablec356d0ea372a60437fdf28745e5e9ae3.tex}}}}

\newcommand{\sailRISCVvalwithinHtifReadable}{\saildoclabelled{sailRISCVzwithinzyhtifzyreadable}{\saildocval{}{\lstinputlisting[language=sail]{sail_latex_riscv/valzwithin_htif_readable2f6131e40985c12d9270943521113c33.tex}}}}

\newcommand{\sailRISCVfnwithinHtifReadable}{\saildoclabelled{sailRISCVfnzwithinzyhtifzyreadable}{\saildocfn{}{\lstinputlisting[language=sail]{sail_latex_riscv/fnzwithin_htif_readable2f6131e40985c12d9270943521113c33.tex}}}}

\newcommand{\sailRISCVvalplatInsnsPerTick}{\saildoclabelled{sailRISCVzplatzyinsnszyperzytick}{\saildocval{}{\lstinputlisting[language=sail]{sail_latex_riscv/valzplat_insns_per_tick0a43cf4e39e3447031fdc4fb6c22d20e.tex}}}}

\newcommand{\sailRISCVregistermtimecmp}{\saildoclabelled{sailRISCVregisterzmtimecmp}{\saildocregister{}{\lstinputlisting[language=sail]{sail_latex_riscv/registerzmtimecmp490b548fc68eb77d9ad0fb131ac5b176.tex}}}}

\newcommand{\sailRISCVletMSIPBASE}{\saildoclabelled{sailRISCVletzMSIPzyBASE}{\saildoclet{}{\lstinputlisting[language=sail]{sail_latex_riscv/letzmsip_basefc27c8e50a1b936c6f7b30a736bd93b3.tex}}}}

\newcommand{\sailRISCVletMTIMECMPBASE}{\saildoclabelled{sailRISCVletzMTIMECMPzyBASE}{\saildoclet{}{\lstinputlisting[language=sail]{sail_latex_riscv/letzmtimecmp_base6c3f1082c76878efdad5a05e5541234f.tex}}}}

\newcommand{\sailRISCVletMTIMECMPBASEHI}{\saildoclabelled{sailRISCVletzMTIMECMPzyBASEzyHI}{\saildoclet{}{\lstinputlisting[language=sail]{sail_latex_riscv/letzmtimecmp_base_hi2f6ebba64ab8b6b097dd0cd20b46485e.tex}}}}

\newcommand{\sailRISCVletMTIMEBASE}{\saildoclabelled{sailRISCVletzMTIMEzyBASE}{\saildoclet{}{\lstinputlisting[language=sail]{sail_latex_riscv/letzmtime_base9ce6a45a625f58bf27a364dcf6e442b9.tex}}}}

\newcommand{\sailRISCVletMTIMEBASEHI}{\saildoclabelled{sailRISCVletzMTIMEzyBASEzyHI}{\saildoclet{}{\lstinputlisting[language=sail]{sail_latex_riscv/letzmtime_base_hi5bcf3133a671d7190ead2f2fe4ca386b.tex}}}}

\newcommand{\sailRISCVvalclintLoad}{\saildoclabelled{sailRISCVzclintzyload}{\saildocval{}{\lstinputlisting[language=sail]{sail_latex_riscv/valzclint_load21de915eadac54aac5354dd7bcbb8d32.tex}}}}

\newcommand{\sailRISCVfnclintLoad}{\saildoclabelled{sailRISCVfnzclintzyload}{\saildocfn{}{\lstinputlisting[language=sail]{sail_latex_riscv/fnzclint_load21de915eadac54aac5354dd7bcbb8d32.tex}}}}

\newcommand{\sailRISCVvalclintDispatch}{\saildoclabelled{sailRISCVzclintzydispatch}{\saildocval{}{\lstinputlisting[language=sail]{sail_latex_riscv/valzclint_dispatch8f07ea27f21c1842cfbd7abdf66f1da6.tex}}}}

\newcommand{\sailRISCVfnclintDispatch}{\saildoclabelled{sailRISCVfnzclintzydispatch}{\saildocfn{}{\lstinputlisting[language=sail]{sail_latex_riscv/fnzclint_dispatch8f07ea27f21c1842cfbd7abdf66f1da6.tex}}}}

\newcommand{\sailRISCVvalclintStore}{\saildoclabelled{sailRISCVzclintzystore}{\saildocval{}{\lstinputlisting[language=sail]{sail_latex_riscv/valzclint_store5ac6a4caa2fe222e7b924cd9a27ec52f.tex}}}}

\newcommand{\sailRISCVfnclintStore}{\saildoclabelled{sailRISCVfnzclintzystore}{\saildocfn{}{\lstinputlisting[language=sail]{sail_latex_riscv/fnzclint_store5ac6a4caa2fe222e7b924cd9a27ec52f.tex}}}}

\newcommand{\sailRISCVvaltickClock}{\saildoclabelled{sailRISCVztickzyclock}{\saildocval{}{\lstinputlisting[language=sail]{sail_latex_riscv/valztick_clocka855f1c53aa2515a7a67cd69b5b3d663.tex}}}}

\newcommand{\sailRISCVfntickClock}{\saildoclabelled{sailRISCVfnztickzyclock}{\saildocfn{}{\lstinputlisting[language=sail]{sail_latex_riscv/fnztick_clocka855f1c53aa2515a7a67cd69b5b3d663.tex}}}}

\newcommand{\sailRISCVvalplatTermWrite}{\saildoclabelled{sailRISCVzplatzytermzywrite}{\saildocval{}{\lstinputlisting[language=sail]{sail_latex_riscv/valzplat_term_write036fbb8469eac6f64f37bf65565a5f02.tex}}}}

\newcommand{\sailRISCVvalplatTermRead}{\saildoclabelled{sailRISCVzplatzytermzyread}{\saildocval{}{\lstinputlisting[language=sail]{sail_latex_riscv/valzplat_term_readbc174c8d489d235edca54e912406e103.tex}}}}

\newcommand{\sailRISCVtypehtifCmd}{\saildoclabelled{sailRISCVtypezhtifzycmd}{\saildoctype{}{\lstinputlisting[language=sail]{sail_latex_riscv/typezhtif_cmd11bed6786969720c8c78224801520c9a.tex}}}}

\newcommand{\sailRISCVregisterhtifTohost}{\saildoclabelled{sailRISCVregisterzhtifzytohost}{\saildocregister{}{\lstinputlisting[language=sail]{sail_latex_riscv/registerzhtif_tohost8f4c187d54fc7660be7cb8ba7477bc9c.tex}}}}

\newcommand{\sailRISCVregisterhtifDone}{\saildoclabelled{sailRISCVregisterzhtifzydone}{\saildocregister{}{\lstinputlisting[language=sail]{sail_latex_riscv/registerzhtif_donee35d746818e476f4601316006140c798.tex}}}}

\newcommand{\sailRISCVregisterhtifExitCode}{\saildoclabelled{sailRISCVregisterzhtifzyexitzycode}{\saildocregister{}{\lstinputlisting[language=sail]{sail_latex_riscv/registerzhtif_exit_code4c49ae09bde37d07ed5c03fcc8471ff0.tex}}}}

\newcommand{\sailRISCVregisterhtifCmdWrite}{\saildoclabelled{sailRISCVregisterzhtifzycmdzywrite}{\saildocregister{}{\lstinputlisting[language=sail]{sail_latex_riscv/registerzhtif_cmd_writeedd1bc7316afddbceddc8c0b7455dcaa.tex}}}}

\newcommand{\sailRISCVregisterhtifPayloadWrites}{\saildoclabelled{sailRISCVregisterzhtifzypayloadzywrites}{\saildocregister{}{\lstinputlisting[language=sail]{sail_latex_riscv/registerzhtif_payload_writesa7d10d6d20d4ae40fd32b7afaa3e7f4f.tex}}}}

\newcommand{\sailRISCVvalresetHtif}{\saildoclabelled{sailRISCVzresetzyhtif}{\saildocval{}{\lstinputlisting[language=sail]{sail_latex_riscv/valzreset_htif438618d53329f3a89294df72f6af5cac.tex}}}}

\newcommand{\sailRISCVfnresetHtif}{\saildoclabelled{sailRISCVfnzresetzyhtif}{\saildocfn{}{\lstinputlisting[language=sail]{sail_latex_riscv/fnzreset_htif438618d53329f3a89294df72f6af5cac.tex}}}}

\newcommand{\sailRISCVvalhtifLoad}{\saildoclabelled{sailRISCVzhtifzyload}{\saildocval{}{\lstinputlisting[language=sail]{sail_latex_riscv/valzhtif_load7bc11b4853a5dae019f61722a0a6d6d7.tex}}}}

\newcommand{\sailRISCVfnhtifLoad}{\saildoclabelled{sailRISCVfnzhtifzyload}{\saildocfn{}{\lstinputlisting[language=sail]{sail_latex_riscv/fnzhtif_load7bc11b4853a5dae019f61722a0a6d6d7.tex}}}}

\newcommand{\sailRISCVvalhtifStore}{\saildoclabelled{sailRISCVzhtifzystore}{\saildocval{}{\lstinputlisting[language=sail]{sail_latex_riscv/valzhtif_storeab9d062182e5f884583204ccd435221d.tex}}}}

\newcommand{\sailRISCVfnhtifStore}{\saildoclabelled{sailRISCVfnzhtifzystore}{\saildocfn{}{\lstinputlisting[language=sail]{sail_latex_riscv/fnzhtif_storeab9d062182e5f884583204ccd435221d.tex}}}}

\newcommand{\sailRISCVvalhtifTick}{\saildoclabelled{sailRISCVzhtifzytick}{\saildocval{}{\lstinputlisting[language=sail]{sail_latex_riscv/valzhtif_tick227711b74637bcd9c79730b4942f90fb.tex}}}}

\newcommand{\sailRISCVfnhtifTick}{\saildoclabelled{sailRISCVfnzhtifzytick}{\saildocfn{}{\lstinputlisting[language=sail]{sail_latex_riscv/fnzhtif_tick227711b74637bcd9c79730b4942f90fb.tex}}}}

\newcommand{\sailRISCVvalwithinMmioReadable}{\saildoclabelled{sailRISCVzwithinzymmiozyreadable}{\saildocval{}{\lstinputlisting[language=sail]{sail_latex_riscv/valzwithin_mmio_readable2afaf2bf016b6ead5a5708ec8508d184.tex}}}}

\newcommand{\sailRISCVfnwithinMmioReadable}{\saildoclabelled{sailRISCVfnzwithinzymmiozyreadable}{\saildocfn{}{\lstinputlisting[language=sail]{sail_latex_riscv/fnzwithin_mmio_readable2afaf2bf016b6ead5a5708ec8508d184.tex}}}}

\newcommand{\sailRISCVvalwithinMmioWritable}{\saildoclabelled{sailRISCVzwithinzymmiozywritable}{\saildocval{}{\lstinputlisting[language=sail]{sail_latex_riscv/valzwithin_mmio_writable310089204ce2d4e6a33811b1982373ad.tex}}}}

\newcommand{\sailRISCVfnwithinMmioWritable}{\saildoclabelled{sailRISCVfnzwithinzymmiozywritable}{\saildocfn{}{\lstinputlisting[language=sail]{sail_latex_riscv/fnzwithin_mmio_writable310089204ce2d4e6a33811b1982373ad.tex}}}}

\newcommand{\sailRISCVvalmmioRead}{\saildoclabelled{sailRISCVzmmiozyread}{\saildocval{}{\lstinputlisting[language=sail]{sail_latex_riscv/valzmmio_read910c976398a0cf73e8b4d12641a665c1.tex}}}}

\newcommand{\sailRISCVfnmmioRead}{\saildoclabelled{sailRISCVfnzmmiozyread}{\saildocfn{}{\lstinputlisting[language=sail]{sail_latex_riscv/fnzmmio_read910c976398a0cf73e8b4d12641a665c1.tex}}}}

\newcommand{\sailRISCVvalmmioWrite}{\saildoclabelled{sailRISCVzmmiozywrite}{\saildocval{}{\lstinputlisting[language=sail]{sail_latex_riscv/valzmmio_writec1bd5fc64a027d200e43ae1730320ed7.tex}}}}

\newcommand{\sailRISCVfnmmioWrite}{\saildoclabelled{sailRISCVfnzmmiozywrite}{\saildocfn{}{\lstinputlisting[language=sail]{sail_latex_riscv/fnzmmio_writec1bd5fc64a027d200e43ae1730320ed7.tex}}}}

\newcommand{\sailRISCVvalinitPlatform}{\saildoclabelled{sailRISCVzinitzyplatform}{\saildocval{}{\lstinputlisting[language=sail]{sail_latex_riscv/valzinit_platform487cad4292d23a20d6a3d0b81157a250.tex}}}}

\newcommand{\sailRISCVfninitPlatform}{\saildoclabelled{sailRISCVfnzinitzyplatform}{\saildocfn{}{\lstinputlisting[language=sail]{sail_latex_riscv/fnzinit_platform487cad4292d23a20d6a3d0b81157a250.tex}}}}

\newcommand{\sailRISCVvaltickPlatform}{\saildoclabelled{sailRISCVztickzyplatform}{\saildocval{}{\lstinputlisting[language=sail]{sail_latex_riscv/valztick_platformc49251d76e66d78fce3dd1f18a27869c.tex}}}}

\newcommand{\sailRISCVfntickPlatform}{\saildoclabelled{sailRISCVfnztickzyplatform}{\saildocfn{}{\lstinputlisting[language=sail]{sail_latex_riscv/fnztick_platformc49251d76e66d78fce3dd1f18a27869c.tex}}}}

\newcommand{\sailRISCVvalhandleIllegal}{\saildoclabelled{sailRISCVzhandlezyillegal}{\saildocval{}{\lstinputlisting[language=sail]{sail_latex_riscv/valzhandle_illegal5907526c9d8e4989ca440b67ea4948c3.tex}}}}

\newcommand{\sailRISCVfnhandleIllegal}{\saildoclabelled{sailRISCVfnzhandlezyillegal}{\saildocfn{}{\lstinputlisting[language=sail]{sail_latex_riscv/fnzhandle_illegal5907526c9d8e4989ca440b67ea4948c3.tex}}}}

\newcommand{\sailRISCVvalplatformWfi}{\saildoclabelled{sailRISCVzplatformzywfi}{\saildocval{}{\lstinputlisting[language=sail]{sail_latex_riscv/valzplatform_wfi377b23f6619d58844892f179f2934ac6.tex}}}}

\newcommand{\sailRISCVfnplatformWfi}{\saildoclabelled{sailRISCVfnzplatformzywfi}{\saildocfn{}{\lstinputlisting[language=sail]{sail_latex_riscv/fnzplatform_wfi377b23f6619d58844892f179f2934ac6.tex}}}}

\newcommand{\sailRISCVvalisAlignedAddr}{\saildoclabelled{sailRISCVziszyalignedzyaddr}{\saildocval{}{\lstinputlisting[language=sail]{sail_latex_riscv/valzis_aligned_addr6fae0ea13237382ac6720d04123fd943.tex}}}}

\newcommand{\sailRISCVfnisAlignedAddr}{\saildoclabelled{sailRISCVfnziszyalignedzyaddr}{\saildocfn{}{\lstinputlisting[language=sail]{sail_latex_riscv/fnzis_aligned_addr6fae0ea13237382ac6720d04123fd943.tex}}}}

\newcommand{\sailRISCVvalreadKindOfFlags}{\saildoclabelled{sailRISCVzreadzykindzyofzyflags}{\saildocval{}{\lstinputlisting[language=sail]{sail_latex_riscv/valzread_kind_of_flagsbdfde0c548450764d5ed916ca1cb98ac.tex}}}}

\newcommand{\sailRISCVfnreadKindOfFlags}{\saildoclabelled{sailRISCVfnzreadzykindzyofzyflags}{\saildocfn{}{\lstinputlisting[language=sail]{sail_latex_riscv/fnzread_kind_of_flagsbdfde0c548450764d5ed916ca1cb98ac.tex}}}}

\newcommand{\sailRISCVvalphysMemRead}{\saildoclabelled{sailRISCVzphyszymemzyread}{\saildocval{}{\lstinputlisting[language=sail]{sail_latex_riscv/valzphys_mem_readdcad862ae3a42c22bfc78bad8e3328db.tex}}}}

\newcommand{\sailRISCVfnphysMemRead}{\saildoclabelled{sailRISCVfnzphyszymemzyread}{\saildocfn{}{\lstinputlisting[language=sail]{sail_latex_riscv/fnzphys_mem_readdcad862ae3a42c22bfc78bad8e3328db.tex}}}}

\newcommand{\sailRISCVvalcheckedMemRead}{\saildoclabelled{sailRISCVzcheckedzymemzyread}{\saildocval{}{\lstinputlisting[language=sail]{sail_latex_riscv/valzchecked_mem_read46a92fcd62c31279edfc3bc18c424fa0.tex}}}}

\newcommand{\sailRISCVfncheckedMemRead}{\saildoclabelled{sailRISCVfnzcheckedzymemzyread}{\saildocfn{}{\lstinputlisting[language=sail]{sail_latex_riscv/fnzchecked_mem_read46a92fcd62c31279edfc3bc18c424fa0.tex}}}}

\newcommand{\sailRISCVvalpmpMemRead}{\saildoclabelled{sailRISCVzpmpzymemzyread}{\saildocval{}{\lstinputlisting[language=sail]{sail_latex_riscv/valzpmp_mem_readc45533831bda1c394396c536ba168b7a.tex}}}}

\newcommand{\sailRISCVfnpmpMemRead}{\saildoclabelled{sailRISCVfnzpmpzymemzyread}{\saildocfn{}{\lstinputlisting[language=sail]{sail_latex_riscv/fnzpmp_mem_readc45533831bda1c394396c536ba168b7a.tex}}}}

\newcommand{\sailRISCVvalrvfiRead}{\saildoclabelled{sailRISCVzrvfizyread}{\saildocval{}{\lstinputlisting[language=sail]{sail_latex_riscv/valzrvfi_readaee7411fcd70e67f9e6d2f7f9f563435.tex}}}}

\newcommand{\sailRISCVfnrvfiRead}{\saildoclabelled{sailRISCVfnzrvfizyread}{\saildocfn{}{\lstinputlisting[language=sail]{sail_latex_riscv/fnzrvfi_readaee7411fcd70e67f9e6d2f7f9f563435.tex}}}}

\newcommand{\sailRISCVvalmemRead}{\saildoclabelled{sailRISCVzmemzyread}{\saildocval{}{\lstinputlisting[language=sail]{sail_latex_riscv/valzmem_readbc59b9b8e622af015b97ceb8dcd5c69e.tex}}}}

\newcommand{\sailRISCVvalmemReadPriv}{\saildoclabelled{sailRISCVzmemzyreadzypriv}{\saildocval{}{\lstinputlisting[language=sail]{sail_latex_riscv/valzmem_read_priv0ea70ac99462900b0ee737fe6e1d211b.tex}}}}

\newcommand{\sailRISCVvalmemReadMeta}{\saildoclabelled{sailRISCVzmemzyreadzymeta}{\saildocval{}{\lstinputlisting[language=sail]{sail_latex_riscv/valzmem_read_metaa66233a97620233d3f80f4bf1a13c232.tex}}}}

\newcommand{\sailRISCVvalmemReadPrivMeta}{\saildoclabelled{sailRISCVzmemzyreadzyprivzymeta}{\saildocval{}{\lstinputlisting[language=sail]{sail_latex_riscv/valzmem_read_priv_metad38f116b4c7942cd9d9f7441528c45e1.tex}}}}

\newcommand{\sailRISCVfnmemReadPrivMeta}{\saildoclabelled{sailRISCVfnzmemzyreadzyprivzymeta}{\saildocfn{}{\lstinputlisting[language=sail]{sail_latex_riscv/fnzmem_read_priv_metad38f116b4c7942cd9d9f7441528c45e1.tex}}}}

\newcommand{\sailRISCVfnmemReadMeta}{\saildoclabelled{sailRISCVfnzmemzyreadzymeta}{\saildocfn{}{\lstinputlisting[language=sail]{sail_latex_riscv/fnzmem_read_metaa66233a97620233d3f80f4bf1a13c232.tex}}}}

\newcommand{\sailRISCVfnmemReadPriv}{\saildoclabelled{sailRISCVfnzmemzyreadzypriv}{\saildocfn{}{\lstinputlisting[language=sail]{sail_latex_riscv/fnzmem_read_priv0ea70ac99462900b0ee737fe6e1d211b.tex}}}}

\newcommand{\sailRISCVfnmemRead}{\saildoclabelled{sailRISCVfnzmemzyread}{\saildocfn{}{\lstinputlisting[language=sail]{sail_latex_riscv/fnzmem_readbc59b9b8e622af015b97ceb8dcd5c69e.tex}}}}

\newcommand{\sailRISCVvalmemWriteEa}{\saildoclabelled{sailRISCVzmemzywritezyea}{\saildocval{}{\lstinputlisting[language=sail]{sail_latex_riscv/valzmem_write_eaf1486ee81ccf925e874de8977b0270e9.tex}}}}

\newcommand{\sailRISCVfnmemWriteEa}{\saildoclabelled{sailRISCVfnzmemzywritezyea}{\saildocfn{}{\lstinputlisting[language=sail]{sail_latex_riscv/fnzmem_write_eaf1486ee81ccf925e874de8977b0270e9.tex}}}}

\newcommand{\sailRISCVvalrvfiWrite}{\saildoclabelled{sailRISCVzrvfizywrite}{\saildocval{}{\lstinputlisting[language=sail]{sail_latex_riscv/valzrvfi_write8e76a07b5a6f2a7b76947099108996b1.tex}}}}

\newcommand{\sailRISCVfnrvfiWrite}{\saildoclabelled{sailRISCVfnzrvfizywrite}{\saildocfn{}{\lstinputlisting[language=sail]{sail_latex_riscv/fnzrvfi_write8e76a07b5a6f2a7b76947099108996b1.tex}}}}

\newcommand{\sailRISCVvalphysMemWrite}{\saildoclabelled{sailRISCVzphyszymemzywrite}{\saildocval{}{\lstinputlisting[language=sail]{sail_latex_riscv/valzphys_mem_writefae7815adda192bed56129eba4b7bb01.tex}}}}

\newcommand{\sailRISCVfnphysMemWrite}{\saildoclabelled{sailRISCVfnzphyszymemzywrite}{\saildocfn{}{\lstinputlisting[language=sail]{sail_latex_riscv/fnzphys_mem_writefae7815adda192bed56129eba4b7bb01.tex}}}}

\newcommand{\sailRISCVvalcheckedMemWrite}{\saildoclabelled{sailRISCVzcheckedzymemzywrite}{\saildocval{}{\lstinputlisting[language=sail]{sail_latex_riscv/valzchecked_mem_write765e0693788c9f4b01c2243ece39909e.tex}}}}

\newcommand{\sailRISCVfncheckedMemWrite}{\saildoclabelled{sailRISCVfnzcheckedzymemzywrite}{\saildocfn{}{\lstinputlisting[language=sail]{sail_latex_riscv/fnzchecked_mem_write765e0693788c9f4b01c2243ece39909e.tex}}}}

\newcommand{\sailRISCVvalpmpMemWrite}{\saildoclabelled{sailRISCVzpmpzymemzywrite}{\saildocval{}{\lstinputlisting[language=sail]{sail_latex_riscv/valzpmp_mem_write4ae53a6de2384826cc3e765eccd350e8.tex}}}}

\newcommand{\sailRISCVfnpmpMemWrite}{\saildoclabelled{sailRISCVfnzpmpzymemzywrite}{\saildocfn{}{\lstinputlisting[language=sail]{sail_latex_riscv/fnzpmp_mem_write4ae53a6de2384826cc3e765eccd350e8.tex}}}}

\newcommand{\sailRISCVvalmemWriteValuePrivMeta}{\saildoclabelled{sailRISCVzmemzywritezyvaluezyprivzymeta}{\saildocval{}{\lstinputlisting[language=sail]{sail_latex_riscv/valzmem_write_value_priv_meta8704060509807d04b1fcb48d80cdcd15.tex}}}}

\newcommand{\sailRISCVfnmemWriteValuePrivMeta}{\saildoclabelled{sailRISCVfnzmemzywritezyvaluezyprivzymeta}{\saildocfn{}{\lstinputlisting[language=sail]{sail_latex_riscv/fnzmem_write_value_priv_meta8704060509807d04b1fcb48d80cdcd15.tex}}}}

\newcommand{\sailRISCVvalmemWriteValuePriv}{\saildoclabelled{sailRISCVzmemzywritezyvaluezypriv}{\saildocval{}{\lstinputlisting[language=sail]{sail_latex_riscv/valzmem_write_value_privd13a5127f92750616ff239409609b843.tex}}}}

\newcommand{\sailRISCVfnmemWriteValuePriv}{\saildoclabelled{sailRISCVfnzmemzywritezyvaluezypriv}{\saildocfn{}{\lstinputlisting[language=sail]{sail_latex_riscv/fnzmem_write_value_privd13a5127f92750616ff239409609b843.tex}}}}

\newcommand{\sailRISCVvalmemWriteValueMeta}{\saildoclabelled{sailRISCVzmemzywritezyvaluezymeta}{\saildocval{}{\lstinputlisting[language=sail]{sail_latex_riscv/valzmem_write_value_meta586f37dd78d9b5be2a948e83778e6186.tex}}}}

\newcommand{\sailRISCVfnmemWriteValueMeta}{\saildoclabelled{sailRISCVfnzmemzywritezyvaluezymeta}{\saildocfn{}{\lstinputlisting[language=sail]{sail_latex_riscv/fnzmem_write_value_meta586f37dd78d9b5be2a948e83778e6186.tex}}}}

\newcommand{\sailRISCVvalmemWriteValue}{\saildoclabelled{sailRISCVzmemzywritezyvalue}{\saildocval{}{\lstinputlisting[language=sail]{sail_latex_riscv/valzmem_write_valuec32f59efd2dcb7ea78ea4c9778bdf2e1.tex}}}}

\newcommand{\sailRISCVfnmemWriteValue}{\saildoclabelled{sailRISCVfnzmemzywritezyvalue}{\saildocfn{}{\lstinputlisting[language=sail]{sail_latex_riscv/fnzmem_write_valuec32f59efd2dcb7ea78ea4c9778bdf2e1.tex}}}}

\newcommand{\sailRISCVvalmemReadCap}{\saildoclabelled{sailRISCVzmemzyreadzycap}{\saildocval{}{\lstinputlisting[language=sail]{sail_latex_riscv/valzmem_read_cap518c2935d72d61e3ceee2abb80c9fce5.tex}}}}

\newcommand{\sailRISCVfnmemReadCap}{\saildoclabelled{sailRISCVfnzmemzyreadzycap}{\saildocfn{}{\lstinputlisting[language=sail]{sail_latex_riscv/fnzmem_read_cap518c2935d72d61e3ceee2abb80c9fce5.tex}}}}

\newcommand{\sailRISCVvalmemWriteEaCap}{\saildoclabelled{sailRISCVzmemzywritezyeazycap}{\saildocval{}{\lstinputlisting[language=sail]{sail_latex_riscv/valzmem_write_ea_capaa2cfa0cf86165d13bd79024ad44b71d.tex}}}}

\newcommand{\sailRISCVfnmemWriteEaCap}{\saildoclabelled{sailRISCVfnzmemzywritezyeazycap}{\saildocfn{}{\lstinputlisting[language=sail]{sail_latex_riscv/fnzmem_write_ea_capaa2cfa0cf86165d13bd79024ad44b71d.tex}}}}

\newcommand{\sailRISCVvalmemWriteCap}{\saildoclabelled{sailRISCVzmemzywritezycap}{\saildocval{}{\lstinputlisting[language=sail]{sail_latex_riscv/valzmem_write_cap1a1d6143df72b48afdcbdae2e99e77f4.tex}}}}

\newcommand{\sailRISCVfnmemWriteCap}{\saildoclabelled{sailRISCVfnzmemzywritezycap}{\saildocfn{}{\lstinputlisting[language=sail]{sail_latex_riscv/fnzmem_write_cap1a1d6143df72b48afdcbdae2e99e77f4.tex}}}}

\newcommand{\sailRISCVtypepteAttribs}{\saildoclabelled{sailRISCVtypezpteAttribs}{\saildoctype{}{\lstinputlisting[language=sail]{sail_latex_riscv/typezpteattribs7d3296aef3ee8195982b63fe09ada4c1.tex}}}}

\newcommand{\sailRISCVtypePTEBits}{\saildoclabelled{sailRISCVtypezPTEzyBits}{\saildoctype{}{\lstinputlisting[language=sail]{sail_latex_riscv/typezpte_bits8fb42d02184f92f56f1b4720e0954290.tex}}}}

\newcommand{\sailRISCVtypeextPte}{\saildoclabelled{sailRISCVtypezextPte}{\saildoctype{}{\lstinputlisting[language=sail]{sail_latex_riscv/typezextpte08291a23c157d4bccbd0a79786be8e37.tex}}}}

\newcommand{\sailRISCVtypeExtPTEBits}{\saildoclabelled{sailRISCVtypezExtzyPTEzyBits}{\saildoctype{}{\lstinputlisting[language=sail]{sail_latex_riscv/typezext_pte_bitsa7158098bccf12e34e307673a27a3abd.tex}}}}

\newcommand{\sailRISCVletdefaultSvThreeTwoExtPte}{\saildoclabelled{sailRISCVletzdefaultzysv32zyextzypte}{\saildoclet{}{\lstinputlisting[language=sail]{sail_latex_riscv/letzdefault_sv32_ext_ptef5811ea386bcee45bcef0e0d448074f6.tex}}}}

\newcommand{\sailRISCVvalisPTEPtr}{\saildoclabelled{sailRISCVzisPTEPtr}{\saildocval{}{\lstinputlisting[language=sail]{sail_latex_riscv/valzispteptrba43877f6dc42c078f937aa41c879446.tex}}}}

\newcommand{\sailRISCVfnisPTEPtr}{\saildoclabelled{sailRISCVfnzisPTEPtr}{\saildocfn{}{\lstinputlisting[language=sail]{sail_latex_riscv/fnzispteptrba43877f6dc42c078f937aa41c879446.tex}}}}

\newcommand{\sailRISCVvalisInvalidPTE}{\saildoclabelled{sailRISCVzisInvalidPTE}{\saildocval{}{\lstinputlisting[language=sail]{sail_latex_riscv/valzisinvalidpte046e5b8d20df2f7ed228312986edeed3.tex}}}}

\newcommand{\sailRISCVfnisInvalidPTE}{\saildoclabelled{sailRISCVfnzisInvalidPTE}{\saildocfn{}{\lstinputlisting[language=sail]{sail_latex_riscv/fnzisinvalidpte046e5b8d20df2f7ed228312986edeed3.tex}}}}

\newcommand{\sailRISCVtypePTECheck}{\saildoclabelled{sailRISCVtypezPTEzyCheck}{\saildoctype{}{\lstinputlisting[language=sail]{sail_latex_riscv/typezpte_checke4890d6be3f8a927bdcbd51b1a5be9d3.tex}}}}

\newcommand{\sailRISCVvalcheckPTEPermission}{\saildoclabelled{sailRISCVzcheckPTEPermission}{\saildocval{}{\lstinputlisting[language=sail]{sail_latex_riscv/valzcheckptepermissione94004ed8067442121c54ab1b95848d0.tex}}}}

\newcommand{\sailRISCVfncheckPTEPermission}{\saildoclabelled{sailRISCVfnzcheckPTEPermission}{\saildocfn{}{\lstinputlisting[language=sail]{sail_latex_riscv/fnzcheckptepermissione94004ed8067442121c54ab1b95848d0.tex}}}}

\newcommand{\sailRISCVvalupdatePTEBits}{\saildoclabelled{sailRISCVzupdatezyPTEzyBits}{\saildocval{}{\lstinputlisting[language=sail]{sail_latex_riscv/valzupdate_pte_bitsd84d357c2799412dbae43ca95464282b.tex}}}}

\newcommand{\sailRISCVfnupdatePTEBits}{\saildoclabelled{sailRISCVfnzupdatezyPTEzyBits}{\saildocfn{}{\lstinputlisting[language=sail]{sail_latex_riscv/fnzupdate_pte_bitsd84d357c2799412dbae43ca95464282b.tex}}}}

\newcommand{\sailRISCVtypePTWError}{\saildoclabelled{sailRISCVtypezPTWzyError}{\saildoctype{}{\lstinputlisting[language=sail]{sail_latex_riscv/typezptw_errore6acdb6d9897199828c918d43d6e0475.tex}}}}

\newcommand{\sailRISCVvalptwErrorToStr}{\saildoclabelled{sailRISCVzptwzyerrorzytozystr}{\saildocval{}{\lstinputlisting[language=sail]{sail_latex_riscv/valzptw_error_to_stra1d14633b5815718af0734f3fe6896fb.tex}}}}

\newcommand{\sailRISCVfnptwErrorToStr}{\saildoclabelled{sailRISCVfnzptwzyerrorzytozystr}{\saildocfn{}{\lstinputlisting[language=sail]{sail_latex_riscv/fnzptw_error_to_stra1d14633b5815718af0734f3fe6896fb.tex}}}}

\newcommand{\sailRISCVoverloadTTTtoStr}{\saildoclabelled{sailRISCVoverloadTTTztozystr}{\saildocoverload{}{\lstinputlisting[language=sail]{sail_latex_riscv/overloadTTTzto_str8b7a6895ae35945bd4740e9f790c43ee.tex}}}}

\newcommand{\sailRISCVvalextGetPtwError}{\saildoclabelled{sailRISCVzextzygetzyptwzyerror}{\saildocval{}{\lstinputlisting[language=sail]{sail_latex_riscv/valzext_get_ptw_errorb38503fe4519ddae4ed13f9933d3c0a5.tex}}}}

\newcommand{\sailRISCVfnextGetPtwError}{\saildoclabelled{sailRISCVfnzextzygetzyptwzyerror}{\saildocfn{}{\lstinputlisting[language=sail]{sail_latex_riscv/fnzext_get_ptw_errorb38503fe4519ddae4ed13f9933d3c0a5.tex}}}}

\newcommand{\sailRISCVvaltranslationException}{\saildoclabelled{sailRISCVztranslationException}{\saildocval{}{\lstinputlisting[language=sail]{sail_latex_riscv/valztranslationexceptionbd47ba58dcb6fcc0be14f6efdd551ad8.tex}}}}

\newcommand{\sailRISCVfntranslationException}{\saildoclabelled{sailRISCVfnztranslationException}{\saildocfn{}{\lstinputlisting[language=sail]{sail_latex_riscv/fnztranslationexceptionbd47ba58dcb6fcc0be14f6efdd551ad8.tex}}}}

\newcommand{\sailRISCVletPAGESIZEBITS}{\saildoclabelled{sailRISCVletzPAGESIZEzyBITS}{\saildoclet{}{\lstinputlisting[language=sail]{sail_latex_riscv/letzpagesize_bits1fe5a2b7d022405eec5fb9efa9146c41.tex}}}}

\newcommand{\sailRISCVtypevaddrThreeTwo}{\saildoclabelled{sailRISCVtypezvaddr32}{\saildoctype{}{\lstinputlisting[language=sail]{sail_latex_riscv/typezvaddr321f5ce490361ac653dc4efb96fb3a4c98.tex}}}}

\newcommand{\sailRISCVtypepaddrThreeTwo}{\saildoclabelled{sailRISCVtypezpaddr32}{\saildoctype{}{\lstinputlisting[language=sail]{sail_latex_riscv/typezpaddr326acc54e31f94adeb46900dacee685a4f.tex}}}}

\newcommand{\sailRISCVtypepteThreeTwo}{\saildoclabelled{sailRISCVtypezpte32}{\saildoctype{}{\lstinputlisting[language=sail]{sail_latex_riscv/typezpte3289880ebfd8a537e73fa0b03b58db09a5.tex}}}}

\newcommand{\sailRISCVtypeasidThreeTwo}{\saildoclabelled{sailRISCVtypezasid32}{\saildoctype{}{\lstinputlisting[language=sail]{sail_latex_riscv/typezasid322158fcb044d3d182ba538bfe0883c5fc.tex}}}}

\newcommand{\sailRISCVvalcurAsidThreeTwo}{\saildoclabelled{sailRISCVzcurAsid32}{\saildocval{}{\lstinputlisting[language=sail]{sail_latex_riscv/valzcurasid32f5f0e43e5813461351dff485d90e4aa4.tex}}}}

\newcommand{\sailRISCVfncurAsidThreeTwo}{\saildoclabelled{sailRISCVfnzcurAsid32}{\saildocfn{}{\lstinputlisting[language=sail]{sail_latex_riscv/fnzcurasid32f5f0e43e5813461351dff485d90e4aa4.tex}}}}

\newcommand{\sailRISCVvalcurPTBThreeTwo}{\saildoclabelled{sailRISCVzcurPTB32}{\saildocval{}{\lstinputlisting[language=sail]{sail_latex_riscv/valzcurptb320f32995fafcf5ebca7eeda72aee3f74f.tex}}}}

\newcommand{\sailRISCVfncurPTBThreeTwo}{\saildoclabelled{sailRISCVfnzcurPTB32}{\saildocfn{}{\lstinputlisting[language=sail]{sail_latex_riscv/fnzcurptb320f32995fafcf5ebca7eeda72aee3f74f.tex}}}}

\newcommand{\sailRISCVletSVThreeTwoLEVELBITS}{\saildoclabelled{sailRISCVletzSV32zyLEVELzyBITS}{\saildoclet{}{\lstinputlisting[language=sail]{sail_latex_riscv/letzsv32_level_bits7af6b65d24fb6005b06c4bb5360a2851.tex}}}}

\newcommand{\sailRISCVletSVThreeTwoLEVELS}{\saildoclabelled{sailRISCVletzSV32zyLEVELS}{\saildoclet{}{\lstinputlisting[language=sail]{sail_latex_riscv/letzsv32_levels64809ecb87a72a19b33c42c931bfe215.tex}}}}

\newcommand{\sailRISCVletPTEThreeTwoLOGSIZE}{\saildoclabelled{sailRISCVletzPTE32zyLOGzySIZE}{\saildoclet{}{\lstinputlisting[language=sail]{sail_latex_riscv/letzpte32_log_sizec1c0bae5116bacf8a2b119b16fd8ea5a.tex}}}}

\newcommand{\sailRISCVletPTEThreeTwoSIZE}{\saildoclabelled{sailRISCVletzPTE32zySIZE}{\saildoclet{}{\lstinputlisting[language=sail]{sail_latex_riscv/letzpte32_sizecbedd4089b99ea14685e6c55c42999b1.tex}}}}

\newcommand{\sailRISCVtypeSVThreeTwoVaddr}{\saildoclabelled{sailRISCVtypezSV32zyVaddr}{\saildoctype{}{\lstinputlisting[language=sail]{sail_latex_riscv/typezsv32_vaddre98811ac62644b6d3e91ca2d91e5afb3.tex}}}}

\newcommand{\sailRISCVtypeSVThreeTwoPaddr}{\saildoclabelled{sailRISCVtypezSV32zyPaddr}{\saildoctype{}{\lstinputlisting[language=sail]{sail_latex_riscv/typezsv32_paddr9f5cfa9b0e96f55368dd46c5e00148a5.tex}}}}

\newcommand{\sailRISCVtypeSVThreeTwoPTE}{\saildoclabelled{sailRISCVtypezSV32zyPTE}{\saildoctype{}{\lstinputlisting[language=sail]{sail_latex_riscv/typezsv32_pte054d26e26929f0189a30c1313f7f54e1.tex}}}}

\newcommand{\sailRISCVtypepaddrSixFour}{\saildoclabelled{sailRISCVtypezpaddr64}{\saildoctype{}{\lstinputlisting[language=sail]{sail_latex_riscv/typezpaddr64f58cecb2121bad03c9e64da6cd15339b.tex}}}}

\newcommand{\sailRISCVtypepteSixFour}{\saildoclabelled{sailRISCVtypezpte64}{\saildoctype{}{\lstinputlisting[language=sail]{sail_latex_riscv/typezpte6458e3ea8a5bd698d68e67a410a526fd60.tex}}}}

\newcommand{\sailRISCVtypeasidSixFour}{\saildoclabelled{sailRISCVtypezasid64}{\saildoctype{}{\lstinputlisting[language=sail]{sail_latex_riscv/typezasid64774ad52daab98e87d2a0c714473b041f.tex}}}}

\newcommand{\sailRISCVvalcurAsidSixFour}{\saildoclabelled{sailRISCVzcurAsid64}{\saildocval{}{\lstinputlisting[language=sail]{sail_latex_riscv/valzcurasid64b2c4a622c945e46b0aea25ce9a07740f.tex}}}}

\newcommand{\sailRISCVfncurAsidSixFour}{\saildoclabelled{sailRISCVfnzcurAsid64}{\saildocfn{}{\lstinputlisting[language=sail]{sail_latex_riscv/fnzcurasid64b2c4a622c945e46b0aea25ce9a07740f.tex}}}}

\newcommand{\sailRISCVvalcurPTBSixFour}{\saildoclabelled{sailRISCVzcurPTB64}{\saildocval{}{\lstinputlisting[language=sail]{sail_latex_riscv/valzcurptb6479eaa1d344e911b626ff56a8856bcd50.tex}}}}

\newcommand{\sailRISCVfncurPTBSixFour}{\saildoclabelled{sailRISCVfnzcurPTB64}{\saildocfn{}{\lstinputlisting[language=sail]{sail_latex_riscv/fnzcurptb6479eaa1d344e911b626ff56a8856bcd50.tex}}}}

\newcommand{\sailRISCVletSVThreeNineLEVELBITS}{\saildoclabelled{sailRISCVletzSV39zyLEVELzyBITS}{\saildoclet{}{\lstinputlisting[language=sail]{sail_latex_riscv/letzsv39_level_bits770c9fd85cef0ff2b4d7a95079d3ae0e.tex}}}}

\newcommand{\sailRISCVletSVThreeNineLEVELS}{\saildoclabelled{sailRISCVletzSV39zyLEVELS}{\saildoclet{}{\lstinputlisting[language=sail]{sail_latex_riscv/letzsv39_levelsa18aaaf885aa1841ed5a91be9bfe252f.tex}}}}

\newcommand{\sailRISCVletPTEThreeNineLOGSIZE}{\saildoclabelled{sailRISCVletzPTE39zyLOGzySIZE}{\saildoclet{}{\lstinputlisting[language=sail]{sail_latex_riscv/letzpte39_log_size48deebfa921eeef530f20ea8d54bb13e.tex}}}}

\newcommand{\sailRISCVletPTEThreeNineSIZE}{\saildoclabelled{sailRISCVletzPTE39zySIZE}{\saildoclet{}{\lstinputlisting[language=sail]{sail_latex_riscv/letzpte39_size52e305796983095f49220ea66ee6800d.tex}}}}

\newcommand{\sailRISCVtypevaddrThreeNine}{\saildoclabelled{sailRISCVtypezvaddr39}{\saildoctype{}{\lstinputlisting[language=sail]{sail_latex_riscv/typezvaddr39d9a5d0d949c4cbbbc71c053903cf73cb.tex}}}}

\newcommand{\sailRISCVtypeSVThreeNineVaddr}{\saildoclabelled{sailRISCVtypezSV39zyVaddr}{\saildoctype{}{\lstinputlisting[language=sail]{sail_latex_riscv/typezsv39_vaddr9f1c1a92c2c683b5b23ddacf71051c03.tex}}}}

\newcommand{\sailRISCVtypeSVThreeNinePaddr}{\saildoclabelled{sailRISCVtypezSV39zyPaddr}{\saildoctype{}{\lstinputlisting[language=sail]{sail_latex_riscv/typezsv39_paddr8e7ac2ddc4a718336a4f3894cdd91edd.tex}}}}

\newcommand{\sailRISCVtypeSVThreeNinePTE}{\saildoclabelled{sailRISCVtypezSV39zyPTE}{\saildoctype{}{\lstinputlisting[language=sail]{sail_latex_riscv/typezsv39_ptea9e6dd73ae92b5be63b2a7edb5ccc8a3.tex}}}}

\newcommand{\sailRISCVletSVFourEightLEVELBITS}{\saildoclabelled{sailRISCVletzSV48zyLEVELzyBITS}{\saildoclet{}{\lstinputlisting[language=sail]{sail_latex_riscv/letzsv48_level_bits8d42ea35a47abd9bf98a0e69297778f9.tex}}}}

\newcommand{\sailRISCVletSVFourEightLEVELS}{\saildoclabelled{sailRISCVletzSV48zyLEVELS}{\saildoclet{}{\lstinputlisting[language=sail]{sail_latex_riscv/letzsv48_levels6840b0c0dc7823e2a3360367a3bf9fde.tex}}}}

\newcommand{\sailRISCVletPTEFourEightLOGSIZE}{\saildoclabelled{sailRISCVletzPTE48zyLOGzySIZE}{\saildoclet{}{\lstinputlisting[language=sail]{sail_latex_riscv/letzpte48_log_size4488de195e7d6006961ab6f6e1c11121.tex}}}}

\newcommand{\sailRISCVletPTEFourEightSIZE}{\saildoclabelled{sailRISCVletzPTE48zySIZE}{\saildoclet{}{\lstinputlisting[language=sail]{sail_latex_riscv/letzpte48_sizeb6377c3b52e81b37c5e1857b9a769f07.tex}}}}

\newcommand{\sailRISCVtypevaddrFourEight}{\saildoclabelled{sailRISCVtypezvaddr48}{\saildoctype{}{\lstinputlisting[language=sail]{sail_latex_riscv/typezvaddr4805a60a23d20e81c5ff9e6a5609aea22f.tex}}}}

\newcommand{\sailRISCVtypepteFourEight}{\saildoclabelled{sailRISCVtypezpte48}{\saildoctype{}{\lstinputlisting[language=sail]{sail_latex_riscv/typezpte48f7bf6ee72ef0ed2c1101c9e605f2f127.tex}}}}

\newcommand{\sailRISCVtypeSVFourEightVaddr}{\saildoclabelled{sailRISCVtypezSV48zyVaddr}{\saildoctype{}{\lstinputlisting[language=sail]{sail_latex_riscv/typezsv48_vaddrd3fdeb6a4cba35ed677a6112bb677bc2.tex}}}}

\newcommand{\sailRISCVtypeSVFourEightPaddr}{\saildoclabelled{sailRISCVtypezSV48zyPaddr}{\saildoctype{}{\lstinputlisting[language=sail]{sail_latex_riscv/typezsv48_paddrb31db8c73707db172008fed5607c9247.tex}}}}

\newcommand{\sailRISCVtypeSVFourEightPTE}{\saildoclabelled{sailRISCVtypezSV48zyPTE}{\saildoctype{}{\lstinputlisting[language=sail]{sail_latex_riscv/typezsv48_pte3a2f757f22affd3d18dd7c046956b713.tex}}}}

\newcommand{\sailRISCVtypePTWResult}{\saildoclabelled{sailRISCVtypezPTWzyResult}{\saildoctype{}{\lstinputlisting[language=sail]{sail_latex_riscv/typezptw_result387dae5ea1a379e54bc31c478988b63d.tex}}}}

\newcommand{\sailRISCVtypeTRResult}{\saildoclabelled{sailRISCVtypezTRzyResult}{\saildoctype{}{\lstinputlisting[language=sail]{sail_latex_riscv/typeztr_result3a5e84bc36b624bd12e1a801e779c4c4.tex}}}}

\newcommand{\sailRISCVtypeTLBEntry}{\saildoclabelled{sailRISCVtypezTLBzyEntry}{\saildoctype{}{\lstinputlisting[language=sail]{sail_latex_riscv/typeztlb_entryd6a3e312d7372e7db4df8f6c3707b5fa.tex}}}}

\newcommand{\sailRISCVvalmakeTLBEntry}{\saildoclabelled{sailRISCVzmakezyTLBzyEntry}{\saildocval{}{\lstinputlisting[language=sail]{sail_latex_riscv/valzmake_tlb_entry6b22aedb2f264f70f3e1bd2d5dd6f057.tex}}}}

\newcommand{\sailRISCVfnmakeTLBEntry}{\saildoclabelled{sailRISCVfnzmakezyTLBzyEntry}{\saildocfn{}{\lstinputlisting[language=sail]{sail_latex_riscv/fnzmake_tlb_entry6b22aedb2f264f70f3e1bd2d5dd6f057.tex}}}}

\newcommand{\sailRISCVvalmatchTLBEntry}{\saildoclabelled{sailRISCVzmatchzyTLBzyEntry}{\saildocval{}{\lstinputlisting[language=sail]{sail_latex_riscv/valzmatch_tlb_entry15cb314356c461bfc7d1b30f45fecb98.tex}}}}

\newcommand{\sailRISCVfnmatchTLBEntry}{\saildoclabelled{sailRISCVfnzmatchzyTLBzyEntry}{\saildocfn{}{\lstinputlisting[language=sail]{sail_latex_riscv/fnzmatch_tlb_entry15cb314356c461bfc7d1b30f45fecb98.tex}}}}

\newcommand{\sailRISCVvalflushTLBEntry}{\saildoclabelled{sailRISCVzflushzyTLBzyEntry}{\saildocval{}{\lstinputlisting[language=sail]{sail_latex_riscv/valzflush_tlb_entry8073b262f27cbbda50ac10a3cc21c463.tex}}}}

\newcommand{\sailRISCVfnflushTLBEntry}{\saildoclabelled{sailRISCVfnzflushzyTLBzyEntry}{\saildocfn{}{\lstinputlisting[language=sail]{sail_latex_riscv/fnzflush_tlb_entry8073b262f27cbbda50ac10a3cc21c463.tex}}}}

\newcommand{\sailRISCVvalwalkThreeNine}{\saildoclabelled{sailRISCVzwalk39}{\saildocval{}{\lstinputlisting[language=sail]{sail_latex_riscv/valzwalk398233e1f75321f48773213830a045bfac.tex}}}}

\newcommand{\sailRISCVfnwalkThreeNine}{\saildoclabelled{sailRISCVfnzwalk39}{\saildocfn{}{\lstinputlisting[language=sail]{sail_latex_riscv/fnzwalk398233e1f75321f48773213830a045bfac.tex}}}}

\newcommand{\sailRISCVtypeTLBThreeNineEntry}{\saildoclabelled{sailRISCVtypezTLB39zyEntry}{\saildoctype{}{\lstinputlisting[language=sail]{sail_latex_riscv/typeztlb39_entrya6d01947ae0af49073403c37a147bc7a.tex}}}}

\newcommand{\sailRISCVregistertlbThreeNine}{\saildoclabelled{sailRISCVregisterztlb39}{\saildocregister{}{\lstinputlisting[language=sail]{sail_latex_riscv/registerztlb39f472715eef1441f08dbf377dfa8b65ed.tex}}}}

\newcommand{\sailRISCVvallookupTLBThreeNine}{\saildoclabelled{sailRISCVzlookupzyTLB39}{\saildocval{}{\lstinputlisting[language=sail]{sail_latex_riscv/valzlookup_tlb39701cebd3fa43ee6815ecaa975a199f5b.tex}}}}

\newcommand{\sailRISCVfnlookupTLBThreeNine}{\saildoclabelled{sailRISCVfnzlookupzyTLB39}{\saildocfn{}{\lstinputlisting[language=sail]{sail_latex_riscv/fnzlookup_tlb39701cebd3fa43ee6815ecaa975a199f5b.tex}}}}

\newcommand{\sailRISCVvaladdToTLBThreeNine}{\saildoclabelled{sailRISCVzaddzytozyTLB39}{\saildocval{}{\lstinputlisting[language=sail]{sail_latex_riscv/valzadd_to_tlb39f45d8d071e2986dede49d7c2109cd4f5.tex}}}}

\newcommand{\sailRISCVfnaddToTLBThreeNine}{\saildoclabelled{sailRISCVfnzaddzytozyTLB39}{\saildocfn{}{\lstinputlisting[language=sail]{sail_latex_riscv/fnzadd_to_tlb39f45d8d071e2986dede49d7c2109cd4f5.tex}}}}

\newcommand{\sailRISCVvalwriteTLBThreeNine}{\saildoclabelled{sailRISCVzwritezyTLB39}{\saildocval{}{\lstinputlisting[language=sail]{sail_latex_riscv/valzwrite_tlb39d07866694d7288e6a3cb2ac08a58c288.tex}}}}

\newcommand{\sailRISCVfnwriteTLBThreeNine}{\saildoclabelled{sailRISCVfnzwritezyTLB39}{\saildocfn{}{\lstinputlisting[language=sail]{sail_latex_riscv/fnzwrite_tlb39d07866694d7288e6a3cb2ac08a58c288.tex}}}}

\newcommand{\sailRISCVvalflushTLBThreeNine}{\saildoclabelled{sailRISCVzflushzyTLB39}{\saildocval{}{\lstinputlisting[language=sail]{sail_latex_riscv/valzflush_tlb39155d346596461ed0e3ab611745b6738a.tex}}}}

\newcommand{\sailRISCVfnflushTLBThreeNine}{\saildoclabelled{sailRISCVfnzflushzyTLB39}{\saildocfn{}{\lstinputlisting[language=sail]{sail_latex_riscv/fnzflush_tlb39155d346596461ed0e3ab611745b6738a.tex}}}}

\newcommand{\sailRISCVvaltranslateThreeNine}{\saildoclabelled{sailRISCVztranslate39}{\saildocval{}{\lstinputlisting[language=sail]{sail_latex_riscv/valztranslate39daa42428c4ec23bd5def028158b476c6.tex}}}}

\newcommand{\sailRISCVfntranslateThreeNine}{\saildoclabelled{sailRISCVfnztranslate39}{\saildocfn{}{\lstinputlisting[language=sail]{sail_latex_riscv/fnztranslate39daa42428c4ec23bd5def028158b476c6.tex}}}}

\newcommand{\sailRISCVvalinitVmemSvThreeNine}{\saildoclabelled{sailRISCVzinitzyvmemzysv39}{\saildocval{}{\lstinputlisting[language=sail]{sail_latex_riscv/valzinit_vmem_sv390a83d23f548fedfc48c542ac764df587.tex}}}}

\newcommand{\sailRISCVfninitVmemSvThreeNine}{\saildoclabelled{sailRISCVfnzinitzyvmemzysv39}{\saildocfn{}{\lstinputlisting[language=sail]{sail_latex_riscv/fnzinit_vmem_sv390a83d23f548fedfc48c542ac764df587.tex}}}}

\newcommand{\sailRISCVvalwalkFourEight}{\saildoclabelled{sailRISCVzwalk48}{\saildocval{}{\lstinputlisting[language=sail]{sail_latex_riscv/valzwalk486c4033235ca01e713873e89320a939ac.tex}}}}

\newcommand{\sailRISCVfnwalkFourEight}{\saildoclabelled{sailRISCVfnzwalk48}{\saildocfn{}{\lstinputlisting[language=sail]{sail_latex_riscv/fnzwalk486c4033235ca01e713873e89320a939ac.tex}}}}

\newcommand{\sailRISCVtypeTLBFourEightEntry}{\saildoclabelled{sailRISCVtypezTLB48zyEntry}{\saildoctype{}{\lstinputlisting[language=sail]{sail_latex_riscv/typeztlb48_entry26abc9d10a3e265edd1da0902c832cee.tex}}}}

\newcommand{\sailRISCVregistertlbFourEight}{\saildoclabelled{sailRISCVregisterztlb48}{\saildocregister{}{\lstinputlisting[language=sail]{sail_latex_riscv/registerztlb48a3524f22d6ad4586a54b808f10a6e105.tex}}}}

\newcommand{\sailRISCVvallookupTLBFourEight}{\saildoclabelled{sailRISCVzlookupzyTLB48}{\saildocval{}{\lstinputlisting[language=sail]{sail_latex_riscv/valzlookup_tlb481be2085cbc29568c5c522b87bd854b70.tex}}}}

\newcommand{\sailRISCVfnlookupTLBFourEight}{\saildoclabelled{sailRISCVfnzlookupzyTLB48}{\saildocfn{}{\lstinputlisting[language=sail]{sail_latex_riscv/fnzlookup_tlb481be2085cbc29568c5c522b87bd854b70.tex}}}}

\newcommand{\sailRISCVvaladdToTLBFourEight}{\saildoclabelled{sailRISCVzaddzytozyTLB48}{\saildocval{}{\lstinputlisting[language=sail]{sail_latex_riscv/valzadd_to_tlb48a08cb99b91f9c80121525cc743b92593.tex}}}}

\newcommand{\sailRISCVfnaddToTLBFourEight}{\saildoclabelled{sailRISCVfnzaddzytozyTLB48}{\saildocfn{}{\lstinputlisting[language=sail]{sail_latex_riscv/fnzadd_to_tlb48a08cb99b91f9c80121525cc743b92593.tex}}}}

\newcommand{\sailRISCVvalwriteTLBFourEight}{\saildoclabelled{sailRISCVzwritezyTLB48}{\saildocval{}{\lstinputlisting[language=sail]{sail_latex_riscv/valzwrite_tlb48c2deeba6fb9156609619ba23dcc84e5a.tex}}}}

\newcommand{\sailRISCVfnwriteTLBFourEight}{\saildoclabelled{sailRISCVfnzwritezyTLB48}{\saildocfn{}{\lstinputlisting[language=sail]{sail_latex_riscv/fnzwrite_tlb48c2deeba6fb9156609619ba23dcc84e5a.tex}}}}

\newcommand{\sailRISCVvalflushTLBFourEight}{\saildoclabelled{sailRISCVzflushzyTLB48}{\saildocval{}{\lstinputlisting[language=sail]{sail_latex_riscv/valzflush_tlb482377e6a21b0e61c1f15d8ef64c6ac044.tex}}}}

\newcommand{\sailRISCVfnflushTLBFourEight}{\saildoclabelled{sailRISCVfnzflushzyTLB48}{\saildocfn{}{\lstinputlisting[language=sail]{sail_latex_riscv/fnzflush_tlb482377e6a21b0e61c1f15d8ef64c6ac044.tex}}}}

\newcommand{\sailRISCVvaltranslateFourEight}{\saildoclabelled{sailRISCVztranslate48}{\saildocval{}{\lstinputlisting[language=sail]{sail_latex_riscv/valztranslate488637bc30f662c37b22a80c3d053c14e5.tex}}}}

\newcommand{\sailRISCVfntranslateFourEight}{\saildoclabelled{sailRISCVfnztranslate48}{\saildocfn{}{\lstinputlisting[language=sail]{sail_latex_riscv/fnztranslate488637bc30f662c37b22a80c3d053c14e5.tex}}}}

\newcommand{\sailRISCVvalinitVmemSvFourEight}{\saildoclabelled{sailRISCVzinitzyvmemzysv48}{\saildocval{}{\lstinputlisting[language=sail]{sail_latex_riscv/valzinit_vmem_sv487afaa3d38e7e9ff21a6cbfad52504311.tex}}}}

\newcommand{\sailRISCVfninitVmemSvFourEight}{\saildoclabelled{sailRISCVfnzinitzyvmemzysv48}{\saildocfn{}{\lstinputlisting[language=sail]{sail_latex_riscv/fnzinit_vmem_sv487afaa3d38e7e9ff21a6cbfad52504311.tex}}}}

\newcommand{\sailRISCVregistersatp}{\saildoclabelled{sailRISCVregisterzsatp}{\saildocregister{}{\lstinputlisting[language=sail]{sail_latex_riscv/registerzsatp2234a3f325753d93675f367c04a81033.tex}}}}

\newcommand{\sailRISCVvallegalizzeSatp}{\saildoclabelled{sailRISCVzlegalizzezysatp}{\saildocval{}{\lstinputlisting[language=sail]{sail_latex_riscv/valzlegalizze_satp1fbfb541ef401311caafca983cb812d6.tex}}}}

\newcommand{\sailRISCVfnlegalizzeSatp}{\saildoclabelled{sailRISCVfnzlegalizzezysatp}{\saildocfn{}{\lstinputlisting[language=sail]{sail_latex_riscv/fnzlegalizze_satp1fbfb541ef401311caafca983cb812d6.tex}}}}

\newcommand{\sailRISCVvalisValidSvThreeNineAddr}{\saildoclabelled{sailRISCVzisValidSv39Addr}{\saildocval{}{\lstinputlisting[language=sail]{sail_latex_riscv/valzisvalidsv39addr6225e53e2f01610f83ab9bea1f3201bf.tex}}}}

\newcommand{\sailRISCVfnisValidSvThreeNineAddr}{\saildoclabelled{sailRISCVfnzisValidSv39Addr}{\saildocfn{}{\lstinputlisting[language=sail]{sail_latex_riscv/fnzisvalidsv39addr6225e53e2f01610f83ab9bea1f3201bf.tex}}}}

\newcommand{\sailRISCVvalisValidSvFourEightAddr}{\saildoclabelled{sailRISCVzisValidSv48Addr}{\saildocval{}{\lstinputlisting[language=sail]{sail_latex_riscv/valzisvalidsv48addr5c09db742963fd3c2cd1457f6411e837.tex}}}}

\newcommand{\sailRISCVfnisValidSvFourEightAddr}{\saildoclabelled{sailRISCVfnzisValidSv48Addr}{\saildocfn{}{\lstinputlisting[language=sail]{sail_latex_riscv/fnzisvalidsv48addr5c09db742963fd3c2cd1457f6411e837.tex}}}}

\newcommand{\sailRISCVvaltranslationMode}{\saildoclabelled{sailRISCVztranslationMode}{\saildocval{}{\lstinputlisting[language=sail]{sail_latex_riscv/valztranslationmode51f0fd652f39ec2f6d4c16847f0d4345.tex}}}}

\newcommand{\sailRISCVfntranslationMode}{\saildoclabelled{sailRISCVfnztranslationMode}{\saildocfn{}{\lstinputlisting[language=sail]{sail_latex_riscv/fnztranslationmode51f0fd652f39ec2f6d4c16847f0d4345.tex}}}}

\newcommand{\sailRISCVvaltranslateAddrPriv}{\saildoclabelled{sailRISCVztranslateAddrzypriv}{\saildocval{}{\lstinputlisting[language=sail]{sail_latex_riscv/valztranslateaddr_priv058a80c00bb7770a29f29e63a3ac342b.tex}}}}

\newcommand{\sailRISCVfntranslateAddrPriv}{\saildoclabelled{sailRISCVfnztranslateAddrzypriv}{\saildocfn{}{\lstinputlisting[language=sail]{sail_latex_riscv/fnztranslateaddr_priv058a80c00bb7770a29f29e63a3ac342b.tex}}}}

\newcommand{\sailRISCVvaltranslateAddr}{\saildoclabelled{sailRISCVztranslateAddr}{\saildocval{}{\lstinputlisting[language=sail]{sail_latex_riscv/valztranslateaddr7dc6bd4ea43d006224000f7b68f6a187.tex}}}}

\newcommand{\sailRISCVfntranslateAddr}{\saildoclabelled{sailRISCVfnztranslateAddr}{\saildocfn{}{\lstinputlisting[language=sail]{sail_latex_riscv/fnztranslateaddr7dc6bd4ea43d006224000f7b68f6a187.tex}}}}

\newcommand{\sailRISCVvalflushTLB}{\saildoclabelled{sailRISCVzflushzyTLB}{\saildocval{}{\lstinputlisting[language=sail]{sail_latex_riscv/valzflush_tlbf2c831dee428b5971141e383ef962e36.tex}}}}

\newcommand{\sailRISCVfnflushTLB}{\saildoclabelled{sailRISCVfnzflushzyTLB}{\saildocfn{}{\lstinputlisting[language=sail]{sail_latex_riscv/fnzflush_tlbf2c831dee428b5971141e383ef962e36.tex}}}}

\newcommand{\sailRISCVvalinitVmem}{\saildoclabelled{sailRISCVzinitzyvmem}{\saildocval{}{\lstinputlisting[language=sail]{sail_latex_riscv/valzinit_vmem811d98ebf1d4d536d0e4070a3b67fe03.tex}}}}

\newcommand{\sailRISCVfninitVmem}{\saildoclabelled{sailRISCVfnzinitzyvmem}{\saildocfn{}{\lstinputlisting[language=sail]{sail_latex_riscv/fnzinit_vmem811d98ebf1d4d536d0e4070a3b67fe03.tex}}}}

\newcommand{\sailRISCVtypeast}{\saildoclabelled{sailRISCVtypezast}{\saildoctype{}{\lstinputlisting[language=sail]{sail_latex_riscv/typezast6bb070d12e82e4887160cdfd016230c8.tex}}}}

\newcommand{\sailRISCVvalexecute}{\saildoclabelled{sailRISCVzexecute}{\saildocval{}{\lstinputlisting[language=sail]{sail_latex_riscv/valzexecute33a689e3a631b9b905b85461d3814943.tex}}}}

\newcommand{\sailRISCVvalassembly}{\saildoclabelled{sailRISCVzassembly}{\saildocval{}{\lstinputlisting[language=sail]{sail_latex_riscv/valzassembly6c256353098ca1294b0a3873338d670c.tex}}}}

\newcommand{\sailRISCVvalencdec}{\saildoclabelled{sailRISCVzencdec}{\saildocval{}{\lstinputlisting[language=sail]{sail_latex_riscv/valzencdeca7ceb0009f9b533ad47d2b69e8881c04.tex}}}}

\newcommand{\sailRISCVvalencdecCompressed}{\saildoclabelled{sailRISCVzencdeczycompressed}{\saildocval{}{\lstinputlisting[language=sail]{sail_latex_riscv/valzencdec_compressed480f14a33b2969971592c7ca63bcfde9.tex}}}}

\newcommand{\sailRISCVvalencdecUop}{\saildoclabelled{sailRISCVzencdeczyuop}{\saildocval{}{\lstinputlisting[language=sail]{sail_latex_riscv/valzencdec_uop0feeafb72397448d1e686117bd04bd8d.tex}}}}

\newcommand{\sailRISCVvalutypeMnemonic}{\saildoclabelled{sailRISCVzutypezymnemonic}{\saildocval{}{\lstinputlisting[language=sail]{sail_latex_riscv/valzutype_mnemonic5740211feaadc8d830fd698383ea27eb.tex}}}}

\newcommand{\sailRISCVvalencdecBop}{\saildoclabelled{sailRISCVzencdeczybop}{\saildocval{}{\lstinputlisting[language=sail]{sail_latex_riscv/valzencdec_bop37935afdc8fb9d403964a671d7b8ef6f.tex}}}}

\newcommand{\sailRISCVvalbtypeMnemonic}{\saildoclabelled{sailRISCVzbtypezymnemonic}{\saildocval{}{\lstinputlisting[language=sail]{sail_latex_riscv/valzbtype_mnemonicee542cf061481e1e2c0e4ec1302a928b.tex}}}}

\newcommand{\sailRISCVvalencdecIop}{\saildoclabelled{sailRISCVzencdeczyiop}{\saildocval{}{\lstinputlisting[language=sail]{sail_latex_riscv/valzencdec_iopdc6254ab3f3dfdad4df376ed9499c048.tex}}}}

\newcommand{\sailRISCVvalitypeMnemonic}{\saildoclabelled{sailRISCVzitypezymnemonic}{\saildocval{}{\lstinputlisting[language=sail]{sail_latex_riscv/valzitype_mnemonicb2266da97e58b5824550ef451a0000db.tex}}}}

\newcommand{\sailRISCVvalencdecSop}{\saildoclabelled{sailRISCVzencdeczysop}{\saildocval{}{\lstinputlisting[language=sail]{sail_latex_riscv/valzencdec_sop073a7abc17ca60aa3455773907eb78c7.tex}}}}

\newcommand{\sailRISCVvalshiftiopMnemonic}{\saildoclabelled{sailRISCVzshiftiopzymnemonic}{\saildocval{}{\lstinputlisting[language=sail]{sail_latex_riscv/valzshiftiop_mnemonicbeafdfe24bd90ed73f232559498ac819.tex}}}}

\newcommand{\sailRISCVvalrtypeMnemonic}{\saildoclabelled{sailRISCVzrtypezymnemonic}{\saildocval{}{\lstinputlisting[language=sail]{sail_latex_riscv/valzrtype_mnemonic539b9ad883876e5c6be9237e5c98ffbb.tex}}}}

\newcommand{\sailRISCVvalextendValue}{\saildoclabelled{sailRISCVzextendzyvalue}{\saildocval{}{\lstinputlisting[language=sail]{sail_latex_riscv/valzextend_value8ddb26f3f92f6848beaff0fbcaa992f6.tex}}}}

\newcommand{\sailRISCVfnextendValue}{\saildoclabelled{sailRISCVfnzextendzyvalue}{\saildocfn{}{\lstinputlisting[language=sail]{sail_latex_riscv/fnzextend_value8ddb26f3f92f6848beaff0fbcaa992f6.tex}}}}

\newcommand{\sailRISCVvalprocessLoad}{\saildoclabelled{sailRISCVzprocesszyload}{\saildocval{}{\lstinputlisting[language=sail]{sail_latex_riscv/valzprocess_load7d9288eb90dd41d1aa3c47eda679c483.tex}}}}

\newcommand{\sailRISCVfnprocessLoad}{\saildoclabelled{sailRISCVfnzprocesszyload}{\saildocfn{}{\lstinputlisting[language=sail]{sail_latex_riscv/fnzprocess_load7d9288eb90dd41d1aa3c47eda679c483.tex}}}}

\newcommand{\sailRISCVvalcheckMisaligned}{\saildoclabelled{sailRISCVzcheckzymisaligned}{\saildocval{}{\lstinputlisting[language=sail]{sail_latex_riscv/valzcheck_misaligned6730e75ccee79325a38a992b6314fd91.tex}}}}

\newcommand{\sailRISCVfncheckMisaligned}{\saildoclabelled{sailRISCVfnzcheckzymisaligned}{\saildocfn{}{\lstinputlisting[language=sail]{sail_latex_riscv/fnzcheck_misaligned6730e75ccee79325a38a992b6314fd91.tex}}}}

\newcommand{\sailRISCVvalmaybeAq}{\saildoclabelled{sailRISCVzmaybezyaq}{\saildocval{}{\lstinputlisting[language=sail]{sail_latex_riscv/valzmaybe_aq7807a00a0c8e402132f155781a4acc6b.tex}}}}

\newcommand{\sailRISCVvalmaybeRl}{\saildoclabelled{sailRISCVzmaybezyrl}{\saildocval{}{\lstinputlisting[language=sail]{sail_latex_riscv/valzmaybe_rl471fbde2d5b588b1be3bdbcbad1b5b40.tex}}}}

\newcommand{\sailRISCVvalmaybeU}{\saildoclabelled{sailRISCVzmaybezyu}{\saildocval{}{\lstinputlisting[language=sail]{sail_latex_riscv/valzmaybe_u271e5e54fd4c9b9e611d3ad0e98a3503.tex}}}}

\newcommand{\sailRISCVvalshiftwMnemonic}{\saildoclabelled{sailRISCVzshiftwzymnemonic}{\saildocval{}{\lstinputlisting[language=sail]{sail_latex_riscv/valzshiftw_mnemonice823d8fe4e9165665d0ca244aa353baa.tex}}}}

\newcommand{\sailRISCVvalrtypewMnemonic}{\saildoclabelled{sailRISCVzrtypewzymnemonic}{\saildocval{}{\lstinputlisting[language=sail]{sail_latex_riscv/valzrtypew_mnemonic72f42d5e018398fdfceb25edf2e10caf.tex}}}}

\newcommand{\sailRISCVvalshiftiwopMnemonic}{\saildoclabelled{sailRISCVzshiftiwopzymnemonic}{\saildocval{}{\lstinputlisting[language=sail]{sail_latex_riscv/valzshiftiwop_mnemonic59f60b5dc1f36ebc6fe3d32d9df1d608.tex}}}}

\newcommand{\sailRISCVvalbitMaybeR}{\saildoclabelled{sailRISCVzbitzymaybezyr}{\saildocval{}{\lstinputlisting[language=sail]{sail_latex_riscv/valzbit_maybe_r1c52279a1272ff324e99d5b1b65881cd.tex}}}}

\newcommand{\sailRISCVvalbitMaybeW}{\saildoclabelled{sailRISCVzbitzymaybezyw}{\saildocval{}{\lstinputlisting[language=sail]{sail_latex_riscv/valzbit_maybe_w6d81c7531ae5006f39930eecd7114080.tex}}}}

\newcommand{\sailRISCVvalbitMaybeI}{\saildoclabelled{sailRISCVzbitzymaybezyi}{\saildocval{}{\lstinputlisting[language=sail]{sail_latex_riscv/valzbit_maybe_iea1ce4e78632791b6873db323516744a.tex}}}}

\newcommand{\sailRISCVvalbitMaybeO}{\saildoclabelled{sailRISCVzbitzymaybezyo}{\saildocval{}{\lstinputlisting[language=sail]{sail_latex_riscv/valzbit_maybe_oa643ed77970ec3375ca02eb2a3d6d7e3.tex}}}}

\newcommand{\sailRISCVvalfenceBits}{\saildoclabelled{sailRISCVzfencezybits}{\saildocval{}{\lstinputlisting[language=sail]{sail_latex_riscv/valzfence_bits39f0871d2c2e4d36bb4622567377397d.tex}}}}

\newcommand{\sailRISCVvalaqrlStr}{\saildoclabelled{sailRISCVzaqrlzystr}{\saildocval{}{\lstinputlisting[language=sail]{sail_latex_riscv/valzaqrl_str43f7a950ecb4ae033c3d54ba744ac285.tex}}}}

\newcommand{\sailRISCVfnaqrlStr}{\saildoclabelled{sailRISCVfnzaqrlzystr}{\saildocfn{}{\lstinputlisting[language=sail]{sail_latex_riscv/fnzaqrl_str43f7a950ecb4ae033c3d54ba744ac285.tex}}}}

\newcommand{\sailRISCVvallrscWidthStr}{\saildoclabelled{sailRISCVzlrsczywidthzystr}{\saildocval{}{\lstinputlisting[language=sail]{sail_latex_riscv/valzlrsc_width_str1a32b7e5b18a83477d13203cd0eca601.tex}}}}

\newcommand{\sailRISCVfnlrscWidthStr}{\saildoclabelled{sailRISCVfnzlrsczywidthzystr}{\saildocfn{}{\lstinputlisting[language=sail]{sail_latex_riscv/fnzlrsc_width_str1a32b7e5b18a83477d13203cd0eca601.tex}}}}

\newcommand{\sailRISCVvalamoWidthValid}{\saildoclabelled{sailRISCVzamozywidthzyvalid}{\saildocval{}{\lstinputlisting[language=sail]{sail_latex_riscv/valzamo_width_validcc8985d6c05c87b227cb9f46a588a56a.tex}}}}

\newcommand{\sailRISCVfnamoWidthValid}{\saildoclabelled{sailRISCVfnzamozywidthzyvalid}{\saildocfn{}{\lstinputlisting[language=sail]{sail_latex_riscv/fnzamo_width_validcc8985d6c05c87b227cb9f46a588a56a.tex}}}}

\newcommand{\sailRISCVvalprocessLoadres}{\saildoclabelled{sailRISCVzprocesszyloadres}{\saildocval{}{\lstinputlisting[language=sail]{sail_latex_riscv/valzprocess_loadres3acadfd67bd540642036cf41405a27c0.tex}}}}

\newcommand{\sailRISCVfnprocessLoadres}{\saildoclabelled{sailRISCVfnzprocesszyloadres}{\saildocfn{}{\lstinputlisting[language=sail]{sail_latex_riscv/fnzprocess_loadres3acadfd67bd540642036cf41405a27c0.tex}}}}

\newcommand{\sailRISCVvalencdecAmoop}{\saildoclabelled{sailRISCVzencdeczyamoop}{\saildocval{}{\lstinputlisting[language=sail]{sail_latex_riscv/valzencdec_amoopad56ea38f11b2d2533d3cea1e6e89079.tex}}}}

\newcommand{\sailRISCVvalamoMnemonic}{\saildoclabelled{sailRISCVzamozymnemonic}{\saildocval{}{\lstinputlisting[language=sail]{sail_latex_riscv/valzamo_mnemonicac3b9dc5cf93b937e8a5514efa62f568.tex}}}}

\newcommand{\sailRISCVvalencdecMulOp}{\saildoclabelled{sailRISCVzencdeczymulzyop}{\saildocval{}{\lstinputlisting[language=sail]{sail_latex_riscv/valzencdec_mul_op4fddc1c61135e80f6618dd3c6fca770c.tex}}}}

\newcommand{\sailRISCVvalmulMnemonic}{\saildoclabelled{sailRISCVzmulzymnemonic}{\saildocval{}{\lstinputlisting[language=sail]{sail_latex_riscv/valzmul_mnemonic8e8a5c0cf101d49d6656287f35956c53.tex}}}}

\newcommand{\sailRISCVvalmaybeNotU}{\saildoclabelled{sailRISCVzmaybezynotzyu}{\saildocval{}{\lstinputlisting[language=sail]{sail_latex_riscv/valzmaybe_not_ud1aa0681e785549479247e44134e0663.tex}}}}

\newcommand{\sailRISCVvalencdecCsrop}{\saildoclabelled{sailRISCVzencdeczycsrop}{\saildocval{}{\lstinputlisting[language=sail]{sail_latex_riscv/valzencdec_csrop402749395fac229088b2e6f6e6206c72.tex}}}}

\newcommand{\sailRISCVvalreadCSR}{\saildoclabelled{sailRISCVzreadCSR}{\saildocval{}{\lstinputlisting[language=sail]{sail_latex_riscv/valzreadcsr1a9ed1f2dac4690038fbe34a4617edca.tex}}}}

\newcommand{\sailRISCVfnreadCSR}{\saildoclabelled{sailRISCVfnzreadCSR}{\saildocfn{}{\lstinputlisting[language=sail]{sail_latex_riscv/fnzreadcsr1a9ed1f2dac4690038fbe34a4617edca.tex}}}}

\newcommand{\sailRISCVvalwriteCSR}{\saildoclabelled{sailRISCVzwriteCSR}{\saildocval{}{\lstinputlisting[language=sail]{sail_latex_riscv/valzwritecsr7af48520171f4dd0cd06c1b6876196a7.tex}}}}

\newcommand{\sailRISCVfnwriteCSR}{\saildoclabelled{sailRISCVfnzwriteCSR}{\saildocfn{}{\lstinputlisting[language=sail]{sail_latex_riscv/fnzwritecsr7af48520171f4dd0cd06c1b6876196a7.tex}}}}

\newcommand{\sailRISCVvalmaybeI}{\saildoclabelled{sailRISCVzmaybezyi}{\saildocval{}{\lstinputlisting[language=sail]{sail_latex_riscv/valzmaybe_i65644ebcc11e5b25fca853fb9aeea917.tex}}}}

\newcommand{\sailRISCVvalcsrMnemonic}{\saildoclabelled{sailRISCVzcsrzymnemonic}{\saildocval{}{\lstinputlisting[language=sail]{sail_latex_riscv/valzcsr_mnemonic3a1dd26f3ac0095deadcf2bffb0adbc8.tex}}}}

\newcommand{\sailRISCVvalencdecRoundingMode}{\saildoclabelled{sailRISCVzencdeczyroundingzymode}{\saildocval{}{\lstinputlisting[language=sail]{sail_latex_riscv/valzencdec_rounding_mode4e9de2b381b2971c0047c4587dc1aeff.tex}}}}

\newcommand{\sailRISCVvalfrmMnemonic}{\saildoclabelled{sailRISCVzfrmzymnemonic}{\saildocval{}{\lstinputlisting[language=sail]{sail_latex_riscv/valzfrm_mnemonic1efa3419adbb7f3793e39728661edbd6.tex}}}}

\newcommand{\sailRISCVvalvalidRoundingMode}{\saildoclabelled{sailRISCVzvalidzyroundingzymode}{\saildocval{}{\lstinputlisting[language=sail]{sail_latex_riscv/valzvalid_rounding_mode99914dcf92c0a4b7285f2212ad9e630f.tex}}}}

\newcommand{\sailRISCVfnvalidRoundingMode}{\saildoclabelled{sailRISCVfnzvalidzyroundingzymode}{\saildocfn{}{\lstinputlisting[language=sail]{sail_latex_riscv/fnzvalid_rounding_mode99914dcf92c0a4b7285f2212ad9e630f.tex}}}}

\newcommand{\sailRISCVvalselectInstrOrFcsrRm}{\saildoclabelled{sailRISCVzselectzyinstrzyorzyfcsrzyrm}{\saildocval{}{\lstinputlisting[language=sail]{sail_latex_riscv/valzselect_instr_or_fcsr_rm8a489b01095486a7dc41b6fb5b17b9a7.tex}}}}

\newcommand{\sailRISCVfnselectInstrOrFcsrRm}{\saildoclabelled{sailRISCVfnzselectzyinstrzyorzyfcsrzyrm}{\saildocfn{}{\lstinputlisting[language=sail]{sail_latex_riscv/fnzselect_instr_or_fcsr_rm8a489b01095486a7dc41b6fb5b17b9a7.tex}}}}

\newcommand{\sailRISCVvalnxFlag}{\saildoclabelled{sailRISCVznxFlag}{\saildocval{}{\lstinputlisting[language=sail]{sail_latex_riscv/valznxflaga9ef0edc4edc03b79abaa230aaad20a6.tex}}}}

\newcommand{\sailRISCVfnnxFlag}{\saildoclabelled{sailRISCVfnznxFlag}{\saildocfn{}{\lstinputlisting[language=sail]{sail_latex_riscv/fnznxflaga9ef0edc4edc03b79abaa230aaad20a6.tex}}}}

\newcommand{\sailRISCVvalufFlag}{\saildoclabelled{sailRISCVzufFlag}{\saildocval{}{\lstinputlisting[language=sail]{sail_latex_riscv/valzufflagb3d552180ae10d9ed1b23a8e10ad2e9c.tex}}}}

\newcommand{\sailRISCVfnufFlag}{\saildoclabelled{sailRISCVfnzufFlag}{\saildocfn{}{\lstinputlisting[language=sail]{sail_latex_riscv/fnzufflagb3d552180ae10d9ed1b23a8e10ad2e9c.tex}}}}

\newcommand{\sailRISCVvalofFlag}{\saildoclabelled{sailRISCVzofFlag}{\saildocval{}{\lstinputlisting[language=sail]{sail_latex_riscv/valzofflag1be1156688e569fa940a9118708be17d.tex}}}}

\newcommand{\sailRISCVfnofFlag}{\saildoclabelled{sailRISCVfnzofFlag}{\saildocfn{}{\lstinputlisting[language=sail]{sail_latex_riscv/fnzofflag1be1156688e569fa940a9118708be17d.tex}}}}

\newcommand{\sailRISCVvaldzzFlag}{\saildoclabelled{sailRISCVzdzzFlag}{\saildocval{}{\lstinputlisting[language=sail]{sail_latex_riscv/valzdzzflagcea95ae2581be09607c095ea1558f21b.tex}}}}

\newcommand{\sailRISCVfndzzFlag}{\saildoclabelled{sailRISCVfnzdzzFlag}{\saildocfn{}{\lstinputlisting[language=sail]{sail_latex_riscv/fnzdzzflagcea95ae2581be09607c095ea1558f21b.tex}}}}

\newcommand{\sailRISCVvalnvFlag}{\saildoclabelled{sailRISCVznvFlag}{\saildocval{}{\lstinputlisting[language=sail]{sail_latex_riscv/valznvflag96f5e3b2efdbb80ea87eb99f361ac158.tex}}}}

\newcommand{\sailRISCVfnnvFlag}{\saildoclabelled{sailRISCVfnznvFlag}{\saildocfn{}{\lstinputlisting[language=sail]{sail_latex_riscv/fnznvflag96f5e3b2efdbb80ea87eb99f361ac158.tex}}}}

\newcommand{\sailRISCVvalfsplitS}{\saildoclabelled{sailRISCVzfsplitzyS}{\saildocval{}{\lstinputlisting[language=sail]{sail_latex_riscv/valzfsplit_s587b8f3a581d4f6ab15969163c75b5ff.tex}}}}

\newcommand{\sailRISCVfnfsplitS}{\saildoclabelled{sailRISCVfnzfsplitzyS}{\saildocfn{}{\lstinputlisting[language=sail]{sail_latex_riscv/fnzfsplit_s587b8f3a581d4f6ab15969163c75b5ff.tex}}}}

\newcommand{\sailRISCVvalfmakeS}{\saildoclabelled{sailRISCVzfmakezyS}{\saildocval{}{\lstinputlisting[language=sail]{sail_latex_riscv/valzfmake_sf09ffe3c3e52a17eb1e248d73c150f66.tex}}}}

\newcommand{\sailRISCVfnfmakeS}{\saildoclabelled{sailRISCVfnzfmakezyS}{\saildocfn{}{\lstinputlisting[language=sail]{sail_latex_riscv/fnzfmake_sf09ffe3c3e52a17eb1e248d73c150f66.tex}}}}

\newcommand{\sailRISCVvalfIsNegInfS}{\saildoclabelled{sailRISCVzfzyiszynegzyinfzyS}{\saildocval{}{\lstinputlisting[language=sail]{sail_latex_riscv/valzf_is_neg_inf_sf1e927a0ea24a891ce2c85c8d22d9613.tex}}}}

\newcommand{\sailRISCVfnfIsNegInfS}{\saildoclabelled{sailRISCVfnzfzyiszynegzyinfzyS}{\saildocfn{}{\lstinputlisting[language=sail]{sail_latex_riscv/fnzf_is_neg_inf_sf1e927a0ea24a891ce2c85c8d22d9613.tex}}}}

\newcommand{\sailRISCVvalfIsNegNormS}{\saildoclabelled{sailRISCVzfzyiszynegzynormzyS}{\saildocval{}{\lstinputlisting[language=sail]{sail_latex_riscv/valzf_is_neg_norm_s8bfccfb981e547ad5f5c42e4a17c2a1f.tex}}}}

\newcommand{\sailRISCVfnfIsNegNormS}{\saildoclabelled{sailRISCVfnzfzyiszynegzynormzyS}{\saildocfn{}{\lstinputlisting[language=sail]{sail_latex_riscv/fnzf_is_neg_norm_s8bfccfb981e547ad5f5c42e4a17c2a1f.tex}}}}

\newcommand{\sailRISCVvalfIsNegSubnormS}{\saildoclabelled{sailRISCVzfzyiszynegzysubnormzyS}{\saildocval{}{\lstinputlisting[language=sail]{sail_latex_riscv/valzf_is_neg_subnorm_sd8c8d47f6284f9e99142031238418d6e.tex}}}}

\newcommand{\sailRISCVfnfIsNegSubnormS}{\saildoclabelled{sailRISCVfnzfzyiszynegzysubnormzyS}{\saildocfn{}{\lstinputlisting[language=sail]{sail_latex_riscv/fnzf_is_neg_subnorm_sd8c8d47f6284f9e99142031238418d6e.tex}}}}

\newcommand{\sailRISCVvalfIsNegZeroS}{\saildoclabelled{sailRISCVzfzyiszynegzyzzerozyS}{\saildocval{}{\lstinputlisting[language=sail]{sail_latex_riscv/valzf_is_neg_zzero_s01b2d7b1def55db2428888908b29c669.tex}}}}

\newcommand{\sailRISCVfnfIsNegZeroS}{\saildoclabelled{sailRISCVfnzfzyiszynegzyzzerozyS}{\saildocfn{}{\lstinputlisting[language=sail]{sail_latex_riscv/fnzf_is_neg_zzero_s01b2d7b1def55db2428888908b29c669.tex}}}}

\newcommand{\sailRISCVvalfIsPosZeroS}{\saildoclabelled{sailRISCVzfzyiszyposzyzzerozyS}{\saildocval{}{\lstinputlisting[language=sail]{sail_latex_riscv/valzf_is_pos_zzero_s92eacf7245b38cd8fbe2b7411cc24794.tex}}}}

\newcommand{\sailRISCVfnfIsPosZeroS}{\saildoclabelled{sailRISCVfnzfzyiszyposzyzzerozyS}{\saildocfn{}{\lstinputlisting[language=sail]{sail_latex_riscv/fnzf_is_pos_zzero_s92eacf7245b38cd8fbe2b7411cc24794.tex}}}}

\newcommand{\sailRISCVvalfIsPosSubnormS}{\saildoclabelled{sailRISCVzfzyiszyposzysubnormzyS}{\saildocval{}{\lstinputlisting[language=sail]{sail_latex_riscv/valzf_is_pos_subnorm_s4efdacb98629e85ea864b456e8377a98.tex}}}}

\newcommand{\sailRISCVfnfIsPosSubnormS}{\saildoclabelled{sailRISCVfnzfzyiszyposzysubnormzyS}{\saildocfn{}{\lstinputlisting[language=sail]{sail_latex_riscv/fnzf_is_pos_subnorm_s4efdacb98629e85ea864b456e8377a98.tex}}}}

\newcommand{\sailRISCVvalfIsPosNormS}{\saildoclabelled{sailRISCVzfzyiszyposzynormzyS}{\saildocval{}{\lstinputlisting[language=sail]{sail_latex_riscv/valzf_is_pos_norm_s19872015fda671aa1dca05b90a2680c4.tex}}}}

\newcommand{\sailRISCVfnfIsPosNormS}{\saildoclabelled{sailRISCVfnzfzyiszyposzynormzyS}{\saildocfn{}{\lstinputlisting[language=sail]{sail_latex_riscv/fnzf_is_pos_norm_s19872015fda671aa1dca05b90a2680c4.tex}}}}

\newcommand{\sailRISCVvalfIsPosInfS}{\saildoclabelled{sailRISCVzfzyiszyposzyinfzyS}{\saildocval{}{\lstinputlisting[language=sail]{sail_latex_riscv/valzf_is_pos_inf_sf714fc249168edfd360262aca5e55c4d.tex}}}}

\newcommand{\sailRISCVfnfIsPosInfS}{\saildoclabelled{sailRISCVfnzfzyiszyposzyinfzyS}{\saildocfn{}{\lstinputlisting[language=sail]{sail_latex_riscv/fnzf_is_pos_inf_sf714fc249168edfd360262aca5e55c4d.tex}}}}

\newcommand{\sailRISCVvalfIsSNaNS}{\saildoclabelled{sailRISCVzfzyiszySNaNzyS}{\saildocval{}{\lstinputlisting[language=sail]{sail_latex_riscv/valzf_is_snan_s4f74c8b27b066b5e2d6856d63c0d8113.tex}}}}

\newcommand{\sailRISCVfnfIsSNaNS}{\saildoclabelled{sailRISCVfnzfzyiszySNaNzyS}{\saildocfn{}{\lstinputlisting[language=sail]{sail_latex_riscv/fnzf_is_snan_s4f74c8b27b066b5e2d6856d63c0d8113.tex}}}}

\newcommand{\sailRISCVvalfIsQNaNS}{\saildoclabelled{sailRISCVzfzyiszyQNaNzyS}{\saildocval{}{\lstinputlisting[language=sail]{sail_latex_riscv/valzf_is_qnan_s36343ea7477dd88c6d9f512ad8587fd3.tex}}}}

\newcommand{\sailRISCVfnfIsQNaNS}{\saildoclabelled{sailRISCVfnzfzyiszyQNaNzyS}{\saildocfn{}{\lstinputlisting[language=sail]{sail_latex_riscv/fnzf_is_qnan_s36343ea7477dd88c6d9f512ad8587fd3.tex}}}}

\newcommand{\sailRISCVvalfIsNaNS}{\saildoclabelled{sailRISCVzfzyiszyNaNzyS}{\saildocval{}{\lstinputlisting[language=sail]{sail_latex_riscv/valzf_is_nan_s0d165c8ab9495379c496e3667f21236f.tex}}}}

\newcommand{\sailRISCVfnfIsNaNS}{\saildoclabelled{sailRISCVfnzfzyiszyNaNzyS}{\saildocfn{}{\lstinputlisting[language=sail]{sail_latex_riscv/fnzf_is_nan_s0d165c8ab9495379c496e3667f21236f.tex}}}}

\newcommand{\sailRISCVvalnegateS}{\saildoclabelled{sailRISCVznegatezyS}{\saildocval{}{\lstinputlisting[language=sail]{sail_latex_riscv/valznegate_s602acdc6547e76adb79aa6072014fa3e.tex}}}}

\newcommand{\sailRISCVfnnegateS}{\saildoclabelled{sailRISCVfnznegatezyS}{\saildocfn{}{\lstinputlisting[language=sail]{sail_latex_riscv/fnznegate_s602acdc6547e76adb79aa6072014fa3e.tex}}}}

\newcommand{\sailRISCVvalfeqQuietS}{\saildoclabelled{sailRISCVzfeqzyquietzyS}{\saildocval{}{\lstinputlisting[language=sail]{sail_latex_riscv/valzfeq_quiet_saed522b017dd40dd277d80aee28e4fe9.tex}}}}

\newcommand{\sailRISCVfnfeqQuietS}{\saildoclabelled{sailRISCVfnzfeqzyquietzyS}{\saildocfn{}{\lstinputlisting[language=sail]{sail_latex_riscv/fnzfeq_quiet_saed522b017dd40dd277d80aee28e4fe9.tex}}}}

\newcommand{\sailRISCVvalfltS}{\saildoclabelled{sailRISCVzfltzyS}{\saildocval{}{\lstinputlisting[language=sail]{sail_latex_riscv/valzflt_sb42cf5853deeee1a9c0ed2ff1e36a52b.tex}}}}

\newcommand{\sailRISCVfnfltS}{\saildoclabelled{sailRISCVfnzfltzyS}{\saildocfn{}{\lstinputlisting[language=sail]{sail_latex_riscv/fnzflt_sb42cf5853deeee1a9c0ed2ff1e36a52b.tex}}}}

\newcommand{\sailRISCVvalfleS}{\saildoclabelled{sailRISCVzflezyS}{\saildocval{}{\lstinputlisting[language=sail]{sail_latex_riscv/valzfle_s93e035a8ca13e7d64b965c7c750ae56d.tex}}}}

\newcommand{\sailRISCVfnfleS}{\saildoclabelled{sailRISCVfnzflezyS}{\saildocfn{}{\lstinputlisting[language=sail]{sail_latex_riscv/fnzfle_s93e035a8ca13e7d64b965c7c750ae56d.tex}}}}

\newcommand{\sailRISCVvalhaveSingleFPU}{\saildoclabelled{sailRISCVzhaveSingleFPU}{\saildocval{}{\lstinputlisting[language=sail]{sail_latex_riscv/valzhavesinglefpu440df2159eb5f81db4a3e22825aa5674.tex}}}}

\newcommand{\sailRISCVfnhaveSingleFPU}{\saildoclabelled{sailRISCVfnzhaveSingleFPU}{\saildocfn{}{\lstinputlisting[language=sail]{sail_latex_riscv/fnzhavesinglefpu440df2159eb5f81db4a3e22825aa5674.tex}}}}

\newcommand{\sailRISCVvalprocessFloadSixFour}{\saildoclabelled{sailRISCVzprocesszyfload64}{\saildocval{}{\lstinputlisting[language=sail]{sail_latex_riscv/valzprocess_fload64aaf81fa4052296eb2993198993f6472c.tex}}}}

\newcommand{\sailRISCVfnprocessFloadSixFour}{\saildoclabelled{sailRISCVfnzprocesszyfload64}{\saildocfn{}{\lstinputlisting[language=sail]{sail_latex_riscv/fnzprocess_fload64aaf81fa4052296eb2993198993f6472c.tex}}}}

\newcommand{\sailRISCVvalprocessFloadThreeTwo}{\saildoclabelled{sailRISCVzprocesszyfload32}{\saildocval{}{\lstinputlisting[language=sail]{sail_latex_riscv/valzprocess_fload322b1b2657ee5e1571b2af7b0d3a6631b7.tex}}}}

\newcommand{\sailRISCVfnprocessFloadThreeTwo}{\saildoclabelled{sailRISCVfnzprocesszyfload32}{\saildocfn{}{\lstinputlisting[language=sail]{sail_latex_riscv/fnzprocess_fload322b1b2657ee5e1571b2af7b0d3a6631b7.tex}}}}

\newcommand{\sailRISCVvalprocessFloadOneSix}{\saildoclabelled{sailRISCVzprocesszyfload16}{\saildocval{}{\lstinputlisting[language=sail]{sail_latex_riscv/valzprocess_fload16a3bf21b336eb480800c836d0e811ce78.tex}}}}

\newcommand{\sailRISCVfnprocessFloadOneSix}{\saildoclabelled{sailRISCVfnzprocesszyfload16}{\saildocfn{}{\lstinputlisting[language=sail]{sail_latex_riscv/fnzprocess_fload16a3bf21b336eb480800c836d0e811ce78.tex}}}}

\newcommand{\sailRISCVvalprocessFstore}{\saildoclabelled{sailRISCVzprocesszyfstore}{\saildocval{}{\lstinputlisting[language=sail]{sail_latex_riscv/valzprocess_fstoree9440c013cfdcc23312bc61d5762f1d1.tex}}}}

\newcommand{\sailRISCVfnprocessFstore}{\saildoclabelled{sailRISCVfnzprocesszyfstore}{\saildocfn{}{\lstinputlisting[language=sail]{sail_latex_riscv/fnzprocess_fstoree9440c013cfdcc23312bc61d5762f1d1.tex}}}}

\newcommand{\sailRISCVvalfMaddTypeMnemonicS}{\saildoclabelled{sailRISCVzfzymaddzytypezymnemoniczyS}{\saildocval{}{\lstinputlisting[language=sail]{sail_latex_riscv/valzf_madd_type_mnemonic_se8cab142783011d68d65bf9f55ceaf5f.tex}}}}

\newcommand{\sailRISCVvalfBinRmTypeMnemonicS}{\saildoclabelled{sailRISCVzfzybinzyrmzytypezymnemoniczyS}{\saildocval{}{\lstinputlisting[language=sail]{sail_latex_riscv/valzf_bin_rm_type_mnemonic_s2d0dda4f6d202d3b11b80b4a78a91df1.tex}}}}

\newcommand{\sailRISCVvalfUnRmTypeMnemonicS}{\saildoclabelled{sailRISCVzfzyunzyrmzytypezymnemoniczyS}{\saildocval{}{\lstinputlisting[language=sail]{sail_latex_riscv/valzf_un_rm_type_mnemonic_s24ea00e2ecedf00b96871ba799645a81.tex}}}}

\newcommand{\sailRISCVvalfBinTypeMnemonicS}{\saildoclabelled{sailRISCVzfzybinzytypezymnemoniczyS}{\saildocval{}{\lstinputlisting[language=sail]{sail_latex_riscv/valzf_bin_type_mnemonic_s7707fb27bf62e8e8783b4425caefe10c.tex}}}}

\newcommand{\sailRISCVvalfUnTypeMnemonicS}{\saildoclabelled{sailRISCVzfzyunzytypezymnemoniczyS}{\saildocval{}{\lstinputlisting[language=sail]{sail_latex_riscv/valzf_un_type_mnemonic_s68f8c6b309ddb2dc8d9cab3341f41b29.tex}}}}

\newcommand{\sailRISCVvalfsplitD}{\saildoclabelled{sailRISCVzfsplitzyD}{\saildocval{}{\lstinputlisting[language=sail]{sail_latex_riscv/valzfsplit_d774cb8ca3d70fb4590f5725c3fd93ad0.tex}}}}

\newcommand{\sailRISCVfnfsplitD}{\saildoclabelled{sailRISCVfnzfsplitzyD}{\saildocfn{}{\lstinputlisting[language=sail]{sail_latex_riscv/fnzfsplit_d774cb8ca3d70fb4590f5725c3fd93ad0.tex}}}}

\newcommand{\sailRISCVvalfmakeD}{\saildoclabelled{sailRISCVzfmakezyD}{\saildocval{}{\lstinputlisting[language=sail]{sail_latex_riscv/valzfmake_de1abf4e676ae7d02ecd6d0037a7d77ee.tex}}}}

\newcommand{\sailRISCVfnfmakeD}{\saildoclabelled{sailRISCVfnzfmakezyD}{\saildocfn{}{\lstinputlisting[language=sail]{sail_latex_riscv/fnzfmake_de1abf4e676ae7d02ecd6d0037a7d77ee.tex}}}}

\newcommand{\sailRISCVvalfIsNegInfD}{\saildoclabelled{sailRISCVzfzyiszynegzyinfzyD}{\saildocval{}{\lstinputlisting[language=sail]{sail_latex_riscv/valzf_is_neg_inf_d3f44c40462ec32ab41b8fa4dc4e6f998.tex}}}}

\newcommand{\sailRISCVfnfIsNegInfD}{\saildoclabelled{sailRISCVfnzfzyiszynegzyinfzyD}{\saildocfn{}{\lstinputlisting[language=sail]{sail_latex_riscv/fnzf_is_neg_inf_d3f44c40462ec32ab41b8fa4dc4e6f998.tex}}}}

\newcommand{\sailRISCVvalfIsNegNormD}{\saildoclabelled{sailRISCVzfzyiszynegzynormzyD}{\saildocval{}{\lstinputlisting[language=sail]{sail_latex_riscv/valzf_is_neg_norm_d9b9e067af3304bb1fae5d1501327fd53.tex}}}}

\newcommand{\sailRISCVfnfIsNegNormD}{\saildoclabelled{sailRISCVfnzfzyiszynegzynormzyD}{\saildocfn{}{\lstinputlisting[language=sail]{sail_latex_riscv/fnzf_is_neg_norm_d9b9e067af3304bb1fae5d1501327fd53.tex}}}}

\newcommand{\sailRISCVvalfIsNegSubnormD}{\saildoclabelled{sailRISCVzfzyiszynegzysubnormzyD}{\saildocval{}{\lstinputlisting[language=sail]{sail_latex_riscv/valzf_is_neg_subnorm_d808d9c9ea664491fe0b23b650d1e996c.tex}}}}

\newcommand{\sailRISCVfnfIsNegSubnormD}{\saildoclabelled{sailRISCVfnzfzyiszynegzysubnormzyD}{\saildocfn{}{\lstinputlisting[language=sail]{sail_latex_riscv/fnzf_is_neg_subnorm_d808d9c9ea664491fe0b23b650d1e996c.tex}}}}

\newcommand{\sailRISCVvalfIsNegZeroD}{\saildoclabelled{sailRISCVzfzyiszynegzyzzerozyD}{\saildocval{}{\lstinputlisting[language=sail]{sail_latex_riscv/valzf_is_neg_zzero_d7855947d702bd41cc96dbb033e17918f.tex}}}}

\newcommand{\sailRISCVfnfIsNegZeroD}{\saildoclabelled{sailRISCVfnzfzyiszynegzyzzerozyD}{\saildocfn{}{\lstinputlisting[language=sail]{sail_latex_riscv/fnzf_is_neg_zzero_d7855947d702bd41cc96dbb033e17918f.tex}}}}

\newcommand{\sailRISCVvalfIsPosZeroD}{\saildoclabelled{sailRISCVzfzyiszyposzyzzerozyD}{\saildocval{}{\lstinputlisting[language=sail]{sail_latex_riscv/valzf_is_pos_zzero_d7e379cb7f8f90c1af7e79d78e2a86f37.tex}}}}

\newcommand{\sailRISCVfnfIsPosZeroD}{\saildoclabelled{sailRISCVfnzfzyiszyposzyzzerozyD}{\saildocfn{}{\lstinputlisting[language=sail]{sail_latex_riscv/fnzf_is_pos_zzero_d7e379cb7f8f90c1af7e79d78e2a86f37.tex}}}}

\newcommand{\sailRISCVvalfIsPosSubnormD}{\saildoclabelled{sailRISCVzfzyiszyposzysubnormzyD}{\saildocval{}{\lstinputlisting[language=sail]{sail_latex_riscv/valzf_is_pos_subnorm_d7e83f87299dae2035e33b7c97dd6be45.tex}}}}

\newcommand{\sailRISCVfnfIsPosSubnormD}{\saildoclabelled{sailRISCVfnzfzyiszyposzysubnormzyD}{\saildocfn{}{\lstinputlisting[language=sail]{sail_latex_riscv/fnzf_is_pos_subnorm_d7e83f87299dae2035e33b7c97dd6be45.tex}}}}

\newcommand{\sailRISCVvalfIsPosNormD}{\saildoclabelled{sailRISCVzfzyiszyposzynormzyD}{\saildocval{}{\lstinputlisting[language=sail]{sail_latex_riscv/valzf_is_pos_norm_d6611be4e31c69209487c037c5336c370.tex}}}}

\newcommand{\sailRISCVfnfIsPosNormD}{\saildoclabelled{sailRISCVfnzfzyiszyposzynormzyD}{\saildocfn{}{\lstinputlisting[language=sail]{sail_latex_riscv/fnzf_is_pos_norm_d6611be4e31c69209487c037c5336c370.tex}}}}

\newcommand{\sailRISCVvalfIsPosInfD}{\saildoclabelled{sailRISCVzfzyiszyposzyinfzyD}{\saildocval{}{\lstinputlisting[language=sail]{sail_latex_riscv/valzf_is_pos_inf_ddc033c6bfe555dfe790113dec0ddb7e3.tex}}}}

\newcommand{\sailRISCVfnfIsPosInfD}{\saildoclabelled{sailRISCVfnzfzyiszyposzyinfzyD}{\saildocfn{}{\lstinputlisting[language=sail]{sail_latex_riscv/fnzf_is_pos_inf_ddc033c6bfe555dfe790113dec0ddb7e3.tex}}}}

\newcommand{\sailRISCVvalfIsSNaND}{\saildoclabelled{sailRISCVzfzyiszySNaNzyD}{\saildocval{}{\lstinputlisting[language=sail]{sail_latex_riscv/valzf_is_snan_d3e7f4d132d4d3fe4846581a123c48b56.tex}}}}

\newcommand{\sailRISCVfnfIsSNaND}{\saildoclabelled{sailRISCVfnzfzyiszySNaNzyD}{\saildocfn{}{\lstinputlisting[language=sail]{sail_latex_riscv/fnzf_is_snan_d3e7f4d132d4d3fe4846581a123c48b56.tex}}}}

\newcommand{\sailRISCVvalfIsQNaND}{\saildoclabelled{sailRISCVzfzyiszyQNaNzyD}{\saildocval{}{\lstinputlisting[language=sail]{sail_latex_riscv/valzf_is_qnan_d5bbb21fc7537d62baacf94741f71226d.tex}}}}

\newcommand{\sailRISCVfnfIsQNaND}{\saildoclabelled{sailRISCVfnzfzyiszyQNaNzyD}{\saildocfn{}{\lstinputlisting[language=sail]{sail_latex_riscv/fnzf_is_qnan_d5bbb21fc7537d62baacf94741f71226d.tex}}}}

\newcommand{\sailRISCVvalfIsNaND}{\saildoclabelled{sailRISCVzfzyiszyNaNzyD}{\saildocval{}{\lstinputlisting[language=sail]{sail_latex_riscv/valzf_is_nan_d970e43a6bafb89b645fab556a4ad623c.tex}}}}

\newcommand{\sailRISCVfnfIsNaND}{\saildoclabelled{sailRISCVfnzfzyiszyNaNzyD}{\saildocfn{}{\lstinputlisting[language=sail]{sail_latex_riscv/fnzf_is_nan_d970e43a6bafb89b645fab556a4ad623c.tex}}}}

\newcommand{\sailRISCVvalnegateD}{\saildoclabelled{sailRISCVznegatezyD}{\saildocval{}{\lstinputlisting[language=sail]{sail_latex_riscv/valznegate_dc9622c2a4f7fec13696be94bec5ae96c.tex}}}}

\newcommand{\sailRISCVfnnegateD}{\saildoclabelled{sailRISCVfnznegatezyD}{\saildocfn{}{\lstinputlisting[language=sail]{sail_latex_riscv/fnznegate_dc9622c2a4f7fec13696be94bec5ae96c.tex}}}}

\newcommand{\sailRISCVvalfeqQuietD}{\saildoclabelled{sailRISCVzfeqzyquietzyD}{\saildocval{}{\lstinputlisting[language=sail]{sail_latex_riscv/valzfeq_quiet_d0b6b5eb435e8b9c9bcf3b77e16fff044.tex}}}}

\newcommand{\sailRISCVfnfeqQuietD}{\saildoclabelled{sailRISCVfnzfeqzyquietzyD}{\saildocfn{}{\lstinputlisting[language=sail]{sail_latex_riscv/fnzfeq_quiet_d0b6b5eb435e8b9c9bcf3b77e16fff044.tex}}}}

\newcommand{\sailRISCVvalfltD}{\saildoclabelled{sailRISCVzfltzyD}{\saildocval{}{\lstinputlisting[language=sail]{sail_latex_riscv/valzflt_dc45e113a7fb9ece6e62166679b975f44.tex}}}}

\newcommand{\sailRISCVfnfltD}{\saildoclabelled{sailRISCVfnzfltzyD}{\saildocfn{}{\lstinputlisting[language=sail]{sail_latex_riscv/fnzflt_dc45e113a7fb9ece6e62166679b975f44.tex}}}}

\newcommand{\sailRISCVvalfleD}{\saildoclabelled{sailRISCVzflezyD}{\saildocval{}{\lstinputlisting[language=sail]{sail_latex_riscv/valzfle_dc504d6eb55e9a730eda7235ae654dd1c.tex}}}}

\newcommand{\sailRISCVfnfleD}{\saildoclabelled{sailRISCVfnzflezyD}{\saildocfn{}{\lstinputlisting[language=sail]{sail_latex_riscv/fnzfle_dc504d6eb55e9a730eda7235ae654dd1c.tex}}}}

\newcommand{\sailRISCVvalhaveDoubleFPU}{\saildoclabelled{sailRISCVzhaveDoubleFPU}{\saildocval{}{\lstinputlisting[language=sail]{sail_latex_riscv/valzhavedoublefpu83ee4ffd2eb1361c47ebb4f92891e68d.tex}}}}

\newcommand{\sailRISCVfnhaveDoubleFPU}{\saildoclabelled{sailRISCVfnzhaveDoubleFPU}{\saildocfn{}{\lstinputlisting[language=sail]{sail_latex_riscv/fnzhavedoublefpu83ee4ffd2eb1361c47ebb4f92891e68d.tex}}}}

\newcommand{\sailRISCVvalvalidDoubleRegs}{\saildoclabelled{sailRISCVzvalidDoubleRegs}{\saildocval{}{\lstinputlisting[language=sail]{sail_latex_riscv/valzvaliddoubleregscd4e0fc960d20d254e143f01d81b4d44.tex}}}}

\newcommand{\sailRISCVfnvalidDoubleRegs}{\saildoclabelled{sailRISCVfnzvalidDoubleRegs}{\saildocfn{}{\lstinputlisting[language=sail]{sail_latex_riscv/fnzvaliddoubleregscd4e0fc960d20d254e143f01d81b4d44.tex}}}}

\newcommand{\sailRISCVvalfMaddTypeMnemonicD}{\saildoclabelled{sailRISCVzfzymaddzytypezymnemoniczyD}{\saildocval{}{\lstinputlisting[language=sail]{sail_latex_riscv/valzf_madd_type_mnemonic_de10f87b5527cad531eb1f3e055843354.tex}}}}

\newcommand{\sailRISCVvalfBinRmTypeMnemonicD}{\saildoclabelled{sailRISCVzfzybinzyrmzytypezymnemoniczyD}{\saildocval{}{\lstinputlisting[language=sail]{sail_latex_riscv/valzf_bin_rm_type_mnemonic_deb9e3f6f9179ea4b09697f358f9f6a0b.tex}}}}

\newcommand{\sailRISCVvalfUnRmTypeMnemonicD}{\saildoclabelled{sailRISCVzfzyunzyrmzytypezymnemoniczyD}{\saildocval{}{\lstinputlisting[language=sail]{sail_latex_riscv/valzf_un_rm_type_mnemonic_d4a9ba98f21ea36db9609bd6233749d86.tex}}}}

\newcommand{\sailRISCVvalfBinTypeMnemonicD}{\saildoclabelled{sailRISCVzfzybinzytypezymnemoniczyD}{\saildocval{}{\lstinputlisting[language=sail]{sail_latex_riscv/valzf_bin_type_mnemonic_d4c3459a461182e50755d18d2795fe9de.tex}}}}

\newcommand{\sailRISCVvalfUnTypeMnemonicD}{\saildoclabelled{sailRISCVzfzyunzytypezymnemoniczyD}{\saildocval{}{\lstinputlisting[language=sail]{sail_latex_riscv/valzf_un_type_mnemonic_d67aa9b49056aabdddaa065ba03ecbadb.tex}}}}

\newcommand{\sailRISCVvalhandleLoadDataViaCap}{\saildoclabelled{sailRISCVzhandlezyloadzydatazyviazycap}{\saildocval{}{\lstinputlisting[language=sail]{sail_latex_riscv/valzhandle_load_data_via_capf2827ba9c795213ba1703c9ae04ec80e.tex}}}}

\newcommand{\sailRISCVfnhandleLoadDataViaCap}{\saildoclabelled{sailRISCVfnzhandlezyloadzydatazyviazycap}{\saildocfn{}{\lstinputlisting[language=sail]{sail_latex_riscv/fnzhandle_load_data_via_capf2827ba9c795213ba1703c9ae04ec80e.tex}}}}

\newcommand{\sailRISCVvalhandleLoadCapViaCap}{\saildoclabelled{sailRISCVzhandlezyloadzycapzyviazycap}{\saildocval{}{\lstinputlisting[language=sail]{sail_latex_riscv/valzhandle_load_cap_via_capc983c32af845a3dec84f2d1241c33390.tex}}}}

\newcommand{\sailRISCVfnhandleLoadCapViaCap}{\saildoclabelled{sailRISCVfnzhandlezyloadzycapzyviazycap}{\saildocfn{}{\lstinputlisting[language=sail]{sail_latex_riscv/fnzhandle_load_cap_via_capc983c32af845a3dec84f2d1241c33390.tex}}}}

\newcommand{\sailRISCVvalcheckResMisaligned}{\saildoclabelled{sailRISCVzcheckzyreszymisaligned}{\saildocval{}{\lstinputlisting[language=sail]{sail_latex_riscv/valzcheck_res_misalignede05ad7adc37fbdff34459806aaddc074.tex}}}}

\newcommand{\sailRISCVfncheckResMisaligned}{\saildoclabelled{sailRISCVfnzcheckzyreszymisaligned}{\saildocfn{}{\lstinputlisting[language=sail]{sail_latex_riscv/fnzcheck_res_misalignede05ad7adc37fbdff34459806aaddc074.tex}}}}

\newcommand{\sailRISCVvalhandleLoadresDataViaCap}{\saildoclabelled{sailRISCVzhandlezyloadreszydatazyviazycap}{\saildocval{}{\lstinputlisting[language=sail]{sail_latex_riscv/valzhandle_loadres_data_via_cap08189f30ce7dd627d16d9228026d4005.tex}}}}

\newcommand{\sailRISCVfnhandleLoadresDataViaCap}{\saildoclabelled{sailRISCVfnzhandlezyloadreszydatazyviazycap}{\saildocfn{}{\lstinputlisting[language=sail]{sail_latex_riscv/fnzhandle_loadres_data_via_cap08189f30ce7dd627d16d9228026d4005.tex}}}}

\newcommand{\sailRISCVvalhandleLoadresCapViaCap}{\saildoclabelled{sailRISCVzhandlezyloadreszycapzyviazycap}{\saildocval{}{\lstinputlisting[language=sail]{sail_latex_riscv/valzhandle_loadres_cap_via_capcff54276a28e84c26d372dc8f545b3f0.tex}}}}

\newcommand{\sailRISCVfnhandleLoadresCapViaCap}{\saildoclabelled{sailRISCVfnzhandlezyloadreszycapzyviazycap}{\saildocfn{}{\lstinputlisting[language=sail]{sail_latex_riscv/fnzhandle_loadres_cap_via_capcff54276a28e84c26d372dc8f545b3f0.tex}}}}

\newcommand{\sailRISCVvalhandleStoreDataViaCap}{\saildoclabelled{sailRISCVzhandlezystorezydatazyviazycap}{\saildocval{}{\lstinputlisting[language=sail]{sail_latex_riscv/valzhandle_store_data_via_capa375706a8b6644ee610b608c6152f081.tex}}}}

\newcommand{\sailRISCVfnhandleStoreDataViaCap}{\saildoclabelled{sailRISCVfnzhandlezystorezydatazyviazycap}{\saildocfn{}{\lstinputlisting[language=sail]{sail_latex_riscv/fnzhandle_store_data_via_capa375706a8b6644ee610b608c6152f081.tex}}}}

\newcommand{\sailRISCVvalhandleStoreCapViaCap}{\saildoclabelled{sailRISCVzhandlezystorezycapzyviazycap}{\saildocval{}{\lstinputlisting[language=sail]{sail_latex_riscv/valzhandle_store_cap_via_cap4d789b2f59a39e53239d073296e45c38.tex}}}}

\newcommand{\sailRISCVfnhandleStoreCapViaCap}{\saildoclabelled{sailRISCVfnzhandlezystorezycapzyviazycap}{\saildocfn{}{\lstinputlisting[language=sail]{sail_latex_riscv/fnzhandle_store_cap_via_cap4d789b2f59a39e53239d073296e45c38.tex}}}}

\newcommand{\sailRISCVvalhandleStoreCondDataViaCap}{\saildoclabelled{sailRISCVzhandlezystorezycondzydatazyviazycap}{\saildocval{}{\lstinputlisting[language=sail]{sail_latex_riscv/valzhandle_store_cond_data_via_cap90bf472704e0e0e2072d2afbb0123121.tex}}}}

\newcommand{\sailRISCVfnhandleStoreCondDataViaCap}{\saildoclabelled{sailRISCVfnzhandlezystorezycondzydatazyviazycap}{\saildocfn{}{\lstinputlisting[language=sail]{sail_latex_riscv/fnzhandle_store_cond_data_via_cap90bf472704e0e0e2072d2afbb0123121.tex}}}}

\newcommand{\sailRISCVvalwriteScCapResult}{\saildoclabelled{sailRISCVzwritezysczycapzyresult}{\saildocval{}{\lstinputlisting[language=sail]{sail_latex_riscv/valzwrite_sc_cap_result1d0d547672de98c1b674eca63a52126b.tex}}}}

\newcommand{\sailRISCVfnwriteScCapResult}{\saildoclabelled{sailRISCVfnzwritezysczycapzyresult}{\saildocfn{}{\lstinputlisting[language=sail]{sail_latex_riscv/fnzwrite_sc_cap_result1d0d547672de98c1b674eca63a52126b.tex}}}}

\newcommand{\sailRISCVvalhandleStoreCondCapViaCap}{\saildoclabelled{sailRISCVzhandlezystorezycondzycapzyviazycap}{\saildocval{}{\lstinputlisting[language=sail]{sail_latex_riscv/valzhandle_store_cond_cap_via_capdfc2c7d93322ab0a720ded29c24438b2.tex}}}}

\newcommand{\sailRISCVfnhandleStoreCondCapViaCap}{\saildoclabelled{sailRISCVfnzhandlezystorezycondzycapzyviazycap}{\saildocfn{}{\lstinputlisting[language=sail]{sail_latex_riscv/fnzhandle_store_cond_cap_via_capdfc2c7d93322ab0a720ded29c24438b2.tex}}}}

\newcommand{\sailRISCVlethaveRVOneTwoEight}{\saildoclabelled{sailRISCVletzhaveRV128}{\saildoclet{}{\lstinputlisting[language=sail]{sail_latex_riscv/letzhaverv12838a7ffddeb07ffd272bcde9de50c3456.tex}}}}

\newcommand{\sailRISCVlethaveRVSixFour}{\saildoclabelled{sailRISCVletzhaveRV64}{\saildoclet{}{\lstinputlisting[language=sail]{sail_latex_riscv/letzhaverv646386e65b1d6070ad5d157a0f6e0a1c6d.tex}}}}

\newcommand{\sailRISCVfclUTYPEexecute}{\saildoclabelled{sailRISCVfclUTYPEzexecute}{\saildocfcl{}{\lstinputlisting[language=sail]{sail_latex_riscv/fclUTYPEzexecute33a689e3a631b9b905b85461d3814943.tex}}}}

\newcommand{\sailRISCVfclRISCVUnderscoreJALexecute}{\saildoclabelled{sailRISCVfclRISCVUnderscoreJALzexecute}{\saildocfcl{}{\lstinputlisting[language=sail]{sail_latex_riscv/fclRISCVUnderscoreJALzexecute33a689e3a631b9b905b85461d3814943.tex}}}}

\newcommand{\sailRISCVfclBTYPEexecute}{\saildoclabelled{sailRISCVfclBTYPEzexecute}{\saildocfcl{}{\lstinputlisting[language=sail]{sail_latex_riscv/fclBTYPEzexecute33a689e3a631b9b905b85461d3814943.tex}}}}

\newcommand{\sailRISCVfclITYPEexecute}{\saildoclabelled{sailRISCVfclITYPEzexecute}{\saildocfcl{}{\lstinputlisting[language=sail]{sail_latex_riscv/fclITYPEzexecute33a689e3a631b9b905b85461d3814943.tex}}}}

\newcommand{\sailRISCVfclSHIFTIOPexecute}{\saildoclabelled{sailRISCVfclSHIFTIOPzexecute}{\saildocfcl{}{\lstinputlisting[language=sail]{sail_latex_riscv/fclSHIFTIOPzexecute33a689e3a631b9b905b85461d3814943.tex}}}}

\newcommand{\sailRISCVfclRTYPEexecute}{\saildoclabelled{sailRISCVfclRTYPEzexecute}{\saildocfcl{}{\lstinputlisting[language=sail]{sail_latex_riscv/fclRTYPEzexecute33a689e3a631b9b905b85461d3814943.tex}}}}

\newcommand{\sailRISCVfclLOADexecute}{\saildoclabelled{sailRISCVfclLOADzexecute}{\saildocfcl{}{\lstinputlisting[language=sail]{sail_latex_riscv/fclLOADzexecute33a689e3a631b9b905b85461d3814943.tex}}}}

\newcommand{\sailRISCVfclSTOREexecute}{\saildoclabelled{sailRISCVfclSTOREzexecute}{\saildocfcl{}{\lstinputlisting[language=sail]{sail_latex_riscv/fclSTOREzexecute33a689e3a631b9b905b85461d3814943.tex}}}}

\newcommand{\sailRISCVfclADDIWexecute}{\saildoclabelled{sailRISCVfclADDIWzexecute}{\saildocfcl{}{\lstinputlisting[language=sail]{sail_latex_riscv/fclADDIWzexecute33a689e3a631b9b905b85461d3814943.tex}}}}

\newcommand{\sailRISCVfclSHIFTWexecute}{\saildoclabelled{sailRISCVfclSHIFTWzexecute}{\saildocfcl{}{\lstinputlisting[language=sail]{sail_latex_riscv/fclSHIFTWzexecute33a689e3a631b9b905b85461d3814943.tex}}}}

\newcommand{\sailRISCVfclRTYPEWexecute}{\saildoclabelled{sailRISCVfclRTYPEWzexecute}{\saildocfcl{}{\lstinputlisting[language=sail]{sail_latex_riscv/fclRTYPEWzexecute33a689e3a631b9b905b85461d3814943.tex}}}}

\newcommand{\sailRISCVfclSHIFTIWOPexecute}{\saildoclabelled{sailRISCVfclSHIFTIWOPzexecute}{\saildocfcl{}{\lstinputlisting[language=sail]{sail_latex_riscv/fclSHIFTIWOPzexecute33a689e3a631b9b905b85461d3814943.tex}}}}

\newcommand{\sailRISCVfclFENCEexecute}{\saildoclabelled{sailRISCVfclFENCEzexecute}{\saildocfcl{}{\lstinputlisting[language=sail]{sail_latex_riscv/fclFENCEzexecute33a689e3a631b9b905b85461d3814943.tex}}}}

\newcommand{\sailRISCVfclFENCEUnderscoreTSOexecute}{\saildoclabelled{sailRISCVfclFENCEUnderscoreTSOzexecute}{\saildocfcl{}{\lstinputlisting[language=sail]{sail_latex_riscv/fclFENCEUnderscoreTSOzexecute33a689e3a631b9b905b85461d3814943.tex}}}}

\newcommand{\sailRISCVfclFENCEIexecute}{\saildoclabelled{sailRISCVfclFENCEIzexecute}{\saildocfcl{}{\lstinputlisting[language=sail]{sail_latex_riscv/fclFENCEIzexecute33a689e3a631b9b905b85461d3814943.tex}}}}

\newcommand{\sailRISCVfclECALLexecute}{\saildoclabelled{sailRISCVfclECALLzexecute}{\saildocfcl{}{\lstinputlisting[language=sail]{sail_latex_riscv/fclECALLzexecute33a689e3a631b9b905b85461d3814943.tex}}}}

\newcommand{\sailRISCVfclMRETexecute}{\saildoclabelled{sailRISCVfclMRETzexecute}{\saildocfcl{}{\lstinputlisting[language=sail]{sail_latex_riscv/fclMRETzexecute33a689e3a631b9b905b85461d3814943.tex}}}}

\newcommand{\sailRISCVfclSRETexecute}{\saildoclabelled{sailRISCVfclSRETzexecute}{\saildocfcl{}{\lstinputlisting[language=sail]{sail_latex_riscv/fclSRETzexecute33a689e3a631b9b905b85461d3814943.tex}}}}

\newcommand{\sailRISCVfclEBREAKexecute}{\saildoclabelled{sailRISCVfclEBREAKzexecute}{\saildocfcl{}{\lstinputlisting[language=sail]{sail_latex_riscv/fclEBREAKzexecute33a689e3a631b9b905b85461d3814943.tex}}}}

\newcommand{\sailRISCVfclWFIexecute}{\saildoclabelled{sailRISCVfclWFIzexecute}{\saildocfcl{}{\lstinputlisting[language=sail]{sail_latex_riscv/fclWFIzexecute33a689e3a631b9b905b85461d3814943.tex}}}}

\newcommand{\sailRISCVfclSFENCEUnderscoreVMAexecute}{\saildoclabelled{sailRISCVfclSFENCEUnderscoreVMAzexecute}{\saildocfcl{}{\lstinputlisting[language=sail]{sail_latex_riscv/fclSFENCEUnderscoreVMAzexecute33a689e3a631b9b905b85461d3814943.tex}}}}

\newcommand{\sailRISCVfclLOADRESexecute}{\saildoclabelled{sailRISCVfclLOADRESzexecute}{\saildocfcl{}{\lstinputlisting[language=sail]{sail_latex_riscv/fclLOADRESzexecute33a689e3a631b9b905b85461d3814943.tex}}}}

\newcommand{\sailRISCVfclSTORECONexecute}{\saildoclabelled{sailRISCVfclSTORECONzexecute}{\saildocfcl{}{\lstinputlisting[language=sail]{sail_latex_riscv/fclSTORECONzexecute33a689e3a631b9b905b85461d3814943.tex}}}}

\newcommand{\sailRISCVfclAMOexecute}{\saildoclabelled{sailRISCVfclAMOzexecute}{\saildocfcl{}{\lstinputlisting[language=sail]{sail_latex_riscv/fclAMOzexecute33a689e3a631b9b905b85461d3814943.tex}}}}

\newcommand{\sailRISCVfclCUnderscoreNOPexecute}{\saildoclabelled{sailRISCVfclCUnderscoreNOPzexecute}{\saildocfcl{}{\lstinputlisting[language=sail]{sail_latex_riscv/fclCUnderscoreNOPzexecute33a689e3a631b9b905b85461d3814943.tex}}}}

\newcommand{\sailRISCVfclCUnderscoreADDIFourSPNexecute}{\saildoclabelled{sailRISCVfclCUnderscoreADDIFourSPNzexecute}{\saildocfcl{}{\lstinputlisting[language=sail]{sail_latex_riscv/fclCUnderscoreADDIFourSPNzexecute33a689e3a631b9b905b85461d3814943.tex}}}}

\newcommand{\sailRISCVfclCUnderscoreLWexecute}{\saildoclabelled{sailRISCVfclCUnderscoreLWzexecute}{\saildocfcl{}{\lstinputlisting[language=sail]{sail_latex_riscv/fclCUnderscoreLWzexecute33a689e3a631b9b905b85461d3814943.tex}}}}

\newcommand{\sailRISCVfclCUnderscoreLDexecute}{\saildoclabelled{sailRISCVfclCUnderscoreLDzexecute}{\saildocfcl{}{\lstinputlisting[language=sail]{sail_latex_riscv/fclCUnderscoreLDzexecute33a689e3a631b9b905b85461d3814943.tex}}}}

\newcommand{\sailRISCVfclCUnderscoreSWexecute}{\saildoclabelled{sailRISCVfclCUnderscoreSWzexecute}{\saildocfcl{}{\lstinputlisting[language=sail]{sail_latex_riscv/fclCUnderscoreSWzexecute33a689e3a631b9b905b85461d3814943.tex}}}}

\newcommand{\sailRISCVfclCUnderscoreSDexecute}{\saildoclabelled{sailRISCVfclCUnderscoreSDzexecute}{\saildocfcl{}{\lstinputlisting[language=sail]{sail_latex_riscv/fclCUnderscoreSDzexecute33a689e3a631b9b905b85461d3814943.tex}}}}

\newcommand{\sailRISCVfclCUnderscoreADDIexecute}{\saildoclabelled{sailRISCVfclCUnderscoreADDIzexecute}{\saildocfcl{}{\lstinputlisting[language=sail]{sail_latex_riscv/fclCUnderscoreADDIzexecute33a689e3a631b9b905b85461d3814943.tex}}}}

\newcommand{\sailRISCVfclCUnderscoreJALexecute}{\saildoclabelled{sailRISCVfclCUnderscoreJALzexecute}{\saildocfcl{}{\lstinputlisting[language=sail]{sail_latex_riscv/fclCUnderscoreJALzexecute33a689e3a631b9b905b85461d3814943.tex}}}}

\newcommand{\sailRISCVfclCUnderscoreADDIWexecute}{\saildoclabelled{sailRISCVfclCUnderscoreADDIWzexecute}{\saildocfcl{}{\lstinputlisting[language=sail]{sail_latex_riscv/fclCUnderscoreADDIWzexecute33a689e3a631b9b905b85461d3814943.tex}}}}

\newcommand{\sailRISCVfclCUnderscoreLIexecute}{\saildoclabelled{sailRISCVfclCUnderscoreLIzexecute}{\saildocfcl{}{\lstinputlisting[language=sail]{sail_latex_riscv/fclCUnderscoreLIzexecute33a689e3a631b9b905b85461d3814943.tex}}}}

\newcommand{\sailRISCVfclCUnderscoreADDIOneSixSPexecute}{\saildoclabelled{sailRISCVfclCUnderscoreADDIOneSixSPzexecute}{\saildocfcl{}{\lstinputlisting[language=sail]{sail_latex_riscv/fclCUnderscoreADDIOneSixSPzexecute33a689e3a631b9b905b85461d3814943.tex}}}}

\newcommand{\sailRISCVfclCUnderscoreLUIexecute}{\saildoclabelled{sailRISCVfclCUnderscoreLUIzexecute}{\saildocfcl{}{\lstinputlisting[language=sail]{sail_latex_riscv/fclCUnderscoreLUIzexecute33a689e3a631b9b905b85461d3814943.tex}}}}

\newcommand{\sailRISCVfclCUnderscoreSRLIexecute}{\saildoclabelled{sailRISCVfclCUnderscoreSRLIzexecute}{\saildocfcl{}{\lstinputlisting[language=sail]{sail_latex_riscv/fclCUnderscoreSRLIzexecute33a689e3a631b9b905b85461d3814943.tex}}}}

\newcommand{\sailRISCVfclCUnderscoreSRAIexecute}{\saildoclabelled{sailRISCVfclCUnderscoreSRAIzexecute}{\saildocfcl{}{\lstinputlisting[language=sail]{sail_latex_riscv/fclCUnderscoreSRAIzexecute33a689e3a631b9b905b85461d3814943.tex}}}}

\newcommand{\sailRISCVfclCUnderscoreANDIexecute}{\saildoclabelled{sailRISCVfclCUnderscoreANDIzexecute}{\saildocfcl{}{\lstinputlisting[language=sail]{sail_latex_riscv/fclCUnderscoreANDIzexecute33a689e3a631b9b905b85461d3814943.tex}}}}

\newcommand{\sailRISCVfclCUnderscoreSUBexecute}{\saildoclabelled{sailRISCVfclCUnderscoreSUBzexecute}{\saildocfcl{}{\lstinputlisting[language=sail]{sail_latex_riscv/fclCUnderscoreSUBzexecute33a689e3a631b9b905b85461d3814943.tex}}}}

\newcommand{\sailRISCVfclCUnderscoreXORexecute}{\saildoclabelled{sailRISCVfclCUnderscoreXORzexecute}{\saildocfcl{}{\lstinputlisting[language=sail]{sail_latex_riscv/fclCUnderscoreXORzexecute33a689e3a631b9b905b85461d3814943.tex}}}}

\newcommand{\sailRISCVfclCUnderscoreORexecute}{\saildoclabelled{sailRISCVfclCUnderscoreORzexecute}{\saildocfcl{}{\lstinputlisting[language=sail]{sail_latex_riscv/fclCUnderscoreORzexecute33a689e3a631b9b905b85461d3814943.tex}}}}

\newcommand{\sailRISCVfclCUnderscoreANDexecute}{\saildoclabelled{sailRISCVfclCUnderscoreANDzexecute}{\saildocfcl{}{\lstinputlisting[language=sail]{sail_latex_riscv/fclCUnderscoreANDzexecute33a689e3a631b9b905b85461d3814943.tex}}}}

\newcommand{\sailRISCVfclCUnderscoreSUBWexecute}{\saildoclabelled{sailRISCVfclCUnderscoreSUBWzexecute}{\saildocfcl{}{\lstinputlisting[language=sail]{sail_latex_riscv/fclCUnderscoreSUBWzexecute33a689e3a631b9b905b85461d3814943.tex}}}}

\newcommand{\sailRISCVfclCUnderscoreADDWexecute}{\saildoclabelled{sailRISCVfclCUnderscoreADDWzexecute}{\saildocfcl{}{\lstinputlisting[language=sail]{sail_latex_riscv/fclCUnderscoreADDWzexecute33a689e3a631b9b905b85461d3814943.tex}}}}

\newcommand{\sailRISCVfclCUnderscoreJexecute}{\saildoclabelled{sailRISCVfclCUnderscoreJzexecute}{\saildocfcl{}{\lstinputlisting[language=sail]{sail_latex_riscv/fclCUnderscoreJzexecute33a689e3a631b9b905b85461d3814943.tex}}}}

\newcommand{\sailRISCVfclCUnderscoreBEQZexecute}{\saildoclabelled{sailRISCVfclCUnderscoreBEQZzexecute}{\saildocfcl{}{\lstinputlisting[language=sail]{sail_latex_riscv/fclCUnderscoreBEQZzexecute33a689e3a631b9b905b85461d3814943.tex}}}}

\newcommand{\sailRISCVfclCUnderscoreBNEZexecute}{\saildoclabelled{sailRISCVfclCUnderscoreBNEZzexecute}{\saildocfcl{}{\lstinputlisting[language=sail]{sail_latex_riscv/fclCUnderscoreBNEZzexecute33a689e3a631b9b905b85461d3814943.tex}}}}

\newcommand{\sailRISCVfclCUnderscoreSLLIexecute}{\saildoclabelled{sailRISCVfclCUnderscoreSLLIzexecute}{\saildocfcl{}{\lstinputlisting[language=sail]{sail_latex_riscv/fclCUnderscoreSLLIzexecute33a689e3a631b9b905b85461d3814943.tex}}}}

\newcommand{\sailRISCVfclCUnderscoreLWSPexecute}{\saildoclabelled{sailRISCVfclCUnderscoreLWSPzexecute}{\saildocfcl{}{\lstinputlisting[language=sail]{sail_latex_riscv/fclCUnderscoreLWSPzexecute33a689e3a631b9b905b85461d3814943.tex}}}}

\newcommand{\sailRISCVfclCUnderscoreLDSPexecute}{\saildoclabelled{sailRISCVfclCUnderscoreLDSPzexecute}{\saildocfcl{}{\lstinputlisting[language=sail]{sail_latex_riscv/fclCUnderscoreLDSPzexecute33a689e3a631b9b905b85461d3814943.tex}}}}

\newcommand{\sailRISCVfclCUnderscoreSWSPexecute}{\saildoclabelled{sailRISCVfclCUnderscoreSWSPzexecute}{\saildocfcl{}{\lstinputlisting[language=sail]{sail_latex_riscv/fclCUnderscoreSWSPzexecute33a689e3a631b9b905b85461d3814943.tex}}}}

\newcommand{\sailRISCVfclCUnderscoreSDSPexecute}{\saildoclabelled{sailRISCVfclCUnderscoreSDSPzexecute}{\saildocfcl{}{\lstinputlisting[language=sail]{sail_latex_riscv/fclCUnderscoreSDSPzexecute33a689e3a631b9b905b85461d3814943.tex}}}}

\newcommand{\sailRISCVfclCUnderscoreJRexecute}{\saildoclabelled{sailRISCVfclCUnderscoreJRzexecute}{\saildocfcl{}{\lstinputlisting[language=sail]{sail_latex_riscv/fclCUnderscoreJRzexecute33a689e3a631b9b905b85461d3814943.tex}}}}

\newcommand{\sailRISCVfclCUnderscoreJALRexecute}{\saildoclabelled{sailRISCVfclCUnderscoreJALRzexecute}{\saildocfcl{}{\lstinputlisting[language=sail]{sail_latex_riscv/fclCUnderscoreJALRzexecute33a689e3a631b9b905b85461d3814943.tex}}}}

\newcommand{\sailRISCVfclCUnderscoreMVexecute}{\saildoclabelled{sailRISCVfclCUnderscoreMVzexecute}{\saildocfcl{}{\lstinputlisting[language=sail]{sail_latex_riscv/fclCUnderscoreMVzexecute33a689e3a631b9b905b85461d3814943.tex}}}}

\newcommand{\sailRISCVfclCUnderscoreEBREAKexecute}{\saildoclabelled{sailRISCVfclCUnderscoreEBREAKzexecute}{\saildocfcl{}{\lstinputlisting[language=sail]{sail_latex_riscv/fclCUnderscoreEBREAKzexecute33a689e3a631b9b905b85461d3814943.tex}}}}

\newcommand{\sailRISCVfclCUnderscoreADDexecute}{\saildoclabelled{sailRISCVfclCUnderscoreADDzexecute}{\saildocfcl{}{\lstinputlisting[language=sail]{sail_latex_riscv/fclCUnderscoreADDzexecute33a689e3a631b9b905b85461d3814943.tex}}}}

\newcommand{\sailRISCVfclMULexecute}{\saildoclabelled{sailRISCVfclMULzexecute}{\saildocfcl{}{\lstinputlisting[language=sail]{sail_latex_riscv/fclMULzexecute33a689e3a631b9b905b85461d3814943.tex}}}}

\newcommand{\sailRISCVfclDIVexecute}{\saildoclabelled{sailRISCVfclDIVzexecute}{\saildocfcl{}{\lstinputlisting[language=sail]{sail_latex_riscv/fclDIVzexecute33a689e3a631b9b905b85461d3814943.tex}}}}

\newcommand{\sailRISCVfclREMexecute}{\saildoclabelled{sailRISCVfclREMzexecute}{\saildocfcl{}{\lstinputlisting[language=sail]{sail_latex_riscv/fclREMzexecute33a689e3a631b9b905b85461d3814943.tex}}}}

\newcommand{\sailRISCVfclMULWexecute}{\saildoclabelled{sailRISCVfclMULWzexecute}{\saildocfcl{}{\lstinputlisting[language=sail]{sail_latex_riscv/fclMULWzexecute33a689e3a631b9b905b85461d3814943.tex}}}}

\newcommand{\sailRISCVfclDIVWexecute}{\saildoclabelled{sailRISCVfclDIVWzexecute}{\saildocfcl{}{\lstinputlisting[language=sail]{sail_latex_riscv/fclDIVWzexecute33a689e3a631b9b905b85461d3814943.tex}}}}

\newcommand{\sailRISCVfclREMWexecute}{\saildoclabelled{sailRISCVfclREMWzexecute}{\saildocfcl{}{\lstinputlisting[language=sail]{sail_latex_riscv/fclREMWzexecute33a689e3a631b9b905b85461d3814943.tex}}}}

\newcommand{\sailRISCVfclCSRexecute}{\saildoclabelled{sailRISCVfclCSRzexecute}{\saildocfcl{}{\lstinputlisting[language=sail]{sail_latex_riscv/fclCSRzexecute33a689e3a631b9b905b85461d3814943.tex}}}}

\newcommand{\sailRISCVfclLOADUnderscoreFPexecute}{\saildoclabelled{sailRISCVfclLOADUnderscoreFPzexecute}{\saildocfcl{}{\lstinputlisting[language=sail]{sail_latex_riscv/fclLOADUnderscoreFPzexecute33a689e3a631b9b905b85461d3814943.tex}}}}

\newcommand{\sailRISCVfclSTOREUnderscoreFPexecute}{\saildoclabelled{sailRISCVfclSTOREUnderscoreFPzexecute}{\saildocfcl{}{\lstinputlisting[language=sail]{sail_latex_riscv/fclSTOREUnderscoreFPzexecute33a689e3a631b9b905b85461d3814943.tex}}}}

\newcommand{\sailRISCVfclFUnderscoreMADDUnderscoreTYPEUnderscoreSexecute}{\saildoclabelled{sailRISCVfclFUnderscoreMADDUnderscoreTYPEUnderscoreSzexecute}{\saildocfcl{}{\lstinputlisting[language=sail]{sail_latex_riscv/fclFUnderscoreMADDUnderscoreTYPEUnderscoreSzexecute33a689e3a631b9b905b85461d3814943.tex}}}}

\newcommand{\sailRISCVfclFUnderscoreBINUnderscoreRMUnderscoreTYPEUnderscoreSexecute}{\saildoclabelled{sailRISCVfclFUnderscoreBINUnderscoreRMUnderscoreTYPEUnderscoreSzexecute}{\saildocfcl{}{\lstinputlisting[language=sail]{sail_latex_riscv/fclFUnderscoreBINUnderscoreRMUnderscoreTYPEUnderscoreSzexecute33a689e3a631b9b905b85461d3814943.tex}}}}

\newcommand{\sailRISCVfclFUnderscoreUNUnderscoreRMUnderscoreTYPEUnderscoreSexecute}{\saildoclabelled{sailRISCVfclFUnderscoreUNUnderscoreRMUnderscoreTYPEUnderscoreSzexecute}{\saildocfcl{}{\lstinputlisting[language=sail]{sail_latex_riscv/fclFUnderscoreUNUnderscoreRMUnderscoreTYPEUnderscoreSzexecute33a689e3a631b9b905b85461d3814943.tex}}}}

\newcommand{\sailRISCVfclFUnderscoreUNUnderscoreRMUnderscoreTYPEUnderscoreSAexecute}{\saildoclabelled{sailRISCVfclFUnderscoreUNUnderscoreRMUnderscoreTYPEUnderscoreSAzexecute}{\saildocfcl{}{\lstinputlisting[language=sail]{sail_latex_riscv/fclFUnderscoreUNUnderscoreRMUnderscoreTYPEUnderscoreSAzexecute33a689e3a631b9b905b85461d3814943.tex}}}}

\newcommand{\sailRISCVfclFUnderscoreUNUnderscoreRMUnderscoreTYPEUnderscoreSBexecute}{\saildoclabelled{sailRISCVfclFUnderscoreUNUnderscoreRMUnderscoreTYPEUnderscoreSBzexecute}{\saildocfcl{}{\lstinputlisting[language=sail]{sail_latex_riscv/fclFUnderscoreUNUnderscoreRMUnderscoreTYPEUnderscoreSBzexecute33a689e3a631b9b905b85461d3814943.tex}}}}

\newcommand{\sailRISCVfclFUnderscoreUNUnderscoreRMUnderscoreTYPEUnderscoreSCexecute}{\saildoclabelled{sailRISCVfclFUnderscoreUNUnderscoreRMUnderscoreTYPEUnderscoreSCzexecute}{\saildocfcl{}{\lstinputlisting[language=sail]{sail_latex_riscv/fclFUnderscoreUNUnderscoreRMUnderscoreTYPEUnderscoreSCzexecute33a689e3a631b9b905b85461d3814943.tex}}}}

\newcommand{\sailRISCVfclFUnderscoreUNUnderscoreRMUnderscoreTYPEUnderscoreSDexecute}{\saildoclabelled{sailRISCVfclFUnderscoreUNUnderscoreRMUnderscoreTYPEUnderscoreSDzexecute}{\saildocfcl{}{\lstinputlisting[language=sail]{sail_latex_riscv/fclFUnderscoreUNUnderscoreRMUnderscoreTYPEUnderscoreSDzexecute33a689e3a631b9b905b85461d3814943.tex}}}}

\newcommand{\sailRISCVfclFUnderscoreUNUnderscoreRMUnderscoreTYPEUnderscoreSEexecute}{\saildoclabelled{sailRISCVfclFUnderscoreUNUnderscoreRMUnderscoreTYPEUnderscoreSEzexecute}{\saildocfcl{}{\lstinputlisting[language=sail]{sail_latex_riscv/fclFUnderscoreUNUnderscoreRMUnderscoreTYPEUnderscoreSEzexecute33a689e3a631b9b905b85461d3814943.tex}}}}

\newcommand{\sailRISCVfclFUnderscoreUNUnderscoreRMUnderscoreTYPEUnderscoreSFexecute}{\saildoclabelled{sailRISCVfclFUnderscoreUNUnderscoreRMUnderscoreTYPEUnderscoreSFzexecute}{\saildocfcl{}{\lstinputlisting[language=sail]{sail_latex_riscv/fclFUnderscoreUNUnderscoreRMUnderscoreTYPEUnderscoreSFzexecute33a689e3a631b9b905b85461d3814943.tex}}}}

\newcommand{\sailRISCVfclFUnderscoreUNUnderscoreRMUnderscoreTYPEUnderscoreSGexecute}{\saildoclabelled{sailRISCVfclFUnderscoreUNUnderscoreRMUnderscoreTYPEUnderscoreSGzexecute}{\saildocfcl{}{\lstinputlisting[language=sail]{sail_latex_riscv/fclFUnderscoreUNUnderscoreRMUnderscoreTYPEUnderscoreSGzexecute33a689e3a631b9b905b85461d3814943.tex}}}}

\newcommand{\sailRISCVfclFUnderscoreUNUnderscoreRMUnderscoreTYPEUnderscoreSHexecute}{\saildoclabelled{sailRISCVfclFUnderscoreUNUnderscoreRMUnderscoreTYPEUnderscoreSHzexecute}{\saildocfcl{}{\lstinputlisting[language=sail]{sail_latex_riscv/fclFUnderscoreUNUnderscoreRMUnderscoreTYPEUnderscoreSHzexecute33a689e3a631b9b905b85461d3814943.tex}}}}

\newcommand{\sailRISCVfclFUnderscoreBINUnderscoreTYPEUnderscoreSexecute}{\saildoclabelled{sailRISCVfclFUnderscoreBINUnderscoreTYPEUnderscoreSzexecute}{\saildocfcl{}{\lstinputlisting[language=sail]{sail_latex_riscv/fclFUnderscoreBINUnderscoreTYPEUnderscoreSzexecute33a689e3a631b9b905b85461d3814943.tex}}}}

\newcommand{\sailRISCVfclFUnderscoreBINUnderscoreTYPEUnderscoreSAexecute}{\saildoclabelled{sailRISCVfclFUnderscoreBINUnderscoreTYPEUnderscoreSAzexecute}{\saildocfcl{}{\lstinputlisting[language=sail]{sail_latex_riscv/fclFUnderscoreBINUnderscoreTYPEUnderscoreSAzexecute33a689e3a631b9b905b85461d3814943.tex}}}}

\newcommand{\sailRISCVfclFUnderscoreBINUnderscoreTYPEUnderscoreSBexecute}{\saildoclabelled{sailRISCVfclFUnderscoreBINUnderscoreTYPEUnderscoreSBzexecute}{\saildocfcl{}{\lstinputlisting[language=sail]{sail_latex_riscv/fclFUnderscoreBINUnderscoreTYPEUnderscoreSBzexecute33a689e3a631b9b905b85461d3814943.tex}}}}

\newcommand{\sailRISCVfclFUnderscoreBINUnderscoreTYPEUnderscoreSCexecute}{\saildoclabelled{sailRISCVfclFUnderscoreBINUnderscoreTYPEUnderscoreSCzexecute}{\saildocfcl{}{\lstinputlisting[language=sail]{sail_latex_riscv/fclFUnderscoreBINUnderscoreTYPEUnderscoreSCzexecute33a689e3a631b9b905b85461d3814943.tex}}}}

\newcommand{\sailRISCVfclFUnderscoreBINUnderscoreTYPEUnderscoreSDexecute}{\saildoclabelled{sailRISCVfclFUnderscoreBINUnderscoreTYPEUnderscoreSDzexecute}{\saildocfcl{}{\lstinputlisting[language=sail]{sail_latex_riscv/fclFUnderscoreBINUnderscoreTYPEUnderscoreSDzexecute33a689e3a631b9b905b85461d3814943.tex}}}}

\newcommand{\sailRISCVfclFUnderscoreBINUnderscoreTYPEUnderscoreSEexecute}{\saildoclabelled{sailRISCVfclFUnderscoreBINUnderscoreTYPEUnderscoreSEzexecute}{\saildocfcl{}{\lstinputlisting[language=sail]{sail_latex_riscv/fclFUnderscoreBINUnderscoreTYPEUnderscoreSEzexecute33a689e3a631b9b905b85461d3814943.tex}}}}

\newcommand{\sailRISCVfclFUnderscoreBINUnderscoreTYPEUnderscoreSFexecute}{\saildoclabelled{sailRISCVfclFUnderscoreBINUnderscoreTYPEUnderscoreSFzexecute}{\saildocfcl{}{\lstinputlisting[language=sail]{sail_latex_riscv/fclFUnderscoreBINUnderscoreTYPEUnderscoreSFzexecute33a689e3a631b9b905b85461d3814943.tex}}}}

\newcommand{\sailRISCVfclFUnderscoreBINUnderscoreTYPEUnderscoreSGexecute}{\saildoclabelled{sailRISCVfclFUnderscoreBINUnderscoreTYPEUnderscoreSGzexecute}{\saildocfcl{}{\lstinputlisting[language=sail]{sail_latex_riscv/fclFUnderscoreBINUnderscoreTYPEUnderscoreSGzexecute33a689e3a631b9b905b85461d3814943.tex}}}}

\newcommand{\sailRISCVfclFUnderscoreUNUnderscoreTYPEUnderscoreSexecute}{\saildoclabelled{sailRISCVfclFUnderscoreUNUnderscoreTYPEUnderscoreSzexecute}{\saildocfcl{}{\lstinputlisting[language=sail]{sail_latex_riscv/fclFUnderscoreUNUnderscoreTYPEUnderscoreSzexecute33a689e3a631b9b905b85461d3814943.tex}}}}

\newcommand{\sailRISCVfclFUnderscoreUNUnderscoreTYPEUnderscoreSAexecute}{\saildoclabelled{sailRISCVfclFUnderscoreUNUnderscoreTYPEUnderscoreSAzexecute}{\saildocfcl{}{\lstinputlisting[language=sail]{sail_latex_riscv/fclFUnderscoreUNUnderscoreTYPEUnderscoreSAzexecute33a689e3a631b9b905b85461d3814943.tex}}}}

\newcommand{\sailRISCVfclFUnderscoreUNUnderscoreTYPEUnderscoreSBexecute}{\saildoclabelled{sailRISCVfclFUnderscoreUNUnderscoreTYPEUnderscoreSBzexecute}{\saildocfcl{}{\lstinputlisting[language=sail]{sail_latex_riscv/fclFUnderscoreUNUnderscoreTYPEUnderscoreSBzexecute33a689e3a631b9b905b85461d3814943.tex}}}}

\newcommand{\sailRISCVfclFUnderscoreMADDUnderscoreTYPEUnderscoreDexecute}{\saildoclabelled{sailRISCVfclFUnderscoreMADDUnderscoreTYPEUnderscoreDzexecute}{\saildocfcl{}{\lstinputlisting[language=sail]{sail_latex_riscv/fclFUnderscoreMADDUnderscoreTYPEUnderscoreDzexecute33a689e3a631b9b905b85461d3814943.tex}}}}

\newcommand{\sailRISCVfclFUnderscoreBINUnderscoreRMUnderscoreTYPEUnderscoreDexecute}{\saildoclabelled{sailRISCVfclFUnderscoreBINUnderscoreRMUnderscoreTYPEUnderscoreDzexecute}{\saildocfcl{}{\lstinputlisting[language=sail]{sail_latex_riscv/fclFUnderscoreBINUnderscoreRMUnderscoreTYPEUnderscoreDzexecute33a689e3a631b9b905b85461d3814943.tex}}}}

\newcommand{\sailRISCVfclFUnderscoreUNUnderscoreRMUnderscoreTYPEUnderscoreDexecute}{\saildoclabelled{sailRISCVfclFUnderscoreUNUnderscoreRMUnderscoreTYPEUnderscoreDzexecute}{\saildocfcl{}{\lstinputlisting[language=sail]{sail_latex_riscv/fclFUnderscoreUNUnderscoreRMUnderscoreTYPEUnderscoreDzexecute33a689e3a631b9b905b85461d3814943.tex}}}}

\newcommand{\sailRISCVfclFUnderscoreUNUnderscoreRMUnderscoreTYPEUnderscoreDAexecute}{\saildoclabelled{sailRISCVfclFUnderscoreUNUnderscoreRMUnderscoreTYPEUnderscoreDAzexecute}{\saildocfcl{}{\lstinputlisting[language=sail]{sail_latex_riscv/fclFUnderscoreUNUnderscoreRMUnderscoreTYPEUnderscoreDAzexecute33a689e3a631b9b905b85461d3814943.tex}}}}

\newcommand{\sailRISCVfclFUnderscoreUNUnderscoreRMUnderscoreTYPEUnderscoreDBexecute}{\saildoclabelled{sailRISCVfclFUnderscoreUNUnderscoreRMUnderscoreTYPEUnderscoreDBzexecute}{\saildocfcl{}{\lstinputlisting[language=sail]{sail_latex_riscv/fclFUnderscoreUNUnderscoreRMUnderscoreTYPEUnderscoreDBzexecute33a689e3a631b9b905b85461d3814943.tex}}}}

\newcommand{\sailRISCVfclFUnderscoreUNUnderscoreRMUnderscoreTYPEUnderscoreDCexecute}{\saildoclabelled{sailRISCVfclFUnderscoreUNUnderscoreRMUnderscoreTYPEUnderscoreDCzexecute}{\saildocfcl{}{\lstinputlisting[language=sail]{sail_latex_riscv/fclFUnderscoreUNUnderscoreRMUnderscoreTYPEUnderscoreDCzexecute33a689e3a631b9b905b85461d3814943.tex}}}}

\newcommand{\sailRISCVfclFUnderscoreUNUnderscoreRMUnderscoreTYPEUnderscoreDDexecute}{\saildoclabelled{sailRISCVfclFUnderscoreUNUnderscoreRMUnderscoreTYPEUnderscoreDDzexecute}{\saildocfcl{}{\lstinputlisting[language=sail]{sail_latex_riscv/fclFUnderscoreUNUnderscoreRMUnderscoreTYPEUnderscoreDDzexecute33a689e3a631b9b905b85461d3814943.tex}}}}

\newcommand{\sailRISCVfclFUnderscoreUNUnderscoreRMUnderscoreTYPEUnderscoreDEexecute}{\saildoclabelled{sailRISCVfclFUnderscoreUNUnderscoreRMUnderscoreTYPEUnderscoreDEzexecute}{\saildocfcl{}{\lstinputlisting[language=sail]{sail_latex_riscv/fclFUnderscoreUNUnderscoreRMUnderscoreTYPEUnderscoreDEzexecute33a689e3a631b9b905b85461d3814943.tex}}}}

\newcommand{\sailRISCVfclFUnderscoreUNUnderscoreRMUnderscoreTYPEUnderscoreDFexecute}{\saildoclabelled{sailRISCVfclFUnderscoreUNUnderscoreRMUnderscoreTYPEUnderscoreDFzexecute}{\saildocfcl{}{\lstinputlisting[language=sail]{sail_latex_riscv/fclFUnderscoreUNUnderscoreRMUnderscoreTYPEUnderscoreDFzexecute33a689e3a631b9b905b85461d3814943.tex}}}}

\newcommand{\sailRISCVfclFUnderscoreUNUnderscoreRMUnderscoreTYPEUnderscoreDGexecute}{\saildoclabelled{sailRISCVfclFUnderscoreUNUnderscoreRMUnderscoreTYPEUnderscoreDGzexecute}{\saildocfcl{}{\lstinputlisting[language=sail]{sail_latex_riscv/fclFUnderscoreUNUnderscoreRMUnderscoreTYPEUnderscoreDGzexecute33a689e3a631b9b905b85461d3814943.tex}}}}

\newcommand{\sailRISCVfclFUnderscoreUNUnderscoreRMUnderscoreTYPEUnderscoreDHexecute}{\saildoclabelled{sailRISCVfclFUnderscoreUNUnderscoreRMUnderscoreTYPEUnderscoreDHzexecute}{\saildocfcl{}{\lstinputlisting[language=sail]{sail_latex_riscv/fclFUnderscoreUNUnderscoreRMUnderscoreTYPEUnderscoreDHzexecute33a689e3a631b9b905b85461d3814943.tex}}}}

\newcommand{\sailRISCVfclFUnderscoreUNUnderscoreRMUnderscoreTYPEUnderscoreDIexecute}{\saildoclabelled{sailRISCVfclFUnderscoreUNUnderscoreRMUnderscoreTYPEUnderscoreDIzexecute}{\saildocfcl{}{\lstinputlisting[language=sail]{sail_latex_riscv/fclFUnderscoreUNUnderscoreRMUnderscoreTYPEUnderscoreDIzexecute33a689e3a631b9b905b85461d3814943.tex}}}}

\newcommand{\sailRISCVfclFUnderscoreUNUnderscoreRMUnderscoreTYPEUnderscoreDJexecute}{\saildoclabelled{sailRISCVfclFUnderscoreUNUnderscoreRMUnderscoreTYPEUnderscoreDJzexecute}{\saildocfcl{}{\lstinputlisting[language=sail]{sail_latex_riscv/fclFUnderscoreUNUnderscoreRMUnderscoreTYPEUnderscoreDJzexecute33a689e3a631b9b905b85461d3814943.tex}}}}

\newcommand{\sailRISCVfclFUnderscoreBINUnderscoreTYPEUnderscoreDexecute}{\saildoclabelled{sailRISCVfclFUnderscoreBINUnderscoreTYPEUnderscoreDzexecute}{\saildocfcl{}{\lstinputlisting[language=sail]{sail_latex_riscv/fclFUnderscoreBINUnderscoreTYPEUnderscoreDzexecute33a689e3a631b9b905b85461d3814943.tex}}}}

\newcommand{\sailRISCVfclFUnderscoreBINUnderscoreTYPEUnderscoreDAexecute}{\saildoclabelled{sailRISCVfclFUnderscoreBINUnderscoreTYPEUnderscoreDAzexecute}{\saildocfcl{}{\lstinputlisting[language=sail]{sail_latex_riscv/fclFUnderscoreBINUnderscoreTYPEUnderscoreDAzexecute33a689e3a631b9b905b85461d3814943.tex}}}}

\newcommand{\sailRISCVfclFUnderscoreBINUnderscoreTYPEUnderscoreDBexecute}{\saildoclabelled{sailRISCVfclFUnderscoreBINUnderscoreTYPEUnderscoreDBzexecute}{\saildocfcl{}{\lstinputlisting[language=sail]{sail_latex_riscv/fclFUnderscoreBINUnderscoreTYPEUnderscoreDBzexecute33a689e3a631b9b905b85461d3814943.tex}}}}

\newcommand{\sailRISCVfclFUnderscoreBINUnderscoreTYPEUnderscoreDCexecute}{\saildoclabelled{sailRISCVfclFUnderscoreBINUnderscoreTYPEUnderscoreDCzexecute}{\saildocfcl{}{\lstinputlisting[language=sail]{sail_latex_riscv/fclFUnderscoreBINUnderscoreTYPEUnderscoreDCzexecute33a689e3a631b9b905b85461d3814943.tex}}}}

\newcommand{\sailRISCVfclFUnderscoreBINUnderscoreTYPEUnderscoreDDexecute}{\saildoclabelled{sailRISCVfclFUnderscoreBINUnderscoreTYPEUnderscoreDDzexecute}{\saildocfcl{}{\lstinputlisting[language=sail]{sail_latex_riscv/fclFUnderscoreBINUnderscoreTYPEUnderscoreDDzexecute33a689e3a631b9b905b85461d3814943.tex}}}}

\newcommand{\sailRISCVfclFUnderscoreBINUnderscoreTYPEUnderscoreDEexecute}{\saildoclabelled{sailRISCVfclFUnderscoreBINUnderscoreTYPEUnderscoreDEzexecute}{\saildocfcl{}{\lstinputlisting[language=sail]{sail_latex_riscv/fclFUnderscoreBINUnderscoreTYPEUnderscoreDEzexecute33a689e3a631b9b905b85461d3814943.tex}}}}

\newcommand{\sailRISCVfclFUnderscoreBINUnderscoreTYPEUnderscoreDFexecute}{\saildoclabelled{sailRISCVfclFUnderscoreBINUnderscoreTYPEUnderscoreDFzexecute}{\saildocfcl{}{\lstinputlisting[language=sail]{sail_latex_riscv/fclFUnderscoreBINUnderscoreTYPEUnderscoreDFzexecute33a689e3a631b9b905b85461d3814943.tex}}}}

\newcommand{\sailRISCVfclFUnderscoreBINUnderscoreTYPEUnderscoreDGexecute}{\saildoclabelled{sailRISCVfclFUnderscoreBINUnderscoreTYPEUnderscoreDGzexecute}{\saildocfcl{}{\lstinputlisting[language=sail]{sail_latex_riscv/fclFUnderscoreBINUnderscoreTYPEUnderscoreDGzexecute33a689e3a631b9b905b85461d3814943.tex}}}}

\newcommand{\sailRISCVfclFUnderscoreUNUnderscoreTYPEUnderscoreDexecute}{\saildoclabelled{sailRISCVfclFUnderscoreUNUnderscoreTYPEUnderscoreDzexecute}{\saildocfcl{}{\lstinputlisting[language=sail]{sail_latex_riscv/fclFUnderscoreUNUnderscoreTYPEUnderscoreDzexecute33a689e3a631b9b905b85461d3814943.tex}}}}

\newcommand{\sailRISCVfclFUnderscoreUNUnderscoreTYPEUnderscoreDAexecute}{\saildoclabelled{sailRISCVfclFUnderscoreUNUnderscoreTYPEUnderscoreDAzexecute}{\saildocfcl{}{\lstinputlisting[language=sail]{sail_latex_riscv/fclFUnderscoreUNUnderscoreTYPEUnderscoreDAzexecute33a689e3a631b9b905b85461d3814943.tex}}}}

\newcommand{\sailRISCVfclFUnderscoreUNUnderscoreTYPEUnderscoreDBexecute}{\saildoclabelled{sailRISCVfclFUnderscoreUNUnderscoreTYPEUnderscoreDBzexecute}{\saildocfcl{}{\lstinputlisting[language=sail]{sail_latex_riscv/fclFUnderscoreUNUnderscoreTYPEUnderscoreDBzexecute33a689e3a631b9b905b85461d3814943.tex}}}}

\newcommand{\sailRISCVfclAUIPCCexecute}{\saildoclabelled{sailRISCVfclAUIPCCzexecute}{\saildocfcl{Capability register \emph{cd} is replaced with the contents of \textbf{PCC}, with the
\textbf{address} replaced with \textbf{PCC}.\textbf{address} $+$ \emph{imm} $\times$ 4096.

}{\lstinputlisting[language=sail]{sail_latex_riscv/fclAUIPCCzexecute33a689e3a631b9b905b85461d3814943.tex}}}}

\newcommand{\sailRISCVfclCJALexecute}{\saildoclabelled{sailRISCVfclCJALzexecute}{\saildocfcl{Capability register \emph{cd} is replaced with the next instruction's \textbf{PCC} and
sealed as a sentry. \textbf{PCC}.\textbf{address} is incremented by \emph{imm}.

\subsection*{Exceptions}


An exception is raised if:

\begin{itemize}
\item \textbf{PCC}.\textbf{address} $+$ \emph{imm} $\lt$ \textbf{PCC}.\textbf{base}.
\item \textbf{PCC}.\textbf{address} $+$ \emph{imm} $+$ min\_instruction\_bytes $\gt$ \textbf{PCC}.\textbf{top}.
\item \textbf{PCC}.\textbf{address} $+$ \emph{imm} is unaligned, ignoring bit 0.
\end{itemize}
}{\lstinputlisting[language=sail]{sail_latex_riscv/fclCJALzexecute33a689e3a631b9b905b85461d3814943.tex}}}}

\newcommand{\sailRISCVfclCJALRexecute}{\saildoclabelled{sailRISCVfclCJALRzexecute}{\saildocfcl{Capability register \emph{cd} is replaced with the next instruction's \textbf{PCC} and
sealed as a sentry. \textbf{PCC} is replaced with the value of capability
register \emph{cs1} with its \textbf{address} incremented by \emph{imm} and the 0th bit of
its \textbf{address} set to 0, and is unsealed if it is a sentry.

\subsection*{Exceptions}


An exception is raised if:

\begin{itemize}
\item \emph{cs1}.\textbf{tag} is not set.
\item \emph{cs1} is sealed and is not a sentry.
\item \emph{cs1} is a sentry and \emph{imm} $\ne$ 0.
\item \emph{cs1}.\textbf{perms} does not grant \textbf{Permit\_Execute}.
\item \emph{cs1}.\textbf{address} $+$ \emph{imm} $\lt$ \emph{cs1}.\textbf{base}.
\item \emph{cs1}.\textbf{address} $+$ \emph{imm} $+$ min\_instruction\_bytes $\gt$ \emph{cs1}.\textbf{top}.
\item \emph{cs1}.\textbf{base} is unaligned.
\item \emph{cs1}.\textbf{address} $+$ \emph{imm} is unaligned, ignoring bit 0.
\end{itemize}
}{\lstinputlisting[language=sail]{sail_latex_riscv/fclCJALRzexecute33a689e3a631b9b905b85461d3814943.tex}}}}

\newcommand{\sailRISCVfclCGetPermexecute}{\saildoclabelled{sailRISCVfclCGetPermzexecute}{\saildocfcl{The least significant \hyperref[sailRISCVzcapzyhpermszywidth]{\lstinline{cap_hperms_width}} bits of integer register \emph{rd} are
set equal to the \textbf{perms} field of capability register \emph{cs1}; bits
\hyperref[sailRISCVzcapzyupermszyshift]{\lstinline{cap_uperms_shift}} to \hyperref[sailRISCVzcapzyupermszyshift]{\lstinline{cap_uperms_shift}}+\hyperref[sailRISCVzcapzyupermszywidth]{\lstinline{cap_uperms_width}}-1 of \emph{rd} are set
equal to the \textbf{uperms} field of \emph{cs1}.
The other bits of \emph{rd} are set to zero.

}{\lstinputlisting[language=sail]{sail_latex_riscv/fclCGetPermzexecute33a689e3a631b9b905b85461d3814943.tex}}}}

\newcommand{\sailRISCVfclCGetFlagsexecute}{\saildoclabelled{sailRISCVfclCGetFlagszexecute}{\saildocfcl{Integer register \emph{rd} is set equal to the zero-extended \textbf{flags} field of
capability register \emph{cs1}.

}{\lstinputlisting[language=sail]{sail_latex_riscv/fclCGetFlagszexecute33a689e3a631b9b905b85461d3814943.tex}}}}

\newcommand{\sailRISCVfclCGetTypeexecute}{\saildoclabelled{sailRISCVfclCGetTypezexecute}{\saildocfcl{Integer register \emph{rd} is set equal to the \textbf{otype} field of capability
register \emph{cs1}.

}{\lstinputlisting[language=sail]{sail_latex_riscv/fclCGetTypezexecute33a689e3a631b9b905b85461d3814943.tex}}}}

\newcommand{\sailRISCVfclCGetBaseexecute}{\saildoclabelled{sailRISCVfclCGetBasezexecute}{\saildocfcl{Integer register \emph{rd} is set equal to the \textbf{base} field of capability
register \emph{cs1}.

}{\lstinputlisting[language=sail]{sail_latex_riscv/fclCGetBasezexecute33a689e3a631b9b905b85461d3814943.tex}}}}

\newcommand{\sailRISCVfclCGetOffsetexecute}{\saildoclabelled{sailRISCVfclCGetOffsetzexecute}{\saildocfcl{Integer register \emph{rd} is set equal to the \textbf{offset} field of capability
register \emph{cs1}.

}{\lstinputlisting[language=sail]{sail_latex_riscv/fclCGetOffsetzexecute33a689e3a631b9b905b85461d3814943.tex}}}}

\newcommand{\sailRISCVfclCGetHighexecute}{\saildoclabelled{sailRISCVfclCGetHighzexecute}{\saildocfcl{Integer register \emph{rd} is set equal to the \textbf{high half} of capability
register \emph{cs1}.

The bits returned here are of the \textbf{in-memory} form of the capability, which
may differ from microarchitectural forms in use within implementations.
(Notably, in the sail implementation, see the distinction between \hyperref[sailRISCVzcapToBits]{\lstinline{capToBits}}
and \hyperref[sailRISCVzcapToMemBits]{\lstinline{capToMemBits}}.)  That is, applying \hyperref[sailRISCVzCGetHigh]{\lstinline{CGetHigh}} to a capability loaded
from address \emph{m} will yield the same result as loading the high half of the
capability-sized granule at \emph{m} (that is, bits above \textbf{XLEN} when a
capability is interpreted as a twice-\textbf{XLEN}-bit integer).

}{\lstinputlisting[language=sail]{sail_latex_riscv/fclCGetHighzexecute33a689e3a631b9b905b85461d3814943.tex}}}}

\newcommand{\sailRISCVfclCSetHighexecute}{\saildoclabelled{sailRISCVfclCSetHighzexecute}{\saildocfcl{Capability register \emph{cd} comes to hold the capability from \emph{cs1} with its
high bits replaced with the value in the integer register \emph{rs2}.  The tag
of \emph{cd} is cleared.

\emph{rs2} holds the \textbf{in-memory} form of capability bits.  That is, this
instruction yields the same result as writing \emph{cs1} out to memory,
overwriting the high word with \emph{rs2}, and loading that capability-sized
granule into \emph{cd}, although without the memory mutation side-effects.

}{\lstinputlisting[language=sail]{sail_latex_riscv/fclCSetHighzexecute33a689e3a631b9b905b85461d3814943.tex}}}}

\newcommand{\sailRISCVfclCGetLenexecute}{\saildoclabelled{sailRISCVfclCGetLenzexecute}{\saildocfcl{Integer register \emph{rd} is set equal to the \textbf{length} field of capability
register \emph{cs1}.

\subsection*{Notes}


\begin{itemize}
\item Due to the compressed representation of capabilities, the actual length
 of capabilities can be $2^{\hyperref[sailRISCVzxlen]{{xlen}}}$; \hyperref[sailRISCVzCGetLen]{\lstinline{CGetLen}} will return the
 maximum value of $2^{\hyperref[sailRISCVzxlen]{{xlen}}}-1$ in this case.
\end{itemize}
}{\lstinputlisting[language=sail]{sail_latex_riscv/fclCGetLenzexecute33a689e3a631b9b905b85461d3814943.tex}}}}

\newcommand{\sailRISCVfclCGetTopexecute}{\saildoclabelled{sailRISCVfclCGetTopzexecute}{\saildocfcl{Integer register \emph{rd} is set equal to the \textbf{top} field (\saildocabbrev{i.e.} one past the
last addressable byte) of capability register \emph{cs1}.

\subsection*{Notes}


\begin{itemize}
\item Due to the compressed representation of capabilities, the actual top
 of capabilities can be $2^{\hyperref[sailRISCVzxlen]{{xlen}}}$; \hyperref[sailRISCVzCGetTop]{\lstinline{CGetTop}} will return the
 maximum value of $2^{\hyperref[sailRISCVzxlen]{{xlen}}}-1$ in this case.
\end{itemize}
}{\lstinputlisting[language=sail]{sail_latex_riscv/fclCGetTopzexecute33a689e3a631b9b905b85461d3814943.tex}}}}

\newcommand{\sailRISCVfclCGetTagexecute}{\saildoclabelled{sailRISCVfclCGetTagzexecute}{\saildocfcl{The low bit of integer register \emph{rd} is set to the \textbf{tag} field of \emph{cs1}.
All other bits of \emph{rd} are cleared.

}{\lstinputlisting[language=sail]{sail_latex_riscv/fclCGetTagzexecute33a689e3a631b9b905b85461d3814943.tex}}}}

\newcommand{\sailRISCVfclCGetSealedexecute}{\saildoclabelled{sailRISCVfclCGetSealedzexecute}{\saildocfcl{The low bit of integer register \emph{rd} is set to 0 if \emph{cs1} is unsealed
and to 1 otherwise.
All other bits of \emph{rd} are cleared.

}{\lstinputlisting[language=sail]{sail_latex_riscv/fclCGetSealedzexecute33a689e3a631b9b905b85461d3814943.tex}}}}

\newcommand{\sailRISCVfclCGetAddrexecute}{\saildoclabelled{sailRISCVfclCGetAddrzexecute}{\saildocfcl{Integer register \emph{rd} is set equal to the \textbf{address} field of capability
register \emph{cs1}.

}{\lstinputlisting[language=sail]{sail_latex_riscv/fclCGetAddrzexecute33a689e3a631b9b905b85461d3814943.tex}}}}

\newcommand{\sailRISCVfclCSpecialRWexecute}{\saildoclabelled{sailRISCVfclCSpecialRWzexecute}{\saildocfcl{Capability register \emph{cd} is set equal to special capability register \emph{scr},
and \emph{scr} is set equal to capability register \emph{cs1} if \emph{cs1} is not \textbf{C0}.

\subsection*{Exceptions}


An exception is raised if:

\begin{itemize}
\item \emph{scr} does not exist.
\item \emph{scr} is read-only and \emph{cs1} is not \textbf{C0}.
\item \emph{scr} is only accessible to a higher privilege mode.
\item \emph{scr} requires \textbf{Permit\_Access\_System\_Registers} and that is not
 granted by \textbf{PCC}.\textbf{perms}.
\end{itemize}


\subsection*{Notes}


\begin{itemize}
\item Writing \textbf{NULL} to a special capability register cannot be done with \textbf{C0}
 as that only performs a read. An alternative implementation would allocate
 a separate two-operand CSpecialR instruction and interpret \emph{cs1} being
 \textbf{C0} as a write of \textbf{NULL} if the need to use a temporary capability
 register proves to be overly problematic for software. For U-mode
 transitions to domains without \textbf{Permit\_Access\_System\_Registers} only
 \textbf{DDC} should need clearing, which can be done with \hyperref[sailRISCVzCClear]{\lstinline{CClear}}.
\end{itemize}
}{\lstinputlisting[language=sail]{sail_latex_riscv/fclCSpecialRWzexecute33a689e3a631b9b905b85461d3814943.tex}}}}

\newcommand{\sailRISCVfclCAndPermexecute}{\saildoclabelled{sailRISCVfclCAndPermzexecute}{\saildocfcl{Capability register \emph{cd} is replaced with the contents of capability
register \emph{cs1} with the \textbf{perms} field set to the bitwise and of its
previous value and bits 0 to \hyperref[sailRISCVzcapzyhpermszywidth]{\lstinline{cap_hperms_width}}-1 of integer register \emph{rs2}
and the \textbf{uperms} field set to the bitwise and of its previous value and
bits \hyperref[sailRISCVzcapzyupermszyshift]{\lstinline{cap_uperms_shift}} to \hyperref[sailRISCVzcapzyupermszyshift]{\lstinline{cap_uperms_shift}}+\hyperref[sailRISCVzcapzyupermszywidth]{\lstinline{cap_uperms_width}}-1 of \emph{rs2}.
If \emph{cs1} was sealed then \emph{cd}.\textbf{tag} is cleared.

}{\lstinputlisting[language=sail]{sail_latex_riscv/fclCAndPermzexecute33a689e3a631b9b905b85461d3814943.tex}}}}

\newcommand{\sailRISCVfclCSetFlagsexecute}{\saildoclabelled{sailRISCVfclCSetFlagszexecute}{\saildocfcl{Capability register \emph{cd} is replaced with the contents of capability
register \emph{cs1} with the \textbf{flags} field set to bits 0 to \hyperref[sailRISCVzcapzyflagszywidth]{\lstinline{cap_flags_width}}-1
of integer register \emph{rs2}. If \emph{cs1} was sealed then \emph{cd}.\textbf{tag} is cleared.

}{\lstinputlisting[language=sail]{sail_latex_riscv/fclCSetFlagszexecute33a689e3a631b9b905b85461d3814943.tex}}}}

\newcommand{\sailRISCVfclCToPtrexecute}{\saildoclabelled{sailRISCVfclCToPtrzexecute}{\saildocfcl{If the \textbf{tag} field of capability register \emph{cs1} is not set, integer
register \emph{rd} is set to 0, otherwise integer register \emph{rd} is set to
\emph{cs1}.\textbf{address} $-$ \emph{cs2}.\textbf{base}.

\subsection*{Notes}


\begin{itemize}
\item \emph{cs2} being sealed will not set \emph{rd} to 0. This is for further study.
\end{itemize}
}{\lstinputlisting[language=sail]{sail_latex_riscv/fclCToPtrzexecute33a689e3a631b9b905b85461d3814943.tex}}}}

\newcommand{\sailRISCVfclCSubexecute}{\saildoclabelled{sailRISCVfclCSubzexecute}{\saildocfcl{Integer register \emph{rd} is set equal to (\emph{cs1}.\textbf{address} $-$
\emph{cs2}.\textbf{address}) $\bmod~2^{\hyperref[sailRISCVzxlen]{{xlen}}}$.

}{\lstinputlisting[language=sail]{sail_latex_riscv/fclCSubzexecute33a689e3a631b9b905b85461d3814943.tex}}}}

\newcommand{\sailRISCVfclCIncOffsetexecute}{\saildoclabelled{sailRISCVfclCIncOffsetzexecute}{\saildocfcl{Capability register \emph{cd} is set equal to capability register \emph{cs1} with its
\textbf{address} replaced with \emph{cs1}.\textbf{address} $+$ \emph{rs2}.
If the resulting capability cannot be represented exactly, or if \emph{cs1} was
sealed, then \emph{cd}.\textbf{tag} is cleared. The remaining capability fields are
set to what the in-memory representation of \emph{cs1} with the address set to
\emph{cs1}.\textbf{address} $+$ \emph{rs2} decodes to.

}{\lstinputlisting[language=sail]{sail_latex_riscv/fclCIncOffsetzexecute33a689e3a631b9b905b85461d3814943.tex}}}}

\newcommand{\sailRISCVfclCIncOffsetImmediateexecute}{\saildoclabelled{sailRISCVfclCIncOffsetImmediatezexecute}{\saildocfcl{Capability register \emph{cd} is set equal to capability register \emph{cs1} with its
\textbf{address} replaced with \emph{cs1}.\textbf{address} $+$ \emph{imm}.
If the resulting capability cannot be represented exactly, or if \emph{cs1} was
sealed, then \emph{cd}.\textbf{tag} is cleared. The remaining capability fields are
set to what the in-memory representation of \emph{cs1} with the address set to
\emph{cs1}.\textbf{address} $+$ \emph{imm} decodes to.

}{\lstinputlisting[language=sail]{sail_latex_riscv/fclCIncOffsetImmediatezexecute33a689e3a631b9b905b85461d3814943.tex}}}}

\newcommand{\sailRISCVfclCSetOffsetexecute}{\saildoclabelled{sailRISCVfclCSetOffsetzexecute}{\saildocfcl{Capability register \emph{cd} is set equal to capability register \emph{cs1} with its
\textbf{address} replaced with \emph{cs1}.\textbf{base} $+$ \emph{rs2}.
If the resulting capability cannot be represented exactly, or if \emph{cs1} was
sealed, then \emph{cd}.\textbf{tag} is cleared. The remaining capability fields are
set to what the in-memory representation of \emph{cs1} with the address set to
\emph{cs1}.\textbf{base} $+$ \emph{rs2} decodes to.

}{\lstinputlisting[language=sail]{sail_latex_riscv/fclCSetOffsetzexecute33a689e3a631b9b905b85461d3814943.tex}}}}

\newcommand{\sailRISCVfclCSetAddrexecute}{\saildoclabelled{sailRISCVfclCSetAddrzexecute}{\saildocfcl{Capability register \emph{cd} is set equal to capability register \emph{cs1} with its
\textbf{address} replaced with \emph{rs2}.
If the resulting capability cannot be represented exactly, or if \emph{cs1} was
sealed, then \emph{cd}.\textbf{tag} is cleared. The remaining capability fields are
set to what the in-memory representation of \emph{cs1} with the address set to
\emph{rs2} decodes to.

}{\lstinputlisting[language=sail]{sail_latex_riscv/fclCSetAddrzexecute33a689e3a631b9b905b85461d3814943.tex}}}}

\newcommand{\sailRISCVfclCSetBoundsexecute}{\saildoclabelled{sailRISCVfclCSetBoundszexecute}{\saildocfcl{Capability register \emph{cd} is set to capability register \emph{cs1} with its
\textbf{base} field replaced with \emph{cs1}.\textbf{address} and its \textbf{length} field
replaced with integer register \emph{rs2}. If the resulting capability cannot be
represented exactly the \textbf{base} will be rounded down and the \textbf{length}
will be rounded up by the smallest amount needed to form a representable
capability covering the requested bounds. The \textbf{tag} field of the result
is cleared if the bounds of the result exceed the bounds of \emph{cs1}, or if
\emph{cs1} was sealed.

}{\lstinputlisting[language=sail]{sail_latex_riscv/fclCSetBoundszexecute33a689e3a631b9b905b85461d3814943.tex}}}}

\newcommand{\sailRISCVfclCSetBoundsImmediateexecute}{\saildoclabelled{sailRISCVfclCSetBoundsImmediatezexecute}{\saildocfcl{Capability register \emph{cd} is set to capability register \emph{cs1} with its
\textbf{base} field replaced with \emph{cs1}.\textbf{address} and its \textbf{length} field
replaced with \emph{uimm}. If the resulting capability cannot be represented
exactly the \textbf{base} will be rounded down and the \textbf{length} will be rounded
up by the smallest amount needed to form a representable capability covering
the requested bounds. The \textbf{tag} field of the result is cleared if the
bounds of the result exceed the bounds of \emph{cs1}, or if \emph{cs1} was sealed.

}{\lstinputlisting[language=sail]{sail_latex_riscv/fclCSetBoundsImmediatezexecute33a689e3a631b9b905b85461d3814943.tex}}}}

\newcommand{\sailRISCVfclCSetBoundsExactexecute}{\saildoclabelled{sailRISCVfclCSetBoundsExactzexecute}{\saildocfcl{Capability register \emph{cd} is set to capability register \emph{cs1} with its
\textbf{base} field replaced with \emph{cs1}.\textbf{address} and its \textbf{length} field
replaced with integer register \emph{rs2}. If the resulting capability cannot be
represented exactly, the \textbf{tag} field will be cleared (unlike
\hyperref[sailRISCVzCSetBounds]{\lstinline{CSetBounds}}), the \textbf{base} will be rounded down and the \textbf{length} will be
rounded up by the smallest amount needed to form a representable capability
covering the requested bounds. The \textbf{tag} field of the result is cleared
if the bounds of the result exceed the bounds of \emph{cs1}, or if \emph{cs1} was
sealed.

}{\lstinputlisting[language=sail]{sail_latex_riscv/fclCSetBoundsExactzexecute33a689e3a631b9b905b85461d3814943.tex}}}}

\newcommand{\sailRISCVfclCClearTagexecute}{\saildoclabelled{sailRISCVfclCClearTagzexecute}{\saildocfcl{Capability register \emph{cd} is replaced with the contents of \emph{cs1}, with
the \textbf{tag} field cleared.

}{\lstinputlisting[language=sail]{sail_latex_riscv/fclCClearTagzexecute33a689e3a631b9b905b85461d3814943.tex}}}}

\newcommand{\sailRISCVfclCMoveexecute}{\saildoclabelled{sailRISCVfclCMovezexecute}{\saildocfcl{Capability register \emph{cd} is replaced with the contents of \emph{cs1}.

}{\lstinputlisting[language=sail]{sail_latex_riscv/fclCMovezexecute33a689e3a631b9b905b85461d3814943.tex}}}}

\newcommand{\sailRISCVfclClearexecute}{\saildoclabelled{sailRISCVfclClearzexecute}{\saildocfcl{Integer registers 8 $\times$ \emph{q} $+$ \emph{i} are each set to 0 if the \emph{i}th bit
of \emph{m} is set. This instruction is only present on implementations with a
split register file. On implementations with a merged register file the
functionality of this instruction is covered by \hyperref[sailRISCVzCClear]{\lstinline{CClear}}.

\subsection*{Notes}


\begin{itemize}
\item This instruction is designed to accelerate the register clearing that is
 required for secure domain transitions. It is expected that it can be
 implemented efficiently in hardware using a single `valid' bit per
 register that is cleared by this instruction and set on any subsequent
 write to the register.
\end{itemize}
}{\lstinputlisting[language=sail]{sail_latex_riscv/fclClearzexecute33a689e3a631b9b905b85461d3814943.tex}}}}

\newcommand{\sailRISCVfclCClearexecute}{\saildoclabelled{sailRISCVfclCClearzexecute}{\saildocfcl{Capability registers 8 $\times$ \emph{q} $+$ \emph{i} are each set to \textbf{NULL} if the
\emph{i}th bit of \emph{m} is set, with the exception that the 0th bit of \emph{m} refers
to \textbf{DDC} when \emph{q} is 0, rather than \textbf{C0}.

\subsection*{Notes}


\begin{itemize}
\item This instruction is designed to accelerate the register clearing that is
 required for secure domain transitions. It is expected that it can be
 implemented efficiently in hardware using a single `valid' bit per
 register that is cleared by this instruction and set on any subsequent
 write to the register.
\end{itemize}
}{\lstinputlisting[language=sail]{sail_latex_riscv/fclCClearzexecute33a689e3a631b9b905b85461d3814943.tex}}}}

\newcommand{\sailRISCVfclFPClearexecute}{\saildoclabelled{sailRISCVfclFPClearzexecute}{\saildocfcl{Floating-point registers 8 $\times$ \emph{q} $+$ \emph{i} are each set to 0 if the
\emph{i}th bit of \emph{m} is set.

\subsection*{Notes}


\begin{itemize}
\item This instruction is designed to accelerate the register clearing that is
 required for secure domain transitions. It is expected that it can be
 implemented efficiently in hardware using a single `valid' bit per
 register that is cleared by this instruction and set on any subsequent
 write to the register.


\item The 0 value written is FLEN bits wide, the largest supported by the
 implementation, such that the in-memory representation of the register is
 0, rather than a NaN-boxed narrower value.


\end{itemize}
}{\lstinputlisting[language=sail]{sail_latex_riscv/fclFPClearzexecute33a689e3a631b9b905b85461d3814943.tex}}}}

\newcommand{\sailRISCVfclCFromPtrexecute}{\saildoclabelled{sailRISCVfclCFromPtrzexecute}{\saildocfcl{If the value of integer register \emph{rs2} is 0 then capability register \emph{cd} is
set to \textbf{NULL}. Otherwise capability register \emph{cd} is set to capability
register \emph{cs1} with its \textbf{offset} replaced with \emph{rs2}. If the resulting
capability cannot be represented exactly, or if \emph{cs1} was sealed, then
\emph{cd}.\textbf{tag} is cleared. The remaining capability fields are set to what the
in-memory representation of \emph{cs1} with the address set to \emph{cd}.\textbf{address}
decodes to.

}{\lstinputlisting[language=sail]{sail_latex_riscv/fclCFromPtrzexecute33a689e3a631b9b905b85461d3814943.tex}}}}

\newcommand{\sailRISCVfclCBuildCapexecute}{\saildoclabelled{sailRISCVfclCBuildCapzexecute}{\saildocfcl{Capability register \emph{cd} is set equal to capability register \emph{cs1} with its
\textbf{base}, \textbf{length}, \textbf{address}, \textbf{perms}, \textbf{uperms} and \textbf{flags}
replaced with the corresponding fields in capability register \emph{cs2}. If
\emph{cs2} is a sentry then \emph{cd} is also sealed as a sentry. If the resulting
capability is not a subset of \emph{cs1} in bounds or permissions, or is not a
legally derivable capability, or if \emph{cs1} did not have its \textbf{tag} field
set, or if \emph{cs1} was sealed, \emph{cd} is replaced with \emph{cs2} with its \textbf{tag}
field cleared.

\subsection*{Notes}


\begin{itemize}
\item Implementations may instead choose to set \emph{cd} to \emph{cs2} with its \textbf{tag}
 set after performing all checks, but the specification derives the result
 from \emph{cs1} in order to convey the provenance associated with this
 operation.
\end{itemize}
}{\lstinputlisting[language=sail]{sail_latex_riscv/fclCBuildCapzexecute33a689e3a631b9b905b85461d3814943.tex}}}}

\newcommand{\sailRISCVfclCCopyTypeexecute}{\saildoclabelled{sailRISCVfclCCopyTypezexecute}{\saildocfcl{Capability register \emph{cd} is replaced with the contents of capability
register \emph{cs1} with the \textbf{address} set to \emph{cs2}.\textbf{otype} and the
\textbf{tag} field cleared if \emph{cs2} has a reserved \textbf{otype} or if \emph{cs1}
was sealed.

\subsection*{Notes}


\begin{itemize}
\item Reserved otypes always result in untagged capabilities, as, at the
 moment, all reserved otypes are constructed using ambiently-available
 actions. \hyperref[sailRISCVzCCSeal]{\lstinline{CCSeal}} knows how to work with these.
\end{itemize}
}{\lstinputlisting[language=sail]{sail_latex_riscv/fclCCopyTypezexecute33a689e3a631b9b905b85461d3814943.tex}}}}

\newcommand{\sailRISCVfclCRRLexecute}{\saildoclabelled{sailRISCVfclCRRLzexecute}{\saildocfcl{Integer register \emph{rd} is set to the smallest value greater or equal to \emph{rs1}
that can be used as a length to set exact bounds on a capability that has a
suitably aligned base (as obtained with the help of \hyperref[sailRISCVzCRAM]{\lstinline{CRAM}}).

}{\lstinputlisting[language=sail]{sail_latex_riscv/fclCRRLzexecute33a689e3a631b9b905b85461d3814943.tex}}}}

\newcommand{\sailRISCVfclCRAMexecute}{\saildoclabelled{sailRISCVfclCRAMzexecute}{\saildocfcl{Integer register \emph{rd} is set to a mask that can be used to round addresses
down to to a value that is sufficiently aligned to set exact bounds for the
nearest representable length of \emph{rs1} (as obtained by \hyperref[sailRISCVzCRRL]{\lstinline{CRRL}}).

}{\lstinputlisting[language=sail]{sail_latex_riscv/fclCRAMzexecute33a689e3a631b9b905b85461d3814943.tex}}}}

\newcommand{\sailRISCVfclCTestSubsetexecute}{\saildoclabelled{sailRISCVfclCTestSubsetzexecute}{\saildocfcl{Integer register \emph{rd} is set to 1 if the \textbf{tag} fields of capability
registers \emph{cs1} and \emph{cs2} are the same and the bounds and permissions of
\emph{cs2} are a subset of those of \emph{cs1}.

\subsection*{Notes}


\begin{itemize}
\item The operand order for this instruction is reversed compared with the
 normal RISC-V comparison instructions, but this may be changed in future.


\item The \textbf{otype} field is ignored for this instruction, but an alternative
 implementation might wish to consider capabilities with distinct
 \textbf{otype}s as unordered as is done for the \textbf{tag} field.


\end{itemize}
}{\lstinputlisting[language=sail]{sail_latex_riscv/fclCTestSubsetzexecute33a689e3a631b9b905b85461d3814943.tex}}}}

\newcommand{\sailRISCVfclCSEQXexecute}{\saildoclabelled{sailRISCVfclCSEQXzexecute}{\saildocfcl{Integer register \emph{rd} is set to 1 if the \textbf{tag} fields and in-memory
representations of capability registers \emph{cs1} and \emph{cs2} are identical,
including any reserved encoding bits, otherwise it is set to 0.

}{\lstinputlisting[language=sail]{sail_latex_riscv/fclCSEQXzexecute33a689e3a631b9b905b85461d3814943.tex}}}}

\newcommand{\sailRISCVfclCSealexecute}{\saildoclabelled{sailRISCVfclCSealzexecute}{\saildocfcl{Capability register \emph{cd} is replaced with capability register \emph{cs1}, and is
sealed with \textbf{otype} equal to the \textbf{address} field of capability register
\emph{cs2}. If \emph{cs2} is unable to authorize the sealing, or if \emph{cs1} was already
sealed, then the \textbf{tag} field of \emph{cd} is cleared.

}{\lstinputlisting[language=sail]{sail_latex_riscv/fclCSealzexecute33a689e3a631b9b905b85461d3814943.tex}}}}

\newcommand{\sailRISCVfclCCSealexecute}{\saildoclabelled{sailRISCVfclCCSealzexecute}{\saildocfcl{Capability register \emph{cd} is replaced with capability register \emph{cs1}, and is
conditionally sealed with \textbf{otype} equal to the \textbf{address} field of
capability register \emph{cs2}. The conditions under which the input is passed
through unaltered are intended to permit a fast branchless rederivation
sequence with multiple sealing authorities with a single \hyperref[sailRISCVzCBuildCap]{\lstinline{CBuildCap}} and a
set of \hyperref[sailRISCVzCCopyType]{\lstinline{CCopyType}} and \hyperref[sailRISCVzCCSeal]{\lstinline{CCSeal}} pairs when swapping capabilities in from
disk. Other than these conditions, if \emph{cs2} is unable to authorize the
sealing, the \textbf{tag} field of \emph{cd} is cleared.

\subsection*{Notes}


\begin{itemize}
\item The intent is that this is used for rederiving swapped-out capabilities,
 so the expectation is that this whole sequence is guarded by a check on
 whether the \textbf{tag} field of the capability was valid.


\item If the input to be conditionally sealed is already sealed it is passed
 through before any futher checks are made. This allows multiple \hyperref[sailRISCVzCCSeal]{\lstinline{CCSeal}}s
 in a chain, any of which can be the one to seal the initial input. The
 intent is that all of these \hyperref[sailRISCVzCCSeal]{\lstinline{CCSeal}}s' authorities will have been produced
 by \hyperref[sailRISCVzCCopyType]{\lstinline{CCopyType}}s of the same input (i.e., they will all attempt to seal to
 the same type), but that's not, strictly, required. Sealed capabilities
 with a reserved \textbf{otype} are also constructed directly by \hyperref[sailRISCVzCBuildCap]{\lstinline{CBuildCap}}.


\item To avoid the need to branch on whether the original capability was sealed,
 attempts to seal with the reserved unsealed \textbf{otype} will leave the
 capability unmodified rather than trap.


\item To avoid the need to check which is the correct authority, any sealing
 request where the \textbf{address} of capability register \emph{cs2} is out of
 bounds will leave the capability unmodified rather than trap, as will
 attempts to seal with an invalid capability since it may have become
 unrepresentable but be within its reinterpreted bounds.


\end{itemize}
}{\lstinputlisting[language=sail]{sail_latex_riscv/fclCCSealzexecute33a689e3a631b9b905b85461d3814943.tex}}}}

\newcommand{\sailRISCVfclCUnsealexecute}{\saildoclabelled{sailRISCVfclCUnsealzexecute}{\saildocfcl{Capability register \emph{cd} is replaced with capability register \emph{cs1} and is
unsealed, using capability register \emph{cs2} as the authority for the unsealing
operation. If \emph{cs2}.\textbf{perms} does not grant \textbf{Global} then \emph{cd}.\textbf{perms}
is stripped of \textbf{Global}. If \emph{cs2} is unable to authorize the unsealing,
the \textbf{tag} field of \emph{cd} is cleared.

}{\lstinputlisting[language=sail]{sail_latex_riscv/fclCUnsealzexecute33a689e3a631b9b905b85461d3814943.tex}}}}

\newcommand{\sailRISCVfclCSealEntryexecute}{\saildoclabelled{sailRISCVfclCSealEntryzexecute}{\saildocfcl{Capability register \emph{cd} is replaced with capability register \emph{cs1} and
sealed as a sentry.

}{\lstinputlisting[language=sail]{sail_latex_riscv/fclCSealEntryzexecute33a689e3a631b9b905b85461d3814943.tex}}}}

\newcommand{\sailRISCVfclCInvokeexecute}{\saildoclabelled{sailRISCVfclCInvokezexecute}{\saildocfcl{\textbf{PCC} is set equal to capability register \emph{cs1} and unsealed with the 0th
bit of its \textbf{address} set to 0, whilst \textbf{C31} is set equal to capability
register \emph{cs2} and unsealed. This provides a constrained form of
non-monotonicity, allowing for fast jumps between protection domains, with
\emph{cs1} providing the target domain's code and \emph{cs2} providing the target
domain's data. The capabilities must have a matching \textbf{otype} to ensure the
right data is provided for the given jump target.

\subsection*{Exceptions}


An exception is raised if:

\begin{itemize}
\item \emph{cs1}.\textbf{tag} is not set.
\item \emph{cs2}.\textbf{tag} is not set.
\item \emph{cs1}.\textbf{otype} is reserved.
\item \emph{cs2}.\textbf{otype} is reserved.
\item \emph{cs1}.\textbf{otype} $\ne$ \emph{cs2}.\textbf{otype}.
\item \emph{cs1}.\textbf{perms} does not grant \textbf{Permit\_CInvoke}.
\item \emph{cs2}.\textbf{perms} does not grant \textbf{Permit\_CInvoke}.
\item \emph{cs1}.\textbf{perms} does not grant \textbf{Permit\_Execute}.
\item \emph{cs2}.\textbf{perms} grants \textbf{Permit\_Execute}.
\item \emph{cs1}.\textbf{address} $\lt$ \emph{cs1}.\textbf{base}.
\item \emph{cs1}.\textbf{address} $+$ min\_instruction\_bytes $\gt$ \emph{cs1}.\textbf{top}.
\item \emph{cs1}.\textbf{base} is unaligned.
\item \emph{cs1}.\textbf{address} is unaligned, ignoring bit 0.
\end{itemize}


\subsection*{Notes}


\begin{itemize}
\item From the point of view of security, this needs to be an atomic operation
 (i.e. the caller cannot decide to just do some of it, because partial
 execution could put the system into an insecure state). From a hardware
 perspective, more complex domain-transition implementations (e.g., to
 implement function-call semantics or message passing) may need to perform
 multiple memory reads and writes, which might take multiple cycles and
 complicate control logic.
\end{itemize}
}{\lstinputlisting[language=sail]{sail_latex_riscv/fclCInvokezexecute33a689e3a631b9b905b85461d3814943.tex}}}}

\newcommand{\sailRISCVfclJALRUnderscoreCAPexecute}{\saildoclabelled{sailRISCVfclJALRUnderscoreCAPzexecute}{\saildocfcl{Capability register \emph{cd} is replaced with the next instruction's \textbf{PCC} and
sealed as a sentry. \textbf{PCC} is replaced with the value of capability
register \emph{cs1} with the 0th bit of its \textbf{address} set to 0 and is unsealed
if it is a sentry.

\subsection*{Exceptions}


An exception is raised if:

\begin{itemize}
\item \emph{cs1}.\textbf{tag} is not set.
\item \emph{cs1} is sealed and is not a sentry.
\item \emph{cs1}.\textbf{perms} does not grant \textbf{Permit\_Execute}.
\item \emph{cs1}.\textbf{address} $\lt$ \emph{cs1}.\textbf{base}.
\item \emph{cs1}.\textbf{address} $+$ min\_instruction\_bytes $\gt$ \emph{cs1}.\textbf{top}.
\item \emph{cs1}.\textbf{base} is unaligned.
\item \emph{cs1}.\textbf{address} is unaligned, ignoring bit 0.
\end{itemize}
}{\lstinputlisting[language=sail]{sail_latex_riscv/fclJALRUnderscoreCAPzexecute33a689e3a631b9b905b85461d3814943.tex}}}}

\newcommand{\sailRISCVfclJALRUnderscorePCCexecute}{\saildoclabelled{sailRISCVfclJALRUnderscorePCCzexecute}{\saildocfcl{Integer register \emph{rd} is replaced with the next instruction's
\textbf{PCC}.\textbf{offset}. \textbf{PCC}.\textbf{offset} is replaced with the value of
register \emph{rs1} with the 0th bit set to 0.

\subsection*{Exceptions}


An exception is raised if:

\begin{itemize}
\item \textbf{PCC}.\textbf{address} $+$ \emph{rs1} $\lt$ \textbf{PCC}.\textbf{base}.
\item \textbf{PCC}.\textbf{address} $+$ \emph{rs1} $+$ min\_instruction\_bytes $\gt$ \textbf{PCC}.\textbf{top}.
\item \textbf{PCC}.\textbf{address} $+$ \emph{rs1} is unaligned, ignoring bit 0.
\end{itemize}
}{\lstinputlisting[language=sail]{sail_latex_riscv/fclJALRUnderscorePCCzexecute33a689e3a631b9b905b85461d3814943.tex}}}}

\newcommand{\sailRISCVfclLoadDataDDCexecute}{\saildoclabelled{sailRISCVfclLoadDataDDCzexecute}{\saildocfcl{Integer register \emph{rd} is replaced with the signed or unsigned byte,
halfword, word or doubleword located in memory at \textbf{DDC}.\textbf{address} $+$
\emph{rs1}.

\subsection*{Exceptions}


An exception is raised if:

\begin{itemize}
\item \textbf{DDC}.\textbf{tag} is not set.
\item \textbf{DDC} is sealed.
\item \textbf{DDC}.\textbf{perms} does not grant \textbf{Permit\_Load}.
\item \textbf{DDC}.\textbf{address} $+$ \emph{rs1} $\lt$ \textbf{DDC}.\textbf{base}.
\item \textbf{DDC}.\textbf{address} $+$ \emph{rs1} $+$ \emph{size} $\gt$ \textbf{DDC}.\textbf{top}.
\end{itemize}
}{\lstinputlisting[language=sail]{sail_latex_riscv/fclLoadDataDDCzexecute33a689e3a631b9b905b85461d3814943.tex}}}}

\newcommand{\sailRISCVfclLoadDataCapexecute}{\saildoclabelled{sailRISCVfclLoadDataCapzexecute}{\saildocfcl{Integer register \emph{rd} is replaced with the signed or unsigned byte,
halfword, word or doubleword located in memory at \emph{cs1}.\textbf{address}.

\subsection*{Exceptions}


An exception is raised if:

\begin{itemize}
\item \emph{cs1}.\textbf{tag} is not set.
\item \emph{cs1} is sealed.
\item \emph{cs1}.\textbf{perms} does not grant \textbf{Permit\_Load}.
\item \emph{cs1}.\textbf{address} $\lt$ \emph{cs1}.\textbf{base}.
\item \emph{cs1}.\textbf{address} $+$ \emph{size} $\gt$ \emph{cs1}.\textbf{top}.
\end{itemize}
}{\lstinputlisting[language=sail]{sail_latex_riscv/fclLoadDataCapzexecute33a689e3a631b9b905b85461d3814943.tex}}}}

\newcommand{\sailRISCVfclLoadCapDDCexecute}{\saildoclabelled{sailRISCVfclLoadCapDDCzexecute}{\saildocfcl{Capability register \emph{cd} is replaced with the capability located in memory
at \textbf{DDC}.\textbf{address} $+$ \emph{rs1}, and if \textbf{DDC}.\textbf{perms} does not grant
\textbf{Permit\_Load\_Capability} then \emph{cd}.\textbf{tag} is cleared.

\subsection*{Exceptions}


An exception is raised if:

\begin{itemize}
\item \textbf{DDC}.\textbf{tag} is not set.
\item \textbf{DDC} is sealed.
\item \textbf{DDC}.\textbf{perms} does not grant \textbf{Permit\_Load}.
\item \textbf{DDC}.\textbf{address} $+$ \emph{rs1} $\lt$ \textbf{DDC}.\textbf{base}.
\item \textbf{DDC}.\textbf{address} $+$ \emph{rs1} $+$ \textbf{CLEN} $/$ 8 $\gt$ \textbf{DDC}.\textbf{top}.
\item \textbf{DDC}.\textbf{address} $+$ \emph{rs1} is unaligned, regardless of whether the
 implementation supports unaligned data accesses.
\end{itemize}
}{\lstinputlisting[language=sail]{sail_latex_riscv/fclLoadCapDDCzexecute33a689e3a631b9b905b85461d3814943.tex}}}}

\newcommand{\sailRISCVfclLoadCapCapexecute}{\saildoclabelled{sailRISCVfclLoadCapCapzexecute}{\saildocfcl{Capability register \emph{cd} is replaced with the capability located in memory
at \emph{cs1}.\textbf{address}, and if \emph{cs1}.\textbf{perms} does not grant
\textbf{Permit\_Load\_Capability} then \emph{cd}.\textbf{tag} is cleared.

\subsection*{Exceptions}


An exception is raised if:

\begin{itemize}
\item \emph{cs1}.\textbf{tag} is not set.
\item \emph{cs1} is sealed.
\item \emph{cs1}.\textbf{perms} does not grant \textbf{Permit\_Load}.
\item \emph{cs1}.\textbf{address} $\lt$ \emph{cs1}.\textbf{base}.
\item \emph{cs1}.\textbf{address} $+$ \textbf{CLEN} $/$ 8 $\gt$ \emph{cs1}.\textbf{top}.
\item \emph{cs1}.\textbf{address} is unaligned, regardless of whether the implementation
 supports unaligned data accesses.
\end{itemize}
}{\lstinputlisting[language=sail]{sail_latex_riscv/fclLoadCapCapzexecute33a689e3a631b9b905b85461d3814943.tex}}}}

\newcommand{\sailRISCVfclCLoadTagsexecute}{\saildoclabelled{sailRISCVfclCLoadTagszexecute}{\saildocfcl{Integer register \emph{rd} is replaced with the tags of the capabilities located
in memory at and above \emph{cs1}.\textbf{address}. The 0th bit corresponds to the
first capability in memory. The result is coherent with other processors, as
if the corresponding data words had also been loaded. The number of tags
loaded is implementation-defined; typical implementations are expected to
return the tags held in an L1 cache line, and so we use the constant
\hyperref[sailRISCVzcapszyperzycachezyline]{\lstinline{caps_per_cache_line}}. The number of tags loaded must be a power of two, at
least 1, and no more than \textbf{XLEN}.

\subsection*{Exceptions}


An exception is raised if:

\begin{itemize}
\item \emph{cs1}.\textbf{tag} is not set.
\item \emph{cs1} is sealed.
\item \emph{cs1}.\textbf{perms} does not grant both \textbf{Permit\_Load} and
 \textbf{Permit\_Load\_Capability}.
\item \emph{cs1}.\textbf{address} $\lt$ \emph{cs1}.\textbf{base}.
\item \emph{cs1}.\textbf{address} $+$ \hyperref[sailRISCVzcapszyperzycachezyline]{\lstinline{caps_per_cache_line}} $\times$ \textbf{CLEN} $/$ 8
 $\gt$ \emph{cs1}.\textbf{top}.
\item \emph{cs1}.\textbf{address} is unaligned.
\item The page table entry for \emph{cs1}.\textbf{address} would cause the tag to be
 cleared.
\end{itemize}


\subsection*{Notes}


\begin{itemize}
\item In order to reduce DRAM traffic, implementations may choose to load only
 the tags and not the corresponding data, and may wish to not evict other
 cache lines by treating it as a non-temporal/streaming load.


\item Software can easily discover the number of tags loaded by an
 implementation by storing a series of \textbf{XLEN} capabilities to an aligned
 array and performing a \hyperref[sailRISCVzCLoadTags]{\lstinline{CLoadTags}} operation. This need only be done once.


\item For heterogeneous multi-core or multi-processor systems, all cores must
 return the same number of tags, which will often be based on the smallest
 cache line size in the system.


\item Unlike \hyperref[sailRISCVzLoadCapImm]{CLC}, this instruction traps if tags will always be
 unset due to lacking \textbf{Permit\_Load\_Capability} or page table entry
 permissions, since that is likely indicative of a software bug that could
 lead to temporal safety vulnerabilities if capabilities are erroneously
 missed.


\end{itemize}
}{\lstinputlisting[language=sail]{sail_latex_riscv/fclCLoadTagszexecute33a689e3a631b9b905b85461d3814943.tex}}}}

\newcommand{\sailRISCVfclLoadResDataDDCexecute}{\saildoclabelled{sailRISCVfclLoadResDataDDCzexecute}{\saildocfcl{}{\lstinputlisting[language=sail]{sail_latex_riscv/fclLoadResDataDDCzexecute33a689e3a631b9b905b85461d3814943.tex}}}}

\newcommand{\sailRISCVfclLoadResCapDDCexecute}{\saildoclabelled{sailRISCVfclLoadResCapDDCzexecute}{\saildocfcl{}{\lstinputlisting[language=sail]{sail_latex_riscv/fclLoadResCapDDCzexecute33a689e3a631b9b905b85461d3814943.tex}}}}

\newcommand{\sailRISCVfclLoadResCapexecute}{\saildoclabelled{sailRISCVfclLoadResCapzexecute}{\saildocfcl{}{\lstinputlisting[language=sail]{sail_latex_riscv/fclLoadResCapzexecute33a689e3a631b9b905b85461d3814943.tex}}}}

\newcommand{\sailRISCVfclLoadResCapCapexecute}{\saildoclabelled{sailRISCVfclLoadResCapCapzexecute}{\saildocfcl{}{\lstinputlisting[language=sail]{sail_latex_riscv/fclLoadResCapCapzexecute33a689e3a631b9b905b85461d3814943.tex}}}}

\newcommand{\sailRISCVfclLoadResCapModeexecute}{\saildoclabelled{sailRISCVfclLoadResCapModezexecute}{\saildocfcl{}{\lstinputlisting[language=sail]{sail_latex_riscv/fclLoadResCapModezexecute33a689e3a631b9b905b85461d3814943.tex}}}}

\newcommand{\sailRISCVfclStoreDataDDCexecute}{\saildoclabelled{sailRISCVfclStoreDataDDCzexecute}{\saildocfcl{The byte, halfword, word or doubleword located in memory at
\textbf{DDC}.\textbf{address} $+$ \emph{rs1} is replaced with integer register \emph{rs2}.

\subsection*{Exceptions}


An exception is raised if:

\begin{itemize}
\item \textbf{DDC}.\textbf{tag} is not set.
\item \textbf{DDC} is sealed.
\item \textbf{DDC}.\textbf{perms} does not grant \textbf{Permit\_Store}.
\item \textbf{DDC}.\textbf{address} $+$ \emph{rs1} $\lt$ \textbf{DDC}.\textbf{base}.
\item \textbf{DDC}.\textbf{address} $+$ \emph{rs1} $+$ \emph{size} $\gt$ \textbf{DDC}.\textbf{top}.
\end{itemize}
}{\lstinputlisting[language=sail]{sail_latex_riscv/fclStoreDataDDCzexecute33a689e3a631b9b905b85461d3814943.tex}}}}

\newcommand{\sailRISCVfclStoreDataCapexecute}{\saildoclabelled{sailRISCVfclStoreDataCapzexecute}{\saildocfcl{The byte, halfword, word or doubleword located in memory at
\emph{cs1}.\textbf{address} is replaced with integer register \emph{rs2}.

\subsection*{Exceptions}


An exception is raised if:

\begin{itemize}
\item \emph{cs1}.\textbf{tag} is not set.
\item \emph{cs1} is sealed.
\item \emph{cs1}.\textbf{perms} does not grant \textbf{Permit\_Store}.
\item \emph{cs1}.\textbf{address} $\lt$ \emph{cs1}.\textbf{base}.
\item \emph{cs1}.\textbf{address} $+$ \emph{size} $\gt$ \emph{cs1}.\textbf{top}.
\end{itemize}
}{\lstinputlisting[language=sail]{sail_latex_riscv/fclStoreDataCapzexecute33a689e3a631b9b905b85461d3814943.tex}}}}

\newcommand{\sailRISCVfclStoreCapDDCexecute}{\saildoclabelled{sailRISCVfclStoreCapDDCzexecute}{\saildocfcl{The capability located in memory at \textbf{DDC}.\textbf{address} $+$ \emph{rs1} is
replaced with capability register \emph{cs2}.

\subsection*{Exceptions}


An exception is raised if:

\begin{itemize}
\item \textbf{DDC}.\textbf{tag} is not set.
\item \textbf{DDC} is sealed.
\item \textbf{DDC}.\textbf{perms} does not grant \textbf{Permit\_Store}.
\item \textbf{DDC}.\textbf{perms} does not grant \textbf{Permit\_Store\_Capability} and
 \emph{cs2}.\textbf{tag} is set.
\item \textbf{DDC}.\textbf{perms} does not grant \textbf{Permit\_Store\_Local\_Capability},
 \emph{cs2}.\textbf{tag} is set and \emph{cs2}.\textbf{perms} does not grant \textbf{Global}.
\item \textbf{DDC}.\textbf{address} $+$ \emph{rs1} $\lt$ \textbf{DDC}.\textbf{base}.
\item \textbf{DDC}.\textbf{address} $+$ \emph{rs1} $+$ \textbf{CLEN} $/$ 8 $\gt$ \textbf{DDC}.\textbf{top}.
\end{itemize}
}{\lstinputlisting[language=sail]{sail_latex_riscv/fclStoreCapDDCzexecute33a689e3a631b9b905b85461d3814943.tex}}}}

\newcommand{\sailRISCVfclStoreCapCapexecute}{\saildoclabelled{sailRISCVfclStoreCapCapzexecute}{\saildocfcl{The capability located in memory at \emph{cs1}.\textbf{address} is replaced with
capability register \emph{cs2}.

\subsection*{Exceptions}


An exception is raised if:

\begin{itemize}
\item \emph{cs1}.\textbf{tag} is not set.
\item \emph{cs1} is sealed.
\item \emph{cs1}.\textbf{perms} does not grant \textbf{Permit\_Store}.
\item \emph{cs1}.\textbf{perms} does not grant \textbf{Permit\_Store\_Capability} and
 \emph{cs2}.\textbf{tag} is set.
\item \emph{cs1}.\textbf{perms} does not grant \textbf{Permit\_Store\_Local\_Capability},
 \emph{cs2}.\textbf{tag} is set and \emph{cs2}.\textbf{perms} does not grant \textbf{Global}.
\item \emph{cs1}.\textbf{address} $\lt$ \emph{cs1}.\textbf{base}.
\item \emph{cs1}.\textbf{address} $+$ \textbf{CLEN} $\gt$ \emph{cs1}.\textbf{top}.
\end{itemize}
}{\lstinputlisting[language=sail]{sail_latex_riscv/fclStoreCapCapzexecute33a689e3a631b9b905b85461d3814943.tex}}}}

\newcommand{\sailRISCVfclLoadCapImmexecute}{\saildoclabelled{sailRISCVfclLoadCapImmzexecute}{\saildocfcl{In integer mode, capability register \emph{cd} is replaced with the capability
located in memory at \textbf{DDC}.\textbf{address} $+$ \emph{rs1} $+$ \emph{imm}, and if
\textbf{DDC}.\textbf{perms} does not grant \textbf{Permit\_Load\_Capability} then
\emph{cd}.\textbf{tag} is cleared. In capability mode, capability register \emph{cd} is
replaced with the capability located in memory at \emph{cs1}.\textbf{address} $+$
\emph{imm}, and if \emph{cs1}.\textbf{perms} does not grant \textbf{Permit\_Load\_Capability} then
\emph{cd}.\textbf{tag} is cleared.

\subsection*{Exceptions}


In integer mode, an exception is raised if:

\begin{itemize}
\item \textbf{DDC}.\textbf{tag} is not set.
\item \textbf{DDC} is sealed.
\item \textbf{DDC}.\textbf{perms} does not grant \textbf{Permit\_Load}.
\item \textbf{DDC}.\textbf{address} $+$ \emph{rs1} $+$ \emph{imm} $\lt$ \textbf{DDC}.\textbf{base}.
\item \textbf{DDC}.\textbf{address} $+$ \emph{rs1} $+$ \emph{imm} $+$ \textbf{CLEN} $/$ 8 $\gt$
 \textbf{DDC}.\textbf{top}.
\item \textbf{DDC}.\textbf{address} $+$ \emph{rs1} $+$ \emph{imm} is unaligned, regardless of
 whether the implementation supports unaligned data accesses.
\end{itemize}


In capability mode, an exception is raised if:

\begin{itemize}
\item \emph{cs1}.\textbf{tag} is not set.
\item \emph{cs1} is sealed.
\item \emph{cs1}.\textbf{perms} does not grant \textbf{Permit\_Load}.
\item \emph{cs1}.\textbf{address} $+$ \emph{imm} $\lt$ \emph{cs1}.\textbf{base}.
\item \emph{cs1}.\textbf{address} $+$ \emph{imm} $+$ \textbf{CLEN} $/$ 8 $\gt$ \emph{cs1}.\textbf{top}.
\item \emph{cs1}.\textbf{address} $+$ \emph{imm} is unaligned, regardless of whether the
 implementation supports unaligned data accesses.
\end{itemize}
}{\lstinputlisting[language=sail]{sail_latex_riscv/fclLoadCapImmzexecute33a689e3a631b9b905b85461d3814943.tex}}}}

\newcommand{\sailRISCVfclStoreCapImmexecute}{\saildoclabelled{sailRISCVfclStoreCapImmzexecute}{\saildocfcl{In integer mode, the capability located in memory at \textbf{DDC}.\textbf{address} $+$
\emph{rs1} $+$ \emph{imm} is replaced with capability register \emph{cs2}. In capability
mode, the capability located in memory at \emph{cs1}.\textbf{address} $+$ \emph{imm} is
replaced with capability register \emph{cs2}.

\subsection*{Exceptions}


In integer mode, an exception is raised if:

\begin{itemize}
\item \textbf{DDC}.\textbf{tag} is not set.
\item \textbf{DDC} is sealed.
\item \textbf{DDC}.\textbf{perms} does not grant \textbf{Permit\_Store}.
\item \textbf{DDC}.\textbf{perms} does not grant \textbf{Permit\_Store\_Capability} and
 \emph{cs2}.\textbf{tag} is set.
\item \textbf{DDC}.\textbf{perms} does not grant \textbf{Permit\_Store\_Local\_Capability},
 \emph{cs2}.\textbf{tag} is set and \emph{cs2}.\textbf{perms} does not grant \textbf{Global}.
\item \textbf{DDC}.\textbf{address} $+$ \emph{rs1} $+$ \emph{imm} $\lt$ \textbf{DDC}.\textbf{base}.
\item \textbf{DDC}.\textbf{address} $+$ \emph{rs1} $+$ \emph{imm} $+$ \textbf{CLEN} $/$ 8 $\gt$
 \textbf{DDC}.\textbf{top}.
\end{itemize}


In capability mode, an exception is raised if:

\begin{itemize}
\item \emph{cs1}.\textbf{tag} is not set.
\item \emph{cs1} is sealed.
\item \emph{cs1}.\textbf{perms} does not grant \textbf{Permit\_Store}.
\item \emph{cs1}.\textbf{perms} does not grant \textbf{Permit\_Store\_Capability} and
 \emph{cs2}.\textbf{tag} is set.
\item \emph{cs1}.\textbf{perms} does not grant \textbf{Permit\_Store\_Local\_Capability},
 \emph{cs2}.\textbf{tag} is set and \emph{cs2}.\textbf{perms} does not grant \textbf{Global}.
\item \emph{cs1}.\textbf{address} $+$ \emph{imm} $\lt$ \emph{cs1}.\textbf{base}.
\item \emph{cs1}.\textbf{address} $+$ \emph{imm} $+$ \textbf{CLEN} $/$ 8 $\gt$ \emph{cs1}.\textbf{top}.
\end{itemize}
}{\lstinputlisting[language=sail]{sail_latex_riscv/fclStoreCapImmzexecute33a689e3a631b9b905b85461d3814943.tex}}}}

\newcommand{\sailRISCVfclStoreCondDataDDCexecute}{\saildoclabelled{sailRISCVfclStoreCondDataDDCzexecute}{\saildocfcl{}{\lstinputlisting[language=sail]{sail_latex_riscv/fclStoreCondDataDDCzexecute33a689e3a631b9b905b85461d3814943.tex}}}}

\newcommand{\sailRISCVfclStoreCondCapDDCexecute}{\saildoclabelled{sailRISCVfclStoreCondCapDDCzexecute}{\saildocfcl{}{\lstinputlisting[language=sail]{sail_latex_riscv/fclStoreCondCapDDCzexecute33a689e3a631b9b905b85461d3814943.tex}}}}

\newcommand{\sailRISCVfclStoreCondCapexecute}{\saildoclabelled{sailRISCVfclStoreCondCapzexecute}{\saildocfcl{}{\lstinputlisting[language=sail]{sail_latex_riscv/fclStoreCondCapzexecute33a689e3a631b9b905b85461d3814943.tex}}}}

\newcommand{\sailRISCVfclStoreCondCapCapexecute}{\saildoclabelled{sailRISCVfclStoreCondCapCapzexecute}{\saildocfcl{}{\lstinputlisting[language=sail]{sail_latex_riscv/fclStoreCondCapCapzexecute33a689e3a631b9b905b85461d3814943.tex}}}}

\newcommand{\sailRISCVfclStoreCondCapModeexecute}{\saildoclabelled{sailRISCVfclStoreCondCapModezexecute}{\saildocfcl{}{\lstinputlisting[language=sail]{sail_latex_riscv/fclStoreCondCapModezexecute33a689e3a631b9b905b85461d3814943.tex}}}}

\newcommand{\sailRISCVfclAMOSwapCapexecute}{\saildoclabelled{sailRISCVfclAMOSwapCapzexecute}{\saildocfcl{}{\lstinputlisting[language=sail]{sail_latex_riscv/fclAMOSwapCapzexecute33a689e3a631b9b905b85461d3814943.tex}}}}

\newcommand{\sailRISCVfclRISCVUnderscoreJALRexecute}{\saildoclabelled{sailRISCVfclRISCVUnderscoreJALRzexecute}{\saildocfcl{}{\lstinputlisting[language=sail]{sail_latex_riscv/fclRISCVUnderscoreJALRzexecute33a689e3a631b9b905b85461d3814943.tex}}}}

\newcommand{\sailRISCVfclILLEGALexecute}{\saildoclabelled{sailRISCVfclILLEGALzexecute}{\saildocfcl{}{\lstinputlisting[language=sail]{sail_latex_riscv/fclILLEGALzexecute33a689e3a631b9b905b85461d3814943.tex}}}}

\newcommand{\sailRISCVfclCUnderscoreILLEGALexecute}{\saildoclabelled{sailRISCVfclCUnderscoreILLEGALzexecute}{\saildocfcl{}{\lstinputlisting[language=sail]{sail_latex_riscv/fclCUnderscoreILLEGALzexecute33a689e3a631b9b905b85461d3814943.tex}}}}



\newcommand{\sailRISCVvalprintInsn}{\saildoclabelled{sailRISCVzprintzyinsn}{\saildocval{}{\lstinputlisting[language=sail]{sail_latex_riscv/valzprint_insn34adb9871c343ddeeb08d9e768ad4c92.tex}}}}

\newcommand{\sailRISCVfnprintInsn}{\saildoclabelled{sailRISCVfnzprintzyinsn}{\saildocfn{}{\lstinputlisting[language=sail]{sail_latex_riscv/fnzprint_insn34adb9871c343ddeeb08d9e768ad4c92.tex}}}}

\newcommand{\sailRISCVoverloadUUUtoStr}{\saildoclabelled{sailRISCVoverloadUUUztozystr}{\saildocoverload{}{\lstinputlisting[language=sail]{sail_latex_riscv/overloadUUUzto_str8b7a6895ae35945bd4740e9f790c43ee.tex}}}}

\newcommand{\sailRISCVvaldecode}{\saildoclabelled{sailRISCVzdecode}{\saildocval{}{\lstinputlisting[language=sail]{sail_latex_riscv/valzdecodec2b8e713fb767e340b7b488fdcba3aab.tex}}}}

\newcommand{\sailRISCVfndecode}{\saildoclabelled{sailRISCVfnzdecode}{\saildocfn{}{\lstinputlisting[language=sail]{sail_latex_riscv/fnzdecodec2b8e713fb767e340b7b488fdcba3aab.tex}}}}

\newcommand{\sailRISCVvaldecodeCompressed}{\saildoclabelled{sailRISCVzdecodeCompressed}{\saildocval{}{\lstinputlisting[language=sail]{sail_latex_riscv/valzdecodecompresseda0c5feab498dd29a69efe0353146e981.tex}}}}

\newcommand{\sailRISCVfndecodeCompressed}{\saildoclabelled{sailRISCVfnzdecodeCompressed}{\saildocfn{}{\lstinputlisting[language=sail]{sail_latex_riscv/fnzdecodecompresseda0c5feab498dd29a69efe0353146e981.tex}}}}

\newcommand{\sailRISCVtypeFetchResult}{\saildoclabelled{sailRISCVtypezFetchResult}{\saildoctype{}{\lstinputlisting[language=sail]{sail_latex_riscv/typezfetchresult170a566bc6b6cd6bbb373e14725091ab.tex}}}}

\newcommand{\sailRISCVvalextInit}{\saildoclabelled{sailRISCVzextzyinit}{\saildocval{}{\lstinputlisting[language=sail]{sail_latex_riscv/valzext_initaf8e3807fa5c1bbef01331f40e0f99a4.tex}}}}

\newcommand{\sailRISCVfnextInit}{\saildoclabelled{sailRISCVfnzextzyinit}{\saildocfn{}{\lstinputlisting[language=sail]{sail_latex_riscv/fnzext_initaf8e3807fa5c1bbef01331f40e0f99a4.tex}}}}

\newcommand{\sailRISCVvalextFetchHook}{\saildoclabelled{sailRISCVzextzyfetchzyhook}{\saildocval{}{\lstinputlisting[language=sail]{sail_latex_riscv/valzext_fetch_hookb78ddd15c7be769c4a0783ef122b9767.tex}}}}

\newcommand{\sailRISCVfnextFetchHook}{\saildoclabelled{sailRISCVfnzextzyfetchzyhook}{\saildocfn{}{\lstinputlisting[language=sail]{sail_latex_riscv/fnzext_fetch_hookb78ddd15c7be769c4a0783ef122b9767.tex}}}}

\newcommand{\sailRISCVvalextPreStepHook}{\saildoclabelled{sailRISCVzextzyprezystepzyhook}{\saildocval{}{\lstinputlisting[language=sail]{sail_latex_riscv/valzext_pre_step_hookbaa423b1418b5f3048ae14ba522eabb5.tex}}}}

\newcommand{\sailRISCVfnextPreStepHook}{\saildoclabelled{sailRISCVfnzextzyprezystepzyhook}{\saildocfn{}{\lstinputlisting[language=sail]{sail_latex_riscv/fnzext_pre_step_hookbaa423b1418b5f3048ae14ba522eabb5.tex}}}}

\newcommand{\sailRISCVvalextPostStepHook}{\saildoclabelled{sailRISCVzextzypostzystepzyhook}{\saildocval{}{\lstinputlisting[language=sail]{sail_latex_riscv/valzext_post_step_hook12041b61939f7dc96fabf0eb4cecd40e.tex}}}}

\newcommand{\sailRISCVfnextPostStepHook}{\saildoclabelled{sailRISCVfnzextzypostzystepzyhook}{\saildocfn{}{\lstinputlisting[language=sail]{sail_latex_riscv/fnzext_post_step_hook12041b61939f7dc96fabf0eb4cecd40e.tex}}}}

\newcommand{\sailRISCVvalextPostDecodeHook}{\saildoclabelled{sailRISCVzextzypostzydecodezyhook}{\saildocval{}{\lstinputlisting[language=sail]{sail_latex_riscv/valzext_post_decode_hook0ea81b52fdd64a9c28fe27bce7cc93bb.tex}}}}

\newcommand{\sailRISCVfnextPostDecodeHook}{\saildoclabelled{sailRISCVfnzextzypostzydecodezyhook}{\saildocfn{}{\lstinputlisting[language=sail]{sail_latex_riscv/fnzext_post_decode_hook0ea81b52fdd64a9c28fe27bce7cc93bb.tex}}}}

\newcommand{\sailRISCVvalisRVC}{\saildoclabelled{sailRISCVzisRVC}{\saildocval{}{\lstinputlisting[language=sail]{sail_latex_riscv/valzisrvcd64dda5fc24513d78f480d3583dee004.tex}}}}

\newcommand{\sailRISCVfnisRVC}{\saildoclabelled{sailRISCVfnzisRVC}{\saildocfn{}{\lstinputlisting[language=sail]{sail_latex_riscv/fnzisrvcd64dda5fc24513d78f480d3583dee004.tex}}}}

\newcommand{\sailRISCVvalfetch}{\saildoclabelled{sailRISCVzfetch}{\saildocval{}{\lstinputlisting[language=sail]{sail_latex_riscv/valzfetch5e1d71b1ad15beedbd2dacb5ddbcd2b6.tex}}}}

\newcommand{\sailRISCVfnfetch}{\saildoclabelled{sailRISCVfnzfetch}{\saildocfn{}{\lstinputlisting[language=sail]{sail_latex_riscv/fnzfetch5e1d71b1ad15beedbd2dacb5ddbcd2b6.tex}}}}

\newcommand{\sailRISCVvalstep}{\saildoclabelled{sailRISCVzstep}{\saildocval{}{\lstinputlisting[language=sail]{sail_latex_riscv/valzstepb001f84c8bf78b78b44b98d2b7f1f7d7.tex}}}}

\newcommand{\sailRISCVfnstep}{\saildoclabelled{sailRISCVfnzstep}{\saildocfn{}{\lstinputlisting[language=sail]{sail_latex_riscv/fnzstepb001f84c8bf78b78b44b98d2b7f1f7d7.tex}}}}

\newcommand{\sailRISCVvalloop}{\saildoclabelled{sailRISCVzloop}{\saildocval{}{\lstinputlisting[language=sail]{sail_latex_riscv/valzloop939222c31aec6e03415219d7c5a4ee7a.tex}}}}

\newcommand{\sailRISCVfnloop}{\saildoclabelled{sailRISCVfnzloop}{\saildocfn{}{\lstinputlisting[language=sail]{sail_latex_riscv/fnzloop939222c31aec6e03415219d7c5a4ee7a.tex}}}}

\newcommand{\sailRISCVvalinitModel}{\saildoclabelled{sailRISCVzinitzymodel}{\saildocval{}{\lstinputlisting[language=sail]{sail_latex_riscv/valzinit_model2343c87c630c8dc589bf21c69bd047d3.tex}}}}

\newcommand{\sailRISCVfninitModel}{\saildoclabelled{sailRISCVfnzinitzymodel}{\saildocfn{}{\lstinputlisting[language=sail]{sail_latex_riscv/fnzinit_model2343c87c630c8dc589bf21c69bd047d3.tex}}}}

\newcommand{\sailRISCVletGPRstrs}{\saildoclabelled{sailRISCVletzGPRstrs}{\saildoclet{}{\lstinputlisting[language=sail]{sail_latex_riscv/letzgprstrsf6c1c171605e33b57e641a43684ba7c7.tex}}}}

\newcommand{\sailRISCVvalGPRstr}{\saildoclabelled{sailRISCVzGPRstr}{\saildocval{}{\lstinputlisting[language=sail]{sail_latex_riscv/valzgprstr8a694e087c131a3070bde7e75ad3a570.tex}}}}

\newcommand{\sailRISCVfnGPRstr}{\saildoclabelled{sailRISCVfnzGPRstr}{\saildocfn{}{\lstinputlisting[language=sail]{sail_latex_riscv/fnzgprstr8a694e087c131a3070bde7e75ad3a570.tex}}}}

\newcommand{\sailRISCVletCIAFp}{\saildoclabelled{sailRISCVletzCIAzyfp}{\saildoclet{}{\lstinputlisting[language=sail]{sail_latex_riscv/letzcia_fp476de1e96cc36dc8fb254f31212ff378.tex}}}}

\newcommand{\sailRISCVletNIAFp}{\saildoclabelled{sailRISCVletzNIAzyfp}{\saildoclet{}{\lstinputlisting[language=sail]{sail_latex_riscv/letznia_fpb4e6847a1c22c2b1b88d46df558f490d.tex}}}}

\newcommand{\sailRISCVvalinitialAnalysis}{\saildoclabelled{sailRISCVzinitialzyanalysis}{\saildocval{}{\lstinputlisting[language=sail]{sail_latex_riscv/valzinitial_analysis58ef2bf9252095b4ead796191551d1ec.tex}}}}

\newcommand{\sailRISCVfninitialAnalysis}{\saildoclabelled{sailRISCVfnzinitialzyanalysis}{\saildocfn{}{\lstinputlisting[language=sail]{sail_latex_riscv/fnzinitial_analysis58ef2bf9252095b4ead796191551d1ec.tex}}}}

\newcommand{\sailRISCVval}[1]{
  \ifstrequal{#1}{Architecture_of_num}{\sailRISCVvalArchitectureOfNum}{}%
  \ifstrequal{#1}{Architecture\_of\_num}{\sailRISCVvalArchitectureOfNum}{}%
  \ifstrequal{#1}{CPtrCmpOp_of_num}{\sailRISCVvalCPtrCmpOpOfNum}{}%
  \ifstrequal{#1}{CPtrCmpOp\_of\_num}{\sailRISCVvalCPtrCmpOpOfNum}{}%
  \ifstrequal{#1}{CapExCode}{\sailRISCVvalCapExCode}{}%
  \ifstrequal{#1}{CapEx_of_num}{\sailRISCVvalCapExOfNum}{}%
  \ifstrequal{#1}{CapEx\_of\_num}{\sailRISCVvalCapExOfNum}{}%
  \ifstrequal{#1}{ClearRegSet_of_num}{\sailRISCVvalClearRegSetOfNum}{}%
  \ifstrequal{#1}{ClearRegSet\_of\_num}{\sailRISCVvalClearRegSetOfNum}{}%
  \ifstrequal{#1}{EXTS}{\sailRISCVvalEXTS}{}%
  \ifstrequal{#1}{EXTZ}{\sailRISCVvalEXTZ}{}%
  \ifstrequal{#1}{ExtStatus_of_num}{\sailRISCVvalExtStatusOfNum}{}%
  \ifstrequal{#1}{ExtStatus\_of\_num}{\sailRISCVvalExtStatusOfNum}{}%
  \ifstrequal{#1}{FRegStr}{\sailRISCVvalFRegStr}{}%
  \ifstrequal{#1}{GPRstr}{\sailRISCVvalGPRstr}{}%
  \ifstrequal{#1}{InterruptType_of_num}{\sailRISCVvalInterruptTypeOfNum}{}%
  \ifstrequal{#1}{InterruptType\_of\_num}{\sailRISCVvalInterruptTypeOfNum}{}%
  \ifstrequal{#1}{MAX}{\sailRISCVvalMAX}{}%
  \ifstrequal{#1}{MEMr_tag}{\sailRISCVvalMEMrTag}{}%
  \ifstrequal{#1}{MEMr\_tag}{\sailRISCVvalMEMrTag}{}%
  \ifstrequal{#1}{MEMw_tag}{\sailRISCVvalMEMwTag}{}%
  \ifstrequal{#1}{MEMw\_tag}{\sailRISCVvalMEMwTag}{}%
  \ifstrequal{#1}{MemoryOpResult_add_meta}{\sailRISCVvalMemoryOpResultAddMeta}{}%
  \ifstrequal{#1}{MemoryOpResult\_add\_meta}{\sailRISCVvalMemoryOpResultAddMeta}{}%
  \ifstrequal{#1}{MemoryOpResult_drop_meta}{\sailRISCVvalMemoryOpResultDropMeta}{}%
  \ifstrequal{#1}{MemoryOpResult\_drop\_meta}{\sailRISCVvalMemoryOpResultDropMeta}{}%
  \ifstrequal{#1}{PmpAddrMatchType_of_num}{\sailRISCVvalPmpAddrMatchTypeOfNum}{}%
  \ifstrequal{#1}{PmpAddrMatchType\_of\_num}{\sailRISCVvalPmpAddrMatchTypeOfNum}{}%
  \ifstrequal{#1}{Privilege_of_num}{\sailRISCVvalPrivilegeOfNum}{}%
  \ifstrequal{#1}{Privilege\_of\_num}{\sailRISCVvalPrivilegeOfNum}{}%
  \ifstrequal{#1}{RegStr}{\sailRISCVvalRegStr}{}%
  \ifstrequal{#1}{Retired_of_num}{\sailRISCVvalRetiredOfNum}{}%
  \ifstrequal{#1}{Retired\_of\_num}{\sailRISCVvalRetiredOfNum}{}%
  \ifstrequal{#1}{SATPMode_of_num}{\sailRISCVvalSATPModeOfNum}{}%
  \ifstrequal{#1}{SATPMode\_of\_num}{\sailRISCVvalSATPModeOfNum}{}%
  \ifstrequal{#1}{TrapVectorMode_of_num}{\sailRISCVvalTrapVectorModeOfNum}{}%
  \ifstrequal{#1}{TrapVectorMode\_of\_num}{\sailRISCVvalTrapVectorModeOfNum}{}%
  \ifstrequal{#1}{__ReadRAM_Meta}{\sailRISCVvalReadRAMMeta}{}%
  \ifstrequal{#1}{\_\_ReadRAM\_Meta}{\sailRISCVvalReadRAMMeta}{}%
  \ifstrequal{#1}{__TraceMemoryRead}{\sailRISCVvalTraceMemoryRead}{}%
  \ifstrequal{#1}{\_\_TraceMemoryRead}{\sailRISCVvalTraceMemoryRead}{}%
  \ifstrequal{#1}{__TraceMemoryWrite}{\sailRISCVvalTraceMemoryWrite}{}%
  \ifstrequal{#1}{\_\_TraceMemoryWrite}{\sailRISCVvalTraceMemoryWrite}{}%
  \ifstrequal{#1}{__WriteRAM_Meta}{\sailRISCVvalWriteRAMMeta}{}%
  \ifstrequal{#1}{\_\_WriteRAM\_Meta}{\sailRISCVvalWriteRAMMeta}{}%
  \ifstrequal{#1}{__barrier}{\sailRISCVvalBarrier}{}%
  \ifstrequal{#1}{\_\_barrier}{\sailRISCVvalBarrier}{}%
  \ifstrequal{#1}{__branch_announce}{\sailRISCVvalBranchAnnounce}{}%
  \ifstrequal{#1}{\_\_branch\_announce}{\sailRISCVvalBranchAnnounce}{}%
  \ifstrequal{#1}{__cache_maintenance}{\sailRISCVvalCacheMaintenance}{}%
  \ifstrequal{#1}{\_\_cache\_maintenance}{\sailRISCVvalCacheMaintenance}{}%
  \ifstrequal{#1}{__deref}{\sailRISCVvalDeref}{}%
  \ifstrequal{#1}{\_\_deref}{\sailRISCVvalDeref}{}%
  \ifstrequal{#1}{__excl_res}{\sailRISCVvalExclRes}{}%
  \ifstrequal{#1}{\_\_excl\_res}{\sailRISCVvalExclRes}{}%
  \ifstrequal{#1}{__id}{\sailRISCVvalId}{}%
  \ifstrequal{#1}{\_\_id}{\sailRISCVvalId}{}%
  \ifstrequal{#1}{__instr_announce}{\sailRISCVvalInstrAnnounce}{}%
  \ifstrequal{#1}{\_\_instr\_announce}{\sailRISCVvalInstrAnnounce}{}%
  \ifstrequal{#1}{__read_mem}{\sailRISCVvalReadMem}{}%
  \ifstrequal{#1}{\_\_read\_mem}{\sailRISCVvalReadMem}{}%
  \ifstrequal{#1}{__read_memt}{\sailRISCVvalReadMemt}{}%
  \ifstrequal{#1}{\_\_read\_memt}{\sailRISCVvalReadMemt}{}%
  \ifstrequal{#1}{__write_mem}{\sailRISCVvalWriteMem}{}%
  \ifstrequal{#1}{\_\_write\_mem}{\sailRISCVvalWriteMem}{}%
  \ifstrequal{#1}{__write_mem_ea}{\sailRISCVvalWriteMemEa}{}%
  \ifstrequal{#1}{\_\_write\_mem\_ea}{\sailRISCVvalWriteMemEa}{}%
  \ifstrequal{#1}{__write_memt}{\sailRISCVvalWriteMemt}{}%
  \ifstrequal{#1}{\_\_write\_memt}{\sailRISCVvalWriteMemt}{}%
  \ifstrequal{#1}{__write_tag}{\sailRISCVvalWriteTag}{}%
  \ifstrequal{#1}{\_\_write\_tag}{\sailRISCVvalWriteTag}{}%
  \ifstrequal{#1}{_reg_deref}{\sailRISCVvalRegDeref}{}%
  \ifstrequal{#1}{\_reg\_deref}{\sailRISCVvalRegDeref}{}%
  \ifstrequal{#1}{_shl1}{\sailRISCVvalShlOne}{}%
  \ifstrequal{#1}{\_shl1}{\sailRISCVvalShlOne}{}%
  \ifstrequal{#1}{_shl32}{\sailRISCVvalShlThreeTwo}{}%
  \ifstrequal{#1}{\_shl32}{\sailRISCVvalShlThreeTwo}{}%
  \ifstrequal{#1}{_shl8}{\sailRISCVvalShlEight}{}%
  \ifstrequal{#1}{\_shl8}{\sailRISCVvalShlEight}{}%
  \ifstrequal{#1}{_shl_int}{\sailRISCVvalShlInt}{}%
  \ifstrequal{#1}{\_shl\_int}{\sailRISCVvalShlInt}{}%
  \ifstrequal{#1}{_shl_int_general}{\sailRISCVvalShlIntGeneral}{}%
  \ifstrequal{#1}{\_shl\_int\_general}{\sailRISCVvalShlIntGeneral}{}%
  \ifstrequal{#1}{_shr32}{\sailRISCVvalShrThreeTwo}{}%
  \ifstrequal{#1}{\_shr32}{\sailRISCVvalShrThreeTwo}{}%
  \ifstrequal{#1}{_shr_int}{\sailRISCVvalShrInt}{}%
  \ifstrequal{#1}{\_shr\_int}{\sailRISCVvalShrInt}{}%
  \ifstrequal{#1}{_shr_int_general}{\sailRISCVvalShrIntGeneral}{}%
  \ifstrequal{#1}{\_shr\_int\_general}{\sailRISCVvalShrIntGeneral}{}%
  \ifstrequal{#1}{_tmod_int}{\sailRISCVvalTmodInt}{}%
  \ifstrequal{#1}{\_tmod\_int}{\sailRISCVvalTmodInt}{}%
  \ifstrequal{#1}{_tmod_int_positive}{\sailRISCVvalTmodIntPositive}{}%
  \ifstrequal{#1}{\_tmod\_int\_positive}{\sailRISCVvalTmodIntPositive}{}%
  \ifstrequal{#1}{a64_barrier_domain_of_num}{\sailRISCVvalaSixFourBarrierDomainOfNum}{}%
  \ifstrequal{#1}{a64\_barrier\_domain\_of\_num}{\sailRISCVvalaSixFourBarrierDomainOfNum}{}%
  \ifstrequal{#1}{a64_barrier_type_of_num}{\sailRISCVvalaSixFourBarrierTypeOfNum}{}%
  \ifstrequal{#1}{a64\_barrier\_type\_of\_num}{\sailRISCVvalaSixFourBarrierTypeOfNum}{}%
  \ifstrequal{#1}{abs_int_atom}{\sailRISCVvalabsIntAtom}{}%
  \ifstrequal{#1}{abs\_int\_atom}{\sailRISCVvalabsIntAtom}{}%
  \ifstrequal{#1}{abs_int_plain}{\sailRISCVvalabsIntPlain}{}%
  \ifstrequal{#1}{abs\_int\_plain}{\sailRISCVvalabsIntPlain}{}%
  \ifstrequal{#1}{accessType_to_str}{\sailRISCVvalaccessTypeToStr}{}%
  \ifstrequal{#1}{accessType\_to\_str}{\sailRISCVvalaccessTypeToStr}{}%
  \ifstrequal{#1}{accrue_fflags}{\sailRISCVvalaccrueFflags}{}%
  \ifstrequal{#1}{accrue\_fflags}{\sailRISCVvalaccrueFflags}{}%
  \ifstrequal{#1}{add_atom}{\sailRISCVvaladdAtom}{}%
  \ifstrequal{#1}{add\_atom}{\sailRISCVvaladdAtom}{}%
  \ifstrequal{#1}{add_bits}{\sailRISCVvaladdBits}{}%
  \ifstrequal{#1}{add\_bits}{\sailRISCVvaladdBits}{}%
  \ifstrequal{#1}{add_bits_int}{\sailRISCVvaladdBitsInt}{}%
  \ifstrequal{#1}{add\_bits\_int}{\sailRISCVvaladdBitsInt}{}%
  \ifstrequal{#1}{add_int}{\sailRISCVvaladdInt}{}%
  \ifstrequal{#1}{add\_int}{\sailRISCVvaladdInt}{}%
  \ifstrequal{#1}{add_to_TLB39}{\sailRISCVvaladdToTLBThreeNine}{}%
  \ifstrequal{#1}{add\_to\_TLB39}{\sailRISCVvaladdToTLBThreeNine}{}%
  \ifstrequal{#1}{add_to_TLB48}{\sailRISCVvaladdToTLBFourEight}{}%
  \ifstrequal{#1}{add\_to\_TLB48}{\sailRISCVvaladdToTLBFourEight}{}%
  \ifstrequal{#1}{addr_to_tag_addr}{\sailRISCVvaladdrToTagAddr}{}%
  \ifstrequal{#1}{addr\_to\_tag\_addr}{\sailRISCVvaladdrToTagAddr}{}%
  \ifstrequal{#1}{amo_mnemonic}{\sailRISCVvalamoMnemonic}{}%
  \ifstrequal{#1}{amo\_mnemonic}{\sailRISCVvalamoMnemonic}{}%
  \ifstrequal{#1}{amo_width_valid}{\sailRISCVvalamoWidthValid}{}%
  \ifstrequal{#1}{amo\_width\_valid}{\sailRISCVvalamoWidthValid}{}%
  \ifstrequal{#1}{amoop_of_num}{\sailRISCVvalamoopOfNum}{}%
  \ifstrequal{#1}{amoop\_of\_num}{\sailRISCVvalamoopOfNum}{}%
  \ifstrequal{#1}{and_bool}{\sailRISCVvalandBool}{}%
  \ifstrequal{#1}{and\_bool}{\sailRISCVvalandBool}{}%
  \ifstrequal{#1}{and_bool_no_flow}{\sailRISCVvalandBoolNoFlow}{}%
  \ifstrequal{#1}{and\_bool\_no\_flow}{\sailRISCVvalandBoolNoFlow}{}%
  \ifstrequal{#1}{and_vec}{\sailRISCVvalandVec}{}%
  \ifstrequal{#1}{and\_vec}{\sailRISCVvalandVec}{}%
  \ifstrequal{#1}{any_vector_update}{\sailRISCVvalanyVectorUpdate}{}%
  \ifstrequal{#1}{any\_vector\_update}{\sailRISCVvalanyVectorUpdate}{}%
  \ifstrequal{#1}{append_64}{\sailRISCVvalappendSixFour}{}%
  \ifstrequal{#1}{append\_64}{\sailRISCVvalappendSixFour}{}%
  \ifstrequal{#1}{aqrl_str}{\sailRISCVvalaqrlStr}{}%
  \ifstrequal{#1}{aqrl\_str}{\sailRISCVvalaqrlStr}{}%
  \ifstrequal{#1}{arch_to_bits}{\sailRISCVvalarchToBits}{}%
  \ifstrequal{#1}{arch\_to\_bits}{\sailRISCVvalarchToBits}{}%
  \ifstrequal{#1}{architecture}{\sailRISCVvalarchitecture}{}%
  \ifstrequal{#1}{assembly}{\sailRISCVvalassembly}{}%
  \ifstrequal{#1}{biop_zbs_of_num}{\sailRISCVvalbiopZbsOfNum}{}%
  \ifstrequal{#1}{biop\_zbs\_of\_num}{\sailRISCVvalbiopZbsOfNum}{}%
  \ifstrequal{#1}{bit_maybe_i}{\sailRISCVvalbitMaybeI}{}%
  \ifstrequal{#1}{bit\_maybe\_i}{\sailRISCVvalbitMaybeI}{}%
  \ifstrequal{#1}{bit_maybe_o}{\sailRISCVvalbitMaybeO}{}%
  \ifstrequal{#1}{bit\_maybe\_o}{\sailRISCVvalbitMaybeO}{}%
  \ifstrequal{#1}{bit_maybe_r}{\sailRISCVvalbitMaybeR}{}%
  \ifstrequal{#1}{bit\_maybe\_r}{\sailRISCVvalbitMaybeR}{}%
  \ifstrequal{#1}{bit_maybe_w}{\sailRISCVvalbitMaybeW}{}%
  \ifstrequal{#1}{bit\_maybe\_w}{\sailRISCVvalbitMaybeW}{}%
  \ifstrequal{#1}{bit_to_bool}{\sailRISCVvalbitToBool}{}%
  \ifstrequal{#1}{bit\_to\_bool}{\sailRISCVvalbitToBool}{}%
  \ifstrequal{#1}{bits_str}{\sailRISCVvalbitsStr}{}%
  \ifstrequal{#1}{bits\_str}{\sailRISCVvalbitsStr}{}%
  \ifstrequal{#1}{bitvector_access}{\sailRISCVvalbitvectorAccess}{}%
  \ifstrequal{#1}{bitvector\_access}{\sailRISCVvalbitvectorAccess}{}%
  \ifstrequal{#1}{bitvector_concat}{\sailRISCVvalbitvectorConcat}{}%
  \ifstrequal{#1}{bitvector\_concat}{\sailRISCVvalbitvectorConcat}{}%
  \ifstrequal{#1}{bitvector_length}{\sailRISCVvalbitvectorLength}{}%
  \ifstrequal{#1}{bitvector\_length}{\sailRISCVvalbitvectorLength}{}%
  \ifstrequal{#1}{bitvector_update}{\sailRISCVvalbitvectorUpdate}{}%
  \ifstrequal{#1}{bitvector\_update}{\sailRISCVvalbitvectorUpdate}{}%
  \ifstrequal{#1}{bool_bits}{\sailRISCVvalboolBits}{}%
  \ifstrequal{#1}{bool\_bits}{\sailRISCVvalboolBits}{}%
  \ifstrequal{#1}{bool_not_bits}{\sailRISCVvalboolNotBits}{}%
  \ifstrequal{#1}{bool\_not\_bits}{\sailRISCVvalboolNotBits}{}%
  \ifstrequal{#1}{bool_to_bit}{\sailRISCVvalboolToBit}{}%
  \ifstrequal{#1}{bool\_to\_bit}{\sailRISCVvalboolToBit}{}%
  \ifstrequal{#1}{bool_to_bits}{\sailRISCVvalboolToBits}{}%
  \ifstrequal{#1}{bool\_to\_bits}{\sailRISCVvalboolToBits}{}%
  \ifstrequal{#1}{bop_of_num}{\sailRISCVvalbopOfNum}{}%
  \ifstrequal{#1}{bop\_of\_num}{\sailRISCVvalbopOfNum}{}%
  \ifstrequal{#1}{brop_zba_of_num}{\sailRISCVvalbropZbaOfNum}{}%
  \ifstrequal{#1}{brop\_zba\_of\_num}{\sailRISCVvalbropZbaOfNum}{}%
  \ifstrequal{#1}{brop_zbb_of_num}{\sailRISCVvalbropZbbOfNum}{}%
  \ifstrequal{#1}{brop\_zbb\_of\_num}{\sailRISCVvalbropZbbOfNum}{}%
  \ifstrequal{#1}{brop_zbkb_of_num}{\sailRISCVvalbropZbkbOfNum}{}%
  \ifstrequal{#1}{brop\_zbkb\_of\_num}{\sailRISCVvalbropZbkbOfNum}{}%
  \ifstrequal{#1}{brop_zbs_of_num}{\sailRISCVvalbropZbsOfNum}{}%
  \ifstrequal{#1}{brop\_zbs\_of\_num}{\sailRISCVvalbropZbsOfNum}{}%
  \ifstrequal{#1}{bropw_zba_of_num}{\sailRISCVvalbropwZbaOfNum}{}%
  \ifstrequal{#1}{bropw\_zba\_of\_num}{\sailRISCVvalbropwZbaOfNum}{}%
  \ifstrequal{#1}{bropw_zbb_of_num}{\sailRISCVvalbropwZbbOfNum}{}%
  \ifstrequal{#1}{bropw\_zbb\_of\_num}{\sailRISCVvalbropwZbbOfNum}{}%
  \ifstrequal{#1}{btype_mnemonic}{\sailRISCVvalbtypeMnemonic}{}%
  \ifstrequal{#1}{btype\_mnemonic}{\sailRISCVvalbtypeMnemonic}{}%
  \ifstrequal{#1}{cache_op_kind_of_num}{\sailRISCVvalcacheOpKindOfNum}{}%
  \ifstrequal{#1}{cache\_op\_kind\_of\_num}{\sailRISCVvalcacheOpKindOfNum}{}%
  \ifstrequal{#1}{cancel_reservation}{\sailRISCVvalcancelReservation}{}%
  \ifstrequal{#1}{cancel\_reservation}{\sailRISCVvalcancelReservation}{}%
  \ifstrequal{#1}{canonical_NaN_D}{\sailRISCVvalcanonicalNaND}{}%
  \ifstrequal{#1}{canonical\_NaN\_D}{\sailRISCVvalcanonicalNaND}{}%
  \ifstrequal{#1}{canonical_NaN_H}{\sailRISCVvalcanonicalNaNH}{}%
  \ifstrequal{#1}{canonical\_NaN\_H}{\sailRISCVvalcanonicalNaNH}{}%
  \ifstrequal{#1}{canonical_NaN_S}{\sailRISCVvalcanonicalNaNS}{}%
  \ifstrequal{#1}{canonical\_NaN\_S}{\sailRISCVvalcanonicalNaNS}{}%
  \ifstrequal{#1}{capBitsToCapability}{\sailRISCVvalcapBitsToCapability}{}%
  \ifstrequal{#1}{capBitsToEncCapability}{\sailRISCVvalcapBitsToEncCapability}{}%
  \ifstrequal{#1}{capBoundsEqual}{\sailRISCVvalcapBoundsEqual}{}%
  \ifstrequal{#1}{capToBits}{\sailRISCVvalcapToBits}{}%
  \ifstrequal{#1}{capToEncCap}{\sailRISCVvalcapToEncCap}{}%
  \ifstrequal{#1}{capToMemBits}{\sailRISCVvalcapToMemBits}{}%
  \ifstrequal{#1}{capToString}{\sailRISCVvalcapToString}{}%
  \ifstrequal{#1}{cap_creg_name}{\sailRISCVvalcapCregName}{}%
  \ifstrequal{#1}{cap\_creg\_name}{\sailRISCVvalcapCregName}{}%
  \ifstrequal{#1}{cap_reg_name}{\sailRISCVvalcapRegName}{}%
  \ifstrequal{#1}{cap\_reg\_name}{\sailRISCVvalcapRegName}{}%
  \ifstrequal{#1}{cap_reg_name_abi}{\sailRISCVvalcapRegNameAbi}{}%
  \ifstrequal{#1}{cap\_reg\_name\_abi}{\sailRISCVvalcapRegNameAbi}{}%
  \ifstrequal{#1}{checkPTEPermission}{\sailRISCVvalcheckPTEPermission}{}%
  \ifstrequal{#1}{check_CSR}{\sailRISCVvalcheckCSR}{}%
  \ifstrequal{#1}{check\_CSR}{\sailRISCVvalcheckCSR}{}%
  \ifstrequal{#1}{check_CSR_access}{\sailRISCVvalcheckCSRAccess}{}%
  \ifstrequal{#1}{check\_CSR\_access}{\sailRISCVvalcheckCSRAccess}{}%
  \ifstrequal{#1}{check_Counteren}{\sailRISCVvalcheckCounteren}{}%
  \ifstrequal{#1}{check\_Counteren}{\sailRISCVvalcheckCounteren}{}%
  \ifstrequal{#1}{check_TVM_SATP}{\sailRISCVvalcheckTVMSATP}{}%
  \ifstrequal{#1}{check\_TVM\_SATP}{\sailRISCVvalcheckTVMSATP}{}%
  \ifstrequal{#1}{check_misaligned}{\sailRISCVvalcheckMisaligned}{}%
  \ifstrequal{#1}{check\_misaligned}{\sailRISCVvalcheckMisaligned}{}%
  \ifstrequal{#1}{check_res_misaligned}{\sailRISCVvalcheckResMisaligned}{}%
  \ifstrequal{#1}{check\_res\_misaligned}{\sailRISCVvalcheckResMisaligned}{}%
  \ifstrequal{#1}{check_seed_CSR}{\sailRISCVvalcheckSeedCSR}{}%
  \ifstrequal{#1}{check\_seed\_CSR}{\sailRISCVvalcheckSeedCSR}{}%
  \ifstrequal{#1}{checked_mem_read}{\sailRISCVvalcheckedMemRead}{}%
  \ifstrequal{#1}{checked\_mem\_read}{\sailRISCVvalcheckedMemRead}{}%
  \ifstrequal{#1}{checked_mem_write}{\sailRISCVvalcheckedMemWrite}{}%
  \ifstrequal{#1}{checked\_mem\_write}{\sailRISCVvalcheckedMemWrite}{}%
  \ifstrequal{#1}{clearTag}{\sailRISCVvalclearTag}{}%
  \ifstrequal{#1}{clearTagIf}{\sailRISCVvalclearTagIf}{}%
  \ifstrequal{#1}{clearTagIfSealed}{\sailRISCVvalclearTagIfSealed}{}%
  \ifstrequal{#1}{clint_dispatch}{\sailRISCVvalclintDispatch}{}%
  \ifstrequal{#1}{clint\_dispatch}{\sailRISCVvalclintDispatch}{}%
  \ifstrequal{#1}{clint_load}{\sailRISCVvalclintLoad}{}%
  \ifstrequal{#1}{clint\_load}{\sailRISCVvalclintLoad}{}%
  \ifstrequal{#1}{clint_store}{\sailRISCVvalclintStore}{}%
  \ifstrequal{#1}{clint\_store}{\sailRISCVvalclintStore}{}%
  \ifstrequal{#1}{concat_str}{\sailRISCVvalconcatStr}{}%
  \ifstrequal{#1}{concat\_str}{\sailRISCVvalconcatStr}{}%
  \ifstrequal{#1}{concat_str_bits}{\sailRISCVvalconcatStrBits}{}%
  \ifstrequal{#1}{concat\_str\_bits}{\sailRISCVvalconcatStrBits}{}%
  \ifstrequal{#1}{concat_str_dec}{\sailRISCVvalconcatStrDec}{}%
  \ifstrequal{#1}{concat\_str\_dec}{\sailRISCVvalconcatStrDec}{}%
  \ifstrequal{#1}{count_leading_zeros}{\sailRISCVvalcountLeadingZeros}{}%
  \ifstrequal{#1}{count\_leading\_zeros}{\sailRISCVvalcountLeadingZeros}{}%
  \ifstrequal{#1}{creg2reg_idx}{\sailRISCVvalcregTworegIdx}{}%
  \ifstrequal{#1}{creg2reg\_idx}{\sailRISCVvalcregTworegIdx}{}%
  \ifstrequal{#1}{creg_name}{\sailRISCVvalcregName}{}%
  \ifstrequal{#1}{creg\_name}{\sailRISCVvalcregName}{}%
  \ifstrequal{#1}{csrAccess}{\sailRISCVvalcsrAccess}{}%
  \ifstrequal{#1}{csrPriv}{\sailRISCVvalcsrPriv}{}%
  \ifstrequal{#1}{csr_mnemonic}{\sailRISCVvalcsrMnemonic}{}%
  \ifstrequal{#1}{csr\_mnemonic}{\sailRISCVvalcsrMnemonic}{}%
  \ifstrequal{#1}{csr_name}{\sailRISCVvalcsrName}{}%
  \ifstrequal{#1}{csr\_name}{\sailRISCVvalcsrName}{}%
  \ifstrequal{#1}{csr_name_map}{\sailRISCVvalcsrNameMap}{}%
  \ifstrequal{#1}{csr\_name\_map}{\sailRISCVvalcsrNameMap}{}%
  \ifstrequal{#1}{csrop_of_num}{\sailRISCVvalcsropOfNum}{}%
  \ifstrequal{#1}{csrop\_of\_num}{\sailRISCVvalcsropOfNum}{}%
  \ifstrequal{#1}{curAsid32}{\sailRISCVvalcurAsidThreeTwo}{}%
  \ifstrequal{#1}{curAsid64}{\sailRISCVvalcurAsidSixFour}{}%
  \ifstrequal{#1}{curPTB32}{\sailRISCVvalcurPTBThreeTwo}{}%
  \ifstrequal{#1}{curPTB64}{\sailRISCVvalcurPTBSixFour}{}%
  \ifstrequal{#1}{cur_Architecture}{\sailRISCVvalcurArchitecture}{}%
  \ifstrequal{#1}{cur\_Architecture}{\sailRISCVvalcurArchitecture}{}%
  \ifstrequal{#1}{dec_str}{\sailRISCVvaldecStr}{}%
  \ifstrequal{#1}{dec\_str}{\sailRISCVvaldecStr}{}%
  \ifstrequal{#1}{decimal_string_of_bits}{\sailRISCVvaldecimalStringOfBits}{}%
  \ifstrequal{#1}{decimal\_string\_of\_bits}{\sailRISCVvaldecimalStringOfBits}{}%
  \ifstrequal{#1}{decode}{\sailRISCVvaldecode}{}%
  \ifstrequal{#1}{decodeCompressed}{\sailRISCVvaldecodeCompressed}{}%
  \ifstrequal{#1}{def_spc}{\sailRISCVvaldefSpc}{}%
  \ifstrequal{#1}{def\_spc}{\sailRISCVvaldefSpc}{}%
  \ifstrequal{#1}{def_spc_backwards}{\sailRISCVvaldefSpcBackwards}{}%
  \ifstrequal{#1}{def\_spc\_backwards}{\sailRISCVvaldefSpcBackwards}{}%
  \ifstrequal{#1}{def_spc_forwards}{\sailRISCVvaldefSpcForwards}{}%
  \ifstrequal{#1}{def\_spc\_forwards}{\sailRISCVvaldefSpcForwards}{}%
  \ifstrequal{#1}{def_spc_matches_prefix}{\sailRISCVvaldefSpcMatchesPrefix}{}%
  \ifstrequal{#1}{def\_spc\_matches\_prefix}{\sailRISCVvaldefSpcMatchesPrefix}{}%
  \ifstrequal{#1}{dirty_fd_context}{\sailRISCVvaldirtyFdContext}{}%
  \ifstrequal{#1}{dirty\_fd\_context}{\sailRISCVvaldirtyFdContext}{}%
  \ifstrequal{#1}{dirty_fd_context_if_present}{\sailRISCVvaldirtyFdContextIfPresent}{}%
  \ifstrequal{#1}{dirty\_fd\_context\_if\_present}{\sailRISCVvaldirtyFdContextIfPresent}{}%
  \ifstrequal{#1}{dispatchInterrupt}{\sailRISCVvaldispatchInterrupt}{}%
  \ifstrequal{#1}{dzFlag}{\sailRISCVvaldzzFlag}{}%
  \ifstrequal{#1}{ediv_int}{\sailRISCVvaledivInt}{}%
  \ifstrequal{#1}{ediv\_int}{\sailRISCVvaledivInt}{}%
  \ifstrequal{#1}{effectivePrivilege}{\sailRISCVvaleffectivePrivilege}{}%
  \ifstrequal{#1}{elf_entry}{\sailRISCVvalelfEntry}{}%
  \ifstrequal{#1}{elf\_entry}{\sailRISCVvalelfEntry}{}%
  \ifstrequal{#1}{elf_tohost}{\sailRISCVvalelfTohost}{}%
  \ifstrequal{#1}{elf\_tohost}{\sailRISCVvalelfTohost}{}%
  \ifstrequal{#1}{emod_int}{\sailRISCVvalemodInt}{}%
  \ifstrequal{#1}{emod\_int}{\sailRISCVvalemodInt}{}%
  \ifstrequal{#1}{encCapToBits}{\sailRISCVvalencCapToBits}{}%
  \ifstrequal{#1}{encCapabilityToCapability}{\sailRISCVvalencCapabilityToCapability}{}%
  \ifstrequal{#1}{encdec}{\sailRISCVvalencdec}{}%
  \ifstrequal{#1}{encdec_amoop}{\sailRISCVvalencdecAmoop}{}%
  \ifstrequal{#1}{encdec\_amoop}{\sailRISCVvalencdecAmoop}{}%
  \ifstrequal{#1}{encdec_bop}{\sailRISCVvalencdecBop}{}%
  \ifstrequal{#1}{encdec\_bop}{\sailRISCVvalencdecBop}{}%
  \ifstrequal{#1}{encdec_compressed}{\sailRISCVvalencdecCompressed}{}%
  \ifstrequal{#1}{encdec\_compressed}{\sailRISCVvalencdecCompressed}{}%
  \ifstrequal{#1}{encdec_csrop}{\sailRISCVvalencdecCsrop}{}%
  \ifstrequal{#1}{encdec\_csrop}{\sailRISCVvalencdecCsrop}{}%
  \ifstrequal{#1}{encdec_iop}{\sailRISCVvalencdecIop}{}%
  \ifstrequal{#1}{encdec\_iop}{\sailRISCVvalencdecIop}{}%
  \ifstrequal{#1}{encdec_mul_op}{\sailRISCVvalencdecMulOp}{}%
  \ifstrequal{#1}{encdec\_mul\_op}{\sailRISCVvalencdecMulOp}{}%
  \ifstrequal{#1}{encdec_rounding_mode}{\sailRISCVvalencdecRoundingMode}{}%
  \ifstrequal{#1}{encdec\_rounding\_mode}{\sailRISCVvalencdecRoundingMode}{}%
  \ifstrequal{#1}{encdec_sop}{\sailRISCVvalencdecSop}{}%
  \ifstrequal{#1}{encdec\_sop}{\sailRISCVvalencdecSop}{}%
  \ifstrequal{#1}{encdec_uop}{\sailRISCVvalencdecUop}{}%
  \ifstrequal{#1}{encdec\_uop}{\sailRISCVvalencdecUop}{}%
  \ifstrequal{#1}{eq_anything}{\sailRISCVvaleqAnything}{}%
  \ifstrequal{#1}{eq\_anything}{\sailRISCVvaleqAnything}{}%
  \ifstrequal{#1}{eq_bit}{\sailRISCVvaleqBit}{}%
  \ifstrequal{#1}{eq\_bit}{\sailRISCVvaleqBit}{}%
  \ifstrequal{#1}{eq_bits}{\sailRISCVvaleqBits}{}%
  \ifstrequal{#1}{eq\_bits}{\sailRISCVvaleqBits}{}%
  \ifstrequal{#1}{eq_bool}{\sailRISCVvaleqBool}{}%
  \ifstrequal{#1}{eq\_bool}{\sailRISCVvaleqBool}{}%
  \ifstrequal{#1}{eq_int}{\sailRISCVvaleqInt}{}%
  \ifstrequal{#1}{eq\_int}{\sailRISCVvaleqInt}{}%
  \ifstrequal{#1}{eq_string}{\sailRISCVvaleqString}{}%
  \ifstrequal{#1}{eq\_string}{\sailRISCVvaleqString}{}%
  \ifstrequal{#1}{eq_unit}{\sailRISCVvaleqUnit}{}%
  \ifstrequal{#1}{eq\_unit}{\sailRISCVvaleqUnit}{}%
  \ifstrequal{#1}{exceptionType_to_bits}{\sailRISCVvalexceptionTypeToBits}{}%
  \ifstrequal{#1}{exceptionType\_to\_bits}{\sailRISCVvalexceptionTypeToBits}{}%
  \ifstrequal{#1}{exceptionType_to_str}{\sailRISCVvalexceptionTypeToStr}{}%
  \ifstrequal{#1}{exceptionType\_to\_str}{\sailRISCVvalexceptionTypeToStr}{}%
  \ifstrequal{#1}{exception_delegatee}{\sailRISCVvalexceptionDelegatee}{}%
  \ifstrequal{#1}{exception\_delegatee}{\sailRISCVvalexceptionDelegatee}{}%
  \ifstrequal{#1}{exception_handler}{\sailRISCVvalexceptionHandler}{}%
  \ifstrequal{#1}{exception\_handler}{\sailRISCVvalexceptionHandler}{}%
  \ifstrequal{#1}{execute}{\sailRISCVvalexecute}{}%
  \ifstrequal{#1}{extStatus_of_bits}{\sailRISCVvalextStatusOfBits}{}%
  \ifstrequal{#1}{extStatus\_of\_bits}{\sailRISCVvalextStatusOfBits}{}%
  \ifstrequal{#1}{extStatus_to_bits}{\sailRISCVvalextStatusToBits}{}%
  \ifstrequal{#1}{extStatus\_to\_bits}{\sailRISCVvalextStatusToBits}{}%
  \ifstrequal{#1}{ext_access_type_of_num}{\sailRISCVvalextAccessTypeOfNum}{}%
  \ifstrequal{#1}{ext\_access\_type\_of\_num}{\sailRISCVvalextAccessTypeOfNum}{}%
  \ifstrequal{#1}{ext_check_CSR}{\sailRISCVvalextCheckCSR}{}%
  \ifstrequal{#1}{ext\_check\_CSR}{\sailRISCVvalextCheckCSR}{}%
  \ifstrequal{#1}{ext_check_CSR_fail}{\sailRISCVvalextCheckCSRFail}{}%
  \ifstrequal{#1}{ext\_check\_CSR\_fail}{\sailRISCVvalextCheckCSRFail}{}%
  \ifstrequal{#1}{ext_check_phys_mem_read}{\sailRISCVvalextCheckPhysMemRead}{}%
  \ifstrequal{#1}{ext\_check\_phys\_mem\_read}{\sailRISCVvalextCheckPhysMemRead}{}%
  \ifstrequal{#1}{ext_check_phys_mem_write}{\sailRISCVvalextCheckPhysMemWrite}{}%
  \ifstrequal{#1}{ext\_check\_phys\_mem\_write}{\sailRISCVvalextCheckPhysMemWrite}{}%
  \ifstrequal{#1}{ext_check_xret_priv}{\sailRISCVvalextCheckXretPriv}{}%
  \ifstrequal{#1}{ext\_check\_xret\_priv}{\sailRISCVvalextCheckXretPriv}{}%
  \ifstrequal{#1}{ext_control_check_addr}{\sailRISCVvalextControlCheckAddr}{}%
  \ifstrequal{#1}{ext\_control\_check\_addr}{\sailRISCVvalextControlCheckAddr}{}%
  \ifstrequal{#1}{ext_control_check_pc}{\sailRISCVvalextControlCheckPc}{}%
  \ifstrequal{#1}{ext\_control\_check\_pc}{\sailRISCVvalextControlCheckPc}{}%
  \ifstrequal{#1}{ext_data_get_addr}{\sailRISCVvalextDataGetAddr}{}%
  \ifstrequal{#1}{ext\_data\_get\_addr}{\sailRISCVvalextDataGetAddr}{}%
  \ifstrequal{#1}{ext_exc_type_of_num}{\sailRISCVvalextExcTypeOfNum}{}%
  \ifstrequal{#1}{ext\_exc\_type\_of\_num}{\sailRISCVvalextExcTypeOfNum}{}%
  \ifstrequal{#1}{ext_exc_type_to_bits}{\sailRISCVvalextExcTypeToBits}{}%
  \ifstrequal{#1}{ext\_exc\_type\_to\_bits}{\sailRISCVvalextExcTypeToBits}{}%
  \ifstrequal{#1}{ext_exc_type_to_str}{\sailRISCVvalextExcTypeToStr}{}%
  \ifstrequal{#1}{ext\_exc\_type\_to\_str}{\sailRISCVvalextExcTypeToStr}{}%
  \ifstrequal{#1}{ext_fail_xret_priv}{\sailRISCVvalextFailXretPriv}{}%
  \ifstrequal{#1}{ext\_fail\_xret\_priv}{\sailRISCVvalextFailXretPriv}{}%
  \ifstrequal{#1}{ext_fetch_check_pc}{\sailRISCVvalextFetchCheckPc}{}%
  \ifstrequal{#1}{ext\_fetch\_check\_pc}{\sailRISCVvalextFetchCheckPc}{}%
  \ifstrequal{#1}{ext_fetch_hook}{\sailRISCVvalextFetchHook}{}%
  \ifstrequal{#1}{ext\_fetch\_hook}{\sailRISCVvalextFetchHook}{}%
  \ifstrequal{#1}{ext_get_ptw_error}{\sailRISCVvalextGetPtwError}{}%
  \ifstrequal{#1}{ext\_get\_ptw\_error}{\sailRISCVvalextGetPtwError}{}%
  \ifstrequal{#1}{ext_handle_control_check_error}{\sailRISCVvalextHandleControlCheckError}{}%
  \ifstrequal{#1}{ext\_handle\_control\_check\_error}{\sailRISCVvalextHandleControlCheckError}{}%
  \ifstrequal{#1}{ext_handle_data_check_error}{\sailRISCVvalextHandleDataCheckError}{}%
  \ifstrequal{#1}{ext\_handle\_data\_check\_error}{\sailRISCVvalextHandleDataCheckError}{}%
  \ifstrequal{#1}{ext_handle_fetch_check_error}{\sailRISCVvalextHandleFetchCheckError}{}%
  \ifstrequal{#1}{ext\_handle\_fetch\_check\_error}{\sailRISCVvalextHandleFetchCheckError}{}%
  \ifstrequal{#1}{ext_init}{\sailRISCVvalextInit}{}%
  \ifstrequal{#1}{ext\_init}{\sailRISCVvalextInit}{}%
  \ifstrequal{#1}{ext_init_regs}{\sailRISCVvalextInitRegs}{}%
  \ifstrequal{#1}{ext\_init\_regs}{\sailRISCVvalextInitRegs}{}%
  \ifstrequal{#1}{ext_is_CSR_defined}{\sailRISCVvalextIsCSRDefined}{}%
  \ifstrequal{#1}{ext\_is\_CSR\_defined}{\sailRISCVvalextIsCSRDefined}{}%
  \ifstrequal{#1}{ext_post_decode_hook}{\sailRISCVvalextPostDecodeHook}{}%
  \ifstrequal{#1}{ext\_post\_decode\_hook}{\sailRISCVvalextPostDecodeHook}{}%
  \ifstrequal{#1}{ext_post_step_hook}{\sailRISCVvalextPostStepHook}{}%
  \ifstrequal{#1}{ext\_post\_step\_hook}{\sailRISCVvalextPostStepHook}{}%
  \ifstrequal{#1}{ext_pre_step_hook}{\sailRISCVvalextPreStepHook}{}%
  \ifstrequal{#1}{ext\_pre\_step\_hook}{\sailRISCVvalextPreStepHook}{}%
  \ifstrequal{#1}{ext_ptw_error_of_num}{\sailRISCVvalextPtwErrorOfNum}{}%
  \ifstrequal{#1}{ext\_ptw\_error\_of\_num}{\sailRISCVvalextPtwErrorOfNum}{}%
  \ifstrequal{#1}{ext_ptw_fail_of_num}{\sailRISCVvalextPtwFailOfNum}{}%
  \ifstrequal{#1}{ext\_ptw\_fail\_of\_num}{\sailRISCVvalextPtwFailOfNum}{}%
  \ifstrequal{#1}{ext_ptw_lc_join}{\sailRISCVvalextPtwLcJoin}{}%
  \ifstrequal{#1}{ext\_ptw\_lc\_join}{\sailRISCVvalextPtwLcJoin}{}%
  \ifstrequal{#1}{ext_ptw_lc_of_num}{\sailRISCVvalextPtwLcOfNum}{}%
  \ifstrequal{#1}{ext\_ptw\_lc\_of\_num}{\sailRISCVvalextPtwLcOfNum}{}%
  \ifstrequal{#1}{ext_ptw_sc_join}{\sailRISCVvalextPtwScJoin}{}%
  \ifstrequal{#1}{ext\_ptw\_sc\_join}{\sailRISCVvalextPtwScJoin}{}%
  \ifstrequal{#1}{ext_ptw_sc_of_num}{\sailRISCVvalextPtwScOfNum}{}%
  \ifstrequal{#1}{ext\_ptw\_sc\_of\_num}{\sailRISCVvalextPtwScOfNum}{}%
  \ifstrequal{#1}{ext_read_CSR}{\sailRISCVvalextReadCSR}{}%
  \ifstrequal{#1}{ext\_read\_CSR}{\sailRISCVvalextReadCSR}{}%
  \ifstrequal{#1}{ext_rvfi_init}{\sailRISCVvalextRvfiInit}{}%
  \ifstrequal{#1}{ext\_rvfi\_init}{\sailRISCVvalextRvfiInit}{}%
  \ifstrequal{#1}{ext_veto_disable_C}{\sailRISCVvalextVetoDisableC}{}%
  \ifstrequal{#1}{ext\_veto\_disable\_C}{\sailRISCVvalextVetoDisableC}{}%
  \ifstrequal{#1}{ext_write_CSR}{\sailRISCVvalextWriteCSR}{}%
  \ifstrequal{#1}{ext\_write\_CSR}{\sailRISCVvalextWriteCSR}{}%
  \ifstrequal{#1}{ext_write_fcsr}{\sailRISCVvalextWriteFcsr}{}%
  \ifstrequal{#1}{ext\_write\_fcsr}{\sailRISCVvalextWriteFcsr}{}%
  \ifstrequal{#1}{extend_value}{\sailRISCVvalextendValue}{}%
  \ifstrequal{#1}{extend\_value}{\sailRISCVvalextendValue}{}%
  \ifstrequal{#1}{extern_f16Add}{\sailRISCVvalexternFOneSixAdd}{}%
  \ifstrequal{#1}{extern\_f16Add}{\sailRISCVvalexternFOneSixAdd}{}%
  \ifstrequal{#1}{extern_f16Div}{\sailRISCVvalexternFOneSixDiv}{}%
  \ifstrequal{#1}{extern\_f16Div}{\sailRISCVvalexternFOneSixDiv}{}%
  \ifstrequal{#1}{extern_f16Eq}{\sailRISCVvalexternFOneSixEq}{}%
  \ifstrequal{#1}{extern\_f16Eq}{\sailRISCVvalexternFOneSixEq}{}%
  \ifstrequal{#1}{extern_f16Le}{\sailRISCVvalexternFOneSixLe}{}%
  \ifstrequal{#1}{extern\_f16Le}{\sailRISCVvalexternFOneSixLe}{}%
  \ifstrequal{#1}{extern_f16Lt}{\sailRISCVvalexternFOneSixLt}{}%
  \ifstrequal{#1}{extern\_f16Lt}{\sailRISCVvalexternFOneSixLt}{}%
  \ifstrequal{#1}{extern_f16Mul}{\sailRISCVvalexternFOneSixMul}{}%
  \ifstrequal{#1}{extern\_f16Mul}{\sailRISCVvalexternFOneSixMul}{}%
  \ifstrequal{#1}{extern_f16MulAdd}{\sailRISCVvalexternFOneSixMulAdd}{}%
  \ifstrequal{#1}{extern\_f16MulAdd}{\sailRISCVvalexternFOneSixMulAdd}{}%
  \ifstrequal{#1}{extern_f16Sqrt}{\sailRISCVvalexternFOneSixSqrt}{}%
  \ifstrequal{#1}{extern\_f16Sqrt}{\sailRISCVvalexternFOneSixSqrt}{}%
  \ifstrequal{#1}{extern_f16Sub}{\sailRISCVvalexternFOneSixSub}{}%
  \ifstrequal{#1}{extern\_f16Sub}{\sailRISCVvalexternFOneSixSub}{}%
  \ifstrequal{#1}{extern_f16ToF32}{\sailRISCVvalexternFOneSixToFThreeTwo}{}%
  \ifstrequal{#1}{extern\_f16ToF32}{\sailRISCVvalexternFOneSixToFThreeTwo}{}%
  \ifstrequal{#1}{extern_f16ToF64}{\sailRISCVvalexternFOneSixToFSixFour}{}%
  \ifstrequal{#1}{extern\_f16ToF64}{\sailRISCVvalexternFOneSixToFSixFour}{}%
  \ifstrequal{#1}{extern_f16ToI32}{\sailRISCVvalexternFOneSixToIThreeTwo}{}%
  \ifstrequal{#1}{extern\_f16ToI32}{\sailRISCVvalexternFOneSixToIThreeTwo}{}%
  \ifstrequal{#1}{extern_f16ToI64}{\sailRISCVvalexternFOneSixToISixFour}{}%
  \ifstrequal{#1}{extern\_f16ToI64}{\sailRISCVvalexternFOneSixToISixFour}{}%
  \ifstrequal{#1}{extern_f16ToUi32}{\sailRISCVvalexternFOneSixToUiThreeTwo}{}%
  \ifstrequal{#1}{extern\_f16ToUi32}{\sailRISCVvalexternFOneSixToUiThreeTwo}{}%
  \ifstrequal{#1}{extern_f16ToUi64}{\sailRISCVvalexternFOneSixToUiSixFour}{}%
  \ifstrequal{#1}{extern\_f16ToUi64}{\sailRISCVvalexternFOneSixToUiSixFour}{}%
  \ifstrequal{#1}{extern_f32Add}{\sailRISCVvalexternFThreeTwoAdd}{}%
  \ifstrequal{#1}{extern\_f32Add}{\sailRISCVvalexternFThreeTwoAdd}{}%
  \ifstrequal{#1}{extern_f32Div}{\sailRISCVvalexternFThreeTwoDiv}{}%
  \ifstrequal{#1}{extern\_f32Div}{\sailRISCVvalexternFThreeTwoDiv}{}%
  \ifstrequal{#1}{extern_f32Eq}{\sailRISCVvalexternFThreeTwoEq}{}%
  \ifstrequal{#1}{extern\_f32Eq}{\sailRISCVvalexternFThreeTwoEq}{}%
  \ifstrequal{#1}{extern_f32Le}{\sailRISCVvalexternFThreeTwoLe}{}%
  \ifstrequal{#1}{extern\_f32Le}{\sailRISCVvalexternFThreeTwoLe}{}%
  \ifstrequal{#1}{extern_f32Lt}{\sailRISCVvalexternFThreeTwoLt}{}%
  \ifstrequal{#1}{extern\_f32Lt}{\sailRISCVvalexternFThreeTwoLt}{}%
  \ifstrequal{#1}{extern_f32Mul}{\sailRISCVvalexternFThreeTwoMul}{}%
  \ifstrequal{#1}{extern\_f32Mul}{\sailRISCVvalexternFThreeTwoMul}{}%
  \ifstrequal{#1}{extern_f32MulAdd}{\sailRISCVvalexternFThreeTwoMulAdd}{}%
  \ifstrequal{#1}{extern\_f32MulAdd}{\sailRISCVvalexternFThreeTwoMulAdd}{}%
  \ifstrequal{#1}{extern_f32Sqrt}{\sailRISCVvalexternFThreeTwoSqrt}{}%
  \ifstrequal{#1}{extern\_f32Sqrt}{\sailRISCVvalexternFThreeTwoSqrt}{}%
  \ifstrequal{#1}{extern_f32Sub}{\sailRISCVvalexternFThreeTwoSub}{}%
  \ifstrequal{#1}{extern\_f32Sub}{\sailRISCVvalexternFThreeTwoSub}{}%
  \ifstrequal{#1}{extern_f32ToF16}{\sailRISCVvalexternFThreeTwoToFOneSix}{}%
  \ifstrequal{#1}{extern\_f32ToF16}{\sailRISCVvalexternFThreeTwoToFOneSix}{}%
  \ifstrequal{#1}{extern_f32ToF64}{\sailRISCVvalexternFThreeTwoToFSixFour}{}%
  \ifstrequal{#1}{extern\_f32ToF64}{\sailRISCVvalexternFThreeTwoToFSixFour}{}%
  \ifstrequal{#1}{extern_f32ToI32}{\sailRISCVvalexternFThreeTwoToIThreeTwo}{}%
  \ifstrequal{#1}{extern\_f32ToI32}{\sailRISCVvalexternFThreeTwoToIThreeTwo}{}%
  \ifstrequal{#1}{extern_f32ToI64}{\sailRISCVvalexternFThreeTwoToISixFour}{}%
  \ifstrequal{#1}{extern\_f32ToI64}{\sailRISCVvalexternFThreeTwoToISixFour}{}%
  \ifstrequal{#1}{extern_f32ToUi32}{\sailRISCVvalexternFThreeTwoToUiThreeTwo}{}%
  \ifstrequal{#1}{extern\_f32ToUi32}{\sailRISCVvalexternFThreeTwoToUiThreeTwo}{}%
  \ifstrequal{#1}{extern_f32ToUi64}{\sailRISCVvalexternFThreeTwoToUiSixFour}{}%
  \ifstrequal{#1}{extern\_f32ToUi64}{\sailRISCVvalexternFThreeTwoToUiSixFour}{}%
  \ifstrequal{#1}{extern_f64Add}{\sailRISCVvalexternFSixFourAdd}{}%
  \ifstrequal{#1}{extern\_f64Add}{\sailRISCVvalexternFSixFourAdd}{}%
  \ifstrequal{#1}{extern_f64Div}{\sailRISCVvalexternFSixFourDiv}{}%
  \ifstrequal{#1}{extern\_f64Div}{\sailRISCVvalexternFSixFourDiv}{}%
  \ifstrequal{#1}{extern_f64Eq}{\sailRISCVvalexternFSixFourEq}{}%
  \ifstrequal{#1}{extern\_f64Eq}{\sailRISCVvalexternFSixFourEq}{}%
  \ifstrequal{#1}{extern_f64Le}{\sailRISCVvalexternFSixFourLe}{}%
  \ifstrequal{#1}{extern\_f64Le}{\sailRISCVvalexternFSixFourLe}{}%
  \ifstrequal{#1}{extern_f64Lt}{\sailRISCVvalexternFSixFourLt}{}%
  \ifstrequal{#1}{extern\_f64Lt}{\sailRISCVvalexternFSixFourLt}{}%
  \ifstrequal{#1}{extern_f64Mul}{\sailRISCVvalexternFSixFourMul}{}%
  \ifstrequal{#1}{extern\_f64Mul}{\sailRISCVvalexternFSixFourMul}{}%
  \ifstrequal{#1}{extern_f64MulAdd}{\sailRISCVvalexternFSixFourMulAdd}{}%
  \ifstrequal{#1}{extern\_f64MulAdd}{\sailRISCVvalexternFSixFourMulAdd}{}%
  \ifstrequal{#1}{extern_f64Sqrt}{\sailRISCVvalexternFSixFourSqrt}{}%
  \ifstrequal{#1}{extern\_f64Sqrt}{\sailRISCVvalexternFSixFourSqrt}{}%
  \ifstrequal{#1}{extern_f64Sub}{\sailRISCVvalexternFSixFourSub}{}%
  \ifstrequal{#1}{extern\_f64Sub}{\sailRISCVvalexternFSixFourSub}{}%
  \ifstrequal{#1}{extern_f64ToF16}{\sailRISCVvalexternFSixFourToFOneSix}{}%
  \ifstrequal{#1}{extern\_f64ToF16}{\sailRISCVvalexternFSixFourToFOneSix}{}%
  \ifstrequal{#1}{extern_f64ToF32}{\sailRISCVvalexternFSixFourToFThreeTwo}{}%
  \ifstrequal{#1}{extern\_f64ToF32}{\sailRISCVvalexternFSixFourToFThreeTwo}{}%
  \ifstrequal{#1}{extern_f64ToI32}{\sailRISCVvalexternFSixFourToIThreeTwo}{}%
  \ifstrequal{#1}{extern\_f64ToI32}{\sailRISCVvalexternFSixFourToIThreeTwo}{}%
  \ifstrequal{#1}{extern_f64ToI64}{\sailRISCVvalexternFSixFourToISixFour}{}%
  \ifstrequal{#1}{extern\_f64ToI64}{\sailRISCVvalexternFSixFourToISixFour}{}%
  \ifstrequal{#1}{extern_f64ToUi32}{\sailRISCVvalexternFSixFourToUiThreeTwo}{}%
  \ifstrequal{#1}{extern\_f64ToUi32}{\sailRISCVvalexternFSixFourToUiThreeTwo}{}%
  \ifstrequal{#1}{extern_f64ToUi64}{\sailRISCVvalexternFSixFourToUiSixFour}{}%
  \ifstrequal{#1}{extern\_f64ToUi64}{\sailRISCVvalexternFSixFourToUiSixFour}{}%
  \ifstrequal{#1}{extern_i32ToF16}{\sailRISCVvalexternIThreeTwoToFOneSix}{}%
  \ifstrequal{#1}{extern\_i32ToF16}{\sailRISCVvalexternIThreeTwoToFOneSix}{}%
  \ifstrequal{#1}{extern_i32ToF32}{\sailRISCVvalexternIThreeTwoToFThreeTwo}{}%
  \ifstrequal{#1}{extern\_i32ToF32}{\sailRISCVvalexternIThreeTwoToFThreeTwo}{}%
  \ifstrequal{#1}{extern_i32ToF64}{\sailRISCVvalexternIThreeTwoToFSixFour}{}%
  \ifstrequal{#1}{extern\_i32ToF64}{\sailRISCVvalexternIThreeTwoToFSixFour}{}%
  \ifstrequal{#1}{extern_i64ToF16}{\sailRISCVvalexternISixFourToFOneSix}{}%
  \ifstrequal{#1}{extern\_i64ToF16}{\sailRISCVvalexternISixFourToFOneSix}{}%
  \ifstrequal{#1}{extern_i64ToF32}{\sailRISCVvalexternISixFourToFThreeTwo}{}%
  \ifstrequal{#1}{extern\_i64ToF32}{\sailRISCVvalexternISixFourToFThreeTwo}{}%
  \ifstrequal{#1}{extern_i64ToF64}{\sailRISCVvalexternISixFourToFSixFour}{}%
  \ifstrequal{#1}{extern\_i64ToF64}{\sailRISCVvalexternISixFourToFSixFour}{}%
  \ifstrequal{#1}{extern_ui32ToF16}{\sailRISCVvalexternUiThreeTwoToFOneSix}{}%
  \ifstrequal{#1}{extern\_ui32ToF16}{\sailRISCVvalexternUiThreeTwoToFOneSix}{}%
  \ifstrequal{#1}{extern_ui32ToF32}{\sailRISCVvalexternUiThreeTwoToFThreeTwo}{}%
  \ifstrequal{#1}{extern\_ui32ToF32}{\sailRISCVvalexternUiThreeTwoToFThreeTwo}{}%
  \ifstrequal{#1}{extern_ui32ToF64}{\sailRISCVvalexternUiThreeTwoToFSixFour}{}%
  \ifstrequal{#1}{extern\_ui32ToF64}{\sailRISCVvalexternUiThreeTwoToFSixFour}{}%
  \ifstrequal{#1}{extern_ui64ToF16}{\sailRISCVvalexternUiSixFourToFOneSix}{}%
  \ifstrequal{#1}{extern\_ui64ToF16}{\sailRISCVvalexternUiSixFourToFOneSix}{}%
  \ifstrequal{#1}{extern_ui64ToF32}{\sailRISCVvalexternUiSixFourToFThreeTwo}{}%
  \ifstrequal{#1}{extern\_ui64ToF32}{\sailRISCVvalexternUiSixFourToFThreeTwo}{}%
  \ifstrequal{#1}{extern_ui64ToF64}{\sailRISCVvalexternUiSixFourToFSixFour}{}%
  \ifstrequal{#1}{extern\_ui64ToF64}{\sailRISCVvalexternUiSixFourToFSixFour}{}%
  \ifstrequal{#1}{extop_zbb_of_num}{\sailRISCVvalextopZbbOfNum}{}%
  \ifstrequal{#1}{extop\_zbb\_of\_num}{\sailRISCVvalextopZbbOfNum}{}%
  \ifstrequal{#1}{f_bin_op_D_of_num}{\sailRISCVvalfBinOpDOfNum}{}%
  \ifstrequal{#1}{f\_bin\_op\_D\_of\_num}{\sailRISCVvalfBinOpDOfNum}{}%
  \ifstrequal{#1}{f_bin_op_H_of_num}{\sailRISCVvalfBinOpHOfNum}{}%
  \ifstrequal{#1}{f\_bin\_op\_H\_of\_num}{\sailRISCVvalfBinOpHOfNum}{}%
  \ifstrequal{#1}{f_bin_op_S_of_num}{\sailRISCVvalfBinOpSOfNum}{}%
  \ifstrequal{#1}{f\_bin\_op\_S\_of\_num}{\sailRISCVvalfBinOpSOfNum}{}%
  \ifstrequal{#1}{f_bin_rm_op_D_of_num}{\sailRISCVvalfBinRmOpDOfNum}{}%
  \ifstrequal{#1}{f\_bin\_rm\_op\_D\_of\_num}{\sailRISCVvalfBinRmOpDOfNum}{}%
  \ifstrequal{#1}{f_bin_rm_op_H_of_num}{\sailRISCVvalfBinRmOpHOfNum}{}%
  \ifstrequal{#1}{f\_bin\_rm\_op\_H\_of\_num}{\sailRISCVvalfBinRmOpHOfNum}{}%
  \ifstrequal{#1}{f_bin_rm_op_S_of_num}{\sailRISCVvalfBinRmOpSOfNum}{}%
  \ifstrequal{#1}{f\_bin\_rm\_op\_S\_of\_num}{\sailRISCVvalfBinRmOpSOfNum}{}%
  \ifstrequal{#1}{f_bin_rm_type_mnemonic_D}{\sailRISCVvalfBinRmTypeMnemonicD}{}%
  \ifstrequal{#1}{f\_bin\_rm\_type\_mnemonic\_D}{\sailRISCVvalfBinRmTypeMnemonicD}{}%
  \ifstrequal{#1}{f_bin_rm_type_mnemonic_S}{\sailRISCVvalfBinRmTypeMnemonicS}{}%
  \ifstrequal{#1}{f\_bin\_rm\_type\_mnemonic\_S}{\sailRISCVvalfBinRmTypeMnemonicS}{}%
  \ifstrequal{#1}{f_bin_type_mnemonic_D}{\sailRISCVvalfBinTypeMnemonicD}{}%
  \ifstrequal{#1}{f\_bin\_type\_mnemonic\_D}{\sailRISCVvalfBinTypeMnemonicD}{}%
  \ifstrequal{#1}{f_bin_type_mnemonic_S}{\sailRISCVvalfBinTypeMnemonicS}{}%
  \ifstrequal{#1}{f\_bin\_type\_mnemonic\_S}{\sailRISCVvalfBinTypeMnemonicS}{}%
  \ifstrequal{#1}{f_is_NaN_D}{\sailRISCVvalfIsNaND}{}%
  \ifstrequal{#1}{f\_is\_NaN\_D}{\sailRISCVvalfIsNaND}{}%
  \ifstrequal{#1}{f_is_NaN_S}{\sailRISCVvalfIsNaNS}{}%
  \ifstrequal{#1}{f\_is\_NaN\_S}{\sailRISCVvalfIsNaNS}{}%
  \ifstrequal{#1}{f_is_QNaN_D}{\sailRISCVvalfIsQNaND}{}%
  \ifstrequal{#1}{f\_is\_QNaN\_D}{\sailRISCVvalfIsQNaND}{}%
  \ifstrequal{#1}{f_is_QNaN_S}{\sailRISCVvalfIsQNaNS}{}%
  \ifstrequal{#1}{f\_is\_QNaN\_S}{\sailRISCVvalfIsQNaNS}{}%
  \ifstrequal{#1}{f_is_SNaN_D}{\sailRISCVvalfIsSNaND}{}%
  \ifstrequal{#1}{f\_is\_SNaN\_D}{\sailRISCVvalfIsSNaND}{}%
  \ifstrequal{#1}{f_is_SNaN_S}{\sailRISCVvalfIsSNaNS}{}%
  \ifstrequal{#1}{f\_is\_SNaN\_S}{\sailRISCVvalfIsSNaNS}{}%
  \ifstrequal{#1}{f_is_neg_inf_D}{\sailRISCVvalfIsNegInfD}{}%
  \ifstrequal{#1}{f\_is\_neg\_inf\_D}{\sailRISCVvalfIsNegInfD}{}%
  \ifstrequal{#1}{f_is_neg_inf_S}{\sailRISCVvalfIsNegInfS}{}%
  \ifstrequal{#1}{f\_is\_neg\_inf\_S}{\sailRISCVvalfIsNegInfS}{}%
  \ifstrequal{#1}{f_is_neg_norm_D}{\sailRISCVvalfIsNegNormD}{}%
  \ifstrequal{#1}{f\_is\_neg\_norm\_D}{\sailRISCVvalfIsNegNormD}{}%
  \ifstrequal{#1}{f_is_neg_norm_S}{\sailRISCVvalfIsNegNormS}{}%
  \ifstrequal{#1}{f\_is\_neg\_norm\_S}{\sailRISCVvalfIsNegNormS}{}%
  \ifstrequal{#1}{f_is_neg_subnorm_D}{\sailRISCVvalfIsNegSubnormD}{}%
  \ifstrequal{#1}{f\_is\_neg\_subnorm\_D}{\sailRISCVvalfIsNegSubnormD}{}%
  \ifstrequal{#1}{f_is_neg_subnorm_S}{\sailRISCVvalfIsNegSubnormS}{}%
  \ifstrequal{#1}{f\_is\_neg\_subnorm\_S}{\sailRISCVvalfIsNegSubnormS}{}%
  \ifstrequal{#1}{f_is_neg_zero_D}{\sailRISCVvalfIsNegZeroD}{}%
  \ifstrequal{#1}{f\_is\_neg\_zero\_D}{\sailRISCVvalfIsNegZeroD}{}%
  \ifstrequal{#1}{f_is_neg_zero_S}{\sailRISCVvalfIsNegZeroS}{}%
  \ifstrequal{#1}{f\_is\_neg\_zero\_S}{\sailRISCVvalfIsNegZeroS}{}%
  \ifstrequal{#1}{f_is_pos_inf_D}{\sailRISCVvalfIsPosInfD}{}%
  \ifstrequal{#1}{f\_is\_pos\_inf\_D}{\sailRISCVvalfIsPosInfD}{}%
  \ifstrequal{#1}{f_is_pos_inf_S}{\sailRISCVvalfIsPosInfS}{}%
  \ifstrequal{#1}{f\_is\_pos\_inf\_S}{\sailRISCVvalfIsPosInfS}{}%
  \ifstrequal{#1}{f_is_pos_norm_D}{\sailRISCVvalfIsPosNormD}{}%
  \ifstrequal{#1}{f\_is\_pos\_norm\_D}{\sailRISCVvalfIsPosNormD}{}%
  \ifstrequal{#1}{f_is_pos_norm_S}{\sailRISCVvalfIsPosNormS}{}%
  \ifstrequal{#1}{f\_is\_pos\_norm\_S}{\sailRISCVvalfIsPosNormS}{}%
  \ifstrequal{#1}{f_is_pos_subnorm_D}{\sailRISCVvalfIsPosSubnormD}{}%
  \ifstrequal{#1}{f\_is\_pos\_subnorm\_D}{\sailRISCVvalfIsPosSubnormD}{}%
  \ifstrequal{#1}{f_is_pos_subnorm_S}{\sailRISCVvalfIsPosSubnormS}{}%
  \ifstrequal{#1}{f\_is\_pos\_subnorm\_S}{\sailRISCVvalfIsPosSubnormS}{}%
  \ifstrequal{#1}{f_is_pos_zero_D}{\sailRISCVvalfIsPosZeroD}{}%
  \ifstrequal{#1}{f\_is\_pos\_zero\_D}{\sailRISCVvalfIsPosZeroD}{}%
  \ifstrequal{#1}{f_is_pos_zero_S}{\sailRISCVvalfIsPosZeroS}{}%
  \ifstrequal{#1}{f\_is\_pos\_zero\_S}{\sailRISCVvalfIsPosZeroS}{}%
  \ifstrequal{#1}{f_madd_op_D_of_num}{\sailRISCVvalfMaddOpDOfNum}{}%
  \ifstrequal{#1}{f\_madd\_op\_D\_of\_num}{\sailRISCVvalfMaddOpDOfNum}{}%
  \ifstrequal{#1}{f_madd_op_H_of_num}{\sailRISCVvalfMaddOpHOfNum}{}%
  \ifstrequal{#1}{f\_madd\_op\_H\_of\_num}{\sailRISCVvalfMaddOpHOfNum}{}%
  \ifstrequal{#1}{f_madd_op_S_of_num}{\sailRISCVvalfMaddOpSOfNum}{}%
  \ifstrequal{#1}{f\_madd\_op\_S\_of\_num}{\sailRISCVvalfMaddOpSOfNum}{}%
  \ifstrequal{#1}{f_madd_type_mnemonic_D}{\sailRISCVvalfMaddTypeMnemonicD}{}%
  \ifstrequal{#1}{f\_madd\_type\_mnemonic\_D}{\sailRISCVvalfMaddTypeMnemonicD}{}%
  \ifstrequal{#1}{f_madd_type_mnemonic_S}{\sailRISCVvalfMaddTypeMnemonicS}{}%
  \ifstrequal{#1}{f\_madd\_type\_mnemonic\_S}{\sailRISCVvalfMaddTypeMnemonicS}{}%
  \ifstrequal{#1}{f_un_op_D_of_num}{\sailRISCVvalfUnOpDOfNum}{}%
  \ifstrequal{#1}{f\_un\_op\_D\_of\_num}{\sailRISCVvalfUnOpDOfNum}{}%
  \ifstrequal{#1}{f_un_op_H_of_num}{\sailRISCVvalfUnOpHOfNum}{}%
  \ifstrequal{#1}{f\_un\_op\_H\_of\_num}{\sailRISCVvalfUnOpHOfNum}{}%
  \ifstrequal{#1}{f_un_op_S_of_num}{\sailRISCVvalfUnOpSOfNum}{}%
  \ifstrequal{#1}{f\_un\_op\_S\_of\_num}{\sailRISCVvalfUnOpSOfNum}{}%
  \ifstrequal{#1}{f_un_rm_op_D_of_num}{\sailRISCVvalfUnRmOpDOfNum}{}%
  \ifstrequal{#1}{f\_un\_rm\_op\_D\_of\_num}{\sailRISCVvalfUnRmOpDOfNum}{}%
  \ifstrequal{#1}{f_un_rm_op_H_of_num}{\sailRISCVvalfUnRmOpHOfNum}{}%
  \ifstrequal{#1}{f\_un\_rm\_op\_H\_of\_num}{\sailRISCVvalfUnRmOpHOfNum}{}%
  \ifstrequal{#1}{f_un_rm_op_S_of_num}{\sailRISCVvalfUnRmOpSOfNum}{}%
  \ifstrequal{#1}{f\_un\_rm\_op\_S\_of\_num}{\sailRISCVvalfUnRmOpSOfNum}{}%
  \ifstrequal{#1}{f_un_rm_type_mnemonic_D}{\sailRISCVvalfUnRmTypeMnemonicD}{}%
  \ifstrequal{#1}{f\_un\_rm\_type\_mnemonic\_D}{\sailRISCVvalfUnRmTypeMnemonicD}{}%
  \ifstrequal{#1}{f_un_rm_type_mnemonic_S}{\sailRISCVvalfUnRmTypeMnemonicS}{}%
  \ifstrequal{#1}{f\_un\_rm\_type\_mnemonic\_S}{\sailRISCVvalfUnRmTypeMnemonicS}{}%
  \ifstrequal{#1}{f_un_type_mnemonic_D}{\sailRISCVvalfUnTypeMnemonicD}{}%
  \ifstrequal{#1}{f\_un\_type\_mnemonic\_D}{\sailRISCVvalfUnTypeMnemonicD}{}%
  \ifstrequal{#1}{f_un_type_mnemonic_S}{\sailRISCVvalfUnTypeMnemonicS}{}%
  \ifstrequal{#1}{f\_un\_type\_mnemonic\_S}{\sailRISCVvalfUnTypeMnemonicS}{}%
  \ifstrequal{#1}{fastRepCheck}{\sailRISCVvalfastRepCheck}{}%
  \ifstrequal{#1}{fdiv_int}{\sailRISCVvalfdivInt}{}%
  \ifstrequal{#1}{fdiv\_int}{\sailRISCVvalfdivInt}{}%
  \ifstrequal{#1}{fence_bits}{\sailRISCVvalfenceBits}{}%
  \ifstrequal{#1}{fence\_bits}{\sailRISCVvalfenceBits}{}%
  \ifstrequal{#1}{feq_quiet_D}{\sailRISCVvalfeqQuietD}{}%
  \ifstrequal{#1}{feq\_quiet\_D}{\sailRISCVvalfeqQuietD}{}%
  \ifstrequal{#1}{feq_quiet_S}{\sailRISCVvalfeqQuietS}{}%
  \ifstrequal{#1}{feq\_quiet\_S}{\sailRISCVvalfeqQuietS}{}%
  \ifstrequal{#1}{fetch}{\sailRISCVvalfetch}{}%
  \ifstrequal{#1}{findPendingInterrupt}{\sailRISCVvalfindPendingInterrupt}{}%
  \ifstrequal{#1}{fle_D}{\sailRISCVvalfleD}{}%
  \ifstrequal{#1}{fle\_D}{\sailRISCVvalfleD}{}%
  \ifstrequal{#1}{fle_S}{\sailRISCVvalfleS}{}%
  \ifstrequal{#1}{fle\_S}{\sailRISCVvalfleS}{}%
  \ifstrequal{#1}{flt_D}{\sailRISCVvalfltD}{}%
  \ifstrequal{#1}{flt\_D}{\sailRISCVvalfltD}{}%
  \ifstrequal{#1}{flt_S}{\sailRISCVvalfltS}{}%
  \ifstrequal{#1}{flt\_S}{\sailRISCVvalfltS}{}%
  \ifstrequal{#1}{flush_TLB}{\sailRISCVvalflushTLB}{}%
  \ifstrequal{#1}{flush\_TLB}{\sailRISCVvalflushTLB}{}%
  \ifstrequal{#1}{flush_TLB39}{\sailRISCVvalflushTLBThreeNine}{}%
  \ifstrequal{#1}{flush\_TLB39}{\sailRISCVvalflushTLBThreeNine}{}%
  \ifstrequal{#1}{flush_TLB48}{\sailRISCVvalflushTLBFourEight}{}%
  \ifstrequal{#1}{flush\_TLB48}{\sailRISCVvalflushTLBFourEight}{}%
  \ifstrequal{#1}{flush_TLB_Entry}{\sailRISCVvalflushTLBEntry}{}%
  \ifstrequal{#1}{flush\_TLB\_Entry}{\sailRISCVvalflushTLBEntry}{}%
  \ifstrequal{#1}{fmake_D}{\sailRISCVvalfmakeD}{}%
  \ifstrequal{#1}{fmake\_D}{\sailRISCVvalfmakeD}{}%
  \ifstrequal{#1}{fmake_S}{\sailRISCVvalfmakeS}{}%
  \ifstrequal{#1}{fmake\_S}{\sailRISCVvalfmakeS}{}%
  \ifstrequal{#1}{fmod_int}{\sailRISCVvalfmodInt}{}%
  \ifstrequal{#1}{fmod\_int}{\sailRISCVvalfmodInt}{}%
  \ifstrequal{#1}{freg_name}{\sailRISCVvalfregName}{}%
  \ifstrequal{#1}{freg\_name}{\sailRISCVvalfregName}{}%
  \ifstrequal{#1}{freg_name_abi}{\sailRISCVvalfregNameAbi}{}%
  \ifstrequal{#1}{freg\_name\_abi}{\sailRISCVvalfregNameAbi}{}%
  \ifstrequal{#1}{freg_or_reg_name}{\sailRISCVvalfregOrRegName}{}%
  \ifstrequal{#1}{freg\_or\_reg\_name}{\sailRISCVvalfregOrRegName}{}%
  \ifstrequal{#1}{fregval_from_freg}{\sailRISCVvalfregvalFromFreg}{}%
  \ifstrequal{#1}{fregval\_from\_freg}{\sailRISCVvalfregvalFromFreg}{}%
  \ifstrequal{#1}{fregval_into_freg}{\sailRISCVvalfregvalIntoFreg}{}%
  \ifstrequal{#1}{fregval\_into\_freg}{\sailRISCVvalfregvalIntoFreg}{}%
  \ifstrequal{#1}{frm_mnemonic}{\sailRISCVvalfrmMnemonic}{}%
  \ifstrequal{#1}{frm\_mnemonic}{\sailRISCVvalfrmMnemonic}{}%
  \ifstrequal{#1}{fsplit_D}{\sailRISCVvalfsplitD}{}%
  \ifstrequal{#1}{fsplit\_D}{\sailRISCVvalfsplitD}{}%
  \ifstrequal{#1}{fsplit_S}{\sailRISCVvalfsplitS}{}%
  \ifstrequal{#1}{fsplit\_S}{\sailRISCVvalfsplitS}{}%
  \ifstrequal{#1}{getCapBase}{\sailRISCVvalgetCapBase}{}%
  \ifstrequal{#1}{getCapBaseBits}{\sailRISCVvalgetCapBaseBits}{}%
  \ifstrequal{#1}{getCapBounds}{\sailRISCVvalgetCapBounds}{}%
  \ifstrequal{#1}{getCapBoundsBits}{\sailRISCVvalgetCapBoundsBits}{}%
  \ifstrequal{#1}{getCapCursor}{\sailRISCVvalgetCapCursor}{}%
  \ifstrequal{#1}{getCapFlags}{\sailRISCVvalgetCapFlags}{}%
  \ifstrequal{#1}{getCapHardPerms}{\sailRISCVvalgetCapHardPerms}{}%
  \ifstrequal{#1}{getCapLength}{\sailRISCVvalgetCapLength}{}%
  \ifstrequal{#1}{getCapOffset}{\sailRISCVvalgetCapOffset}{}%
  \ifstrequal{#1}{getCapOffsetBits}{\sailRISCVvalgetCapOffsetBits}{}%
  \ifstrequal{#1}{getCapPerms}{\sailRISCVvalgetCapPerms}{}%
  \ifstrequal{#1}{getCapTop}{\sailRISCVvalgetCapTop}{}%
  \ifstrequal{#1}{getCapTopBits}{\sailRISCVvalgetCapTopBits}{}%
  \ifstrequal{#1}{getPendingSet}{\sailRISCVvalgetPendingSet}{}%
  \ifstrequal{#1}{getRepresentableAlignmentMask}{\sailRISCVvalgetRepresentableAlignmentMask}{}%
  \ifstrequal{#1}{getRepresentableLength}{\sailRISCVvalgetRepresentableLength}{}%
  \ifstrequal{#1}{get_16_random_bits}{\sailRISCVvalgetOneSixRandomBits}{}%
  \ifstrequal{#1}{get\_16\_random\_bits}{\sailRISCVvalgetOneSixRandomBits}{}%
  \ifstrequal{#1}{get_arch_pc}{\sailRISCVvalgetArchPc}{}%
  \ifstrequal{#1}{get\_arch\_pc}{\sailRISCVvalgetArchPc}{}%
  \ifstrequal{#1}{get_cheri_mode_cap_addr}{\sailRISCVvalgetCheriModeCapAddr}{}%
  \ifstrequal{#1}{get\_cheri\_mode\_cap\_addr}{\sailRISCVvalgetCheriModeCapAddr}{}%
  \ifstrequal{#1}{get_config_print_instr}{\sailRISCVvalgetConfigPrintInstr}{}%
  \ifstrequal{#1}{get\_config\_print\_instr}{\sailRISCVvalgetConfigPrintInstr}{}%
  \ifstrequal{#1}{get_config_print_mem}{\sailRISCVvalgetConfigPrintMem}{}%
  \ifstrequal{#1}{get\_config\_print\_mem}{\sailRISCVvalgetConfigPrintMem}{}%
  \ifstrequal{#1}{get_config_print_platform}{\sailRISCVvalgetConfigPrintPlatform}{}%
  \ifstrequal{#1}{get\_config\_print\_platform}{\sailRISCVvalgetConfigPrintPlatform}{}%
  \ifstrequal{#1}{get_config_print_reg}{\sailRISCVvalgetConfigPrintReg}{}%
  \ifstrequal{#1}{get\_config\_print\_reg}{\sailRISCVvalgetConfigPrintReg}{}%
  \ifstrequal{#1}{get_mstatus_SXL}{\sailRISCVvalgetMstatusSXL}{}%
  \ifstrequal{#1}{get\_mstatus\_SXL}{\sailRISCVvalgetMstatusSXL}{}%
  \ifstrequal{#1}{get_mstatus_UXL}{\sailRISCVvalgetMstatusUXL}{}%
  \ifstrequal{#1}{get\_mstatus\_UXL}{\sailRISCVvalgetMstatusUXL}{}%
  \ifstrequal{#1}{get_mtvec}{\sailRISCVvalgetMtvec}{}%
  \ifstrequal{#1}{get\_mtvec}{\sailRISCVvalgetMtvec}{}%
  \ifstrequal{#1}{get_next_pc}{\sailRISCVvalgetNextPc}{}%
  \ifstrequal{#1}{get\_next\_pc}{\sailRISCVvalgetNextPc}{}%
  \ifstrequal{#1}{get_slice_int}{\sailRISCVvalgetSliceInt}{}%
  \ifstrequal{#1}{get\_slice\_int}{\sailRISCVvalgetSliceInt}{}%
  \ifstrequal{#1}{get_sstatus_UXL}{\sailRISCVvalgetSstatusUXL}{}%
  \ifstrequal{#1}{get\_sstatus\_UXL}{\sailRISCVvalgetSstatusUXL}{}%
  \ifstrequal{#1}{get_stvec}{\sailRISCVvalgetStvec}{}%
  \ifstrequal{#1}{get\_stvec}{\sailRISCVvalgetStvec}{}%
  \ifstrequal{#1}{get_utvec}{\sailRISCVvalgetUtvec}{}%
  \ifstrequal{#1}{get\_utvec}{\sailRISCVvalgetUtvec}{}%
  \ifstrequal{#1}{get_xret_target}{\sailRISCVvalgetXretTarget}{}%
  \ifstrequal{#1}{get\_xret\_target}{\sailRISCVvalgetXretTarget}{}%
  \ifstrequal{#1}{gt_int}{\sailRISCVvalgtInt}{}%
  \ifstrequal{#1}{gt\_int}{\sailRISCVvalgtInt}{}%
  \ifstrequal{#1}{gteq_int}{\sailRISCVvalgteqInt}{}%
  \ifstrequal{#1}{gteq\_int}{\sailRISCVvalgteqInt}{}%
  \ifstrequal{#1}{handle_cheri_cap_exception}{\sailRISCVvalhandleCheriCapException}{}%
  \ifstrequal{#1}{handle\_cheri\_cap\_exception}{\sailRISCVvalhandleCheriCapException}{}%
  \ifstrequal{#1}{handle_cheri_pcc_exception}{\sailRISCVvalhandleCheriPccException}{}%
  \ifstrequal{#1}{handle\_cheri\_pcc\_exception}{\sailRISCVvalhandleCheriPccException}{}%
  \ifstrequal{#1}{handle_cheri_reg_exception}{\sailRISCVvalhandleCheriRegException}{}%
  \ifstrequal{#1}{handle\_cheri\_reg\_exception}{\sailRISCVvalhandleCheriRegException}{}%
  \ifstrequal{#1}{handle_exception}{\sailRISCVvalhandleException}{}%
  \ifstrequal{#1}{handle\_exception}{\sailRISCVvalhandleException}{}%
  \ifstrequal{#1}{handle_illegal}{\sailRISCVvalhandleIllegal}{}%
  \ifstrequal{#1}{handle\_illegal}{\sailRISCVvalhandleIllegal}{}%
  \ifstrequal{#1}{handle_interrupt}{\sailRISCVvalhandleInterrupt}{}%
  \ifstrequal{#1}{handle\_interrupt}{\sailRISCVvalhandleInterrupt}{}%
  \ifstrequal{#1}{handle_load_cap_via_cap}{\sailRISCVvalhandleLoadCapViaCap}{}%
  \ifstrequal{#1}{handle\_load\_cap\_via\_cap}{\sailRISCVvalhandleLoadCapViaCap}{}%
  \ifstrequal{#1}{handle_load_data_via_cap}{\sailRISCVvalhandleLoadDataViaCap}{}%
  \ifstrequal{#1}{handle\_load\_data\_via\_cap}{\sailRISCVvalhandleLoadDataViaCap}{}%
  \ifstrequal{#1}{handle_loadres_cap_via_cap}{\sailRISCVvalhandleLoadresCapViaCap}{}%
  \ifstrequal{#1}{handle\_loadres\_cap\_via\_cap}{\sailRISCVvalhandleLoadresCapViaCap}{}%
  \ifstrequal{#1}{handle_loadres_data_via_cap}{\sailRISCVvalhandleLoadresDataViaCap}{}%
  \ifstrequal{#1}{handle\_loadres\_data\_via\_cap}{\sailRISCVvalhandleLoadresDataViaCap}{}%
  \ifstrequal{#1}{handle_mem_exception}{\sailRISCVvalhandleMemException}{}%
  \ifstrequal{#1}{handle\_mem\_exception}{\sailRISCVvalhandleMemException}{}%
  \ifstrequal{#1}{handle_store_cap_via_cap}{\sailRISCVvalhandleStoreCapViaCap}{}%
  \ifstrequal{#1}{handle\_store\_cap\_via\_cap}{\sailRISCVvalhandleStoreCapViaCap}{}%
  \ifstrequal{#1}{handle_store_cond_cap_via_cap}{\sailRISCVvalhandleStoreCondCapViaCap}{}%
  \ifstrequal{#1}{handle\_store\_cond\_cap\_via\_cap}{\sailRISCVvalhandleStoreCondCapViaCap}{}%
  \ifstrequal{#1}{handle_store_cond_data_via_cap}{\sailRISCVvalhandleStoreCondDataViaCap}{}%
  \ifstrequal{#1}{handle\_store\_cond\_data\_via\_cap}{\sailRISCVvalhandleStoreCondDataViaCap}{}%
  \ifstrequal{#1}{handle_store_data_via_cap}{\sailRISCVvalhandleStoreDataViaCap}{}%
  \ifstrequal{#1}{handle\_store\_data\_via\_cap}{\sailRISCVvalhandleStoreDataViaCap}{}%
  \ifstrequal{#1}{handle_trap_extension}{\sailRISCVvalhandleTrapExtension}{}%
  \ifstrequal{#1}{handle\_trap\_extension}{\sailRISCVvalhandleTrapExtension}{}%
  \ifstrequal{#1}{hasReservedOType}{\sailRISCVvalhasReservedOType}{}%
  \ifstrequal{#1}{haveAtomics}{\sailRISCVvalhaveAtomics}{}%
  \ifstrequal{#1}{haveDExt}{\sailRISCVvalhaveDExt}{}%
  \ifstrequal{#1}{haveDoubleFPU}{\sailRISCVvalhaveDoubleFPU}{}%
  \ifstrequal{#1}{haveFExt}{\sailRISCVvalhaveFExt}{}%
  \ifstrequal{#1}{haveMulDiv}{\sailRISCVvalhaveMulDiv}{}%
  \ifstrequal{#1}{haveNExt}{\sailRISCVvalhaveNExt}{}%
  \ifstrequal{#1}{haveRVC}{\sailRISCVvalhaveRVC}{}%
  \ifstrequal{#1}{haveSingleFPU}{\sailRISCVvalhaveSingleFPU}{}%
  \ifstrequal{#1}{haveSupMode}{\sailRISCVvalhaveSupMode}{}%
  \ifstrequal{#1}{haveUsrMode}{\sailRISCVvalhaveUsrMode}{}%
  \ifstrequal{#1}{haveXcheri}{\sailRISCVvalhaveXcheri}{}%
  \ifstrequal{#1}{haveZba}{\sailRISCVvalhaveZba}{}%
  \ifstrequal{#1}{haveZbb}{\sailRISCVvalhaveZbb}{}%
  \ifstrequal{#1}{haveZbc}{\sailRISCVvalhaveZbc}{}%
  \ifstrequal{#1}{haveZbkb}{\sailRISCVvalhaveZbkb}{}%
  \ifstrequal{#1}{haveZbkc}{\sailRISCVvalhaveZbkc}{}%
  \ifstrequal{#1}{haveZbkx}{\sailRISCVvalhaveZbkx}{}%
  \ifstrequal{#1}{haveZbs}{\sailRISCVvalhaveZbs}{}%
  \ifstrequal{#1}{haveZdinx}{\sailRISCVvalhaveZdinx}{}%
  \ifstrequal{#1}{haveZfh}{\sailRISCVvalhaveZfh}{}%
  \ifstrequal{#1}{haveZfinx}{\sailRISCVvalhaveZfinx}{}%
  \ifstrequal{#1}{haveZhinx}{\sailRISCVvalhaveZhinx}{}%
  \ifstrequal{#1}{haveZknd}{\sailRISCVvalhaveZknd}{}%
  \ifstrequal{#1}{haveZkne}{\sailRISCVvalhaveZkne}{}%
  \ifstrequal{#1}{haveZknh}{\sailRISCVvalhaveZknh}{}%
  \ifstrequal{#1}{haveZkr}{\sailRISCVvalhaveZkr}{}%
  \ifstrequal{#1}{haveZksed}{\sailRISCVvalhaveZksed}{}%
  \ifstrequal{#1}{haveZksh}{\sailRISCVvalhaveZksh}{}%
  \ifstrequal{#1}{haveZmmul}{\sailRISCVvalhaveZmmul}{}%
  \ifstrequal{#1}{hex_bits}{\sailRISCVvalhexBits}{}%
  \ifstrequal{#1}{hex\_bits}{\sailRISCVvalhexBits}{}%
  \ifstrequal{#1}{hex_bits_1}{\sailRISCVvalhexBitsOne}{}%
  \ifstrequal{#1}{hex\_bits\_1}{\sailRISCVvalhexBitsOne}{}%
  \ifstrequal{#1}{hex_bits_10}{\sailRISCVvalhexBitsOneZero}{}%
  \ifstrequal{#1}{hex\_bits\_10}{\sailRISCVvalhexBitsOneZero}{}%
  \ifstrequal{#1}{hex_bits_10_backwards}{\sailRISCVvalhexBitsOneZeroBackwards}{}%
  \ifstrequal{#1}{hex\_bits\_10\_backwards}{\sailRISCVvalhexBitsOneZeroBackwards}{}%
  \ifstrequal{#1}{hex_bits_10_backwards_matches}{\sailRISCVvalhexBitsOneZeroBackwardsMatches}{}%
  \ifstrequal{#1}{hex\_bits\_10\_backwards\_matches}{\sailRISCVvalhexBitsOneZeroBackwardsMatches}{}%
  \ifstrequal{#1}{hex_bits_10_forwards}{\sailRISCVvalhexBitsOneZeroForwards}{}%
  \ifstrequal{#1}{hex\_bits\_10\_forwards}{\sailRISCVvalhexBitsOneZeroForwards}{}%
  \ifstrequal{#1}{hex_bits_10_forwards_matches}{\sailRISCVvalhexBitsOneZeroForwardsMatches}{}%
  \ifstrequal{#1}{hex\_bits\_10\_forwards\_matches}{\sailRISCVvalhexBitsOneZeroForwardsMatches}{}%
  \ifstrequal{#1}{hex_bits_10_matches_prefix}{\sailRISCVvalhexBitsOneZeroMatchesPrefix}{}%
  \ifstrequal{#1}{hex\_bits\_10\_matches\_prefix}{\sailRISCVvalhexBitsOneZeroMatchesPrefix}{}%
  \ifstrequal{#1}{hex_bits_11}{\sailRISCVvalhexBitsOneOne}{}%
  \ifstrequal{#1}{hex\_bits\_11}{\sailRISCVvalhexBitsOneOne}{}%
  \ifstrequal{#1}{hex_bits_11_backwards}{\sailRISCVvalhexBitsOneOneBackwards}{}%
  \ifstrequal{#1}{hex\_bits\_11\_backwards}{\sailRISCVvalhexBitsOneOneBackwards}{}%
  \ifstrequal{#1}{hex_bits_11_backwards_matches}{\sailRISCVvalhexBitsOneOneBackwardsMatches}{}%
  \ifstrequal{#1}{hex\_bits\_11\_backwards\_matches}{\sailRISCVvalhexBitsOneOneBackwardsMatches}{}%
  \ifstrequal{#1}{hex_bits_11_forwards}{\sailRISCVvalhexBitsOneOneForwards}{}%
  \ifstrequal{#1}{hex\_bits\_11\_forwards}{\sailRISCVvalhexBitsOneOneForwards}{}%
  \ifstrequal{#1}{hex_bits_11_forwards_matches}{\sailRISCVvalhexBitsOneOneForwardsMatches}{}%
  \ifstrequal{#1}{hex\_bits\_11\_forwards\_matches}{\sailRISCVvalhexBitsOneOneForwardsMatches}{}%
  \ifstrequal{#1}{hex_bits_11_matches_prefix}{\sailRISCVvalhexBitsOneOneMatchesPrefix}{}%
  \ifstrequal{#1}{hex\_bits\_11\_matches\_prefix}{\sailRISCVvalhexBitsOneOneMatchesPrefix}{}%
  \ifstrequal{#1}{hex_bits_12}{\sailRISCVvalhexBitsOneTwo}{}%
  \ifstrequal{#1}{hex\_bits\_12}{\sailRISCVvalhexBitsOneTwo}{}%
  \ifstrequal{#1}{hex_bits_12_backwards}{\sailRISCVvalhexBitsOneTwoBackwards}{}%
  \ifstrequal{#1}{hex\_bits\_12\_backwards}{\sailRISCVvalhexBitsOneTwoBackwards}{}%
  \ifstrequal{#1}{hex_bits_12_backwards_matches}{\sailRISCVvalhexBitsOneTwoBackwardsMatches}{}%
  \ifstrequal{#1}{hex\_bits\_12\_backwards\_matches}{\sailRISCVvalhexBitsOneTwoBackwardsMatches}{}%
  \ifstrequal{#1}{hex_bits_12_forwards}{\sailRISCVvalhexBitsOneTwoForwards}{}%
  \ifstrequal{#1}{hex\_bits\_12\_forwards}{\sailRISCVvalhexBitsOneTwoForwards}{}%
  \ifstrequal{#1}{hex_bits_12_forwards_matches}{\sailRISCVvalhexBitsOneTwoForwardsMatches}{}%
  \ifstrequal{#1}{hex\_bits\_12\_forwards\_matches}{\sailRISCVvalhexBitsOneTwoForwardsMatches}{}%
  \ifstrequal{#1}{hex_bits_13}{\sailRISCVvalhexBitsOneThree}{}%
  \ifstrequal{#1}{hex\_bits\_13}{\sailRISCVvalhexBitsOneThree}{}%
  \ifstrequal{#1}{hex_bits_13_backwards}{\sailRISCVvalhexBitsOneThreeBackwards}{}%
  \ifstrequal{#1}{hex\_bits\_13\_backwards}{\sailRISCVvalhexBitsOneThreeBackwards}{}%
  \ifstrequal{#1}{hex_bits_13_backwards_matches}{\sailRISCVvalhexBitsOneThreeBackwardsMatches}{}%
  \ifstrequal{#1}{hex\_bits\_13\_backwards\_matches}{\sailRISCVvalhexBitsOneThreeBackwardsMatches}{}%
  \ifstrequal{#1}{hex_bits_13_forwards}{\sailRISCVvalhexBitsOneThreeForwards}{}%
  \ifstrequal{#1}{hex\_bits\_13\_forwards}{\sailRISCVvalhexBitsOneThreeForwards}{}%
  \ifstrequal{#1}{hex_bits_13_forwards_matches}{\sailRISCVvalhexBitsOneThreeForwardsMatches}{}%
  \ifstrequal{#1}{hex\_bits\_13\_forwards\_matches}{\sailRISCVvalhexBitsOneThreeForwardsMatches}{}%
  \ifstrequal{#1}{hex_bits_13_matches_prefix}{\sailRISCVvalhexBitsOneThreeMatchesPrefix}{}%
  \ifstrequal{#1}{hex\_bits\_13\_matches\_prefix}{\sailRISCVvalhexBitsOneThreeMatchesPrefix}{}%
  \ifstrequal{#1}{hex_bits_14}{\sailRISCVvalhexBitsOneFour}{}%
  \ifstrequal{#1}{hex\_bits\_14}{\sailRISCVvalhexBitsOneFour}{}%
  \ifstrequal{#1}{hex_bits_14_backwards}{\sailRISCVvalhexBitsOneFourBackwards}{}%
  \ifstrequal{#1}{hex\_bits\_14\_backwards}{\sailRISCVvalhexBitsOneFourBackwards}{}%
  \ifstrequal{#1}{hex_bits_14_backwards_matches}{\sailRISCVvalhexBitsOneFourBackwardsMatches}{}%
  \ifstrequal{#1}{hex\_bits\_14\_backwards\_matches}{\sailRISCVvalhexBitsOneFourBackwardsMatches}{}%
  \ifstrequal{#1}{hex_bits_14_forwards}{\sailRISCVvalhexBitsOneFourForwards}{}%
  \ifstrequal{#1}{hex\_bits\_14\_forwards}{\sailRISCVvalhexBitsOneFourForwards}{}%
  \ifstrequal{#1}{hex_bits_14_forwards_matches}{\sailRISCVvalhexBitsOneFourForwardsMatches}{}%
  \ifstrequal{#1}{hex\_bits\_14\_forwards\_matches}{\sailRISCVvalhexBitsOneFourForwardsMatches}{}%
  \ifstrequal{#1}{hex_bits_14_matches_prefix}{\sailRISCVvalhexBitsOneFourMatchesPrefix}{}%
  \ifstrequal{#1}{hex\_bits\_14\_matches\_prefix}{\sailRISCVvalhexBitsOneFourMatchesPrefix}{}%
  \ifstrequal{#1}{hex_bits_15}{\sailRISCVvalhexBitsOneFive}{}%
  \ifstrequal{#1}{hex\_bits\_15}{\sailRISCVvalhexBitsOneFive}{}%
  \ifstrequal{#1}{hex_bits_15_backwards}{\sailRISCVvalhexBitsOneFiveBackwards}{}%
  \ifstrequal{#1}{hex\_bits\_15\_backwards}{\sailRISCVvalhexBitsOneFiveBackwards}{}%
  \ifstrequal{#1}{hex_bits_15_backwards_matches}{\sailRISCVvalhexBitsOneFiveBackwardsMatches}{}%
  \ifstrequal{#1}{hex\_bits\_15\_backwards\_matches}{\sailRISCVvalhexBitsOneFiveBackwardsMatches}{}%
  \ifstrequal{#1}{hex_bits_15_forwards}{\sailRISCVvalhexBitsOneFiveForwards}{}%
  \ifstrequal{#1}{hex\_bits\_15\_forwards}{\sailRISCVvalhexBitsOneFiveForwards}{}%
  \ifstrequal{#1}{hex_bits_15_forwards_matches}{\sailRISCVvalhexBitsOneFiveForwardsMatches}{}%
  \ifstrequal{#1}{hex\_bits\_15\_forwards\_matches}{\sailRISCVvalhexBitsOneFiveForwardsMatches}{}%
  \ifstrequal{#1}{hex_bits_15_matches_prefix}{\sailRISCVvalhexBitsOneFiveMatchesPrefix}{}%
  \ifstrequal{#1}{hex\_bits\_15\_matches\_prefix}{\sailRISCVvalhexBitsOneFiveMatchesPrefix}{}%
  \ifstrequal{#1}{hex_bits_16}{\sailRISCVvalhexBitsOneSix}{}%
  \ifstrequal{#1}{hex\_bits\_16}{\sailRISCVvalhexBitsOneSix}{}%
  \ifstrequal{#1}{hex_bits_16_backwards}{\sailRISCVvalhexBitsOneSixBackwards}{}%
  \ifstrequal{#1}{hex\_bits\_16\_backwards}{\sailRISCVvalhexBitsOneSixBackwards}{}%
  \ifstrequal{#1}{hex_bits_16_backwards_matches}{\sailRISCVvalhexBitsOneSixBackwardsMatches}{}%
  \ifstrequal{#1}{hex\_bits\_16\_backwards\_matches}{\sailRISCVvalhexBitsOneSixBackwardsMatches}{}%
  \ifstrequal{#1}{hex_bits_16_forwards}{\sailRISCVvalhexBitsOneSixForwards}{}%
  \ifstrequal{#1}{hex\_bits\_16\_forwards}{\sailRISCVvalhexBitsOneSixForwards}{}%
  \ifstrequal{#1}{hex_bits_16_forwards_matches}{\sailRISCVvalhexBitsOneSixForwardsMatches}{}%
  \ifstrequal{#1}{hex\_bits\_16\_forwards\_matches}{\sailRISCVvalhexBitsOneSixForwardsMatches}{}%
  \ifstrequal{#1}{hex_bits_16_matches_prefix}{\sailRISCVvalhexBitsOneSixMatchesPrefix}{}%
  \ifstrequal{#1}{hex\_bits\_16\_matches\_prefix}{\sailRISCVvalhexBitsOneSixMatchesPrefix}{}%
  \ifstrequal{#1}{hex_bits_17}{\sailRISCVvalhexBitsOneSeven}{}%
  \ifstrequal{#1}{hex\_bits\_17}{\sailRISCVvalhexBitsOneSeven}{}%
  \ifstrequal{#1}{hex_bits_17_backwards}{\sailRISCVvalhexBitsOneSevenBackwards}{}%
  \ifstrequal{#1}{hex\_bits\_17\_backwards}{\sailRISCVvalhexBitsOneSevenBackwards}{}%
  \ifstrequal{#1}{hex_bits_17_backwards_matches}{\sailRISCVvalhexBitsOneSevenBackwardsMatches}{}%
  \ifstrequal{#1}{hex\_bits\_17\_backwards\_matches}{\sailRISCVvalhexBitsOneSevenBackwardsMatches}{}%
  \ifstrequal{#1}{hex_bits_17_forwards}{\sailRISCVvalhexBitsOneSevenForwards}{}%
  \ifstrequal{#1}{hex\_bits\_17\_forwards}{\sailRISCVvalhexBitsOneSevenForwards}{}%
  \ifstrequal{#1}{hex_bits_17_forwards_matches}{\sailRISCVvalhexBitsOneSevenForwardsMatches}{}%
  \ifstrequal{#1}{hex\_bits\_17\_forwards\_matches}{\sailRISCVvalhexBitsOneSevenForwardsMatches}{}%
  \ifstrequal{#1}{hex_bits_17_matches_prefix}{\sailRISCVvalhexBitsOneSevenMatchesPrefix}{}%
  \ifstrequal{#1}{hex\_bits\_17\_matches\_prefix}{\sailRISCVvalhexBitsOneSevenMatchesPrefix}{}%
  \ifstrequal{#1}{hex_bits_18}{\sailRISCVvalhexBitsOneEight}{}%
  \ifstrequal{#1}{hex\_bits\_18}{\sailRISCVvalhexBitsOneEight}{}%
  \ifstrequal{#1}{hex_bits_18_backwards}{\sailRISCVvalhexBitsOneEightBackwards}{}%
  \ifstrequal{#1}{hex\_bits\_18\_backwards}{\sailRISCVvalhexBitsOneEightBackwards}{}%
  \ifstrequal{#1}{hex_bits_18_backwards_matches}{\sailRISCVvalhexBitsOneEightBackwardsMatches}{}%
  \ifstrequal{#1}{hex\_bits\_18\_backwards\_matches}{\sailRISCVvalhexBitsOneEightBackwardsMatches}{}%
  \ifstrequal{#1}{hex_bits_18_forwards}{\sailRISCVvalhexBitsOneEightForwards}{}%
  \ifstrequal{#1}{hex\_bits\_18\_forwards}{\sailRISCVvalhexBitsOneEightForwards}{}%
  \ifstrequal{#1}{hex_bits_18_forwards_matches}{\sailRISCVvalhexBitsOneEightForwardsMatches}{}%
  \ifstrequal{#1}{hex\_bits\_18\_forwards\_matches}{\sailRISCVvalhexBitsOneEightForwardsMatches}{}%
  \ifstrequal{#1}{hex_bits_18_matches_prefix}{\sailRISCVvalhexBitsOneEightMatchesPrefix}{}%
  \ifstrequal{#1}{hex\_bits\_18\_matches\_prefix}{\sailRISCVvalhexBitsOneEightMatchesPrefix}{}%
  \ifstrequal{#1}{hex_bits_19}{\sailRISCVvalhexBitsOneNine}{}%
  \ifstrequal{#1}{hex\_bits\_19}{\sailRISCVvalhexBitsOneNine}{}%
  \ifstrequal{#1}{hex_bits_19_backwards}{\sailRISCVvalhexBitsOneNineBackwards}{}%
  \ifstrequal{#1}{hex\_bits\_19\_backwards}{\sailRISCVvalhexBitsOneNineBackwards}{}%
  \ifstrequal{#1}{hex_bits_19_backwards_matches}{\sailRISCVvalhexBitsOneNineBackwardsMatches}{}%
  \ifstrequal{#1}{hex\_bits\_19\_backwards\_matches}{\sailRISCVvalhexBitsOneNineBackwardsMatches}{}%
  \ifstrequal{#1}{hex_bits_19_forwards}{\sailRISCVvalhexBitsOneNineForwards}{}%
  \ifstrequal{#1}{hex\_bits\_19\_forwards}{\sailRISCVvalhexBitsOneNineForwards}{}%
  \ifstrequal{#1}{hex_bits_19_forwards_matches}{\sailRISCVvalhexBitsOneNineForwardsMatches}{}%
  \ifstrequal{#1}{hex\_bits\_19\_forwards\_matches}{\sailRISCVvalhexBitsOneNineForwardsMatches}{}%
  \ifstrequal{#1}{hex_bits_19_matches_prefix}{\sailRISCVvalhexBitsOneNineMatchesPrefix}{}%
  \ifstrequal{#1}{hex\_bits\_19\_matches\_prefix}{\sailRISCVvalhexBitsOneNineMatchesPrefix}{}%
  \ifstrequal{#1}{hex_bits_1_backwards}{\sailRISCVvalhexBitsOneBackwards}{}%
  \ifstrequal{#1}{hex\_bits\_1\_backwards}{\sailRISCVvalhexBitsOneBackwards}{}%
  \ifstrequal{#1}{hex_bits_1_backwards_matches}{\sailRISCVvalhexBitsOneBackwardsMatches}{}%
  \ifstrequal{#1}{hex\_bits\_1\_backwards\_matches}{\sailRISCVvalhexBitsOneBackwardsMatches}{}%
  \ifstrequal{#1}{hex_bits_1_forwards}{\sailRISCVvalhexBitsOneForwards}{}%
  \ifstrequal{#1}{hex\_bits\_1\_forwards}{\sailRISCVvalhexBitsOneForwards}{}%
  \ifstrequal{#1}{hex_bits_1_forwards_matches}{\sailRISCVvalhexBitsOneForwardsMatches}{}%
  \ifstrequal{#1}{hex\_bits\_1\_forwards\_matches}{\sailRISCVvalhexBitsOneForwardsMatches}{}%
  \ifstrequal{#1}{hex_bits_1_matches_prefix}{\sailRISCVvalhexBitsOneMatchesPrefix}{}%
  \ifstrequal{#1}{hex\_bits\_1\_matches\_prefix}{\sailRISCVvalhexBitsOneMatchesPrefix}{}%
  \ifstrequal{#1}{hex_bits_2}{\sailRISCVvalhexBitsTwo}{}%
  \ifstrequal{#1}{hex\_bits\_2}{\sailRISCVvalhexBitsTwo}{}%
  \ifstrequal{#1}{hex_bits_20}{\sailRISCVvalhexBitsTwoZero}{}%
  \ifstrequal{#1}{hex\_bits\_20}{\sailRISCVvalhexBitsTwoZero}{}%
  \ifstrequal{#1}{hex_bits_20_backwards}{\sailRISCVvalhexBitsTwoZeroBackwards}{}%
  \ifstrequal{#1}{hex\_bits\_20\_backwards}{\sailRISCVvalhexBitsTwoZeroBackwards}{}%
  \ifstrequal{#1}{hex_bits_20_backwards_matches}{\sailRISCVvalhexBitsTwoZeroBackwardsMatches}{}%
  \ifstrequal{#1}{hex\_bits\_20\_backwards\_matches}{\sailRISCVvalhexBitsTwoZeroBackwardsMatches}{}%
  \ifstrequal{#1}{hex_bits_20_forwards}{\sailRISCVvalhexBitsTwoZeroForwards}{}%
  \ifstrequal{#1}{hex\_bits\_20\_forwards}{\sailRISCVvalhexBitsTwoZeroForwards}{}%
  \ifstrequal{#1}{hex_bits_20_forwards_matches}{\sailRISCVvalhexBitsTwoZeroForwardsMatches}{}%
  \ifstrequal{#1}{hex\_bits\_20\_forwards\_matches}{\sailRISCVvalhexBitsTwoZeroForwardsMatches}{}%
  \ifstrequal{#1}{hex_bits_20_matches_prefix}{\sailRISCVvalhexBitsTwoZeroMatchesPrefix}{}%
  \ifstrequal{#1}{hex\_bits\_20\_matches\_prefix}{\sailRISCVvalhexBitsTwoZeroMatchesPrefix}{}%
  \ifstrequal{#1}{hex_bits_21}{\sailRISCVvalhexBitsTwoOne}{}%
  \ifstrequal{#1}{hex\_bits\_21}{\sailRISCVvalhexBitsTwoOne}{}%
  \ifstrequal{#1}{hex_bits_21_backwards}{\sailRISCVvalhexBitsTwoOneBackwards}{}%
  \ifstrequal{#1}{hex\_bits\_21\_backwards}{\sailRISCVvalhexBitsTwoOneBackwards}{}%
  \ifstrequal{#1}{hex_bits_21_backwards_matches}{\sailRISCVvalhexBitsTwoOneBackwardsMatches}{}%
  \ifstrequal{#1}{hex\_bits\_21\_backwards\_matches}{\sailRISCVvalhexBitsTwoOneBackwardsMatches}{}%
  \ifstrequal{#1}{hex_bits_21_forwards}{\sailRISCVvalhexBitsTwoOneForwards}{}%
  \ifstrequal{#1}{hex\_bits\_21\_forwards}{\sailRISCVvalhexBitsTwoOneForwards}{}%
  \ifstrequal{#1}{hex_bits_21_forwards_matches}{\sailRISCVvalhexBitsTwoOneForwardsMatches}{}%
  \ifstrequal{#1}{hex\_bits\_21\_forwards\_matches}{\sailRISCVvalhexBitsTwoOneForwardsMatches}{}%
  \ifstrequal{#1}{hex_bits_21_matches_prefix}{\sailRISCVvalhexBitsTwoOneMatchesPrefix}{}%
  \ifstrequal{#1}{hex\_bits\_21\_matches\_prefix}{\sailRISCVvalhexBitsTwoOneMatchesPrefix}{}%
  \ifstrequal{#1}{hex_bits_22}{\sailRISCVvalhexBitsTwoTwo}{}%
  \ifstrequal{#1}{hex\_bits\_22}{\sailRISCVvalhexBitsTwoTwo}{}%
  \ifstrequal{#1}{hex_bits_22_backwards}{\sailRISCVvalhexBitsTwoTwoBackwards}{}%
  \ifstrequal{#1}{hex\_bits\_22\_backwards}{\sailRISCVvalhexBitsTwoTwoBackwards}{}%
  \ifstrequal{#1}{hex_bits_22_backwards_matches}{\sailRISCVvalhexBitsTwoTwoBackwardsMatches}{}%
  \ifstrequal{#1}{hex\_bits\_22\_backwards\_matches}{\sailRISCVvalhexBitsTwoTwoBackwardsMatches}{}%
  \ifstrequal{#1}{hex_bits_22_forwards}{\sailRISCVvalhexBitsTwoTwoForwards}{}%
  \ifstrequal{#1}{hex\_bits\_22\_forwards}{\sailRISCVvalhexBitsTwoTwoForwards}{}%
  \ifstrequal{#1}{hex_bits_22_forwards_matches}{\sailRISCVvalhexBitsTwoTwoForwardsMatches}{}%
  \ifstrequal{#1}{hex\_bits\_22\_forwards\_matches}{\sailRISCVvalhexBitsTwoTwoForwardsMatches}{}%
  \ifstrequal{#1}{hex_bits_22_matches_prefix}{\sailRISCVvalhexBitsTwoTwoMatchesPrefix}{}%
  \ifstrequal{#1}{hex\_bits\_22\_matches\_prefix}{\sailRISCVvalhexBitsTwoTwoMatchesPrefix}{}%
  \ifstrequal{#1}{hex_bits_23}{\sailRISCVvalhexBitsTwoThree}{}%
  \ifstrequal{#1}{hex\_bits\_23}{\sailRISCVvalhexBitsTwoThree}{}%
  \ifstrequal{#1}{hex_bits_23_backwards}{\sailRISCVvalhexBitsTwoThreeBackwards}{}%
  \ifstrequal{#1}{hex\_bits\_23\_backwards}{\sailRISCVvalhexBitsTwoThreeBackwards}{}%
  \ifstrequal{#1}{hex_bits_23_backwards_matches}{\sailRISCVvalhexBitsTwoThreeBackwardsMatches}{}%
  \ifstrequal{#1}{hex\_bits\_23\_backwards\_matches}{\sailRISCVvalhexBitsTwoThreeBackwardsMatches}{}%
  \ifstrequal{#1}{hex_bits_23_forwards}{\sailRISCVvalhexBitsTwoThreeForwards}{}%
  \ifstrequal{#1}{hex\_bits\_23\_forwards}{\sailRISCVvalhexBitsTwoThreeForwards}{}%
  \ifstrequal{#1}{hex_bits_23_forwards_matches}{\sailRISCVvalhexBitsTwoThreeForwardsMatches}{}%
  \ifstrequal{#1}{hex\_bits\_23\_forwards\_matches}{\sailRISCVvalhexBitsTwoThreeForwardsMatches}{}%
  \ifstrequal{#1}{hex_bits_23_matches_prefix}{\sailRISCVvalhexBitsTwoThreeMatchesPrefix}{}%
  \ifstrequal{#1}{hex\_bits\_23\_matches\_prefix}{\sailRISCVvalhexBitsTwoThreeMatchesPrefix}{}%
  \ifstrequal{#1}{hex_bits_24}{\sailRISCVvalhexBitsTwoFour}{}%
  \ifstrequal{#1}{hex\_bits\_24}{\sailRISCVvalhexBitsTwoFour}{}%
  \ifstrequal{#1}{hex_bits_24_backwards}{\sailRISCVvalhexBitsTwoFourBackwards}{}%
  \ifstrequal{#1}{hex\_bits\_24\_backwards}{\sailRISCVvalhexBitsTwoFourBackwards}{}%
  \ifstrequal{#1}{hex_bits_24_backwards_matches}{\sailRISCVvalhexBitsTwoFourBackwardsMatches}{}%
  \ifstrequal{#1}{hex\_bits\_24\_backwards\_matches}{\sailRISCVvalhexBitsTwoFourBackwardsMatches}{}%
  \ifstrequal{#1}{hex_bits_24_forwards}{\sailRISCVvalhexBitsTwoFourForwards}{}%
  \ifstrequal{#1}{hex\_bits\_24\_forwards}{\sailRISCVvalhexBitsTwoFourForwards}{}%
  \ifstrequal{#1}{hex_bits_24_forwards_matches}{\sailRISCVvalhexBitsTwoFourForwardsMatches}{}%
  \ifstrequal{#1}{hex\_bits\_24\_forwards\_matches}{\sailRISCVvalhexBitsTwoFourForwardsMatches}{}%
  \ifstrequal{#1}{hex_bits_24_matches_prefix}{\sailRISCVvalhexBitsTwoFourMatchesPrefix}{}%
  \ifstrequal{#1}{hex\_bits\_24\_matches\_prefix}{\sailRISCVvalhexBitsTwoFourMatchesPrefix}{}%
  \ifstrequal{#1}{hex_bits_25}{\sailRISCVvalhexBitsTwoFive}{}%
  \ifstrequal{#1}{hex\_bits\_25}{\sailRISCVvalhexBitsTwoFive}{}%
  \ifstrequal{#1}{hex_bits_25_backwards}{\sailRISCVvalhexBitsTwoFiveBackwards}{}%
  \ifstrequal{#1}{hex\_bits\_25\_backwards}{\sailRISCVvalhexBitsTwoFiveBackwards}{}%
  \ifstrequal{#1}{hex_bits_25_backwards_matches}{\sailRISCVvalhexBitsTwoFiveBackwardsMatches}{}%
  \ifstrequal{#1}{hex\_bits\_25\_backwards\_matches}{\sailRISCVvalhexBitsTwoFiveBackwardsMatches}{}%
  \ifstrequal{#1}{hex_bits_25_forwards}{\sailRISCVvalhexBitsTwoFiveForwards}{}%
  \ifstrequal{#1}{hex\_bits\_25\_forwards}{\sailRISCVvalhexBitsTwoFiveForwards}{}%
  \ifstrequal{#1}{hex_bits_25_forwards_matches}{\sailRISCVvalhexBitsTwoFiveForwardsMatches}{}%
  \ifstrequal{#1}{hex\_bits\_25\_forwards\_matches}{\sailRISCVvalhexBitsTwoFiveForwardsMatches}{}%
  \ifstrequal{#1}{hex_bits_25_matches_prefix}{\sailRISCVvalhexBitsTwoFiveMatchesPrefix}{}%
  \ifstrequal{#1}{hex\_bits\_25\_matches\_prefix}{\sailRISCVvalhexBitsTwoFiveMatchesPrefix}{}%
  \ifstrequal{#1}{hex_bits_26}{\sailRISCVvalhexBitsTwoSix}{}%
  \ifstrequal{#1}{hex\_bits\_26}{\sailRISCVvalhexBitsTwoSix}{}%
  \ifstrequal{#1}{hex_bits_26_backwards}{\sailRISCVvalhexBitsTwoSixBackwards}{}%
  \ifstrequal{#1}{hex\_bits\_26\_backwards}{\sailRISCVvalhexBitsTwoSixBackwards}{}%
  \ifstrequal{#1}{hex_bits_26_backwards_matches}{\sailRISCVvalhexBitsTwoSixBackwardsMatches}{}%
  \ifstrequal{#1}{hex\_bits\_26\_backwards\_matches}{\sailRISCVvalhexBitsTwoSixBackwardsMatches}{}%
  \ifstrequal{#1}{hex_bits_26_forwards}{\sailRISCVvalhexBitsTwoSixForwards}{}%
  \ifstrequal{#1}{hex\_bits\_26\_forwards}{\sailRISCVvalhexBitsTwoSixForwards}{}%
  \ifstrequal{#1}{hex_bits_26_forwards_matches}{\sailRISCVvalhexBitsTwoSixForwardsMatches}{}%
  \ifstrequal{#1}{hex\_bits\_26\_forwards\_matches}{\sailRISCVvalhexBitsTwoSixForwardsMatches}{}%
  \ifstrequal{#1}{hex_bits_26_matches_prefix}{\sailRISCVvalhexBitsTwoSixMatchesPrefix}{}%
  \ifstrequal{#1}{hex\_bits\_26\_matches\_prefix}{\sailRISCVvalhexBitsTwoSixMatchesPrefix}{}%
  \ifstrequal{#1}{hex_bits_27}{\sailRISCVvalhexBitsTwoSeven}{}%
  \ifstrequal{#1}{hex\_bits\_27}{\sailRISCVvalhexBitsTwoSeven}{}%
  \ifstrequal{#1}{hex_bits_27_backwards}{\sailRISCVvalhexBitsTwoSevenBackwards}{}%
  \ifstrequal{#1}{hex\_bits\_27\_backwards}{\sailRISCVvalhexBitsTwoSevenBackwards}{}%
  \ifstrequal{#1}{hex_bits_27_backwards_matches}{\sailRISCVvalhexBitsTwoSevenBackwardsMatches}{}%
  \ifstrequal{#1}{hex\_bits\_27\_backwards\_matches}{\sailRISCVvalhexBitsTwoSevenBackwardsMatches}{}%
  \ifstrequal{#1}{hex_bits_27_forwards}{\sailRISCVvalhexBitsTwoSevenForwards}{}%
  \ifstrequal{#1}{hex\_bits\_27\_forwards}{\sailRISCVvalhexBitsTwoSevenForwards}{}%
  \ifstrequal{#1}{hex_bits_27_forwards_matches}{\sailRISCVvalhexBitsTwoSevenForwardsMatches}{}%
  \ifstrequal{#1}{hex\_bits\_27\_forwards\_matches}{\sailRISCVvalhexBitsTwoSevenForwardsMatches}{}%
  \ifstrequal{#1}{hex_bits_27_matches_prefix}{\sailRISCVvalhexBitsTwoSevenMatchesPrefix}{}%
  \ifstrequal{#1}{hex\_bits\_27\_matches\_prefix}{\sailRISCVvalhexBitsTwoSevenMatchesPrefix}{}%
  \ifstrequal{#1}{hex_bits_28}{\sailRISCVvalhexBitsTwoEight}{}%
  \ifstrequal{#1}{hex\_bits\_28}{\sailRISCVvalhexBitsTwoEight}{}%
  \ifstrequal{#1}{hex_bits_28_backwards}{\sailRISCVvalhexBitsTwoEightBackwards}{}%
  \ifstrequal{#1}{hex\_bits\_28\_backwards}{\sailRISCVvalhexBitsTwoEightBackwards}{}%
  \ifstrequal{#1}{hex_bits_28_backwards_matches}{\sailRISCVvalhexBitsTwoEightBackwardsMatches}{}%
  \ifstrequal{#1}{hex\_bits\_28\_backwards\_matches}{\sailRISCVvalhexBitsTwoEightBackwardsMatches}{}%
  \ifstrequal{#1}{hex_bits_28_forwards}{\sailRISCVvalhexBitsTwoEightForwards}{}%
  \ifstrequal{#1}{hex\_bits\_28\_forwards}{\sailRISCVvalhexBitsTwoEightForwards}{}%
  \ifstrequal{#1}{hex_bits_28_forwards_matches}{\sailRISCVvalhexBitsTwoEightForwardsMatches}{}%
  \ifstrequal{#1}{hex\_bits\_28\_forwards\_matches}{\sailRISCVvalhexBitsTwoEightForwardsMatches}{}%
  \ifstrequal{#1}{hex_bits_28_matches_prefix}{\sailRISCVvalhexBitsTwoEightMatchesPrefix}{}%
  \ifstrequal{#1}{hex\_bits\_28\_matches\_prefix}{\sailRISCVvalhexBitsTwoEightMatchesPrefix}{}%
  \ifstrequal{#1}{hex_bits_29}{\sailRISCVvalhexBitsTwoNine}{}%
  \ifstrequal{#1}{hex\_bits\_29}{\sailRISCVvalhexBitsTwoNine}{}%
  \ifstrequal{#1}{hex_bits_29_backwards}{\sailRISCVvalhexBitsTwoNineBackwards}{}%
  \ifstrequal{#1}{hex\_bits\_29\_backwards}{\sailRISCVvalhexBitsTwoNineBackwards}{}%
  \ifstrequal{#1}{hex_bits_29_backwards_matches}{\sailRISCVvalhexBitsTwoNineBackwardsMatches}{}%
  \ifstrequal{#1}{hex\_bits\_29\_backwards\_matches}{\sailRISCVvalhexBitsTwoNineBackwardsMatches}{}%
  \ifstrequal{#1}{hex_bits_29_forwards}{\sailRISCVvalhexBitsTwoNineForwards}{}%
  \ifstrequal{#1}{hex\_bits\_29\_forwards}{\sailRISCVvalhexBitsTwoNineForwards}{}%
  \ifstrequal{#1}{hex_bits_29_forwards_matches}{\sailRISCVvalhexBitsTwoNineForwardsMatches}{}%
  \ifstrequal{#1}{hex\_bits\_29\_forwards\_matches}{\sailRISCVvalhexBitsTwoNineForwardsMatches}{}%
  \ifstrequal{#1}{hex_bits_29_matches_prefix}{\sailRISCVvalhexBitsTwoNineMatchesPrefix}{}%
  \ifstrequal{#1}{hex\_bits\_29\_matches\_prefix}{\sailRISCVvalhexBitsTwoNineMatchesPrefix}{}%
  \ifstrequal{#1}{hex_bits_2_backwards}{\sailRISCVvalhexBitsTwoBackwards}{}%
  \ifstrequal{#1}{hex\_bits\_2\_backwards}{\sailRISCVvalhexBitsTwoBackwards}{}%
  \ifstrequal{#1}{hex_bits_2_backwards_matches}{\sailRISCVvalhexBitsTwoBackwardsMatches}{}%
  \ifstrequal{#1}{hex\_bits\_2\_backwards\_matches}{\sailRISCVvalhexBitsTwoBackwardsMatches}{}%
  \ifstrequal{#1}{hex_bits_2_forwards}{\sailRISCVvalhexBitsTwoForwards}{}%
  \ifstrequal{#1}{hex\_bits\_2\_forwards}{\sailRISCVvalhexBitsTwoForwards}{}%
  \ifstrequal{#1}{hex_bits_2_forwards_matches}{\sailRISCVvalhexBitsTwoForwardsMatches}{}%
  \ifstrequal{#1}{hex\_bits\_2\_forwards\_matches}{\sailRISCVvalhexBitsTwoForwardsMatches}{}%
  \ifstrequal{#1}{hex_bits_2_matches_prefix}{\sailRISCVvalhexBitsTwoMatchesPrefix}{}%
  \ifstrequal{#1}{hex\_bits\_2\_matches\_prefix}{\sailRISCVvalhexBitsTwoMatchesPrefix}{}%
  \ifstrequal{#1}{hex_bits_3}{\sailRISCVvalhexBitsThree}{}%
  \ifstrequal{#1}{hex\_bits\_3}{\sailRISCVvalhexBitsThree}{}%
  \ifstrequal{#1}{hex_bits_30}{\sailRISCVvalhexBitsThreeZero}{}%
  \ifstrequal{#1}{hex\_bits\_30}{\sailRISCVvalhexBitsThreeZero}{}%
  \ifstrequal{#1}{hex_bits_30_backwards}{\sailRISCVvalhexBitsThreeZeroBackwards}{}%
  \ifstrequal{#1}{hex\_bits\_30\_backwards}{\sailRISCVvalhexBitsThreeZeroBackwards}{}%
  \ifstrequal{#1}{hex_bits_30_backwards_matches}{\sailRISCVvalhexBitsThreeZeroBackwardsMatches}{}%
  \ifstrequal{#1}{hex\_bits\_30\_backwards\_matches}{\sailRISCVvalhexBitsThreeZeroBackwardsMatches}{}%
  \ifstrequal{#1}{hex_bits_30_forwards}{\sailRISCVvalhexBitsThreeZeroForwards}{}%
  \ifstrequal{#1}{hex\_bits\_30\_forwards}{\sailRISCVvalhexBitsThreeZeroForwards}{}%
  \ifstrequal{#1}{hex_bits_30_forwards_matches}{\sailRISCVvalhexBitsThreeZeroForwardsMatches}{}%
  \ifstrequal{#1}{hex\_bits\_30\_forwards\_matches}{\sailRISCVvalhexBitsThreeZeroForwardsMatches}{}%
  \ifstrequal{#1}{hex_bits_30_matches_prefix}{\sailRISCVvalhexBitsThreeZeroMatchesPrefix}{}%
  \ifstrequal{#1}{hex\_bits\_30\_matches\_prefix}{\sailRISCVvalhexBitsThreeZeroMatchesPrefix}{}%
  \ifstrequal{#1}{hex_bits_31}{\sailRISCVvalhexBitsThreeOne}{}%
  \ifstrequal{#1}{hex\_bits\_31}{\sailRISCVvalhexBitsThreeOne}{}%
  \ifstrequal{#1}{hex_bits_31_backwards}{\sailRISCVvalhexBitsThreeOneBackwards}{}%
  \ifstrequal{#1}{hex\_bits\_31\_backwards}{\sailRISCVvalhexBitsThreeOneBackwards}{}%
  \ifstrequal{#1}{hex_bits_31_backwards_matches}{\sailRISCVvalhexBitsThreeOneBackwardsMatches}{}%
  \ifstrequal{#1}{hex\_bits\_31\_backwards\_matches}{\sailRISCVvalhexBitsThreeOneBackwardsMatches}{}%
  \ifstrequal{#1}{hex_bits_31_forwards}{\sailRISCVvalhexBitsThreeOneForwards}{}%
  \ifstrequal{#1}{hex\_bits\_31\_forwards}{\sailRISCVvalhexBitsThreeOneForwards}{}%
  \ifstrequal{#1}{hex_bits_31_forwards_matches}{\sailRISCVvalhexBitsThreeOneForwardsMatches}{}%
  \ifstrequal{#1}{hex\_bits\_31\_forwards\_matches}{\sailRISCVvalhexBitsThreeOneForwardsMatches}{}%
  \ifstrequal{#1}{hex_bits_31_matches_prefix}{\sailRISCVvalhexBitsThreeOneMatchesPrefix}{}%
  \ifstrequal{#1}{hex\_bits\_31\_matches\_prefix}{\sailRISCVvalhexBitsThreeOneMatchesPrefix}{}%
  \ifstrequal{#1}{hex_bits_32}{\sailRISCVvalhexBitsThreeTwo}{}%
  \ifstrequal{#1}{hex\_bits\_32}{\sailRISCVvalhexBitsThreeTwo}{}%
  \ifstrequal{#1}{hex_bits_32_backwards}{\sailRISCVvalhexBitsThreeTwoBackwards}{}%
  \ifstrequal{#1}{hex\_bits\_32\_backwards}{\sailRISCVvalhexBitsThreeTwoBackwards}{}%
  \ifstrequal{#1}{hex_bits_32_backwards_matches}{\sailRISCVvalhexBitsThreeTwoBackwardsMatches}{}%
  \ifstrequal{#1}{hex\_bits\_32\_backwards\_matches}{\sailRISCVvalhexBitsThreeTwoBackwardsMatches}{}%
  \ifstrequal{#1}{hex_bits_32_forwards}{\sailRISCVvalhexBitsThreeTwoForwards}{}%
  \ifstrequal{#1}{hex\_bits\_32\_forwards}{\sailRISCVvalhexBitsThreeTwoForwards}{}%
  \ifstrequal{#1}{hex_bits_32_forwards_matches}{\sailRISCVvalhexBitsThreeTwoForwardsMatches}{}%
  \ifstrequal{#1}{hex\_bits\_32\_forwards\_matches}{\sailRISCVvalhexBitsThreeTwoForwardsMatches}{}%
  \ifstrequal{#1}{hex_bits_32_matches_prefix}{\sailRISCVvalhexBitsThreeTwoMatchesPrefix}{}%
  \ifstrequal{#1}{hex\_bits\_32\_matches\_prefix}{\sailRISCVvalhexBitsThreeTwoMatchesPrefix}{}%
  \ifstrequal{#1}{hex_bits_33}{\sailRISCVvalhexBitsThreeThree}{}%
  \ifstrequal{#1}{hex\_bits\_33}{\sailRISCVvalhexBitsThreeThree}{}%
  \ifstrequal{#1}{hex_bits_33_backwards}{\sailRISCVvalhexBitsThreeThreeBackwards}{}%
  \ifstrequal{#1}{hex\_bits\_33\_backwards}{\sailRISCVvalhexBitsThreeThreeBackwards}{}%
  \ifstrequal{#1}{hex_bits_33_backwards_matches}{\sailRISCVvalhexBitsThreeThreeBackwardsMatches}{}%
  \ifstrequal{#1}{hex\_bits\_33\_backwards\_matches}{\sailRISCVvalhexBitsThreeThreeBackwardsMatches}{}%
  \ifstrequal{#1}{hex_bits_33_forwards}{\sailRISCVvalhexBitsThreeThreeForwards}{}%
  \ifstrequal{#1}{hex\_bits\_33\_forwards}{\sailRISCVvalhexBitsThreeThreeForwards}{}%
  \ifstrequal{#1}{hex_bits_33_forwards_matches}{\sailRISCVvalhexBitsThreeThreeForwardsMatches}{}%
  \ifstrequal{#1}{hex\_bits\_33\_forwards\_matches}{\sailRISCVvalhexBitsThreeThreeForwardsMatches}{}%
  \ifstrequal{#1}{hex_bits_33_matches_prefix}{\sailRISCVvalhexBitsThreeThreeMatchesPrefix}{}%
  \ifstrequal{#1}{hex\_bits\_33\_matches\_prefix}{\sailRISCVvalhexBitsThreeThreeMatchesPrefix}{}%
  \ifstrequal{#1}{hex_bits_3_backwards}{\sailRISCVvalhexBitsThreeBackwards}{}%
  \ifstrequal{#1}{hex\_bits\_3\_backwards}{\sailRISCVvalhexBitsThreeBackwards}{}%
  \ifstrequal{#1}{hex_bits_3_backwards_matches}{\sailRISCVvalhexBitsThreeBackwardsMatches}{}%
  \ifstrequal{#1}{hex\_bits\_3\_backwards\_matches}{\sailRISCVvalhexBitsThreeBackwardsMatches}{}%
  \ifstrequal{#1}{hex_bits_3_forwards}{\sailRISCVvalhexBitsThreeForwards}{}%
  \ifstrequal{#1}{hex\_bits\_3\_forwards}{\sailRISCVvalhexBitsThreeForwards}{}%
  \ifstrequal{#1}{hex_bits_3_forwards_matches}{\sailRISCVvalhexBitsThreeForwardsMatches}{}%
  \ifstrequal{#1}{hex\_bits\_3\_forwards\_matches}{\sailRISCVvalhexBitsThreeForwardsMatches}{}%
  \ifstrequal{#1}{hex_bits_3_matches_prefix}{\sailRISCVvalhexBitsThreeMatchesPrefix}{}%
  \ifstrequal{#1}{hex\_bits\_3\_matches\_prefix}{\sailRISCVvalhexBitsThreeMatchesPrefix}{}%
  \ifstrequal{#1}{hex_bits_4}{\sailRISCVvalhexBitsFour}{}%
  \ifstrequal{#1}{hex\_bits\_4}{\sailRISCVvalhexBitsFour}{}%
  \ifstrequal{#1}{hex_bits_48}{\sailRISCVvalhexBitsFourEight}{}%
  \ifstrequal{#1}{hex\_bits\_48}{\sailRISCVvalhexBitsFourEight}{}%
  \ifstrequal{#1}{hex_bits_48_backwards}{\sailRISCVvalhexBitsFourEightBackwards}{}%
  \ifstrequal{#1}{hex\_bits\_48\_backwards}{\sailRISCVvalhexBitsFourEightBackwards}{}%
  \ifstrequal{#1}{hex_bits_48_backwards_matches}{\sailRISCVvalhexBitsFourEightBackwardsMatches}{}%
  \ifstrequal{#1}{hex\_bits\_48\_backwards\_matches}{\sailRISCVvalhexBitsFourEightBackwardsMatches}{}%
  \ifstrequal{#1}{hex_bits_48_forwards}{\sailRISCVvalhexBitsFourEightForwards}{}%
  \ifstrequal{#1}{hex\_bits\_48\_forwards}{\sailRISCVvalhexBitsFourEightForwards}{}%
  \ifstrequal{#1}{hex_bits_48_forwards_matches}{\sailRISCVvalhexBitsFourEightForwardsMatches}{}%
  \ifstrequal{#1}{hex\_bits\_48\_forwards\_matches}{\sailRISCVvalhexBitsFourEightForwardsMatches}{}%
  \ifstrequal{#1}{hex_bits_48_matches_prefix}{\sailRISCVvalhexBitsFourEightMatchesPrefix}{}%
  \ifstrequal{#1}{hex\_bits\_48\_matches\_prefix}{\sailRISCVvalhexBitsFourEightMatchesPrefix}{}%
  \ifstrequal{#1}{hex_bits_4_backwards}{\sailRISCVvalhexBitsFourBackwards}{}%
  \ifstrequal{#1}{hex\_bits\_4\_backwards}{\sailRISCVvalhexBitsFourBackwards}{}%
  \ifstrequal{#1}{hex_bits_4_backwards_matches}{\sailRISCVvalhexBitsFourBackwardsMatches}{}%
  \ifstrequal{#1}{hex\_bits\_4\_backwards\_matches}{\sailRISCVvalhexBitsFourBackwardsMatches}{}%
  \ifstrequal{#1}{hex_bits_4_forwards}{\sailRISCVvalhexBitsFourForwards}{}%
  \ifstrequal{#1}{hex\_bits\_4\_forwards}{\sailRISCVvalhexBitsFourForwards}{}%
  \ifstrequal{#1}{hex_bits_4_forwards_matches}{\sailRISCVvalhexBitsFourForwardsMatches}{}%
  \ifstrequal{#1}{hex\_bits\_4\_forwards\_matches}{\sailRISCVvalhexBitsFourForwardsMatches}{}%
  \ifstrequal{#1}{hex_bits_4_matches_prefix}{\sailRISCVvalhexBitsFourMatchesPrefix}{}%
  \ifstrequal{#1}{hex\_bits\_4\_matches\_prefix}{\sailRISCVvalhexBitsFourMatchesPrefix}{}%
  \ifstrequal{#1}{hex_bits_5}{\sailRISCVvalhexBitsFive}{}%
  \ifstrequal{#1}{hex\_bits\_5}{\sailRISCVvalhexBitsFive}{}%
  \ifstrequal{#1}{hex_bits_5_backwards}{\sailRISCVvalhexBitsFiveBackwards}{}%
  \ifstrequal{#1}{hex\_bits\_5\_backwards}{\sailRISCVvalhexBitsFiveBackwards}{}%
  \ifstrequal{#1}{hex_bits_5_backwards_matches}{\sailRISCVvalhexBitsFiveBackwardsMatches}{}%
  \ifstrequal{#1}{hex\_bits\_5\_backwards\_matches}{\sailRISCVvalhexBitsFiveBackwardsMatches}{}%
  \ifstrequal{#1}{hex_bits_5_forwards}{\sailRISCVvalhexBitsFiveForwards}{}%
  \ifstrequal{#1}{hex\_bits\_5\_forwards}{\sailRISCVvalhexBitsFiveForwards}{}%
  \ifstrequal{#1}{hex_bits_5_forwards_matches}{\sailRISCVvalhexBitsFiveForwardsMatches}{}%
  \ifstrequal{#1}{hex\_bits\_5\_forwards\_matches}{\sailRISCVvalhexBitsFiveForwardsMatches}{}%
  \ifstrequal{#1}{hex_bits_5_matches_prefix}{\sailRISCVvalhexBitsFiveMatchesPrefix}{}%
  \ifstrequal{#1}{hex\_bits\_5\_matches\_prefix}{\sailRISCVvalhexBitsFiveMatchesPrefix}{}%
  \ifstrequal{#1}{hex_bits_6}{\sailRISCVvalhexBitsSix}{}%
  \ifstrequal{#1}{hex\_bits\_6}{\sailRISCVvalhexBitsSix}{}%
  \ifstrequal{#1}{hex_bits_64}{\sailRISCVvalhexBitsSixFour}{}%
  \ifstrequal{#1}{hex\_bits\_64}{\sailRISCVvalhexBitsSixFour}{}%
  \ifstrequal{#1}{hex_bits_64_backwards}{\sailRISCVvalhexBitsSixFourBackwards}{}%
  \ifstrequal{#1}{hex\_bits\_64\_backwards}{\sailRISCVvalhexBitsSixFourBackwards}{}%
  \ifstrequal{#1}{hex_bits_64_backwards_matches}{\sailRISCVvalhexBitsSixFourBackwardsMatches}{}%
  \ifstrequal{#1}{hex\_bits\_64\_backwards\_matches}{\sailRISCVvalhexBitsSixFourBackwardsMatches}{}%
  \ifstrequal{#1}{hex_bits_64_forwards}{\sailRISCVvalhexBitsSixFourForwards}{}%
  \ifstrequal{#1}{hex\_bits\_64\_forwards}{\sailRISCVvalhexBitsSixFourForwards}{}%
  \ifstrequal{#1}{hex_bits_64_forwards_matches}{\sailRISCVvalhexBitsSixFourForwardsMatches}{}%
  \ifstrequal{#1}{hex\_bits\_64\_forwards\_matches}{\sailRISCVvalhexBitsSixFourForwardsMatches}{}%
  \ifstrequal{#1}{hex_bits_64_matches_prefix}{\sailRISCVvalhexBitsSixFourMatchesPrefix}{}%
  \ifstrequal{#1}{hex\_bits\_64\_matches\_prefix}{\sailRISCVvalhexBitsSixFourMatchesPrefix}{}%
  \ifstrequal{#1}{hex_bits_6_backwards}{\sailRISCVvalhexBitsSixBackwards}{}%
  \ifstrequal{#1}{hex\_bits\_6\_backwards}{\sailRISCVvalhexBitsSixBackwards}{}%
  \ifstrequal{#1}{hex_bits_6_backwards_matches}{\sailRISCVvalhexBitsSixBackwardsMatches}{}%
  \ifstrequal{#1}{hex\_bits\_6\_backwards\_matches}{\sailRISCVvalhexBitsSixBackwardsMatches}{}%
  \ifstrequal{#1}{hex_bits_6_forwards}{\sailRISCVvalhexBitsSixForwards}{}%
  \ifstrequal{#1}{hex\_bits\_6\_forwards}{\sailRISCVvalhexBitsSixForwards}{}%
  \ifstrequal{#1}{hex_bits_6_forwards_matches}{\sailRISCVvalhexBitsSixForwardsMatches}{}%
  \ifstrequal{#1}{hex\_bits\_6\_forwards\_matches}{\sailRISCVvalhexBitsSixForwardsMatches}{}%
  \ifstrequal{#1}{hex_bits_6_matches_prefix}{\sailRISCVvalhexBitsSixMatchesPrefix}{}%
  \ifstrequal{#1}{hex\_bits\_6\_matches\_prefix}{\sailRISCVvalhexBitsSixMatchesPrefix}{}%
  \ifstrequal{#1}{hex_bits_7}{\sailRISCVvalhexBitsSeven}{}%
  \ifstrequal{#1}{hex\_bits\_7}{\sailRISCVvalhexBitsSeven}{}%
  \ifstrequal{#1}{hex_bits_7_backwards}{\sailRISCVvalhexBitsSevenBackwards}{}%
  \ifstrequal{#1}{hex\_bits\_7\_backwards}{\sailRISCVvalhexBitsSevenBackwards}{}%
  \ifstrequal{#1}{hex_bits_7_backwards_matches}{\sailRISCVvalhexBitsSevenBackwardsMatches}{}%
  \ifstrequal{#1}{hex\_bits\_7\_backwards\_matches}{\sailRISCVvalhexBitsSevenBackwardsMatches}{}%
  \ifstrequal{#1}{hex_bits_7_forwards}{\sailRISCVvalhexBitsSevenForwards}{}%
  \ifstrequal{#1}{hex\_bits\_7\_forwards}{\sailRISCVvalhexBitsSevenForwards}{}%
  \ifstrequal{#1}{hex_bits_7_forwards_matches}{\sailRISCVvalhexBitsSevenForwardsMatches}{}%
  \ifstrequal{#1}{hex\_bits\_7\_forwards\_matches}{\sailRISCVvalhexBitsSevenForwardsMatches}{}%
  \ifstrequal{#1}{hex_bits_7_matches_prefix}{\sailRISCVvalhexBitsSevenMatchesPrefix}{}%
  \ifstrequal{#1}{hex\_bits\_7\_matches\_prefix}{\sailRISCVvalhexBitsSevenMatchesPrefix}{}%
  \ifstrequal{#1}{hex_bits_8}{\sailRISCVvalhexBitsEight}{}%
  \ifstrequal{#1}{hex\_bits\_8}{\sailRISCVvalhexBitsEight}{}%
  \ifstrequal{#1}{hex_bits_8_backwards}{\sailRISCVvalhexBitsEightBackwards}{}%
  \ifstrequal{#1}{hex\_bits\_8\_backwards}{\sailRISCVvalhexBitsEightBackwards}{}%
  \ifstrequal{#1}{hex_bits_8_backwards_matches}{\sailRISCVvalhexBitsEightBackwardsMatches}{}%
  \ifstrequal{#1}{hex\_bits\_8\_backwards\_matches}{\sailRISCVvalhexBitsEightBackwardsMatches}{}%
  \ifstrequal{#1}{hex_bits_8_forwards}{\sailRISCVvalhexBitsEightForwards}{}%
  \ifstrequal{#1}{hex\_bits\_8\_forwards}{\sailRISCVvalhexBitsEightForwards}{}%
  \ifstrequal{#1}{hex_bits_8_forwards_matches}{\sailRISCVvalhexBitsEightForwardsMatches}{}%
  \ifstrequal{#1}{hex\_bits\_8\_forwards\_matches}{\sailRISCVvalhexBitsEightForwardsMatches}{}%
  \ifstrequal{#1}{hex_bits_8_matches_prefix}{\sailRISCVvalhexBitsEightMatchesPrefix}{}%
  \ifstrequal{#1}{hex\_bits\_8\_matches\_prefix}{\sailRISCVvalhexBitsEightMatchesPrefix}{}%
  \ifstrequal{#1}{hex_bits_9}{\sailRISCVvalhexBitsNine}{}%
  \ifstrequal{#1}{hex\_bits\_9}{\sailRISCVvalhexBitsNine}{}%
  \ifstrequal{#1}{hex_bits_9_backwards}{\sailRISCVvalhexBitsNineBackwards}{}%
  \ifstrequal{#1}{hex\_bits\_9\_backwards}{\sailRISCVvalhexBitsNineBackwards}{}%
  \ifstrequal{#1}{hex_bits_9_backwards_matches}{\sailRISCVvalhexBitsNineBackwardsMatches}{}%
  \ifstrequal{#1}{hex\_bits\_9\_backwards\_matches}{\sailRISCVvalhexBitsNineBackwardsMatches}{}%
  \ifstrequal{#1}{hex_bits_9_forwards}{\sailRISCVvalhexBitsNineForwards}{}%
  \ifstrequal{#1}{hex\_bits\_9\_forwards}{\sailRISCVvalhexBitsNineForwards}{}%
  \ifstrequal{#1}{hex_bits_9_forwards_matches}{\sailRISCVvalhexBitsNineForwardsMatches}{}%
  \ifstrequal{#1}{hex\_bits\_9\_forwards\_matches}{\sailRISCVvalhexBitsNineForwardsMatches}{}%
  \ifstrequal{#1}{hex_bits_9_matches_prefix}{\sailRISCVvalhexBitsNineMatchesPrefix}{}%
  \ifstrequal{#1}{hex\_bits\_9\_matches\_prefix}{\sailRISCVvalhexBitsNineMatchesPrefix}{}%
  \ifstrequal{#1}{hex_str}{\sailRISCVvalhexStr}{}%
  \ifstrequal{#1}{hex\_str}{\sailRISCVvalhexStr}{}%
  \ifstrequal{#1}{htif_load}{\sailRISCVvalhtifLoad}{}%
  \ifstrequal{#1}{htif\_load}{\sailRISCVvalhtifLoad}{}%
  \ifstrequal{#1}{htif_store}{\sailRISCVvalhtifStore}{}%
  \ifstrequal{#1}{htif\_store}{\sailRISCVvalhtifStore}{}%
  \ifstrequal{#1}{htif_tick}{\sailRISCVvalhtifTick}{}%
  \ifstrequal{#1}{htif\_tick}{\sailRISCVvalhtifTick}{}%
  \ifstrequal{#1}{in32BitMode}{\sailRISCVvalinThreeTwoBitMode}{}%
  \ifstrequal{#1}{inCapBounds}{\sailRISCVvalinCapBounds}{}%
  \ifstrequal{#1}{incCapOffset}{\sailRISCVvalincCapOffset}{}%
  \ifstrequal{#1}{init_base_regs}{\sailRISCVvalinitBaseRegs}{}%
  \ifstrequal{#1}{init\_base\_regs}{\sailRISCVvalinitBaseRegs}{}%
  \ifstrequal{#1}{init_fdext_regs}{\sailRISCVvalinitFdextRegs}{}%
  \ifstrequal{#1}{init\_fdext\_regs}{\sailRISCVvalinitFdextRegs}{}%
  \ifstrequal{#1}{init_model}{\sailRISCVvalinitModel}{}%
  \ifstrequal{#1}{init\_model}{\sailRISCVvalinitModel}{}%
  \ifstrequal{#1}{init_platform}{\sailRISCVvalinitPlatform}{}%
  \ifstrequal{#1}{init\_platform}{\sailRISCVvalinitPlatform}{}%
  \ifstrequal{#1}{init_pmp}{\sailRISCVvalinitPmp}{}%
  \ifstrequal{#1}{init\_pmp}{\sailRISCVvalinitPmp}{}%
  \ifstrequal{#1}{init_sys}{\sailRISCVvalinitSys}{}%
  \ifstrequal{#1}{init\_sys}{\sailRISCVvalinitSys}{}%
  \ifstrequal{#1}{init_vmem}{\sailRISCVvalinitVmem}{}%
  \ifstrequal{#1}{init\_vmem}{\sailRISCVvalinitVmem}{}%
  \ifstrequal{#1}{init_vmem_sv39}{\sailRISCVvalinitVmemSvThreeNine}{}%
  \ifstrequal{#1}{init\_vmem\_sv39}{\sailRISCVvalinitVmemSvThreeNine}{}%
  \ifstrequal{#1}{init_vmem_sv48}{\sailRISCVvalinitVmemSvFourEight}{}%
  \ifstrequal{#1}{init\_vmem\_sv48}{\sailRISCVvalinitVmemSvFourEight}{}%
  \ifstrequal{#1}{initial_analysis}{\sailRISCVvalinitialAnalysis}{}%
  \ifstrequal{#1}{initial\_analysis}{\sailRISCVvalinitialAnalysis}{}%
  \ifstrequal{#1}{int_power}{\sailRISCVvalintPower}{}%
  \ifstrequal{#1}{int\_power}{\sailRISCVvalintPower}{}%
  \ifstrequal{#1}{int_to_cap}{\sailRISCVvalintToCap}{}%
  \ifstrequal{#1}{int\_to\_cap}{\sailRISCVvalintToCap}{}%
  \ifstrequal{#1}{internal_error}{\sailRISCVvalinternalError}{}%
  \ifstrequal{#1}{internal\_error}{\sailRISCVvalinternalError}{}%
  \ifstrequal{#1}{interruptType_to_bits}{\sailRISCVvalinterruptTypeToBits}{}%
  \ifstrequal{#1}{interruptType\_to\_bits}{\sailRISCVvalinterruptTypeToBits}{}%
  \ifstrequal{#1}{iop_of_num}{\sailRISCVvaliopOfNum}{}%
  \ifstrequal{#1}{iop\_of\_num}{\sailRISCVvaliopOfNum}{}%
  \ifstrequal{#1}{isCapSealed}{\sailRISCVvalisCapSealed}{}%
  \ifstrequal{#1}{isInvalidPTE}{\sailRISCVvalisInvalidPTE}{}%
  \ifstrequal{#1}{isPTEPtr}{\sailRISCVvalisPTEPtr}{}%
  \ifstrequal{#1}{isRVC}{\sailRISCVvalisRVC}{}%
  \ifstrequal{#1}{isValidSv39Addr}{\sailRISCVvalisValidSvThreeNineAddr}{}%
  \ifstrequal{#1}{isValidSv48Addr}{\sailRISCVvalisValidSvFourEightAddr}{}%
  \ifstrequal{#1}{is_CSR_defined}{\sailRISCVvalisCSRDefined}{}%
  \ifstrequal{#1}{is\_CSR\_defined}{\sailRISCVvalisCSRDefined}{}%
  \ifstrequal{#1}{is_aligned_addr}{\sailRISCVvalisAlignedAddr}{}%
  \ifstrequal{#1}{is\_aligned\_addr}{\sailRISCVvalisAlignedAddr}{}%
  \ifstrequal{#1}{is_none}{\sailRISCVvalisNone}{}%
  \ifstrequal{#1}{is\_none}{\sailRISCVvalisNone}{}%
  \ifstrequal{#1}{is_some}{\sailRISCVvalisSome}{}%
  \ifstrequal{#1}{is\_some}{\sailRISCVvalisSome}{}%
  \ifstrequal{#1}{itype_mnemonic}{\sailRISCVvalitypeMnemonic}{}%
  \ifstrequal{#1}{itype\_mnemonic}{\sailRISCVvalitypeMnemonic}{}%
  \ifstrequal{#1}{legalize_ccsr}{\sailRISCVvallegalizzeCcsr}{}%
  \ifstrequal{#1}{legalize\_ccsr}{\sailRISCVvallegalizzeCcsr}{}%
  \ifstrequal{#1}{legalize_epcc}{\sailRISCVvallegalizzeEpcc}{}%
  \ifstrequal{#1}{legalize\_epcc}{\sailRISCVvallegalizzeEpcc}{}%
  \ifstrequal{#1}{legalize_mcounteren}{\sailRISCVvallegalizzeMcounteren}{}%
  \ifstrequal{#1}{legalize\_mcounteren}{\sailRISCVvallegalizzeMcounteren}{}%
  \ifstrequal{#1}{legalize_mcountinhibit}{\sailRISCVvallegalizzeMcountinhibit}{}%
  \ifstrequal{#1}{legalize\_mcountinhibit}{\sailRISCVvallegalizzeMcountinhibit}{}%
  \ifstrequal{#1}{legalize_medeleg}{\sailRISCVvallegalizzeMedeleg}{}%
  \ifstrequal{#1}{legalize\_medeleg}{\sailRISCVvallegalizzeMedeleg}{}%
  \ifstrequal{#1}{legalize_mideleg}{\sailRISCVvallegalizzeMideleg}{}%
  \ifstrequal{#1}{legalize\_mideleg}{\sailRISCVvallegalizzeMideleg}{}%
  \ifstrequal{#1}{legalize_mie}{\sailRISCVvallegalizzeMie}{}%
  \ifstrequal{#1}{legalize\_mie}{\sailRISCVvallegalizzeMie}{}%
  \ifstrequal{#1}{legalize_mip}{\sailRISCVvallegalizzeMip}{}%
  \ifstrequal{#1}{legalize\_mip}{\sailRISCVvallegalizzeMip}{}%
  \ifstrequal{#1}{legalize_misa}{\sailRISCVvallegalizzeMisa}{}%
  \ifstrequal{#1}{legalize\_misa}{\sailRISCVvallegalizzeMisa}{}%
  \ifstrequal{#1}{legalize_mstatus}{\sailRISCVvallegalizzeMstatus}{}%
  \ifstrequal{#1}{legalize\_mstatus}{\sailRISCVvallegalizzeMstatus}{}%
  \ifstrequal{#1}{legalize_satp}{\sailRISCVvallegalizzeSatp}{}%
  \ifstrequal{#1}{legalize\_satp}{\sailRISCVvallegalizzeSatp}{}%
  \ifstrequal{#1}{legalize_satp32}{\sailRISCVvallegalizzeSatpThreeTwo}{}%
  \ifstrequal{#1}{legalize\_satp32}{\sailRISCVvallegalizzeSatpThreeTwo}{}%
  \ifstrequal{#1}{legalize_satp64}{\sailRISCVvallegalizzeSatpSixFour}{}%
  \ifstrequal{#1}{legalize\_satp64}{\sailRISCVvallegalizzeSatpSixFour}{}%
  \ifstrequal{#1}{legalize_scounteren}{\sailRISCVvallegalizzeScounteren}{}%
  \ifstrequal{#1}{legalize\_scounteren}{\sailRISCVvallegalizzeScounteren}{}%
  \ifstrequal{#1}{legalize_sedeleg}{\sailRISCVvallegalizzeSedeleg}{}%
  \ifstrequal{#1}{legalize\_sedeleg}{\sailRISCVvallegalizzeSedeleg}{}%
  \ifstrequal{#1}{legalize_sie}{\sailRISCVvallegalizzeSie}{}%
  \ifstrequal{#1}{legalize\_sie}{\sailRISCVvallegalizzeSie}{}%
  \ifstrequal{#1}{legalize_sip}{\sailRISCVvallegalizzeSip}{}%
  \ifstrequal{#1}{legalize\_sip}{\sailRISCVvallegalizzeSip}{}%
  \ifstrequal{#1}{legalize_sstatus}{\sailRISCVvallegalizzeSstatus}{}%
  \ifstrequal{#1}{legalize\_sstatus}{\sailRISCVvallegalizzeSstatus}{}%
  \ifstrequal{#1}{legalize_tcc}{\sailRISCVvallegalizzeTcc}{}%
  \ifstrequal{#1}{legalize\_tcc}{\sailRISCVvallegalizzeTcc}{}%
  \ifstrequal{#1}{legalize_tvec}{\sailRISCVvallegalizzeTvec}{}%
  \ifstrequal{#1}{legalize\_tvec}{\sailRISCVvallegalizzeTvec}{}%
  \ifstrequal{#1}{legalize_uie}{\sailRISCVvallegalizzeUie}{}%
  \ifstrequal{#1}{legalize\_uie}{\sailRISCVvallegalizzeUie}{}%
  \ifstrequal{#1}{legalize_uip}{\sailRISCVvallegalizzeUip}{}%
  \ifstrequal{#1}{legalize\_uip}{\sailRISCVvallegalizzeUip}{}%
  \ifstrequal{#1}{legalize_ustatus}{\sailRISCVvallegalizzeUstatus}{}%
  \ifstrequal{#1}{legalize\_ustatus}{\sailRISCVvallegalizzeUstatus}{}%
  \ifstrequal{#1}{legalize_xepc}{\sailRISCVvallegalizzeXepc}{}%
  \ifstrequal{#1}{legalize\_xepc}{\sailRISCVvallegalizzeXepc}{}%
  \ifstrequal{#1}{lift_sie}{\sailRISCVvalliftSie}{}%
  \ifstrequal{#1}{lift\_sie}{\sailRISCVvalliftSie}{}%
  \ifstrequal{#1}{lift_sip}{\sailRISCVvalliftSip}{}%
  \ifstrequal{#1}{lift\_sip}{\sailRISCVvalliftSip}{}%
  \ifstrequal{#1}{lift_sstatus}{\sailRISCVvalliftSstatus}{}%
  \ifstrequal{#1}{lift\_sstatus}{\sailRISCVvalliftSstatus}{}%
  \ifstrequal{#1}{lift_uie}{\sailRISCVvalliftUie}{}%
  \ifstrequal{#1}{lift\_uie}{\sailRISCVvalliftUie}{}%
  \ifstrequal{#1}{lift_uip}{\sailRISCVvalliftUip}{}%
  \ifstrequal{#1}{lift\_uip}{\sailRISCVvalliftUip}{}%
  \ifstrequal{#1}{lift_ustatus}{\sailRISCVvalliftUstatus}{}%
  \ifstrequal{#1}{lift\_ustatus}{\sailRISCVvalliftUstatus}{}%
  \ifstrequal{#1}{load_reservation}{\sailRISCVvalloadReservation}{}%
  \ifstrequal{#1}{load\_reservation}{\sailRISCVvalloadReservation}{}%
  \ifstrequal{#1}{lookup_TLB39}{\sailRISCVvallookupTLBThreeNine}{}%
  \ifstrequal{#1}{lookup\_TLB39}{\sailRISCVvallookupTLBThreeNine}{}%
  \ifstrequal{#1}{lookup_TLB48}{\sailRISCVvallookupTLBFourEight}{}%
  \ifstrequal{#1}{lookup\_TLB48}{\sailRISCVvallookupTLBFourEight}{}%
  \ifstrequal{#1}{loop}{\sailRISCVvalloop}{}%
  \ifstrequal{#1}{lower_mie}{\sailRISCVvallowerMie}{}%
  \ifstrequal{#1}{lower\_mie}{\sailRISCVvallowerMie}{}%
  \ifstrequal{#1}{lower_mip}{\sailRISCVvallowerMip}{}%
  \ifstrequal{#1}{lower\_mip}{\sailRISCVvallowerMip}{}%
  \ifstrequal{#1}{lower_mstatus}{\sailRISCVvallowerMstatus}{}%
  \ifstrequal{#1}{lower\_mstatus}{\sailRISCVvallowerMstatus}{}%
  \ifstrequal{#1}{lower_sie}{\sailRISCVvallowerSie}{}%
  \ifstrequal{#1}{lower\_sie}{\sailRISCVvallowerSie}{}%
  \ifstrequal{#1}{lower_sip}{\sailRISCVvallowerSip}{}%
  \ifstrequal{#1}{lower\_sip}{\sailRISCVvallowerSip}{}%
  \ifstrequal{#1}{lower_sstatus}{\sailRISCVvallowerSstatus}{}%
  \ifstrequal{#1}{lower\_sstatus}{\sailRISCVvallowerSstatus}{}%
  \ifstrequal{#1}{lrsc_width_str}{\sailRISCVvallrscWidthStr}{}%
  \ifstrequal{#1}{lrsc\_width\_str}{\sailRISCVvallrscWidthStr}{}%
  \ifstrequal{#1}{lt_int}{\sailRISCVvalltInt}{}%
  \ifstrequal{#1}{lt\_int}{\sailRISCVvalltInt}{}%
  \ifstrequal{#1}{lteq_int}{\sailRISCVvallteqInt}{}%
  \ifstrequal{#1}{lteq\_int}{\sailRISCVvallteqInt}{}%
  \ifstrequal{#1}{make_TLB_Entry}{\sailRISCVvalmakeTLBEntry}{}%
  \ifstrequal{#1}{make\_TLB\_Entry}{\sailRISCVvalmakeTLBEntry}{}%
  \ifstrequal{#1}{match_TLB_Entry}{\sailRISCVvalmatchTLBEntry}{}%
  \ifstrequal{#1}{match\_TLB\_Entry}{\sailRISCVvalmatchTLBEntry}{}%
  \ifstrequal{#1}{match_reservation}{\sailRISCVvalmatchReservation}{}%
  \ifstrequal{#1}{match\_reservation}{\sailRISCVvalmatchReservation}{}%
  \ifstrequal{#1}{max_int}{\sailRISCVvalmaxInt}{}%
  \ifstrequal{#1}{max\_int}{\sailRISCVvalmaxInt}{}%
  \ifstrequal{#1}{maybe_aq}{\sailRISCVvalmaybeAq}{}%
  \ifstrequal{#1}{maybe\_aq}{\sailRISCVvalmaybeAq}{}%
  \ifstrequal{#1}{maybe_i}{\sailRISCVvalmaybeI}{}%
  \ifstrequal{#1}{maybe\_i}{\sailRISCVvalmaybeI}{}%
  \ifstrequal{#1}{maybe_not_u}{\sailRISCVvalmaybeNotU}{}%
  \ifstrequal{#1}{maybe\_not\_u}{\sailRISCVvalmaybeNotU}{}%
  \ifstrequal{#1}{maybe_rl}{\sailRISCVvalmaybeRl}{}%
  \ifstrequal{#1}{maybe\_rl}{\sailRISCVvalmaybeRl}{}%
  \ifstrequal{#1}{maybe_u}{\sailRISCVvalmaybeU}{}%
  \ifstrequal{#1}{maybe\_u}{\sailRISCVvalmaybeU}{}%
  \ifstrequal{#1}{memBitsToCapability}{\sailRISCVvalmemBitsToCapability}{}%
  \ifstrequal{#1}{mem_read}{\sailRISCVvalmemRead}{}%
  \ifstrequal{#1}{mem\_read}{\sailRISCVvalmemRead}{}%
  \ifstrequal{#1}{mem_read_cap}{\sailRISCVvalmemReadCap}{}%
  \ifstrequal{#1}{mem\_read\_cap}{\sailRISCVvalmemReadCap}{}%
  \ifstrequal{#1}{mem_read_meta}{\sailRISCVvalmemReadMeta}{}%
  \ifstrequal{#1}{mem\_read\_meta}{\sailRISCVvalmemReadMeta}{}%
  \ifstrequal{#1}{mem_read_priv}{\sailRISCVvalmemReadPriv}{}%
  \ifstrequal{#1}{mem\_read\_priv}{\sailRISCVvalmemReadPriv}{}%
  \ifstrequal{#1}{mem_read_priv_meta}{\sailRISCVvalmemReadPrivMeta}{}%
  \ifstrequal{#1}{mem\_read\_priv\_meta}{\sailRISCVvalmemReadPrivMeta}{}%
  \ifstrequal{#1}{mem_write_cap}{\sailRISCVvalmemWriteCap}{}%
  \ifstrequal{#1}{mem\_write\_cap}{\sailRISCVvalmemWriteCap}{}%
  \ifstrequal{#1}{mem_write_ea}{\sailRISCVvalmemWriteEa}{}%
  \ifstrequal{#1}{mem\_write\_ea}{\sailRISCVvalmemWriteEa}{}%
  \ifstrequal{#1}{mem_write_ea_cap}{\sailRISCVvalmemWriteEaCap}{}%
  \ifstrequal{#1}{mem\_write\_ea\_cap}{\sailRISCVvalmemWriteEaCap}{}%
  \ifstrequal{#1}{mem_write_value}{\sailRISCVvalmemWriteValue}{}%
  \ifstrequal{#1}{mem\_write\_value}{\sailRISCVvalmemWriteValue}{}%
  \ifstrequal{#1}{mem_write_value_meta}{\sailRISCVvalmemWriteValueMeta}{}%
  \ifstrequal{#1}{mem\_write\_value\_meta}{\sailRISCVvalmemWriteValueMeta}{}%
  \ifstrequal{#1}{mem_write_value_priv}{\sailRISCVvalmemWriteValuePriv}{}%
  \ifstrequal{#1}{mem\_write\_value\_priv}{\sailRISCVvalmemWriteValuePriv}{}%
  \ifstrequal{#1}{mem_write_value_priv_meta}{\sailRISCVvalmemWriteValuePrivMeta}{}%
  \ifstrequal{#1}{mem\_write\_value\_priv\_meta}{\sailRISCVvalmemWriteValuePrivMeta}{}%
  \ifstrequal{#1}{min_instruction_bytes}{\sailRISCVvalminInstructionBytes}{}%
  \ifstrequal{#1}{min\_instruction\_bytes}{\sailRISCVvalminInstructionBytes}{}%
  \ifstrequal{#1}{min_int}{\sailRISCVvalminInt}{}%
  \ifstrequal{#1}{min\_int}{\sailRISCVvalminInt}{}%
  \ifstrequal{#1}{mmio_read}{\sailRISCVvalmmioRead}{}%
  \ifstrequal{#1}{mmio\_read}{\sailRISCVvalmmioRead}{}%
  \ifstrequal{#1}{mmio_write}{\sailRISCVvalmmioWrite}{}%
  \ifstrequal{#1}{mmio\_write}{\sailRISCVvalmmioWrite}{}%
  \ifstrequal{#1}{mul_mnemonic}{\sailRISCVvalmulMnemonic}{}%
  \ifstrequal{#1}{mul\_mnemonic}{\sailRISCVvalmulMnemonic}{}%
  \ifstrequal{#1}{mult_atom}{\sailRISCVvalmultAtom}{}%
  \ifstrequal{#1}{mult\_atom}{\sailRISCVvalmultAtom}{}%
  \ifstrequal{#1}{mult_int}{\sailRISCVvalmultInt}{}%
  \ifstrequal{#1}{mult\_int}{\sailRISCVvalmultInt}{}%
  \ifstrequal{#1}{n_leading_spaces}{\sailRISCVvalnLeadingSpaces}{}%
  \ifstrequal{#1}{n\_leading\_spaces}{\sailRISCVvalnLeadingSpaces}{}%
  \ifstrequal{#1}{nan_box_H}{\sailRISCVvalnanBoxH}{}%
  \ifstrequal{#1}{nan\_box\_H}{\sailRISCVvalnanBoxH}{}%
  \ifstrequal{#1}{nan_box_S}{\sailRISCVvalnanBoxS}{}%
  \ifstrequal{#1}{nan\_box\_S}{\sailRISCVvalnanBoxS}{}%
  \ifstrequal{#1}{nan_unbox_H}{\sailRISCVvalnanUnboxH}{}%
  \ifstrequal{#1}{nan\_unbox\_H}{\sailRISCVvalnanUnboxH}{}%
  \ifstrequal{#1}{nan_unbox_S}{\sailRISCVvalnanUnboxS}{}%
  \ifstrequal{#1}{nan\_unbox\_S}{\sailRISCVvalnanUnboxS}{}%
  \ifstrequal{#1}{negate_D}{\sailRISCVvalnegateD}{}%
  \ifstrequal{#1}{negate\_D}{\sailRISCVvalnegateD}{}%
  \ifstrequal{#1}{negate_S}{\sailRISCVvalnegateS}{}%
  \ifstrequal{#1}{negate\_S}{\sailRISCVvalnegateS}{}%
  \ifstrequal{#1}{negate_atom}{\sailRISCVvalnegateAtom}{}%
  \ifstrequal{#1}{negate\_atom}{\sailRISCVvalnegateAtom}{}%
  \ifstrequal{#1}{negate_int}{\sailRISCVvalnegateInt}{}%
  \ifstrequal{#1}{negate\_int}{\sailRISCVvalnegateInt}{}%
  \ifstrequal{#1}{neq_anything}{\sailRISCVvalneqAnything}{}%
  \ifstrequal{#1}{neq\_anything}{\sailRISCVvalneqAnything}{}%
  \ifstrequal{#1}{neq_bits}{\sailRISCVvalneqBits}{}%
  \ifstrequal{#1}{neq\_bits}{\sailRISCVvalneqBits}{}%
  \ifstrequal{#1}{neq_bool}{\sailRISCVvalneqBool}{}%
  \ifstrequal{#1}{neq\_bool}{\sailRISCVvalneqBool}{}%
  \ifstrequal{#1}{neq_int}{\sailRISCVvalneqInt}{}%
  \ifstrequal{#1}{neq\_int}{\sailRISCVvalneqInt}{}%
  \ifstrequal{#1}{neq_vec}{\sailRISCVvalneqVec}{}%
  \ifstrequal{#1}{neq\_vec}{\sailRISCVvalneqVec}{}%
  \ifstrequal{#1}{not}{\sailRISCVvalnot}{}%
  \ifstrequal{#1}{not_bit}{\sailRISCVvalnotBit}{}%
  \ifstrequal{#1}{not\_bit}{\sailRISCVvalnotBit}{}%
  \ifstrequal{#1}{not_bool}{\sailRISCVvalnotBool}{}%
  \ifstrequal{#1}{not\_bool}{\sailRISCVvalnotBool}{}%
  \ifstrequal{#1}{not_implemented}{\sailRISCVvalnotImplemented}{}%
  \ifstrequal{#1}{not\_implemented}{\sailRISCVvalnotImplemented}{}%
  \ifstrequal{#1}{not_vec}{\sailRISCVvalnotVec}{}%
  \ifstrequal{#1}{not\_vec}{\sailRISCVvalnotVec}{}%
  \ifstrequal{#1}{num_of_Architecture}{\sailRISCVvalnumOfArchitecture}{}%
  \ifstrequal{#1}{num\_of\_Architecture}{\sailRISCVvalnumOfArchitecture}{}%
  \ifstrequal{#1}{num_of_CPtrCmpOp}{\sailRISCVvalnumOfCPtrCmpOp}{}%
  \ifstrequal{#1}{num\_of\_CPtrCmpOp}{\sailRISCVvalnumOfCPtrCmpOp}{}%
  \ifstrequal{#1}{num_of_CapEx}{\sailRISCVvalnumOfCapEx}{}%
  \ifstrequal{#1}{num\_of\_CapEx}{\sailRISCVvalnumOfCapEx}{}%
  \ifstrequal{#1}{num_of_ClearRegSet}{\sailRISCVvalnumOfClearRegSet}{}%
  \ifstrequal{#1}{num\_of\_ClearRegSet}{\sailRISCVvalnumOfClearRegSet}{}%
  \ifstrequal{#1}{num_of_ExceptionType}{\sailRISCVvalnumOfExceptionType}{}%
  \ifstrequal{#1}{num\_of\_ExceptionType}{\sailRISCVvalnumOfExceptionType}{}%
  \ifstrequal{#1}{num_of_ExtStatus}{\sailRISCVvalnumOfExtStatus}{}%
  \ifstrequal{#1}{num\_of\_ExtStatus}{\sailRISCVvalnumOfExtStatus}{}%
  \ifstrequal{#1}{num_of_InterruptType}{\sailRISCVvalnumOfInterruptType}{}%
  \ifstrequal{#1}{num\_of\_InterruptType}{\sailRISCVvalnumOfInterruptType}{}%
  \ifstrequal{#1}{num_of_PmpAddrMatchType}{\sailRISCVvalnumOfPmpAddrMatchType}{}%
  \ifstrequal{#1}{num\_of\_PmpAddrMatchType}{\sailRISCVvalnumOfPmpAddrMatchType}{}%
  \ifstrequal{#1}{num_of_Privilege}{\sailRISCVvalnumOfPrivilege}{}%
  \ifstrequal{#1}{num\_of\_Privilege}{\sailRISCVvalnumOfPrivilege}{}%
  \ifstrequal{#1}{num_of_Retired}{\sailRISCVvalnumOfRetired}{}%
  \ifstrequal{#1}{num\_of\_Retired}{\sailRISCVvalnumOfRetired}{}%
  \ifstrequal{#1}{num_of_SATPMode}{\sailRISCVvalnumOfSATPMode}{}%
  \ifstrequal{#1}{num\_of\_SATPMode}{\sailRISCVvalnumOfSATPMode}{}%
  \ifstrequal{#1}{num_of_TrapVectorMode}{\sailRISCVvalnumOfTrapVectorMode}{}%
  \ifstrequal{#1}{num\_of\_TrapVectorMode}{\sailRISCVvalnumOfTrapVectorMode}{}%
  \ifstrequal{#1}{num_of_a64_barrier_domain}{\sailRISCVvalnumOfASixFourBarrierDomain}{}%
  \ifstrequal{#1}{num\_of\_a64\_barrier\_domain}{\sailRISCVvalnumOfASixFourBarrierDomain}{}%
  \ifstrequal{#1}{num_of_a64_barrier_type}{\sailRISCVvalnumOfASixFourBarrierType}{}%
  \ifstrequal{#1}{num\_of\_a64\_barrier\_type}{\sailRISCVvalnumOfASixFourBarrierType}{}%
  \ifstrequal{#1}{num_of_amoop}{\sailRISCVvalnumOfAmoop}{}%
  \ifstrequal{#1}{num\_of\_amoop}{\sailRISCVvalnumOfAmoop}{}%
  \ifstrequal{#1}{num_of_biop_zbs}{\sailRISCVvalnumOfBiopZbs}{}%
  \ifstrequal{#1}{num\_of\_biop\_zbs}{\sailRISCVvalnumOfBiopZbs}{}%
  \ifstrequal{#1}{num_of_bop}{\sailRISCVvalnumOfBop}{}%
  \ifstrequal{#1}{num\_of\_bop}{\sailRISCVvalnumOfBop}{}%
  \ifstrequal{#1}{num_of_brop_zba}{\sailRISCVvalnumOfBropZba}{}%
  \ifstrequal{#1}{num\_of\_brop\_zba}{\sailRISCVvalnumOfBropZba}{}%
  \ifstrequal{#1}{num_of_brop_zbb}{\sailRISCVvalnumOfBropZbb}{}%
  \ifstrequal{#1}{num\_of\_brop\_zbb}{\sailRISCVvalnumOfBropZbb}{}%
  \ifstrequal{#1}{num_of_brop_zbkb}{\sailRISCVvalnumOfBropZbkb}{}%
  \ifstrequal{#1}{num\_of\_brop\_zbkb}{\sailRISCVvalnumOfBropZbkb}{}%
  \ifstrequal{#1}{num_of_brop_zbs}{\sailRISCVvalnumOfBropZbs}{}%
  \ifstrequal{#1}{num\_of\_brop\_zbs}{\sailRISCVvalnumOfBropZbs}{}%
  \ifstrequal{#1}{num_of_bropw_zba}{\sailRISCVvalnumOfBropwZba}{}%
  \ifstrequal{#1}{num\_of\_bropw\_zba}{\sailRISCVvalnumOfBropwZba}{}%
  \ifstrequal{#1}{num_of_bropw_zbb}{\sailRISCVvalnumOfBropwZbb}{}%
  \ifstrequal{#1}{num\_of\_bropw\_zbb}{\sailRISCVvalnumOfBropwZbb}{}%
  \ifstrequal{#1}{num_of_cache_op_kind}{\sailRISCVvalnumOfCacheOpKind}{}%
  \ifstrequal{#1}{num\_of\_cache\_op\_kind}{\sailRISCVvalnumOfCacheOpKind}{}%
  \ifstrequal{#1}{num_of_csrop}{\sailRISCVvalnumOfCsrop}{}%
  \ifstrequal{#1}{num\_of\_csrop}{\sailRISCVvalnumOfCsrop}{}%
  \ifstrequal{#1}{num_of_ext_access_type}{\sailRISCVvalnumOfExtAccessType}{}%
  \ifstrequal{#1}{num\_of\_ext\_access\_type}{\sailRISCVvalnumOfExtAccessType}{}%
  \ifstrequal{#1}{num_of_ext_exc_type}{\sailRISCVvalnumOfExtExcType}{}%
  \ifstrequal{#1}{num\_of\_ext\_exc\_type}{\sailRISCVvalnumOfExtExcType}{}%
  \ifstrequal{#1}{num_of_ext_ptw_error}{\sailRISCVvalnumOfExtPtwError}{}%
  \ifstrequal{#1}{num\_of\_ext\_ptw\_error}{\sailRISCVvalnumOfExtPtwError}{}%
  \ifstrequal{#1}{num_of_ext_ptw_fail}{\sailRISCVvalnumOfExtPtwFail}{}%
  \ifstrequal{#1}{num\_of\_ext\_ptw\_fail}{\sailRISCVvalnumOfExtPtwFail}{}%
  \ifstrequal{#1}{num_of_ext_ptw_lc}{\sailRISCVvalnumOfExtPtwLc}{}%
  \ifstrequal{#1}{num\_of\_ext\_ptw\_lc}{\sailRISCVvalnumOfExtPtwLc}{}%
  \ifstrequal{#1}{num_of_ext_ptw_sc}{\sailRISCVvalnumOfExtPtwSc}{}%
  \ifstrequal{#1}{num\_of\_ext\_ptw\_sc}{\sailRISCVvalnumOfExtPtwSc}{}%
  \ifstrequal{#1}{num_of_extop_zbb}{\sailRISCVvalnumOfExtopZbb}{}%
  \ifstrequal{#1}{num\_of\_extop\_zbb}{\sailRISCVvalnumOfExtopZbb}{}%
  \ifstrequal{#1}{num_of_f_bin_op_D}{\sailRISCVvalnumOfFBinOpD}{}%
  \ifstrequal{#1}{num\_of\_f\_bin\_op\_D}{\sailRISCVvalnumOfFBinOpD}{}%
  \ifstrequal{#1}{num_of_f_bin_op_H}{\sailRISCVvalnumOfFBinOpH}{}%
  \ifstrequal{#1}{num\_of\_f\_bin\_op\_H}{\sailRISCVvalnumOfFBinOpH}{}%
  \ifstrequal{#1}{num_of_f_bin_op_S}{\sailRISCVvalnumOfFBinOpS}{}%
  \ifstrequal{#1}{num\_of\_f\_bin\_op\_S}{\sailRISCVvalnumOfFBinOpS}{}%
  \ifstrequal{#1}{num_of_f_bin_rm_op_D}{\sailRISCVvalnumOfFBinRmOpD}{}%
  \ifstrequal{#1}{num\_of\_f\_bin\_rm\_op\_D}{\sailRISCVvalnumOfFBinRmOpD}{}%
  \ifstrequal{#1}{num_of_f_bin_rm_op_H}{\sailRISCVvalnumOfFBinRmOpH}{}%
  \ifstrequal{#1}{num\_of\_f\_bin\_rm\_op\_H}{\sailRISCVvalnumOfFBinRmOpH}{}%
  \ifstrequal{#1}{num_of_f_bin_rm_op_S}{\sailRISCVvalnumOfFBinRmOpS}{}%
  \ifstrequal{#1}{num\_of\_f\_bin\_rm\_op\_S}{\sailRISCVvalnumOfFBinRmOpS}{}%
  \ifstrequal{#1}{num_of_f_madd_op_D}{\sailRISCVvalnumOfFMaddOpD}{}%
  \ifstrequal{#1}{num\_of\_f\_madd\_op\_D}{\sailRISCVvalnumOfFMaddOpD}{}%
  \ifstrequal{#1}{num_of_f_madd_op_H}{\sailRISCVvalnumOfFMaddOpH}{}%
  \ifstrequal{#1}{num\_of\_f\_madd\_op\_H}{\sailRISCVvalnumOfFMaddOpH}{}%
  \ifstrequal{#1}{num_of_f_madd_op_S}{\sailRISCVvalnumOfFMaddOpS}{}%
  \ifstrequal{#1}{num\_of\_f\_madd\_op\_S}{\sailRISCVvalnumOfFMaddOpS}{}%
  \ifstrequal{#1}{num_of_f_un_op_D}{\sailRISCVvalnumOfFUnOpD}{}%
  \ifstrequal{#1}{num\_of\_f\_un\_op\_D}{\sailRISCVvalnumOfFUnOpD}{}%
  \ifstrequal{#1}{num_of_f_un_op_H}{\sailRISCVvalnumOfFUnOpH}{}%
  \ifstrequal{#1}{num\_of\_f\_un\_op\_H}{\sailRISCVvalnumOfFUnOpH}{}%
  \ifstrequal{#1}{num_of_f_un_op_S}{\sailRISCVvalnumOfFUnOpS}{}%
  \ifstrequal{#1}{num\_of\_f\_un\_op\_S}{\sailRISCVvalnumOfFUnOpS}{}%
  \ifstrequal{#1}{num_of_f_un_rm_op_D}{\sailRISCVvalnumOfFUnRmOpD}{}%
  \ifstrequal{#1}{num\_of\_f\_un\_rm\_op\_D}{\sailRISCVvalnumOfFUnRmOpD}{}%
  \ifstrequal{#1}{num_of_f_un_rm_op_H}{\sailRISCVvalnumOfFUnRmOpH}{}%
  \ifstrequal{#1}{num\_of\_f\_un\_rm\_op\_H}{\sailRISCVvalnumOfFUnRmOpH}{}%
  \ifstrequal{#1}{num_of_f_un_rm_op_S}{\sailRISCVvalnumOfFUnRmOpS}{}%
  \ifstrequal{#1}{num\_of\_f\_un\_rm\_op\_S}{\sailRISCVvalnumOfFUnRmOpS}{}%
  \ifstrequal{#1}{num_of_iop}{\sailRISCVvalnumOfIop}{}%
  \ifstrequal{#1}{num\_of\_iop}{\sailRISCVvalnumOfIop}{}%
  \ifstrequal{#1}{num_of_pmpAddrMatch}{\sailRISCVvalnumOfPmpAddrMatch}{}%
  \ifstrequal{#1}{num\_of\_pmpAddrMatch}{\sailRISCVvalnumOfPmpAddrMatch}{}%
  \ifstrequal{#1}{num_of_pmpMatch}{\sailRISCVvalnumOfPmpMatch}{}%
  \ifstrequal{#1}{num\_of\_pmpMatch}{\sailRISCVvalnumOfPmpMatch}{}%
  \ifstrequal{#1}{num_of_read_kind}{\sailRISCVvalnumOfReadKind}{}%
  \ifstrequal{#1}{num\_of\_read\_kind}{\sailRISCVvalnumOfReadKind}{}%
  \ifstrequal{#1}{num_of_rop}{\sailRISCVvalnumOfRop}{}%
  \ifstrequal{#1}{num\_of\_rop}{\sailRISCVvalnumOfRop}{}%
  \ifstrequal{#1}{num_of_ropw}{\sailRISCVvalnumOfRopw}{}%
  \ifstrequal{#1}{num\_of\_ropw}{\sailRISCVvalnumOfRopw}{}%
  \ifstrequal{#1}{num_of_rounding_mode}{\sailRISCVvalnumOfRoundingMode}{}%
  \ifstrequal{#1}{num\_of\_rounding\_mode}{\sailRISCVvalnumOfRoundingMode}{}%
  \ifstrequal{#1}{num_of_seed_opst}{\sailRISCVvalnumOfSeedOpst}{}%
  \ifstrequal{#1}{num\_of\_seed\_opst}{\sailRISCVvalnumOfSeedOpst}{}%
  \ifstrequal{#1}{num_of_sop}{\sailRISCVvalnumOfSop}{}%
  \ifstrequal{#1}{num\_of\_sop}{\sailRISCVvalnumOfSop}{}%
  \ifstrequal{#1}{num_of_sopw}{\sailRISCVvalnumOfSopw}{}%
  \ifstrequal{#1}{num\_of\_sopw}{\sailRISCVvalnumOfSopw}{}%
  \ifstrequal{#1}{num_of_trans_kind}{\sailRISCVvalnumOfTransKind}{}%
  \ifstrequal{#1}{num\_of\_trans\_kind}{\sailRISCVvalnumOfTransKind}{}%
  \ifstrequal{#1}{num_of_uop}{\sailRISCVvalnumOfUop}{}%
  \ifstrequal{#1}{num\_of\_uop}{\sailRISCVvalnumOfUop}{}%
  \ifstrequal{#1}{num_of_word_width}{\sailRISCVvalnumOfWordWidth}{}%
  \ifstrequal{#1}{num\_of\_word\_width}{\sailRISCVvalnumOfWordWidth}{}%
  \ifstrequal{#1}{num_of_write_kind}{\sailRISCVvalnumOfWriteKind}{}%
  \ifstrequal{#1}{num\_of\_write\_kind}{\sailRISCVvalnumOfWriteKind}{}%
  \ifstrequal{#1}{nvFlag}{\sailRISCVvalnvFlag}{}%
  \ifstrequal{#1}{nxFlag}{\sailRISCVvalnxFlag}{}%
  \ifstrequal{#1}{ofFlag}{\sailRISCVvalofFlag}{}%
  \ifstrequal{#1}{ones}{\sailRISCVvalones}{}%
  \ifstrequal{#1}{opst_code}{\sailRISCVvalopstCode}{}%
  \ifstrequal{#1}{opst\_code}{\sailRISCVvalopstCode}{}%
  \ifstrequal{#1}{opt_spc}{\sailRISCVvaloptSpc}{}%
  \ifstrequal{#1}{opt\_spc}{\sailRISCVvaloptSpc}{}%
  \ifstrequal{#1}{opt_spc_backwards}{\sailRISCVvaloptSpcBackwards}{}%
  \ifstrequal{#1}{opt\_spc\_backwards}{\sailRISCVvaloptSpcBackwards}{}%
  \ifstrequal{#1}{opt_spc_forwards}{\sailRISCVvaloptSpcForwards}{}%
  \ifstrequal{#1}{opt\_spc\_forwards}{\sailRISCVvaloptSpcForwards}{}%
  \ifstrequal{#1}{opt_spc_matches_prefix}{\sailRISCVvaloptSpcMatchesPrefix}{}%
  \ifstrequal{#1}{opt\_spc\_matches\_prefix}{\sailRISCVvaloptSpcMatchesPrefix}{}%
  \ifstrequal{#1}{or_bool}{\sailRISCVvalorBool}{}%
  \ifstrequal{#1}{or\_bool}{\sailRISCVvalorBool}{}%
  \ifstrequal{#1}{or_vec}{\sailRISCVvalorVec}{}%
  \ifstrequal{#1}{or\_vec}{\sailRISCVvalorVec}{}%
  \ifstrequal{#1}{pc_alignment_mask}{\sailRISCVvalpcAlignmentMask}{}%
  \ifstrequal{#1}{pc\_alignment\_mask}{\sailRISCVvalpcAlignmentMask}{}%
  \ifstrequal{#1}{pcc_access_system_regs}{\sailRISCVvalpccAccessSystemRegs}{}%
  \ifstrequal{#1}{pcc\_access\_system\_regs}{\sailRISCVvalpccAccessSystemRegs}{}%
  \ifstrequal{#1}{phys_mem_read}{\sailRISCVvalphysMemRead}{}%
  \ifstrequal{#1}{phys\_mem\_read}{\sailRISCVvalphysMemRead}{}%
  \ifstrequal{#1}{phys_mem_segments}{\sailRISCVvalphysMemSegments}{}%
  \ifstrequal{#1}{phys\_mem\_segments}{\sailRISCVvalphysMemSegments}{}%
  \ifstrequal{#1}{phys_mem_write}{\sailRISCVvalphysMemWrite}{}%
  \ifstrequal{#1}{phys\_mem\_write}{\sailRISCVvalphysMemWrite}{}%
  \ifstrequal{#1}{plain_vector_access}{\sailRISCVvalplainVectorAccess}{}%
  \ifstrequal{#1}{plain\_vector\_access}{\sailRISCVvalplainVectorAccess}{}%
  \ifstrequal{#1}{plain_vector_update}{\sailRISCVvalplainVectorUpdate}{}%
  \ifstrequal{#1}{plain\_vector\_update}{\sailRISCVvalplainVectorUpdate}{}%
  \ifstrequal{#1}{plat_clint_base}{\sailRISCVvalplatClintBase}{}%
  \ifstrequal{#1}{plat\_clint\_base}{\sailRISCVvalplatClintBase}{}%
  \ifstrequal{#1}{plat_clint_size}{\sailRISCVvalplatClintSizze}{}%
  \ifstrequal{#1}{plat\_clint\_size}{\sailRISCVvalplatClintSizze}{}%
  \ifstrequal{#1}{plat_enable_dirty_update}{\sailRISCVvalplatEnableDirtyUpdate}{}%
  \ifstrequal{#1}{plat\_enable\_dirty\_update}{\sailRISCVvalplatEnableDirtyUpdate}{}%
  \ifstrequal{#1}{plat_enable_misaligned_access}{\sailRISCVvalplatEnableMisalignedAccess}{}%
  \ifstrequal{#1}{plat\_enable\_misaligned\_access}{\sailRISCVvalplatEnableMisalignedAccess}{}%
  \ifstrequal{#1}{plat_enable_pmp}{\sailRISCVvalplatEnablePmp}{}%
  \ifstrequal{#1}{plat\_enable\_pmp}{\sailRISCVvalplatEnablePmp}{}%
  \ifstrequal{#1}{plat_htif_tohost}{\sailRISCVvalplatHtifTohost}{}%
  \ifstrequal{#1}{plat\_htif\_tohost}{\sailRISCVvalplatHtifTohost}{}%
  \ifstrequal{#1}{plat_insns_per_tick}{\sailRISCVvalplatInsnsPerTick}{}%
  \ifstrequal{#1}{plat\_insns\_per\_tick}{\sailRISCVvalplatInsnsPerTick}{}%
  \ifstrequal{#1}{plat_mtval_has_illegal_inst_bits}{\sailRISCVvalplatMtvalHasIllegalInstBits}{}%
  \ifstrequal{#1}{plat\_mtval\_has\_illegal\_inst\_bits}{\sailRISCVvalplatMtvalHasIllegalInstBits}{}%
  \ifstrequal{#1}{plat_ram_base}{\sailRISCVvalplatRamBase}{}%
  \ifstrequal{#1}{plat\_ram\_base}{\sailRISCVvalplatRamBase}{}%
  \ifstrequal{#1}{plat_ram_size}{\sailRISCVvalplatRamSizze}{}%
  \ifstrequal{#1}{plat\_ram\_size}{\sailRISCVvalplatRamSizze}{}%
  \ifstrequal{#1}{plat_rom_base}{\sailRISCVvalplatRomBase}{}%
  \ifstrequal{#1}{plat\_rom\_base}{\sailRISCVvalplatRomBase}{}%
  \ifstrequal{#1}{plat_rom_size}{\sailRISCVvalplatRomSizze}{}%
  \ifstrequal{#1}{plat\_rom\_size}{\sailRISCVvalplatRomSizze}{}%
  \ifstrequal{#1}{plat_term_read}{\sailRISCVvalplatTermRead}{}%
  \ifstrequal{#1}{plat\_term\_read}{\sailRISCVvalplatTermRead}{}%
  \ifstrequal{#1}{plat_term_write}{\sailRISCVvalplatTermWrite}{}%
  \ifstrequal{#1}{plat\_term\_write}{\sailRISCVvalplatTermWrite}{}%
  \ifstrequal{#1}{platform_wfi}{\sailRISCVvalplatformWfi}{}%
  \ifstrequal{#1}{platform\_wfi}{\sailRISCVvalplatformWfi}{}%
  \ifstrequal{#1}{pmpAddrMatchType_of_bits}{\sailRISCVvalpmpAddrMatchTypeOfBits}{}%
  \ifstrequal{#1}{pmpAddrMatchType\_of\_bits}{\sailRISCVvalpmpAddrMatchTypeOfBits}{}%
  \ifstrequal{#1}{pmpAddrMatchType_to_bits}{\sailRISCVvalpmpAddrMatchTypeToBits}{}%
  \ifstrequal{#1}{pmpAddrMatchType\_to\_bits}{\sailRISCVvalpmpAddrMatchTypeToBits}{}%
  \ifstrequal{#1}{pmpAddrMatch_of_num}{\sailRISCVvalpmpAddrMatchOfNum}{}%
  \ifstrequal{#1}{pmpAddrMatch\_of\_num}{\sailRISCVvalpmpAddrMatchOfNum}{}%
  \ifstrequal{#1}{pmpAddrRange}{\sailRISCVvalpmpAddrRangeA}{}%
  \ifstrequal{#1}{pmpCheck}{\sailRISCVvalpmpCheck}{}%
  \ifstrequal{#1}{pmpCheckPerms}{\sailRISCVvalpmpCheckPerms}{}%
  \ifstrequal{#1}{pmpCheckRWX}{\sailRISCVvalpmpCheckRWX}{}%
  \ifstrequal{#1}{pmpLocked}{\sailRISCVvalpmpLocked}{}%
  \ifstrequal{#1}{pmpMatchAddr}{\sailRISCVvalpmpMatchAddr}{}%
  \ifstrequal{#1}{pmpMatchEntry}{\sailRISCVvalpmpMatchEntry}{}%
  \ifstrequal{#1}{pmpMatch_of_num}{\sailRISCVvalpmpMatchOfNum}{}%
  \ifstrequal{#1}{pmpMatch\_of\_num}{\sailRISCVvalpmpMatchOfNum}{}%
  \ifstrequal{#1}{pmpReadCfgReg}{\sailRISCVvalpmpReadCfgReg}{}%
  \ifstrequal{#1}{pmpTORLocked}{\sailRISCVvalpmpTORLocked}{}%
  \ifstrequal{#1}{pmpWriteAddr}{\sailRISCVvalpmpWriteAddr}{}%
  \ifstrequal{#1}{pmpWriteCfg}{\sailRISCVvalpmpWriteCfg}{}%
  \ifstrequal{#1}{pmpWriteCfgReg}{\sailRISCVvalpmpWriteCfgReg}{}%
  \ifstrequal{#1}{pmp_mem_read}{\sailRISCVvalpmpMemRead}{}%
  \ifstrequal{#1}{pmp\_mem\_read}{\sailRISCVvalpmpMemRead}{}%
  \ifstrequal{#1}{pmp_mem_write}{\sailRISCVvalpmpMemWrite}{}%
  \ifstrequal{#1}{pmp\_mem\_write}{\sailRISCVvalpmpMemWrite}{}%
  \ifstrequal{#1}{pow2}{\sailRISCVvalpowTwo}{}%
  \ifstrequal{#1}{prepare_trap_vector}{\sailRISCVvalprepareTrapVector}{}%
  \ifstrequal{#1}{prepare\_trap\_vector}{\sailRISCVvalprepareTrapVector}{}%
  \ifstrequal{#1}{prepare_xret_target}{\sailRISCVvalprepareXretTarget}{}%
  \ifstrequal{#1}{prepare\_xret\_target}{\sailRISCVvalprepareXretTarget}{}%
  \ifstrequal{#1}{prerr_bits}{\sailRISCVvalprerrBits}{}%
  \ifstrequal{#1}{prerr\_bits}{\sailRISCVvalprerrBits}{}%
  \ifstrequal{#1}{prerr_endline}{\sailRISCVvalprerrEndline}{}%
  \ifstrequal{#1}{prerr\_endline}{\sailRISCVvalprerrEndline}{}%
  \ifstrequal{#1}{prerr_int}{\sailRISCVvalprerrInt}{}%
  \ifstrequal{#1}{prerr\_int}{\sailRISCVvalprerrInt}{}%
  \ifstrequal{#1}{print}{\sailRISCVvalprint}{}%
  \ifstrequal{#1}{print_bits}{\sailRISCVvalprintBits}{}%
  \ifstrequal{#1}{print\_bits}{\sailRISCVvalprintBits}{}%
  \ifstrequal{#1}{print_endline}{\sailRISCVvalprintEndline}{}%
  \ifstrequal{#1}{print\_endline}{\sailRISCVvalprintEndline}{}%
  \ifstrequal{#1}{print_insn}{\sailRISCVvalprintInsn}{}%
  \ifstrequal{#1}{print\_insn}{\sailRISCVvalprintInsn}{}%
  \ifstrequal{#1}{print_instr}{\sailRISCVvalprintInstr}{}%
  \ifstrequal{#1}{print\_instr}{\sailRISCVvalprintInstr}{}%
  \ifstrequal{#1}{print_int}{\sailRISCVvalprintInt}{}%
  \ifstrequal{#1}{print\_int}{\sailRISCVvalprintInt}{}%
  \ifstrequal{#1}{print_mem}{\sailRISCVvalprintMem}{}%
  \ifstrequal{#1}{print\_mem}{\sailRISCVvalprintMem}{}%
  \ifstrequal{#1}{print_platform}{\sailRISCVvalprintPlatform}{}%
  \ifstrequal{#1}{print\_platform}{\sailRISCVvalprintPlatform}{}%
  \ifstrequal{#1}{print_reg}{\sailRISCVvalprintReg}{}%
  \ifstrequal{#1}{print\_reg}{\sailRISCVvalprintReg}{}%
  \ifstrequal{#1}{print_string}{\sailRISCVvalprintString}{}%
  \ifstrequal{#1}{print\_string}{\sailRISCVvalprintString}{}%
  \ifstrequal{#1}{privLevel_of_bits}{\sailRISCVvalprivLevelOfBits}{}%
  \ifstrequal{#1}{privLevel\_of\_bits}{\sailRISCVvalprivLevelOfBits}{}%
  \ifstrequal{#1}{privLevel_to_bits}{\sailRISCVvalprivLevelToBits}{}%
  \ifstrequal{#1}{privLevel\_to\_bits}{\sailRISCVvalprivLevelToBits}{}%
  \ifstrequal{#1}{privLevel_to_str}{\sailRISCVvalprivLevelToStr}{}%
  \ifstrequal{#1}{privLevel\_to\_str}{\sailRISCVvalprivLevelToStr}{}%
  \ifstrequal{#1}{processPending}{\sailRISCVvalprocessPending}{}%
  \ifstrequal{#1}{process_fload16}{\sailRISCVvalprocessFloadOneSix}{}%
  \ifstrequal{#1}{process\_fload16}{\sailRISCVvalprocessFloadOneSix}{}%
  \ifstrequal{#1}{process_fload32}{\sailRISCVvalprocessFloadThreeTwo}{}%
  \ifstrequal{#1}{process\_fload32}{\sailRISCVvalprocessFloadThreeTwo}{}%
  \ifstrequal{#1}{process_fload64}{\sailRISCVvalprocessFloadSixFour}{}%
  \ifstrequal{#1}{process\_fload64}{\sailRISCVvalprocessFloadSixFour}{}%
  \ifstrequal{#1}{process_fstore}{\sailRISCVvalprocessFstore}{}%
  \ifstrequal{#1}{process\_fstore}{\sailRISCVvalprocessFstore}{}%
  \ifstrequal{#1}{process_load}{\sailRISCVvalprocessLoad}{}%
  \ifstrequal{#1}{process\_load}{\sailRISCVvalprocessLoad}{}%
  \ifstrequal{#1}{process_loadres}{\sailRISCVvalprocessLoadres}{}%
  \ifstrequal{#1}{process\_loadres}{\sailRISCVvalprocessLoadres}{}%
  \ifstrequal{#1}{ptw_error_to_str}{\sailRISCVvalptwErrorToStr}{}%
  \ifstrequal{#1}{ptw\_error\_to\_str}{\sailRISCVvalptwErrorToStr}{}%
  \ifstrequal{#1}{quot_round_zero}{\sailRISCVvalquotRoundZero}{}%
  \ifstrequal{#1}{quot\_round\_zero}{\sailRISCVvalquotRoundZero}{}%
  \ifstrequal{#1}{rC}{\sailRISCVvalrC}{}%
  \ifstrequal{#1}{rC_bits}{\sailRISCVvalrCBits}{}%
  \ifstrequal{#1}{rC\_bits}{\sailRISCVvalrCBits}{}%
  \ifstrequal{#1}{rF}{\sailRISCVvalrF}{}%
  \ifstrequal{#1}{rF_bits}{\sailRISCVvalrFBits}{}%
  \ifstrequal{#1}{rF\_bits}{\sailRISCVvalrFBits}{}%
  \ifstrequal{#1}{rF_or_X_D}{\sailRISCVvalrFOrXD}{}%
  \ifstrequal{#1}{rF\_or\_X\_D}{\sailRISCVvalrFOrXD}{}%
  \ifstrequal{#1}{rF_or_X_H}{\sailRISCVvalrFOrXH}{}%
  \ifstrequal{#1}{rF\_or\_X\_H}{\sailRISCVvalrFOrXH}{}%
  \ifstrequal{#1}{rF_or_X_S}{\sailRISCVvalrFOrXS}{}%
  \ifstrequal{#1}{rF\_or\_X\_S}{\sailRISCVvalrFOrXS}{}%
  \ifstrequal{#1}{rX}{\sailRISCVvalrX}{}%
  \ifstrequal{#1}{rX_bits}{\sailRISCVvalrXBits}{}%
  \ifstrequal{#1}{rX\_bits}{\sailRISCVvalrXBits}{}%
  \ifstrequal{#1}{readCSR}{\sailRISCVvalreadCSR}{}%
  \ifstrequal{#1}{read_kind_of_flags}{\sailRISCVvalreadKindOfFlags}{}%
  \ifstrequal{#1}{read\_kind\_of\_flags}{\sailRISCVvalreadKindOfFlags}{}%
  \ifstrequal{#1}{read_kind_of_num}{\sailRISCVvalreadKindOfNum}{}%
  \ifstrequal{#1}{read\_kind\_of\_num}{\sailRISCVvalreadKindOfNum}{}%
  \ifstrequal{#1}{read_ram}{\sailRISCVvalreadRam}{}%
  \ifstrequal{#1}{read\_ram}{\sailRISCVvalreadRam}{}%
  \ifstrequal{#1}{read_seed_csr}{\sailRISCVvalreadSeedCsr}{}%
  \ifstrequal{#1}{read\_seed\_csr}{\sailRISCVvalreadSeedCsr}{}%
  \ifstrequal{#1}{reg_deref}{\sailRISCVvalregDeref}{}%
  \ifstrequal{#1}{reg\_deref}{\sailRISCVvalregDeref}{}%
  \ifstrequal{#1}{reg_name}{\sailRISCVvalregName}{}%
  \ifstrequal{#1}{reg\_name}{\sailRISCVvalregName}{}%
  \ifstrequal{#1}{reg_name_abi}{\sailRISCVvalregNameAbi}{}%
  \ifstrequal{#1}{reg\_name\_abi}{\sailRISCVvalregNameAbi}{}%
  \ifstrequal{#1}{regidx_to_regno}{\sailRISCVvalregidxToRegno}{}%
  \ifstrequal{#1}{regidx\_to\_regno}{\sailRISCVvalregidxToRegno}{}%
  \ifstrequal{#1}{regval_from_reg}{\sailRISCVvalregvalFromReg}{}%
  \ifstrequal{#1}{regval\_from\_reg}{\sailRISCVvalregvalFromReg}{}%
  \ifstrequal{#1}{regval_into_reg}{\sailRISCVvalregvalIntoReg}{}%
  \ifstrequal{#1}{regval\_into\_reg}{\sailRISCVvalregvalIntoReg}{}%
  \ifstrequal{#1}{rem_round_zero}{\sailRISCVvalremRoundZero}{}%
  \ifstrequal{#1}{rem\_round\_zero}{\sailRISCVvalremRoundZero}{}%
  \ifstrequal{#1}{replicate_bits}{\sailRISCVvalreplicateBits}{}%
  \ifstrequal{#1}{replicate\_bits}{\sailRISCVvalreplicateBits}{}%
  \ifstrequal{#1}{reset_htif}{\sailRISCVvalresetHtif}{}%
  \ifstrequal{#1}{reset\_htif}{\sailRISCVvalresetHtif}{}%
  \ifstrequal{#1}{retire_instruction}{\sailRISCVvalretireInstruction}{}%
  \ifstrequal{#1}{retire\_instruction}{\sailRISCVvalretireInstruction}{}%
  \ifstrequal{#1}{reverse_bits_in_byte}{\sailRISCVvalreverseBitsInByte}{}%
  \ifstrequal{#1}{reverse\_bits\_in\_byte}{\sailRISCVvalreverseBitsInByte}{}%
  \ifstrequal{#1}{riscv_f16Add}{\sailRISCVvalriscvFOneSixAdd}{}%
  \ifstrequal{#1}{riscv\_f16Add}{\sailRISCVvalriscvFOneSixAdd}{}%
  \ifstrequal{#1}{riscv_f16Div}{\sailRISCVvalriscvFOneSixDiv}{}%
  \ifstrequal{#1}{riscv\_f16Div}{\sailRISCVvalriscvFOneSixDiv}{}%
  \ifstrequal{#1}{riscv_f16Eq}{\sailRISCVvalriscvFOneSixEq}{}%
  \ifstrequal{#1}{riscv\_f16Eq}{\sailRISCVvalriscvFOneSixEq}{}%
  \ifstrequal{#1}{riscv_f16Le}{\sailRISCVvalriscvFOneSixLe}{}%
  \ifstrequal{#1}{riscv\_f16Le}{\sailRISCVvalriscvFOneSixLe}{}%
  \ifstrequal{#1}{riscv_f16Lt}{\sailRISCVvalriscvFOneSixLt}{}%
  \ifstrequal{#1}{riscv\_f16Lt}{\sailRISCVvalriscvFOneSixLt}{}%
  \ifstrequal{#1}{riscv_f16Mul}{\sailRISCVvalriscvFOneSixMul}{}%
  \ifstrequal{#1}{riscv\_f16Mul}{\sailRISCVvalriscvFOneSixMul}{}%
  \ifstrequal{#1}{riscv_f16MulAdd}{\sailRISCVvalriscvFOneSixMulAdd}{}%
  \ifstrequal{#1}{riscv\_f16MulAdd}{\sailRISCVvalriscvFOneSixMulAdd}{}%
  \ifstrequal{#1}{riscv_f16Sqrt}{\sailRISCVvalriscvFOneSixSqrt}{}%
  \ifstrequal{#1}{riscv\_f16Sqrt}{\sailRISCVvalriscvFOneSixSqrt}{}%
  \ifstrequal{#1}{riscv_f16Sub}{\sailRISCVvalriscvFOneSixSub}{}%
  \ifstrequal{#1}{riscv\_f16Sub}{\sailRISCVvalriscvFOneSixSub}{}%
  \ifstrequal{#1}{riscv_f16ToF32}{\sailRISCVvalriscvFOneSixToFThreeTwo}{}%
  \ifstrequal{#1}{riscv\_f16ToF32}{\sailRISCVvalriscvFOneSixToFThreeTwo}{}%
  \ifstrequal{#1}{riscv_f16ToF64}{\sailRISCVvalriscvFOneSixToFSixFour}{}%
  \ifstrequal{#1}{riscv\_f16ToF64}{\sailRISCVvalriscvFOneSixToFSixFour}{}%
  \ifstrequal{#1}{riscv_f16ToI32}{\sailRISCVvalriscvFOneSixToIThreeTwo}{}%
  \ifstrequal{#1}{riscv\_f16ToI32}{\sailRISCVvalriscvFOneSixToIThreeTwo}{}%
  \ifstrequal{#1}{riscv_f16ToI64}{\sailRISCVvalriscvFOneSixToISixFour}{}%
  \ifstrequal{#1}{riscv\_f16ToI64}{\sailRISCVvalriscvFOneSixToISixFour}{}%
  \ifstrequal{#1}{riscv_f16ToUi32}{\sailRISCVvalriscvFOneSixToUiThreeTwo}{}%
  \ifstrequal{#1}{riscv\_f16ToUi32}{\sailRISCVvalriscvFOneSixToUiThreeTwo}{}%
  \ifstrequal{#1}{riscv_f16ToUi64}{\sailRISCVvalriscvFOneSixToUiSixFour}{}%
  \ifstrequal{#1}{riscv\_f16ToUi64}{\sailRISCVvalriscvFOneSixToUiSixFour}{}%
  \ifstrequal{#1}{riscv_f32Add}{\sailRISCVvalriscvFThreeTwoAdd}{}%
  \ifstrequal{#1}{riscv\_f32Add}{\sailRISCVvalriscvFThreeTwoAdd}{}%
  \ifstrequal{#1}{riscv_f32Div}{\sailRISCVvalriscvFThreeTwoDiv}{}%
  \ifstrequal{#1}{riscv\_f32Div}{\sailRISCVvalriscvFThreeTwoDiv}{}%
  \ifstrequal{#1}{riscv_f32Eq}{\sailRISCVvalriscvFThreeTwoEq}{}%
  \ifstrequal{#1}{riscv\_f32Eq}{\sailRISCVvalriscvFThreeTwoEq}{}%
  \ifstrequal{#1}{riscv_f32Le}{\sailRISCVvalriscvFThreeTwoLe}{}%
  \ifstrequal{#1}{riscv\_f32Le}{\sailRISCVvalriscvFThreeTwoLe}{}%
  \ifstrequal{#1}{riscv_f32Lt}{\sailRISCVvalriscvFThreeTwoLt}{}%
  \ifstrequal{#1}{riscv\_f32Lt}{\sailRISCVvalriscvFThreeTwoLt}{}%
  \ifstrequal{#1}{riscv_f32Mul}{\sailRISCVvalriscvFThreeTwoMul}{}%
  \ifstrequal{#1}{riscv\_f32Mul}{\sailRISCVvalriscvFThreeTwoMul}{}%
  \ifstrequal{#1}{riscv_f32MulAdd}{\sailRISCVvalriscvFThreeTwoMulAdd}{}%
  \ifstrequal{#1}{riscv\_f32MulAdd}{\sailRISCVvalriscvFThreeTwoMulAdd}{}%
  \ifstrequal{#1}{riscv_f32Sqrt}{\sailRISCVvalriscvFThreeTwoSqrt}{}%
  \ifstrequal{#1}{riscv\_f32Sqrt}{\sailRISCVvalriscvFThreeTwoSqrt}{}%
  \ifstrequal{#1}{riscv_f32Sub}{\sailRISCVvalriscvFThreeTwoSub}{}%
  \ifstrequal{#1}{riscv\_f32Sub}{\sailRISCVvalriscvFThreeTwoSub}{}%
  \ifstrequal{#1}{riscv_f32ToF16}{\sailRISCVvalriscvFThreeTwoToFOneSix}{}%
  \ifstrequal{#1}{riscv\_f32ToF16}{\sailRISCVvalriscvFThreeTwoToFOneSix}{}%
  \ifstrequal{#1}{riscv_f32ToF64}{\sailRISCVvalriscvFThreeTwoToFSixFour}{}%
  \ifstrequal{#1}{riscv\_f32ToF64}{\sailRISCVvalriscvFThreeTwoToFSixFour}{}%
  \ifstrequal{#1}{riscv_f32ToI32}{\sailRISCVvalriscvFThreeTwoToIThreeTwo}{}%
  \ifstrequal{#1}{riscv\_f32ToI32}{\sailRISCVvalriscvFThreeTwoToIThreeTwo}{}%
  \ifstrequal{#1}{riscv_f32ToI64}{\sailRISCVvalriscvFThreeTwoToISixFour}{}%
  \ifstrequal{#1}{riscv\_f32ToI64}{\sailRISCVvalriscvFThreeTwoToISixFour}{}%
  \ifstrequal{#1}{riscv_f32ToUi32}{\sailRISCVvalriscvFThreeTwoToUiThreeTwo}{}%
  \ifstrequal{#1}{riscv\_f32ToUi32}{\sailRISCVvalriscvFThreeTwoToUiThreeTwo}{}%
  \ifstrequal{#1}{riscv_f32ToUi64}{\sailRISCVvalriscvFThreeTwoToUiSixFour}{}%
  \ifstrequal{#1}{riscv\_f32ToUi64}{\sailRISCVvalriscvFThreeTwoToUiSixFour}{}%
  \ifstrequal{#1}{riscv_f64Add}{\sailRISCVvalriscvFSixFourAdd}{}%
  \ifstrequal{#1}{riscv\_f64Add}{\sailRISCVvalriscvFSixFourAdd}{}%
  \ifstrequal{#1}{riscv_f64Div}{\sailRISCVvalriscvFSixFourDiv}{}%
  \ifstrequal{#1}{riscv\_f64Div}{\sailRISCVvalriscvFSixFourDiv}{}%
  \ifstrequal{#1}{riscv_f64Eq}{\sailRISCVvalriscvFSixFourEq}{}%
  \ifstrequal{#1}{riscv\_f64Eq}{\sailRISCVvalriscvFSixFourEq}{}%
  \ifstrequal{#1}{riscv_f64Le}{\sailRISCVvalriscvFSixFourLe}{}%
  \ifstrequal{#1}{riscv\_f64Le}{\sailRISCVvalriscvFSixFourLe}{}%
  \ifstrequal{#1}{riscv_f64Lt}{\sailRISCVvalriscvFSixFourLt}{}%
  \ifstrequal{#1}{riscv\_f64Lt}{\sailRISCVvalriscvFSixFourLt}{}%
  \ifstrequal{#1}{riscv_f64Mul}{\sailRISCVvalriscvFSixFourMul}{}%
  \ifstrequal{#1}{riscv\_f64Mul}{\sailRISCVvalriscvFSixFourMul}{}%
  \ifstrequal{#1}{riscv_f64MulAdd}{\sailRISCVvalriscvFSixFourMulAdd}{}%
  \ifstrequal{#1}{riscv\_f64MulAdd}{\sailRISCVvalriscvFSixFourMulAdd}{}%
  \ifstrequal{#1}{riscv_f64Sqrt}{\sailRISCVvalriscvFSixFourSqrt}{}%
  \ifstrequal{#1}{riscv\_f64Sqrt}{\sailRISCVvalriscvFSixFourSqrt}{}%
  \ifstrequal{#1}{riscv_f64Sub}{\sailRISCVvalriscvFSixFourSub}{}%
  \ifstrequal{#1}{riscv\_f64Sub}{\sailRISCVvalriscvFSixFourSub}{}%
  \ifstrequal{#1}{riscv_f64ToF16}{\sailRISCVvalriscvFSixFourToFOneSix}{}%
  \ifstrequal{#1}{riscv\_f64ToF16}{\sailRISCVvalriscvFSixFourToFOneSix}{}%
  \ifstrequal{#1}{riscv_f64ToF32}{\sailRISCVvalriscvFSixFourToFThreeTwo}{}%
  \ifstrequal{#1}{riscv\_f64ToF32}{\sailRISCVvalriscvFSixFourToFThreeTwo}{}%
  \ifstrequal{#1}{riscv_f64ToI32}{\sailRISCVvalriscvFSixFourToIThreeTwo}{}%
  \ifstrequal{#1}{riscv\_f64ToI32}{\sailRISCVvalriscvFSixFourToIThreeTwo}{}%
  \ifstrequal{#1}{riscv_f64ToI64}{\sailRISCVvalriscvFSixFourToISixFour}{}%
  \ifstrequal{#1}{riscv\_f64ToI64}{\sailRISCVvalriscvFSixFourToISixFour}{}%
  \ifstrequal{#1}{riscv_f64ToUi32}{\sailRISCVvalriscvFSixFourToUiThreeTwo}{}%
  \ifstrequal{#1}{riscv\_f64ToUi32}{\sailRISCVvalriscvFSixFourToUiThreeTwo}{}%
  \ifstrequal{#1}{riscv_f64ToUi64}{\sailRISCVvalriscvFSixFourToUiSixFour}{}%
  \ifstrequal{#1}{riscv\_f64ToUi64}{\sailRISCVvalriscvFSixFourToUiSixFour}{}%
  \ifstrequal{#1}{riscv_i32ToF16}{\sailRISCVvalriscvIThreeTwoToFOneSix}{}%
  \ifstrequal{#1}{riscv\_i32ToF16}{\sailRISCVvalriscvIThreeTwoToFOneSix}{}%
  \ifstrequal{#1}{riscv_i32ToF32}{\sailRISCVvalriscvIThreeTwoToFThreeTwo}{}%
  \ifstrequal{#1}{riscv\_i32ToF32}{\sailRISCVvalriscvIThreeTwoToFThreeTwo}{}%
  \ifstrequal{#1}{riscv_i32ToF64}{\sailRISCVvalriscvIThreeTwoToFSixFour}{}%
  \ifstrequal{#1}{riscv\_i32ToF64}{\sailRISCVvalriscvIThreeTwoToFSixFour}{}%
  \ifstrequal{#1}{riscv_i64ToF16}{\sailRISCVvalriscvISixFourToFOneSix}{}%
  \ifstrequal{#1}{riscv\_i64ToF16}{\sailRISCVvalriscvISixFourToFOneSix}{}%
  \ifstrequal{#1}{riscv_i64ToF32}{\sailRISCVvalriscvISixFourToFThreeTwo}{}%
  \ifstrequal{#1}{riscv\_i64ToF32}{\sailRISCVvalriscvISixFourToFThreeTwo}{}%
  \ifstrequal{#1}{riscv_i64ToF64}{\sailRISCVvalriscvISixFourToFSixFour}{}%
  \ifstrequal{#1}{riscv\_i64ToF64}{\sailRISCVvalriscvISixFourToFSixFour}{}%
  \ifstrequal{#1}{riscv_ui32ToF16}{\sailRISCVvalriscvUiThreeTwoToFOneSix}{}%
  \ifstrequal{#1}{riscv\_ui32ToF16}{\sailRISCVvalriscvUiThreeTwoToFOneSix}{}%
  \ifstrequal{#1}{riscv_ui32ToF32}{\sailRISCVvalriscvUiThreeTwoToFThreeTwo}{}%
  \ifstrequal{#1}{riscv\_ui32ToF32}{\sailRISCVvalriscvUiThreeTwoToFThreeTwo}{}%
  \ifstrequal{#1}{riscv_ui32ToF64}{\sailRISCVvalriscvUiThreeTwoToFSixFour}{}%
  \ifstrequal{#1}{riscv\_ui32ToF64}{\sailRISCVvalriscvUiThreeTwoToFSixFour}{}%
  \ifstrequal{#1}{riscv_ui64ToF16}{\sailRISCVvalriscvUiSixFourToFOneSix}{}%
  \ifstrequal{#1}{riscv\_ui64ToF16}{\sailRISCVvalriscvUiSixFourToFOneSix}{}%
  \ifstrequal{#1}{riscv_ui64ToF32}{\sailRISCVvalriscvUiSixFourToFThreeTwo}{}%
  \ifstrequal{#1}{riscv\_ui64ToF32}{\sailRISCVvalriscvUiSixFourToFThreeTwo}{}%
  \ifstrequal{#1}{riscv_ui64ToF64}{\sailRISCVvalriscvUiSixFourToFSixFour}{}%
  \ifstrequal{#1}{riscv\_ui64ToF64}{\sailRISCVvalriscvUiSixFourToFSixFour}{}%
  \ifstrequal{#1}{rop_of_num}{\sailRISCVvalropOfNum}{}%
  \ifstrequal{#1}{rop\_of\_num}{\sailRISCVvalropOfNum}{}%
  \ifstrequal{#1}{ropw_of_num}{\sailRISCVvalropwOfNum}{}%
  \ifstrequal{#1}{ropw\_of\_num}{\sailRISCVvalropwOfNum}{}%
  \ifstrequal{#1}{rotate_bits_left}{\sailRISCVvalrotateBitsLeft}{}%
  \ifstrequal{#1}{rotate\_bits\_left}{\sailRISCVvalrotateBitsLeft}{}%
  \ifstrequal{#1}{rotate_bits_right}{\sailRISCVvalrotateBitsRight}{}%
  \ifstrequal{#1}{rotate\_bits\_right}{\sailRISCVvalrotateBitsRight}{}%
  \ifstrequal{#1}{rotatel}{\sailRISCVvalrotatel}{}%
  \ifstrequal{#1}{rotater}{\sailRISCVvalrotater}{}%
  \ifstrequal{#1}{rounding_mode_of_num}{\sailRISCVvalroundingModeOfNum}{}%
  \ifstrequal{#1}{rounding\_mode\_of\_num}{\sailRISCVvalroundingModeOfNum}{}%
  \ifstrequal{#1}{rtype_mnemonic}{\sailRISCVvalrtypeMnemonic}{}%
  \ifstrequal{#1}{rtype\_mnemonic}{\sailRISCVvalrtypeMnemonic}{}%
  \ifstrequal{#1}{rtypew_mnemonic}{\sailRISCVvalrtypewMnemonic}{}%
  \ifstrequal{#1}{rtypew\_mnemonic}{\sailRISCVvalrtypewMnemonic}{}%
  \ifstrequal{#1}{rvfi_read}{\sailRISCVvalrvfiRead}{}%
  \ifstrequal{#1}{rvfi\_read}{\sailRISCVvalrvfiRead}{}%
  \ifstrequal{#1}{rvfi_trap}{\sailRISCVvalrvfiTrap}{}%
  \ifstrequal{#1}{rvfi\_trap}{\sailRISCVvalrvfiTrap}{}%
  \ifstrequal{#1}{rvfi_wX}{\sailRISCVvalrvfiWX}{}%
  \ifstrequal{#1}{rvfi\_wX}{\sailRISCVvalrvfiWX}{}%
  \ifstrequal{#1}{rvfi_write}{\sailRISCVvalrvfiWrite}{}%
  \ifstrequal{#1}{rvfi\_write}{\sailRISCVvalrvfiWrite}{}%
  \ifstrequal{#1}{sail_arith_shiftright}{\sailRISCVvalsailArithShiftright}{}%
  \ifstrequal{#1}{sail\_arith\_shiftright}{\sailRISCVvalsailArithShiftright}{}%
  \ifstrequal{#1}{sail_mask}{\sailRISCVvalsailMask}{}%
  \ifstrequal{#1}{sail\_mask}{\sailRISCVvalsailMask}{}%
  \ifstrequal{#1}{sail_ones}{\sailRISCVvalsailOnes}{}%
  \ifstrequal{#1}{sail\_ones}{\sailRISCVvalsailOnes}{}%
  \ifstrequal{#1}{sail_shiftleft}{\sailRISCVvalsailShiftleft}{}%
  \ifstrequal{#1}{sail\_shiftleft}{\sailRISCVvalsailShiftleft}{}%
  \ifstrequal{#1}{sail_shiftright}{\sailRISCVvalsailShiftright}{}%
  \ifstrequal{#1}{sail\_shiftright}{\sailRISCVvalsailShiftright}{}%
  \ifstrequal{#1}{sail_sign_extend}{\sailRISCVvalsailSignExtend}{}%
  \ifstrequal{#1}{sail\_sign\_extend}{\sailRISCVvalsailSignExtend}{}%
  \ifstrequal{#1}{sail_zero_extend}{\sailRISCVvalsailZeroExtend}{}%
  \ifstrequal{#1}{sail\_zero\_extend}{\sailRISCVvalsailZeroExtend}{}%
  \ifstrequal{#1}{sail_zeros}{\sailRISCVvalsailZeros}{}%
  \ifstrequal{#1}{sail\_zeros}{\sailRISCVvalsailZeros}{}%
  \ifstrequal{#1}{satp64Mode_of_bits}{\sailRISCVvalsatpSixFourModeOfBits}{}%
  \ifstrequal{#1}{satp64Mode\_of\_bits}{\sailRISCVvalsatpSixFourModeOfBits}{}%
  \ifstrequal{#1}{scr_name}{\sailRISCVvalscrName}{}%
  \ifstrequal{#1}{scr\_name}{\sailRISCVvalscrName}{}%
  \ifstrequal{#1}{scr_name_map}{\sailRISCVvalscrNameMap}{}%
  \ifstrequal{#1}{scr\_name\_map}{\sailRISCVvalscrNameMap}{}%
  \ifstrequal{#1}{sealCap}{\sailRISCVvalsealCap}{}%
  \ifstrequal{#1}{seed_opst_of_num}{\sailRISCVvalseedOpstOfNum}{}%
  \ifstrequal{#1}{seed\_opst\_of\_num}{\sailRISCVvalseedOpstOfNum}{}%
  \ifstrequal{#1}{select_instr_or_fcsr_rm}{\sailRISCVvalselectInstrOrFcsrRm}{}%
  \ifstrequal{#1}{select\_instr\_or\_fcsr\_rm}{\sailRISCVvalselectInstrOrFcsrRm}{}%
  \ifstrequal{#1}{sep}{\sailRISCVvalsep}{}%
  \ifstrequal{#1}{setCapAddr}{\sailRISCVvalsetCapAddr}{}%
  \ifstrequal{#1}{setCapBounds}{\sailRISCVvalsetCapBounds}{}%
  \ifstrequal{#1}{setCapFlags}{\sailRISCVvalsetCapFlags}{}%
  \ifstrequal{#1}{setCapOffset}{\sailRISCVvalsetCapOffset}{}%
  \ifstrequal{#1}{setCapOffsetChecked}{\sailRISCVvalsetCapOffsetChecked}{}%
  \ifstrequal{#1}{setCapPerms}{\sailRISCVvalsetCapPerms}{}%
  \ifstrequal{#1}{set_mstatus_SXL}{\sailRISCVvalsetMstatusSXL}{}%
  \ifstrequal{#1}{set\_mstatus\_SXL}{\sailRISCVvalsetMstatusSXL}{}%
  \ifstrequal{#1}{set_mstatus_UXL}{\sailRISCVvalsetMstatusUXL}{}%
  \ifstrequal{#1}{set\_mstatus\_UXL}{\sailRISCVvalsetMstatusUXL}{}%
  \ifstrequal{#1}{set_mtvec}{\sailRISCVvalsetMtvec}{}%
  \ifstrequal{#1}{set\_mtvec}{\sailRISCVvalsetMtvec}{}%
  \ifstrequal{#1}{set_next_pc}{\sailRISCVvalsetNextPc}{}%
  \ifstrequal{#1}{set\_next\_pc}{\sailRISCVvalsetNextPc}{}%
  \ifstrequal{#1}{set_slice_bits}{\sailRISCVvalsetSliceBits}{}%
  \ifstrequal{#1}{set\_slice\_bits}{\sailRISCVvalsetSliceBits}{}%
  \ifstrequal{#1}{set_slice_int}{\sailRISCVvalsetSliceInt}{}%
  \ifstrequal{#1}{set\_slice\_int}{\sailRISCVvalsetSliceInt}{}%
  \ifstrequal{#1}{set_sstatus_UXL}{\sailRISCVvalsetSstatusUXL}{}%
  \ifstrequal{#1}{set\_sstatus\_UXL}{\sailRISCVvalsetSstatusUXL}{}%
  \ifstrequal{#1}{set_stvec}{\sailRISCVvalsetStvec}{}%
  \ifstrequal{#1}{set\_stvec}{\sailRISCVvalsetStvec}{}%
  \ifstrequal{#1}{set_utvec}{\sailRISCVvalsetUtvec}{}%
  \ifstrequal{#1}{set\_utvec}{\sailRISCVvalsetUtvec}{}%
  \ifstrequal{#1}{set_xret_target}{\sailRISCVvalsetXretTarget}{}%
  \ifstrequal{#1}{set\_xret\_target}{\sailRISCVvalsetXretTarget}{}%
  \ifstrequal{#1}{shift_bits_left}{\sailRISCVvalshiftBitsLeft}{}%
  \ifstrequal{#1}{shift\_bits\_left}{\sailRISCVvalshiftBitsLeft}{}%
  \ifstrequal{#1}{shift_bits_right}{\sailRISCVvalshiftBitsRight}{}%
  \ifstrequal{#1}{shift\_bits\_right}{\sailRISCVvalshiftBitsRight}{}%
  \ifstrequal{#1}{shift_right_arith32}{\sailRISCVvalshiftRightArithThreeTwo}{}%
  \ifstrequal{#1}{shift\_right\_arith32}{\sailRISCVvalshiftRightArithThreeTwo}{}%
  \ifstrequal{#1}{shift_right_arith64}{\sailRISCVvalshiftRightArithSixFour}{}%
  \ifstrequal{#1}{shift\_right\_arith64}{\sailRISCVvalshiftRightArithSixFour}{}%
  \ifstrequal{#1}{shiftiop_mnemonic}{\sailRISCVvalshiftiopMnemonic}{}%
  \ifstrequal{#1}{shiftiop\_mnemonic}{\sailRISCVvalshiftiopMnemonic}{}%
  \ifstrequal{#1}{shiftiwop_mnemonic}{\sailRISCVvalshiftiwopMnemonic}{}%
  \ifstrequal{#1}{shiftiwop\_mnemonic}{\sailRISCVvalshiftiwopMnemonic}{}%
  \ifstrequal{#1}{shiftl}{\sailRISCVvalshiftl}{}%
  \ifstrequal{#1}{shiftr}{\sailRISCVvalshiftr}{}%
  \ifstrequal{#1}{shiftw_mnemonic}{\sailRISCVvalshiftwMnemonic}{}%
  \ifstrequal{#1}{shiftw\_mnemonic}{\sailRISCVvalshiftwMnemonic}{}%
  \ifstrequal{#1}{signed}{\sailRISCVvalsigned}{}%
  \ifstrequal{#1}{size_bits}{\sailRISCVvalsizzeBits}{}%
  \ifstrequal{#1}{size\_bits}{\sailRISCVvalsizzeBits}{}%
  \ifstrequal{#1}{size_mnemonic}{\sailRISCVvalsizzeMnemonic}{}%
  \ifstrequal{#1}{size\_mnemonic}{\sailRISCVvalsizzeMnemonic}{}%
  \ifstrequal{#1}{slice}{\sailRISCVvalslice}{}%
  \ifstrequal{#1}{slice_mask}{\sailRISCVvalsliceMask}{}%
  \ifstrequal{#1}{slice\_mask}{\sailRISCVvalsliceMask}{}%
  \ifstrequal{#1}{sop_of_num}{\sailRISCVvalsopOfNum}{}%
  \ifstrequal{#1}{sop\_of\_num}{\sailRISCVvalsopOfNum}{}%
  \ifstrequal{#1}{sopw_of_num}{\sailRISCVvalsopwOfNum}{}%
  \ifstrequal{#1}{sopw\_of\_num}{\sailRISCVvalsopwOfNum}{}%
  \ifstrequal{#1}{spc}{\sailRISCVvalspc}{}%
  \ifstrequal{#1}{spc_backwards}{\sailRISCVvalspcBackwards}{}%
  \ifstrequal{#1}{spc\_backwards}{\sailRISCVvalspcBackwards}{}%
  \ifstrequal{#1}{spc_forwards}{\sailRISCVvalspcForwards}{}%
  \ifstrequal{#1}{spc\_forwards}{\sailRISCVvalspcForwards}{}%
  \ifstrequal{#1}{spc_matches_prefix}{\sailRISCVvalspcMatchesPrefix}{}%
  \ifstrequal{#1}{spc\_matches\_prefix}{\sailRISCVvalspcMatchesPrefix}{}%
  \ifstrequal{#1}{speculate_conditional}{\sailRISCVvalspeculateConditional}{}%
  \ifstrequal{#1}{speculate\_conditional}{\sailRISCVvalspeculateConditional}{}%
  \ifstrequal{#1}{step}{\sailRISCVvalstep}{}%
  \ifstrequal{#1}{string_append}{\sailRISCVvalstringAppend}{}%
  \ifstrequal{#1}{string\_append}{\sailRISCVvalstringAppend}{}%
  \ifstrequal{#1}{string_drop}{\sailRISCVvalstringDrop}{}%
  \ifstrequal{#1}{string\_drop}{\sailRISCVvalstringDrop}{}%
  \ifstrequal{#1}{string_length}{\sailRISCVvalstringLength}{}%
  \ifstrequal{#1}{string\_length}{\sailRISCVvalstringLength}{}%
  \ifstrequal{#1}{string_of_bit}{\sailRISCVvalstringOfBit}{}%
  \ifstrequal{#1}{string\_of\_bit}{\sailRISCVvalstringOfBit}{}%
  \ifstrequal{#1}{string_of_bits}{\sailRISCVvalstringOfBits}{}%
  \ifstrequal{#1}{string\_of\_bits}{\sailRISCVvalstringOfBits}{}%
  \ifstrequal{#1}{string_of_capex}{\sailRISCVvalstringOfCapex}{}%
  \ifstrequal{#1}{string\_of\_capex}{\sailRISCVvalstringOfCapex}{}%
  \ifstrequal{#1}{string_of_int}{\sailRISCVvalstringOfInt}{}%
  \ifstrequal{#1}{string\_of\_int}{\sailRISCVvalstringOfInt}{}%
  \ifstrequal{#1}{string_startswith}{\sailRISCVvalstringStartswith}{}%
  \ifstrequal{#1}{string\_startswith}{\sailRISCVvalstringStartswith}{}%
  \ifstrequal{#1}{string_take}{\sailRISCVvalstringTake}{}%
  \ifstrequal{#1}{string\_take}{\sailRISCVvalstringTake}{}%
  \ifstrequal{#1}{sub_atom}{\sailRISCVvalsubAtom}{}%
  \ifstrequal{#1}{sub\_atom}{\sailRISCVvalsubAtom}{}%
  \ifstrequal{#1}{sub_bits}{\sailRISCVvalsubBits}{}%
  \ifstrequal{#1}{sub\_bits}{\sailRISCVvalsubBits}{}%
  \ifstrequal{#1}{sub_int}{\sailRISCVvalsubInt}{}%
  \ifstrequal{#1}{sub\_int}{\sailRISCVvalsubInt}{}%
  \ifstrequal{#1}{sub_nat}{\sailRISCVvalsubNat}{}%
  \ifstrequal{#1}{sub\_nat}{\sailRISCVvalsubNat}{}%
  \ifstrequal{#1}{sub_vec}{\sailRISCVvalsubVec}{}%
  \ifstrequal{#1}{sub\_vec}{\sailRISCVvalsubVec}{}%
  \ifstrequal{#1}{sub_vec_int}{\sailRISCVvalsubVecInt}{}%
  \ifstrequal{#1}{sub\_vec\_int}{\sailRISCVvalsubVecInt}{}%
  \ifstrequal{#1}{subrange_bits}{\sailRISCVvalsubrangeBits}{}%
  \ifstrequal{#1}{subrange\_bits}{\sailRISCVvalsubrangeBits}{}%
  \ifstrequal{#1}{sys_enable_fdext}{\sailRISCVvalsysEnableFdext}{}%
  \ifstrequal{#1}{sys\_enable\_fdext}{\sailRISCVvalsysEnableFdext}{}%
  \ifstrequal{#1}{sys_enable_next}{\sailRISCVvalsysEnableNext}{}%
  \ifstrequal{#1}{sys\_enable\_next}{\sailRISCVvalsysEnableNext}{}%
  \ifstrequal{#1}{sys_enable_rvc}{\sailRISCVvalsysEnableRvc}{}%
  \ifstrequal{#1}{sys\_enable\_rvc}{\sailRISCVvalsysEnableRvc}{}%
  \ifstrequal{#1}{sys_enable_writable_misa}{\sailRISCVvalsysEnableWritableMisa}{}%
  \ifstrequal{#1}{sys\_enable\_writable\_misa}{\sailRISCVvalsysEnableWritableMisa}{}%
  \ifstrequal{#1}{sys_enable_zfinx}{\sailRISCVvalsysEnableZfinx}{}%
  \ifstrequal{#1}{sys\_enable\_zfinx}{\sailRISCVvalsysEnableZfinx}{}%
  \ifstrequal{#1}{tag_addr_to_addr}{\sailRISCVvaltagAddrToAddr}{}%
  \ifstrequal{#1}{tag\_addr\_to\_addr}{\sailRISCVvaltagAddrToAddr}{}%
  \ifstrequal{#1}{tdiv_int}{\sailRISCVvaltdivInt}{}%
  \ifstrequal{#1}{tdiv\_int}{\sailRISCVvaltdivInt}{}%
  \ifstrequal{#1}{tick_clock}{\sailRISCVvaltickClock}{}%
  \ifstrequal{#1}{tick\_clock}{\sailRISCVvaltickClock}{}%
  \ifstrequal{#1}{tick_pc}{\sailRISCVvaltickPc}{}%
  \ifstrequal{#1}{tick\_pc}{\sailRISCVvaltickPc}{}%
  \ifstrequal{#1}{tick_platform}{\sailRISCVvaltickPlatform}{}%
  \ifstrequal{#1}{tick\_platform}{\sailRISCVvaltickPlatform}{}%
  \ifstrequal{#1}{to_bits}{\sailRISCVvaltoBits}{}%
  \ifstrequal{#1}{to\_bits}{\sailRISCVvaltoBits}{}%
  \ifstrequal{#1}{trans_kind_of_num}{\sailRISCVvaltransKindOfNum}{}%
  \ifstrequal{#1}{trans\_kind\_of\_num}{\sailRISCVvaltransKindOfNum}{}%
  \ifstrequal{#1}{translate39}{\sailRISCVvaltranslateThreeNine}{}%
  \ifstrequal{#1}{translate48}{\sailRISCVvaltranslateFourEight}{}%
  \ifstrequal{#1}{translateAddr}{\sailRISCVvaltranslateAddr}{}%
  \ifstrequal{#1}{translateAddr_priv}{\sailRISCVvaltranslateAddrPriv}{}%
  \ifstrequal{#1}{translateAddr\_priv}{\sailRISCVvaltranslateAddrPriv}{}%
  \ifstrequal{#1}{translationException}{\sailRISCVvaltranslationException}{}%
  \ifstrequal{#1}{translationMode}{\sailRISCVvaltranslationMode}{}%
  \ifstrequal{#1}{trapVectorMode_of_bits}{\sailRISCVvaltrapVectorModeOfBits}{}%
  \ifstrequal{#1}{trapVectorMode\_of\_bits}{\sailRISCVvaltrapVectorModeOfBits}{}%
  \ifstrequal{#1}{trap_handler}{\sailRISCVvaltrapHandler}{}%
  \ifstrequal{#1}{trap\_handler}{\sailRISCVvaltrapHandler}{}%
  \ifstrequal{#1}{truncate}{\sailRISCVvaltruncate}{}%
  \ifstrequal{#1}{truncateLSB}{\sailRISCVvaltruncateLSB}{}%
  \ifstrequal{#1}{tval}{\sailRISCVvaltval}{}%
  \ifstrequal{#1}{tvec_addr}{\sailRISCVvaltvecAddr}{}%
  \ifstrequal{#1}{tvec\_addr}{\sailRISCVvaltvecAddr}{}%
  \ifstrequal{#1}{ufFlag}{\sailRISCVvalufFlag}{}%
  \ifstrequal{#1}{unsealCap}{\sailRISCVvalunsealCap}{}%
  \ifstrequal{#1}{unsigned}{\sailRISCVvalunsigned}{}%
  \ifstrequal{#1}{uop_of_num}{\sailRISCVvaluopOfNum}{}%
  \ifstrequal{#1}{uop\_of\_num}{\sailRISCVvaluopOfNum}{}%
  \ifstrequal{#1}{update_PTE_Bits}{\sailRISCVvalupdatePTEBits}{}%
  \ifstrequal{#1}{update\_PTE\_Bits}{\sailRISCVvalupdatePTEBits}{}%
  \ifstrequal{#1}{update_softfloat_fflags}{\sailRISCVvalupdateSoftfloatFflags}{}%
  \ifstrequal{#1}{update\_softfloat\_fflags}{\sailRISCVvalupdateSoftfloatFflags}{}%
  \ifstrequal{#1}{update_subrange}{\sailRISCVvalupdateSubrange}{}%
  \ifstrequal{#1}{update\_subrange}{\sailRISCVvalupdateSubrange}{}%
  \ifstrequal{#1}{update_subrange_bits}{\sailRISCVvalupdateSubrangeBits}{}%
  \ifstrequal{#1}{update\_subrange\_bits}{\sailRISCVvalupdateSubrangeBits}{}%
  \ifstrequal{#1}{utype_mnemonic}{\sailRISCVvalutypeMnemonic}{}%
  \ifstrequal{#1}{utype\_mnemonic}{\sailRISCVvalutypeMnemonic}{}%
  \ifstrequal{#1}{validDoubleRegs}{\sailRISCVvalvalidDoubleRegs}{}%
  \ifstrequal{#1}{valid_rounding_mode}{\sailRISCVvalvalidRoundingMode}{}%
  \ifstrequal{#1}{valid\_rounding\_mode}{\sailRISCVvalvalidRoundingMode}{}%
  \ifstrequal{#1}{vector_concat}{\sailRISCVvalvectorConcat}{}%
  \ifstrequal{#1}{vector\_concat}{\sailRISCVvalvectorConcat}{}%
  \ifstrequal{#1}{vector_length}{\sailRISCVvalvectorLength}{}%
  \ifstrequal{#1}{vector\_length}{\sailRISCVvalvectorLength}{}%
  \ifstrequal{#1}{wC}{\sailRISCVvalwC}{}%
  \ifstrequal{#1}{wC_bits}{\sailRISCVvalwCBits}{}%
  \ifstrequal{#1}{wC\_bits}{\sailRISCVvalwCBits}{}%
  \ifstrequal{#1}{wF}{\sailRISCVvalwF}{}%
  \ifstrequal{#1}{wF_bits}{\sailRISCVvalwFBits}{}%
  \ifstrequal{#1}{wF\_bits}{\sailRISCVvalwFBits}{}%
  \ifstrequal{#1}{wF_or_X_D}{\sailRISCVvalwFOrXD}{}%
  \ifstrequal{#1}{wF\_or\_X\_D}{\sailRISCVvalwFOrXD}{}%
  \ifstrequal{#1}{wF_or_X_H}{\sailRISCVvalwFOrXH}{}%
  \ifstrequal{#1}{wF\_or\_X\_H}{\sailRISCVvalwFOrXH}{}%
  \ifstrequal{#1}{wF_or_X_S}{\sailRISCVvalwFOrXS}{}%
  \ifstrequal{#1}{wF\_or\_X\_S}{\sailRISCVvalwFOrXS}{}%
  \ifstrequal{#1}{wX}{\sailRISCVvalwX}{}%
  \ifstrequal{#1}{wX_bits}{\sailRISCVvalwXBits}{}%
  \ifstrequal{#1}{wX\_bits}{\sailRISCVvalwXBits}{}%
  \ifstrequal{#1}{walk39}{\sailRISCVvalwalkThreeNine}{}%
  \ifstrequal{#1}{walk48}{\sailRISCVvalwalkFourEight}{}%
  \ifstrequal{#1}{within_clint}{\sailRISCVvalwithinClint}{}%
  \ifstrequal{#1}{within\_clint}{\sailRISCVvalwithinClint}{}%
  \ifstrequal{#1}{within_htif_readable}{\sailRISCVvalwithinHtifReadable}{}%
  \ifstrequal{#1}{within\_htif\_readable}{\sailRISCVvalwithinHtifReadable}{}%
  \ifstrequal{#1}{within_htif_writable}{\sailRISCVvalwithinHtifWritable}{}%
  \ifstrequal{#1}{within\_htif\_writable}{\sailRISCVvalwithinHtifWritable}{}%
  \ifstrequal{#1}{within_mmio_readable}{\sailRISCVvalwithinMmioReadable}{}%
  \ifstrequal{#1}{within\_mmio\_readable}{\sailRISCVvalwithinMmioReadable}{}%
  \ifstrequal{#1}{within_mmio_writable}{\sailRISCVvalwithinMmioWritable}{}%
  \ifstrequal{#1}{within\_mmio\_writable}{\sailRISCVvalwithinMmioWritable}{}%
  \ifstrequal{#1}{within_phys_mem}{\sailRISCVvalwithinPhysMem}{}%
  \ifstrequal{#1}{within\_phys\_mem}{\sailRISCVvalwithinPhysMem}{}%
  \ifstrequal{#1}{word_width_bytes}{\sailRISCVvalwordWidthBytes}{}%
  \ifstrequal{#1}{word\_width\_bytes}{\sailRISCVvalwordWidthBytes}{}%
  \ifstrequal{#1}{word_width_of_num}{\sailRISCVvalwordWidthOfNum}{}%
  \ifstrequal{#1}{word\_width\_of\_num}{\sailRISCVvalwordWidthOfNum}{}%
  \ifstrequal{#1}{writeCSR}{\sailRISCVvalwriteCSR}{}%
  \ifstrequal{#1}{write_TLB39}{\sailRISCVvalwriteTLBThreeNine}{}%
  \ifstrequal{#1}{write\_TLB39}{\sailRISCVvalwriteTLBThreeNine}{}%
  \ifstrequal{#1}{write_TLB48}{\sailRISCVvalwriteTLBFourEight}{}%
  \ifstrequal{#1}{write\_TLB48}{\sailRISCVvalwriteTLBFourEight}{}%
  \ifstrequal{#1}{write_fflags}{\sailRISCVvalwriteFflags}{}%
  \ifstrequal{#1}{write\_fflags}{\sailRISCVvalwriteFflags}{}%
  \ifstrequal{#1}{write_kind_of_num}{\sailRISCVvalwriteKindOfNum}{}%
  \ifstrequal{#1}{write\_kind\_of\_num}{\sailRISCVvalwriteKindOfNum}{}%
  \ifstrequal{#1}{write_ram}{\sailRISCVvalwriteRam}{}%
  \ifstrequal{#1}{write\_ram}{\sailRISCVvalwriteRam}{}%
  \ifstrequal{#1}{write_ram_ea}{\sailRISCVvalwriteRamEa}{}%
  \ifstrequal{#1}{write\_ram\_ea}{\sailRISCVvalwriteRamEa}{}%
  \ifstrequal{#1}{write_sc_cap_result}{\sailRISCVvalwriteScCapResult}{}%
  \ifstrequal{#1}{write\_sc\_cap\_result}{\sailRISCVvalwriteScCapResult}{}%
  \ifstrequal{#1}{write_seed_csr}{\sailRISCVvalwriteSeedCsr}{}%
  \ifstrequal{#1}{write\_seed\_csr}{\sailRISCVvalwriteSeedCsr}{}%
  \ifstrequal{#1}{xor_vec}{\sailRISCVvalxorVec}{}%
  \ifstrequal{#1}{xor\_vec}{\sailRISCVvalxorVec}{}%
  \ifstrequal{#1}{zeros_implicit}{\sailRISCVvalzzerosImplicit}{}%
  \ifstrequal{#1}{zeros\_implicit}{\sailRISCVvalzzerosImplicit}{}%
  \ifstrequal{#1}{(operator <=_u)}{\sailRISCVvalzEightoperatorzZerozIzJUzNine}{}%
  \ifstrequal{#1}{(operator $>$=\_u)}{\sailRISCVvalzEightoperatorzZerozIzJUzNine}{}%
  \ifstrequal{#1}{(operator <_s)}{\sailRISCVvalzEightoperatorzZerozISzNine}{}%
  \ifstrequal{#1}{(operator $>$\_s)}{\sailRISCVvalzEightoperatorzZerozISzNine}{}%
  \ifstrequal{#1}{(operator <_u)}{\sailRISCVvalzEightoperatorzZerozIUzNine}{}%
  \ifstrequal{#1}{(operator $>$\_u)}{\sailRISCVvalzEightoperatorzZerozIUzNine}{}%
  \ifstrequal{#1}{(operator >=_s)}{\sailRISCVvalzEightoperatorzZerozKzJSzNine}{}%
  \ifstrequal{#1}{(operator $$>$$=\_s)}{\sailRISCVvalzEightoperatorzZerozKzJSzNine}{}%
  \ifstrequal{#1}{(operator >=_u)}{\sailRISCVvalzEightoperatorzZerozKzJUzNine}{}%
  \ifstrequal{#1}{(operator $$>$$=\_u)}{\sailRISCVvalzEightoperatorzZerozKzJUzNine}{}}

\newcommand{\sailRISCVrefval}[2]{
  \ifstrequal{#1}{Architecture_of_num}{\hyperref[sailRISCVzArchitecturezyofzynum]{#2}}{}%
  \ifstrequal{#1}{Architecture\_of\_num}{\hyperref[sailRISCVzArchitecturezyofzynum]{#2}}{}%
  \ifstrequal{#1}{CPtrCmpOp_of_num}{\hyperref[sailRISCVzCPtrCmpOpzyofzynum]{#2}}{}%
  \ifstrequal{#1}{CPtrCmpOp\_of\_num}{\hyperref[sailRISCVzCPtrCmpOpzyofzynum]{#2}}{}%
  \ifstrequal{#1}{CapExCode}{\hyperref[sailRISCVzCapExCode]{#2}}{}%
  \ifstrequal{#1}{CapEx_of_num}{\hyperref[sailRISCVzCapExzyofzynum]{#2}}{}%
  \ifstrequal{#1}{CapEx\_of\_num}{\hyperref[sailRISCVzCapExzyofzynum]{#2}}{}%
  \ifstrequal{#1}{ClearRegSet_of_num}{\hyperref[sailRISCVzClearRegSetzyofzynum]{#2}}{}%
  \ifstrequal{#1}{ClearRegSet\_of\_num}{\hyperref[sailRISCVzClearRegSetzyofzynum]{#2}}{}%
  \ifstrequal{#1}{EXTS}{\hyperref[sailRISCVzEXTS]{#2}}{}%
  \ifstrequal{#1}{EXTZ}{\hyperref[sailRISCVzEXTZ]{#2}}{}%
  \ifstrequal{#1}{ExtStatus_of_num}{\hyperref[sailRISCVzExtStatuszyofzynum]{#2}}{}%
  \ifstrequal{#1}{ExtStatus\_of\_num}{\hyperref[sailRISCVzExtStatuszyofzynum]{#2}}{}%
  \ifstrequal{#1}{FRegStr}{\hyperref[sailRISCVzFRegStr]{#2}}{}%
  \ifstrequal{#1}{GPRstr}{\hyperref[sailRISCVzGPRstr]{#2}}{}%
  \ifstrequal{#1}{InterruptType_of_num}{\hyperref[sailRISCVzInterruptTypezyofzynum]{#2}}{}%
  \ifstrequal{#1}{InterruptType\_of\_num}{\hyperref[sailRISCVzInterruptTypezyofzynum]{#2}}{}%
  \ifstrequal{#1}{MAX}{\hyperref[sailRISCVzMAX]{#2}}{}%
  \ifstrequal{#1}{MEMr_tag}{\hyperref[sailRISCVzMEMrzytag]{#2}}{}%
  \ifstrequal{#1}{MEMr\_tag}{\hyperref[sailRISCVzMEMrzytag]{#2}}{}%
  \ifstrequal{#1}{MEMw_tag}{\hyperref[sailRISCVzMEMwzytag]{#2}}{}%
  \ifstrequal{#1}{MEMw\_tag}{\hyperref[sailRISCVzMEMwzytag]{#2}}{}%
  \ifstrequal{#1}{MemoryOpResult_add_meta}{\hyperref[sailRISCVzMemoryOpResultzyaddzymeta]{#2}}{}%
  \ifstrequal{#1}{MemoryOpResult\_add\_meta}{\hyperref[sailRISCVzMemoryOpResultzyaddzymeta]{#2}}{}%
  \ifstrequal{#1}{MemoryOpResult_drop_meta}{\hyperref[sailRISCVzMemoryOpResultzydropzymeta]{#2}}{}%
  \ifstrequal{#1}{MemoryOpResult\_drop\_meta}{\hyperref[sailRISCVzMemoryOpResultzydropzymeta]{#2}}{}%
  \ifstrequal{#1}{PmpAddrMatchType_of_num}{\hyperref[sailRISCVzPmpAddrMatchTypezyofzynum]{#2}}{}%
  \ifstrequal{#1}{PmpAddrMatchType\_of\_num}{\hyperref[sailRISCVzPmpAddrMatchTypezyofzynum]{#2}}{}%
  \ifstrequal{#1}{Privilege_of_num}{\hyperref[sailRISCVzPrivilegezyofzynum]{#2}}{}%
  \ifstrequal{#1}{Privilege\_of\_num}{\hyperref[sailRISCVzPrivilegezyofzynum]{#2}}{}%
  \ifstrequal{#1}{RegStr}{\hyperref[sailRISCVzRegStr]{#2}}{}%
  \ifstrequal{#1}{Retired_of_num}{\hyperref[sailRISCVzRetiredzyofzynum]{#2}}{}%
  \ifstrequal{#1}{Retired\_of\_num}{\hyperref[sailRISCVzRetiredzyofzynum]{#2}}{}%
  \ifstrequal{#1}{SATPMode_of_num}{\hyperref[sailRISCVzSATPModezyofzynum]{#2}}{}%
  \ifstrequal{#1}{SATPMode\_of\_num}{\hyperref[sailRISCVzSATPModezyofzynum]{#2}}{}%
  \ifstrequal{#1}{TrapVectorMode_of_num}{\hyperref[sailRISCVzTrapVectorModezyofzynum]{#2}}{}%
  \ifstrequal{#1}{TrapVectorMode\_of\_num}{\hyperref[sailRISCVzTrapVectorModezyofzynum]{#2}}{}%
  \ifstrequal{#1}{__ReadRAM_Meta}{\hyperref[sailRISCVzzyzyReadRAMzyMeta]{#2}}{}%
  \ifstrequal{#1}{\_\_ReadRAM\_Meta}{\hyperref[sailRISCVzzyzyReadRAMzyMeta]{#2}}{}%
  \ifstrequal{#1}{__TraceMemoryRead}{\hyperref[sailRISCVzzyzyTraceMemoryRead]{#2}}{}%
  \ifstrequal{#1}{\_\_TraceMemoryRead}{\hyperref[sailRISCVzzyzyTraceMemoryRead]{#2}}{}%
  \ifstrequal{#1}{__TraceMemoryWrite}{\hyperref[sailRISCVzzyzyTraceMemoryWrite]{#2}}{}%
  \ifstrequal{#1}{\_\_TraceMemoryWrite}{\hyperref[sailRISCVzzyzyTraceMemoryWrite]{#2}}{}%
  \ifstrequal{#1}{__WriteRAM_Meta}{\hyperref[sailRISCVzzyzyWriteRAMzyMeta]{#2}}{}%
  \ifstrequal{#1}{\_\_WriteRAM\_Meta}{\hyperref[sailRISCVzzyzyWriteRAMzyMeta]{#2}}{}%
  \ifstrequal{#1}{__barrier}{\hyperref[sailRISCVzzyzybarrier]{#2}}{}%
  \ifstrequal{#1}{\_\_barrier}{\hyperref[sailRISCVzzyzybarrier]{#2}}{}%
  \ifstrequal{#1}{__branch_announce}{\hyperref[sailRISCVzzyzybranchzyannounce]{#2}}{}%
  \ifstrequal{#1}{\_\_branch\_announce}{\hyperref[sailRISCVzzyzybranchzyannounce]{#2}}{}%
  \ifstrequal{#1}{__cache_maintenance}{\hyperref[sailRISCVzzyzycachezymaintenance]{#2}}{}%
  \ifstrequal{#1}{\_\_cache\_maintenance}{\hyperref[sailRISCVzzyzycachezymaintenance]{#2}}{}%
  \ifstrequal{#1}{__deref}{\hyperref[sailRISCVzzyzyderef]{#2}}{}%
  \ifstrequal{#1}{\_\_deref}{\hyperref[sailRISCVzzyzyderef]{#2}}{}%
  \ifstrequal{#1}{__excl_res}{\hyperref[sailRISCVzzyzyexclzyres]{#2}}{}%
  \ifstrequal{#1}{\_\_excl\_res}{\hyperref[sailRISCVzzyzyexclzyres]{#2}}{}%
  \ifstrequal{#1}{__id}{\hyperref[sailRISCVzzyzyid]{#2}}{}%
  \ifstrequal{#1}{\_\_id}{\hyperref[sailRISCVzzyzyid]{#2}}{}%
  \ifstrequal{#1}{__instr_announce}{\hyperref[sailRISCVzzyzyinstrzyannounce]{#2}}{}%
  \ifstrequal{#1}{\_\_instr\_announce}{\hyperref[sailRISCVzzyzyinstrzyannounce]{#2}}{}%
  \ifstrequal{#1}{__read_mem}{\hyperref[sailRISCVzzyzyreadzymem]{#2}}{}%
  \ifstrequal{#1}{\_\_read\_mem}{\hyperref[sailRISCVzzyzyreadzymem]{#2}}{}%
  \ifstrequal{#1}{__read_memt}{\hyperref[sailRISCVzzyzyreadzymemt]{#2}}{}%
  \ifstrequal{#1}{\_\_read\_memt}{\hyperref[sailRISCVzzyzyreadzymemt]{#2}}{}%
  \ifstrequal{#1}{__write_mem}{\hyperref[sailRISCVzzyzywritezymem]{#2}}{}%
  \ifstrequal{#1}{\_\_write\_mem}{\hyperref[sailRISCVzzyzywritezymem]{#2}}{}%
  \ifstrequal{#1}{__write_mem_ea}{\hyperref[sailRISCVzzyzywritezymemzyea]{#2}}{}%
  \ifstrequal{#1}{\_\_write\_mem\_ea}{\hyperref[sailRISCVzzyzywritezymemzyea]{#2}}{}%
  \ifstrequal{#1}{__write_memt}{\hyperref[sailRISCVzzyzywritezymemt]{#2}}{}%
  \ifstrequal{#1}{\_\_write\_memt}{\hyperref[sailRISCVzzyzywritezymemt]{#2}}{}%
  \ifstrequal{#1}{__write_tag}{\hyperref[sailRISCVzzyzywritezytag]{#2}}{}%
  \ifstrequal{#1}{\_\_write\_tag}{\hyperref[sailRISCVzzyzywritezytag]{#2}}{}%
  \ifstrequal{#1}{_reg_deref}{\hyperref[sailRISCVzzyregzyderef]{#2}}{}%
  \ifstrequal{#1}{\_reg\_deref}{\hyperref[sailRISCVzzyregzyderef]{#2}}{}%
  \ifstrequal{#1}{_shl1}{\hyperref[sailRISCVzzyshl1]{#2}}{}%
  \ifstrequal{#1}{\_shl1}{\hyperref[sailRISCVzzyshl1]{#2}}{}%
  \ifstrequal{#1}{_shl32}{\hyperref[sailRISCVzzyshl32]{#2}}{}%
  \ifstrequal{#1}{\_shl32}{\hyperref[sailRISCVzzyshl32]{#2}}{}%
  \ifstrequal{#1}{_shl8}{\hyperref[sailRISCVzzyshl8]{#2}}{}%
  \ifstrequal{#1}{\_shl8}{\hyperref[sailRISCVzzyshl8]{#2}}{}%
  \ifstrequal{#1}{_shl_int}{\hyperref[sailRISCVzzyshlzyint]{#2}}{}%
  \ifstrequal{#1}{\_shl\_int}{\hyperref[sailRISCVzzyshlzyint]{#2}}{}%
  \ifstrequal{#1}{_shl_int_general}{\hyperref[sailRISCVzzyshlzyintzygeneral]{#2}}{}%
  \ifstrequal{#1}{\_shl\_int\_general}{\hyperref[sailRISCVzzyshlzyintzygeneral]{#2}}{}%
  \ifstrequal{#1}{_shr32}{\hyperref[sailRISCVzzyshr32]{#2}}{}%
  \ifstrequal{#1}{\_shr32}{\hyperref[sailRISCVzzyshr32]{#2}}{}%
  \ifstrequal{#1}{_shr_int}{\hyperref[sailRISCVzzyshrzyint]{#2}}{}%
  \ifstrequal{#1}{\_shr\_int}{\hyperref[sailRISCVzzyshrzyint]{#2}}{}%
  \ifstrequal{#1}{_shr_int_general}{\hyperref[sailRISCVzzyshrzyintzygeneral]{#2}}{}%
  \ifstrequal{#1}{\_shr\_int\_general}{\hyperref[sailRISCVzzyshrzyintzygeneral]{#2}}{}%
  \ifstrequal{#1}{_tmod_int}{\hyperref[sailRISCVzzytmodzyint]{#2}}{}%
  \ifstrequal{#1}{\_tmod\_int}{\hyperref[sailRISCVzzytmodzyint]{#2}}{}%
  \ifstrequal{#1}{_tmod_int_positive}{\hyperref[sailRISCVzzytmodzyintzypositive]{#2}}{}%
  \ifstrequal{#1}{\_tmod\_int\_positive}{\hyperref[sailRISCVzzytmodzyintzypositive]{#2}}{}%
  \ifstrequal{#1}{a64_barrier_domain_of_num}{\hyperref[sailRISCVza64zybarrierzydomainzyofzynum]{#2}}{}%
  \ifstrequal{#1}{a64\_barrier\_domain\_of\_num}{\hyperref[sailRISCVza64zybarrierzydomainzyofzynum]{#2}}{}%
  \ifstrequal{#1}{a64_barrier_type_of_num}{\hyperref[sailRISCVza64zybarrierzytypezyofzynum]{#2}}{}%
  \ifstrequal{#1}{a64\_barrier\_type\_of\_num}{\hyperref[sailRISCVza64zybarrierzytypezyofzynum]{#2}}{}%
  \ifstrequal{#1}{abs_int_atom}{\hyperref[sailRISCVzabszyintzyatom]{#2}}{}%
  \ifstrequal{#1}{abs\_int\_atom}{\hyperref[sailRISCVzabszyintzyatom]{#2}}{}%
  \ifstrequal{#1}{abs_int_plain}{\hyperref[sailRISCVzabszyintzyplain]{#2}}{}%
  \ifstrequal{#1}{abs\_int\_plain}{\hyperref[sailRISCVzabszyintzyplain]{#2}}{}%
  \ifstrequal{#1}{accessType_to_str}{\hyperref[sailRISCVzaccessTypezytozystr]{#2}}{}%
  \ifstrequal{#1}{accessType\_to\_str}{\hyperref[sailRISCVzaccessTypezytozystr]{#2}}{}%
  \ifstrequal{#1}{accrue_fflags}{\hyperref[sailRISCVzaccruezyfflags]{#2}}{}%
  \ifstrequal{#1}{accrue\_fflags}{\hyperref[sailRISCVzaccruezyfflags]{#2}}{}%
  \ifstrequal{#1}{add_atom}{\hyperref[sailRISCVzaddzyatom]{#2}}{}%
  \ifstrequal{#1}{add\_atom}{\hyperref[sailRISCVzaddzyatom]{#2}}{}%
  \ifstrequal{#1}{add_bits}{\hyperref[sailRISCVzaddzybits]{#2}}{}%
  \ifstrequal{#1}{add\_bits}{\hyperref[sailRISCVzaddzybits]{#2}}{}%
  \ifstrequal{#1}{add_bits_int}{\hyperref[sailRISCVzaddzybitszyint]{#2}}{}%
  \ifstrequal{#1}{add\_bits\_int}{\hyperref[sailRISCVzaddzybitszyint]{#2}}{}%
  \ifstrequal{#1}{add_int}{\hyperref[sailRISCVzaddzyint]{#2}}{}%
  \ifstrequal{#1}{add\_int}{\hyperref[sailRISCVzaddzyint]{#2}}{}%
  \ifstrequal{#1}{add_to_TLB39}{\hyperref[sailRISCVzaddzytozyTLB39]{#2}}{}%
  \ifstrequal{#1}{add\_to\_TLB39}{\hyperref[sailRISCVzaddzytozyTLB39]{#2}}{}%
  \ifstrequal{#1}{add_to_TLB48}{\hyperref[sailRISCVzaddzytozyTLB48]{#2}}{}%
  \ifstrequal{#1}{add\_to\_TLB48}{\hyperref[sailRISCVzaddzytozyTLB48]{#2}}{}%
  \ifstrequal{#1}{addr_to_tag_addr}{\hyperref[sailRISCVzaddrzytozytagzyaddr]{#2}}{}%
  \ifstrequal{#1}{addr\_to\_tag\_addr}{\hyperref[sailRISCVzaddrzytozytagzyaddr]{#2}}{}%
  \ifstrequal{#1}{amo_mnemonic}{\hyperref[sailRISCVzamozymnemonic]{#2}}{}%
  \ifstrequal{#1}{amo\_mnemonic}{\hyperref[sailRISCVzamozymnemonic]{#2}}{}%
  \ifstrequal{#1}{amo_width_valid}{\hyperref[sailRISCVzamozywidthzyvalid]{#2}}{}%
  \ifstrequal{#1}{amo\_width\_valid}{\hyperref[sailRISCVzamozywidthzyvalid]{#2}}{}%
  \ifstrequal{#1}{amoop_of_num}{\hyperref[sailRISCVzamoopzyofzynum]{#2}}{}%
  \ifstrequal{#1}{amoop\_of\_num}{\hyperref[sailRISCVzamoopzyofzynum]{#2}}{}%
  \ifstrequal{#1}{and_bool}{\hyperref[sailRISCVzandzybool]{#2}}{}%
  \ifstrequal{#1}{and\_bool}{\hyperref[sailRISCVzandzybool]{#2}}{}%
  \ifstrequal{#1}{and_bool_no_flow}{\hyperref[sailRISCVzandzyboolzynozyflow]{#2}}{}%
  \ifstrequal{#1}{and\_bool\_no\_flow}{\hyperref[sailRISCVzandzyboolzynozyflow]{#2}}{}%
  \ifstrequal{#1}{and_vec}{\hyperref[sailRISCVzandzyvec]{#2}}{}%
  \ifstrequal{#1}{and\_vec}{\hyperref[sailRISCVzandzyvec]{#2}}{}%
  \ifstrequal{#1}{any_vector_update}{\hyperref[sailRISCVzanyzyvectorzyupdate]{#2}}{}%
  \ifstrequal{#1}{any\_vector\_update}{\hyperref[sailRISCVzanyzyvectorzyupdate]{#2}}{}%
  \ifstrequal{#1}{append_64}{\hyperref[sailRISCVzappendzy64]{#2}}{}%
  \ifstrequal{#1}{append\_64}{\hyperref[sailRISCVzappendzy64]{#2}}{}%
  \ifstrequal{#1}{aqrl_str}{\hyperref[sailRISCVzaqrlzystr]{#2}}{}%
  \ifstrequal{#1}{aqrl\_str}{\hyperref[sailRISCVzaqrlzystr]{#2}}{}%
  \ifstrequal{#1}{arch_to_bits}{\hyperref[sailRISCVzarchzytozybits]{#2}}{}%
  \ifstrequal{#1}{arch\_to\_bits}{\hyperref[sailRISCVzarchzytozybits]{#2}}{}%
  \ifstrequal{#1}{architecture}{\hyperref[sailRISCVzarchitecture]{#2}}{}%
  \ifstrequal{#1}{assembly}{\hyperref[sailRISCVzassembly]{#2}}{}%
  \ifstrequal{#1}{biop_zbs_of_num}{\hyperref[sailRISCVzbiopzyzzbszyofzynum]{#2}}{}%
  \ifstrequal{#1}{biop\_zbs\_of\_num}{\hyperref[sailRISCVzbiopzyzzbszyofzynum]{#2}}{}%
  \ifstrequal{#1}{bit_maybe_i}{\hyperref[sailRISCVzbitzymaybezyi]{#2}}{}%
  \ifstrequal{#1}{bit\_maybe\_i}{\hyperref[sailRISCVzbitzymaybezyi]{#2}}{}%
  \ifstrequal{#1}{bit_maybe_o}{\hyperref[sailRISCVzbitzymaybezyo]{#2}}{}%
  \ifstrequal{#1}{bit\_maybe\_o}{\hyperref[sailRISCVzbitzymaybezyo]{#2}}{}%
  \ifstrequal{#1}{bit_maybe_r}{\hyperref[sailRISCVzbitzymaybezyr]{#2}}{}%
  \ifstrequal{#1}{bit\_maybe\_r}{\hyperref[sailRISCVzbitzymaybezyr]{#2}}{}%
  \ifstrequal{#1}{bit_maybe_w}{\hyperref[sailRISCVzbitzymaybezyw]{#2}}{}%
  \ifstrequal{#1}{bit\_maybe\_w}{\hyperref[sailRISCVzbitzymaybezyw]{#2}}{}%
  \ifstrequal{#1}{bit_to_bool}{\hyperref[sailRISCVzbitzytozybool]{#2}}{}%
  \ifstrequal{#1}{bit\_to\_bool}{\hyperref[sailRISCVzbitzytozybool]{#2}}{}%
  \ifstrequal{#1}{bits_str}{\hyperref[sailRISCVzbitszystr]{#2}}{}%
  \ifstrequal{#1}{bits\_str}{\hyperref[sailRISCVzbitszystr]{#2}}{}%
  \ifstrequal{#1}{bitvector_access}{\hyperref[sailRISCVzbitvectorzyaccess]{#2}}{}%
  \ifstrequal{#1}{bitvector\_access}{\hyperref[sailRISCVzbitvectorzyaccess]{#2}}{}%
  \ifstrequal{#1}{bitvector_concat}{\hyperref[sailRISCVzbitvectorzyconcat]{#2}}{}%
  \ifstrequal{#1}{bitvector\_concat}{\hyperref[sailRISCVzbitvectorzyconcat]{#2}}{}%
  \ifstrequal{#1}{bitvector_length}{\hyperref[sailRISCVzbitvectorzylength]{#2}}{}%
  \ifstrequal{#1}{bitvector\_length}{\hyperref[sailRISCVzbitvectorzylength]{#2}}{}%
  \ifstrequal{#1}{bitvector_update}{\hyperref[sailRISCVzbitvectorzyupdate]{#2}}{}%
  \ifstrequal{#1}{bitvector\_update}{\hyperref[sailRISCVzbitvectorzyupdate]{#2}}{}%
  \ifstrequal{#1}{bool_bits}{\hyperref[sailRISCVzboolzybits]{#2}}{}%
  \ifstrequal{#1}{bool\_bits}{\hyperref[sailRISCVzboolzybits]{#2}}{}%
  \ifstrequal{#1}{bool_not_bits}{\hyperref[sailRISCVzboolzynotzybits]{#2}}{}%
  \ifstrequal{#1}{bool\_not\_bits}{\hyperref[sailRISCVzboolzynotzybits]{#2}}{}%
  \ifstrequal{#1}{bool_to_bit}{\hyperref[sailRISCVzboolzytozybit]{#2}}{}%
  \ifstrequal{#1}{bool\_to\_bit}{\hyperref[sailRISCVzboolzytozybit]{#2}}{}%
  \ifstrequal{#1}{bool_to_bits}{\hyperref[sailRISCVzboolzytozybits]{#2}}{}%
  \ifstrequal{#1}{bool\_to\_bits}{\hyperref[sailRISCVzboolzytozybits]{#2}}{}%
  \ifstrequal{#1}{bop_of_num}{\hyperref[sailRISCVzbopzyofzynum]{#2}}{}%
  \ifstrequal{#1}{bop\_of\_num}{\hyperref[sailRISCVzbopzyofzynum]{#2}}{}%
  \ifstrequal{#1}{brop_zba_of_num}{\hyperref[sailRISCVzbropzyzzbazyofzynum]{#2}}{}%
  \ifstrequal{#1}{brop\_zba\_of\_num}{\hyperref[sailRISCVzbropzyzzbazyofzynum]{#2}}{}%
  \ifstrequal{#1}{brop_zbb_of_num}{\hyperref[sailRISCVzbropzyzzbbzyofzynum]{#2}}{}%
  \ifstrequal{#1}{brop\_zbb\_of\_num}{\hyperref[sailRISCVzbropzyzzbbzyofzynum]{#2}}{}%
  \ifstrequal{#1}{brop_zbkb_of_num}{\hyperref[sailRISCVzbropzyzzbkbzyofzynum]{#2}}{}%
  \ifstrequal{#1}{brop\_zbkb\_of\_num}{\hyperref[sailRISCVzbropzyzzbkbzyofzynum]{#2}}{}%
  \ifstrequal{#1}{brop_zbs_of_num}{\hyperref[sailRISCVzbropzyzzbszyofzynum]{#2}}{}%
  \ifstrequal{#1}{brop\_zbs\_of\_num}{\hyperref[sailRISCVzbropzyzzbszyofzynum]{#2}}{}%
  \ifstrequal{#1}{bropw_zba_of_num}{\hyperref[sailRISCVzbropwzyzzbazyofzynum]{#2}}{}%
  \ifstrequal{#1}{bropw\_zba\_of\_num}{\hyperref[sailRISCVzbropwzyzzbazyofzynum]{#2}}{}%
  \ifstrequal{#1}{bropw_zbb_of_num}{\hyperref[sailRISCVzbropwzyzzbbzyofzynum]{#2}}{}%
  \ifstrequal{#1}{bropw\_zbb\_of\_num}{\hyperref[sailRISCVzbropwzyzzbbzyofzynum]{#2}}{}%
  \ifstrequal{#1}{btype_mnemonic}{\hyperref[sailRISCVzbtypezymnemonic]{#2}}{}%
  \ifstrequal{#1}{btype\_mnemonic}{\hyperref[sailRISCVzbtypezymnemonic]{#2}}{}%
  \ifstrequal{#1}{cache_op_kind_of_num}{\hyperref[sailRISCVzcachezyopzykindzyofzynum]{#2}}{}%
  \ifstrequal{#1}{cache\_op\_kind\_of\_num}{\hyperref[sailRISCVzcachezyopzykindzyofzynum]{#2}}{}%
  \ifstrequal{#1}{cancel_reservation}{\hyperref[sailRISCVzcancelzyreservation]{#2}}{}%
  \ifstrequal{#1}{cancel\_reservation}{\hyperref[sailRISCVzcancelzyreservation]{#2}}{}%
  \ifstrequal{#1}{canonical_NaN_D}{\hyperref[sailRISCVzcanonicalzyNaNzyD]{#2}}{}%
  \ifstrequal{#1}{canonical\_NaN\_D}{\hyperref[sailRISCVzcanonicalzyNaNzyD]{#2}}{}%
  \ifstrequal{#1}{canonical_NaN_H}{\hyperref[sailRISCVzcanonicalzyNaNzyH]{#2}}{}%
  \ifstrequal{#1}{canonical\_NaN\_H}{\hyperref[sailRISCVzcanonicalzyNaNzyH]{#2}}{}%
  \ifstrequal{#1}{canonical_NaN_S}{\hyperref[sailRISCVzcanonicalzyNaNzyS]{#2}}{}%
  \ifstrequal{#1}{canonical\_NaN\_S}{\hyperref[sailRISCVzcanonicalzyNaNzyS]{#2}}{}%
  \ifstrequal{#1}{capBitsToCapability}{\hyperref[sailRISCVzcapBitsToCapability]{#2}}{}%
  \ifstrequal{#1}{capBitsToEncCapability}{\hyperref[sailRISCVzcapBitsToEncCapability]{#2}}{}%
  \ifstrequal{#1}{capBoundsEqual}{\hyperref[sailRISCVzcapBoundsEqual]{#2}}{}%
  \ifstrequal{#1}{capToBits}{\hyperref[sailRISCVzcapToBits]{#2}}{}%
  \ifstrequal{#1}{capToEncCap}{\hyperref[sailRISCVzcapToEncCap]{#2}}{}%
  \ifstrequal{#1}{capToMemBits}{\hyperref[sailRISCVzcapToMemBits]{#2}}{}%
  \ifstrequal{#1}{capToString}{\hyperref[sailRISCVzcapToString]{#2}}{}%
  \ifstrequal{#1}{cap_creg_name}{\hyperref[sailRISCVzcapzycregzyname]{#2}}{}%
  \ifstrequal{#1}{cap\_creg\_name}{\hyperref[sailRISCVzcapzycregzyname]{#2}}{}%
  \ifstrequal{#1}{cap_reg_name}{\hyperref[sailRISCVzcapzyregzyname]{#2}}{}%
  \ifstrequal{#1}{cap\_reg\_name}{\hyperref[sailRISCVzcapzyregzyname]{#2}}{}%
  \ifstrequal{#1}{cap_reg_name_abi}{\hyperref[sailRISCVzcapzyregzynamezyabi]{#2}}{}%
  \ifstrequal{#1}{cap\_reg\_name\_abi}{\hyperref[sailRISCVzcapzyregzynamezyabi]{#2}}{}%
  \ifstrequal{#1}{checkPTEPermission}{\hyperref[sailRISCVzcheckPTEPermission]{#2}}{}%
  \ifstrequal{#1}{check_CSR}{\hyperref[sailRISCVzcheckzyCSR]{#2}}{}%
  \ifstrequal{#1}{check\_CSR}{\hyperref[sailRISCVzcheckzyCSR]{#2}}{}%
  \ifstrequal{#1}{check_CSR_access}{\hyperref[sailRISCVzcheckzyCSRzyaccess]{#2}}{}%
  \ifstrequal{#1}{check\_CSR\_access}{\hyperref[sailRISCVzcheckzyCSRzyaccess]{#2}}{}%
  \ifstrequal{#1}{check_Counteren}{\hyperref[sailRISCVzcheckzyCounteren]{#2}}{}%
  \ifstrequal{#1}{check\_Counteren}{\hyperref[sailRISCVzcheckzyCounteren]{#2}}{}%
  \ifstrequal{#1}{check_TVM_SATP}{\hyperref[sailRISCVzcheckzyTVMzySATP]{#2}}{}%
  \ifstrequal{#1}{check\_TVM\_SATP}{\hyperref[sailRISCVzcheckzyTVMzySATP]{#2}}{}%
  \ifstrequal{#1}{check_misaligned}{\hyperref[sailRISCVzcheckzymisaligned]{#2}}{}%
  \ifstrequal{#1}{check\_misaligned}{\hyperref[sailRISCVzcheckzymisaligned]{#2}}{}%
  \ifstrequal{#1}{check_res_misaligned}{\hyperref[sailRISCVzcheckzyreszymisaligned]{#2}}{}%
  \ifstrequal{#1}{check\_res\_misaligned}{\hyperref[sailRISCVzcheckzyreszymisaligned]{#2}}{}%
  \ifstrequal{#1}{check_seed_CSR}{\hyperref[sailRISCVzcheckzyseedzyCSR]{#2}}{}%
  \ifstrequal{#1}{check\_seed\_CSR}{\hyperref[sailRISCVzcheckzyseedzyCSR]{#2}}{}%
  \ifstrequal{#1}{checked_mem_read}{\hyperref[sailRISCVzcheckedzymemzyread]{#2}}{}%
  \ifstrequal{#1}{checked\_mem\_read}{\hyperref[sailRISCVzcheckedzymemzyread]{#2}}{}%
  \ifstrequal{#1}{checked_mem_write}{\hyperref[sailRISCVzcheckedzymemzywrite]{#2}}{}%
  \ifstrequal{#1}{checked\_mem\_write}{\hyperref[sailRISCVzcheckedzymemzywrite]{#2}}{}%
  \ifstrequal{#1}{clearTag}{\hyperref[sailRISCVzclearTag]{#2}}{}%
  \ifstrequal{#1}{clearTagIf}{\hyperref[sailRISCVzclearTagIf]{#2}}{}%
  \ifstrequal{#1}{clearTagIfSealed}{\hyperref[sailRISCVzclearTagIfSealed]{#2}}{}%
  \ifstrequal{#1}{clint_dispatch}{\hyperref[sailRISCVzclintzydispatch]{#2}}{}%
  \ifstrequal{#1}{clint\_dispatch}{\hyperref[sailRISCVzclintzydispatch]{#2}}{}%
  \ifstrequal{#1}{clint_load}{\hyperref[sailRISCVzclintzyload]{#2}}{}%
  \ifstrequal{#1}{clint\_load}{\hyperref[sailRISCVzclintzyload]{#2}}{}%
  \ifstrequal{#1}{clint_store}{\hyperref[sailRISCVzclintzystore]{#2}}{}%
  \ifstrequal{#1}{clint\_store}{\hyperref[sailRISCVzclintzystore]{#2}}{}%
  \ifstrequal{#1}{concat_str}{\hyperref[sailRISCVzconcatzystr]{#2}}{}%
  \ifstrequal{#1}{concat\_str}{\hyperref[sailRISCVzconcatzystr]{#2}}{}%
  \ifstrequal{#1}{concat_str_bits}{\hyperref[sailRISCVzconcatzystrzybits]{#2}}{}%
  \ifstrequal{#1}{concat\_str\_bits}{\hyperref[sailRISCVzconcatzystrzybits]{#2}}{}%
  \ifstrequal{#1}{concat_str_dec}{\hyperref[sailRISCVzconcatzystrzydec]{#2}}{}%
  \ifstrequal{#1}{concat\_str\_dec}{\hyperref[sailRISCVzconcatzystrzydec]{#2}}{}%
  \ifstrequal{#1}{count_leading_zeros}{\hyperref[sailRISCVzcountzyleadingzyzzeros]{#2}}{}%
  \ifstrequal{#1}{count\_leading\_zeros}{\hyperref[sailRISCVzcountzyleadingzyzzeros]{#2}}{}%
  \ifstrequal{#1}{creg2reg_idx}{\hyperref[sailRISCVzcreg2regzyidx]{#2}}{}%
  \ifstrequal{#1}{creg2reg\_idx}{\hyperref[sailRISCVzcreg2regzyidx]{#2}}{}%
  \ifstrequal{#1}{creg_name}{\hyperref[sailRISCVzcregzyname]{#2}}{}%
  \ifstrequal{#1}{creg\_name}{\hyperref[sailRISCVzcregzyname]{#2}}{}%
  \ifstrequal{#1}{csrAccess}{\hyperref[sailRISCVzcsrAccess]{#2}}{}%
  \ifstrequal{#1}{csrPriv}{\hyperref[sailRISCVzcsrPriv]{#2}}{}%
  \ifstrequal{#1}{csr_mnemonic}{\hyperref[sailRISCVzcsrzymnemonic]{#2}}{}%
  \ifstrequal{#1}{csr\_mnemonic}{\hyperref[sailRISCVzcsrzymnemonic]{#2}}{}%
  \ifstrequal{#1}{csr_name}{\hyperref[sailRISCVzcsrzyname]{#2}}{}%
  \ifstrequal{#1}{csr\_name}{\hyperref[sailRISCVzcsrzyname]{#2}}{}%
  \ifstrequal{#1}{csr_name_map}{\hyperref[sailRISCVzcsrzynamezymap]{#2}}{}%
  \ifstrequal{#1}{csr\_name\_map}{\hyperref[sailRISCVzcsrzynamezymap]{#2}}{}%
  \ifstrequal{#1}{csrop_of_num}{\hyperref[sailRISCVzcsropzyofzynum]{#2}}{}%
  \ifstrequal{#1}{csrop\_of\_num}{\hyperref[sailRISCVzcsropzyofzynum]{#2}}{}%
  \ifstrequal{#1}{curAsid32}{\hyperref[sailRISCVzcurAsid32]{#2}}{}%
  \ifstrequal{#1}{curAsid64}{\hyperref[sailRISCVzcurAsid64]{#2}}{}%
  \ifstrequal{#1}{curPTB32}{\hyperref[sailRISCVzcurPTB32]{#2}}{}%
  \ifstrequal{#1}{curPTB64}{\hyperref[sailRISCVzcurPTB64]{#2}}{}%
  \ifstrequal{#1}{cur_Architecture}{\hyperref[sailRISCVzcurzyArchitecture]{#2}}{}%
  \ifstrequal{#1}{cur\_Architecture}{\hyperref[sailRISCVzcurzyArchitecture]{#2}}{}%
  \ifstrequal{#1}{dec_str}{\hyperref[sailRISCVzdeczystr]{#2}}{}%
  \ifstrequal{#1}{dec\_str}{\hyperref[sailRISCVzdeczystr]{#2}}{}%
  \ifstrequal{#1}{decimal_string_of_bits}{\hyperref[sailRISCVzdecimalzystringzyofzybits]{#2}}{}%
  \ifstrequal{#1}{decimal\_string\_of\_bits}{\hyperref[sailRISCVzdecimalzystringzyofzybits]{#2}}{}%
  \ifstrequal{#1}{decode}{\hyperref[sailRISCVzdecode]{#2}}{}%
  \ifstrequal{#1}{decodeCompressed}{\hyperref[sailRISCVzdecodeCompressed]{#2}}{}%
  \ifstrequal{#1}{def_spc}{\hyperref[sailRISCVzdefzyspc]{#2}}{}%
  \ifstrequal{#1}{def\_spc}{\hyperref[sailRISCVzdefzyspc]{#2}}{}%
  \ifstrequal{#1}{def_spc_backwards}{\hyperref[sailRISCVzdefzyspczybackwards]{#2}}{}%
  \ifstrequal{#1}{def\_spc\_backwards}{\hyperref[sailRISCVzdefzyspczybackwards]{#2}}{}%
  \ifstrequal{#1}{def_spc_forwards}{\hyperref[sailRISCVzdefzyspczyforwards]{#2}}{}%
  \ifstrequal{#1}{def\_spc\_forwards}{\hyperref[sailRISCVzdefzyspczyforwards]{#2}}{}%
  \ifstrequal{#1}{def_spc_matches_prefix}{\hyperref[sailRISCVzdefzyspczymatcheszyprefix]{#2}}{}%
  \ifstrequal{#1}{def\_spc\_matches\_prefix}{\hyperref[sailRISCVzdefzyspczymatcheszyprefix]{#2}}{}%
  \ifstrequal{#1}{dirty_fd_context}{\hyperref[sailRISCVzdirtyzyfdzycontext]{#2}}{}%
  \ifstrequal{#1}{dirty\_fd\_context}{\hyperref[sailRISCVzdirtyzyfdzycontext]{#2}}{}%
  \ifstrequal{#1}{dirty_fd_context_if_present}{\hyperref[sailRISCVzdirtyzyfdzycontextzyifzypresent]{#2}}{}%
  \ifstrequal{#1}{dirty\_fd\_context\_if\_present}{\hyperref[sailRISCVzdirtyzyfdzycontextzyifzypresent]{#2}}{}%
  \ifstrequal{#1}{dispatchInterrupt}{\hyperref[sailRISCVzdispatchInterrupt]{#2}}{}%
  \ifstrequal{#1}{dzFlag}{\hyperref[sailRISCVzdzzFlag]{#2}}{}%
  \ifstrequal{#1}{ediv_int}{\hyperref[sailRISCVzedivzyint]{#2}}{}%
  \ifstrequal{#1}{ediv\_int}{\hyperref[sailRISCVzedivzyint]{#2}}{}%
  \ifstrequal{#1}{effectivePrivilege}{\hyperref[sailRISCVzeffectivePrivilege]{#2}}{}%
  \ifstrequal{#1}{elf_entry}{\hyperref[sailRISCVzelfzyentry]{#2}}{}%
  \ifstrequal{#1}{elf\_entry}{\hyperref[sailRISCVzelfzyentry]{#2}}{}%
  \ifstrequal{#1}{elf_tohost}{\hyperref[sailRISCVzelfzytohost]{#2}}{}%
  \ifstrequal{#1}{elf\_tohost}{\hyperref[sailRISCVzelfzytohost]{#2}}{}%
  \ifstrequal{#1}{emod_int}{\hyperref[sailRISCVzemodzyint]{#2}}{}%
  \ifstrequal{#1}{emod\_int}{\hyperref[sailRISCVzemodzyint]{#2}}{}%
  \ifstrequal{#1}{encCapToBits}{\hyperref[sailRISCVzencCapToBits]{#2}}{}%
  \ifstrequal{#1}{encCapabilityToCapability}{\hyperref[sailRISCVzencCapabilityToCapability]{#2}}{}%
  \ifstrequal{#1}{encdec}{\hyperref[sailRISCVzencdec]{#2}}{}%
  \ifstrequal{#1}{encdec_amoop}{\hyperref[sailRISCVzencdeczyamoop]{#2}}{}%
  \ifstrequal{#1}{encdec\_amoop}{\hyperref[sailRISCVzencdeczyamoop]{#2}}{}%
  \ifstrequal{#1}{encdec_bop}{\hyperref[sailRISCVzencdeczybop]{#2}}{}%
  \ifstrequal{#1}{encdec\_bop}{\hyperref[sailRISCVzencdeczybop]{#2}}{}%
  \ifstrequal{#1}{encdec_compressed}{\hyperref[sailRISCVzencdeczycompressed]{#2}}{}%
  \ifstrequal{#1}{encdec\_compressed}{\hyperref[sailRISCVzencdeczycompressed]{#2}}{}%
  \ifstrequal{#1}{encdec_csrop}{\hyperref[sailRISCVzencdeczycsrop]{#2}}{}%
  \ifstrequal{#1}{encdec\_csrop}{\hyperref[sailRISCVzencdeczycsrop]{#2}}{}%
  \ifstrequal{#1}{encdec_iop}{\hyperref[sailRISCVzencdeczyiop]{#2}}{}%
  \ifstrequal{#1}{encdec\_iop}{\hyperref[sailRISCVzencdeczyiop]{#2}}{}%
  \ifstrequal{#1}{encdec_mul_op}{\hyperref[sailRISCVzencdeczymulzyop]{#2}}{}%
  \ifstrequal{#1}{encdec\_mul\_op}{\hyperref[sailRISCVzencdeczymulzyop]{#2}}{}%
  \ifstrequal{#1}{encdec_rounding_mode}{\hyperref[sailRISCVzencdeczyroundingzymode]{#2}}{}%
  \ifstrequal{#1}{encdec\_rounding\_mode}{\hyperref[sailRISCVzencdeczyroundingzymode]{#2}}{}%
  \ifstrequal{#1}{encdec_sop}{\hyperref[sailRISCVzencdeczysop]{#2}}{}%
  \ifstrequal{#1}{encdec\_sop}{\hyperref[sailRISCVzencdeczysop]{#2}}{}%
  \ifstrequal{#1}{encdec_uop}{\hyperref[sailRISCVzencdeczyuop]{#2}}{}%
  \ifstrequal{#1}{encdec\_uop}{\hyperref[sailRISCVzencdeczyuop]{#2}}{}%
  \ifstrequal{#1}{eq_anything}{\hyperref[sailRISCVzeqzyanything]{#2}}{}%
  \ifstrequal{#1}{eq\_anything}{\hyperref[sailRISCVzeqzyanything]{#2}}{}%
  \ifstrequal{#1}{eq_bit}{\hyperref[sailRISCVzeqzybit]{#2}}{}%
  \ifstrequal{#1}{eq\_bit}{\hyperref[sailRISCVzeqzybit]{#2}}{}%
  \ifstrequal{#1}{eq_bits}{\hyperref[sailRISCVzeqzybits]{#2}}{}%
  \ifstrequal{#1}{eq\_bits}{\hyperref[sailRISCVzeqzybits]{#2}}{}%
  \ifstrequal{#1}{eq_bool}{\hyperref[sailRISCVzeqzybool]{#2}}{}%
  \ifstrequal{#1}{eq\_bool}{\hyperref[sailRISCVzeqzybool]{#2}}{}%
  \ifstrequal{#1}{eq_int}{\hyperref[sailRISCVzeqzyint]{#2}}{}%
  \ifstrequal{#1}{eq\_int}{\hyperref[sailRISCVzeqzyint]{#2}}{}%
  \ifstrequal{#1}{eq_string}{\hyperref[sailRISCVzeqzystring]{#2}}{}%
  \ifstrequal{#1}{eq\_string}{\hyperref[sailRISCVzeqzystring]{#2}}{}%
  \ifstrequal{#1}{eq_unit}{\hyperref[sailRISCVzeqzyunit]{#2}}{}%
  \ifstrequal{#1}{eq\_unit}{\hyperref[sailRISCVzeqzyunit]{#2}}{}%
  \ifstrequal{#1}{exceptionType_to_bits}{\hyperref[sailRISCVzexceptionTypezytozybits]{#2}}{}%
  \ifstrequal{#1}{exceptionType\_to\_bits}{\hyperref[sailRISCVzexceptionTypezytozybits]{#2}}{}%
  \ifstrequal{#1}{exceptionType_to_str}{\hyperref[sailRISCVzexceptionTypezytozystr]{#2}}{}%
  \ifstrequal{#1}{exceptionType\_to\_str}{\hyperref[sailRISCVzexceptionTypezytozystr]{#2}}{}%
  \ifstrequal{#1}{exception_delegatee}{\hyperref[sailRISCVzexceptionzydelegatee]{#2}}{}%
  \ifstrequal{#1}{exception\_delegatee}{\hyperref[sailRISCVzexceptionzydelegatee]{#2}}{}%
  \ifstrequal{#1}{exception_handler}{\hyperref[sailRISCVzexceptionzyhandler]{#2}}{}%
  \ifstrequal{#1}{exception\_handler}{\hyperref[sailRISCVzexceptionzyhandler]{#2}}{}%
  \ifstrequal{#1}{execute}{\hyperref[sailRISCVzexecute]{#2}}{}%
  \ifstrequal{#1}{extStatus_of_bits}{\hyperref[sailRISCVzextStatuszyofzybits]{#2}}{}%
  \ifstrequal{#1}{extStatus\_of\_bits}{\hyperref[sailRISCVzextStatuszyofzybits]{#2}}{}%
  \ifstrequal{#1}{extStatus_to_bits}{\hyperref[sailRISCVzextStatuszytozybits]{#2}}{}%
  \ifstrequal{#1}{extStatus\_to\_bits}{\hyperref[sailRISCVzextStatuszytozybits]{#2}}{}%
  \ifstrequal{#1}{ext_access_type_of_num}{\hyperref[sailRISCVzextzyaccesszytypezyofzynum]{#2}}{}%
  \ifstrequal{#1}{ext\_access\_type\_of\_num}{\hyperref[sailRISCVzextzyaccesszytypezyofzynum]{#2}}{}%
  \ifstrequal{#1}{ext_check_CSR}{\hyperref[sailRISCVzextzycheckzyCSR]{#2}}{}%
  \ifstrequal{#1}{ext\_check\_CSR}{\hyperref[sailRISCVzextzycheckzyCSR]{#2}}{}%
  \ifstrequal{#1}{ext_check_CSR_fail}{\hyperref[sailRISCVzextzycheckzyCSRzyfail]{#2}}{}%
  \ifstrequal{#1}{ext\_check\_CSR\_fail}{\hyperref[sailRISCVzextzycheckzyCSRzyfail]{#2}}{}%
  \ifstrequal{#1}{ext_check_phys_mem_read}{\hyperref[sailRISCVzextzycheckzyphyszymemzyread]{#2}}{}%
  \ifstrequal{#1}{ext\_check\_phys\_mem\_read}{\hyperref[sailRISCVzextzycheckzyphyszymemzyread]{#2}}{}%
  \ifstrequal{#1}{ext_check_phys_mem_write}{\hyperref[sailRISCVzextzycheckzyphyszymemzywrite]{#2}}{}%
  \ifstrequal{#1}{ext\_check\_phys\_mem\_write}{\hyperref[sailRISCVzextzycheckzyphyszymemzywrite]{#2}}{}%
  \ifstrequal{#1}{ext_check_xret_priv}{\hyperref[sailRISCVzextzycheckzyxretzypriv]{#2}}{}%
  \ifstrequal{#1}{ext\_check\_xret\_priv}{\hyperref[sailRISCVzextzycheckzyxretzypriv]{#2}}{}%
  \ifstrequal{#1}{ext_control_check_addr}{\hyperref[sailRISCVzextzycontrolzycheckzyaddr]{#2}}{}%
  \ifstrequal{#1}{ext\_control\_check\_addr}{\hyperref[sailRISCVzextzycontrolzycheckzyaddr]{#2}}{}%
  \ifstrequal{#1}{ext_control_check_pc}{\hyperref[sailRISCVzextzycontrolzycheckzypc]{#2}}{}%
  \ifstrequal{#1}{ext\_control\_check\_pc}{\hyperref[sailRISCVzextzycontrolzycheckzypc]{#2}}{}%
  \ifstrequal{#1}{ext_data_get_addr}{\hyperref[sailRISCVzextzydatazygetzyaddr]{#2}}{}%
  \ifstrequal{#1}{ext\_data\_get\_addr}{\hyperref[sailRISCVzextzydatazygetzyaddr]{#2}}{}%
  \ifstrequal{#1}{ext_exc_type_of_num}{\hyperref[sailRISCVzextzyexczytypezyofzynum]{#2}}{}%
  \ifstrequal{#1}{ext\_exc\_type\_of\_num}{\hyperref[sailRISCVzextzyexczytypezyofzynum]{#2}}{}%
  \ifstrequal{#1}{ext_exc_type_to_bits}{\hyperref[sailRISCVzextzyexczytypezytozybits]{#2}}{}%
  \ifstrequal{#1}{ext\_exc\_type\_to\_bits}{\hyperref[sailRISCVzextzyexczytypezytozybits]{#2}}{}%
  \ifstrequal{#1}{ext_exc_type_to_str}{\hyperref[sailRISCVzextzyexczytypezytozystr]{#2}}{}%
  \ifstrequal{#1}{ext\_exc\_type\_to\_str}{\hyperref[sailRISCVzextzyexczytypezytozystr]{#2}}{}%
  \ifstrequal{#1}{ext_fail_xret_priv}{\hyperref[sailRISCVzextzyfailzyxretzypriv]{#2}}{}%
  \ifstrequal{#1}{ext\_fail\_xret\_priv}{\hyperref[sailRISCVzextzyfailzyxretzypriv]{#2}}{}%
  \ifstrequal{#1}{ext_fetch_check_pc}{\hyperref[sailRISCVzextzyfetchzycheckzypc]{#2}}{}%
  \ifstrequal{#1}{ext\_fetch\_check\_pc}{\hyperref[sailRISCVzextzyfetchzycheckzypc]{#2}}{}%
  \ifstrequal{#1}{ext_fetch_hook}{\hyperref[sailRISCVzextzyfetchzyhook]{#2}}{}%
  \ifstrequal{#1}{ext\_fetch\_hook}{\hyperref[sailRISCVzextzyfetchzyhook]{#2}}{}%
  \ifstrequal{#1}{ext_get_ptw_error}{\hyperref[sailRISCVzextzygetzyptwzyerror]{#2}}{}%
  \ifstrequal{#1}{ext\_get\_ptw\_error}{\hyperref[sailRISCVzextzygetzyptwzyerror]{#2}}{}%
  \ifstrequal{#1}{ext_handle_control_check_error}{\hyperref[sailRISCVzextzyhandlezycontrolzycheckzyerror]{#2}}{}%
  \ifstrequal{#1}{ext\_handle\_control\_check\_error}{\hyperref[sailRISCVzextzyhandlezycontrolzycheckzyerror]{#2}}{}%
  \ifstrequal{#1}{ext_handle_data_check_error}{\hyperref[sailRISCVzextzyhandlezydatazycheckzyerror]{#2}}{}%
  \ifstrequal{#1}{ext\_handle\_data\_check\_error}{\hyperref[sailRISCVzextzyhandlezydatazycheckzyerror]{#2}}{}%
  \ifstrequal{#1}{ext_handle_fetch_check_error}{\hyperref[sailRISCVzextzyhandlezyfetchzycheckzyerror]{#2}}{}%
  \ifstrequal{#1}{ext\_handle\_fetch\_check\_error}{\hyperref[sailRISCVzextzyhandlezyfetchzycheckzyerror]{#2}}{}%
  \ifstrequal{#1}{ext_init}{\hyperref[sailRISCVzextzyinit]{#2}}{}%
  \ifstrequal{#1}{ext\_init}{\hyperref[sailRISCVzextzyinit]{#2}}{}%
  \ifstrequal{#1}{ext_init_regs}{\hyperref[sailRISCVzextzyinitzyregs]{#2}}{}%
  \ifstrequal{#1}{ext\_init\_regs}{\hyperref[sailRISCVzextzyinitzyregs]{#2}}{}%
  \ifstrequal{#1}{ext_is_CSR_defined}{\hyperref[sailRISCVzextzyiszyCSRzydefined]{#2}}{}%
  \ifstrequal{#1}{ext\_is\_CSR\_defined}{\hyperref[sailRISCVzextzyiszyCSRzydefined]{#2}}{}%
  \ifstrequal{#1}{ext_post_decode_hook}{\hyperref[sailRISCVzextzypostzydecodezyhook]{#2}}{}%
  \ifstrequal{#1}{ext\_post\_decode\_hook}{\hyperref[sailRISCVzextzypostzydecodezyhook]{#2}}{}%
  \ifstrequal{#1}{ext_post_step_hook}{\hyperref[sailRISCVzextzypostzystepzyhook]{#2}}{}%
  \ifstrequal{#1}{ext\_post\_step\_hook}{\hyperref[sailRISCVzextzypostzystepzyhook]{#2}}{}%
  \ifstrequal{#1}{ext_pre_step_hook}{\hyperref[sailRISCVzextzyprezystepzyhook]{#2}}{}%
  \ifstrequal{#1}{ext\_pre\_step\_hook}{\hyperref[sailRISCVzextzyprezystepzyhook]{#2}}{}%
  \ifstrequal{#1}{ext_ptw_error_of_num}{\hyperref[sailRISCVzextzyptwzyerrorzyofzynum]{#2}}{}%
  \ifstrequal{#1}{ext\_ptw\_error\_of\_num}{\hyperref[sailRISCVzextzyptwzyerrorzyofzynum]{#2}}{}%
  \ifstrequal{#1}{ext_ptw_fail_of_num}{\hyperref[sailRISCVzextzyptwzyfailzyofzynum]{#2}}{}%
  \ifstrequal{#1}{ext\_ptw\_fail\_of\_num}{\hyperref[sailRISCVzextzyptwzyfailzyofzynum]{#2}}{}%
  \ifstrequal{#1}{ext_ptw_lc_join}{\hyperref[sailRISCVzextzyptwzylczyjoin]{#2}}{}%
  \ifstrequal{#1}{ext\_ptw\_lc\_join}{\hyperref[sailRISCVzextzyptwzylczyjoin]{#2}}{}%
  \ifstrequal{#1}{ext_ptw_lc_of_num}{\hyperref[sailRISCVzextzyptwzylczyofzynum]{#2}}{}%
  \ifstrequal{#1}{ext\_ptw\_lc\_of\_num}{\hyperref[sailRISCVzextzyptwzylczyofzynum]{#2}}{}%
  \ifstrequal{#1}{ext_ptw_sc_join}{\hyperref[sailRISCVzextzyptwzysczyjoin]{#2}}{}%
  \ifstrequal{#1}{ext\_ptw\_sc\_join}{\hyperref[sailRISCVzextzyptwzysczyjoin]{#2}}{}%
  \ifstrequal{#1}{ext_ptw_sc_of_num}{\hyperref[sailRISCVzextzyptwzysczyofzynum]{#2}}{}%
  \ifstrequal{#1}{ext\_ptw\_sc\_of\_num}{\hyperref[sailRISCVzextzyptwzysczyofzynum]{#2}}{}%
  \ifstrequal{#1}{ext_read_CSR}{\hyperref[sailRISCVzextzyreadzyCSR]{#2}}{}%
  \ifstrequal{#1}{ext\_read\_CSR}{\hyperref[sailRISCVzextzyreadzyCSR]{#2}}{}%
  \ifstrequal{#1}{ext_rvfi_init}{\hyperref[sailRISCVzextzyrvfizyinit]{#2}}{}%
  \ifstrequal{#1}{ext\_rvfi\_init}{\hyperref[sailRISCVzextzyrvfizyinit]{#2}}{}%
  \ifstrequal{#1}{ext_veto_disable_C}{\hyperref[sailRISCVzextzyvetozydisablezyC]{#2}}{}%
  \ifstrequal{#1}{ext\_veto\_disable\_C}{\hyperref[sailRISCVzextzyvetozydisablezyC]{#2}}{}%
  \ifstrequal{#1}{ext_write_CSR}{\hyperref[sailRISCVzextzywritezyCSR]{#2}}{}%
  \ifstrequal{#1}{ext\_write\_CSR}{\hyperref[sailRISCVzextzywritezyCSR]{#2}}{}%
  \ifstrequal{#1}{ext_write_fcsr}{\hyperref[sailRISCVzextzywritezyfcsr]{#2}}{}%
  \ifstrequal{#1}{ext\_write\_fcsr}{\hyperref[sailRISCVzextzywritezyfcsr]{#2}}{}%
  \ifstrequal{#1}{extend_value}{\hyperref[sailRISCVzextendzyvalue]{#2}}{}%
  \ifstrequal{#1}{extend\_value}{\hyperref[sailRISCVzextendzyvalue]{#2}}{}%
  \ifstrequal{#1}{extern_f16Add}{\hyperref[sailRISCVzexternzyf16Add]{#2}}{}%
  \ifstrequal{#1}{extern\_f16Add}{\hyperref[sailRISCVzexternzyf16Add]{#2}}{}%
  \ifstrequal{#1}{extern_f16Div}{\hyperref[sailRISCVzexternzyf16Div]{#2}}{}%
  \ifstrequal{#1}{extern\_f16Div}{\hyperref[sailRISCVzexternzyf16Div]{#2}}{}%
  \ifstrequal{#1}{extern_f16Eq}{\hyperref[sailRISCVzexternzyf16Eq]{#2}}{}%
  \ifstrequal{#1}{extern\_f16Eq}{\hyperref[sailRISCVzexternzyf16Eq]{#2}}{}%
  \ifstrequal{#1}{extern_f16Le}{\hyperref[sailRISCVzexternzyf16Le]{#2}}{}%
  \ifstrequal{#1}{extern\_f16Le}{\hyperref[sailRISCVzexternzyf16Le]{#2}}{}%
  \ifstrequal{#1}{extern_f16Lt}{\hyperref[sailRISCVzexternzyf16Lt]{#2}}{}%
  \ifstrequal{#1}{extern\_f16Lt}{\hyperref[sailRISCVzexternzyf16Lt]{#2}}{}%
  \ifstrequal{#1}{extern_f16Mul}{\hyperref[sailRISCVzexternzyf16Mul]{#2}}{}%
  \ifstrequal{#1}{extern\_f16Mul}{\hyperref[sailRISCVzexternzyf16Mul]{#2}}{}%
  \ifstrequal{#1}{extern_f16MulAdd}{\hyperref[sailRISCVzexternzyf16MulAdd]{#2}}{}%
  \ifstrequal{#1}{extern\_f16MulAdd}{\hyperref[sailRISCVzexternzyf16MulAdd]{#2}}{}%
  \ifstrequal{#1}{extern_f16Sqrt}{\hyperref[sailRISCVzexternzyf16Sqrt]{#2}}{}%
  \ifstrequal{#1}{extern\_f16Sqrt}{\hyperref[sailRISCVzexternzyf16Sqrt]{#2}}{}%
  \ifstrequal{#1}{extern_f16Sub}{\hyperref[sailRISCVzexternzyf16Sub]{#2}}{}%
  \ifstrequal{#1}{extern\_f16Sub}{\hyperref[sailRISCVzexternzyf16Sub]{#2}}{}%
  \ifstrequal{#1}{extern_f16ToF32}{\hyperref[sailRISCVzexternzyf16ToF32]{#2}}{}%
  \ifstrequal{#1}{extern\_f16ToF32}{\hyperref[sailRISCVzexternzyf16ToF32]{#2}}{}%
  \ifstrequal{#1}{extern_f16ToF64}{\hyperref[sailRISCVzexternzyf16ToF64]{#2}}{}%
  \ifstrequal{#1}{extern\_f16ToF64}{\hyperref[sailRISCVzexternzyf16ToF64]{#2}}{}%
  \ifstrequal{#1}{extern_f16ToI32}{\hyperref[sailRISCVzexternzyf16ToI32]{#2}}{}%
  \ifstrequal{#1}{extern\_f16ToI32}{\hyperref[sailRISCVzexternzyf16ToI32]{#2}}{}%
  \ifstrequal{#1}{extern_f16ToI64}{\hyperref[sailRISCVzexternzyf16ToI64]{#2}}{}%
  \ifstrequal{#1}{extern\_f16ToI64}{\hyperref[sailRISCVzexternzyf16ToI64]{#2}}{}%
  \ifstrequal{#1}{extern_f16ToUi32}{\hyperref[sailRISCVzexternzyf16ToUi32]{#2}}{}%
  \ifstrequal{#1}{extern\_f16ToUi32}{\hyperref[sailRISCVzexternzyf16ToUi32]{#2}}{}%
  \ifstrequal{#1}{extern_f16ToUi64}{\hyperref[sailRISCVzexternzyf16ToUi64]{#2}}{}%
  \ifstrequal{#1}{extern\_f16ToUi64}{\hyperref[sailRISCVzexternzyf16ToUi64]{#2}}{}%
  \ifstrequal{#1}{extern_f32Add}{\hyperref[sailRISCVzexternzyf32Add]{#2}}{}%
  \ifstrequal{#1}{extern\_f32Add}{\hyperref[sailRISCVzexternzyf32Add]{#2}}{}%
  \ifstrequal{#1}{extern_f32Div}{\hyperref[sailRISCVzexternzyf32Div]{#2}}{}%
  \ifstrequal{#1}{extern\_f32Div}{\hyperref[sailRISCVzexternzyf32Div]{#2}}{}%
  \ifstrequal{#1}{extern_f32Eq}{\hyperref[sailRISCVzexternzyf32Eq]{#2}}{}%
  \ifstrequal{#1}{extern\_f32Eq}{\hyperref[sailRISCVzexternzyf32Eq]{#2}}{}%
  \ifstrequal{#1}{extern_f32Le}{\hyperref[sailRISCVzexternzyf32Le]{#2}}{}%
  \ifstrequal{#1}{extern\_f32Le}{\hyperref[sailRISCVzexternzyf32Le]{#2}}{}%
  \ifstrequal{#1}{extern_f32Lt}{\hyperref[sailRISCVzexternzyf32Lt]{#2}}{}%
  \ifstrequal{#1}{extern\_f32Lt}{\hyperref[sailRISCVzexternzyf32Lt]{#2}}{}%
  \ifstrequal{#1}{extern_f32Mul}{\hyperref[sailRISCVzexternzyf32Mul]{#2}}{}%
  \ifstrequal{#1}{extern\_f32Mul}{\hyperref[sailRISCVzexternzyf32Mul]{#2}}{}%
  \ifstrequal{#1}{extern_f32MulAdd}{\hyperref[sailRISCVzexternzyf32MulAdd]{#2}}{}%
  \ifstrequal{#1}{extern\_f32MulAdd}{\hyperref[sailRISCVzexternzyf32MulAdd]{#2}}{}%
  \ifstrequal{#1}{extern_f32Sqrt}{\hyperref[sailRISCVzexternzyf32Sqrt]{#2}}{}%
  \ifstrequal{#1}{extern\_f32Sqrt}{\hyperref[sailRISCVzexternzyf32Sqrt]{#2}}{}%
  \ifstrequal{#1}{extern_f32Sub}{\hyperref[sailRISCVzexternzyf32Sub]{#2}}{}%
  \ifstrequal{#1}{extern\_f32Sub}{\hyperref[sailRISCVzexternzyf32Sub]{#2}}{}%
  \ifstrequal{#1}{extern_f32ToF16}{\hyperref[sailRISCVzexternzyf32ToF16]{#2}}{}%
  \ifstrequal{#1}{extern\_f32ToF16}{\hyperref[sailRISCVzexternzyf32ToF16]{#2}}{}%
  \ifstrequal{#1}{extern_f32ToF64}{\hyperref[sailRISCVzexternzyf32ToF64]{#2}}{}%
  \ifstrequal{#1}{extern\_f32ToF64}{\hyperref[sailRISCVzexternzyf32ToF64]{#2}}{}%
  \ifstrequal{#1}{extern_f32ToI32}{\hyperref[sailRISCVzexternzyf32ToI32]{#2}}{}%
  \ifstrequal{#1}{extern\_f32ToI32}{\hyperref[sailRISCVzexternzyf32ToI32]{#2}}{}%
  \ifstrequal{#1}{extern_f32ToI64}{\hyperref[sailRISCVzexternzyf32ToI64]{#2}}{}%
  \ifstrequal{#1}{extern\_f32ToI64}{\hyperref[sailRISCVzexternzyf32ToI64]{#2}}{}%
  \ifstrequal{#1}{extern_f32ToUi32}{\hyperref[sailRISCVzexternzyf32ToUi32]{#2}}{}%
  \ifstrequal{#1}{extern\_f32ToUi32}{\hyperref[sailRISCVzexternzyf32ToUi32]{#2}}{}%
  \ifstrequal{#1}{extern_f32ToUi64}{\hyperref[sailRISCVzexternzyf32ToUi64]{#2}}{}%
  \ifstrequal{#1}{extern\_f32ToUi64}{\hyperref[sailRISCVzexternzyf32ToUi64]{#2}}{}%
  \ifstrequal{#1}{extern_f64Add}{\hyperref[sailRISCVzexternzyf64Add]{#2}}{}%
  \ifstrequal{#1}{extern\_f64Add}{\hyperref[sailRISCVzexternzyf64Add]{#2}}{}%
  \ifstrequal{#1}{extern_f64Div}{\hyperref[sailRISCVzexternzyf64Div]{#2}}{}%
  \ifstrequal{#1}{extern\_f64Div}{\hyperref[sailRISCVzexternzyf64Div]{#2}}{}%
  \ifstrequal{#1}{extern_f64Eq}{\hyperref[sailRISCVzexternzyf64Eq]{#2}}{}%
  \ifstrequal{#1}{extern\_f64Eq}{\hyperref[sailRISCVzexternzyf64Eq]{#2}}{}%
  \ifstrequal{#1}{extern_f64Le}{\hyperref[sailRISCVzexternzyf64Le]{#2}}{}%
  \ifstrequal{#1}{extern\_f64Le}{\hyperref[sailRISCVzexternzyf64Le]{#2}}{}%
  \ifstrequal{#1}{extern_f64Lt}{\hyperref[sailRISCVzexternzyf64Lt]{#2}}{}%
  \ifstrequal{#1}{extern\_f64Lt}{\hyperref[sailRISCVzexternzyf64Lt]{#2}}{}%
  \ifstrequal{#1}{extern_f64Mul}{\hyperref[sailRISCVzexternzyf64Mul]{#2}}{}%
  \ifstrequal{#1}{extern\_f64Mul}{\hyperref[sailRISCVzexternzyf64Mul]{#2}}{}%
  \ifstrequal{#1}{extern_f64MulAdd}{\hyperref[sailRISCVzexternzyf64MulAdd]{#2}}{}%
  \ifstrequal{#1}{extern\_f64MulAdd}{\hyperref[sailRISCVzexternzyf64MulAdd]{#2}}{}%
  \ifstrequal{#1}{extern_f64Sqrt}{\hyperref[sailRISCVzexternzyf64Sqrt]{#2}}{}%
  \ifstrequal{#1}{extern\_f64Sqrt}{\hyperref[sailRISCVzexternzyf64Sqrt]{#2}}{}%
  \ifstrequal{#1}{extern_f64Sub}{\hyperref[sailRISCVzexternzyf64Sub]{#2}}{}%
  \ifstrequal{#1}{extern\_f64Sub}{\hyperref[sailRISCVzexternzyf64Sub]{#2}}{}%
  \ifstrequal{#1}{extern_f64ToF16}{\hyperref[sailRISCVzexternzyf64ToF16]{#2}}{}%
  \ifstrequal{#1}{extern\_f64ToF16}{\hyperref[sailRISCVzexternzyf64ToF16]{#2}}{}%
  \ifstrequal{#1}{extern_f64ToF32}{\hyperref[sailRISCVzexternzyf64ToF32]{#2}}{}%
  \ifstrequal{#1}{extern\_f64ToF32}{\hyperref[sailRISCVzexternzyf64ToF32]{#2}}{}%
  \ifstrequal{#1}{extern_f64ToI32}{\hyperref[sailRISCVzexternzyf64ToI32]{#2}}{}%
  \ifstrequal{#1}{extern\_f64ToI32}{\hyperref[sailRISCVzexternzyf64ToI32]{#2}}{}%
  \ifstrequal{#1}{extern_f64ToI64}{\hyperref[sailRISCVzexternzyf64ToI64]{#2}}{}%
  \ifstrequal{#1}{extern\_f64ToI64}{\hyperref[sailRISCVzexternzyf64ToI64]{#2}}{}%
  \ifstrequal{#1}{extern_f64ToUi32}{\hyperref[sailRISCVzexternzyf64ToUi32]{#2}}{}%
  \ifstrequal{#1}{extern\_f64ToUi32}{\hyperref[sailRISCVzexternzyf64ToUi32]{#2}}{}%
  \ifstrequal{#1}{extern_f64ToUi64}{\hyperref[sailRISCVzexternzyf64ToUi64]{#2}}{}%
  \ifstrequal{#1}{extern\_f64ToUi64}{\hyperref[sailRISCVzexternzyf64ToUi64]{#2}}{}%
  \ifstrequal{#1}{extern_i32ToF16}{\hyperref[sailRISCVzexternzyi32ToF16]{#2}}{}%
  \ifstrequal{#1}{extern\_i32ToF16}{\hyperref[sailRISCVzexternzyi32ToF16]{#2}}{}%
  \ifstrequal{#1}{extern_i32ToF32}{\hyperref[sailRISCVzexternzyi32ToF32]{#2}}{}%
  \ifstrequal{#1}{extern\_i32ToF32}{\hyperref[sailRISCVzexternzyi32ToF32]{#2}}{}%
  \ifstrequal{#1}{extern_i32ToF64}{\hyperref[sailRISCVzexternzyi32ToF64]{#2}}{}%
  \ifstrequal{#1}{extern\_i32ToF64}{\hyperref[sailRISCVzexternzyi32ToF64]{#2}}{}%
  \ifstrequal{#1}{extern_i64ToF16}{\hyperref[sailRISCVzexternzyi64ToF16]{#2}}{}%
  \ifstrequal{#1}{extern\_i64ToF16}{\hyperref[sailRISCVzexternzyi64ToF16]{#2}}{}%
  \ifstrequal{#1}{extern_i64ToF32}{\hyperref[sailRISCVzexternzyi64ToF32]{#2}}{}%
  \ifstrequal{#1}{extern\_i64ToF32}{\hyperref[sailRISCVzexternzyi64ToF32]{#2}}{}%
  \ifstrequal{#1}{extern_i64ToF64}{\hyperref[sailRISCVzexternzyi64ToF64]{#2}}{}%
  \ifstrequal{#1}{extern\_i64ToF64}{\hyperref[sailRISCVzexternzyi64ToF64]{#2}}{}%
  \ifstrequal{#1}{extern_ui32ToF16}{\hyperref[sailRISCVzexternzyui32ToF16]{#2}}{}%
  \ifstrequal{#1}{extern\_ui32ToF16}{\hyperref[sailRISCVzexternzyui32ToF16]{#2}}{}%
  \ifstrequal{#1}{extern_ui32ToF32}{\hyperref[sailRISCVzexternzyui32ToF32]{#2}}{}%
  \ifstrequal{#1}{extern\_ui32ToF32}{\hyperref[sailRISCVzexternzyui32ToF32]{#2}}{}%
  \ifstrequal{#1}{extern_ui32ToF64}{\hyperref[sailRISCVzexternzyui32ToF64]{#2}}{}%
  \ifstrequal{#1}{extern\_ui32ToF64}{\hyperref[sailRISCVzexternzyui32ToF64]{#2}}{}%
  \ifstrequal{#1}{extern_ui64ToF16}{\hyperref[sailRISCVzexternzyui64ToF16]{#2}}{}%
  \ifstrequal{#1}{extern\_ui64ToF16}{\hyperref[sailRISCVzexternzyui64ToF16]{#2}}{}%
  \ifstrequal{#1}{extern_ui64ToF32}{\hyperref[sailRISCVzexternzyui64ToF32]{#2}}{}%
  \ifstrequal{#1}{extern\_ui64ToF32}{\hyperref[sailRISCVzexternzyui64ToF32]{#2}}{}%
  \ifstrequal{#1}{extern_ui64ToF64}{\hyperref[sailRISCVzexternzyui64ToF64]{#2}}{}%
  \ifstrequal{#1}{extern\_ui64ToF64}{\hyperref[sailRISCVzexternzyui64ToF64]{#2}}{}%
  \ifstrequal{#1}{extop_zbb_of_num}{\hyperref[sailRISCVzextopzyzzbbzyofzynum]{#2}}{}%
  \ifstrequal{#1}{extop\_zbb\_of\_num}{\hyperref[sailRISCVzextopzyzzbbzyofzynum]{#2}}{}%
  \ifstrequal{#1}{f_bin_op_D_of_num}{\hyperref[sailRISCVzfzybinzyopzyDzyofzynum]{#2}}{}%
  \ifstrequal{#1}{f\_bin\_op\_D\_of\_num}{\hyperref[sailRISCVzfzybinzyopzyDzyofzynum]{#2}}{}%
  \ifstrequal{#1}{f_bin_op_H_of_num}{\hyperref[sailRISCVzfzybinzyopzyHzyofzynum]{#2}}{}%
  \ifstrequal{#1}{f\_bin\_op\_H\_of\_num}{\hyperref[sailRISCVzfzybinzyopzyHzyofzynum]{#2}}{}%
  \ifstrequal{#1}{f_bin_op_S_of_num}{\hyperref[sailRISCVzfzybinzyopzySzyofzynum]{#2}}{}%
  \ifstrequal{#1}{f\_bin\_op\_S\_of\_num}{\hyperref[sailRISCVzfzybinzyopzySzyofzynum]{#2}}{}%
  \ifstrequal{#1}{f_bin_rm_op_D_of_num}{\hyperref[sailRISCVzfzybinzyrmzyopzyDzyofzynum]{#2}}{}%
  \ifstrequal{#1}{f\_bin\_rm\_op\_D\_of\_num}{\hyperref[sailRISCVzfzybinzyrmzyopzyDzyofzynum]{#2}}{}%
  \ifstrequal{#1}{f_bin_rm_op_H_of_num}{\hyperref[sailRISCVzfzybinzyrmzyopzyHzyofzynum]{#2}}{}%
  \ifstrequal{#1}{f\_bin\_rm\_op\_H\_of\_num}{\hyperref[sailRISCVzfzybinzyrmzyopzyHzyofzynum]{#2}}{}%
  \ifstrequal{#1}{f_bin_rm_op_S_of_num}{\hyperref[sailRISCVzfzybinzyrmzyopzySzyofzynum]{#2}}{}%
  \ifstrequal{#1}{f\_bin\_rm\_op\_S\_of\_num}{\hyperref[sailRISCVzfzybinzyrmzyopzySzyofzynum]{#2}}{}%
  \ifstrequal{#1}{f_bin_rm_type_mnemonic_D}{\hyperref[sailRISCVzfzybinzyrmzytypezymnemoniczyD]{#2}}{}%
  \ifstrequal{#1}{f\_bin\_rm\_type\_mnemonic\_D}{\hyperref[sailRISCVzfzybinzyrmzytypezymnemoniczyD]{#2}}{}%
  \ifstrequal{#1}{f_bin_rm_type_mnemonic_S}{\hyperref[sailRISCVzfzybinzyrmzytypezymnemoniczyS]{#2}}{}%
  \ifstrequal{#1}{f\_bin\_rm\_type\_mnemonic\_S}{\hyperref[sailRISCVzfzybinzyrmzytypezymnemoniczyS]{#2}}{}%
  \ifstrequal{#1}{f_bin_type_mnemonic_D}{\hyperref[sailRISCVzfzybinzytypezymnemoniczyD]{#2}}{}%
  \ifstrequal{#1}{f\_bin\_type\_mnemonic\_D}{\hyperref[sailRISCVzfzybinzytypezymnemoniczyD]{#2}}{}%
  \ifstrequal{#1}{f_bin_type_mnemonic_S}{\hyperref[sailRISCVzfzybinzytypezymnemoniczyS]{#2}}{}%
  \ifstrequal{#1}{f\_bin\_type\_mnemonic\_S}{\hyperref[sailRISCVzfzybinzytypezymnemoniczyS]{#2}}{}%
  \ifstrequal{#1}{f_is_NaN_D}{\hyperref[sailRISCVzfzyiszyNaNzyD]{#2}}{}%
  \ifstrequal{#1}{f\_is\_NaN\_D}{\hyperref[sailRISCVzfzyiszyNaNzyD]{#2}}{}%
  \ifstrequal{#1}{f_is_NaN_S}{\hyperref[sailRISCVzfzyiszyNaNzyS]{#2}}{}%
  \ifstrequal{#1}{f\_is\_NaN\_S}{\hyperref[sailRISCVzfzyiszyNaNzyS]{#2}}{}%
  \ifstrequal{#1}{f_is_QNaN_D}{\hyperref[sailRISCVzfzyiszyQNaNzyD]{#2}}{}%
  \ifstrequal{#1}{f\_is\_QNaN\_D}{\hyperref[sailRISCVzfzyiszyQNaNzyD]{#2}}{}%
  \ifstrequal{#1}{f_is_QNaN_S}{\hyperref[sailRISCVzfzyiszyQNaNzyS]{#2}}{}%
  \ifstrequal{#1}{f\_is\_QNaN\_S}{\hyperref[sailRISCVzfzyiszyQNaNzyS]{#2}}{}%
  \ifstrequal{#1}{f_is_SNaN_D}{\hyperref[sailRISCVzfzyiszySNaNzyD]{#2}}{}%
  \ifstrequal{#1}{f\_is\_SNaN\_D}{\hyperref[sailRISCVzfzyiszySNaNzyD]{#2}}{}%
  \ifstrequal{#1}{f_is_SNaN_S}{\hyperref[sailRISCVzfzyiszySNaNzyS]{#2}}{}%
  \ifstrequal{#1}{f\_is\_SNaN\_S}{\hyperref[sailRISCVzfzyiszySNaNzyS]{#2}}{}%
  \ifstrequal{#1}{f_is_neg_inf_D}{\hyperref[sailRISCVzfzyiszynegzyinfzyD]{#2}}{}%
  \ifstrequal{#1}{f\_is\_neg\_inf\_D}{\hyperref[sailRISCVzfzyiszynegzyinfzyD]{#2}}{}%
  \ifstrequal{#1}{f_is_neg_inf_S}{\hyperref[sailRISCVzfzyiszynegzyinfzyS]{#2}}{}%
  \ifstrequal{#1}{f\_is\_neg\_inf\_S}{\hyperref[sailRISCVzfzyiszynegzyinfzyS]{#2}}{}%
  \ifstrequal{#1}{f_is_neg_norm_D}{\hyperref[sailRISCVzfzyiszynegzynormzyD]{#2}}{}%
  \ifstrequal{#1}{f\_is\_neg\_norm\_D}{\hyperref[sailRISCVzfzyiszynegzynormzyD]{#2}}{}%
  \ifstrequal{#1}{f_is_neg_norm_S}{\hyperref[sailRISCVzfzyiszynegzynormzyS]{#2}}{}%
  \ifstrequal{#1}{f\_is\_neg\_norm\_S}{\hyperref[sailRISCVzfzyiszynegzynormzyS]{#2}}{}%
  \ifstrequal{#1}{f_is_neg_subnorm_D}{\hyperref[sailRISCVzfzyiszynegzysubnormzyD]{#2}}{}%
  \ifstrequal{#1}{f\_is\_neg\_subnorm\_D}{\hyperref[sailRISCVzfzyiszynegzysubnormzyD]{#2}}{}%
  \ifstrequal{#1}{f_is_neg_subnorm_S}{\hyperref[sailRISCVzfzyiszynegzysubnormzyS]{#2}}{}%
  \ifstrequal{#1}{f\_is\_neg\_subnorm\_S}{\hyperref[sailRISCVzfzyiszynegzysubnormzyS]{#2}}{}%
  \ifstrequal{#1}{f_is_neg_zero_D}{\hyperref[sailRISCVzfzyiszynegzyzzerozyD]{#2}}{}%
  \ifstrequal{#1}{f\_is\_neg\_zero\_D}{\hyperref[sailRISCVzfzyiszynegzyzzerozyD]{#2}}{}%
  \ifstrequal{#1}{f_is_neg_zero_S}{\hyperref[sailRISCVzfzyiszynegzyzzerozyS]{#2}}{}%
  \ifstrequal{#1}{f\_is\_neg\_zero\_S}{\hyperref[sailRISCVzfzyiszynegzyzzerozyS]{#2}}{}%
  \ifstrequal{#1}{f_is_pos_inf_D}{\hyperref[sailRISCVzfzyiszyposzyinfzyD]{#2}}{}%
  \ifstrequal{#1}{f\_is\_pos\_inf\_D}{\hyperref[sailRISCVzfzyiszyposzyinfzyD]{#2}}{}%
  \ifstrequal{#1}{f_is_pos_inf_S}{\hyperref[sailRISCVzfzyiszyposzyinfzyS]{#2}}{}%
  \ifstrequal{#1}{f\_is\_pos\_inf\_S}{\hyperref[sailRISCVzfzyiszyposzyinfzyS]{#2}}{}%
  \ifstrequal{#1}{f_is_pos_norm_D}{\hyperref[sailRISCVzfzyiszyposzynormzyD]{#2}}{}%
  \ifstrequal{#1}{f\_is\_pos\_norm\_D}{\hyperref[sailRISCVzfzyiszyposzynormzyD]{#2}}{}%
  \ifstrequal{#1}{f_is_pos_norm_S}{\hyperref[sailRISCVzfzyiszyposzynormzyS]{#2}}{}%
  \ifstrequal{#1}{f\_is\_pos\_norm\_S}{\hyperref[sailRISCVzfzyiszyposzynormzyS]{#2}}{}%
  \ifstrequal{#1}{f_is_pos_subnorm_D}{\hyperref[sailRISCVzfzyiszyposzysubnormzyD]{#2}}{}%
  \ifstrequal{#1}{f\_is\_pos\_subnorm\_D}{\hyperref[sailRISCVzfzyiszyposzysubnormzyD]{#2}}{}%
  \ifstrequal{#1}{f_is_pos_subnorm_S}{\hyperref[sailRISCVzfzyiszyposzysubnormzyS]{#2}}{}%
  \ifstrequal{#1}{f\_is\_pos\_subnorm\_S}{\hyperref[sailRISCVzfzyiszyposzysubnormzyS]{#2}}{}%
  \ifstrequal{#1}{f_is_pos_zero_D}{\hyperref[sailRISCVzfzyiszyposzyzzerozyD]{#2}}{}%
  \ifstrequal{#1}{f\_is\_pos\_zero\_D}{\hyperref[sailRISCVzfzyiszyposzyzzerozyD]{#2}}{}%
  \ifstrequal{#1}{f_is_pos_zero_S}{\hyperref[sailRISCVzfzyiszyposzyzzerozyS]{#2}}{}%
  \ifstrequal{#1}{f\_is\_pos\_zero\_S}{\hyperref[sailRISCVzfzyiszyposzyzzerozyS]{#2}}{}%
  \ifstrequal{#1}{f_madd_op_D_of_num}{\hyperref[sailRISCVzfzymaddzyopzyDzyofzynum]{#2}}{}%
  \ifstrequal{#1}{f\_madd\_op\_D\_of\_num}{\hyperref[sailRISCVzfzymaddzyopzyDzyofzynum]{#2}}{}%
  \ifstrequal{#1}{f_madd_op_H_of_num}{\hyperref[sailRISCVzfzymaddzyopzyHzyofzynum]{#2}}{}%
  \ifstrequal{#1}{f\_madd\_op\_H\_of\_num}{\hyperref[sailRISCVzfzymaddzyopzyHzyofzynum]{#2}}{}%
  \ifstrequal{#1}{f_madd_op_S_of_num}{\hyperref[sailRISCVzfzymaddzyopzySzyofzynum]{#2}}{}%
  \ifstrequal{#1}{f\_madd\_op\_S\_of\_num}{\hyperref[sailRISCVzfzymaddzyopzySzyofzynum]{#2}}{}%
  \ifstrequal{#1}{f_madd_type_mnemonic_D}{\hyperref[sailRISCVzfzymaddzytypezymnemoniczyD]{#2}}{}%
  \ifstrequal{#1}{f\_madd\_type\_mnemonic\_D}{\hyperref[sailRISCVzfzymaddzytypezymnemoniczyD]{#2}}{}%
  \ifstrequal{#1}{f_madd_type_mnemonic_S}{\hyperref[sailRISCVzfzymaddzytypezymnemoniczyS]{#2}}{}%
  \ifstrequal{#1}{f\_madd\_type\_mnemonic\_S}{\hyperref[sailRISCVzfzymaddzytypezymnemoniczyS]{#2}}{}%
  \ifstrequal{#1}{f_un_op_D_of_num}{\hyperref[sailRISCVzfzyunzyopzyDzyofzynum]{#2}}{}%
  \ifstrequal{#1}{f\_un\_op\_D\_of\_num}{\hyperref[sailRISCVzfzyunzyopzyDzyofzynum]{#2}}{}%
  \ifstrequal{#1}{f_un_op_H_of_num}{\hyperref[sailRISCVzfzyunzyopzyHzyofzynum]{#2}}{}%
  \ifstrequal{#1}{f\_un\_op\_H\_of\_num}{\hyperref[sailRISCVzfzyunzyopzyHzyofzynum]{#2}}{}%
  \ifstrequal{#1}{f_un_op_S_of_num}{\hyperref[sailRISCVzfzyunzyopzySzyofzynum]{#2}}{}%
  \ifstrequal{#1}{f\_un\_op\_S\_of\_num}{\hyperref[sailRISCVzfzyunzyopzySzyofzynum]{#2}}{}%
  \ifstrequal{#1}{f_un_rm_op_D_of_num}{\hyperref[sailRISCVzfzyunzyrmzyopzyDzyofzynum]{#2}}{}%
  \ifstrequal{#1}{f\_un\_rm\_op\_D\_of\_num}{\hyperref[sailRISCVzfzyunzyrmzyopzyDzyofzynum]{#2}}{}%
  \ifstrequal{#1}{f_un_rm_op_H_of_num}{\hyperref[sailRISCVzfzyunzyrmzyopzyHzyofzynum]{#2}}{}%
  \ifstrequal{#1}{f\_un\_rm\_op\_H\_of\_num}{\hyperref[sailRISCVzfzyunzyrmzyopzyHzyofzynum]{#2}}{}%
  \ifstrequal{#1}{f_un_rm_op_S_of_num}{\hyperref[sailRISCVzfzyunzyrmzyopzySzyofzynum]{#2}}{}%
  \ifstrequal{#1}{f\_un\_rm\_op\_S\_of\_num}{\hyperref[sailRISCVzfzyunzyrmzyopzySzyofzynum]{#2}}{}%
  \ifstrequal{#1}{f_un_rm_type_mnemonic_D}{\hyperref[sailRISCVzfzyunzyrmzytypezymnemoniczyD]{#2}}{}%
  \ifstrequal{#1}{f\_un\_rm\_type\_mnemonic\_D}{\hyperref[sailRISCVzfzyunzyrmzytypezymnemoniczyD]{#2}}{}%
  \ifstrequal{#1}{f_un_rm_type_mnemonic_S}{\hyperref[sailRISCVzfzyunzyrmzytypezymnemoniczyS]{#2}}{}%
  \ifstrequal{#1}{f\_un\_rm\_type\_mnemonic\_S}{\hyperref[sailRISCVzfzyunzyrmzytypezymnemoniczyS]{#2}}{}%
  \ifstrequal{#1}{f_un_type_mnemonic_D}{\hyperref[sailRISCVzfzyunzytypezymnemoniczyD]{#2}}{}%
  \ifstrequal{#1}{f\_un\_type\_mnemonic\_D}{\hyperref[sailRISCVzfzyunzytypezymnemoniczyD]{#2}}{}%
  \ifstrequal{#1}{f_un_type_mnemonic_S}{\hyperref[sailRISCVzfzyunzytypezymnemoniczyS]{#2}}{}%
  \ifstrequal{#1}{f\_un\_type\_mnemonic\_S}{\hyperref[sailRISCVzfzyunzytypezymnemoniczyS]{#2}}{}%
  \ifstrequal{#1}{fastRepCheck}{\hyperref[sailRISCVzfastRepCheck]{#2}}{}%
  \ifstrequal{#1}{fdiv_int}{\hyperref[sailRISCVzfdivzyint]{#2}}{}%
  \ifstrequal{#1}{fdiv\_int}{\hyperref[sailRISCVzfdivzyint]{#2}}{}%
  \ifstrequal{#1}{fence_bits}{\hyperref[sailRISCVzfencezybits]{#2}}{}%
  \ifstrequal{#1}{fence\_bits}{\hyperref[sailRISCVzfencezybits]{#2}}{}%
  \ifstrequal{#1}{feq_quiet_D}{\hyperref[sailRISCVzfeqzyquietzyD]{#2}}{}%
  \ifstrequal{#1}{feq\_quiet\_D}{\hyperref[sailRISCVzfeqzyquietzyD]{#2}}{}%
  \ifstrequal{#1}{feq_quiet_S}{\hyperref[sailRISCVzfeqzyquietzyS]{#2}}{}%
  \ifstrequal{#1}{feq\_quiet\_S}{\hyperref[sailRISCVzfeqzyquietzyS]{#2}}{}%
  \ifstrequal{#1}{fetch}{\hyperref[sailRISCVzfetch]{#2}}{}%
  \ifstrequal{#1}{findPendingInterrupt}{\hyperref[sailRISCVzfindPendingInterrupt]{#2}}{}%
  \ifstrequal{#1}{fle_D}{\hyperref[sailRISCVzflezyD]{#2}}{}%
  \ifstrequal{#1}{fle\_D}{\hyperref[sailRISCVzflezyD]{#2}}{}%
  \ifstrequal{#1}{fle_S}{\hyperref[sailRISCVzflezyS]{#2}}{}%
  \ifstrequal{#1}{fle\_S}{\hyperref[sailRISCVzflezyS]{#2}}{}%
  \ifstrequal{#1}{flt_D}{\hyperref[sailRISCVzfltzyD]{#2}}{}%
  \ifstrequal{#1}{flt\_D}{\hyperref[sailRISCVzfltzyD]{#2}}{}%
  \ifstrequal{#1}{flt_S}{\hyperref[sailRISCVzfltzyS]{#2}}{}%
  \ifstrequal{#1}{flt\_S}{\hyperref[sailRISCVzfltzyS]{#2}}{}%
  \ifstrequal{#1}{flush_TLB}{\hyperref[sailRISCVzflushzyTLB]{#2}}{}%
  \ifstrequal{#1}{flush\_TLB}{\hyperref[sailRISCVzflushzyTLB]{#2}}{}%
  \ifstrequal{#1}{flush_TLB39}{\hyperref[sailRISCVzflushzyTLB39]{#2}}{}%
  \ifstrequal{#1}{flush\_TLB39}{\hyperref[sailRISCVzflushzyTLB39]{#2}}{}%
  \ifstrequal{#1}{flush_TLB48}{\hyperref[sailRISCVzflushzyTLB48]{#2}}{}%
  \ifstrequal{#1}{flush\_TLB48}{\hyperref[sailRISCVzflushzyTLB48]{#2}}{}%
  \ifstrequal{#1}{flush_TLB_Entry}{\hyperref[sailRISCVzflushzyTLBzyEntry]{#2}}{}%
  \ifstrequal{#1}{flush\_TLB\_Entry}{\hyperref[sailRISCVzflushzyTLBzyEntry]{#2}}{}%
  \ifstrequal{#1}{fmake_D}{\hyperref[sailRISCVzfmakezyD]{#2}}{}%
  \ifstrequal{#1}{fmake\_D}{\hyperref[sailRISCVzfmakezyD]{#2}}{}%
  \ifstrequal{#1}{fmake_S}{\hyperref[sailRISCVzfmakezyS]{#2}}{}%
  \ifstrequal{#1}{fmake\_S}{\hyperref[sailRISCVzfmakezyS]{#2}}{}%
  \ifstrequal{#1}{fmod_int}{\hyperref[sailRISCVzfmodzyint]{#2}}{}%
  \ifstrequal{#1}{fmod\_int}{\hyperref[sailRISCVzfmodzyint]{#2}}{}%
  \ifstrequal{#1}{freg_name}{\hyperref[sailRISCVzfregzyname]{#2}}{}%
  \ifstrequal{#1}{freg\_name}{\hyperref[sailRISCVzfregzyname]{#2}}{}%
  \ifstrequal{#1}{freg_name_abi}{\hyperref[sailRISCVzfregzynamezyabi]{#2}}{}%
  \ifstrequal{#1}{freg\_name\_abi}{\hyperref[sailRISCVzfregzynamezyabi]{#2}}{}%
  \ifstrequal{#1}{freg_or_reg_name}{\hyperref[sailRISCVzfregzyorzyregzyname]{#2}}{}%
  \ifstrequal{#1}{freg\_or\_reg\_name}{\hyperref[sailRISCVzfregzyorzyregzyname]{#2}}{}%
  \ifstrequal{#1}{fregval_from_freg}{\hyperref[sailRISCVzfregvalzyfromzyfreg]{#2}}{}%
  \ifstrequal{#1}{fregval\_from\_freg}{\hyperref[sailRISCVzfregvalzyfromzyfreg]{#2}}{}%
  \ifstrequal{#1}{fregval_into_freg}{\hyperref[sailRISCVzfregvalzyintozyfreg]{#2}}{}%
  \ifstrequal{#1}{fregval\_into\_freg}{\hyperref[sailRISCVzfregvalzyintozyfreg]{#2}}{}%
  \ifstrequal{#1}{frm_mnemonic}{\hyperref[sailRISCVzfrmzymnemonic]{#2}}{}%
  \ifstrequal{#1}{frm\_mnemonic}{\hyperref[sailRISCVzfrmzymnemonic]{#2}}{}%
  \ifstrequal{#1}{fsplit_D}{\hyperref[sailRISCVzfsplitzyD]{#2}}{}%
  \ifstrequal{#1}{fsplit\_D}{\hyperref[sailRISCVzfsplitzyD]{#2}}{}%
  \ifstrequal{#1}{fsplit_S}{\hyperref[sailRISCVzfsplitzyS]{#2}}{}%
  \ifstrequal{#1}{fsplit\_S}{\hyperref[sailRISCVzfsplitzyS]{#2}}{}%
  \ifstrequal{#1}{getCapBase}{\hyperref[sailRISCVzgetCapBase]{#2}}{}%
  \ifstrequal{#1}{getCapBaseBits}{\hyperref[sailRISCVzgetCapBaseBits]{#2}}{}%
  \ifstrequal{#1}{getCapBounds}{\hyperref[sailRISCVzgetCapBounds]{#2}}{}%
  \ifstrequal{#1}{getCapBoundsBits}{\hyperref[sailRISCVzgetCapBoundsBits]{#2}}{}%
  \ifstrequal{#1}{getCapCursor}{\hyperref[sailRISCVzgetCapCursor]{#2}}{}%
  \ifstrequal{#1}{getCapFlags}{\hyperref[sailRISCVzgetCapFlags]{#2}}{}%
  \ifstrequal{#1}{getCapHardPerms}{\hyperref[sailRISCVzgetCapHardPerms]{#2}}{}%
  \ifstrequal{#1}{getCapLength}{\hyperref[sailRISCVzgetCapLength]{#2}}{}%
  \ifstrequal{#1}{getCapOffset}{\hyperref[sailRISCVzgetCapOffset]{#2}}{}%
  \ifstrequal{#1}{getCapOffsetBits}{\hyperref[sailRISCVzgetCapOffsetBits]{#2}}{}%
  \ifstrequal{#1}{getCapPerms}{\hyperref[sailRISCVzgetCapPerms]{#2}}{}%
  \ifstrequal{#1}{getCapTop}{\hyperref[sailRISCVzgetCapTop]{#2}}{}%
  \ifstrequal{#1}{getCapTopBits}{\hyperref[sailRISCVzgetCapTopBits]{#2}}{}%
  \ifstrequal{#1}{getPendingSet}{\hyperref[sailRISCVzgetPendingSet]{#2}}{}%
  \ifstrequal{#1}{getRepresentableAlignmentMask}{\hyperref[sailRISCVzgetRepresentableAlignmentMask]{#2}}{}%
  \ifstrequal{#1}{getRepresentableLength}{\hyperref[sailRISCVzgetRepresentableLength]{#2}}{}%
  \ifstrequal{#1}{get_16_random_bits}{\hyperref[sailRISCVzgetzy16zyrandomzybits]{#2}}{}%
  \ifstrequal{#1}{get\_16\_random\_bits}{\hyperref[sailRISCVzgetzy16zyrandomzybits]{#2}}{}%
  \ifstrequal{#1}{get_arch_pc}{\hyperref[sailRISCVzgetzyarchzypc]{#2}}{}%
  \ifstrequal{#1}{get\_arch\_pc}{\hyperref[sailRISCVzgetzyarchzypc]{#2}}{}%
  \ifstrequal{#1}{get_cheri_mode_cap_addr}{\hyperref[sailRISCVzgetzycherizymodezycapzyaddr]{#2}}{}%
  \ifstrequal{#1}{get\_cheri\_mode\_cap\_addr}{\hyperref[sailRISCVzgetzycherizymodezycapzyaddr]{#2}}{}%
  \ifstrequal{#1}{get_config_print_instr}{\hyperref[sailRISCVzgetzyconfigzyprintzyinstr]{#2}}{}%
  \ifstrequal{#1}{get\_config\_print\_instr}{\hyperref[sailRISCVzgetzyconfigzyprintzyinstr]{#2}}{}%
  \ifstrequal{#1}{get_config_print_mem}{\hyperref[sailRISCVzgetzyconfigzyprintzymem]{#2}}{}%
  \ifstrequal{#1}{get\_config\_print\_mem}{\hyperref[sailRISCVzgetzyconfigzyprintzymem]{#2}}{}%
  \ifstrequal{#1}{get_config_print_platform}{\hyperref[sailRISCVzgetzyconfigzyprintzyplatform]{#2}}{}%
  \ifstrequal{#1}{get\_config\_print\_platform}{\hyperref[sailRISCVzgetzyconfigzyprintzyplatform]{#2}}{}%
  \ifstrequal{#1}{get_config_print_reg}{\hyperref[sailRISCVzgetzyconfigzyprintzyreg]{#2}}{}%
  \ifstrequal{#1}{get\_config\_print\_reg}{\hyperref[sailRISCVzgetzyconfigzyprintzyreg]{#2}}{}%
  \ifstrequal{#1}{get_mstatus_SXL}{\hyperref[sailRISCVzgetzymstatuszySXL]{#2}}{}%
  \ifstrequal{#1}{get\_mstatus\_SXL}{\hyperref[sailRISCVzgetzymstatuszySXL]{#2}}{}%
  \ifstrequal{#1}{get_mstatus_UXL}{\hyperref[sailRISCVzgetzymstatuszyUXL]{#2}}{}%
  \ifstrequal{#1}{get\_mstatus\_UXL}{\hyperref[sailRISCVzgetzymstatuszyUXL]{#2}}{}%
  \ifstrequal{#1}{get_mtvec}{\hyperref[sailRISCVzgetzymtvec]{#2}}{}%
  \ifstrequal{#1}{get\_mtvec}{\hyperref[sailRISCVzgetzymtvec]{#2}}{}%
  \ifstrequal{#1}{get_next_pc}{\hyperref[sailRISCVzgetzynextzypc]{#2}}{}%
  \ifstrequal{#1}{get\_next\_pc}{\hyperref[sailRISCVzgetzynextzypc]{#2}}{}%
  \ifstrequal{#1}{get_slice_int}{\hyperref[sailRISCVzgetzyslicezyint]{#2}}{}%
  \ifstrequal{#1}{get\_slice\_int}{\hyperref[sailRISCVzgetzyslicezyint]{#2}}{}%
  \ifstrequal{#1}{get_sstatus_UXL}{\hyperref[sailRISCVzgetzysstatuszyUXL]{#2}}{}%
  \ifstrequal{#1}{get\_sstatus\_UXL}{\hyperref[sailRISCVzgetzysstatuszyUXL]{#2}}{}%
  \ifstrequal{#1}{get_stvec}{\hyperref[sailRISCVzgetzystvec]{#2}}{}%
  \ifstrequal{#1}{get\_stvec}{\hyperref[sailRISCVzgetzystvec]{#2}}{}%
  \ifstrequal{#1}{get_utvec}{\hyperref[sailRISCVzgetzyutvec]{#2}}{}%
  \ifstrequal{#1}{get\_utvec}{\hyperref[sailRISCVzgetzyutvec]{#2}}{}%
  \ifstrequal{#1}{get_xret_target}{\hyperref[sailRISCVzgetzyxretzytarget]{#2}}{}%
  \ifstrequal{#1}{get\_xret\_target}{\hyperref[sailRISCVzgetzyxretzytarget]{#2}}{}%
  \ifstrequal{#1}{gt_int}{\hyperref[sailRISCVzgtzyint]{#2}}{}%
  \ifstrequal{#1}{gt\_int}{\hyperref[sailRISCVzgtzyint]{#2}}{}%
  \ifstrequal{#1}{gteq_int}{\hyperref[sailRISCVzgteqzyint]{#2}}{}%
  \ifstrequal{#1}{gteq\_int}{\hyperref[sailRISCVzgteqzyint]{#2}}{}%
  \ifstrequal{#1}{handle_cheri_cap_exception}{\hyperref[sailRISCVzhandlezycherizycapzyexception]{#2}}{}%
  \ifstrequal{#1}{handle\_cheri\_cap\_exception}{\hyperref[sailRISCVzhandlezycherizycapzyexception]{#2}}{}%
  \ifstrequal{#1}{handle_cheri_pcc_exception}{\hyperref[sailRISCVzhandlezycherizypcczyexception]{#2}}{}%
  \ifstrequal{#1}{handle\_cheri\_pcc\_exception}{\hyperref[sailRISCVzhandlezycherizypcczyexception]{#2}}{}%
  \ifstrequal{#1}{handle_cheri_reg_exception}{\hyperref[sailRISCVzhandlezycherizyregzyexception]{#2}}{}%
  \ifstrequal{#1}{handle\_cheri\_reg\_exception}{\hyperref[sailRISCVzhandlezycherizyregzyexception]{#2}}{}%
  \ifstrequal{#1}{handle_exception}{\hyperref[sailRISCVzhandlezyexception]{#2}}{}%
  \ifstrequal{#1}{handle\_exception}{\hyperref[sailRISCVzhandlezyexception]{#2}}{}%
  \ifstrequal{#1}{handle_illegal}{\hyperref[sailRISCVzhandlezyillegal]{#2}}{}%
  \ifstrequal{#1}{handle\_illegal}{\hyperref[sailRISCVzhandlezyillegal]{#2}}{}%
  \ifstrequal{#1}{handle_interrupt}{\hyperref[sailRISCVzhandlezyinterrupt]{#2}}{}%
  \ifstrequal{#1}{handle\_interrupt}{\hyperref[sailRISCVzhandlezyinterrupt]{#2}}{}%
  \ifstrequal{#1}{handle_load_cap_via_cap}{\hyperref[sailRISCVzhandlezyloadzycapzyviazycap]{#2}}{}%
  \ifstrequal{#1}{handle\_load\_cap\_via\_cap}{\hyperref[sailRISCVzhandlezyloadzycapzyviazycap]{#2}}{}%
  \ifstrequal{#1}{handle_load_data_via_cap}{\hyperref[sailRISCVzhandlezyloadzydatazyviazycap]{#2}}{}%
  \ifstrequal{#1}{handle\_load\_data\_via\_cap}{\hyperref[sailRISCVzhandlezyloadzydatazyviazycap]{#2}}{}%
  \ifstrequal{#1}{handle_loadres_cap_via_cap}{\hyperref[sailRISCVzhandlezyloadreszycapzyviazycap]{#2}}{}%
  \ifstrequal{#1}{handle\_loadres\_cap\_via\_cap}{\hyperref[sailRISCVzhandlezyloadreszycapzyviazycap]{#2}}{}%
  \ifstrequal{#1}{handle_loadres_data_via_cap}{\hyperref[sailRISCVzhandlezyloadreszydatazyviazycap]{#2}}{}%
  \ifstrequal{#1}{handle\_loadres\_data\_via\_cap}{\hyperref[sailRISCVzhandlezyloadreszydatazyviazycap]{#2}}{}%
  \ifstrequal{#1}{handle_mem_exception}{\hyperref[sailRISCVzhandlezymemzyexception]{#2}}{}%
  \ifstrequal{#1}{handle\_mem\_exception}{\hyperref[sailRISCVzhandlezymemzyexception]{#2}}{}%
  \ifstrequal{#1}{handle_store_cap_via_cap}{\hyperref[sailRISCVzhandlezystorezycapzyviazycap]{#2}}{}%
  \ifstrequal{#1}{handle\_store\_cap\_via\_cap}{\hyperref[sailRISCVzhandlezystorezycapzyviazycap]{#2}}{}%
  \ifstrequal{#1}{handle_store_cond_cap_via_cap}{\hyperref[sailRISCVzhandlezystorezycondzycapzyviazycap]{#2}}{}%
  \ifstrequal{#1}{handle\_store\_cond\_cap\_via\_cap}{\hyperref[sailRISCVzhandlezystorezycondzycapzyviazycap]{#2}}{}%
  \ifstrequal{#1}{handle_store_cond_data_via_cap}{\hyperref[sailRISCVzhandlezystorezycondzydatazyviazycap]{#2}}{}%
  \ifstrequal{#1}{handle\_store\_cond\_data\_via\_cap}{\hyperref[sailRISCVzhandlezystorezycondzydatazyviazycap]{#2}}{}%
  \ifstrequal{#1}{handle_store_data_via_cap}{\hyperref[sailRISCVzhandlezystorezydatazyviazycap]{#2}}{}%
  \ifstrequal{#1}{handle\_store\_data\_via\_cap}{\hyperref[sailRISCVzhandlezystorezydatazyviazycap]{#2}}{}%
  \ifstrequal{#1}{handle_trap_extension}{\hyperref[sailRISCVzhandlezytrapzyextension]{#2}}{}%
  \ifstrequal{#1}{handle\_trap\_extension}{\hyperref[sailRISCVzhandlezytrapzyextension]{#2}}{}%
  \ifstrequal{#1}{hasReservedOType}{\hyperref[sailRISCVzhasReservedOType]{#2}}{}%
  \ifstrequal{#1}{haveAtomics}{\hyperref[sailRISCVzhaveAtomics]{#2}}{}%
  \ifstrequal{#1}{haveDExt}{\hyperref[sailRISCVzhaveDExt]{#2}}{}%
  \ifstrequal{#1}{haveDoubleFPU}{\hyperref[sailRISCVzhaveDoubleFPU]{#2}}{}%
  \ifstrequal{#1}{haveFExt}{\hyperref[sailRISCVzhaveFExt]{#2}}{}%
  \ifstrequal{#1}{haveMulDiv}{\hyperref[sailRISCVzhaveMulDiv]{#2}}{}%
  \ifstrequal{#1}{haveNExt}{\hyperref[sailRISCVzhaveNExt]{#2}}{}%
  \ifstrequal{#1}{haveRVC}{\hyperref[sailRISCVzhaveRVC]{#2}}{}%
  \ifstrequal{#1}{haveSingleFPU}{\hyperref[sailRISCVzhaveSingleFPU]{#2}}{}%
  \ifstrequal{#1}{haveSupMode}{\hyperref[sailRISCVzhaveSupMode]{#2}}{}%
  \ifstrequal{#1}{haveUsrMode}{\hyperref[sailRISCVzhaveUsrMode]{#2}}{}%
  \ifstrequal{#1}{haveXcheri}{\hyperref[sailRISCVzhaveXcheri]{#2}}{}%
  \ifstrequal{#1}{haveZba}{\hyperref[sailRISCVzhaveZba]{#2}}{}%
  \ifstrequal{#1}{haveZbb}{\hyperref[sailRISCVzhaveZbb]{#2}}{}%
  \ifstrequal{#1}{haveZbc}{\hyperref[sailRISCVzhaveZbc]{#2}}{}%
  \ifstrequal{#1}{haveZbkb}{\hyperref[sailRISCVzhaveZbkb]{#2}}{}%
  \ifstrequal{#1}{haveZbkc}{\hyperref[sailRISCVzhaveZbkc]{#2}}{}%
  \ifstrequal{#1}{haveZbkx}{\hyperref[sailRISCVzhaveZbkx]{#2}}{}%
  \ifstrequal{#1}{haveZbs}{\hyperref[sailRISCVzhaveZbs]{#2}}{}%
  \ifstrequal{#1}{haveZdinx}{\hyperref[sailRISCVzhaveZdinx]{#2}}{}%
  \ifstrequal{#1}{haveZfh}{\hyperref[sailRISCVzhaveZfh]{#2}}{}%
  \ifstrequal{#1}{haveZfinx}{\hyperref[sailRISCVzhaveZfinx]{#2}}{}%
  \ifstrequal{#1}{haveZhinx}{\hyperref[sailRISCVzhaveZhinx]{#2}}{}%
  \ifstrequal{#1}{haveZknd}{\hyperref[sailRISCVzhaveZknd]{#2}}{}%
  \ifstrequal{#1}{haveZkne}{\hyperref[sailRISCVzhaveZkne]{#2}}{}%
  \ifstrequal{#1}{haveZknh}{\hyperref[sailRISCVzhaveZknh]{#2}}{}%
  \ifstrequal{#1}{haveZkr}{\hyperref[sailRISCVzhaveZkr]{#2}}{}%
  \ifstrequal{#1}{haveZksed}{\hyperref[sailRISCVzhaveZksed]{#2}}{}%
  \ifstrequal{#1}{haveZksh}{\hyperref[sailRISCVzhaveZksh]{#2}}{}%
  \ifstrequal{#1}{haveZmmul}{\hyperref[sailRISCVzhaveZmmul]{#2}}{}%
  \ifstrequal{#1}{hex_bits}{\hyperref[sailRISCVzhexzybits]{#2}}{}%
  \ifstrequal{#1}{hex\_bits}{\hyperref[sailRISCVzhexzybits]{#2}}{}%
  \ifstrequal{#1}{hex_bits_1}{\hyperref[sailRISCVzhexzybitszy1]{#2}}{}%
  \ifstrequal{#1}{hex\_bits\_1}{\hyperref[sailRISCVzhexzybitszy1]{#2}}{}%
  \ifstrequal{#1}{hex_bits_10}{\hyperref[sailRISCVzhexzybitszy10]{#2}}{}%
  \ifstrequal{#1}{hex\_bits\_10}{\hyperref[sailRISCVzhexzybitszy10]{#2}}{}%
  \ifstrequal{#1}{hex_bits_10_backwards}{\hyperref[sailRISCVzhexzybitszy10zybackwards]{#2}}{}%
  \ifstrequal{#1}{hex\_bits\_10\_backwards}{\hyperref[sailRISCVzhexzybitszy10zybackwards]{#2}}{}%
  \ifstrequal{#1}{hex_bits_10_backwards_matches}{\hyperref[sailRISCVzhexzybitszy10zybackwardszymatches]{#2}}{}%
  \ifstrequal{#1}{hex\_bits\_10\_backwards\_matches}{\hyperref[sailRISCVzhexzybitszy10zybackwardszymatches]{#2}}{}%
  \ifstrequal{#1}{hex_bits_10_forwards}{\hyperref[sailRISCVzhexzybitszy10zyforwards]{#2}}{}%
  \ifstrequal{#1}{hex\_bits\_10\_forwards}{\hyperref[sailRISCVzhexzybitszy10zyforwards]{#2}}{}%
  \ifstrequal{#1}{hex_bits_10_forwards_matches}{\hyperref[sailRISCVzhexzybitszy10zyforwardszymatches]{#2}}{}%
  \ifstrequal{#1}{hex\_bits\_10\_forwards\_matches}{\hyperref[sailRISCVzhexzybitszy10zyforwardszymatches]{#2}}{}%
  \ifstrequal{#1}{hex_bits_10_matches_prefix}{\hyperref[sailRISCVzhexzybitszy10zymatcheszyprefix]{#2}}{}%
  \ifstrequal{#1}{hex\_bits\_10\_matches\_prefix}{\hyperref[sailRISCVzhexzybitszy10zymatcheszyprefix]{#2}}{}%
  \ifstrequal{#1}{hex_bits_11}{\hyperref[sailRISCVzhexzybitszy11]{#2}}{}%
  \ifstrequal{#1}{hex\_bits\_11}{\hyperref[sailRISCVzhexzybitszy11]{#2}}{}%
  \ifstrequal{#1}{hex_bits_11_backwards}{\hyperref[sailRISCVzhexzybitszy11zybackwards]{#2}}{}%
  \ifstrequal{#1}{hex\_bits\_11\_backwards}{\hyperref[sailRISCVzhexzybitszy11zybackwards]{#2}}{}%
  \ifstrequal{#1}{hex_bits_11_backwards_matches}{\hyperref[sailRISCVzhexzybitszy11zybackwardszymatches]{#2}}{}%
  \ifstrequal{#1}{hex\_bits\_11\_backwards\_matches}{\hyperref[sailRISCVzhexzybitszy11zybackwardszymatches]{#2}}{}%
  \ifstrequal{#1}{hex_bits_11_forwards}{\hyperref[sailRISCVzhexzybitszy11zyforwards]{#2}}{}%
  \ifstrequal{#1}{hex\_bits\_11\_forwards}{\hyperref[sailRISCVzhexzybitszy11zyforwards]{#2}}{}%
  \ifstrequal{#1}{hex_bits_11_forwards_matches}{\hyperref[sailRISCVzhexzybitszy11zyforwardszymatches]{#2}}{}%
  \ifstrequal{#1}{hex\_bits\_11\_forwards\_matches}{\hyperref[sailRISCVzhexzybitszy11zyforwardszymatches]{#2}}{}%
  \ifstrequal{#1}{hex_bits_11_matches_prefix}{\hyperref[sailRISCVzhexzybitszy11zymatcheszyprefix]{#2}}{}%
  \ifstrequal{#1}{hex\_bits\_11\_matches\_prefix}{\hyperref[sailRISCVzhexzybitszy11zymatcheszyprefix]{#2}}{}%
  \ifstrequal{#1}{hex_bits_12}{\hyperref[sailRISCVzhexzybitszy12]{#2}}{}%
  \ifstrequal{#1}{hex\_bits\_12}{\hyperref[sailRISCVzhexzybitszy12]{#2}}{}%
  \ifstrequal{#1}{hex_bits_12_backwards}{\hyperref[sailRISCVzhexzybitszy12zybackwards]{#2}}{}%
  \ifstrequal{#1}{hex\_bits\_12\_backwards}{\hyperref[sailRISCVzhexzybitszy12zybackwards]{#2}}{}%
  \ifstrequal{#1}{hex_bits_12_backwards_matches}{\hyperref[sailRISCVzhexzybitszy12zybackwardszymatches]{#2}}{}%
  \ifstrequal{#1}{hex\_bits\_12\_backwards\_matches}{\hyperref[sailRISCVzhexzybitszy12zybackwardszymatches]{#2}}{}%
  \ifstrequal{#1}{hex_bits_12_forwards}{\hyperref[sailRISCVzhexzybitszy12zyforwards]{#2}}{}%
  \ifstrequal{#1}{hex\_bits\_12\_forwards}{\hyperref[sailRISCVzhexzybitszy12zyforwards]{#2}}{}%
  \ifstrequal{#1}{hex_bits_12_forwards_matches}{\hyperref[sailRISCVzhexzybitszy12zyforwardszymatches]{#2}}{}%
  \ifstrequal{#1}{hex\_bits\_12\_forwards\_matches}{\hyperref[sailRISCVzhexzybitszy12zyforwardszymatches]{#2}}{}%
  \ifstrequal{#1}{hex_bits_13}{\hyperref[sailRISCVzhexzybitszy13]{#2}}{}%
  \ifstrequal{#1}{hex\_bits\_13}{\hyperref[sailRISCVzhexzybitszy13]{#2}}{}%
  \ifstrequal{#1}{hex_bits_13_backwards}{\hyperref[sailRISCVzhexzybitszy13zybackwards]{#2}}{}%
  \ifstrequal{#1}{hex\_bits\_13\_backwards}{\hyperref[sailRISCVzhexzybitszy13zybackwards]{#2}}{}%
  \ifstrequal{#1}{hex_bits_13_backwards_matches}{\hyperref[sailRISCVzhexzybitszy13zybackwardszymatches]{#2}}{}%
  \ifstrequal{#1}{hex\_bits\_13\_backwards\_matches}{\hyperref[sailRISCVzhexzybitszy13zybackwardszymatches]{#2}}{}%
  \ifstrequal{#1}{hex_bits_13_forwards}{\hyperref[sailRISCVzhexzybitszy13zyforwards]{#2}}{}%
  \ifstrequal{#1}{hex\_bits\_13\_forwards}{\hyperref[sailRISCVzhexzybitszy13zyforwards]{#2}}{}%
  \ifstrequal{#1}{hex_bits_13_forwards_matches}{\hyperref[sailRISCVzhexzybitszy13zyforwardszymatches]{#2}}{}%
  \ifstrequal{#1}{hex\_bits\_13\_forwards\_matches}{\hyperref[sailRISCVzhexzybitszy13zyforwardszymatches]{#2}}{}%
  \ifstrequal{#1}{hex_bits_13_matches_prefix}{\hyperref[sailRISCVzhexzybitszy13zymatcheszyprefix]{#2}}{}%
  \ifstrequal{#1}{hex\_bits\_13\_matches\_prefix}{\hyperref[sailRISCVzhexzybitszy13zymatcheszyprefix]{#2}}{}%
  \ifstrequal{#1}{hex_bits_14}{\hyperref[sailRISCVzhexzybitszy14]{#2}}{}%
  \ifstrequal{#1}{hex\_bits\_14}{\hyperref[sailRISCVzhexzybitszy14]{#2}}{}%
  \ifstrequal{#1}{hex_bits_14_backwards}{\hyperref[sailRISCVzhexzybitszy14zybackwards]{#2}}{}%
  \ifstrequal{#1}{hex\_bits\_14\_backwards}{\hyperref[sailRISCVzhexzybitszy14zybackwards]{#2}}{}%
  \ifstrequal{#1}{hex_bits_14_backwards_matches}{\hyperref[sailRISCVzhexzybitszy14zybackwardszymatches]{#2}}{}%
  \ifstrequal{#1}{hex\_bits\_14\_backwards\_matches}{\hyperref[sailRISCVzhexzybitszy14zybackwardszymatches]{#2}}{}%
  \ifstrequal{#1}{hex_bits_14_forwards}{\hyperref[sailRISCVzhexzybitszy14zyforwards]{#2}}{}%
  \ifstrequal{#1}{hex\_bits\_14\_forwards}{\hyperref[sailRISCVzhexzybitszy14zyforwards]{#2}}{}%
  \ifstrequal{#1}{hex_bits_14_forwards_matches}{\hyperref[sailRISCVzhexzybitszy14zyforwardszymatches]{#2}}{}%
  \ifstrequal{#1}{hex\_bits\_14\_forwards\_matches}{\hyperref[sailRISCVzhexzybitszy14zyforwardszymatches]{#2}}{}%
  \ifstrequal{#1}{hex_bits_14_matches_prefix}{\hyperref[sailRISCVzhexzybitszy14zymatcheszyprefix]{#2}}{}%
  \ifstrequal{#1}{hex\_bits\_14\_matches\_prefix}{\hyperref[sailRISCVzhexzybitszy14zymatcheszyprefix]{#2}}{}%
  \ifstrequal{#1}{hex_bits_15}{\hyperref[sailRISCVzhexzybitszy15]{#2}}{}%
  \ifstrequal{#1}{hex\_bits\_15}{\hyperref[sailRISCVzhexzybitszy15]{#2}}{}%
  \ifstrequal{#1}{hex_bits_15_backwards}{\hyperref[sailRISCVzhexzybitszy15zybackwards]{#2}}{}%
  \ifstrequal{#1}{hex\_bits\_15\_backwards}{\hyperref[sailRISCVzhexzybitszy15zybackwards]{#2}}{}%
  \ifstrequal{#1}{hex_bits_15_backwards_matches}{\hyperref[sailRISCVzhexzybitszy15zybackwardszymatches]{#2}}{}%
  \ifstrequal{#1}{hex\_bits\_15\_backwards\_matches}{\hyperref[sailRISCVzhexzybitszy15zybackwardszymatches]{#2}}{}%
  \ifstrequal{#1}{hex_bits_15_forwards}{\hyperref[sailRISCVzhexzybitszy15zyforwards]{#2}}{}%
  \ifstrequal{#1}{hex\_bits\_15\_forwards}{\hyperref[sailRISCVzhexzybitszy15zyforwards]{#2}}{}%
  \ifstrequal{#1}{hex_bits_15_forwards_matches}{\hyperref[sailRISCVzhexzybitszy15zyforwardszymatches]{#2}}{}%
  \ifstrequal{#1}{hex\_bits\_15\_forwards\_matches}{\hyperref[sailRISCVzhexzybitszy15zyforwardszymatches]{#2}}{}%
  \ifstrequal{#1}{hex_bits_15_matches_prefix}{\hyperref[sailRISCVzhexzybitszy15zymatcheszyprefix]{#2}}{}%
  \ifstrequal{#1}{hex\_bits\_15\_matches\_prefix}{\hyperref[sailRISCVzhexzybitszy15zymatcheszyprefix]{#2}}{}%
  \ifstrequal{#1}{hex_bits_16}{\hyperref[sailRISCVzhexzybitszy16]{#2}}{}%
  \ifstrequal{#1}{hex\_bits\_16}{\hyperref[sailRISCVzhexzybitszy16]{#2}}{}%
  \ifstrequal{#1}{hex_bits_16_backwards}{\hyperref[sailRISCVzhexzybitszy16zybackwards]{#2}}{}%
  \ifstrequal{#1}{hex\_bits\_16\_backwards}{\hyperref[sailRISCVzhexzybitszy16zybackwards]{#2}}{}%
  \ifstrequal{#1}{hex_bits_16_backwards_matches}{\hyperref[sailRISCVzhexzybitszy16zybackwardszymatches]{#2}}{}%
  \ifstrequal{#1}{hex\_bits\_16\_backwards\_matches}{\hyperref[sailRISCVzhexzybitszy16zybackwardszymatches]{#2}}{}%
  \ifstrequal{#1}{hex_bits_16_forwards}{\hyperref[sailRISCVzhexzybitszy16zyforwards]{#2}}{}%
  \ifstrequal{#1}{hex\_bits\_16\_forwards}{\hyperref[sailRISCVzhexzybitszy16zyforwards]{#2}}{}%
  \ifstrequal{#1}{hex_bits_16_forwards_matches}{\hyperref[sailRISCVzhexzybitszy16zyforwardszymatches]{#2}}{}%
  \ifstrequal{#1}{hex\_bits\_16\_forwards\_matches}{\hyperref[sailRISCVzhexzybitszy16zyforwardszymatches]{#2}}{}%
  \ifstrequal{#1}{hex_bits_16_matches_prefix}{\hyperref[sailRISCVzhexzybitszy16zymatcheszyprefix]{#2}}{}%
  \ifstrequal{#1}{hex\_bits\_16\_matches\_prefix}{\hyperref[sailRISCVzhexzybitszy16zymatcheszyprefix]{#2}}{}%
  \ifstrequal{#1}{hex_bits_17}{\hyperref[sailRISCVzhexzybitszy17]{#2}}{}%
  \ifstrequal{#1}{hex\_bits\_17}{\hyperref[sailRISCVzhexzybitszy17]{#2}}{}%
  \ifstrequal{#1}{hex_bits_17_backwards}{\hyperref[sailRISCVzhexzybitszy17zybackwards]{#2}}{}%
  \ifstrequal{#1}{hex\_bits\_17\_backwards}{\hyperref[sailRISCVzhexzybitszy17zybackwards]{#2}}{}%
  \ifstrequal{#1}{hex_bits_17_backwards_matches}{\hyperref[sailRISCVzhexzybitszy17zybackwardszymatches]{#2}}{}%
  \ifstrequal{#1}{hex\_bits\_17\_backwards\_matches}{\hyperref[sailRISCVzhexzybitszy17zybackwardszymatches]{#2}}{}%
  \ifstrequal{#1}{hex_bits_17_forwards}{\hyperref[sailRISCVzhexzybitszy17zyforwards]{#2}}{}%
  \ifstrequal{#1}{hex\_bits\_17\_forwards}{\hyperref[sailRISCVzhexzybitszy17zyforwards]{#2}}{}%
  \ifstrequal{#1}{hex_bits_17_forwards_matches}{\hyperref[sailRISCVzhexzybitszy17zyforwardszymatches]{#2}}{}%
  \ifstrequal{#1}{hex\_bits\_17\_forwards\_matches}{\hyperref[sailRISCVzhexzybitszy17zyforwardszymatches]{#2}}{}%
  \ifstrequal{#1}{hex_bits_17_matches_prefix}{\hyperref[sailRISCVzhexzybitszy17zymatcheszyprefix]{#2}}{}%
  \ifstrequal{#1}{hex\_bits\_17\_matches\_prefix}{\hyperref[sailRISCVzhexzybitszy17zymatcheszyprefix]{#2}}{}%
  \ifstrequal{#1}{hex_bits_18}{\hyperref[sailRISCVzhexzybitszy18]{#2}}{}%
  \ifstrequal{#1}{hex\_bits\_18}{\hyperref[sailRISCVzhexzybitszy18]{#2}}{}%
  \ifstrequal{#1}{hex_bits_18_backwards}{\hyperref[sailRISCVzhexzybitszy18zybackwards]{#2}}{}%
  \ifstrequal{#1}{hex\_bits\_18\_backwards}{\hyperref[sailRISCVzhexzybitszy18zybackwards]{#2}}{}%
  \ifstrequal{#1}{hex_bits_18_backwards_matches}{\hyperref[sailRISCVzhexzybitszy18zybackwardszymatches]{#2}}{}%
  \ifstrequal{#1}{hex\_bits\_18\_backwards\_matches}{\hyperref[sailRISCVzhexzybitszy18zybackwardszymatches]{#2}}{}%
  \ifstrequal{#1}{hex_bits_18_forwards}{\hyperref[sailRISCVzhexzybitszy18zyforwards]{#2}}{}%
  \ifstrequal{#1}{hex\_bits\_18\_forwards}{\hyperref[sailRISCVzhexzybitszy18zyforwards]{#2}}{}%
  \ifstrequal{#1}{hex_bits_18_forwards_matches}{\hyperref[sailRISCVzhexzybitszy18zyforwardszymatches]{#2}}{}%
  \ifstrequal{#1}{hex\_bits\_18\_forwards\_matches}{\hyperref[sailRISCVzhexzybitszy18zyforwardszymatches]{#2}}{}%
  \ifstrequal{#1}{hex_bits_18_matches_prefix}{\hyperref[sailRISCVzhexzybitszy18zymatcheszyprefix]{#2}}{}%
  \ifstrequal{#1}{hex\_bits\_18\_matches\_prefix}{\hyperref[sailRISCVzhexzybitszy18zymatcheszyprefix]{#2}}{}%
  \ifstrequal{#1}{hex_bits_19}{\hyperref[sailRISCVzhexzybitszy19]{#2}}{}%
  \ifstrequal{#1}{hex\_bits\_19}{\hyperref[sailRISCVzhexzybitszy19]{#2}}{}%
  \ifstrequal{#1}{hex_bits_19_backwards}{\hyperref[sailRISCVzhexzybitszy19zybackwards]{#2}}{}%
  \ifstrequal{#1}{hex\_bits\_19\_backwards}{\hyperref[sailRISCVzhexzybitszy19zybackwards]{#2}}{}%
  \ifstrequal{#1}{hex_bits_19_backwards_matches}{\hyperref[sailRISCVzhexzybitszy19zybackwardszymatches]{#2}}{}%
  \ifstrequal{#1}{hex\_bits\_19\_backwards\_matches}{\hyperref[sailRISCVzhexzybitszy19zybackwardszymatches]{#2}}{}%
  \ifstrequal{#1}{hex_bits_19_forwards}{\hyperref[sailRISCVzhexzybitszy19zyforwards]{#2}}{}%
  \ifstrequal{#1}{hex\_bits\_19\_forwards}{\hyperref[sailRISCVzhexzybitszy19zyforwards]{#2}}{}%
  \ifstrequal{#1}{hex_bits_19_forwards_matches}{\hyperref[sailRISCVzhexzybitszy19zyforwardszymatches]{#2}}{}%
  \ifstrequal{#1}{hex\_bits\_19\_forwards\_matches}{\hyperref[sailRISCVzhexzybitszy19zyforwardszymatches]{#2}}{}%
  \ifstrequal{#1}{hex_bits_19_matches_prefix}{\hyperref[sailRISCVzhexzybitszy19zymatcheszyprefix]{#2}}{}%
  \ifstrequal{#1}{hex\_bits\_19\_matches\_prefix}{\hyperref[sailRISCVzhexzybitszy19zymatcheszyprefix]{#2}}{}%
  \ifstrequal{#1}{hex_bits_1_backwards}{\hyperref[sailRISCVzhexzybitszy1zybackwards]{#2}}{}%
  \ifstrequal{#1}{hex\_bits\_1\_backwards}{\hyperref[sailRISCVzhexzybitszy1zybackwards]{#2}}{}%
  \ifstrequal{#1}{hex_bits_1_backwards_matches}{\hyperref[sailRISCVzhexzybitszy1zybackwardszymatches]{#2}}{}%
  \ifstrequal{#1}{hex\_bits\_1\_backwards\_matches}{\hyperref[sailRISCVzhexzybitszy1zybackwardszymatches]{#2}}{}%
  \ifstrequal{#1}{hex_bits_1_forwards}{\hyperref[sailRISCVzhexzybitszy1zyforwards]{#2}}{}%
  \ifstrequal{#1}{hex\_bits\_1\_forwards}{\hyperref[sailRISCVzhexzybitszy1zyforwards]{#2}}{}%
  \ifstrequal{#1}{hex_bits_1_forwards_matches}{\hyperref[sailRISCVzhexzybitszy1zyforwardszymatches]{#2}}{}%
  \ifstrequal{#1}{hex\_bits\_1\_forwards\_matches}{\hyperref[sailRISCVzhexzybitszy1zyforwardszymatches]{#2}}{}%
  \ifstrequal{#1}{hex_bits_1_matches_prefix}{\hyperref[sailRISCVzhexzybitszy1zymatcheszyprefix]{#2}}{}%
  \ifstrequal{#1}{hex\_bits\_1\_matches\_prefix}{\hyperref[sailRISCVzhexzybitszy1zymatcheszyprefix]{#2}}{}%
  \ifstrequal{#1}{hex_bits_2}{\hyperref[sailRISCVzhexzybitszy2]{#2}}{}%
  \ifstrequal{#1}{hex\_bits\_2}{\hyperref[sailRISCVzhexzybitszy2]{#2}}{}%
  \ifstrequal{#1}{hex_bits_20}{\hyperref[sailRISCVzhexzybitszy20]{#2}}{}%
  \ifstrequal{#1}{hex\_bits\_20}{\hyperref[sailRISCVzhexzybitszy20]{#2}}{}%
  \ifstrequal{#1}{hex_bits_20_backwards}{\hyperref[sailRISCVzhexzybitszy20zybackwards]{#2}}{}%
  \ifstrequal{#1}{hex\_bits\_20\_backwards}{\hyperref[sailRISCVzhexzybitszy20zybackwards]{#2}}{}%
  \ifstrequal{#1}{hex_bits_20_backwards_matches}{\hyperref[sailRISCVzhexzybitszy20zybackwardszymatches]{#2}}{}%
  \ifstrequal{#1}{hex\_bits\_20\_backwards\_matches}{\hyperref[sailRISCVzhexzybitszy20zybackwardszymatches]{#2}}{}%
  \ifstrequal{#1}{hex_bits_20_forwards}{\hyperref[sailRISCVzhexzybitszy20zyforwards]{#2}}{}%
  \ifstrequal{#1}{hex\_bits\_20\_forwards}{\hyperref[sailRISCVzhexzybitszy20zyforwards]{#2}}{}%
  \ifstrequal{#1}{hex_bits_20_forwards_matches}{\hyperref[sailRISCVzhexzybitszy20zyforwardszymatches]{#2}}{}%
  \ifstrequal{#1}{hex\_bits\_20\_forwards\_matches}{\hyperref[sailRISCVzhexzybitszy20zyforwardszymatches]{#2}}{}%
  \ifstrequal{#1}{hex_bits_20_matches_prefix}{\hyperref[sailRISCVzhexzybitszy20zymatcheszyprefix]{#2}}{}%
  \ifstrequal{#1}{hex\_bits\_20\_matches\_prefix}{\hyperref[sailRISCVzhexzybitszy20zymatcheszyprefix]{#2}}{}%
  \ifstrequal{#1}{hex_bits_21}{\hyperref[sailRISCVzhexzybitszy21]{#2}}{}%
  \ifstrequal{#1}{hex\_bits\_21}{\hyperref[sailRISCVzhexzybitszy21]{#2}}{}%
  \ifstrequal{#1}{hex_bits_21_backwards}{\hyperref[sailRISCVzhexzybitszy21zybackwards]{#2}}{}%
  \ifstrequal{#1}{hex\_bits\_21\_backwards}{\hyperref[sailRISCVzhexzybitszy21zybackwards]{#2}}{}%
  \ifstrequal{#1}{hex_bits_21_backwards_matches}{\hyperref[sailRISCVzhexzybitszy21zybackwardszymatches]{#2}}{}%
  \ifstrequal{#1}{hex\_bits\_21\_backwards\_matches}{\hyperref[sailRISCVzhexzybitszy21zybackwardszymatches]{#2}}{}%
  \ifstrequal{#1}{hex_bits_21_forwards}{\hyperref[sailRISCVzhexzybitszy21zyforwards]{#2}}{}%
  \ifstrequal{#1}{hex\_bits\_21\_forwards}{\hyperref[sailRISCVzhexzybitszy21zyforwards]{#2}}{}%
  \ifstrequal{#1}{hex_bits_21_forwards_matches}{\hyperref[sailRISCVzhexzybitszy21zyforwardszymatches]{#2}}{}%
  \ifstrequal{#1}{hex\_bits\_21\_forwards\_matches}{\hyperref[sailRISCVzhexzybitszy21zyforwardszymatches]{#2}}{}%
  \ifstrequal{#1}{hex_bits_21_matches_prefix}{\hyperref[sailRISCVzhexzybitszy21zymatcheszyprefix]{#2}}{}%
  \ifstrequal{#1}{hex\_bits\_21\_matches\_prefix}{\hyperref[sailRISCVzhexzybitszy21zymatcheszyprefix]{#2}}{}%
  \ifstrequal{#1}{hex_bits_22}{\hyperref[sailRISCVzhexzybitszy22]{#2}}{}%
  \ifstrequal{#1}{hex\_bits\_22}{\hyperref[sailRISCVzhexzybitszy22]{#2}}{}%
  \ifstrequal{#1}{hex_bits_22_backwards}{\hyperref[sailRISCVzhexzybitszy22zybackwards]{#2}}{}%
  \ifstrequal{#1}{hex\_bits\_22\_backwards}{\hyperref[sailRISCVzhexzybitszy22zybackwards]{#2}}{}%
  \ifstrequal{#1}{hex_bits_22_backwards_matches}{\hyperref[sailRISCVzhexzybitszy22zybackwardszymatches]{#2}}{}%
  \ifstrequal{#1}{hex\_bits\_22\_backwards\_matches}{\hyperref[sailRISCVzhexzybitszy22zybackwardszymatches]{#2}}{}%
  \ifstrequal{#1}{hex_bits_22_forwards}{\hyperref[sailRISCVzhexzybitszy22zyforwards]{#2}}{}%
  \ifstrequal{#1}{hex\_bits\_22\_forwards}{\hyperref[sailRISCVzhexzybitszy22zyforwards]{#2}}{}%
  \ifstrequal{#1}{hex_bits_22_forwards_matches}{\hyperref[sailRISCVzhexzybitszy22zyforwardszymatches]{#2}}{}%
  \ifstrequal{#1}{hex\_bits\_22\_forwards\_matches}{\hyperref[sailRISCVzhexzybitszy22zyforwardszymatches]{#2}}{}%
  \ifstrequal{#1}{hex_bits_22_matches_prefix}{\hyperref[sailRISCVzhexzybitszy22zymatcheszyprefix]{#2}}{}%
  \ifstrequal{#1}{hex\_bits\_22\_matches\_prefix}{\hyperref[sailRISCVzhexzybitszy22zymatcheszyprefix]{#2}}{}%
  \ifstrequal{#1}{hex_bits_23}{\hyperref[sailRISCVzhexzybitszy23]{#2}}{}%
  \ifstrequal{#1}{hex\_bits\_23}{\hyperref[sailRISCVzhexzybitszy23]{#2}}{}%
  \ifstrequal{#1}{hex_bits_23_backwards}{\hyperref[sailRISCVzhexzybitszy23zybackwards]{#2}}{}%
  \ifstrequal{#1}{hex\_bits\_23\_backwards}{\hyperref[sailRISCVzhexzybitszy23zybackwards]{#2}}{}%
  \ifstrequal{#1}{hex_bits_23_backwards_matches}{\hyperref[sailRISCVzhexzybitszy23zybackwardszymatches]{#2}}{}%
  \ifstrequal{#1}{hex\_bits\_23\_backwards\_matches}{\hyperref[sailRISCVzhexzybitszy23zybackwardszymatches]{#2}}{}%
  \ifstrequal{#1}{hex_bits_23_forwards}{\hyperref[sailRISCVzhexzybitszy23zyforwards]{#2}}{}%
  \ifstrequal{#1}{hex\_bits\_23\_forwards}{\hyperref[sailRISCVzhexzybitszy23zyforwards]{#2}}{}%
  \ifstrequal{#1}{hex_bits_23_forwards_matches}{\hyperref[sailRISCVzhexzybitszy23zyforwardszymatches]{#2}}{}%
  \ifstrequal{#1}{hex\_bits\_23\_forwards\_matches}{\hyperref[sailRISCVzhexzybitszy23zyforwardszymatches]{#2}}{}%
  \ifstrequal{#1}{hex_bits_23_matches_prefix}{\hyperref[sailRISCVzhexzybitszy23zymatcheszyprefix]{#2}}{}%
  \ifstrequal{#1}{hex\_bits\_23\_matches\_prefix}{\hyperref[sailRISCVzhexzybitszy23zymatcheszyprefix]{#2}}{}%
  \ifstrequal{#1}{hex_bits_24}{\hyperref[sailRISCVzhexzybitszy24]{#2}}{}%
  \ifstrequal{#1}{hex\_bits\_24}{\hyperref[sailRISCVzhexzybitszy24]{#2}}{}%
  \ifstrequal{#1}{hex_bits_24_backwards}{\hyperref[sailRISCVzhexzybitszy24zybackwards]{#2}}{}%
  \ifstrequal{#1}{hex\_bits\_24\_backwards}{\hyperref[sailRISCVzhexzybitszy24zybackwards]{#2}}{}%
  \ifstrequal{#1}{hex_bits_24_backwards_matches}{\hyperref[sailRISCVzhexzybitszy24zybackwardszymatches]{#2}}{}%
  \ifstrequal{#1}{hex\_bits\_24\_backwards\_matches}{\hyperref[sailRISCVzhexzybitszy24zybackwardszymatches]{#2}}{}%
  \ifstrequal{#1}{hex_bits_24_forwards}{\hyperref[sailRISCVzhexzybitszy24zyforwards]{#2}}{}%
  \ifstrequal{#1}{hex\_bits\_24\_forwards}{\hyperref[sailRISCVzhexzybitszy24zyforwards]{#2}}{}%
  \ifstrequal{#1}{hex_bits_24_forwards_matches}{\hyperref[sailRISCVzhexzybitszy24zyforwardszymatches]{#2}}{}%
  \ifstrequal{#1}{hex\_bits\_24\_forwards\_matches}{\hyperref[sailRISCVzhexzybitszy24zyforwardszymatches]{#2}}{}%
  \ifstrequal{#1}{hex_bits_24_matches_prefix}{\hyperref[sailRISCVzhexzybitszy24zymatcheszyprefix]{#2}}{}%
  \ifstrequal{#1}{hex\_bits\_24\_matches\_prefix}{\hyperref[sailRISCVzhexzybitszy24zymatcheszyprefix]{#2}}{}%
  \ifstrequal{#1}{hex_bits_25}{\hyperref[sailRISCVzhexzybitszy25]{#2}}{}%
  \ifstrequal{#1}{hex\_bits\_25}{\hyperref[sailRISCVzhexzybitszy25]{#2}}{}%
  \ifstrequal{#1}{hex_bits_25_backwards}{\hyperref[sailRISCVzhexzybitszy25zybackwards]{#2}}{}%
  \ifstrequal{#1}{hex\_bits\_25\_backwards}{\hyperref[sailRISCVzhexzybitszy25zybackwards]{#2}}{}%
  \ifstrequal{#1}{hex_bits_25_backwards_matches}{\hyperref[sailRISCVzhexzybitszy25zybackwardszymatches]{#2}}{}%
  \ifstrequal{#1}{hex\_bits\_25\_backwards\_matches}{\hyperref[sailRISCVzhexzybitszy25zybackwardszymatches]{#2}}{}%
  \ifstrequal{#1}{hex_bits_25_forwards}{\hyperref[sailRISCVzhexzybitszy25zyforwards]{#2}}{}%
  \ifstrequal{#1}{hex\_bits\_25\_forwards}{\hyperref[sailRISCVzhexzybitszy25zyforwards]{#2}}{}%
  \ifstrequal{#1}{hex_bits_25_forwards_matches}{\hyperref[sailRISCVzhexzybitszy25zyforwardszymatches]{#2}}{}%
  \ifstrequal{#1}{hex\_bits\_25\_forwards\_matches}{\hyperref[sailRISCVzhexzybitszy25zyforwardszymatches]{#2}}{}%
  \ifstrequal{#1}{hex_bits_25_matches_prefix}{\hyperref[sailRISCVzhexzybitszy25zymatcheszyprefix]{#2}}{}%
  \ifstrequal{#1}{hex\_bits\_25\_matches\_prefix}{\hyperref[sailRISCVzhexzybitszy25zymatcheszyprefix]{#2}}{}%
  \ifstrequal{#1}{hex_bits_26}{\hyperref[sailRISCVzhexzybitszy26]{#2}}{}%
  \ifstrequal{#1}{hex\_bits\_26}{\hyperref[sailRISCVzhexzybitszy26]{#2}}{}%
  \ifstrequal{#1}{hex_bits_26_backwards}{\hyperref[sailRISCVzhexzybitszy26zybackwards]{#2}}{}%
  \ifstrequal{#1}{hex\_bits\_26\_backwards}{\hyperref[sailRISCVzhexzybitszy26zybackwards]{#2}}{}%
  \ifstrequal{#1}{hex_bits_26_backwards_matches}{\hyperref[sailRISCVzhexzybitszy26zybackwardszymatches]{#2}}{}%
  \ifstrequal{#1}{hex\_bits\_26\_backwards\_matches}{\hyperref[sailRISCVzhexzybitszy26zybackwardszymatches]{#2}}{}%
  \ifstrequal{#1}{hex_bits_26_forwards}{\hyperref[sailRISCVzhexzybitszy26zyforwards]{#2}}{}%
  \ifstrequal{#1}{hex\_bits\_26\_forwards}{\hyperref[sailRISCVzhexzybitszy26zyforwards]{#2}}{}%
  \ifstrequal{#1}{hex_bits_26_forwards_matches}{\hyperref[sailRISCVzhexzybitszy26zyforwardszymatches]{#2}}{}%
  \ifstrequal{#1}{hex\_bits\_26\_forwards\_matches}{\hyperref[sailRISCVzhexzybitszy26zyforwardszymatches]{#2}}{}%
  \ifstrequal{#1}{hex_bits_26_matches_prefix}{\hyperref[sailRISCVzhexzybitszy26zymatcheszyprefix]{#2}}{}%
  \ifstrequal{#1}{hex\_bits\_26\_matches\_prefix}{\hyperref[sailRISCVzhexzybitszy26zymatcheszyprefix]{#2}}{}%
  \ifstrequal{#1}{hex_bits_27}{\hyperref[sailRISCVzhexzybitszy27]{#2}}{}%
  \ifstrequal{#1}{hex\_bits\_27}{\hyperref[sailRISCVzhexzybitszy27]{#2}}{}%
  \ifstrequal{#1}{hex_bits_27_backwards}{\hyperref[sailRISCVzhexzybitszy27zybackwards]{#2}}{}%
  \ifstrequal{#1}{hex\_bits\_27\_backwards}{\hyperref[sailRISCVzhexzybitszy27zybackwards]{#2}}{}%
  \ifstrequal{#1}{hex_bits_27_backwards_matches}{\hyperref[sailRISCVzhexzybitszy27zybackwardszymatches]{#2}}{}%
  \ifstrequal{#1}{hex\_bits\_27\_backwards\_matches}{\hyperref[sailRISCVzhexzybitszy27zybackwardszymatches]{#2}}{}%
  \ifstrequal{#1}{hex_bits_27_forwards}{\hyperref[sailRISCVzhexzybitszy27zyforwards]{#2}}{}%
  \ifstrequal{#1}{hex\_bits\_27\_forwards}{\hyperref[sailRISCVzhexzybitszy27zyforwards]{#2}}{}%
  \ifstrequal{#1}{hex_bits_27_forwards_matches}{\hyperref[sailRISCVzhexzybitszy27zyforwardszymatches]{#2}}{}%
  \ifstrequal{#1}{hex\_bits\_27\_forwards\_matches}{\hyperref[sailRISCVzhexzybitszy27zyforwardszymatches]{#2}}{}%
  \ifstrequal{#1}{hex_bits_27_matches_prefix}{\hyperref[sailRISCVzhexzybitszy27zymatcheszyprefix]{#2}}{}%
  \ifstrequal{#1}{hex\_bits\_27\_matches\_prefix}{\hyperref[sailRISCVzhexzybitszy27zymatcheszyprefix]{#2}}{}%
  \ifstrequal{#1}{hex_bits_28}{\hyperref[sailRISCVzhexzybitszy28]{#2}}{}%
  \ifstrequal{#1}{hex\_bits\_28}{\hyperref[sailRISCVzhexzybitszy28]{#2}}{}%
  \ifstrequal{#1}{hex_bits_28_backwards}{\hyperref[sailRISCVzhexzybitszy28zybackwards]{#2}}{}%
  \ifstrequal{#1}{hex\_bits\_28\_backwards}{\hyperref[sailRISCVzhexzybitszy28zybackwards]{#2}}{}%
  \ifstrequal{#1}{hex_bits_28_backwards_matches}{\hyperref[sailRISCVzhexzybitszy28zybackwardszymatches]{#2}}{}%
  \ifstrequal{#1}{hex\_bits\_28\_backwards\_matches}{\hyperref[sailRISCVzhexzybitszy28zybackwardszymatches]{#2}}{}%
  \ifstrequal{#1}{hex_bits_28_forwards}{\hyperref[sailRISCVzhexzybitszy28zyforwards]{#2}}{}%
  \ifstrequal{#1}{hex\_bits\_28\_forwards}{\hyperref[sailRISCVzhexzybitszy28zyforwards]{#2}}{}%
  \ifstrequal{#1}{hex_bits_28_forwards_matches}{\hyperref[sailRISCVzhexzybitszy28zyforwardszymatches]{#2}}{}%
  \ifstrequal{#1}{hex\_bits\_28\_forwards\_matches}{\hyperref[sailRISCVzhexzybitszy28zyforwardszymatches]{#2}}{}%
  \ifstrequal{#1}{hex_bits_28_matches_prefix}{\hyperref[sailRISCVzhexzybitszy28zymatcheszyprefix]{#2}}{}%
  \ifstrequal{#1}{hex\_bits\_28\_matches\_prefix}{\hyperref[sailRISCVzhexzybitszy28zymatcheszyprefix]{#2}}{}%
  \ifstrequal{#1}{hex_bits_29}{\hyperref[sailRISCVzhexzybitszy29]{#2}}{}%
  \ifstrequal{#1}{hex\_bits\_29}{\hyperref[sailRISCVzhexzybitszy29]{#2}}{}%
  \ifstrequal{#1}{hex_bits_29_backwards}{\hyperref[sailRISCVzhexzybitszy29zybackwards]{#2}}{}%
  \ifstrequal{#1}{hex\_bits\_29\_backwards}{\hyperref[sailRISCVzhexzybitszy29zybackwards]{#2}}{}%
  \ifstrequal{#1}{hex_bits_29_backwards_matches}{\hyperref[sailRISCVzhexzybitszy29zybackwardszymatches]{#2}}{}%
  \ifstrequal{#1}{hex\_bits\_29\_backwards\_matches}{\hyperref[sailRISCVzhexzybitszy29zybackwardszymatches]{#2}}{}%
  \ifstrequal{#1}{hex_bits_29_forwards}{\hyperref[sailRISCVzhexzybitszy29zyforwards]{#2}}{}%
  \ifstrequal{#1}{hex\_bits\_29\_forwards}{\hyperref[sailRISCVzhexzybitszy29zyforwards]{#2}}{}%
  \ifstrequal{#1}{hex_bits_29_forwards_matches}{\hyperref[sailRISCVzhexzybitszy29zyforwardszymatches]{#2}}{}%
  \ifstrequal{#1}{hex\_bits\_29\_forwards\_matches}{\hyperref[sailRISCVzhexzybitszy29zyforwardszymatches]{#2}}{}%
  \ifstrequal{#1}{hex_bits_29_matches_prefix}{\hyperref[sailRISCVzhexzybitszy29zymatcheszyprefix]{#2}}{}%
  \ifstrequal{#1}{hex\_bits\_29\_matches\_prefix}{\hyperref[sailRISCVzhexzybitszy29zymatcheszyprefix]{#2}}{}%
  \ifstrequal{#1}{hex_bits_2_backwards}{\hyperref[sailRISCVzhexzybitszy2zybackwards]{#2}}{}%
  \ifstrequal{#1}{hex\_bits\_2\_backwards}{\hyperref[sailRISCVzhexzybitszy2zybackwards]{#2}}{}%
  \ifstrequal{#1}{hex_bits_2_backwards_matches}{\hyperref[sailRISCVzhexzybitszy2zybackwardszymatches]{#2}}{}%
  \ifstrequal{#1}{hex\_bits\_2\_backwards\_matches}{\hyperref[sailRISCVzhexzybitszy2zybackwardszymatches]{#2}}{}%
  \ifstrequal{#1}{hex_bits_2_forwards}{\hyperref[sailRISCVzhexzybitszy2zyforwards]{#2}}{}%
  \ifstrequal{#1}{hex\_bits\_2\_forwards}{\hyperref[sailRISCVzhexzybitszy2zyforwards]{#2}}{}%
  \ifstrequal{#1}{hex_bits_2_forwards_matches}{\hyperref[sailRISCVzhexzybitszy2zyforwardszymatches]{#2}}{}%
  \ifstrequal{#1}{hex\_bits\_2\_forwards\_matches}{\hyperref[sailRISCVzhexzybitszy2zyforwardszymatches]{#2}}{}%
  \ifstrequal{#1}{hex_bits_2_matches_prefix}{\hyperref[sailRISCVzhexzybitszy2zymatcheszyprefix]{#2}}{}%
  \ifstrequal{#1}{hex\_bits\_2\_matches\_prefix}{\hyperref[sailRISCVzhexzybitszy2zymatcheszyprefix]{#2}}{}%
  \ifstrequal{#1}{hex_bits_3}{\hyperref[sailRISCVzhexzybitszy3]{#2}}{}%
  \ifstrequal{#1}{hex\_bits\_3}{\hyperref[sailRISCVzhexzybitszy3]{#2}}{}%
  \ifstrequal{#1}{hex_bits_30}{\hyperref[sailRISCVzhexzybitszy30]{#2}}{}%
  \ifstrequal{#1}{hex\_bits\_30}{\hyperref[sailRISCVzhexzybitszy30]{#2}}{}%
  \ifstrequal{#1}{hex_bits_30_backwards}{\hyperref[sailRISCVzhexzybitszy30zybackwards]{#2}}{}%
  \ifstrequal{#1}{hex\_bits\_30\_backwards}{\hyperref[sailRISCVzhexzybitszy30zybackwards]{#2}}{}%
  \ifstrequal{#1}{hex_bits_30_backwards_matches}{\hyperref[sailRISCVzhexzybitszy30zybackwardszymatches]{#2}}{}%
  \ifstrequal{#1}{hex\_bits\_30\_backwards\_matches}{\hyperref[sailRISCVzhexzybitszy30zybackwardszymatches]{#2}}{}%
  \ifstrequal{#1}{hex_bits_30_forwards}{\hyperref[sailRISCVzhexzybitszy30zyforwards]{#2}}{}%
  \ifstrequal{#1}{hex\_bits\_30\_forwards}{\hyperref[sailRISCVzhexzybitszy30zyforwards]{#2}}{}%
  \ifstrequal{#1}{hex_bits_30_forwards_matches}{\hyperref[sailRISCVzhexzybitszy30zyforwardszymatches]{#2}}{}%
  \ifstrequal{#1}{hex\_bits\_30\_forwards\_matches}{\hyperref[sailRISCVzhexzybitszy30zyforwardszymatches]{#2}}{}%
  \ifstrequal{#1}{hex_bits_30_matches_prefix}{\hyperref[sailRISCVzhexzybitszy30zymatcheszyprefix]{#2}}{}%
  \ifstrequal{#1}{hex\_bits\_30\_matches\_prefix}{\hyperref[sailRISCVzhexzybitszy30zymatcheszyprefix]{#2}}{}%
  \ifstrequal{#1}{hex_bits_31}{\hyperref[sailRISCVzhexzybitszy31]{#2}}{}%
  \ifstrequal{#1}{hex\_bits\_31}{\hyperref[sailRISCVzhexzybitszy31]{#2}}{}%
  \ifstrequal{#1}{hex_bits_31_backwards}{\hyperref[sailRISCVzhexzybitszy31zybackwards]{#2}}{}%
  \ifstrequal{#1}{hex\_bits\_31\_backwards}{\hyperref[sailRISCVzhexzybitszy31zybackwards]{#2}}{}%
  \ifstrequal{#1}{hex_bits_31_backwards_matches}{\hyperref[sailRISCVzhexzybitszy31zybackwardszymatches]{#2}}{}%
  \ifstrequal{#1}{hex\_bits\_31\_backwards\_matches}{\hyperref[sailRISCVzhexzybitszy31zybackwardszymatches]{#2}}{}%
  \ifstrequal{#1}{hex_bits_31_forwards}{\hyperref[sailRISCVzhexzybitszy31zyforwards]{#2}}{}%
  \ifstrequal{#1}{hex\_bits\_31\_forwards}{\hyperref[sailRISCVzhexzybitszy31zyforwards]{#2}}{}%
  \ifstrequal{#1}{hex_bits_31_forwards_matches}{\hyperref[sailRISCVzhexzybitszy31zyforwardszymatches]{#2}}{}%
  \ifstrequal{#1}{hex\_bits\_31\_forwards\_matches}{\hyperref[sailRISCVzhexzybitszy31zyforwardszymatches]{#2}}{}%
  \ifstrequal{#1}{hex_bits_31_matches_prefix}{\hyperref[sailRISCVzhexzybitszy31zymatcheszyprefix]{#2}}{}%
  \ifstrequal{#1}{hex\_bits\_31\_matches\_prefix}{\hyperref[sailRISCVzhexzybitszy31zymatcheszyprefix]{#2}}{}%
  \ifstrequal{#1}{hex_bits_32}{\hyperref[sailRISCVzhexzybitszy32]{#2}}{}%
  \ifstrequal{#1}{hex\_bits\_32}{\hyperref[sailRISCVzhexzybitszy32]{#2}}{}%
  \ifstrequal{#1}{hex_bits_32_backwards}{\hyperref[sailRISCVzhexzybitszy32zybackwards]{#2}}{}%
  \ifstrequal{#1}{hex\_bits\_32\_backwards}{\hyperref[sailRISCVzhexzybitszy32zybackwards]{#2}}{}%
  \ifstrequal{#1}{hex_bits_32_backwards_matches}{\hyperref[sailRISCVzhexzybitszy32zybackwardszymatches]{#2}}{}%
  \ifstrequal{#1}{hex\_bits\_32\_backwards\_matches}{\hyperref[sailRISCVzhexzybitszy32zybackwardszymatches]{#2}}{}%
  \ifstrequal{#1}{hex_bits_32_forwards}{\hyperref[sailRISCVzhexzybitszy32zyforwards]{#2}}{}%
  \ifstrequal{#1}{hex\_bits\_32\_forwards}{\hyperref[sailRISCVzhexzybitszy32zyforwards]{#2}}{}%
  \ifstrequal{#1}{hex_bits_32_forwards_matches}{\hyperref[sailRISCVzhexzybitszy32zyforwardszymatches]{#2}}{}%
  \ifstrequal{#1}{hex\_bits\_32\_forwards\_matches}{\hyperref[sailRISCVzhexzybitszy32zyforwardszymatches]{#2}}{}%
  \ifstrequal{#1}{hex_bits_32_matches_prefix}{\hyperref[sailRISCVzhexzybitszy32zymatcheszyprefix]{#2}}{}%
  \ifstrequal{#1}{hex\_bits\_32\_matches\_prefix}{\hyperref[sailRISCVzhexzybitszy32zymatcheszyprefix]{#2}}{}%
  \ifstrequal{#1}{hex_bits_33}{\hyperref[sailRISCVzhexzybitszy33]{#2}}{}%
  \ifstrequal{#1}{hex\_bits\_33}{\hyperref[sailRISCVzhexzybitszy33]{#2}}{}%
  \ifstrequal{#1}{hex_bits_33_backwards}{\hyperref[sailRISCVzhexzybitszy33zybackwards]{#2}}{}%
  \ifstrequal{#1}{hex\_bits\_33\_backwards}{\hyperref[sailRISCVzhexzybitszy33zybackwards]{#2}}{}%
  \ifstrequal{#1}{hex_bits_33_backwards_matches}{\hyperref[sailRISCVzhexzybitszy33zybackwardszymatches]{#2}}{}%
  \ifstrequal{#1}{hex\_bits\_33\_backwards\_matches}{\hyperref[sailRISCVzhexzybitszy33zybackwardszymatches]{#2}}{}%
  \ifstrequal{#1}{hex_bits_33_forwards}{\hyperref[sailRISCVzhexzybitszy33zyforwards]{#2}}{}%
  \ifstrequal{#1}{hex\_bits\_33\_forwards}{\hyperref[sailRISCVzhexzybitszy33zyforwards]{#2}}{}%
  \ifstrequal{#1}{hex_bits_33_forwards_matches}{\hyperref[sailRISCVzhexzybitszy33zyforwardszymatches]{#2}}{}%
  \ifstrequal{#1}{hex\_bits\_33\_forwards\_matches}{\hyperref[sailRISCVzhexzybitszy33zyforwardszymatches]{#2}}{}%
  \ifstrequal{#1}{hex_bits_33_matches_prefix}{\hyperref[sailRISCVzhexzybitszy33zymatcheszyprefix]{#2}}{}%
  \ifstrequal{#1}{hex\_bits\_33\_matches\_prefix}{\hyperref[sailRISCVzhexzybitszy33zymatcheszyprefix]{#2}}{}%
  \ifstrequal{#1}{hex_bits_3_backwards}{\hyperref[sailRISCVzhexzybitszy3zybackwards]{#2}}{}%
  \ifstrequal{#1}{hex\_bits\_3\_backwards}{\hyperref[sailRISCVzhexzybitszy3zybackwards]{#2}}{}%
  \ifstrequal{#1}{hex_bits_3_backwards_matches}{\hyperref[sailRISCVzhexzybitszy3zybackwardszymatches]{#2}}{}%
  \ifstrequal{#1}{hex\_bits\_3\_backwards\_matches}{\hyperref[sailRISCVzhexzybitszy3zybackwardszymatches]{#2}}{}%
  \ifstrequal{#1}{hex_bits_3_forwards}{\hyperref[sailRISCVzhexzybitszy3zyforwards]{#2}}{}%
  \ifstrequal{#1}{hex\_bits\_3\_forwards}{\hyperref[sailRISCVzhexzybitszy3zyforwards]{#2}}{}%
  \ifstrequal{#1}{hex_bits_3_forwards_matches}{\hyperref[sailRISCVzhexzybitszy3zyforwardszymatches]{#2}}{}%
  \ifstrequal{#1}{hex\_bits\_3\_forwards\_matches}{\hyperref[sailRISCVzhexzybitszy3zyforwardszymatches]{#2}}{}%
  \ifstrequal{#1}{hex_bits_3_matches_prefix}{\hyperref[sailRISCVzhexzybitszy3zymatcheszyprefix]{#2}}{}%
  \ifstrequal{#1}{hex\_bits\_3\_matches\_prefix}{\hyperref[sailRISCVzhexzybitszy3zymatcheszyprefix]{#2}}{}%
  \ifstrequal{#1}{hex_bits_4}{\hyperref[sailRISCVzhexzybitszy4]{#2}}{}%
  \ifstrequal{#1}{hex\_bits\_4}{\hyperref[sailRISCVzhexzybitszy4]{#2}}{}%
  \ifstrequal{#1}{hex_bits_48}{\hyperref[sailRISCVzhexzybitszy48]{#2}}{}%
  \ifstrequal{#1}{hex\_bits\_48}{\hyperref[sailRISCVzhexzybitszy48]{#2}}{}%
  \ifstrequal{#1}{hex_bits_48_backwards}{\hyperref[sailRISCVzhexzybitszy48zybackwards]{#2}}{}%
  \ifstrequal{#1}{hex\_bits\_48\_backwards}{\hyperref[sailRISCVzhexzybitszy48zybackwards]{#2}}{}%
  \ifstrequal{#1}{hex_bits_48_backwards_matches}{\hyperref[sailRISCVzhexzybitszy48zybackwardszymatches]{#2}}{}%
  \ifstrequal{#1}{hex\_bits\_48\_backwards\_matches}{\hyperref[sailRISCVzhexzybitszy48zybackwardszymatches]{#2}}{}%
  \ifstrequal{#1}{hex_bits_48_forwards}{\hyperref[sailRISCVzhexzybitszy48zyforwards]{#2}}{}%
  \ifstrequal{#1}{hex\_bits\_48\_forwards}{\hyperref[sailRISCVzhexzybitszy48zyforwards]{#2}}{}%
  \ifstrequal{#1}{hex_bits_48_forwards_matches}{\hyperref[sailRISCVzhexzybitszy48zyforwardszymatches]{#2}}{}%
  \ifstrequal{#1}{hex\_bits\_48\_forwards\_matches}{\hyperref[sailRISCVzhexzybitszy48zyforwardszymatches]{#2}}{}%
  \ifstrequal{#1}{hex_bits_48_matches_prefix}{\hyperref[sailRISCVzhexzybitszy48zymatcheszyprefix]{#2}}{}%
  \ifstrequal{#1}{hex\_bits\_48\_matches\_prefix}{\hyperref[sailRISCVzhexzybitszy48zymatcheszyprefix]{#2}}{}%
  \ifstrequal{#1}{hex_bits_4_backwards}{\hyperref[sailRISCVzhexzybitszy4zybackwards]{#2}}{}%
  \ifstrequal{#1}{hex\_bits\_4\_backwards}{\hyperref[sailRISCVzhexzybitszy4zybackwards]{#2}}{}%
  \ifstrequal{#1}{hex_bits_4_backwards_matches}{\hyperref[sailRISCVzhexzybitszy4zybackwardszymatches]{#2}}{}%
  \ifstrequal{#1}{hex\_bits\_4\_backwards\_matches}{\hyperref[sailRISCVzhexzybitszy4zybackwardszymatches]{#2}}{}%
  \ifstrequal{#1}{hex_bits_4_forwards}{\hyperref[sailRISCVzhexzybitszy4zyforwards]{#2}}{}%
  \ifstrequal{#1}{hex\_bits\_4\_forwards}{\hyperref[sailRISCVzhexzybitszy4zyforwards]{#2}}{}%
  \ifstrequal{#1}{hex_bits_4_forwards_matches}{\hyperref[sailRISCVzhexzybitszy4zyforwardszymatches]{#2}}{}%
  \ifstrequal{#1}{hex\_bits\_4\_forwards\_matches}{\hyperref[sailRISCVzhexzybitszy4zyforwardszymatches]{#2}}{}%
  \ifstrequal{#1}{hex_bits_4_matches_prefix}{\hyperref[sailRISCVzhexzybitszy4zymatcheszyprefix]{#2}}{}%
  \ifstrequal{#1}{hex\_bits\_4\_matches\_prefix}{\hyperref[sailRISCVzhexzybitszy4zymatcheszyprefix]{#2}}{}%
  \ifstrequal{#1}{hex_bits_5}{\hyperref[sailRISCVzhexzybitszy5]{#2}}{}%
  \ifstrequal{#1}{hex\_bits\_5}{\hyperref[sailRISCVzhexzybitszy5]{#2}}{}%
  \ifstrequal{#1}{hex_bits_5_backwards}{\hyperref[sailRISCVzhexzybitszy5zybackwards]{#2}}{}%
  \ifstrequal{#1}{hex\_bits\_5\_backwards}{\hyperref[sailRISCVzhexzybitszy5zybackwards]{#2}}{}%
  \ifstrequal{#1}{hex_bits_5_backwards_matches}{\hyperref[sailRISCVzhexzybitszy5zybackwardszymatches]{#2}}{}%
  \ifstrequal{#1}{hex\_bits\_5\_backwards\_matches}{\hyperref[sailRISCVzhexzybitszy5zybackwardszymatches]{#2}}{}%
  \ifstrequal{#1}{hex_bits_5_forwards}{\hyperref[sailRISCVzhexzybitszy5zyforwards]{#2}}{}%
  \ifstrequal{#1}{hex\_bits\_5\_forwards}{\hyperref[sailRISCVzhexzybitszy5zyforwards]{#2}}{}%
  \ifstrequal{#1}{hex_bits_5_forwards_matches}{\hyperref[sailRISCVzhexzybitszy5zyforwardszymatches]{#2}}{}%
  \ifstrequal{#1}{hex\_bits\_5\_forwards\_matches}{\hyperref[sailRISCVzhexzybitszy5zyforwardszymatches]{#2}}{}%
  \ifstrequal{#1}{hex_bits_5_matches_prefix}{\hyperref[sailRISCVzhexzybitszy5zymatcheszyprefix]{#2}}{}%
  \ifstrequal{#1}{hex\_bits\_5\_matches\_prefix}{\hyperref[sailRISCVzhexzybitszy5zymatcheszyprefix]{#2}}{}%
  \ifstrequal{#1}{hex_bits_6}{\hyperref[sailRISCVzhexzybitszy6]{#2}}{}%
  \ifstrequal{#1}{hex\_bits\_6}{\hyperref[sailRISCVzhexzybitszy6]{#2}}{}%
  \ifstrequal{#1}{hex_bits_64}{\hyperref[sailRISCVzhexzybitszy64]{#2}}{}%
  \ifstrequal{#1}{hex\_bits\_64}{\hyperref[sailRISCVzhexzybitszy64]{#2}}{}%
  \ifstrequal{#1}{hex_bits_64_backwards}{\hyperref[sailRISCVzhexzybitszy64zybackwards]{#2}}{}%
  \ifstrequal{#1}{hex\_bits\_64\_backwards}{\hyperref[sailRISCVzhexzybitszy64zybackwards]{#2}}{}%
  \ifstrequal{#1}{hex_bits_64_backwards_matches}{\hyperref[sailRISCVzhexzybitszy64zybackwardszymatches]{#2}}{}%
  \ifstrequal{#1}{hex\_bits\_64\_backwards\_matches}{\hyperref[sailRISCVzhexzybitszy64zybackwardszymatches]{#2}}{}%
  \ifstrequal{#1}{hex_bits_64_forwards}{\hyperref[sailRISCVzhexzybitszy64zyforwards]{#2}}{}%
  \ifstrequal{#1}{hex\_bits\_64\_forwards}{\hyperref[sailRISCVzhexzybitszy64zyforwards]{#2}}{}%
  \ifstrequal{#1}{hex_bits_64_forwards_matches}{\hyperref[sailRISCVzhexzybitszy64zyforwardszymatches]{#2}}{}%
  \ifstrequal{#1}{hex\_bits\_64\_forwards\_matches}{\hyperref[sailRISCVzhexzybitszy64zyforwardszymatches]{#2}}{}%
  \ifstrequal{#1}{hex_bits_64_matches_prefix}{\hyperref[sailRISCVzhexzybitszy64zymatcheszyprefix]{#2}}{}%
  \ifstrequal{#1}{hex\_bits\_64\_matches\_prefix}{\hyperref[sailRISCVzhexzybitszy64zymatcheszyprefix]{#2}}{}%
  \ifstrequal{#1}{hex_bits_6_backwards}{\hyperref[sailRISCVzhexzybitszy6zybackwards]{#2}}{}%
  \ifstrequal{#1}{hex\_bits\_6\_backwards}{\hyperref[sailRISCVzhexzybitszy6zybackwards]{#2}}{}%
  \ifstrequal{#1}{hex_bits_6_backwards_matches}{\hyperref[sailRISCVzhexzybitszy6zybackwardszymatches]{#2}}{}%
  \ifstrequal{#1}{hex\_bits\_6\_backwards\_matches}{\hyperref[sailRISCVzhexzybitszy6zybackwardszymatches]{#2}}{}%
  \ifstrequal{#1}{hex_bits_6_forwards}{\hyperref[sailRISCVzhexzybitszy6zyforwards]{#2}}{}%
  \ifstrequal{#1}{hex\_bits\_6\_forwards}{\hyperref[sailRISCVzhexzybitszy6zyforwards]{#2}}{}%
  \ifstrequal{#1}{hex_bits_6_forwards_matches}{\hyperref[sailRISCVzhexzybitszy6zyforwardszymatches]{#2}}{}%
  \ifstrequal{#1}{hex\_bits\_6\_forwards\_matches}{\hyperref[sailRISCVzhexzybitszy6zyforwardszymatches]{#2}}{}%
  \ifstrequal{#1}{hex_bits_6_matches_prefix}{\hyperref[sailRISCVzhexzybitszy6zymatcheszyprefix]{#2}}{}%
  \ifstrequal{#1}{hex\_bits\_6\_matches\_prefix}{\hyperref[sailRISCVzhexzybitszy6zymatcheszyprefix]{#2}}{}%
  \ifstrequal{#1}{hex_bits_7}{\hyperref[sailRISCVzhexzybitszy7]{#2}}{}%
  \ifstrequal{#1}{hex\_bits\_7}{\hyperref[sailRISCVzhexzybitszy7]{#2}}{}%
  \ifstrequal{#1}{hex_bits_7_backwards}{\hyperref[sailRISCVzhexzybitszy7zybackwards]{#2}}{}%
  \ifstrequal{#1}{hex\_bits\_7\_backwards}{\hyperref[sailRISCVzhexzybitszy7zybackwards]{#2}}{}%
  \ifstrequal{#1}{hex_bits_7_backwards_matches}{\hyperref[sailRISCVzhexzybitszy7zybackwardszymatches]{#2}}{}%
  \ifstrequal{#1}{hex\_bits\_7\_backwards\_matches}{\hyperref[sailRISCVzhexzybitszy7zybackwardszymatches]{#2}}{}%
  \ifstrequal{#1}{hex_bits_7_forwards}{\hyperref[sailRISCVzhexzybitszy7zyforwards]{#2}}{}%
  \ifstrequal{#1}{hex\_bits\_7\_forwards}{\hyperref[sailRISCVzhexzybitszy7zyforwards]{#2}}{}%
  \ifstrequal{#1}{hex_bits_7_forwards_matches}{\hyperref[sailRISCVzhexzybitszy7zyforwardszymatches]{#2}}{}%
  \ifstrequal{#1}{hex\_bits\_7\_forwards\_matches}{\hyperref[sailRISCVzhexzybitszy7zyforwardszymatches]{#2}}{}%
  \ifstrequal{#1}{hex_bits_7_matches_prefix}{\hyperref[sailRISCVzhexzybitszy7zymatcheszyprefix]{#2}}{}%
  \ifstrequal{#1}{hex\_bits\_7\_matches\_prefix}{\hyperref[sailRISCVzhexzybitszy7zymatcheszyprefix]{#2}}{}%
  \ifstrequal{#1}{hex_bits_8}{\hyperref[sailRISCVzhexzybitszy8]{#2}}{}%
  \ifstrequal{#1}{hex\_bits\_8}{\hyperref[sailRISCVzhexzybitszy8]{#2}}{}%
  \ifstrequal{#1}{hex_bits_8_backwards}{\hyperref[sailRISCVzhexzybitszy8zybackwards]{#2}}{}%
  \ifstrequal{#1}{hex\_bits\_8\_backwards}{\hyperref[sailRISCVzhexzybitszy8zybackwards]{#2}}{}%
  \ifstrequal{#1}{hex_bits_8_backwards_matches}{\hyperref[sailRISCVzhexzybitszy8zybackwardszymatches]{#2}}{}%
  \ifstrequal{#1}{hex\_bits\_8\_backwards\_matches}{\hyperref[sailRISCVzhexzybitszy8zybackwardszymatches]{#2}}{}%
  \ifstrequal{#1}{hex_bits_8_forwards}{\hyperref[sailRISCVzhexzybitszy8zyforwards]{#2}}{}%
  \ifstrequal{#1}{hex\_bits\_8\_forwards}{\hyperref[sailRISCVzhexzybitszy8zyforwards]{#2}}{}%
  \ifstrequal{#1}{hex_bits_8_forwards_matches}{\hyperref[sailRISCVzhexzybitszy8zyforwardszymatches]{#2}}{}%
  \ifstrequal{#1}{hex\_bits\_8\_forwards\_matches}{\hyperref[sailRISCVzhexzybitszy8zyforwardszymatches]{#2}}{}%
  \ifstrequal{#1}{hex_bits_8_matches_prefix}{\hyperref[sailRISCVzhexzybitszy8zymatcheszyprefix]{#2}}{}%
  \ifstrequal{#1}{hex\_bits\_8\_matches\_prefix}{\hyperref[sailRISCVzhexzybitszy8zymatcheszyprefix]{#2}}{}%
  \ifstrequal{#1}{hex_bits_9}{\hyperref[sailRISCVzhexzybitszy9]{#2}}{}%
  \ifstrequal{#1}{hex\_bits\_9}{\hyperref[sailRISCVzhexzybitszy9]{#2}}{}%
  \ifstrequal{#1}{hex_bits_9_backwards}{\hyperref[sailRISCVzhexzybitszy9zybackwards]{#2}}{}%
  \ifstrequal{#1}{hex\_bits\_9\_backwards}{\hyperref[sailRISCVzhexzybitszy9zybackwards]{#2}}{}%
  \ifstrequal{#1}{hex_bits_9_backwards_matches}{\hyperref[sailRISCVzhexzybitszy9zybackwardszymatches]{#2}}{}%
  \ifstrequal{#1}{hex\_bits\_9\_backwards\_matches}{\hyperref[sailRISCVzhexzybitszy9zybackwardszymatches]{#2}}{}%
  \ifstrequal{#1}{hex_bits_9_forwards}{\hyperref[sailRISCVzhexzybitszy9zyforwards]{#2}}{}%
  \ifstrequal{#1}{hex\_bits\_9\_forwards}{\hyperref[sailRISCVzhexzybitszy9zyforwards]{#2}}{}%
  \ifstrequal{#1}{hex_bits_9_forwards_matches}{\hyperref[sailRISCVzhexzybitszy9zyforwardszymatches]{#2}}{}%
  \ifstrequal{#1}{hex\_bits\_9\_forwards\_matches}{\hyperref[sailRISCVzhexzybitszy9zyforwardszymatches]{#2}}{}%
  \ifstrequal{#1}{hex_bits_9_matches_prefix}{\hyperref[sailRISCVzhexzybitszy9zymatcheszyprefix]{#2}}{}%
  \ifstrequal{#1}{hex\_bits\_9\_matches\_prefix}{\hyperref[sailRISCVzhexzybitszy9zymatcheszyprefix]{#2}}{}%
  \ifstrequal{#1}{hex_str}{\hyperref[sailRISCVzhexzystr]{#2}}{}%
  \ifstrequal{#1}{hex\_str}{\hyperref[sailRISCVzhexzystr]{#2}}{}%
  \ifstrequal{#1}{htif_load}{\hyperref[sailRISCVzhtifzyload]{#2}}{}%
  \ifstrequal{#1}{htif\_load}{\hyperref[sailRISCVzhtifzyload]{#2}}{}%
  \ifstrequal{#1}{htif_store}{\hyperref[sailRISCVzhtifzystore]{#2}}{}%
  \ifstrequal{#1}{htif\_store}{\hyperref[sailRISCVzhtifzystore]{#2}}{}%
  \ifstrequal{#1}{htif_tick}{\hyperref[sailRISCVzhtifzytick]{#2}}{}%
  \ifstrequal{#1}{htif\_tick}{\hyperref[sailRISCVzhtifzytick]{#2}}{}%
  \ifstrequal{#1}{in32BitMode}{\hyperref[sailRISCVzin32BitMode]{#2}}{}%
  \ifstrequal{#1}{inCapBounds}{\hyperref[sailRISCVzinCapBounds]{#2}}{}%
  \ifstrequal{#1}{incCapOffset}{\hyperref[sailRISCVzincCapOffset]{#2}}{}%
  \ifstrequal{#1}{init_base_regs}{\hyperref[sailRISCVzinitzybasezyregs]{#2}}{}%
  \ifstrequal{#1}{init\_base\_regs}{\hyperref[sailRISCVzinitzybasezyregs]{#2}}{}%
  \ifstrequal{#1}{init_fdext_regs}{\hyperref[sailRISCVzinitzyfdextzyregs]{#2}}{}%
  \ifstrequal{#1}{init\_fdext\_regs}{\hyperref[sailRISCVzinitzyfdextzyregs]{#2}}{}%
  \ifstrequal{#1}{init_model}{\hyperref[sailRISCVzinitzymodel]{#2}}{}%
  \ifstrequal{#1}{init\_model}{\hyperref[sailRISCVzinitzymodel]{#2}}{}%
  \ifstrequal{#1}{init_platform}{\hyperref[sailRISCVzinitzyplatform]{#2}}{}%
  \ifstrequal{#1}{init\_platform}{\hyperref[sailRISCVzinitzyplatform]{#2}}{}%
  \ifstrequal{#1}{init_pmp}{\hyperref[sailRISCVzinitzypmp]{#2}}{}%
  \ifstrequal{#1}{init\_pmp}{\hyperref[sailRISCVzinitzypmp]{#2}}{}%
  \ifstrequal{#1}{init_sys}{\hyperref[sailRISCVzinitzysys]{#2}}{}%
  \ifstrequal{#1}{init\_sys}{\hyperref[sailRISCVzinitzysys]{#2}}{}%
  \ifstrequal{#1}{init_vmem}{\hyperref[sailRISCVzinitzyvmem]{#2}}{}%
  \ifstrequal{#1}{init\_vmem}{\hyperref[sailRISCVzinitzyvmem]{#2}}{}%
  \ifstrequal{#1}{init_vmem_sv39}{\hyperref[sailRISCVzinitzyvmemzysv39]{#2}}{}%
  \ifstrequal{#1}{init\_vmem\_sv39}{\hyperref[sailRISCVzinitzyvmemzysv39]{#2}}{}%
  \ifstrequal{#1}{init_vmem_sv48}{\hyperref[sailRISCVzinitzyvmemzysv48]{#2}}{}%
  \ifstrequal{#1}{init\_vmem\_sv48}{\hyperref[sailRISCVzinitzyvmemzysv48]{#2}}{}%
  \ifstrequal{#1}{initial_analysis}{\hyperref[sailRISCVzinitialzyanalysis]{#2}}{}%
  \ifstrequal{#1}{initial\_analysis}{\hyperref[sailRISCVzinitialzyanalysis]{#2}}{}%
  \ifstrequal{#1}{int_power}{\hyperref[sailRISCVzintzypower]{#2}}{}%
  \ifstrequal{#1}{int\_power}{\hyperref[sailRISCVzintzypower]{#2}}{}%
  \ifstrequal{#1}{int_to_cap}{\hyperref[sailRISCVzintzytozycap]{#2}}{}%
  \ifstrequal{#1}{int\_to\_cap}{\hyperref[sailRISCVzintzytozycap]{#2}}{}%
  \ifstrequal{#1}{internal_error}{\hyperref[sailRISCVzinternalzyerror]{#2}}{}%
  \ifstrequal{#1}{internal\_error}{\hyperref[sailRISCVzinternalzyerror]{#2}}{}%
  \ifstrequal{#1}{interruptType_to_bits}{\hyperref[sailRISCVzinterruptTypezytozybits]{#2}}{}%
  \ifstrequal{#1}{interruptType\_to\_bits}{\hyperref[sailRISCVzinterruptTypezytozybits]{#2}}{}%
  \ifstrequal{#1}{iop_of_num}{\hyperref[sailRISCVziopzyofzynum]{#2}}{}%
  \ifstrequal{#1}{iop\_of\_num}{\hyperref[sailRISCVziopzyofzynum]{#2}}{}%
  \ifstrequal{#1}{isCapSealed}{\hyperref[sailRISCVzisCapSealed]{#2}}{}%
  \ifstrequal{#1}{isInvalidPTE}{\hyperref[sailRISCVzisInvalidPTE]{#2}}{}%
  \ifstrequal{#1}{isPTEPtr}{\hyperref[sailRISCVzisPTEPtr]{#2}}{}%
  \ifstrequal{#1}{isRVC}{\hyperref[sailRISCVzisRVC]{#2}}{}%
  \ifstrequal{#1}{isValidSv39Addr}{\hyperref[sailRISCVzisValidSv39Addr]{#2}}{}%
  \ifstrequal{#1}{isValidSv48Addr}{\hyperref[sailRISCVzisValidSv48Addr]{#2}}{}%
  \ifstrequal{#1}{is_CSR_defined}{\hyperref[sailRISCVziszyCSRzydefined]{#2}}{}%
  \ifstrequal{#1}{is\_CSR\_defined}{\hyperref[sailRISCVziszyCSRzydefined]{#2}}{}%
  \ifstrequal{#1}{is_aligned_addr}{\hyperref[sailRISCVziszyalignedzyaddr]{#2}}{}%
  \ifstrequal{#1}{is\_aligned\_addr}{\hyperref[sailRISCVziszyalignedzyaddr]{#2}}{}%
  \ifstrequal{#1}{is_none}{\hyperref[sailRISCVziszynone]{#2}}{}%
  \ifstrequal{#1}{is\_none}{\hyperref[sailRISCVziszynone]{#2}}{}%
  \ifstrequal{#1}{is_some}{\hyperref[sailRISCVziszysome]{#2}}{}%
  \ifstrequal{#1}{is\_some}{\hyperref[sailRISCVziszysome]{#2}}{}%
  \ifstrequal{#1}{itype_mnemonic}{\hyperref[sailRISCVzitypezymnemonic]{#2}}{}%
  \ifstrequal{#1}{itype\_mnemonic}{\hyperref[sailRISCVzitypezymnemonic]{#2}}{}%
  \ifstrequal{#1}{legalize_ccsr}{\hyperref[sailRISCVzlegalizzezyccsr]{#2}}{}%
  \ifstrequal{#1}{legalize\_ccsr}{\hyperref[sailRISCVzlegalizzezyccsr]{#2}}{}%
  \ifstrequal{#1}{legalize_epcc}{\hyperref[sailRISCVzlegalizzezyepcc]{#2}}{}%
  \ifstrequal{#1}{legalize\_epcc}{\hyperref[sailRISCVzlegalizzezyepcc]{#2}}{}%
  \ifstrequal{#1}{legalize_mcounteren}{\hyperref[sailRISCVzlegalizzezymcounteren]{#2}}{}%
  \ifstrequal{#1}{legalize\_mcounteren}{\hyperref[sailRISCVzlegalizzezymcounteren]{#2}}{}%
  \ifstrequal{#1}{legalize_mcountinhibit}{\hyperref[sailRISCVzlegalizzezymcountinhibit]{#2}}{}%
  \ifstrequal{#1}{legalize\_mcountinhibit}{\hyperref[sailRISCVzlegalizzezymcountinhibit]{#2}}{}%
  \ifstrequal{#1}{legalize_medeleg}{\hyperref[sailRISCVzlegalizzezymedeleg]{#2}}{}%
  \ifstrequal{#1}{legalize\_medeleg}{\hyperref[sailRISCVzlegalizzezymedeleg]{#2}}{}%
  \ifstrequal{#1}{legalize_mideleg}{\hyperref[sailRISCVzlegalizzezymideleg]{#2}}{}%
  \ifstrequal{#1}{legalize\_mideleg}{\hyperref[sailRISCVzlegalizzezymideleg]{#2}}{}%
  \ifstrequal{#1}{legalize_mie}{\hyperref[sailRISCVzlegalizzezymie]{#2}}{}%
  \ifstrequal{#1}{legalize\_mie}{\hyperref[sailRISCVzlegalizzezymie]{#2}}{}%
  \ifstrequal{#1}{legalize_mip}{\hyperref[sailRISCVzlegalizzezymip]{#2}}{}%
  \ifstrequal{#1}{legalize\_mip}{\hyperref[sailRISCVzlegalizzezymip]{#2}}{}%
  \ifstrequal{#1}{legalize_misa}{\hyperref[sailRISCVzlegalizzezymisa]{#2}}{}%
  \ifstrequal{#1}{legalize\_misa}{\hyperref[sailRISCVzlegalizzezymisa]{#2}}{}%
  \ifstrequal{#1}{legalize_mstatus}{\hyperref[sailRISCVzlegalizzezymstatus]{#2}}{}%
  \ifstrequal{#1}{legalize\_mstatus}{\hyperref[sailRISCVzlegalizzezymstatus]{#2}}{}%
  \ifstrequal{#1}{legalize_satp}{\hyperref[sailRISCVzlegalizzezysatp]{#2}}{}%
  \ifstrequal{#1}{legalize\_satp}{\hyperref[sailRISCVzlegalizzezysatp]{#2}}{}%
  \ifstrequal{#1}{legalize_satp32}{\hyperref[sailRISCVzlegalizzezysatp32]{#2}}{}%
  \ifstrequal{#1}{legalize\_satp32}{\hyperref[sailRISCVzlegalizzezysatp32]{#2}}{}%
  \ifstrequal{#1}{legalize_satp64}{\hyperref[sailRISCVzlegalizzezysatp64]{#2}}{}%
  \ifstrequal{#1}{legalize\_satp64}{\hyperref[sailRISCVzlegalizzezysatp64]{#2}}{}%
  \ifstrequal{#1}{legalize_scounteren}{\hyperref[sailRISCVzlegalizzezyscounteren]{#2}}{}%
  \ifstrequal{#1}{legalize\_scounteren}{\hyperref[sailRISCVzlegalizzezyscounteren]{#2}}{}%
  \ifstrequal{#1}{legalize_sedeleg}{\hyperref[sailRISCVzlegalizzezysedeleg]{#2}}{}%
  \ifstrequal{#1}{legalize\_sedeleg}{\hyperref[sailRISCVzlegalizzezysedeleg]{#2}}{}%
  \ifstrequal{#1}{legalize_sie}{\hyperref[sailRISCVzlegalizzezysie]{#2}}{}%
  \ifstrequal{#1}{legalize\_sie}{\hyperref[sailRISCVzlegalizzezysie]{#2}}{}%
  \ifstrequal{#1}{legalize_sip}{\hyperref[sailRISCVzlegalizzezysip]{#2}}{}%
  \ifstrequal{#1}{legalize\_sip}{\hyperref[sailRISCVzlegalizzezysip]{#2}}{}%
  \ifstrequal{#1}{legalize_sstatus}{\hyperref[sailRISCVzlegalizzezysstatus]{#2}}{}%
  \ifstrequal{#1}{legalize\_sstatus}{\hyperref[sailRISCVzlegalizzezysstatus]{#2}}{}%
  \ifstrequal{#1}{legalize_tcc}{\hyperref[sailRISCVzlegalizzezytcc]{#2}}{}%
  \ifstrequal{#1}{legalize\_tcc}{\hyperref[sailRISCVzlegalizzezytcc]{#2}}{}%
  \ifstrequal{#1}{legalize_tvec}{\hyperref[sailRISCVzlegalizzezytvec]{#2}}{}%
  \ifstrequal{#1}{legalize\_tvec}{\hyperref[sailRISCVzlegalizzezytvec]{#2}}{}%
  \ifstrequal{#1}{legalize_uie}{\hyperref[sailRISCVzlegalizzezyuie]{#2}}{}%
  \ifstrequal{#1}{legalize\_uie}{\hyperref[sailRISCVzlegalizzezyuie]{#2}}{}%
  \ifstrequal{#1}{legalize_uip}{\hyperref[sailRISCVzlegalizzezyuip]{#2}}{}%
  \ifstrequal{#1}{legalize\_uip}{\hyperref[sailRISCVzlegalizzezyuip]{#2}}{}%
  \ifstrequal{#1}{legalize_ustatus}{\hyperref[sailRISCVzlegalizzezyustatus]{#2}}{}%
  \ifstrequal{#1}{legalize\_ustatus}{\hyperref[sailRISCVzlegalizzezyustatus]{#2}}{}%
  \ifstrequal{#1}{legalize_xepc}{\hyperref[sailRISCVzlegalizzezyxepc]{#2}}{}%
  \ifstrequal{#1}{legalize\_xepc}{\hyperref[sailRISCVzlegalizzezyxepc]{#2}}{}%
  \ifstrequal{#1}{lift_sie}{\hyperref[sailRISCVzliftzysie]{#2}}{}%
  \ifstrequal{#1}{lift\_sie}{\hyperref[sailRISCVzliftzysie]{#2}}{}%
  \ifstrequal{#1}{lift_sip}{\hyperref[sailRISCVzliftzysip]{#2}}{}%
  \ifstrequal{#1}{lift\_sip}{\hyperref[sailRISCVzliftzysip]{#2}}{}%
  \ifstrequal{#1}{lift_sstatus}{\hyperref[sailRISCVzliftzysstatus]{#2}}{}%
  \ifstrequal{#1}{lift\_sstatus}{\hyperref[sailRISCVzliftzysstatus]{#2}}{}%
  \ifstrequal{#1}{lift_uie}{\hyperref[sailRISCVzliftzyuie]{#2}}{}%
  \ifstrequal{#1}{lift\_uie}{\hyperref[sailRISCVzliftzyuie]{#2}}{}%
  \ifstrequal{#1}{lift_uip}{\hyperref[sailRISCVzliftzyuip]{#2}}{}%
  \ifstrequal{#1}{lift\_uip}{\hyperref[sailRISCVzliftzyuip]{#2}}{}%
  \ifstrequal{#1}{lift_ustatus}{\hyperref[sailRISCVzliftzyustatus]{#2}}{}%
  \ifstrequal{#1}{lift\_ustatus}{\hyperref[sailRISCVzliftzyustatus]{#2}}{}%
  \ifstrequal{#1}{load_reservation}{\hyperref[sailRISCVzloadzyreservation]{#2}}{}%
  \ifstrequal{#1}{load\_reservation}{\hyperref[sailRISCVzloadzyreservation]{#2}}{}%
  \ifstrequal{#1}{lookup_TLB39}{\hyperref[sailRISCVzlookupzyTLB39]{#2}}{}%
  \ifstrequal{#1}{lookup\_TLB39}{\hyperref[sailRISCVzlookupzyTLB39]{#2}}{}%
  \ifstrequal{#1}{lookup_TLB48}{\hyperref[sailRISCVzlookupzyTLB48]{#2}}{}%
  \ifstrequal{#1}{lookup\_TLB48}{\hyperref[sailRISCVzlookupzyTLB48]{#2}}{}%
  \ifstrequal{#1}{loop}{\hyperref[sailRISCVzloop]{#2}}{}%
  \ifstrequal{#1}{lower_mie}{\hyperref[sailRISCVzlowerzymie]{#2}}{}%
  \ifstrequal{#1}{lower\_mie}{\hyperref[sailRISCVzlowerzymie]{#2}}{}%
  \ifstrequal{#1}{lower_mip}{\hyperref[sailRISCVzlowerzymip]{#2}}{}%
  \ifstrequal{#1}{lower\_mip}{\hyperref[sailRISCVzlowerzymip]{#2}}{}%
  \ifstrequal{#1}{lower_mstatus}{\hyperref[sailRISCVzlowerzymstatus]{#2}}{}%
  \ifstrequal{#1}{lower\_mstatus}{\hyperref[sailRISCVzlowerzymstatus]{#2}}{}%
  \ifstrequal{#1}{lower_sie}{\hyperref[sailRISCVzlowerzysie]{#2}}{}%
  \ifstrequal{#1}{lower\_sie}{\hyperref[sailRISCVzlowerzysie]{#2}}{}%
  \ifstrequal{#1}{lower_sip}{\hyperref[sailRISCVzlowerzysip]{#2}}{}%
  \ifstrequal{#1}{lower\_sip}{\hyperref[sailRISCVzlowerzysip]{#2}}{}%
  \ifstrequal{#1}{lower_sstatus}{\hyperref[sailRISCVzlowerzysstatus]{#2}}{}%
  \ifstrequal{#1}{lower\_sstatus}{\hyperref[sailRISCVzlowerzysstatus]{#2}}{}%
  \ifstrequal{#1}{lrsc_width_str}{\hyperref[sailRISCVzlrsczywidthzystr]{#2}}{}%
  \ifstrequal{#1}{lrsc\_width\_str}{\hyperref[sailRISCVzlrsczywidthzystr]{#2}}{}%
  \ifstrequal{#1}{lt_int}{\hyperref[sailRISCVzltzyint]{#2}}{}%
  \ifstrequal{#1}{lt\_int}{\hyperref[sailRISCVzltzyint]{#2}}{}%
  \ifstrequal{#1}{lteq_int}{\hyperref[sailRISCVzlteqzyint]{#2}}{}%
  \ifstrequal{#1}{lteq\_int}{\hyperref[sailRISCVzlteqzyint]{#2}}{}%
  \ifstrequal{#1}{make_TLB_Entry}{\hyperref[sailRISCVzmakezyTLBzyEntry]{#2}}{}%
  \ifstrequal{#1}{make\_TLB\_Entry}{\hyperref[sailRISCVzmakezyTLBzyEntry]{#2}}{}%
  \ifstrequal{#1}{match_TLB_Entry}{\hyperref[sailRISCVzmatchzyTLBzyEntry]{#2}}{}%
  \ifstrequal{#1}{match\_TLB\_Entry}{\hyperref[sailRISCVzmatchzyTLBzyEntry]{#2}}{}%
  \ifstrequal{#1}{match_reservation}{\hyperref[sailRISCVzmatchzyreservation]{#2}}{}%
  \ifstrequal{#1}{match\_reservation}{\hyperref[sailRISCVzmatchzyreservation]{#2}}{}%
  \ifstrequal{#1}{max_int}{\hyperref[sailRISCVzmaxzyint]{#2}}{}%
  \ifstrequal{#1}{max\_int}{\hyperref[sailRISCVzmaxzyint]{#2}}{}%
  \ifstrequal{#1}{maybe_aq}{\hyperref[sailRISCVzmaybezyaq]{#2}}{}%
  \ifstrequal{#1}{maybe\_aq}{\hyperref[sailRISCVzmaybezyaq]{#2}}{}%
  \ifstrequal{#1}{maybe_i}{\hyperref[sailRISCVzmaybezyi]{#2}}{}%
  \ifstrequal{#1}{maybe\_i}{\hyperref[sailRISCVzmaybezyi]{#2}}{}%
  \ifstrequal{#1}{maybe_not_u}{\hyperref[sailRISCVzmaybezynotzyu]{#2}}{}%
  \ifstrequal{#1}{maybe\_not\_u}{\hyperref[sailRISCVzmaybezynotzyu]{#2}}{}%
  \ifstrequal{#1}{maybe_rl}{\hyperref[sailRISCVzmaybezyrl]{#2}}{}%
  \ifstrequal{#1}{maybe\_rl}{\hyperref[sailRISCVzmaybezyrl]{#2}}{}%
  \ifstrequal{#1}{maybe_u}{\hyperref[sailRISCVzmaybezyu]{#2}}{}%
  \ifstrequal{#1}{maybe\_u}{\hyperref[sailRISCVzmaybezyu]{#2}}{}%
  \ifstrequal{#1}{memBitsToCapability}{\hyperref[sailRISCVzmemBitsToCapability]{#2}}{}%
  \ifstrequal{#1}{mem_read}{\hyperref[sailRISCVzmemzyread]{#2}}{}%
  \ifstrequal{#1}{mem\_read}{\hyperref[sailRISCVzmemzyread]{#2}}{}%
  \ifstrequal{#1}{mem_read_cap}{\hyperref[sailRISCVzmemzyreadzycap]{#2}}{}%
  \ifstrequal{#1}{mem\_read\_cap}{\hyperref[sailRISCVzmemzyreadzycap]{#2}}{}%
  \ifstrequal{#1}{mem_read_meta}{\hyperref[sailRISCVzmemzyreadzymeta]{#2}}{}%
  \ifstrequal{#1}{mem\_read\_meta}{\hyperref[sailRISCVzmemzyreadzymeta]{#2}}{}%
  \ifstrequal{#1}{mem_read_priv}{\hyperref[sailRISCVzmemzyreadzypriv]{#2}}{}%
  \ifstrequal{#1}{mem\_read\_priv}{\hyperref[sailRISCVzmemzyreadzypriv]{#2}}{}%
  \ifstrequal{#1}{mem_read_priv_meta}{\hyperref[sailRISCVzmemzyreadzyprivzymeta]{#2}}{}%
  \ifstrequal{#1}{mem\_read\_priv\_meta}{\hyperref[sailRISCVzmemzyreadzyprivzymeta]{#2}}{}%
  \ifstrequal{#1}{mem_write_cap}{\hyperref[sailRISCVzmemzywritezycap]{#2}}{}%
  \ifstrequal{#1}{mem\_write\_cap}{\hyperref[sailRISCVzmemzywritezycap]{#2}}{}%
  \ifstrequal{#1}{mem_write_ea}{\hyperref[sailRISCVzmemzywritezyea]{#2}}{}%
  \ifstrequal{#1}{mem\_write\_ea}{\hyperref[sailRISCVzmemzywritezyea]{#2}}{}%
  \ifstrequal{#1}{mem_write_ea_cap}{\hyperref[sailRISCVzmemzywritezyeazycap]{#2}}{}%
  \ifstrequal{#1}{mem\_write\_ea\_cap}{\hyperref[sailRISCVzmemzywritezyeazycap]{#2}}{}%
  \ifstrequal{#1}{mem_write_value}{\hyperref[sailRISCVzmemzywritezyvalue]{#2}}{}%
  \ifstrequal{#1}{mem\_write\_value}{\hyperref[sailRISCVzmemzywritezyvalue]{#2}}{}%
  \ifstrequal{#1}{mem_write_value_meta}{\hyperref[sailRISCVzmemzywritezyvaluezymeta]{#2}}{}%
  \ifstrequal{#1}{mem\_write\_value\_meta}{\hyperref[sailRISCVzmemzywritezyvaluezymeta]{#2}}{}%
  \ifstrequal{#1}{mem_write_value_priv}{\hyperref[sailRISCVzmemzywritezyvaluezypriv]{#2}}{}%
  \ifstrequal{#1}{mem\_write\_value\_priv}{\hyperref[sailRISCVzmemzywritezyvaluezypriv]{#2}}{}%
  \ifstrequal{#1}{mem_write_value_priv_meta}{\hyperref[sailRISCVzmemzywritezyvaluezyprivzymeta]{#2}}{}%
  \ifstrequal{#1}{mem\_write\_value\_priv\_meta}{\hyperref[sailRISCVzmemzywritezyvaluezyprivzymeta]{#2}}{}%
  \ifstrequal{#1}{min_instruction_bytes}{\hyperref[sailRISCVzminzyinstructionzybytes]{#2}}{}%
  \ifstrequal{#1}{min\_instruction\_bytes}{\hyperref[sailRISCVzminzyinstructionzybytes]{#2}}{}%
  \ifstrequal{#1}{min_int}{\hyperref[sailRISCVzminzyint]{#2}}{}%
  \ifstrequal{#1}{min\_int}{\hyperref[sailRISCVzminzyint]{#2}}{}%
  \ifstrequal{#1}{mmio_read}{\hyperref[sailRISCVzmmiozyread]{#2}}{}%
  \ifstrequal{#1}{mmio\_read}{\hyperref[sailRISCVzmmiozyread]{#2}}{}%
  \ifstrequal{#1}{mmio_write}{\hyperref[sailRISCVzmmiozywrite]{#2}}{}%
  \ifstrequal{#1}{mmio\_write}{\hyperref[sailRISCVzmmiozywrite]{#2}}{}%
  \ifstrequal{#1}{mul_mnemonic}{\hyperref[sailRISCVzmulzymnemonic]{#2}}{}%
  \ifstrequal{#1}{mul\_mnemonic}{\hyperref[sailRISCVzmulzymnemonic]{#2}}{}%
  \ifstrequal{#1}{mult_atom}{\hyperref[sailRISCVzmultzyatom]{#2}}{}%
  \ifstrequal{#1}{mult\_atom}{\hyperref[sailRISCVzmultzyatom]{#2}}{}%
  \ifstrequal{#1}{mult_int}{\hyperref[sailRISCVzmultzyint]{#2}}{}%
  \ifstrequal{#1}{mult\_int}{\hyperref[sailRISCVzmultzyint]{#2}}{}%
  \ifstrequal{#1}{n_leading_spaces}{\hyperref[sailRISCVznzyleadingzyspaces]{#2}}{}%
  \ifstrequal{#1}{n\_leading\_spaces}{\hyperref[sailRISCVznzyleadingzyspaces]{#2}}{}%
  \ifstrequal{#1}{nan_box_H}{\hyperref[sailRISCVznanzyboxzyH]{#2}}{}%
  \ifstrequal{#1}{nan\_box\_H}{\hyperref[sailRISCVznanzyboxzyH]{#2}}{}%
  \ifstrequal{#1}{nan_box_S}{\hyperref[sailRISCVznanzyboxzyS]{#2}}{}%
  \ifstrequal{#1}{nan\_box\_S}{\hyperref[sailRISCVznanzyboxzyS]{#2}}{}%
  \ifstrequal{#1}{nan_unbox_H}{\hyperref[sailRISCVznanzyunboxzyH]{#2}}{}%
  \ifstrequal{#1}{nan\_unbox\_H}{\hyperref[sailRISCVznanzyunboxzyH]{#2}}{}%
  \ifstrequal{#1}{nan_unbox_S}{\hyperref[sailRISCVznanzyunboxzyS]{#2}}{}%
  \ifstrequal{#1}{nan\_unbox\_S}{\hyperref[sailRISCVznanzyunboxzyS]{#2}}{}%
  \ifstrequal{#1}{negate_D}{\hyperref[sailRISCVznegatezyD]{#2}}{}%
  \ifstrequal{#1}{negate\_D}{\hyperref[sailRISCVznegatezyD]{#2}}{}%
  \ifstrequal{#1}{negate_S}{\hyperref[sailRISCVznegatezyS]{#2}}{}%
  \ifstrequal{#1}{negate\_S}{\hyperref[sailRISCVznegatezyS]{#2}}{}%
  \ifstrequal{#1}{negate_atom}{\hyperref[sailRISCVznegatezyatom]{#2}}{}%
  \ifstrequal{#1}{negate\_atom}{\hyperref[sailRISCVznegatezyatom]{#2}}{}%
  \ifstrequal{#1}{negate_int}{\hyperref[sailRISCVznegatezyint]{#2}}{}%
  \ifstrequal{#1}{negate\_int}{\hyperref[sailRISCVznegatezyint]{#2}}{}%
  \ifstrequal{#1}{neq_anything}{\hyperref[sailRISCVzneqzyanything]{#2}}{}%
  \ifstrequal{#1}{neq\_anything}{\hyperref[sailRISCVzneqzyanything]{#2}}{}%
  \ifstrequal{#1}{neq_bits}{\hyperref[sailRISCVzneqzybits]{#2}}{}%
  \ifstrequal{#1}{neq\_bits}{\hyperref[sailRISCVzneqzybits]{#2}}{}%
  \ifstrequal{#1}{neq_bool}{\hyperref[sailRISCVzneqzybool]{#2}}{}%
  \ifstrequal{#1}{neq\_bool}{\hyperref[sailRISCVzneqzybool]{#2}}{}%
  \ifstrequal{#1}{neq_int}{\hyperref[sailRISCVzneqzyint]{#2}}{}%
  \ifstrequal{#1}{neq\_int}{\hyperref[sailRISCVzneqzyint]{#2}}{}%
  \ifstrequal{#1}{neq_vec}{\hyperref[sailRISCVzneqzyvec]{#2}}{}%
  \ifstrequal{#1}{neq\_vec}{\hyperref[sailRISCVzneqzyvec]{#2}}{}%
  \ifstrequal{#1}{not}{\hyperref[sailRISCVznot]{#2}}{}%
  \ifstrequal{#1}{not_bit}{\hyperref[sailRISCVznotzybit]{#2}}{}%
  \ifstrequal{#1}{not\_bit}{\hyperref[sailRISCVznotzybit]{#2}}{}%
  \ifstrequal{#1}{not_bool}{\hyperref[sailRISCVznotzybool]{#2}}{}%
  \ifstrequal{#1}{not\_bool}{\hyperref[sailRISCVznotzybool]{#2}}{}%
  \ifstrequal{#1}{not_implemented}{\hyperref[sailRISCVznotzyimplemented]{#2}}{}%
  \ifstrequal{#1}{not\_implemented}{\hyperref[sailRISCVznotzyimplemented]{#2}}{}%
  \ifstrequal{#1}{not_vec}{\hyperref[sailRISCVznotzyvec]{#2}}{}%
  \ifstrequal{#1}{not\_vec}{\hyperref[sailRISCVznotzyvec]{#2}}{}%
  \ifstrequal{#1}{num_of_Architecture}{\hyperref[sailRISCVznumzyofzyArchitecture]{#2}}{}%
  \ifstrequal{#1}{num\_of\_Architecture}{\hyperref[sailRISCVznumzyofzyArchitecture]{#2}}{}%
  \ifstrequal{#1}{num_of_CPtrCmpOp}{\hyperref[sailRISCVznumzyofzyCPtrCmpOp]{#2}}{}%
  \ifstrequal{#1}{num\_of\_CPtrCmpOp}{\hyperref[sailRISCVznumzyofzyCPtrCmpOp]{#2}}{}%
  \ifstrequal{#1}{num_of_CapEx}{\hyperref[sailRISCVznumzyofzyCapEx]{#2}}{}%
  \ifstrequal{#1}{num\_of\_CapEx}{\hyperref[sailRISCVznumzyofzyCapEx]{#2}}{}%
  \ifstrequal{#1}{num_of_ClearRegSet}{\hyperref[sailRISCVznumzyofzyClearRegSet]{#2}}{}%
  \ifstrequal{#1}{num\_of\_ClearRegSet}{\hyperref[sailRISCVznumzyofzyClearRegSet]{#2}}{}%
  \ifstrequal{#1}{num_of_ExceptionType}{\hyperref[sailRISCVznumzyofzyExceptionType]{#2}}{}%
  \ifstrequal{#1}{num\_of\_ExceptionType}{\hyperref[sailRISCVznumzyofzyExceptionType]{#2}}{}%
  \ifstrequal{#1}{num_of_ExtStatus}{\hyperref[sailRISCVznumzyofzyExtStatus]{#2}}{}%
  \ifstrequal{#1}{num\_of\_ExtStatus}{\hyperref[sailRISCVznumzyofzyExtStatus]{#2}}{}%
  \ifstrequal{#1}{num_of_InterruptType}{\hyperref[sailRISCVznumzyofzyInterruptType]{#2}}{}%
  \ifstrequal{#1}{num\_of\_InterruptType}{\hyperref[sailRISCVznumzyofzyInterruptType]{#2}}{}%
  \ifstrequal{#1}{num_of_PmpAddrMatchType}{\hyperref[sailRISCVznumzyofzyPmpAddrMatchType]{#2}}{}%
  \ifstrequal{#1}{num\_of\_PmpAddrMatchType}{\hyperref[sailRISCVznumzyofzyPmpAddrMatchType]{#2}}{}%
  \ifstrequal{#1}{num_of_Privilege}{\hyperref[sailRISCVznumzyofzyPrivilege]{#2}}{}%
  \ifstrequal{#1}{num\_of\_Privilege}{\hyperref[sailRISCVznumzyofzyPrivilege]{#2}}{}%
  \ifstrequal{#1}{num_of_Retired}{\hyperref[sailRISCVznumzyofzyRetired]{#2}}{}%
  \ifstrequal{#1}{num\_of\_Retired}{\hyperref[sailRISCVznumzyofzyRetired]{#2}}{}%
  \ifstrequal{#1}{num_of_SATPMode}{\hyperref[sailRISCVznumzyofzySATPMode]{#2}}{}%
  \ifstrequal{#1}{num\_of\_SATPMode}{\hyperref[sailRISCVznumzyofzySATPMode]{#2}}{}%
  \ifstrequal{#1}{num_of_TrapVectorMode}{\hyperref[sailRISCVznumzyofzyTrapVectorMode]{#2}}{}%
  \ifstrequal{#1}{num\_of\_TrapVectorMode}{\hyperref[sailRISCVznumzyofzyTrapVectorMode]{#2}}{}%
  \ifstrequal{#1}{num_of_a64_barrier_domain}{\hyperref[sailRISCVznumzyofzya64zybarrierzydomain]{#2}}{}%
  \ifstrequal{#1}{num\_of\_a64\_barrier\_domain}{\hyperref[sailRISCVznumzyofzya64zybarrierzydomain]{#2}}{}%
  \ifstrequal{#1}{num_of_a64_barrier_type}{\hyperref[sailRISCVznumzyofzya64zybarrierzytype]{#2}}{}%
  \ifstrequal{#1}{num\_of\_a64\_barrier\_type}{\hyperref[sailRISCVznumzyofzya64zybarrierzytype]{#2}}{}%
  \ifstrequal{#1}{num_of_amoop}{\hyperref[sailRISCVznumzyofzyamoop]{#2}}{}%
  \ifstrequal{#1}{num\_of\_amoop}{\hyperref[sailRISCVznumzyofzyamoop]{#2}}{}%
  \ifstrequal{#1}{num_of_biop_zbs}{\hyperref[sailRISCVznumzyofzybiopzyzzbs]{#2}}{}%
  \ifstrequal{#1}{num\_of\_biop\_zbs}{\hyperref[sailRISCVznumzyofzybiopzyzzbs]{#2}}{}%
  \ifstrequal{#1}{num_of_bop}{\hyperref[sailRISCVznumzyofzybop]{#2}}{}%
  \ifstrequal{#1}{num\_of\_bop}{\hyperref[sailRISCVznumzyofzybop]{#2}}{}%
  \ifstrequal{#1}{num_of_brop_zba}{\hyperref[sailRISCVznumzyofzybropzyzzba]{#2}}{}%
  \ifstrequal{#1}{num\_of\_brop\_zba}{\hyperref[sailRISCVznumzyofzybropzyzzba]{#2}}{}%
  \ifstrequal{#1}{num_of_brop_zbb}{\hyperref[sailRISCVznumzyofzybropzyzzbb]{#2}}{}%
  \ifstrequal{#1}{num\_of\_brop\_zbb}{\hyperref[sailRISCVznumzyofzybropzyzzbb]{#2}}{}%
  \ifstrequal{#1}{num_of_brop_zbkb}{\hyperref[sailRISCVznumzyofzybropzyzzbkb]{#2}}{}%
  \ifstrequal{#1}{num\_of\_brop\_zbkb}{\hyperref[sailRISCVznumzyofzybropzyzzbkb]{#2}}{}%
  \ifstrequal{#1}{num_of_brop_zbs}{\hyperref[sailRISCVznumzyofzybropzyzzbs]{#2}}{}%
  \ifstrequal{#1}{num\_of\_brop\_zbs}{\hyperref[sailRISCVznumzyofzybropzyzzbs]{#2}}{}%
  \ifstrequal{#1}{num_of_bropw_zba}{\hyperref[sailRISCVznumzyofzybropwzyzzba]{#2}}{}%
  \ifstrequal{#1}{num\_of\_bropw\_zba}{\hyperref[sailRISCVznumzyofzybropwzyzzba]{#2}}{}%
  \ifstrequal{#1}{num_of_bropw_zbb}{\hyperref[sailRISCVznumzyofzybropwzyzzbb]{#2}}{}%
  \ifstrequal{#1}{num\_of\_bropw\_zbb}{\hyperref[sailRISCVznumzyofzybropwzyzzbb]{#2}}{}%
  \ifstrequal{#1}{num_of_cache_op_kind}{\hyperref[sailRISCVznumzyofzycachezyopzykind]{#2}}{}%
  \ifstrequal{#1}{num\_of\_cache\_op\_kind}{\hyperref[sailRISCVznumzyofzycachezyopzykind]{#2}}{}%
  \ifstrequal{#1}{num_of_csrop}{\hyperref[sailRISCVznumzyofzycsrop]{#2}}{}%
  \ifstrequal{#1}{num\_of\_csrop}{\hyperref[sailRISCVznumzyofzycsrop]{#2}}{}%
  \ifstrequal{#1}{num_of_ext_access_type}{\hyperref[sailRISCVznumzyofzyextzyaccesszytype]{#2}}{}%
  \ifstrequal{#1}{num\_of\_ext\_access\_type}{\hyperref[sailRISCVznumzyofzyextzyaccesszytype]{#2}}{}%
  \ifstrequal{#1}{num_of_ext_exc_type}{\hyperref[sailRISCVznumzyofzyextzyexczytype]{#2}}{}%
  \ifstrequal{#1}{num\_of\_ext\_exc\_type}{\hyperref[sailRISCVznumzyofzyextzyexczytype]{#2}}{}%
  \ifstrequal{#1}{num_of_ext_ptw_error}{\hyperref[sailRISCVznumzyofzyextzyptwzyerror]{#2}}{}%
  \ifstrequal{#1}{num\_of\_ext\_ptw\_error}{\hyperref[sailRISCVznumzyofzyextzyptwzyerror]{#2}}{}%
  \ifstrequal{#1}{num_of_ext_ptw_fail}{\hyperref[sailRISCVznumzyofzyextzyptwzyfail]{#2}}{}%
  \ifstrequal{#1}{num\_of\_ext\_ptw\_fail}{\hyperref[sailRISCVznumzyofzyextzyptwzyfail]{#2}}{}%
  \ifstrequal{#1}{num_of_ext_ptw_lc}{\hyperref[sailRISCVznumzyofzyextzyptwzylc]{#2}}{}%
  \ifstrequal{#1}{num\_of\_ext\_ptw\_lc}{\hyperref[sailRISCVznumzyofzyextzyptwzylc]{#2}}{}%
  \ifstrequal{#1}{num_of_ext_ptw_sc}{\hyperref[sailRISCVznumzyofzyextzyptwzysc]{#2}}{}%
  \ifstrequal{#1}{num\_of\_ext\_ptw\_sc}{\hyperref[sailRISCVznumzyofzyextzyptwzysc]{#2}}{}%
  \ifstrequal{#1}{num_of_extop_zbb}{\hyperref[sailRISCVznumzyofzyextopzyzzbb]{#2}}{}%
  \ifstrequal{#1}{num\_of\_extop\_zbb}{\hyperref[sailRISCVznumzyofzyextopzyzzbb]{#2}}{}%
  \ifstrequal{#1}{num_of_f_bin_op_D}{\hyperref[sailRISCVznumzyofzyfzybinzyopzyD]{#2}}{}%
  \ifstrequal{#1}{num\_of\_f\_bin\_op\_D}{\hyperref[sailRISCVznumzyofzyfzybinzyopzyD]{#2}}{}%
  \ifstrequal{#1}{num_of_f_bin_op_H}{\hyperref[sailRISCVznumzyofzyfzybinzyopzyH]{#2}}{}%
  \ifstrequal{#1}{num\_of\_f\_bin\_op\_H}{\hyperref[sailRISCVznumzyofzyfzybinzyopzyH]{#2}}{}%
  \ifstrequal{#1}{num_of_f_bin_op_S}{\hyperref[sailRISCVznumzyofzyfzybinzyopzyS]{#2}}{}%
  \ifstrequal{#1}{num\_of\_f\_bin\_op\_S}{\hyperref[sailRISCVznumzyofzyfzybinzyopzyS]{#2}}{}%
  \ifstrequal{#1}{num_of_f_bin_rm_op_D}{\hyperref[sailRISCVznumzyofzyfzybinzyrmzyopzyD]{#2}}{}%
  \ifstrequal{#1}{num\_of\_f\_bin\_rm\_op\_D}{\hyperref[sailRISCVznumzyofzyfzybinzyrmzyopzyD]{#2}}{}%
  \ifstrequal{#1}{num_of_f_bin_rm_op_H}{\hyperref[sailRISCVznumzyofzyfzybinzyrmzyopzyH]{#2}}{}%
  \ifstrequal{#1}{num\_of\_f\_bin\_rm\_op\_H}{\hyperref[sailRISCVznumzyofzyfzybinzyrmzyopzyH]{#2}}{}%
  \ifstrequal{#1}{num_of_f_bin_rm_op_S}{\hyperref[sailRISCVznumzyofzyfzybinzyrmzyopzyS]{#2}}{}%
  \ifstrequal{#1}{num\_of\_f\_bin\_rm\_op\_S}{\hyperref[sailRISCVznumzyofzyfzybinzyrmzyopzyS]{#2}}{}%
  \ifstrequal{#1}{num_of_f_madd_op_D}{\hyperref[sailRISCVznumzyofzyfzymaddzyopzyD]{#2}}{}%
  \ifstrequal{#1}{num\_of\_f\_madd\_op\_D}{\hyperref[sailRISCVznumzyofzyfzymaddzyopzyD]{#2}}{}%
  \ifstrequal{#1}{num_of_f_madd_op_H}{\hyperref[sailRISCVznumzyofzyfzymaddzyopzyH]{#2}}{}%
  \ifstrequal{#1}{num\_of\_f\_madd\_op\_H}{\hyperref[sailRISCVznumzyofzyfzymaddzyopzyH]{#2}}{}%
  \ifstrequal{#1}{num_of_f_madd_op_S}{\hyperref[sailRISCVznumzyofzyfzymaddzyopzyS]{#2}}{}%
  \ifstrequal{#1}{num\_of\_f\_madd\_op\_S}{\hyperref[sailRISCVznumzyofzyfzymaddzyopzyS]{#2}}{}%
  \ifstrequal{#1}{num_of_f_un_op_D}{\hyperref[sailRISCVznumzyofzyfzyunzyopzyD]{#2}}{}%
  \ifstrequal{#1}{num\_of\_f\_un\_op\_D}{\hyperref[sailRISCVznumzyofzyfzyunzyopzyD]{#2}}{}%
  \ifstrequal{#1}{num_of_f_un_op_H}{\hyperref[sailRISCVznumzyofzyfzyunzyopzyH]{#2}}{}%
  \ifstrequal{#1}{num\_of\_f\_un\_op\_H}{\hyperref[sailRISCVznumzyofzyfzyunzyopzyH]{#2}}{}%
  \ifstrequal{#1}{num_of_f_un_op_S}{\hyperref[sailRISCVznumzyofzyfzyunzyopzyS]{#2}}{}%
  \ifstrequal{#1}{num\_of\_f\_un\_op\_S}{\hyperref[sailRISCVznumzyofzyfzyunzyopzyS]{#2}}{}%
  \ifstrequal{#1}{num_of_f_un_rm_op_D}{\hyperref[sailRISCVznumzyofzyfzyunzyrmzyopzyD]{#2}}{}%
  \ifstrequal{#1}{num\_of\_f\_un\_rm\_op\_D}{\hyperref[sailRISCVznumzyofzyfzyunzyrmzyopzyD]{#2}}{}%
  \ifstrequal{#1}{num_of_f_un_rm_op_H}{\hyperref[sailRISCVznumzyofzyfzyunzyrmzyopzyH]{#2}}{}%
  \ifstrequal{#1}{num\_of\_f\_un\_rm\_op\_H}{\hyperref[sailRISCVznumzyofzyfzyunzyrmzyopzyH]{#2}}{}%
  \ifstrequal{#1}{num_of_f_un_rm_op_S}{\hyperref[sailRISCVznumzyofzyfzyunzyrmzyopzyS]{#2}}{}%
  \ifstrequal{#1}{num\_of\_f\_un\_rm\_op\_S}{\hyperref[sailRISCVznumzyofzyfzyunzyrmzyopzyS]{#2}}{}%
  \ifstrequal{#1}{num_of_iop}{\hyperref[sailRISCVznumzyofzyiop]{#2}}{}%
  \ifstrequal{#1}{num\_of\_iop}{\hyperref[sailRISCVznumzyofzyiop]{#2}}{}%
  \ifstrequal{#1}{num_of_pmpAddrMatch}{\hyperref[sailRISCVznumzyofzypmpAddrMatch]{#2}}{}%
  \ifstrequal{#1}{num\_of\_pmpAddrMatch}{\hyperref[sailRISCVznumzyofzypmpAddrMatch]{#2}}{}%
  \ifstrequal{#1}{num_of_pmpMatch}{\hyperref[sailRISCVznumzyofzypmpMatch]{#2}}{}%
  \ifstrequal{#1}{num\_of\_pmpMatch}{\hyperref[sailRISCVznumzyofzypmpMatch]{#2}}{}%
  \ifstrequal{#1}{num_of_read_kind}{\hyperref[sailRISCVznumzyofzyreadzykind]{#2}}{}%
  \ifstrequal{#1}{num\_of\_read\_kind}{\hyperref[sailRISCVznumzyofzyreadzykind]{#2}}{}%
  \ifstrequal{#1}{num_of_rop}{\hyperref[sailRISCVznumzyofzyrop]{#2}}{}%
  \ifstrequal{#1}{num\_of\_rop}{\hyperref[sailRISCVznumzyofzyrop]{#2}}{}%
  \ifstrequal{#1}{num_of_ropw}{\hyperref[sailRISCVznumzyofzyropw]{#2}}{}%
  \ifstrequal{#1}{num\_of\_ropw}{\hyperref[sailRISCVznumzyofzyropw]{#2}}{}%
  \ifstrequal{#1}{num_of_rounding_mode}{\hyperref[sailRISCVznumzyofzyroundingzymode]{#2}}{}%
  \ifstrequal{#1}{num\_of\_rounding\_mode}{\hyperref[sailRISCVznumzyofzyroundingzymode]{#2}}{}%
  \ifstrequal{#1}{num_of_seed_opst}{\hyperref[sailRISCVznumzyofzyseedzyopst]{#2}}{}%
  \ifstrequal{#1}{num\_of\_seed\_opst}{\hyperref[sailRISCVznumzyofzyseedzyopst]{#2}}{}%
  \ifstrequal{#1}{num_of_sop}{\hyperref[sailRISCVznumzyofzysop]{#2}}{}%
  \ifstrequal{#1}{num\_of\_sop}{\hyperref[sailRISCVznumzyofzysop]{#2}}{}%
  \ifstrequal{#1}{num_of_sopw}{\hyperref[sailRISCVznumzyofzysopw]{#2}}{}%
  \ifstrequal{#1}{num\_of\_sopw}{\hyperref[sailRISCVznumzyofzysopw]{#2}}{}%
  \ifstrequal{#1}{num_of_trans_kind}{\hyperref[sailRISCVznumzyofzytranszykind]{#2}}{}%
  \ifstrequal{#1}{num\_of\_trans\_kind}{\hyperref[sailRISCVznumzyofzytranszykind]{#2}}{}%
  \ifstrequal{#1}{num_of_uop}{\hyperref[sailRISCVznumzyofzyuop]{#2}}{}%
  \ifstrequal{#1}{num\_of\_uop}{\hyperref[sailRISCVznumzyofzyuop]{#2}}{}%
  \ifstrequal{#1}{num_of_word_width}{\hyperref[sailRISCVznumzyofzywordzywidth]{#2}}{}%
  \ifstrequal{#1}{num\_of\_word\_width}{\hyperref[sailRISCVznumzyofzywordzywidth]{#2}}{}%
  \ifstrequal{#1}{num_of_write_kind}{\hyperref[sailRISCVznumzyofzywritezykind]{#2}}{}%
  \ifstrequal{#1}{num\_of\_write\_kind}{\hyperref[sailRISCVznumzyofzywritezykind]{#2}}{}%
  \ifstrequal{#1}{nvFlag}{\hyperref[sailRISCVznvFlag]{#2}}{}%
  \ifstrequal{#1}{nxFlag}{\hyperref[sailRISCVznxFlag]{#2}}{}%
  \ifstrequal{#1}{ofFlag}{\hyperref[sailRISCVzofFlag]{#2}}{}%
  \ifstrequal{#1}{ones}{\hyperref[sailRISCVzones]{#2}}{}%
  \ifstrequal{#1}{opst_code}{\hyperref[sailRISCVzopstzycode]{#2}}{}%
  \ifstrequal{#1}{opst\_code}{\hyperref[sailRISCVzopstzycode]{#2}}{}%
  \ifstrequal{#1}{opt_spc}{\hyperref[sailRISCVzoptzyspc]{#2}}{}%
  \ifstrequal{#1}{opt\_spc}{\hyperref[sailRISCVzoptzyspc]{#2}}{}%
  \ifstrequal{#1}{opt_spc_backwards}{\hyperref[sailRISCVzoptzyspczybackwards]{#2}}{}%
  \ifstrequal{#1}{opt\_spc\_backwards}{\hyperref[sailRISCVzoptzyspczybackwards]{#2}}{}%
  \ifstrequal{#1}{opt_spc_forwards}{\hyperref[sailRISCVzoptzyspczyforwards]{#2}}{}%
  \ifstrequal{#1}{opt\_spc\_forwards}{\hyperref[sailRISCVzoptzyspczyforwards]{#2}}{}%
  \ifstrequal{#1}{opt_spc_matches_prefix}{\hyperref[sailRISCVzoptzyspczymatcheszyprefix]{#2}}{}%
  \ifstrequal{#1}{opt\_spc\_matches\_prefix}{\hyperref[sailRISCVzoptzyspczymatcheszyprefix]{#2}}{}%
  \ifstrequal{#1}{or_bool}{\hyperref[sailRISCVzorzybool]{#2}}{}%
  \ifstrequal{#1}{or\_bool}{\hyperref[sailRISCVzorzybool]{#2}}{}%
  \ifstrequal{#1}{or_vec}{\hyperref[sailRISCVzorzyvec]{#2}}{}%
  \ifstrequal{#1}{or\_vec}{\hyperref[sailRISCVzorzyvec]{#2}}{}%
  \ifstrequal{#1}{pc_alignment_mask}{\hyperref[sailRISCVzpczyalignmentzymask]{#2}}{}%
  \ifstrequal{#1}{pc\_alignment\_mask}{\hyperref[sailRISCVzpczyalignmentzymask]{#2}}{}%
  \ifstrequal{#1}{pcc_access_system_regs}{\hyperref[sailRISCVzpcczyaccesszysystemzyregs]{#2}}{}%
  \ifstrequal{#1}{pcc\_access\_system\_regs}{\hyperref[sailRISCVzpcczyaccesszysystemzyregs]{#2}}{}%
  \ifstrequal{#1}{phys_mem_read}{\hyperref[sailRISCVzphyszymemzyread]{#2}}{}%
  \ifstrequal{#1}{phys\_mem\_read}{\hyperref[sailRISCVzphyszymemzyread]{#2}}{}%
  \ifstrequal{#1}{phys_mem_segments}{\hyperref[sailRISCVzphyszymemzysegments]{#2}}{}%
  \ifstrequal{#1}{phys\_mem\_segments}{\hyperref[sailRISCVzphyszymemzysegments]{#2}}{}%
  \ifstrequal{#1}{phys_mem_write}{\hyperref[sailRISCVzphyszymemzywrite]{#2}}{}%
  \ifstrequal{#1}{phys\_mem\_write}{\hyperref[sailRISCVzphyszymemzywrite]{#2}}{}%
  \ifstrequal{#1}{plain_vector_access}{\hyperref[sailRISCVzplainzyvectorzyaccess]{#2}}{}%
  \ifstrequal{#1}{plain\_vector\_access}{\hyperref[sailRISCVzplainzyvectorzyaccess]{#2}}{}%
  \ifstrequal{#1}{plain_vector_update}{\hyperref[sailRISCVzplainzyvectorzyupdate]{#2}}{}%
  \ifstrequal{#1}{plain\_vector\_update}{\hyperref[sailRISCVzplainzyvectorzyupdate]{#2}}{}%
  \ifstrequal{#1}{plat_clint_base}{\hyperref[sailRISCVzplatzyclintzybase]{#2}}{}%
  \ifstrequal{#1}{plat\_clint\_base}{\hyperref[sailRISCVzplatzyclintzybase]{#2}}{}%
  \ifstrequal{#1}{plat_clint_size}{\hyperref[sailRISCVzplatzyclintzysizze]{#2}}{}%
  \ifstrequal{#1}{plat\_clint\_size}{\hyperref[sailRISCVzplatzyclintzysizze]{#2}}{}%
  \ifstrequal{#1}{plat_enable_dirty_update}{\hyperref[sailRISCVzplatzyenablezydirtyzyupdate]{#2}}{}%
  \ifstrequal{#1}{plat\_enable\_dirty\_update}{\hyperref[sailRISCVzplatzyenablezydirtyzyupdate]{#2}}{}%
  \ifstrequal{#1}{plat_enable_misaligned_access}{\hyperref[sailRISCVzplatzyenablezymisalignedzyaccess]{#2}}{}%
  \ifstrequal{#1}{plat\_enable\_misaligned\_access}{\hyperref[sailRISCVzplatzyenablezymisalignedzyaccess]{#2}}{}%
  \ifstrequal{#1}{plat_enable_pmp}{\hyperref[sailRISCVzplatzyenablezypmp]{#2}}{}%
  \ifstrequal{#1}{plat\_enable\_pmp}{\hyperref[sailRISCVzplatzyenablezypmp]{#2}}{}%
  \ifstrequal{#1}{plat_htif_tohost}{\hyperref[sailRISCVzplatzyhtifzytohost]{#2}}{}%
  \ifstrequal{#1}{plat\_htif\_tohost}{\hyperref[sailRISCVzplatzyhtifzytohost]{#2}}{}%
  \ifstrequal{#1}{plat_insns_per_tick}{\hyperref[sailRISCVzplatzyinsnszyperzytick]{#2}}{}%
  \ifstrequal{#1}{plat\_insns\_per\_tick}{\hyperref[sailRISCVzplatzyinsnszyperzytick]{#2}}{}%
  \ifstrequal{#1}{plat_mtval_has_illegal_inst_bits}{\hyperref[sailRISCVzplatzymtvalzyhaszyillegalzyinstzybits]{#2}}{}%
  \ifstrequal{#1}{plat\_mtval\_has\_illegal\_inst\_bits}{\hyperref[sailRISCVzplatzymtvalzyhaszyillegalzyinstzybits]{#2}}{}%
  \ifstrequal{#1}{plat_ram_base}{\hyperref[sailRISCVzplatzyramzybase]{#2}}{}%
  \ifstrequal{#1}{plat\_ram\_base}{\hyperref[sailRISCVzplatzyramzybase]{#2}}{}%
  \ifstrequal{#1}{plat_ram_size}{\hyperref[sailRISCVzplatzyramzysizze]{#2}}{}%
  \ifstrequal{#1}{plat\_ram\_size}{\hyperref[sailRISCVzplatzyramzysizze]{#2}}{}%
  \ifstrequal{#1}{plat_rom_base}{\hyperref[sailRISCVzplatzyromzybase]{#2}}{}%
  \ifstrequal{#1}{plat\_rom\_base}{\hyperref[sailRISCVzplatzyromzybase]{#2}}{}%
  \ifstrequal{#1}{plat_rom_size}{\hyperref[sailRISCVzplatzyromzysizze]{#2}}{}%
  \ifstrequal{#1}{plat\_rom\_size}{\hyperref[sailRISCVzplatzyromzysizze]{#2}}{}%
  \ifstrequal{#1}{plat_term_read}{\hyperref[sailRISCVzplatzytermzyread]{#2}}{}%
  \ifstrequal{#1}{plat\_term\_read}{\hyperref[sailRISCVzplatzytermzyread]{#2}}{}%
  \ifstrequal{#1}{plat_term_write}{\hyperref[sailRISCVzplatzytermzywrite]{#2}}{}%
  \ifstrequal{#1}{plat\_term\_write}{\hyperref[sailRISCVzplatzytermzywrite]{#2}}{}%
  \ifstrequal{#1}{platform_wfi}{\hyperref[sailRISCVzplatformzywfi]{#2}}{}%
  \ifstrequal{#1}{platform\_wfi}{\hyperref[sailRISCVzplatformzywfi]{#2}}{}%
  \ifstrequal{#1}{pmpAddrMatchType_of_bits}{\hyperref[sailRISCVzpmpAddrMatchTypezyofzybits]{#2}}{}%
  \ifstrequal{#1}{pmpAddrMatchType\_of\_bits}{\hyperref[sailRISCVzpmpAddrMatchTypezyofzybits]{#2}}{}%
  \ifstrequal{#1}{pmpAddrMatchType_to_bits}{\hyperref[sailRISCVzpmpAddrMatchTypezytozybits]{#2}}{}%
  \ifstrequal{#1}{pmpAddrMatchType\_to\_bits}{\hyperref[sailRISCVzpmpAddrMatchTypezytozybits]{#2}}{}%
  \ifstrequal{#1}{pmpAddrMatch_of_num}{\hyperref[sailRISCVzpmpAddrMatchzyofzynum]{#2}}{}%
  \ifstrequal{#1}{pmpAddrMatch\_of\_num}{\hyperref[sailRISCVzpmpAddrMatchzyofzynum]{#2}}{}%
  \ifstrequal{#1}{pmpAddrRange}{\hyperref[sailRISCVzpmpAddrRange]{#2}}{}%
  \ifstrequal{#1}{pmpCheck}{\hyperref[sailRISCVzpmpCheck]{#2}}{}%
  \ifstrequal{#1}{pmpCheckPerms}{\hyperref[sailRISCVzpmpCheckPerms]{#2}}{}%
  \ifstrequal{#1}{pmpCheckRWX}{\hyperref[sailRISCVzpmpCheckRWX]{#2}}{}%
  \ifstrequal{#1}{pmpLocked}{\hyperref[sailRISCVzpmpLocked]{#2}}{}%
  \ifstrequal{#1}{pmpMatchAddr}{\hyperref[sailRISCVzpmpMatchAddr]{#2}}{}%
  \ifstrequal{#1}{pmpMatchEntry}{\hyperref[sailRISCVzpmpMatchEntry]{#2}}{}%
  \ifstrequal{#1}{pmpMatch_of_num}{\hyperref[sailRISCVzpmpMatchzyofzynum]{#2}}{}%
  \ifstrequal{#1}{pmpMatch\_of\_num}{\hyperref[sailRISCVzpmpMatchzyofzynum]{#2}}{}%
  \ifstrequal{#1}{pmpReadCfgReg}{\hyperref[sailRISCVzpmpReadCfgReg]{#2}}{}%
  \ifstrequal{#1}{pmpTORLocked}{\hyperref[sailRISCVzpmpTORLocked]{#2}}{}%
  \ifstrequal{#1}{pmpWriteAddr}{\hyperref[sailRISCVzpmpWriteAddr]{#2}}{}%
  \ifstrequal{#1}{pmpWriteCfg}{\hyperref[sailRISCVzpmpWriteCfg]{#2}}{}%
  \ifstrequal{#1}{pmpWriteCfgReg}{\hyperref[sailRISCVzpmpWriteCfgReg]{#2}}{}%
  \ifstrequal{#1}{pmp_mem_read}{\hyperref[sailRISCVzpmpzymemzyread]{#2}}{}%
  \ifstrequal{#1}{pmp\_mem\_read}{\hyperref[sailRISCVzpmpzymemzyread]{#2}}{}%
  \ifstrequal{#1}{pmp_mem_write}{\hyperref[sailRISCVzpmpzymemzywrite]{#2}}{}%
  \ifstrequal{#1}{pmp\_mem\_write}{\hyperref[sailRISCVzpmpzymemzywrite]{#2}}{}%
  \ifstrequal{#1}{pow2}{\hyperref[sailRISCVzpow2]{#2}}{}%
  \ifstrequal{#1}{prepare_trap_vector}{\hyperref[sailRISCVzpreparezytrapzyvector]{#2}}{}%
  \ifstrequal{#1}{prepare\_trap\_vector}{\hyperref[sailRISCVzpreparezytrapzyvector]{#2}}{}%
  \ifstrequal{#1}{prepare_xret_target}{\hyperref[sailRISCVzpreparezyxretzytarget]{#2}}{}%
  \ifstrequal{#1}{prepare\_xret\_target}{\hyperref[sailRISCVzpreparezyxretzytarget]{#2}}{}%
  \ifstrequal{#1}{prerr_bits}{\hyperref[sailRISCVzprerrzybits]{#2}}{}%
  \ifstrequal{#1}{prerr\_bits}{\hyperref[sailRISCVzprerrzybits]{#2}}{}%
  \ifstrequal{#1}{prerr_endline}{\hyperref[sailRISCVzprerrzyendline]{#2}}{}%
  \ifstrequal{#1}{prerr\_endline}{\hyperref[sailRISCVzprerrzyendline]{#2}}{}%
  \ifstrequal{#1}{prerr_int}{\hyperref[sailRISCVzprerrzyint]{#2}}{}%
  \ifstrequal{#1}{prerr\_int}{\hyperref[sailRISCVzprerrzyint]{#2}}{}%
  \ifstrequal{#1}{print}{\hyperref[sailRISCVzprint]{#2}}{}%
  \ifstrequal{#1}{print_bits}{\hyperref[sailRISCVzprintzybits]{#2}}{}%
  \ifstrequal{#1}{print\_bits}{\hyperref[sailRISCVzprintzybits]{#2}}{}%
  \ifstrequal{#1}{print_endline}{\hyperref[sailRISCVzprintzyendline]{#2}}{}%
  \ifstrequal{#1}{print\_endline}{\hyperref[sailRISCVzprintzyendline]{#2}}{}%
  \ifstrequal{#1}{print_insn}{\hyperref[sailRISCVzprintzyinsn]{#2}}{}%
  \ifstrequal{#1}{print\_insn}{\hyperref[sailRISCVzprintzyinsn]{#2}}{}%
  \ifstrequal{#1}{print_instr}{\hyperref[sailRISCVzprintzyinstr]{#2}}{}%
  \ifstrequal{#1}{print\_instr}{\hyperref[sailRISCVzprintzyinstr]{#2}}{}%
  \ifstrequal{#1}{print_int}{\hyperref[sailRISCVzprintzyint]{#2}}{}%
  \ifstrequal{#1}{print\_int}{\hyperref[sailRISCVzprintzyint]{#2}}{}%
  \ifstrequal{#1}{print_mem}{\hyperref[sailRISCVzprintzymem]{#2}}{}%
  \ifstrequal{#1}{print\_mem}{\hyperref[sailRISCVzprintzymem]{#2}}{}%
  \ifstrequal{#1}{print_platform}{\hyperref[sailRISCVzprintzyplatform]{#2}}{}%
  \ifstrequal{#1}{print\_platform}{\hyperref[sailRISCVzprintzyplatform]{#2}}{}%
  \ifstrequal{#1}{print_reg}{\hyperref[sailRISCVzprintzyreg]{#2}}{}%
  \ifstrequal{#1}{print\_reg}{\hyperref[sailRISCVzprintzyreg]{#2}}{}%
  \ifstrequal{#1}{print_string}{\hyperref[sailRISCVzprintzystring]{#2}}{}%
  \ifstrequal{#1}{print\_string}{\hyperref[sailRISCVzprintzystring]{#2}}{}%
  \ifstrequal{#1}{privLevel_of_bits}{\hyperref[sailRISCVzprivLevelzyofzybits]{#2}}{}%
  \ifstrequal{#1}{privLevel\_of\_bits}{\hyperref[sailRISCVzprivLevelzyofzybits]{#2}}{}%
  \ifstrequal{#1}{privLevel_to_bits}{\hyperref[sailRISCVzprivLevelzytozybits]{#2}}{}%
  \ifstrequal{#1}{privLevel\_to\_bits}{\hyperref[sailRISCVzprivLevelzytozybits]{#2}}{}%
  \ifstrequal{#1}{privLevel_to_str}{\hyperref[sailRISCVzprivLevelzytozystr]{#2}}{}%
  \ifstrequal{#1}{privLevel\_to\_str}{\hyperref[sailRISCVzprivLevelzytozystr]{#2}}{}%
  \ifstrequal{#1}{processPending}{\hyperref[sailRISCVzprocessPending]{#2}}{}%
  \ifstrequal{#1}{process_fload16}{\hyperref[sailRISCVzprocesszyfload16]{#2}}{}%
  \ifstrequal{#1}{process\_fload16}{\hyperref[sailRISCVzprocesszyfload16]{#2}}{}%
  \ifstrequal{#1}{process_fload32}{\hyperref[sailRISCVzprocesszyfload32]{#2}}{}%
  \ifstrequal{#1}{process\_fload32}{\hyperref[sailRISCVzprocesszyfload32]{#2}}{}%
  \ifstrequal{#1}{process_fload64}{\hyperref[sailRISCVzprocesszyfload64]{#2}}{}%
  \ifstrequal{#1}{process\_fload64}{\hyperref[sailRISCVzprocesszyfload64]{#2}}{}%
  \ifstrequal{#1}{process_fstore}{\hyperref[sailRISCVzprocesszyfstore]{#2}}{}%
  \ifstrequal{#1}{process\_fstore}{\hyperref[sailRISCVzprocesszyfstore]{#2}}{}%
  \ifstrequal{#1}{process_load}{\hyperref[sailRISCVzprocesszyload]{#2}}{}%
  \ifstrequal{#1}{process\_load}{\hyperref[sailRISCVzprocesszyload]{#2}}{}%
  \ifstrequal{#1}{process_loadres}{\hyperref[sailRISCVzprocesszyloadres]{#2}}{}%
  \ifstrequal{#1}{process\_loadres}{\hyperref[sailRISCVzprocesszyloadres]{#2}}{}%
  \ifstrequal{#1}{ptw_error_to_str}{\hyperref[sailRISCVzptwzyerrorzytozystr]{#2}}{}%
  \ifstrequal{#1}{ptw\_error\_to\_str}{\hyperref[sailRISCVzptwzyerrorzytozystr]{#2}}{}%
  \ifstrequal{#1}{quot_round_zero}{\hyperref[sailRISCVzquotzyroundzyzzero]{#2}}{}%
  \ifstrequal{#1}{quot\_round\_zero}{\hyperref[sailRISCVzquotzyroundzyzzero]{#2}}{}%
  \ifstrequal{#1}{rC}{\hyperref[sailRISCVzrC]{#2}}{}%
  \ifstrequal{#1}{rC_bits}{\hyperref[sailRISCVzrCzybits]{#2}}{}%
  \ifstrequal{#1}{rC\_bits}{\hyperref[sailRISCVzrCzybits]{#2}}{}%
  \ifstrequal{#1}{rF}{\hyperref[sailRISCVzrF]{#2}}{}%
  \ifstrequal{#1}{rF_bits}{\hyperref[sailRISCVzrFzybits]{#2}}{}%
  \ifstrequal{#1}{rF\_bits}{\hyperref[sailRISCVzrFzybits]{#2}}{}%
  \ifstrequal{#1}{rF_or_X_D}{\hyperref[sailRISCVzrFzyorzyXzyD]{#2}}{}%
  \ifstrequal{#1}{rF\_or\_X\_D}{\hyperref[sailRISCVzrFzyorzyXzyD]{#2}}{}%
  \ifstrequal{#1}{rF_or_X_H}{\hyperref[sailRISCVzrFzyorzyXzyH]{#2}}{}%
  \ifstrequal{#1}{rF\_or\_X\_H}{\hyperref[sailRISCVzrFzyorzyXzyH]{#2}}{}%
  \ifstrequal{#1}{rF_or_X_S}{\hyperref[sailRISCVzrFzyorzyXzyS]{#2}}{}%
  \ifstrequal{#1}{rF\_or\_X\_S}{\hyperref[sailRISCVzrFzyorzyXzyS]{#2}}{}%
  \ifstrequal{#1}{rX}{\hyperref[sailRISCVzrX]{#2}}{}%
  \ifstrequal{#1}{rX_bits}{\hyperref[sailRISCVzrXzybits]{#2}}{}%
  \ifstrequal{#1}{rX\_bits}{\hyperref[sailRISCVzrXzybits]{#2}}{}%
  \ifstrequal{#1}{readCSR}{\hyperref[sailRISCVzreadCSR]{#2}}{}%
  \ifstrequal{#1}{read_kind_of_flags}{\hyperref[sailRISCVzreadzykindzyofzyflags]{#2}}{}%
  \ifstrequal{#1}{read\_kind\_of\_flags}{\hyperref[sailRISCVzreadzykindzyofzyflags]{#2}}{}%
  \ifstrequal{#1}{read_kind_of_num}{\hyperref[sailRISCVzreadzykindzyofzynum]{#2}}{}%
  \ifstrequal{#1}{read\_kind\_of\_num}{\hyperref[sailRISCVzreadzykindzyofzynum]{#2}}{}%
  \ifstrequal{#1}{read_ram}{\hyperref[sailRISCVzreadzyram]{#2}}{}%
  \ifstrequal{#1}{read\_ram}{\hyperref[sailRISCVzreadzyram]{#2}}{}%
  \ifstrequal{#1}{read_seed_csr}{\hyperref[sailRISCVzreadzyseedzycsr]{#2}}{}%
  \ifstrequal{#1}{read\_seed\_csr}{\hyperref[sailRISCVzreadzyseedzycsr]{#2}}{}%
  \ifstrequal{#1}{reg_deref}{\hyperref[sailRISCVzregzyderef]{#2}}{}%
  \ifstrequal{#1}{reg\_deref}{\hyperref[sailRISCVzregzyderef]{#2}}{}%
  \ifstrequal{#1}{reg_name}{\hyperref[sailRISCVzregzyname]{#2}}{}%
  \ifstrequal{#1}{reg\_name}{\hyperref[sailRISCVzregzyname]{#2}}{}%
  \ifstrequal{#1}{reg_name_abi}{\hyperref[sailRISCVzregzynamezyabi]{#2}}{}%
  \ifstrequal{#1}{reg\_name\_abi}{\hyperref[sailRISCVzregzynamezyabi]{#2}}{}%
  \ifstrequal{#1}{regidx_to_regno}{\hyperref[sailRISCVzregidxzytozyregno]{#2}}{}%
  \ifstrequal{#1}{regidx\_to\_regno}{\hyperref[sailRISCVzregidxzytozyregno]{#2}}{}%
  \ifstrequal{#1}{regval_from_reg}{\hyperref[sailRISCVzregvalzyfromzyreg]{#2}}{}%
  \ifstrequal{#1}{regval\_from\_reg}{\hyperref[sailRISCVzregvalzyfromzyreg]{#2}}{}%
  \ifstrequal{#1}{regval_into_reg}{\hyperref[sailRISCVzregvalzyintozyreg]{#2}}{}%
  \ifstrequal{#1}{regval\_into\_reg}{\hyperref[sailRISCVzregvalzyintozyreg]{#2}}{}%
  \ifstrequal{#1}{rem_round_zero}{\hyperref[sailRISCVzremzyroundzyzzero]{#2}}{}%
  \ifstrequal{#1}{rem\_round\_zero}{\hyperref[sailRISCVzremzyroundzyzzero]{#2}}{}%
  \ifstrequal{#1}{replicate_bits}{\hyperref[sailRISCVzreplicatezybits]{#2}}{}%
  \ifstrequal{#1}{replicate\_bits}{\hyperref[sailRISCVzreplicatezybits]{#2}}{}%
  \ifstrequal{#1}{reset_htif}{\hyperref[sailRISCVzresetzyhtif]{#2}}{}%
  \ifstrequal{#1}{reset\_htif}{\hyperref[sailRISCVzresetzyhtif]{#2}}{}%
  \ifstrequal{#1}{retire_instruction}{\hyperref[sailRISCVzretirezyinstruction]{#2}}{}%
  \ifstrequal{#1}{retire\_instruction}{\hyperref[sailRISCVzretirezyinstruction]{#2}}{}%
  \ifstrequal{#1}{reverse_bits_in_byte}{\hyperref[sailRISCVzreversezybitszyinzybyte]{#2}}{}%
  \ifstrequal{#1}{reverse\_bits\_in\_byte}{\hyperref[sailRISCVzreversezybitszyinzybyte]{#2}}{}%
  \ifstrequal{#1}{riscv_f16Add}{\hyperref[sailRISCVzriscvzyf16Add]{#2}}{}%
  \ifstrequal{#1}{riscv\_f16Add}{\hyperref[sailRISCVzriscvzyf16Add]{#2}}{}%
  \ifstrequal{#1}{riscv_f16Div}{\hyperref[sailRISCVzriscvzyf16Div]{#2}}{}%
  \ifstrequal{#1}{riscv\_f16Div}{\hyperref[sailRISCVzriscvzyf16Div]{#2}}{}%
  \ifstrequal{#1}{riscv_f16Eq}{\hyperref[sailRISCVzriscvzyf16Eq]{#2}}{}%
  \ifstrequal{#1}{riscv\_f16Eq}{\hyperref[sailRISCVzriscvzyf16Eq]{#2}}{}%
  \ifstrequal{#1}{riscv_f16Le}{\hyperref[sailRISCVzriscvzyf16Le]{#2}}{}%
  \ifstrequal{#1}{riscv\_f16Le}{\hyperref[sailRISCVzriscvzyf16Le]{#2}}{}%
  \ifstrequal{#1}{riscv_f16Lt}{\hyperref[sailRISCVzriscvzyf16Lt]{#2}}{}%
  \ifstrequal{#1}{riscv\_f16Lt}{\hyperref[sailRISCVzriscvzyf16Lt]{#2}}{}%
  \ifstrequal{#1}{riscv_f16Mul}{\hyperref[sailRISCVzriscvzyf16Mul]{#2}}{}%
  \ifstrequal{#1}{riscv\_f16Mul}{\hyperref[sailRISCVzriscvzyf16Mul]{#2}}{}%
  \ifstrequal{#1}{riscv_f16MulAdd}{\hyperref[sailRISCVzriscvzyf16MulAdd]{#2}}{}%
  \ifstrequal{#1}{riscv\_f16MulAdd}{\hyperref[sailRISCVzriscvzyf16MulAdd]{#2}}{}%
  \ifstrequal{#1}{riscv_f16Sqrt}{\hyperref[sailRISCVzriscvzyf16Sqrt]{#2}}{}%
  \ifstrequal{#1}{riscv\_f16Sqrt}{\hyperref[sailRISCVzriscvzyf16Sqrt]{#2}}{}%
  \ifstrequal{#1}{riscv_f16Sub}{\hyperref[sailRISCVzriscvzyf16Sub]{#2}}{}%
  \ifstrequal{#1}{riscv\_f16Sub}{\hyperref[sailRISCVzriscvzyf16Sub]{#2}}{}%
  \ifstrequal{#1}{riscv_f16ToF32}{\hyperref[sailRISCVzriscvzyf16ToF32]{#2}}{}%
  \ifstrequal{#1}{riscv\_f16ToF32}{\hyperref[sailRISCVzriscvzyf16ToF32]{#2}}{}%
  \ifstrequal{#1}{riscv_f16ToF64}{\hyperref[sailRISCVzriscvzyf16ToF64]{#2}}{}%
  \ifstrequal{#1}{riscv\_f16ToF64}{\hyperref[sailRISCVzriscvzyf16ToF64]{#2}}{}%
  \ifstrequal{#1}{riscv_f16ToI32}{\hyperref[sailRISCVzriscvzyf16ToI32]{#2}}{}%
  \ifstrequal{#1}{riscv\_f16ToI32}{\hyperref[sailRISCVzriscvzyf16ToI32]{#2}}{}%
  \ifstrequal{#1}{riscv_f16ToI64}{\hyperref[sailRISCVzriscvzyf16ToI64]{#2}}{}%
  \ifstrequal{#1}{riscv\_f16ToI64}{\hyperref[sailRISCVzriscvzyf16ToI64]{#2}}{}%
  \ifstrequal{#1}{riscv_f16ToUi32}{\hyperref[sailRISCVzriscvzyf16ToUi32]{#2}}{}%
  \ifstrequal{#1}{riscv\_f16ToUi32}{\hyperref[sailRISCVzriscvzyf16ToUi32]{#2}}{}%
  \ifstrequal{#1}{riscv_f16ToUi64}{\hyperref[sailRISCVzriscvzyf16ToUi64]{#2}}{}%
  \ifstrequal{#1}{riscv\_f16ToUi64}{\hyperref[sailRISCVzriscvzyf16ToUi64]{#2}}{}%
  \ifstrequal{#1}{riscv_f32Add}{\hyperref[sailRISCVzriscvzyf32Add]{#2}}{}%
  \ifstrequal{#1}{riscv\_f32Add}{\hyperref[sailRISCVzriscvzyf32Add]{#2}}{}%
  \ifstrequal{#1}{riscv_f32Div}{\hyperref[sailRISCVzriscvzyf32Div]{#2}}{}%
  \ifstrequal{#1}{riscv\_f32Div}{\hyperref[sailRISCVzriscvzyf32Div]{#2}}{}%
  \ifstrequal{#1}{riscv_f32Eq}{\hyperref[sailRISCVzriscvzyf32Eq]{#2}}{}%
  \ifstrequal{#1}{riscv\_f32Eq}{\hyperref[sailRISCVzriscvzyf32Eq]{#2}}{}%
  \ifstrequal{#1}{riscv_f32Le}{\hyperref[sailRISCVzriscvzyf32Le]{#2}}{}%
  \ifstrequal{#1}{riscv\_f32Le}{\hyperref[sailRISCVzriscvzyf32Le]{#2}}{}%
  \ifstrequal{#1}{riscv_f32Lt}{\hyperref[sailRISCVzriscvzyf32Lt]{#2}}{}%
  \ifstrequal{#1}{riscv\_f32Lt}{\hyperref[sailRISCVzriscvzyf32Lt]{#2}}{}%
  \ifstrequal{#1}{riscv_f32Mul}{\hyperref[sailRISCVzriscvzyf32Mul]{#2}}{}%
  \ifstrequal{#1}{riscv\_f32Mul}{\hyperref[sailRISCVzriscvzyf32Mul]{#2}}{}%
  \ifstrequal{#1}{riscv_f32MulAdd}{\hyperref[sailRISCVzriscvzyf32MulAdd]{#2}}{}%
  \ifstrequal{#1}{riscv\_f32MulAdd}{\hyperref[sailRISCVzriscvzyf32MulAdd]{#2}}{}%
  \ifstrequal{#1}{riscv_f32Sqrt}{\hyperref[sailRISCVzriscvzyf32Sqrt]{#2}}{}%
  \ifstrequal{#1}{riscv\_f32Sqrt}{\hyperref[sailRISCVzriscvzyf32Sqrt]{#2}}{}%
  \ifstrequal{#1}{riscv_f32Sub}{\hyperref[sailRISCVzriscvzyf32Sub]{#2}}{}%
  \ifstrequal{#1}{riscv\_f32Sub}{\hyperref[sailRISCVzriscvzyf32Sub]{#2}}{}%
  \ifstrequal{#1}{riscv_f32ToF16}{\hyperref[sailRISCVzriscvzyf32ToF16]{#2}}{}%
  \ifstrequal{#1}{riscv\_f32ToF16}{\hyperref[sailRISCVzriscvzyf32ToF16]{#2}}{}%
  \ifstrequal{#1}{riscv_f32ToF64}{\hyperref[sailRISCVzriscvzyf32ToF64]{#2}}{}%
  \ifstrequal{#1}{riscv\_f32ToF64}{\hyperref[sailRISCVzriscvzyf32ToF64]{#2}}{}%
  \ifstrequal{#1}{riscv_f32ToI32}{\hyperref[sailRISCVzriscvzyf32ToI32]{#2}}{}%
  \ifstrequal{#1}{riscv\_f32ToI32}{\hyperref[sailRISCVzriscvzyf32ToI32]{#2}}{}%
  \ifstrequal{#1}{riscv_f32ToI64}{\hyperref[sailRISCVzriscvzyf32ToI64]{#2}}{}%
  \ifstrequal{#1}{riscv\_f32ToI64}{\hyperref[sailRISCVzriscvzyf32ToI64]{#2}}{}%
  \ifstrequal{#1}{riscv_f32ToUi32}{\hyperref[sailRISCVzriscvzyf32ToUi32]{#2}}{}%
  \ifstrequal{#1}{riscv\_f32ToUi32}{\hyperref[sailRISCVzriscvzyf32ToUi32]{#2}}{}%
  \ifstrequal{#1}{riscv_f32ToUi64}{\hyperref[sailRISCVzriscvzyf32ToUi64]{#2}}{}%
  \ifstrequal{#1}{riscv\_f32ToUi64}{\hyperref[sailRISCVzriscvzyf32ToUi64]{#2}}{}%
  \ifstrequal{#1}{riscv_f64Add}{\hyperref[sailRISCVzriscvzyf64Add]{#2}}{}%
  \ifstrequal{#1}{riscv\_f64Add}{\hyperref[sailRISCVzriscvzyf64Add]{#2}}{}%
  \ifstrequal{#1}{riscv_f64Div}{\hyperref[sailRISCVzriscvzyf64Div]{#2}}{}%
  \ifstrequal{#1}{riscv\_f64Div}{\hyperref[sailRISCVzriscvzyf64Div]{#2}}{}%
  \ifstrequal{#1}{riscv_f64Eq}{\hyperref[sailRISCVzriscvzyf64Eq]{#2}}{}%
  \ifstrequal{#1}{riscv\_f64Eq}{\hyperref[sailRISCVzriscvzyf64Eq]{#2}}{}%
  \ifstrequal{#1}{riscv_f64Le}{\hyperref[sailRISCVzriscvzyf64Le]{#2}}{}%
  \ifstrequal{#1}{riscv\_f64Le}{\hyperref[sailRISCVzriscvzyf64Le]{#2}}{}%
  \ifstrequal{#1}{riscv_f64Lt}{\hyperref[sailRISCVzriscvzyf64Lt]{#2}}{}%
  \ifstrequal{#1}{riscv\_f64Lt}{\hyperref[sailRISCVzriscvzyf64Lt]{#2}}{}%
  \ifstrequal{#1}{riscv_f64Mul}{\hyperref[sailRISCVzriscvzyf64Mul]{#2}}{}%
  \ifstrequal{#1}{riscv\_f64Mul}{\hyperref[sailRISCVzriscvzyf64Mul]{#2}}{}%
  \ifstrequal{#1}{riscv_f64MulAdd}{\hyperref[sailRISCVzriscvzyf64MulAdd]{#2}}{}%
  \ifstrequal{#1}{riscv\_f64MulAdd}{\hyperref[sailRISCVzriscvzyf64MulAdd]{#2}}{}%
  \ifstrequal{#1}{riscv_f64Sqrt}{\hyperref[sailRISCVzriscvzyf64Sqrt]{#2}}{}%
  \ifstrequal{#1}{riscv\_f64Sqrt}{\hyperref[sailRISCVzriscvzyf64Sqrt]{#2}}{}%
  \ifstrequal{#1}{riscv_f64Sub}{\hyperref[sailRISCVzriscvzyf64Sub]{#2}}{}%
  \ifstrequal{#1}{riscv\_f64Sub}{\hyperref[sailRISCVzriscvzyf64Sub]{#2}}{}%
  \ifstrequal{#1}{riscv_f64ToF16}{\hyperref[sailRISCVzriscvzyf64ToF16]{#2}}{}%
  \ifstrequal{#1}{riscv\_f64ToF16}{\hyperref[sailRISCVzriscvzyf64ToF16]{#2}}{}%
  \ifstrequal{#1}{riscv_f64ToF32}{\hyperref[sailRISCVzriscvzyf64ToF32]{#2}}{}%
  \ifstrequal{#1}{riscv\_f64ToF32}{\hyperref[sailRISCVzriscvzyf64ToF32]{#2}}{}%
  \ifstrequal{#1}{riscv_f64ToI32}{\hyperref[sailRISCVzriscvzyf64ToI32]{#2}}{}%
  \ifstrequal{#1}{riscv\_f64ToI32}{\hyperref[sailRISCVzriscvzyf64ToI32]{#2}}{}%
  \ifstrequal{#1}{riscv_f64ToI64}{\hyperref[sailRISCVzriscvzyf64ToI64]{#2}}{}%
  \ifstrequal{#1}{riscv\_f64ToI64}{\hyperref[sailRISCVzriscvzyf64ToI64]{#2}}{}%
  \ifstrequal{#1}{riscv_f64ToUi32}{\hyperref[sailRISCVzriscvzyf64ToUi32]{#2}}{}%
  \ifstrequal{#1}{riscv\_f64ToUi32}{\hyperref[sailRISCVzriscvzyf64ToUi32]{#2}}{}%
  \ifstrequal{#1}{riscv_f64ToUi64}{\hyperref[sailRISCVzriscvzyf64ToUi64]{#2}}{}%
  \ifstrequal{#1}{riscv\_f64ToUi64}{\hyperref[sailRISCVzriscvzyf64ToUi64]{#2}}{}%
  \ifstrequal{#1}{riscv_i32ToF16}{\hyperref[sailRISCVzriscvzyi32ToF16]{#2}}{}%
  \ifstrequal{#1}{riscv\_i32ToF16}{\hyperref[sailRISCVzriscvzyi32ToF16]{#2}}{}%
  \ifstrequal{#1}{riscv_i32ToF32}{\hyperref[sailRISCVzriscvzyi32ToF32]{#2}}{}%
  \ifstrequal{#1}{riscv\_i32ToF32}{\hyperref[sailRISCVzriscvzyi32ToF32]{#2}}{}%
  \ifstrequal{#1}{riscv_i32ToF64}{\hyperref[sailRISCVzriscvzyi32ToF64]{#2}}{}%
  \ifstrequal{#1}{riscv\_i32ToF64}{\hyperref[sailRISCVzriscvzyi32ToF64]{#2}}{}%
  \ifstrequal{#1}{riscv_i64ToF16}{\hyperref[sailRISCVzriscvzyi64ToF16]{#2}}{}%
  \ifstrequal{#1}{riscv\_i64ToF16}{\hyperref[sailRISCVzriscvzyi64ToF16]{#2}}{}%
  \ifstrequal{#1}{riscv_i64ToF32}{\hyperref[sailRISCVzriscvzyi64ToF32]{#2}}{}%
  \ifstrequal{#1}{riscv\_i64ToF32}{\hyperref[sailRISCVzriscvzyi64ToF32]{#2}}{}%
  \ifstrequal{#1}{riscv_i64ToF64}{\hyperref[sailRISCVzriscvzyi64ToF64]{#2}}{}%
  \ifstrequal{#1}{riscv\_i64ToF64}{\hyperref[sailRISCVzriscvzyi64ToF64]{#2}}{}%
  \ifstrequal{#1}{riscv_ui32ToF16}{\hyperref[sailRISCVzriscvzyui32ToF16]{#2}}{}%
  \ifstrequal{#1}{riscv\_ui32ToF16}{\hyperref[sailRISCVzriscvzyui32ToF16]{#2}}{}%
  \ifstrequal{#1}{riscv_ui32ToF32}{\hyperref[sailRISCVzriscvzyui32ToF32]{#2}}{}%
  \ifstrequal{#1}{riscv\_ui32ToF32}{\hyperref[sailRISCVzriscvzyui32ToF32]{#2}}{}%
  \ifstrequal{#1}{riscv_ui32ToF64}{\hyperref[sailRISCVzriscvzyui32ToF64]{#2}}{}%
  \ifstrequal{#1}{riscv\_ui32ToF64}{\hyperref[sailRISCVzriscvzyui32ToF64]{#2}}{}%
  \ifstrequal{#1}{riscv_ui64ToF16}{\hyperref[sailRISCVzriscvzyui64ToF16]{#2}}{}%
  \ifstrequal{#1}{riscv\_ui64ToF16}{\hyperref[sailRISCVzriscvzyui64ToF16]{#2}}{}%
  \ifstrequal{#1}{riscv_ui64ToF32}{\hyperref[sailRISCVzriscvzyui64ToF32]{#2}}{}%
  \ifstrequal{#1}{riscv\_ui64ToF32}{\hyperref[sailRISCVzriscvzyui64ToF32]{#2}}{}%
  \ifstrequal{#1}{riscv_ui64ToF64}{\hyperref[sailRISCVzriscvzyui64ToF64]{#2}}{}%
  \ifstrequal{#1}{riscv\_ui64ToF64}{\hyperref[sailRISCVzriscvzyui64ToF64]{#2}}{}%
  \ifstrequal{#1}{rop_of_num}{\hyperref[sailRISCVzropzyofzynum]{#2}}{}%
  \ifstrequal{#1}{rop\_of\_num}{\hyperref[sailRISCVzropzyofzynum]{#2}}{}%
  \ifstrequal{#1}{ropw_of_num}{\hyperref[sailRISCVzropwzyofzynum]{#2}}{}%
  \ifstrequal{#1}{ropw\_of\_num}{\hyperref[sailRISCVzropwzyofzynum]{#2}}{}%
  \ifstrequal{#1}{rotate_bits_left}{\hyperref[sailRISCVzrotatezybitszyleft]{#2}}{}%
  \ifstrequal{#1}{rotate\_bits\_left}{\hyperref[sailRISCVzrotatezybitszyleft]{#2}}{}%
  \ifstrequal{#1}{rotate_bits_right}{\hyperref[sailRISCVzrotatezybitszyright]{#2}}{}%
  \ifstrequal{#1}{rotate\_bits\_right}{\hyperref[sailRISCVzrotatezybitszyright]{#2}}{}%
  \ifstrequal{#1}{rotatel}{\hyperref[sailRISCVzrotatel]{#2}}{}%
  \ifstrequal{#1}{rotater}{\hyperref[sailRISCVzrotater]{#2}}{}%
  \ifstrequal{#1}{rounding_mode_of_num}{\hyperref[sailRISCVzroundingzymodezyofzynum]{#2}}{}%
  \ifstrequal{#1}{rounding\_mode\_of\_num}{\hyperref[sailRISCVzroundingzymodezyofzynum]{#2}}{}%
  \ifstrequal{#1}{rtype_mnemonic}{\hyperref[sailRISCVzrtypezymnemonic]{#2}}{}%
  \ifstrequal{#1}{rtype\_mnemonic}{\hyperref[sailRISCVzrtypezymnemonic]{#2}}{}%
  \ifstrequal{#1}{rtypew_mnemonic}{\hyperref[sailRISCVzrtypewzymnemonic]{#2}}{}%
  \ifstrequal{#1}{rtypew\_mnemonic}{\hyperref[sailRISCVzrtypewzymnemonic]{#2}}{}%
  \ifstrequal{#1}{rvfi_read}{\hyperref[sailRISCVzrvfizyread]{#2}}{}%
  \ifstrequal{#1}{rvfi\_read}{\hyperref[sailRISCVzrvfizyread]{#2}}{}%
  \ifstrequal{#1}{rvfi_trap}{\hyperref[sailRISCVzrvfizytrap]{#2}}{}%
  \ifstrequal{#1}{rvfi\_trap}{\hyperref[sailRISCVzrvfizytrap]{#2}}{}%
  \ifstrequal{#1}{rvfi_wX}{\hyperref[sailRISCVzrvfizywX]{#2}}{}%
  \ifstrequal{#1}{rvfi\_wX}{\hyperref[sailRISCVzrvfizywX]{#2}}{}%
  \ifstrequal{#1}{rvfi_write}{\hyperref[sailRISCVzrvfizywrite]{#2}}{}%
  \ifstrequal{#1}{rvfi\_write}{\hyperref[sailRISCVzrvfizywrite]{#2}}{}%
  \ifstrequal{#1}{sail_arith_shiftright}{\hyperref[sailRISCVzsailzyarithzyshiftright]{#2}}{}%
  \ifstrequal{#1}{sail\_arith\_shiftright}{\hyperref[sailRISCVzsailzyarithzyshiftright]{#2}}{}%
  \ifstrequal{#1}{sail_mask}{\hyperref[sailRISCVzsailzymask]{#2}}{}%
  \ifstrequal{#1}{sail\_mask}{\hyperref[sailRISCVzsailzymask]{#2}}{}%
  \ifstrequal{#1}{sail_ones}{\hyperref[sailRISCVzsailzyones]{#2}}{}%
  \ifstrequal{#1}{sail\_ones}{\hyperref[sailRISCVzsailzyones]{#2}}{}%
  \ifstrequal{#1}{sail_shiftleft}{\hyperref[sailRISCVzsailzyshiftleft]{#2}}{}%
  \ifstrequal{#1}{sail\_shiftleft}{\hyperref[sailRISCVzsailzyshiftleft]{#2}}{}%
  \ifstrequal{#1}{sail_shiftright}{\hyperref[sailRISCVzsailzyshiftright]{#2}}{}%
  \ifstrequal{#1}{sail\_shiftright}{\hyperref[sailRISCVzsailzyshiftright]{#2}}{}%
  \ifstrequal{#1}{sail_sign_extend}{\hyperref[sailRISCVzsailzysignzyextend]{#2}}{}%
  \ifstrequal{#1}{sail\_sign\_extend}{\hyperref[sailRISCVzsailzysignzyextend]{#2}}{}%
  \ifstrequal{#1}{sail_zero_extend}{\hyperref[sailRISCVzsailzyzzerozyextend]{#2}}{}%
  \ifstrequal{#1}{sail\_zero\_extend}{\hyperref[sailRISCVzsailzyzzerozyextend]{#2}}{}%
  \ifstrequal{#1}{sail_zeros}{\hyperref[sailRISCVzsailzyzzeros]{#2}}{}%
  \ifstrequal{#1}{sail\_zeros}{\hyperref[sailRISCVzsailzyzzeros]{#2}}{}%
  \ifstrequal{#1}{satp64Mode_of_bits}{\hyperref[sailRISCVzsatp64Modezyofzybits]{#2}}{}%
  \ifstrequal{#1}{satp64Mode\_of\_bits}{\hyperref[sailRISCVzsatp64Modezyofzybits]{#2}}{}%
  \ifstrequal{#1}{scr_name}{\hyperref[sailRISCVzscrzyname]{#2}}{}%
  \ifstrequal{#1}{scr\_name}{\hyperref[sailRISCVzscrzyname]{#2}}{}%
  \ifstrequal{#1}{scr_name_map}{\hyperref[sailRISCVzscrzynamezymap]{#2}}{}%
  \ifstrequal{#1}{scr\_name\_map}{\hyperref[sailRISCVzscrzynamezymap]{#2}}{}%
  \ifstrequal{#1}{sealCap}{\hyperref[sailRISCVzsealCap]{#2}}{}%
  \ifstrequal{#1}{seed_opst_of_num}{\hyperref[sailRISCVzseedzyopstzyofzynum]{#2}}{}%
  \ifstrequal{#1}{seed\_opst\_of\_num}{\hyperref[sailRISCVzseedzyopstzyofzynum]{#2}}{}%
  \ifstrequal{#1}{select_instr_or_fcsr_rm}{\hyperref[sailRISCVzselectzyinstrzyorzyfcsrzyrm]{#2}}{}%
  \ifstrequal{#1}{select\_instr\_or\_fcsr\_rm}{\hyperref[sailRISCVzselectzyinstrzyorzyfcsrzyrm]{#2}}{}%
  \ifstrequal{#1}{sep}{\hyperref[sailRISCVzsep]{#2}}{}%
  \ifstrequal{#1}{setCapAddr}{\hyperref[sailRISCVzsetCapAddr]{#2}}{}%
  \ifstrequal{#1}{setCapBounds}{\hyperref[sailRISCVzsetCapBounds]{#2}}{}%
  \ifstrequal{#1}{setCapFlags}{\hyperref[sailRISCVzsetCapFlags]{#2}}{}%
  \ifstrequal{#1}{setCapOffset}{\hyperref[sailRISCVzsetCapOffset]{#2}}{}%
  \ifstrequal{#1}{setCapOffsetChecked}{\hyperref[sailRISCVzsetCapOffsetChecked]{#2}}{}%
  \ifstrequal{#1}{setCapPerms}{\hyperref[sailRISCVzsetCapPerms]{#2}}{}%
  \ifstrequal{#1}{set_mstatus_SXL}{\hyperref[sailRISCVzsetzymstatuszySXL]{#2}}{}%
  \ifstrequal{#1}{set\_mstatus\_SXL}{\hyperref[sailRISCVzsetzymstatuszySXL]{#2}}{}%
  \ifstrequal{#1}{set_mstatus_UXL}{\hyperref[sailRISCVzsetzymstatuszyUXL]{#2}}{}%
  \ifstrequal{#1}{set\_mstatus\_UXL}{\hyperref[sailRISCVzsetzymstatuszyUXL]{#2}}{}%
  \ifstrequal{#1}{set_mtvec}{\hyperref[sailRISCVzsetzymtvec]{#2}}{}%
  \ifstrequal{#1}{set\_mtvec}{\hyperref[sailRISCVzsetzymtvec]{#2}}{}%
  \ifstrequal{#1}{set_next_pc}{\hyperref[sailRISCVzsetzynextzypc]{#2}}{}%
  \ifstrequal{#1}{set\_next\_pc}{\hyperref[sailRISCVzsetzynextzypc]{#2}}{}%
  \ifstrequal{#1}{set_slice_bits}{\hyperref[sailRISCVzsetzyslicezybits]{#2}}{}%
  \ifstrequal{#1}{set\_slice\_bits}{\hyperref[sailRISCVzsetzyslicezybits]{#2}}{}%
  \ifstrequal{#1}{set_slice_int}{\hyperref[sailRISCVzsetzyslicezyint]{#2}}{}%
  \ifstrequal{#1}{set\_slice\_int}{\hyperref[sailRISCVzsetzyslicezyint]{#2}}{}%
  \ifstrequal{#1}{set_sstatus_UXL}{\hyperref[sailRISCVzsetzysstatuszyUXL]{#2}}{}%
  \ifstrequal{#1}{set\_sstatus\_UXL}{\hyperref[sailRISCVzsetzysstatuszyUXL]{#2}}{}%
  \ifstrequal{#1}{set_stvec}{\hyperref[sailRISCVzsetzystvec]{#2}}{}%
  \ifstrequal{#1}{set\_stvec}{\hyperref[sailRISCVzsetzystvec]{#2}}{}%
  \ifstrequal{#1}{set_utvec}{\hyperref[sailRISCVzsetzyutvec]{#2}}{}%
  \ifstrequal{#1}{set\_utvec}{\hyperref[sailRISCVzsetzyutvec]{#2}}{}%
  \ifstrequal{#1}{set_xret_target}{\hyperref[sailRISCVzsetzyxretzytarget]{#2}}{}%
  \ifstrequal{#1}{set\_xret\_target}{\hyperref[sailRISCVzsetzyxretzytarget]{#2}}{}%
  \ifstrequal{#1}{shift_bits_left}{\hyperref[sailRISCVzshiftzybitszyleft]{#2}}{}%
  \ifstrequal{#1}{shift\_bits\_left}{\hyperref[sailRISCVzshiftzybitszyleft]{#2}}{}%
  \ifstrequal{#1}{shift_bits_right}{\hyperref[sailRISCVzshiftzybitszyright]{#2}}{}%
  \ifstrequal{#1}{shift\_bits\_right}{\hyperref[sailRISCVzshiftzybitszyright]{#2}}{}%
  \ifstrequal{#1}{shift_right_arith32}{\hyperref[sailRISCVzshiftzyrightzyarith32]{#2}}{}%
  \ifstrequal{#1}{shift\_right\_arith32}{\hyperref[sailRISCVzshiftzyrightzyarith32]{#2}}{}%
  \ifstrequal{#1}{shift_right_arith64}{\hyperref[sailRISCVzshiftzyrightzyarith64]{#2}}{}%
  \ifstrequal{#1}{shift\_right\_arith64}{\hyperref[sailRISCVzshiftzyrightzyarith64]{#2}}{}%
  \ifstrequal{#1}{shiftiop_mnemonic}{\hyperref[sailRISCVzshiftiopzymnemonic]{#2}}{}%
  \ifstrequal{#1}{shiftiop\_mnemonic}{\hyperref[sailRISCVzshiftiopzymnemonic]{#2}}{}%
  \ifstrequal{#1}{shiftiwop_mnemonic}{\hyperref[sailRISCVzshiftiwopzymnemonic]{#2}}{}%
  \ifstrequal{#1}{shiftiwop\_mnemonic}{\hyperref[sailRISCVzshiftiwopzymnemonic]{#2}}{}%
  \ifstrequal{#1}{shiftl}{\hyperref[sailRISCVzshiftl]{#2}}{}%
  \ifstrequal{#1}{shiftr}{\hyperref[sailRISCVzshiftr]{#2}}{}%
  \ifstrequal{#1}{shiftw_mnemonic}{\hyperref[sailRISCVzshiftwzymnemonic]{#2}}{}%
  \ifstrequal{#1}{shiftw\_mnemonic}{\hyperref[sailRISCVzshiftwzymnemonic]{#2}}{}%
  \ifstrequal{#1}{signed}{\hyperref[sailRISCVzsigned]{#2}}{}%
  \ifstrequal{#1}{size_bits}{\hyperref[sailRISCVzsizzezybits]{#2}}{}%
  \ifstrequal{#1}{size\_bits}{\hyperref[sailRISCVzsizzezybits]{#2}}{}%
  \ifstrequal{#1}{size_mnemonic}{\hyperref[sailRISCVzsizzezymnemonic]{#2}}{}%
  \ifstrequal{#1}{size\_mnemonic}{\hyperref[sailRISCVzsizzezymnemonic]{#2}}{}%
  \ifstrequal{#1}{slice}{\hyperref[sailRISCVzslice]{#2}}{}%
  \ifstrequal{#1}{slice_mask}{\hyperref[sailRISCVzslicezymask]{#2}}{}%
  \ifstrequal{#1}{slice\_mask}{\hyperref[sailRISCVzslicezymask]{#2}}{}%
  \ifstrequal{#1}{sop_of_num}{\hyperref[sailRISCVzsopzyofzynum]{#2}}{}%
  \ifstrequal{#1}{sop\_of\_num}{\hyperref[sailRISCVzsopzyofzynum]{#2}}{}%
  \ifstrequal{#1}{sopw_of_num}{\hyperref[sailRISCVzsopwzyofzynum]{#2}}{}%
  \ifstrequal{#1}{sopw\_of\_num}{\hyperref[sailRISCVzsopwzyofzynum]{#2}}{}%
  \ifstrequal{#1}{spc}{\hyperref[sailRISCVzspc]{#2}}{}%
  \ifstrequal{#1}{spc_backwards}{\hyperref[sailRISCVzspczybackwards]{#2}}{}%
  \ifstrequal{#1}{spc\_backwards}{\hyperref[sailRISCVzspczybackwards]{#2}}{}%
  \ifstrequal{#1}{spc_forwards}{\hyperref[sailRISCVzspczyforwards]{#2}}{}%
  \ifstrequal{#1}{spc\_forwards}{\hyperref[sailRISCVzspczyforwards]{#2}}{}%
  \ifstrequal{#1}{spc_matches_prefix}{\hyperref[sailRISCVzspczymatcheszyprefix]{#2}}{}%
  \ifstrequal{#1}{spc\_matches\_prefix}{\hyperref[sailRISCVzspczymatcheszyprefix]{#2}}{}%
  \ifstrequal{#1}{speculate_conditional}{\hyperref[sailRISCVzspeculatezyconditional]{#2}}{}%
  \ifstrequal{#1}{speculate\_conditional}{\hyperref[sailRISCVzspeculatezyconditional]{#2}}{}%
  \ifstrequal{#1}{step}{\hyperref[sailRISCVzstep]{#2}}{}%
  \ifstrequal{#1}{string_append}{\hyperref[sailRISCVzstringzyappend]{#2}}{}%
  \ifstrequal{#1}{string\_append}{\hyperref[sailRISCVzstringzyappend]{#2}}{}%
  \ifstrequal{#1}{string_drop}{\hyperref[sailRISCVzstringzydrop]{#2}}{}%
  \ifstrequal{#1}{string\_drop}{\hyperref[sailRISCVzstringzydrop]{#2}}{}%
  \ifstrequal{#1}{string_length}{\hyperref[sailRISCVzstringzylength]{#2}}{}%
  \ifstrequal{#1}{string\_length}{\hyperref[sailRISCVzstringzylength]{#2}}{}%
  \ifstrequal{#1}{string_of_bit}{\hyperref[sailRISCVzstringzyofzybit]{#2}}{}%
  \ifstrequal{#1}{string\_of\_bit}{\hyperref[sailRISCVzstringzyofzybit]{#2}}{}%
  \ifstrequal{#1}{string_of_bits}{\hyperref[sailRISCVzstringzyofzybits]{#2}}{}%
  \ifstrequal{#1}{string\_of\_bits}{\hyperref[sailRISCVzstringzyofzybits]{#2}}{}%
  \ifstrequal{#1}{string_of_capex}{\hyperref[sailRISCVzstringzyofzycapex]{#2}}{}%
  \ifstrequal{#1}{string\_of\_capex}{\hyperref[sailRISCVzstringzyofzycapex]{#2}}{}%
  \ifstrequal{#1}{string_of_int}{\hyperref[sailRISCVzstringzyofzyint]{#2}}{}%
  \ifstrequal{#1}{string\_of\_int}{\hyperref[sailRISCVzstringzyofzyint]{#2}}{}%
  \ifstrequal{#1}{string_startswith}{\hyperref[sailRISCVzstringzystartswith]{#2}}{}%
  \ifstrequal{#1}{string\_startswith}{\hyperref[sailRISCVzstringzystartswith]{#2}}{}%
  \ifstrequal{#1}{string_take}{\hyperref[sailRISCVzstringzytake]{#2}}{}%
  \ifstrequal{#1}{string\_take}{\hyperref[sailRISCVzstringzytake]{#2}}{}%
  \ifstrequal{#1}{sub_atom}{\hyperref[sailRISCVzsubzyatom]{#2}}{}%
  \ifstrequal{#1}{sub\_atom}{\hyperref[sailRISCVzsubzyatom]{#2}}{}%
  \ifstrequal{#1}{sub_bits}{\hyperref[sailRISCVzsubzybits]{#2}}{}%
  \ifstrequal{#1}{sub\_bits}{\hyperref[sailRISCVzsubzybits]{#2}}{}%
  \ifstrequal{#1}{sub_int}{\hyperref[sailRISCVzsubzyint]{#2}}{}%
  \ifstrequal{#1}{sub\_int}{\hyperref[sailRISCVzsubzyint]{#2}}{}%
  \ifstrequal{#1}{sub_nat}{\hyperref[sailRISCVzsubzynat]{#2}}{}%
  \ifstrequal{#1}{sub\_nat}{\hyperref[sailRISCVzsubzynat]{#2}}{}%
  \ifstrequal{#1}{sub_vec}{\hyperref[sailRISCVzsubzyvec]{#2}}{}%
  \ifstrequal{#1}{sub\_vec}{\hyperref[sailRISCVzsubzyvec]{#2}}{}%
  \ifstrequal{#1}{sub_vec_int}{\hyperref[sailRISCVzsubzyveczyint]{#2}}{}%
  \ifstrequal{#1}{sub\_vec\_int}{\hyperref[sailRISCVzsubzyveczyint]{#2}}{}%
  \ifstrequal{#1}{subrange_bits}{\hyperref[sailRISCVzsubrangezybits]{#2}}{}%
  \ifstrequal{#1}{subrange\_bits}{\hyperref[sailRISCVzsubrangezybits]{#2}}{}%
  \ifstrequal{#1}{sys_enable_fdext}{\hyperref[sailRISCVzsyszyenablezyfdext]{#2}}{}%
  \ifstrequal{#1}{sys\_enable\_fdext}{\hyperref[sailRISCVzsyszyenablezyfdext]{#2}}{}%
  \ifstrequal{#1}{sys_enable_next}{\hyperref[sailRISCVzsyszyenablezynext]{#2}}{}%
  \ifstrequal{#1}{sys\_enable\_next}{\hyperref[sailRISCVzsyszyenablezynext]{#2}}{}%
  \ifstrequal{#1}{sys_enable_rvc}{\hyperref[sailRISCVzsyszyenablezyrvc]{#2}}{}%
  \ifstrequal{#1}{sys\_enable\_rvc}{\hyperref[sailRISCVzsyszyenablezyrvc]{#2}}{}%
  \ifstrequal{#1}{sys_enable_writable_misa}{\hyperref[sailRISCVzsyszyenablezywritablezymisa]{#2}}{}%
  \ifstrequal{#1}{sys\_enable\_writable\_misa}{\hyperref[sailRISCVzsyszyenablezywritablezymisa]{#2}}{}%
  \ifstrequal{#1}{sys_enable_zfinx}{\hyperref[sailRISCVzsyszyenablezyzzfinx]{#2}}{}%
  \ifstrequal{#1}{sys\_enable\_zfinx}{\hyperref[sailRISCVzsyszyenablezyzzfinx]{#2}}{}%
  \ifstrequal{#1}{tag_addr_to_addr}{\hyperref[sailRISCVztagzyaddrzytozyaddr]{#2}}{}%
  \ifstrequal{#1}{tag\_addr\_to\_addr}{\hyperref[sailRISCVztagzyaddrzytozyaddr]{#2}}{}%
  \ifstrequal{#1}{tdiv_int}{\hyperref[sailRISCVztdivzyint]{#2}}{}%
  \ifstrequal{#1}{tdiv\_int}{\hyperref[sailRISCVztdivzyint]{#2}}{}%
  \ifstrequal{#1}{tick_clock}{\hyperref[sailRISCVztickzyclock]{#2}}{}%
  \ifstrequal{#1}{tick\_clock}{\hyperref[sailRISCVztickzyclock]{#2}}{}%
  \ifstrequal{#1}{tick_pc}{\hyperref[sailRISCVztickzypc]{#2}}{}%
  \ifstrequal{#1}{tick\_pc}{\hyperref[sailRISCVztickzypc]{#2}}{}%
  \ifstrequal{#1}{tick_platform}{\hyperref[sailRISCVztickzyplatform]{#2}}{}%
  \ifstrequal{#1}{tick\_platform}{\hyperref[sailRISCVztickzyplatform]{#2}}{}%
  \ifstrequal{#1}{to_bits}{\hyperref[sailRISCVztozybits]{#2}}{}%
  \ifstrequal{#1}{to\_bits}{\hyperref[sailRISCVztozybits]{#2}}{}%
  \ifstrequal{#1}{trans_kind_of_num}{\hyperref[sailRISCVztranszykindzyofzynum]{#2}}{}%
  \ifstrequal{#1}{trans\_kind\_of\_num}{\hyperref[sailRISCVztranszykindzyofzynum]{#2}}{}%
  \ifstrequal{#1}{translate39}{\hyperref[sailRISCVztranslate39]{#2}}{}%
  \ifstrequal{#1}{translate48}{\hyperref[sailRISCVztranslate48]{#2}}{}%
  \ifstrequal{#1}{translateAddr}{\hyperref[sailRISCVztranslateAddr]{#2}}{}%
  \ifstrequal{#1}{translateAddr_priv}{\hyperref[sailRISCVztranslateAddrzypriv]{#2}}{}%
  \ifstrequal{#1}{translateAddr\_priv}{\hyperref[sailRISCVztranslateAddrzypriv]{#2}}{}%
  \ifstrequal{#1}{translationException}{\hyperref[sailRISCVztranslationException]{#2}}{}%
  \ifstrequal{#1}{translationMode}{\hyperref[sailRISCVztranslationMode]{#2}}{}%
  \ifstrequal{#1}{trapVectorMode_of_bits}{\hyperref[sailRISCVztrapVectorModezyofzybits]{#2}}{}%
  \ifstrequal{#1}{trapVectorMode\_of\_bits}{\hyperref[sailRISCVztrapVectorModezyofzybits]{#2}}{}%
  \ifstrequal{#1}{trap_handler}{\hyperref[sailRISCVztrapzyhandler]{#2}}{}%
  \ifstrequal{#1}{trap\_handler}{\hyperref[sailRISCVztrapzyhandler]{#2}}{}%
  \ifstrequal{#1}{truncate}{\hyperref[sailRISCVztruncate]{#2}}{}%
  \ifstrequal{#1}{truncateLSB}{\hyperref[sailRISCVztruncateLSB]{#2}}{}%
  \ifstrequal{#1}{tval}{\hyperref[sailRISCVztval]{#2}}{}%
  \ifstrequal{#1}{tvec_addr}{\hyperref[sailRISCVztveczyaddr]{#2}}{}%
  \ifstrequal{#1}{tvec\_addr}{\hyperref[sailRISCVztveczyaddr]{#2}}{}%
  \ifstrequal{#1}{ufFlag}{\hyperref[sailRISCVzufFlag]{#2}}{}%
  \ifstrequal{#1}{unsealCap}{\hyperref[sailRISCVzunsealCap]{#2}}{}%
  \ifstrequal{#1}{unsigned}{\hyperref[sailRISCVzunsigned]{#2}}{}%
  \ifstrequal{#1}{uop_of_num}{\hyperref[sailRISCVzuopzyofzynum]{#2}}{}%
  \ifstrequal{#1}{uop\_of\_num}{\hyperref[sailRISCVzuopzyofzynum]{#2}}{}%
  \ifstrequal{#1}{update_PTE_Bits}{\hyperref[sailRISCVzupdatezyPTEzyBits]{#2}}{}%
  \ifstrequal{#1}{update\_PTE\_Bits}{\hyperref[sailRISCVzupdatezyPTEzyBits]{#2}}{}%
  \ifstrequal{#1}{update_softfloat_fflags}{\hyperref[sailRISCVzupdatezysoftfloatzyfflags]{#2}}{}%
  \ifstrequal{#1}{update\_softfloat\_fflags}{\hyperref[sailRISCVzupdatezysoftfloatzyfflags]{#2}}{}%
  \ifstrequal{#1}{update_subrange}{\hyperref[sailRISCVzupdatezysubrange]{#2}}{}%
  \ifstrequal{#1}{update\_subrange}{\hyperref[sailRISCVzupdatezysubrange]{#2}}{}%
  \ifstrequal{#1}{update_subrange_bits}{\hyperref[sailRISCVzupdatezysubrangezybits]{#2}}{}%
  \ifstrequal{#1}{update\_subrange\_bits}{\hyperref[sailRISCVzupdatezysubrangezybits]{#2}}{}%
  \ifstrequal{#1}{utype_mnemonic}{\hyperref[sailRISCVzutypezymnemonic]{#2}}{}%
  \ifstrequal{#1}{utype\_mnemonic}{\hyperref[sailRISCVzutypezymnemonic]{#2}}{}%
  \ifstrequal{#1}{validDoubleRegs}{\hyperref[sailRISCVzvalidDoubleRegs]{#2}}{}%
  \ifstrequal{#1}{valid_rounding_mode}{\hyperref[sailRISCVzvalidzyroundingzymode]{#2}}{}%
  \ifstrequal{#1}{valid\_rounding\_mode}{\hyperref[sailRISCVzvalidzyroundingzymode]{#2}}{}%
  \ifstrequal{#1}{vector_concat}{\hyperref[sailRISCVzvectorzyconcat]{#2}}{}%
  \ifstrequal{#1}{vector\_concat}{\hyperref[sailRISCVzvectorzyconcat]{#2}}{}%
  \ifstrequal{#1}{vector_length}{\hyperref[sailRISCVzvectorzylength]{#2}}{}%
  \ifstrequal{#1}{vector\_length}{\hyperref[sailRISCVzvectorzylength]{#2}}{}%
  \ifstrequal{#1}{wC}{\hyperref[sailRISCVzwC]{#2}}{}%
  \ifstrequal{#1}{wC_bits}{\hyperref[sailRISCVzwCzybits]{#2}}{}%
  \ifstrequal{#1}{wC\_bits}{\hyperref[sailRISCVzwCzybits]{#2}}{}%
  \ifstrequal{#1}{wF}{\hyperref[sailRISCVzwF]{#2}}{}%
  \ifstrequal{#1}{wF_bits}{\hyperref[sailRISCVzwFzybits]{#2}}{}%
  \ifstrequal{#1}{wF\_bits}{\hyperref[sailRISCVzwFzybits]{#2}}{}%
  \ifstrequal{#1}{wF_or_X_D}{\hyperref[sailRISCVzwFzyorzyXzyD]{#2}}{}%
  \ifstrequal{#1}{wF\_or\_X\_D}{\hyperref[sailRISCVzwFzyorzyXzyD]{#2}}{}%
  \ifstrequal{#1}{wF_or_X_H}{\hyperref[sailRISCVzwFzyorzyXzyH]{#2}}{}%
  \ifstrequal{#1}{wF\_or\_X\_H}{\hyperref[sailRISCVzwFzyorzyXzyH]{#2}}{}%
  \ifstrequal{#1}{wF_or_X_S}{\hyperref[sailRISCVzwFzyorzyXzyS]{#2}}{}%
  \ifstrequal{#1}{wF\_or\_X\_S}{\hyperref[sailRISCVzwFzyorzyXzyS]{#2}}{}%
  \ifstrequal{#1}{wX}{\hyperref[sailRISCVzwX]{#2}}{}%
  \ifstrequal{#1}{wX_bits}{\hyperref[sailRISCVzwXzybits]{#2}}{}%
  \ifstrequal{#1}{wX\_bits}{\hyperref[sailRISCVzwXzybits]{#2}}{}%
  \ifstrequal{#1}{walk39}{\hyperref[sailRISCVzwalk39]{#2}}{}%
  \ifstrequal{#1}{walk48}{\hyperref[sailRISCVzwalk48]{#2}}{}%
  \ifstrequal{#1}{within_clint}{\hyperref[sailRISCVzwithinzyclint]{#2}}{}%
  \ifstrequal{#1}{within\_clint}{\hyperref[sailRISCVzwithinzyclint]{#2}}{}%
  \ifstrequal{#1}{within_htif_readable}{\hyperref[sailRISCVzwithinzyhtifzyreadable]{#2}}{}%
  \ifstrequal{#1}{within\_htif\_readable}{\hyperref[sailRISCVzwithinzyhtifzyreadable]{#2}}{}%
  \ifstrequal{#1}{within_htif_writable}{\hyperref[sailRISCVzwithinzyhtifzywritable]{#2}}{}%
  \ifstrequal{#1}{within\_htif\_writable}{\hyperref[sailRISCVzwithinzyhtifzywritable]{#2}}{}%
  \ifstrequal{#1}{within_mmio_readable}{\hyperref[sailRISCVzwithinzymmiozyreadable]{#2}}{}%
  \ifstrequal{#1}{within\_mmio\_readable}{\hyperref[sailRISCVzwithinzymmiozyreadable]{#2}}{}%
  \ifstrequal{#1}{within_mmio_writable}{\hyperref[sailRISCVzwithinzymmiozywritable]{#2}}{}%
  \ifstrequal{#1}{within\_mmio\_writable}{\hyperref[sailRISCVzwithinzymmiozywritable]{#2}}{}%
  \ifstrequal{#1}{within_phys_mem}{\hyperref[sailRISCVzwithinzyphyszymem]{#2}}{}%
  \ifstrequal{#1}{within\_phys\_mem}{\hyperref[sailRISCVzwithinzyphyszymem]{#2}}{}%
  \ifstrequal{#1}{word_width_bytes}{\hyperref[sailRISCVzwordzywidthzybytes]{#2}}{}%
  \ifstrequal{#1}{word\_width\_bytes}{\hyperref[sailRISCVzwordzywidthzybytes]{#2}}{}%
  \ifstrequal{#1}{word_width_of_num}{\hyperref[sailRISCVzwordzywidthzyofzynum]{#2}}{}%
  \ifstrequal{#1}{word\_width\_of\_num}{\hyperref[sailRISCVzwordzywidthzyofzynum]{#2}}{}%
  \ifstrequal{#1}{writeCSR}{\hyperref[sailRISCVzwriteCSR]{#2}}{}%
  \ifstrequal{#1}{write_TLB39}{\hyperref[sailRISCVzwritezyTLB39]{#2}}{}%
  \ifstrequal{#1}{write\_TLB39}{\hyperref[sailRISCVzwritezyTLB39]{#2}}{}%
  \ifstrequal{#1}{write_TLB48}{\hyperref[sailRISCVzwritezyTLB48]{#2}}{}%
  \ifstrequal{#1}{write\_TLB48}{\hyperref[sailRISCVzwritezyTLB48]{#2}}{}%
  \ifstrequal{#1}{write_fflags}{\hyperref[sailRISCVzwritezyfflags]{#2}}{}%
  \ifstrequal{#1}{write\_fflags}{\hyperref[sailRISCVzwritezyfflags]{#2}}{}%
  \ifstrequal{#1}{write_kind_of_num}{\hyperref[sailRISCVzwritezykindzyofzynum]{#2}}{}%
  \ifstrequal{#1}{write\_kind\_of\_num}{\hyperref[sailRISCVzwritezykindzyofzynum]{#2}}{}%
  \ifstrequal{#1}{write_ram}{\hyperref[sailRISCVzwritezyram]{#2}}{}%
  \ifstrequal{#1}{write\_ram}{\hyperref[sailRISCVzwritezyram]{#2}}{}%
  \ifstrequal{#1}{write_ram_ea}{\hyperref[sailRISCVzwritezyramzyea]{#2}}{}%
  \ifstrequal{#1}{write\_ram\_ea}{\hyperref[sailRISCVzwritezyramzyea]{#2}}{}%
  \ifstrequal{#1}{write_sc_cap_result}{\hyperref[sailRISCVzwritezysczycapzyresult]{#2}}{}%
  \ifstrequal{#1}{write\_sc\_cap\_result}{\hyperref[sailRISCVzwritezysczycapzyresult]{#2}}{}%
  \ifstrequal{#1}{write_seed_csr}{\hyperref[sailRISCVzwritezyseedzycsr]{#2}}{}%
  \ifstrequal{#1}{write\_seed\_csr}{\hyperref[sailRISCVzwritezyseedzycsr]{#2}}{}%
  \ifstrequal{#1}{xor_vec}{\hyperref[sailRISCVzxorzyvec]{#2}}{}%
  \ifstrequal{#1}{xor\_vec}{\hyperref[sailRISCVzxorzyvec]{#2}}{}%
  \ifstrequal{#1}{zeros_implicit}{\hyperref[sailRISCVzzzeroszyimplicit]{#2}}{}%
  \ifstrequal{#1}{zeros\_implicit}{\hyperref[sailRISCVzzzeroszyimplicit]{#2}}{}%
  \ifstrequal{#1}{(operator <=_u)}{\hyperref[sailRISCVzz8operatorz0zIzJzyuz9]{#2}}{}%
  \ifstrequal{#1}{(operator $>$=\_u)}{\hyperref[sailRISCVzz8operatorz0zIzJzyuz9]{#2}}{}%
  \ifstrequal{#1}{(operator <_s)}{\hyperref[sailRISCVzz8operatorz0zIzysz9]{#2}}{}%
  \ifstrequal{#1}{(operator $>$\_s)}{\hyperref[sailRISCVzz8operatorz0zIzysz9]{#2}}{}%
  \ifstrequal{#1}{(operator <_u)}{\hyperref[sailRISCVzz8operatorz0zIzyuz9]{#2}}{}%
  \ifstrequal{#1}{(operator $>$\_u)}{\hyperref[sailRISCVzz8operatorz0zIzyuz9]{#2}}{}%
  \ifstrequal{#1}{(operator >=_s)}{\hyperref[sailRISCVzz8operatorz0zKzJzysz9]{#2}}{}%
  \ifstrequal{#1}{(operator $$>$$=\_s)}{\hyperref[sailRISCVzz8operatorz0zKzJzysz9]{#2}}{}%
  \ifstrequal{#1}{(operator >=_u)}{\hyperref[sailRISCVzz8operatorz0zKzJzyuz9]{#2}}{}%
  \ifstrequal{#1}{(operator $$>$$=\_u)}{\hyperref[sailRISCVzz8operatorz0zKzJzyuz9]{#2}}{}}

\newcommand{\sailRISCVfn}[1]{
  \ifstrequal{#1}{Architecture_of_num}{\sailRISCVfnArchitectureOfNum}{}%
  \ifstrequal{#1}{Architecture\_of\_num}{\sailRISCVfnArchitectureOfNum}{}%
  \ifstrequal{#1}{CPtrCmpOp_of_num}{\sailRISCVfnCPtrCmpOpOfNum}{}%
  \ifstrequal{#1}{CPtrCmpOp\_of\_num}{\sailRISCVfnCPtrCmpOpOfNum}{}%
  \ifstrequal{#1}{CapExCode}{\sailRISCVfnCapExCode}{}%
  \ifstrequal{#1}{CapEx_of_num}{\sailRISCVfnCapExOfNum}{}%
  \ifstrequal{#1}{CapEx\_of\_num}{\sailRISCVfnCapExOfNum}{}%
  \ifstrequal{#1}{ClearRegSet_of_num}{\sailRISCVfnClearRegSetOfNum}{}%
  \ifstrequal{#1}{ClearRegSet\_of\_num}{\sailRISCVfnClearRegSetOfNum}{}%
  \ifstrequal{#1}{EXTS}{\sailRISCVfnEXTS}{}%
  \ifstrequal{#1}{EXTZ}{\sailRISCVfnEXTZ}{}%
  \ifstrequal{#1}{ExtStatus_of_num}{\sailRISCVfnExtStatusOfNum}{}%
  \ifstrequal{#1}{ExtStatus\_of\_num}{\sailRISCVfnExtStatusOfNum}{}%
  \ifstrequal{#1}{FRegStr}{\sailRISCVfnFRegStr}{}%
  \ifstrequal{#1}{GPRstr}{\sailRISCVfnGPRstr}{}%
  \ifstrequal{#1}{InterruptType_of_num}{\sailRISCVfnInterruptTypeOfNum}{}%
  \ifstrequal{#1}{InterruptType\_of\_num}{\sailRISCVfnInterruptTypeOfNum}{}%
  \ifstrequal{#1}{MAX}{\sailRISCVfnMAX}{}%
  \ifstrequal{#1}{MemoryOpResult_add_meta}{\sailRISCVfnMemoryOpResultAddMeta}{}%
  \ifstrequal{#1}{MemoryOpResult\_add\_meta}{\sailRISCVfnMemoryOpResultAddMeta}{}%
  \ifstrequal{#1}{MemoryOpResult_drop_meta}{\sailRISCVfnMemoryOpResultDropMeta}{}%
  \ifstrequal{#1}{MemoryOpResult\_drop\_meta}{\sailRISCVfnMemoryOpResultDropMeta}{}%
  \ifstrequal{#1}{PmpAddrMatchType_of_num}{\sailRISCVfnPmpAddrMatchTypeOfNum}{}%
  \ifstrequal{#1}{PmpAddrMatchType\_of\_num}{\sailRISCVfnPmpAddrMatchTypeOfNum}{}%
  \ifstrequal{#1}{Privilege_of_num}{\sailRISCVfnPrivilegeOfNum}{}%
  \ifstrequal{#1}{Privilege\_of\_num}{\sailRISCVfnPrivilegeOfNum}{}%
  \ifstrequal{#1}{RegStr}{\sailRISCVfnRegStr}{}%
  \ifstrequal{#1}{Retired_of_num}{\sailRISCVfnRetiredOfNum}{}%
  \ifstrequal{#1}{Retired\_of\_num}{\sailRISCVfnRetiredOfNum}{}%
  \ifstrequal{#1}{SATPMode_of_num}{\sailRISCVfnSATPModeOfNum}{}%
  \ifstrequal{#1}{SATPMode\_of\_num}{\sailRISCVfnSATPModeOfNum}{}%
  \ifstrequal{#1}{TrapVectorMode_of_num}{\sailRISCVfnTrapVectorModeOfNum}{}%
  \ifstrequal{#1}{TrapVectorMode\_of\_num}{\sailRISCVfnTrapVectorModeOfNum}{}%
  \ifstrequal{#1}{__ReadRAM_Meta}{\sailRISCVfnReadRAMMeta}{}%
  \ifstrequal{#1}{\_\_ReadRAM\_Meta}{\sailRISCVfnReadRAMMeta}{}%
  \ifstrequal{#1}{__WriteRAM_Meta}{\sailRISCVfnWriteRAMMeta}{}%
  \ifstrequal{#1}{\_\_WriteRAM\_Meta}{\sailRISCVfnWriteRAMMeta}{}%
  \ifstrequal{#1}{__id}{\sailRISCVfnId}{}%
  \ifstrequal{#1}{\_\_id}{\sailRISCVfnId}{}%
  \ifstrequal{#1}{_shl_int_general}{\sailRISCVfnShlIntGeneral}{}%
  \ifstrequal{#1}{\_shl\_int\_general}{\sailRISCVfnShlIntGeneral}{}%
  \ifstrequal{#1}{_shr_int_general}{\sailRISCVfnShrIntGeneral}{}%
  \ifstrequal{#1}{\_shr\_int\_general}{\sailRISCVfnShrIntGeneral}{}%
  \ifstrequal{#1}{a64_barrier_domain_of_num}{\sailRISCVfnaSixFourBarrierDomainOfNum}{}%
  \ifstrequal{#1}{a64\_barrier\_domain\_of\_num}{\sailRISCVfnaSixFourBarrierDomainOfNum}{}%
  \ifstrequal{#1}{a64_barrier_type_of_num}{\sailRISCVfnaSixFourBarrierTypeOfNum}{}%
  \ifstrequal{#1}{a64\_barrier\_type\_of\_num}{\sailRISCVfnaSixFourBarrierTypeOfNum}{}%
  \ifstrequal{#1}{accessType_to_str}{\sailRISCVfnaccessTypeToStr}{}%
  \ifstrequal{#1}{accessType\_to\_str}{\sailRISCVfnaccessTypeToStr}{}%
  \ifstrequal{#1}{accrue_fflags}{\sailRISCVfnaccrueFflags}{}%
  \ifstrequal{#1}{accrue\_fflags}{\sailRISCVfnaccrueFflags}{}%
  \ifstrequal{#1}{add_to_TLB39}{\sailRISCVfnaddToTLBThreeNine}{}%
  \ifstrequal{#1}{add\_to\_TLB39}{\sailRISCVfnaddToTLBThreeNine}{}%
  \ifstrequal{#1}{add_to_TLB48}{\sailRISCVfnaddToTLBFourEight}{}%
  \ifstrequal{#1}{add\_to\_TLB48}{\sailRISCVfnaddToTLBFourEight}{}%
  \ifstrequal{#1}{addr_to_tag_addr}{\sailRISCVfnaddrToTagAddr}{}%
  \ifstrequal{#1}{addr\_to\_tag\_addr}{\sailRISCVfnaddrToTagAddr}{}%
  \ifstrequal{#1}{amo_width_valid}{\sailRISCVfnamoWidthValid}{}%
  \ifstrequal{#1}{amo\_width\_valid}{\sailRISCVfnamoWidthValid}{}%
  \ifstrequal{#1}{amoop_of_num}{\sailRISCVfnamoopOfNum}{}%
  \ifstrequal{#1}{amoop\_of\_num}{\sailRISCVfnamoopOfNum}{}%
  \ifstrequal{#1}{aqrl_str}{\sailRISCVfnaqrlStr}{}%
  \ifstrequal{#1}{aqrl\_str}{\sailRISCVfnaqrlStr}{}%
  \ifstrequal{#1}{arch_to_bits}{\sailRISCVfnarchToBits}{}%
  \ifstrequal{#1}{arch\_to\_bits}{\sailRISCVfnarchToBits}{}%
  \ifstrequal{#1}{architecture}{\sailRISCVfnarchitecture}{}%
  \ifstrequal{#1}{biop_zbs_of_num}{\sailRISCVfnbiopZbsOfNum}{}%
  \ifstrequal{#1}{biop\_zbs\_of\_num}{\sailRISCVfnbiopZbsOfNum}{}%
  \ifstrequal{#1}{bit_to_bool}{\sailRISCVfnbitToBool}{}%
  \ifstrequal{#1}{bit\_to\_bool}{\sailRISCVfnbitToBool}{}%
  \ifstrequal{#1}{bool_to_bit}{\sailRISCVfnboolToBit}{}%
  \ifstrequal{#1}{bool\_to\_bit}{\sailRISCVfnboolToBit}{}%
  \ifstrequal{#1}{bool_to_bits}{\sailRISCVfnboolToBits}{}%
  \ifstrequal{#1}{bool\_to\_bits}{\sailRISCVfnboolToBits}{}%
  \ifstrequal{#1}{bop_of_num}{\sailRISCVfnbopOfNum}{}%
  \ifstrequal{#1}{bop\_of\_num}{\sailRISCVfnbopOfNum}{}%
  \ifstrequal{#1}{brop_zba_of_num}{\sailRISCVfnbropZbaOfNum}{}%
  \ifstrequal{#1}{brop\_zba\_of\_num}{\sailRISCVfnbropZbaOfNum}{}%
  \ifstrequal{#1}{brop_zbb_of_num}{\sailRISCVfnbropZbbOfNum}{}%
  \ifstrequal{#1}{brop\_zbb\_of\_num}{\sailRISCVfnbropZbbOfNum}{}%
  \ifstrequal{#1}{brop_zbkb_of_num}{\sailRISCVfnbropZbkbOfNum}{}%
  \ifstrequal{#1}{brop\_zbkb\_of\_num}{\sailRISCVfnbropZbkbOfNum}{}%
  \ifstrequal{#1}{brop_zbs_of_num}{\sailRISCVfnbropZbsOfNum}{}%
  \ifstrequal{#1}{brop\_zbs\_of\_num}{\sailRISCVfnbropZbsOfNum}{}%
  \ifstrequal{#1}{bropw_zba_of_num}{\sailRISCVfnbropwZbaOfNum}{}%
  \ifstrequal{#1}{bropw\_zba\_of\_num}{\sailRISCVfnbropwZbaOfNum}{}%
  \ifstrequal{#1}{bropw_zbb_of_num}{\sailRISCVfnbropwZbbOfNum}{}%
  \ifstrequal{#1}{bropw\_zbb\_of\_num}{\sailRISCVfnbropwZbbOfNum}{}%
  \ifstrequal{#1}{cache_op_kind_of_num}{\sailRISCVfncacheOpKindOfNum}{}%
  \ifstrequal{#1}{cache\_op\_kind\_of\_num}{\sailRISCVfncacheOpKindOfNum}{}%
  \ifstrequal{#1}{canonical_NaN_D}{\sailRISCVfncanonicalNaND}{}%
  \ifstrequal{#1}{canonical\_NaN\_D}{\sailRISCVfncanonicalNaND}{}%
  \ifstrequal{#1}{canonical_NaN_H}{\sailRISCVfncanonicalNaNH}{}%
  \ifstrequal{#1}{canonical\_NaN\_H}{\sailRISCVfncanonicalNaNH}{}%
  \ifstrequal{#1}{canonical_NaN_S}{\sailRISCVfncanonicalNaNS}{}%
  \ifstrequal{#1}{canonical\_NaN\_S}{\sailRISCVfncanonicalNaNS}{}%
  \ifstrequal{#1}{capBitsToCapability}{\sailRISCVfncapBitsToCapability}{}%
  \ifstrequal{#1}{capBitsToEncCapability}{\sailRISCVfncapBitsToEncCapability}{}%
  \ifstrequal{#1}{capBoundsEqual}{\sailRISCVfncapBoundsEqual}{}%
  \ifstrequal{#1}{capToBits}{\sailRISCVfncapToBits}{}%
  \ifstrequal{#1}{capToEncCap}{\sailRISCVfncapToEncCap}{}%
  \ifstrequal{#1}{capToMemBits}{\sailRISCVfncapToMemBits}{}%
  \ifstrequal{#1}{capToString}{\sailRISCVfncapToString}{}%
  \ifstrequal{#1}{cap_reg_name_abi}{\sailRISCVfncapRegNameAbi}{}%
  \ifstrequal{#1}{cap\_reg\_name\_abi}{\sailRISCVfncapRegNameAbi}{}%
  \ifstrequal{#1}{checkPTEPermission}{\sailRISCVfncheckPTEPermission}{}%
  \ifstrequal{#1}{check_CSR}{\sailRISCVfncheckCSR}{}%
  \ifstrequal{#1}{check\_CSR}{\sailRISCVfncheckCSR}{}%
  \ifstrequal{#1}{check_CSR_access}{\sailRISCVfncheckCSRAccess}{}%
  \ifstrequal{#1}{check\_CSR\_access}{\sailRISCVfncheckCSRAccess}{}%
  \ifstrequal{#1}{check_Counteren}{\sailRISCVfncheckCounteren}{}%
  \ifstrequal{#1}{check\_Counteren}{\sailRISCVfncheckCounteren}{}%
  \ifstrequal{#1}{check_TVM_SATP}{\sailRISCVfncheckTVMSATP}{}%
  \ifstrequal{#1}{check\_TVM\_SATP}{\sailRISCVfncheckTVMSATP}{}%
  \ifstrequal{#1}{check_misaligned}{\sailRISCVfncheckMisaligned}{}%
  \ifstrequal{#1}{check\_misaligned}{\sailRISCVfncheckMisaligned}{}%
  \ifstrequal{#1}{check_res_misaligned}{\sailRISCVfncheckResMisaligned}{}%
  \ifstrequal{#1}{check\_res\_misaligned}{\sailRISCVfncheckResMisaligned}{}%
  \ifstrequal{#1}{check_seed_CSR}{\sailRISCVfncheckSeedCSR}{}%
  \ifstrequal{#1}{check\_seed\_CSR}{\sailRISCVfncheckSeedCSR}{}%
  \ifstrequal{#1}{checked_mem_read}{\sailRISCVfncheckedMemRead}{}%
  \ifstrequal{#1}{checked\_mem\_read}{\sailRISCVfncheckedMemRead}{}%
  \ifstrequal{#1}{checked_mem_write}{\sailRISCVfncheckedMemWrite}{}%
  \ifstrequal{#1}{checked\_mem\_write}{\sailRISCVfncheckedMemWrite}{}%
  \ifstrequal{#1}{clearTag}{\sailRISCVfnclearTag}{}%
  \ifstrequal{#1}{clearTagIf}{\sailRISCVfnclearTagIf}{}%
  \ifstrequal{#1}{clearTagIfSealed}{\sailRISCVfnclearTagIfSealed}{}%
  \ifstrequal{#1}{clint_dispatch}{\sailRISCVfnclintDispatch}{}%
  \ifstrequal{#1}{clint\_dispatch}{\sailRISCVfnclintDispatch}{}%
  \ifstrequal{#1}{clint_load}{\sailRISCVfnclintLoad}{}%
  \ifstrequal{#1}{clint\_load}{\sailRISCVfnclintLoad}{}%
  \ifstrequal{#1}{clint_store}{\sailRISCVfnclintStore}{}%
  \ifstrequal{#1}{clint\_store}{\sailRISCVfnclintStore}{}%
  \ifstrequal{#1}{concat_str_bits}{\sailRISCVfnconcatStrBits}{}%
  \ifstrequal{#1}{concat\_str\_bits}{\sailRISCVfnconcatStrBits}{}%
  \ifstrequal{#1}{concat_str_dec}{\sailRISCVfnconcatStrDec}{}%
  \ifstrequal{#1}{concat\_str\_dec}{\sailRISCVfnconcatStrDec}{}%
  \ifstrequal{#1}{creg2reg_idx}{\sailRISCVfncregTworegIdx}{}%
  \ifstrequal{#1}{creg2reg\_idx}{\sailRISCVfncregTworegIdx}{}%
  \ifstrequal{#1}{csrAccess}{\sailRISCVfncsrAccess}{}%
  \ifstrequal{#1}{csrPriv}{\sailRISCVfncsrPriv}{}%
  \ifstrequal{#1}{csr_name}{\sailRISCVfncsrName}{}%
  \ifstrequal{#1}{csr\_name}{\sailRISCVfncsrName}{}%
  \ifstrequal{#1}{csrop_of_num}{\sailRISCVfncsropOfNum}{}%
  \ifstrequal{#1}{csrop\_of\_num}{\sailRISCVfncsropOfNum}{}%
  \ifstrequal{#1}{curAsid32}{\sailRISCVfncurAsidThreeTwo}{}%
  \ifstrequal{#1}{curAsid64}{\sailRISCVfncurAsidSixFour}{}%
  \ifstrequal{#1}{curPTB32}{\sailRISCVfncurPTBThreeTwo}{}%
  \ifstrequal{#1}{curPTB64}{\sailRISCVfncurPTBSixFour}{}%
  \ifstrequal{#1}{cur_Architecture}{\sailRISCVfncurArchitecture}{}%
  \ifstrequal{#1}{cur\_Architecture}{\sailRISCVfncurArchitecture}{}%
  \ifstrequal{#1}{decode}{\sailRISCVfndecode}{}%
  \ifstrequal{#1}{decodeCompressed}{\sailRISCVfndecodeCompressed}{}%
  \ifstrequal{#1}{def_spc_backwards}{\sailRISCVfndefSpcBackwards}{}%
  \ifstrequal{#1}{def\_spc\_backwards}{\sailRISCVfndefSpcBackwards}{}%
  \ifstrequal{#1}{def_spc_forwards}{\sailRISCVfndefSpcForwards}{}%
  \ifstrequal{#1}{def\_spc\_forwards}{\sailRISCVfndefSpcForwards}{}%
  \ifstrequal{#1}{def_spc_matches_prefix}{\sailRISCVfndefSpcMatchesPrefix}{}%
  \ifstrequal{#1}{def\_spc\_matches\_prefix}{\sailRISCVfndefSpcMatchesPrefix}{}%
  \ifstrequal{#1}{dirty_fd_context}{\sailRISCVfndirtyFdContext}{}%
  \ifstrequal{#1}{dirty\_fd\_context}{\sailRISCVfndirtyFdContext}{}%
  \ifstrequal{#1}{dirty_fd_context_if_present}{\sailRISCVfndirtyFdContextIfPresent}{}%
  \ifstrequal{#1}{dirty\_fd\_context\_if\_present}{\sailRISCVfndirtyFdContextIfPresent}{}%
  \ifstrequal{#1}{dispatchInterrupt}{\sailRISCVfndispatchInterrupt}{}%
  \ifstrequal{#1}{dzFlag}{\sailRISCVfndzzFlag}{}%
  \ifstrequal{#1}{effectivePrivilege}{\sailRISCVfneffectivePrivilege}{}%
  \ifstrequal{#1}{encCapToBits}{\sailRISCVfnencCapToBits}{}%
  \ifstrequal{#1}{encCapabilityToCapability}{\sailRISCVfnencCapabilityToCapability}{}%
  \ifstrequal{#1}{eq_unit}{\sailRISCVfneqUnit}{}%
  \ifstrequal{#1}{eq\_unit}{\sailRISCVfneqUnit}{}%
  \ifstrequal{#1}{exceptionType_to_bits}{\sailRISCVfnexceptionTypeToBits}{}%
  \ifstrequal{#1}{exceptionType\_to\_bits}{\sailRISCVfnexceptionTypeToBits}{}%
  \ifstrequal{#1}{exceptionType_to_str}{\sailRISCVfnexceptionTypeToStr}{}%
  \ifstrequal{#1}{exceptionType\_to\_str}{\sailRISCVfnexceptionTypeToStr}{}%
  \ifstrequal{#1}{exception_delegatee}{\sailRISCVfnexceptionDelegatee}{}%
  \ifstrequal{#1}{exception\_delegatee}{\sailRISCVfnexceptionDelegatee}{}%
  \ifstrequal{#1}{exception_handler}{\sailRISCVfnexceptionHandler}{}%
  \ifstrequal{#1}{exception\_handler}{\sailRISCVfnexceptionHandler}{}%
  \ifstrequal{#1}{extStatus_of_bits}{\sailRISCVfnextStatusOfBits}{}%
  \ifstrequal{#1}{extStatus\_of\_bits}{\sailRISCVfnextStatusOfBits}{}%
  \ifstrequal{#1}{extStatus_to_bits}{\sailRISCVfnextStatusToBits}{}%
  \ifstrequal{#1}{extStatus\_to\_bits}{\sailRISCVfnextStatusToBits}{}%
  \ifstrequal{#1}{ext_access_type_of_num}{\sailRISCVfnextAccessTypeOfNum}{}%
  \ifstrequal{#1}{ext\_access\_type\_of\_num}{\sailRISCVfnextAccessTypeOfNum}{}%
  \ifstrequal{#1}{ext_check_CSR}{\sailRISCVfnextCheckCSR}{}%
  \ifstrequal{#1}{ext\_check\_CSR}{\sailRISCVfnextCheckCSR}{}%
  \ifstrequal{#1}{ext_check_CSR_fail}{\sailRISCVfnextCheckCSRFail}{}%
  \ifstrequal{#1}{ext\_check\_CSR\_fail}{\sailRISCVfnextCheckCSRFail}{}%
  \ifstrequal{#1}{ext_check_phys_mem_read}{\sailRISCVfnextCheckPhysMemRead}{}%
  \ifstrequal{#1}{ext\_check\_phys\_mem\_read}{\sailRISCVfnextCheckPhysMemRead}{}%
  \ifstrequal{#1}{ext_check_phys_mem_write}{\sailRISCVfnextCheckPhysMemWrite}{}%
  \ifstrequal{#1}{ext\_check\_phys\_mem\_write}{\sailRISCVfnextCheckPhysMemWrite}{}%
  \ifstrequal{#1}{ext_check_xret_priv}{\sailRISCVfnextCheckXretPriv}{}%
  \ifstrequal{#1}{ext\_check\_xret\_priv}{\sailRISCVfnextCheckXretPriv}{}%
  \ifstrequal{#1}{ext_control_check_addr}{\sailRISCVfnextControlCheckAddr}{}%
  \ifstrequal{#1}{ext\_control\_check\_addr}{\sailRISCVfnextControlCheckAddr}{}%
  \ifstrequal{#1}{ext_control_check_pc}{\sailRISCVfnextControlCheckPc}{}%
  \ifstrequal{#1}{ext\_control\_check\_pc}{\sailRISCVfnextControlCheckPc}{}%
  \ifstrequal{#1}{ext_data_get_addr}{\sailRISCVfnextDataGetAddr}{}%
  \ifstrequal{#1}{ext\_data\_get\_addr}{\sailRISCVfnextDataGetAddr}{}%
  \ifstrequal{#1}{ext_exc_type_of_num}{\sailRISCVfnextExcTypeOfNum}{}%
  \ifstrequal{#1}{ext\_exc\_type\_of\_num}{\sailRISCVfnextExcTypeOfNum}{}%
  \ifstrequal{#1}{ext_exc_type_to_bits}{\sailRISCVfnextExcTypeToBits}{}%
  \ifstrequal{#1}{ext\_exc\_type\_to\_bits}{\sailRISCVfnextExcTypeToBits}{}%
  \ifstrequal{#1}{ext_exc_type_to_str}{\sailRISCVfnextExcTypeToStr}{}%
  \ifstrequal{#1}{ext\_exc\_type\_to\_str}{\sailRISCVfnextExcTypeToStr}{}%
  \ifstrequal{#1}{ext_fail_xret_priv}{\sailRISCVfnextFailXretPriv}{}%
  \ifstrequal{#1}{ext\_fail\_xret\_priv}{\sailRISCVfnextFailXretPriv}{}%
  \ifstrequal{#1}{ext_fetch_check_pc}{\sailRISCVfnextFetchCheckPc}{}%
  \ifstrequal{#1}{ext\_fetch\_check\_pc}{\sailRISCVfnextFetchCheckPc}{}%
  \ifstrequal{#1}{ext_fetch_hook}{\sailRISCVfnextFetchHook}{}%
  \ifstrequal{#1}{ext\_fetch\_hook}{\sailRISCVfnextFetchHook}{}%
  \ifstrequal{#1}{ext_get_ptw_error}{\sailRISCVfnextGetPtwError}{}%
  \ifstrequal{#1}{ext\_get\_ptw\_error}{\sailRISCVfnextGetPtwError}{}%
  \ifstrequal{#1}{ext_handle_control_check_error}{\sailRISCVfnextHandleControlCheckError}{}%
  \ifstrequal{#1}{ext\_handle\_control\_check\_error}{\sailRISCVfnextHandleControlCheckError}{}%
  \ifstrequal{#1}{ext_handle_data_check_error}{\sailRISCVfnextHandleDataCheckError}{}%
  \ifstrequal{#1}{ext\_handle\_data\_check\_error}{\sailRISCVfnextHandleDataCheckError}{}%
  \ifstrequal{#1}{ext_handle_fetch_check_error}{\sailRISCVfnextHandleFetchCheckError}{}%
  \ifstrequal{#1}{ext\_handle\_fetch\_check\_error}{\sailRISCVfnextHandleFetchCheckError}{}%
  \ifstrequal{#1}{ext_init}{\sailRISCVfnextInit}{}%
  \ifstrequal{#1}{ext\_init}{\sailRISCVfnextInit}{}%
  \ifstrequal{#1}{ext_init_regs}{\sailRISCVfnextInitRegs}{}%
  \ifstrequal{#1}{ext\_init\_regs}{\sailRISCVfnextInitRegs}{}%
  \ifstrequal{#1}{ext_post_decode_hook}{\sailRISCVfnextPostDecodeHook}{}%
  \ifstrequal{#1}{ext\_post\_decode\_hook}{\sailRISCVfnextPostDecodeHook}{}%
  \ifstrequal{#1}{ext_post_step_hook}{\sailRISCVfnextPostStepHook}{}%
  \ifstrequal{#1}{ext\_post\_step\_hook}{\sailRISCVfnextPostStepHook}{}%
  \ifstrequal{#1}{ext_pre_step_hook}{\sailRISCVfnextPreStepHook}{}%
  \ifstrequal{#1}{ext\_pre\_step\_hook}{\sailRISCVfnextPreStepHook}{}%
  \ifstrequal{#1}{ext_ptw_error_of_num}{\sailRISCVfnextPtwErrorOfNum}{}%
  \ifstrequal{#1}{ext\_ptw\_error\_of\_num}{\sailRISCVfnextPtwErrorOfNum}{}%
  \ifstrequal{#1}{ext_ptw_fail_of_num}{\sailRISCVfnextPtwFailOfNum}{}%
  \ifstrequal{#1}{ext\_ptw\_fail\_of\_num}{\sailRISCVfnextPtwFailOfNum}{}%
  \ifstrequal{#1}{ext_ptw_lc_join}{\sailRISCVfnextPtwLcJoin}{}%
  \ifstrequal{#1}{ext\_ptw\_lc\_join}{\sailRISCVfnextPtwLcJoin}{}%
  \ifstrequal{#1}{ext_ptw_lc_of_num}{\sailRISCVfnextPtwLcOfNum}{}%
  \ifstrequal{#1}{ext\_ptw\_lc\_of\_num}{\sailRISCVfnextPtwLcOfNum}{}%
  \ifstrequal{#1}{ext_ptw_sc_join}{\sailRISCVfnextPtwScJoin}{}%
  \ifstrequal{#1}{ext\_ptw\_sc\_join}{\sailRISCVfnextPtwScJoin}{}%
  \ifstrequal{#1}{ext_ptw_sc_of_num}{\sailRISCVfnextPtwScOfNum}{}%
  \ifstrequal{#1}{ext\_ptw\_sc\_of\_num}{\sailRISCVfnextPtwScOfNum}{}%
  \ifstrequal{#1}{ext_rvfi_init}{\sailRISCVfnextRvfiInit}{}%
  \ifstrequal{#1}{ext\_rvfi\_init}{\sailRISCVfnextRvfiInit}{}%
  \ifstrequal{#1}{ext_veto_disable_C}{\sailRISCVfnextVetoDisableC}{}%
  \ifstrequal{#1}{ext\_veto\_disable\_C}{\sailRISCVfnextVetoDisableC}{}%
  \ifstrequal{#1}{ext_write_fcsr}{\sailRISCVfnextWriteFcsr}{}%
  \ifstrequal{#1}{ext\_write\_fcsr}{\sailRISCVfnextWriteFcsr}{}%
  \ifstrequal{#1}{extend_value}{\sailRISCVfnextendValue}{}%
  \ifstrequal{#1}{extend\_value}{\sailRISCVfnextendValue}{}%
  \ifstrequal{#1}{extop_zbb_of_num}{\sailRISCVfnextopZbbOfNum}{}%
  \ifstrequal{#1}{extop\_zbb\_of\_num}{\sailRISCVfnextopZbbOfNum}{}%
  \ifstrequal{#1}{f_bin_op_D_of_num}{\sailRISCVfnfBinOpDOfNum}{}%
  \ifstrequal{#1}{f\_bin\_op\_D\_of\_num}{\sailRISCVfnfBinOpDOfNum}{}%
  \ifstrequal{#1}{f_bin_op_H_of_num}{\sailRISCVfnfBinOpHOfNum}{}%
  \ifstrequal{#1}{f\_bin\_op\_H\_of\_num}{\sailRISCVfnfBinOpHOfNum}{}%
  \ifstrequal{#1}{f_bin_op_S_of_num}{\sailRISCVfnfBinOpSOfNum}{}%
  \ifstrequal{#1}{f\_bin\_op\_S\_of\_num}{\sailRISCVfnfBinOpSOfNum}{}%
  \ifstrequal{#1}{f_bin_rm_op_D_of_num}{\sailRISCVfnfBinRmOpDOfNum}{}%
  \ifstrequal{#1}{f\_bin\_rm\_op\_D\_of\_num}{\sailRISCVfnfBinRmOpDOfNum}{}%
  \ifstrequal{#1}{f_bin_rm_op_H_of_num}{\sailRISCVfnfBinRmOpHOfNum}{}%
  \ifstrequal{#1}{f\_bin\_rm\_op\_H\_of\_num}{\sailRISCVfnfBinRmOpHOfNum}{}%
  \ifstrequal{#1}{f_bin_rm_op_S_of_num}{\sailRISCVfnfBinRmOpSOfNum}{}%
  \ifstrequal{#1}{f\_bin\_rm\_op\_S\_of\_num}{\sailRISCVfnfBinRmOpSOfNum}{}%
  \ifstrequal{#1}{f_is_NaN_D}{\sailRISCVfnfIsNaND}{}%
  \ifstrequal{#1}{f\_is\_NaN\_D}{\sailRISCVfnfIsNaND}{}%
  \ifstrequal{#1}{f_is_NaN_S}{\sailRISCVfnfIsNaNS}{}%
  \ifstrequal{#1}{f\_is\_NaN\_S}{\sailRISCVfnfIsNaNS}{}%
  \ifstrequal{#1}{f_is_QNaN_D}{\sailRISCVfnfIsQNaND}{}%
  \ifstrequal{#1}{f\_is\_QNaN\_D}{\sailRISCVfnfIsQNaND}{}%
  \ifstrequal{#1}{f_is_QNaN_S}{\sailRISCVfnfIsQNaNS}{}%
  \ifstrequal{#1}{f\_is\_QNaN\_S}{\sailRISCVfnfIsQNaNS}{}%
  \ifstrequal{#1}{f_is_SNaN_D}{\sailRISCVfnfIsSNaND}{}%
  \ifstrequal{#1}{f\_is\_SNaN\_D}{\sailRISCVfnfIsSNaND}{}%
  \ifstrequal{#1}{f_is_SNaN_S}{\sailRISCVfnfIsSNaNS}{}%
  \ifstrequal{#1}{f\_is\_SNaN\_S}{\sailRISCVfnfIsSNaNS}{}%
  \ifstrequal{#1}{f_is_neg_inf_D}{\sailRISCVfnfIsNegInfD}{}%
  \ifstrequal{#1}{f\_is\_neg\_inf\_D}{\sailRISCVfnfIsNegInfD}{}%
  \ifstrequal{#1}{f_is_neg_inf_S}{\sailRISCVfnfIsNegInfS}{}%
  \ifstrequal{#1}{f\_is\_neg\_inf\_S}{\sailRISCVfnfIsNegInfS}{}%
  \ifstrequal{#1}{f_is_neg_norm_D}{\sailRISCVfnfIsNegNormD}{}%
  \ifstrequal{#1}{f\_is\_neg\_norm\_D}{\sailRISCVfnfIsNegNormD}{}%
  \ifstrequal{#1}{f_is_neg_norm_S}{\sailRISCVfnfIsNegNormS}{}%
  \ifstrequal{#1}{f\_is\_neg\_norm\_S}{\sailRISCVfnfIsNegNormS}{}%
  \ifstrequal{#1}{f_is_neg_subnorm_D}{\sailRISCVfnfIsNegSubnormD}{}%
  \ifstrequal{#1}{f\_is\_neg\_subnorm\_D}{\sailRISCVfnfIsNegSubnormD}{}%
  \ifstrequal{#1}{f_is_neg_subnorm_S}{\sailRISCVfnfIsNegSubnormS}{}%
  \ifstrequal{#1}{f\_is\_neg\_subnorm\_S}{\sailRISCVfnfIsNegSubnormS}{}%
  \ifstrequal{#1}{f_is_neg_zero_D}{\sailRISCVfnfIsNegZeroD}{}%
  \ifstrequal{#1}{f\_is\_neg\_zero\_D}{\sailRISCVfnfIsNegZeroD}{}%
  \ifstrequal{#1}{f_is_neg_zero_S}{\sailRISCVfnfIsNegZeroS}{}%
  \ifstrequal{#1}{f\_is\_neg\_zero\_S}{\sailRISCVfnfIsNegZeroS}{}%
  \ifstrequal{#1}{f_is_pos_inf_D}{\sailRISCVfnfIsPosInfD}{}%
  \ifstrequal{#1}{f\_is\_pos\_inf\_D}{\sailRISCVfnfIsPosInfD}{}%
  \ifstrequal{#1}{f_is_pos_inf_S}{\sailRISCVfnfIsPosInfS}{}%
  \ifstrequal{#1}{f\_is\_pos\_inf\_S}{\sailRISCVfnfIsPosInfS}{}%
  \ifstrequal{#1}{f_is_pos_norm_D}{\sailRISCVfnfIsPosNormD}{}%
  \ifstrequal{#1}{f\_is\_pos\_norm\_D}{\sailRISCVfnfIsPosNormD}{}%
  \ifstrequal{#1}{f_is_pos_norm_S}{\sailRISCVfnfIsPosNormS}{}%
  \ifstrequal{#1}{f\_is\_pos\_norm\_S}{\sailRISCVfnfIsPosNormS}{}%
  \ifstrequal{#1}{f_is_pos_subnorm_D}{\sailRISCVfnfIsPosSubnormD}{}%
  \ifstrequal{#1}{f\_is\_pos\_subnorm\_D}{\sailRISCVfnfIsPosSubnormD}{}%
  \ifstrequal{#1}{f_is_pos_subnorm_S}{\sailRISCVfnfIsPosSubnormS}{}%
  \ifstrequal{#1}{f\_is\_pos\_subnorm\_S}{\sailRISCVfnfIsPosSubnormS}{}%
  \ifstrequal{#1}{f_is_pos_zero_D}{\sailRISCVfnfIsPosZeroD}{}%
  \ifstrequal{#1}{f\_is\_pos\_zero\_D}{\sailRISCVfnfIsPosZeroD}{}%
  \ifstrequal{#1}{f_is_pos_zero_S}{\sailRISCVfnfIsPosZeroS}{}%
  \ifstrequal{#1}{f\_is\_pos\_zero\_S}{\sailRISCVfnfIsPosZeroS}{}%
  \ifstrequal{#1}{f_madd_op_D_of_num}{\sailRISCVfnfMaddOpDOfNum}{}%
  \ifstrequal{#1}{f\_madd\_op\_D\_of\_num}{\sailRISCVfnfMaddOpDOfNum}{}%
  \ifstrequal{#1}{f_madd_op_H_of_num}{\sailRISCVfnfMaddOpHOfNum}{}%
  \ifstrequal{#1}{f\_madd\_op\_H\_of\_num}{\sailRISCVfnfMaddOpHOfNum}{}%
  \ifstrequal{#1}{f_madd_op_S_of_num}{\sailRISCVfnfMaddOpSOfNum}{}%
  \ifstrequal{#1}{f\_madd\_op\_S\_of\_num}{\sailRISCVfnfMaddOpSOfNum}{}%
  \ifstrequal{#1}{f_un_op_D_of_num}{\sailRISCVfnfUnOpDOfNum}{}%
  \ifstrequal{#1}{f\_un\_op\_D\_of\_num}{\sailRISCVfnfUnOpDOfNum}{}%
  \ifstrequal{#1}{f_un_op_H_of_num}{\sailRISCVfnfUnOpHOfNum}{}%
  \ifstrequal{#1}{f\_un\_op\_H\_of\_num}{\sailRISCVfnfUnOpHOfNum}{}%
  \ifstrequal{#1}{f_un_op_S_of_num}{\sailRISCVfnfUnOpSOfNum}{}%
  \ifstrequal{#1}{f\_un\_op\_S\_of\_num}{\sailRISCVfnfUnOpSOfNum}{}%
  \ifstrequal{#1}{f_un_rm_op_D_of_num}{\sailRISCVfnfUnRmOpDOfNum}{}%
  \ifstrequal{#1}{f\_un\_rm\_op\_D\_of\_num}{\sailRISCVfnfUnRmOpDOfNum}{}%
  \ifstrequal{#1}{f_un_rm_op_H_of_num}{\sailRISCVfnfUnRmOpHOfNum}{}%
  \ifstrequal{#1}{f\_un\_rm\_op\_H\_of\_num}{\sailRISCVfnfUnRmOpHOfNum}{}%
  \ifstrequal{#1}{f_un_rm_op_S_of_num}{\sailRISCVfnfUnRmOpSOfNum}{}%
  \ifstrequal{#1}{f\_un\_rm\_op\_S\_of\_num}{\sailRISCVfnfUnRmOpSOfNum}{}%
  \ifstrequal{#1}{fastRepCheck}{\sailRISCVfnfastRepCheck}{}%
  \ifstrequal{#1}{fdiv_int}{\sailRISCVfnfdivInt}{}%
  \ifstrequal{#1}{fdiv\_int}{\sailRISCVfnfdivInt}{}%
  \ifstrequal{#1}{feq_quiet_D}{\sailRISCVfnfeqQuietD}{}%
  \ifstrequal{#1}{feq\_quiet\_D}{\sailRISCVfnfeqQuietD}{}%
  \ifstrequal{#1}{feq_quiet_S}{\sailRISCVfnfeqQuietS}{}%
  \ifstrequal{#1}{feq\_quiet\_S}{\sailRISCVfnfeqQuietS}{}%
  \ifstrequal{#1}{fetch}{\sailRISCVfnfetch}{}%
  \ifstrequal{#1}{findPendingInterrupt}{\sailRISCVfnfindPendingInterrupt}{}%
  \ifstrequal{#1}{fle_D}{\sailRISCVfnfleD}{}%
  \ifstrequal{#1}{fle\_D}{\sailRISCVfnfleD}{}%
  \ifstrequal{#1}{fle_S}{\sailRISCVfnfleS}{}%
  \ifstrequal{#1}{fle\_S}{\sailRISCVfnfleS}{}%
  \ifstrequal{#1}{flt_D}{\sailRISCVfnfltD}{}%
  \ifstrequal{#1}{flt\_D}{\sailRISCVfnfltD}{}%
  \ifstrequal{#1}{flt_S}{\sailRISCVfnfltS}{}%
  \ifstrequal{#1}{flt\_S}{\sailRISCVfnfltS}{}%
  \ifstrequal{#1}{flush_TLB}{\sailRISCVfnflushTLB}{}%
  \ifstrequal{#1}{flush\_TLB}{\sailRISCVfnflushTLB}{}%
  \ifstrequal{#1}{flush_TLB39}{\sailRISCVfnflushTLBThreeNine}{}%
  \ifstrequal{#1}{flush\_TLB39}{\sailRISCVfnflushTLBThreeNine}{}%
  \ifstrequal{#1}{flush_TLB48}{\sailRISCVfnflushTLBFourEight}{}%
  \ifstrequal{#1}{flush\_TLB48}{\sailRISCVfnflushTLBFourEight}{}%
  \ifstrequal{#1}{flush_TLB_Entry}{\sailRISCVfnflushTLBEntry}{}%
  \ifstrequal{#1}{flush\_TLB\_Entry}{\sailRISCVfnflushTLBEntry}{}%
  \ifstrequal{#1}{fmake_D}{\sailRISCVfnfmakeD}{}%
  \ifstrequal{#1}{fmake\_D}{\sailRISCVfnfmakeD}{}%
  \ifstrequal{#1}{fmake_S}{\sailRISCVfnfmakeS}{}%
  \ifstrequal{#1}{fmake\_S}{\sailRISCVfnfmakeS}{}%
  \ifstrequal{#1}{fmod_int}{\sailRISCVfnfmodInt}{}%
  \ifstrequal{#1}{fmod\_int}{\sailRISCVfnfmodInt}{}%
  \ifstrequal{#1}{fregval_from_freg}{\sailRISCVfnfregvalFromFreg}{}%
  \ifstrequal{#1}{fregval\_from\_freg}{\sailRISCVfnfregvalFromFreg}{}%
  \ifstrequal{#1}{fregval_into_freg}{\sailRISCVfnfregvalIntoFreg}{}%
  \ifstrequal{#1}{fregval\_into\_freg}{\sailRISCVfnfregvalIntoFreg}{}%
  \ifstrequal{#1}{fsplit_D}{\sailRISCVfnfsplitD}{}%
  \ifstrequal{#1}{fsplit\_D}{\sailRISCVfnfsplitD}{}%
  \ifstrequal{#1}{fsplit_S}{\sailRISCVfnfsplitS}{}%
  \ifstrequal{#1}{fsplit\_S}{\sailRISCVfnfsplitS}{}%
  \ifstrequal{#1}{getCapBase}{\sailRISCVfngetCapBase}{}%
  \ifstrequal{#1}{getCapBaseBits}{\sailRISCVfngetCapBaseBits}{}%
  \ifstrequal{#1}{getCapBounds}{\sailRISCVfngetCapBounds}{}%
  \ifstrequal{#1}{getCapBoundsBits}{\sailRISCVfngetCapBoundsBits}{}%
  \ifstrequal{#1}{getCapCursor}{\sailRISCVfngetCapCursor}{}%
  \ifstrequal{#1}{getCapFlags}{\sailRISCVfngetCapFlags}{}%
  \ifstrequal{#1}{getCapHardPerms}{\sailRISCVfngetCapHardPerms}{}%
  \ifstrequal{#1}{getCapLength}{\sailRISCVfngetCapLength}{}%
  \ifstrequal{#1}{getCapOffset}{\sailRISCVfngetCapOffset}{}%
  \ifstrequal{#1}{getCapOffsetBits}{\sailRISCVfngetCapOffsetBits}{}%
  \ifstrequal{#1}{getCapPerms}{\sailRISCVfngetCapPerms}{}%
  \ifstrequal{#1}{getCapTop}{\sailRISCVfngetCapTop}{}%
  \ifstrequal{#1}{getCapTopBits}{\sailRISCVfngetCapTopBits}{}%
  \ifstrequal{#1}{getPendingSet}{\sailRISCVfngetPendingSet}{}%
  \ifstrequal{#1}{getRepresentableAlignmentMask}{\sailRISCVfngetRepresentableAlignmentMask}{}%
  \ifstrequal{#1}{getRepresentableLength}{\sailRISCVfngetRepresentableLength}{}%
  \ifstrequal{#1}{get_arch_pc}{\sailRISCVfngetArchPc}{}%
  \ifstrequal{#1}{get\_arch\_pc}{\sailRISCVfngetArchPc}{}%
  \ifstrequal{#1}{get_cheri_mode_cap_addr}{\sailRISCVfngetCheriModeCapAddr}{}%
  \ifstrequal{#1}{get\_cheri\_mode\_cap\_addr}{\sailRISCVfngetCheriModeCapAddr}{}%
  \ifstrequal{#1}{get_config_print_instr}{\sailRISCVfngetConfigPrintInstr}{}%
  \ifstrequal{#1}{get\_config\_print\_instr}{\sailRISCVfngetConfigPrintInstr}{}%
  \ifstrequal{#1}{get_config_print_mem}{\sailRISCVfngetConfigPrintMem}{}%
  \ifstrequal{#1}{get\_config\_print\_mem}{\sailRISCVfngetConfigPrintMem}{}%
  \ifstrequal{#1}{get_config_print_platform}{\sailRISCVfngetConfigPrintPlatform}{}%
  \ifstrequal{#1}{get\_config\_print\_platform}{\sailRISCVfngetConfigPrintPlatform}{}%
  \ifstrequal{#1}{get_config_print_reg}{\sailRISCVfngetConfigPrintReg}{}%
  \ifstrequal{#1}{get\_config\_print\_reg}{\sailRISCVfngetConfigPrintReg}{}%
  \ifstrequal{#1}{get_mstatus_SXL}{\sailRISCVfngetMstatusSXL}{}%
  \ifstrequal{#1}{get\_mstatus\_SXL}{\sailRISCVfngetMstatusSXL}{}%
  \ifstrequal{#1}{get_mstatus_UXL}{\sailRISCVfngetMstatusUXL}{}%
  \ifstrequal{#1}{get\_mstatus\_UXL}{\sailRISCVfngetMstatusUXL}{}%
  \ifstrequal{#1}{get_mtvec}{\sailRISCVfngetMtvec}{}%
  \ifstrequal{#1}{get\_mtvec}{\sailRISCVfngetMtvec}{}%
  \ifstrequal{#1}{get_next_pc}{\sailRISCVfngetNextPc}{}%
  \ifstrequal{#1}{get\_next\_pc}{\sailRISCVfngetNextPc}{}%
  \ifstrequal{#1}{get_sstatus_UXL}{\sailRISCVfngetSstatusUXL}{}%
  \ifstrequal{#1}{get\_sstatus\_UXL}{\sailRISCVfngetSstatusUXL}{}%
  \ifstrequal{#1}{get_stvec}{\sailRISCVfngetStvec}{}%
  \ifstrequal{#1}{get\_stvec}{\sailRISCVfngetStvec}{}%
  \ifstrequal{#1}{get_utvec}{\sailRISCVfngetUtvec}{}%
  \ifstrequal{#1}{get\_utvec}{\sailRISCVfngetUtvec}{}%
  \ifstrequal{#1}{get_xret_target}{\sailRISCVfngetXretTarget}{}%
  \ifstrequal{#1}{get\_xret\_target}{\sailRISCVfngetXretTarget}{}%
  \ifstrequal{#1}{handle_cheri_cap_exception}{\sailRISCVfnhandleCheriCapException}{}%
  \ifstrequal{#1}{handle\_cheri\_cap\_exception}{\sailRISCVfnhandleCheriCapException}{}%
  \ifstrequal{#1}{handle_cheri_pcc_exception}{\sailRISCVfnhandleCheriPccException}{}%
  \ifstrequal{#1}{handle\_cheri\_pcc\_exception}{\sailRISCVfnhandleCheriPccException}{}%
  \ifstrequal{#1}{handle_cheri_reg_exception}{\sailRISCVfnhandleCheriRegException}{}%
  \ifstrequal{#1}{handle\_cheri\_reg\_exception}{\sailRISCVfnhandleCheriRegException}{}%
  \ifstrequal{#1}{handle_exception}{\sailRISCVfnhandleException}{}%
  \ifstrequal{#1}{handle\_exception}{\sailRISCVfnhandleException}{}%
  \ifstrequal{#1}{handle_illegal}{\sailRISCVfnhandleIllegal}{}%
  \ifstrequal{#1}{handle\_illegal}{\sailRISCVfnhandleIllegal}{}%
  \ifstrequal{#1}{handle_interrupt}{\sailRISCVfnhandleInterrupt}{}%
  \ifstrequal{#1}{handle\_interrupt}{\sailRISCVfnhandleInterrupt}{}%
  \ifstrequal{#1}{handle_load_cap_via_cap}{\sailRISCVfnhandleLoadCapViaCap}{}%
  \ifstrequal{#1}{handle\_load\_cap\_via\_cap}{\sailRISCVfnhandleLoadCapViaCap}{}%
  \ifstrequal{#1}{handle_load_data_via_cap}{\sailRISCVfnhandleLoadDataViaCap}{}%
  \ifstrequal{#1}{handle\_load\_data\_via\_cap}{\sailRISCVfnhandleLoadDataViaCap}{}%
  \ifstrequal{#1}{handle_loadres_cap_via_cap}{\sailRISCVfnhandleLoadresCapViaCap}{}%
  \ifstrequal{#1}{handle\_loadres\_cap\_via\_cap}{\sailRISCVfnhandleLoadresCapViaCap}{}%
  \ifstrequal{#1}{handle_loadres_data_via_cap}{\sailRISCVfnhandleLoadresDataViaCap}{}%
  \ifstrequal{#1}{handle\_loadres\_data\_via\_cap}{\sailRISCVfnhandleLoadresDataViaCap}{}%
  \ifstrequal{#1}{handle_mem_exception}{\sailRISCVfnhandleMemException}{}%
  \ifstrequal{#1}{handle\_mem\_exception}{\sailRISCVfnhandleMemException}{}%
  \ifstrequal{#1}{handle_store_cap_via_cap}{\sailRISCVfnhandleStoreCapViaCap}{}%
  \ifstrequal{#1}{handle\_store\_cap\_via\_cap}{\sailRISCVfnhandleStoreCapViaCap}{}%
  \ifstrequal{#1}{handle_store_cond_cap_via_cap}{\sailRISCVfnhandleStoreCondCapViaCap}{}%
  \ifstrequal{#1}{handle\_store\_cond\_cap\_via\_cap}{\sailRISCVfnhandleStoreCondCapViaCap}{}%
  \ifstrequal{#1}{handle_store_cond_data_via_cap}{\sailRISCVfnhandleStoreCondDataViaCap}{}%
  \ifstrequal{#1}{handle\_store\_cond\_data\_via\_cap}{\sailRISCVfnhandleStoreCondDataViaCap}{}%
  \ifstrequal{#1}{handle_store_data_via_cap}{\sailRISCVfnhandleStoreDataViaCap}{}%
  \ifstrequal{#1}{handle\_store\_data\_via\_cap}{\sailRISCVfnhandleStoreDataViaCap}{}%
  \ifstrequal{#1}{handle_trap_extension}{\sailRISCVfnhandleTrapExtension}{}%
  \ifstrequal{#1}{handle\_trap\_extension}{\sailRISCVfnhandleTrapExtension}{}%
  \ifstrequal{#1}{hasReservedOType}{\sailRISCVfnhasReservedOType}{}%
  \ifstrequal{#1}{haveAtomics}{\sailRISCVfnhaveAtomics}{}%
  \ifstrequal{#1}{haveDExt}{\sailRISCVfnhaveDExt}{}%
  \ifstrequal{#1}{haveDoubleFPU}{\sailRISCVfnhaveDoubleFPU}{}%
  \ifstrequal{#1}{haveFExt}{\sailRISCVfnhaveFExt}{}%
  \ifstrequal{#1}{haveMulDiv}{\sailRISCVfnhaveMulDiv}{}%
  \ifstrequal{#1}{haveNExt}{\sailRISCVfnhaveNExt}{}%
  \ifstrequal{#1}{haveRVC}{\sailRISCVfnhaveRVC}{}%
  \ifstrequal{#1}{haveSingleFPU}{\sailRISCVfnhaveSingleFPU}{}%
  \ifstrequal{#1}{haveSupMode}{\sailRISCVfnhaveSupMode}{}%
  \ifstrequal{#1}{haveUsrMode}{\sailRISCVfnhaveUsrMode}{}%
  \ifstrequal{#1}{haveXcheri}{\sailRISCVfnhaveXcheri}{}%
  \ifstrequal{#1}{haveZba}{\sailRISCVfnhaveZba}{}%
  \ifstrequal{#1}{haveZbb}{\sailRISCVfnhaveZbb}{}%
  \ifstrequal{#1}{haveZbc}{\sailRISCVfnhaveZbc}{}%
  \ifstrequal{#1}{haveZbkb}{\sailRISCVfnhaveZbkb}{}%
  \ifstrequal{#1}{haveZbkc}{\sailRISCVfnhaveZbkc}{}%
  \ifstrequal{#1}{haveZbkx}{\sailRISCVfnhaveZbkx}{}%
  \ifstrequal{#1}{haveZbs}{\sailRISCVfnhaveZbs}{}%
  \ifstrequal{#1}{haveZdinx}{\sailRISCVfnhaveZdinx}{}%
  \ifstrequal{#1}{haveZfh}{\sailRISCVfnhaveZfh}{}%
  \ifstrequal{#1}{haveZfinx}{\sailRISCVfnhaveZfinx}{}%
  \ifstrequal{#1}{haveZhinx}{\sailRISCVfnhaveZhinx}{}%
  \ifstrequal{#1}{haveZknd}{\sailRISCVfnhaveZknd}{}%
  \ifstrequal{#1}{haveZkne}{\sailRISCVfnhaveZkne}{}%
  \ifstrequal{#1}{haveZknh}{\sailRISCVfnhaveZknh}{}%
  \ifstrequal{#1}{haveZkr}{\sailRISCVfnhaveZkr}{}%
  \ifstrequal{#1}{haveZksed}{\sailRISCVfnhaveZksed}{}%
  \ifstrequal{#1}{haveZksh}{\sailRISCVfnhaveZksh}{}%
  \ifstrequal{#1}{haveZmmul}{\sailRISCVfnhaveZmmul}{}%
  \ifstrequal{#1}{hex_bits_10_backwards}{\sailRISCVfnhexBitsOneZeroBackwards}{}%
  \ifstrequal{#1}{hex\_bits\_10\_backwards}{\sailRISCVfnhexBitsOneZeroBackwards}{}%
  \ifstrequal{#1}{hex_bits_10_backwards_matches}{\sailRISCVfnhexBitsOneZeroBackwardsMatches}{}%
  \ifstrequal{#1}{hex\_bits\_10\_backwards\_matches}{\sailRISCVfnhexBitsOneZeroBackwardsMatches}{}%
  \ifstrequal{#1}{hex_bits_10_forwards_matches}{\sailRISCVfnhexBitsOneZeroForwardsMatches}{}%
  \ifstrequal{#1}{hex\_bits\_10\_forwards\_matches}{\sailRISCVfnhexBitsOneZeroForwardsMatches}{}%
  \ifstrequal{#1}{hex_bits_11_backwards}{\sailRISCVfnhexBitsOneOneBackwards}{}%
  \ifstrequal{#1}{hex\_bits\_11\_backwards}{\sailRISCVfnhexBitsOneOneBackwards}{}%
  \ifstrequal{#1}{hex_bits_11_backwards_matches}{\sailRISCVfnhexBitsOneOneBackwardsMatches}{}%
  \ifstrequal{#1}{hex\_bits\_11\_backwards\_matches}{\sailRISCVfnhexBitsOneOneBackwardsMatches}{}%
  \ifstrequal{#1}{hex_bits_11_forwards_matches}{\sailRISCVfnhexBitsOneOneForwardsMatches}{}%
  \ifstrequal{#1}{hex\_bits\_11\_forwards\_matches}{\sailRISCVfnhexBitsOneOneForwardsMatches}{}%
  \ifstrequal{#1}{hex_bits_12_backwards}{\sailRISCVfnhexBitsOneTwoBackwards}{}%
  \ifstrequal{#1}{hex\_bits\_12\_backwards}{\sailRISCVfnhexBitsOneTwoBackwards}{}%
  \ifstrequal{#1}{hex_bits_12_backwards_matches}{\sailRISCVfnhexBitsOneTwoBackwardsMatches}{}%
  \ifstrequal{#1}{hex\_bits\_12\_backwards\_matches}{\sailRISCVfnhexBitsOneTwoBackwardsMatches}{}%
  \ifstrequal{#1}{hex_bits_12_forwards_matches}{\sailRISCVfnhexBitsOneTwoForwardsMatches}{}%
  \ifstrequal{#1}{hex\_bits\_12\_forwards\_matches}{\sailRISCVfnhexBitsOneTwoForwardsMatches}{}%
  \ifstrequal{#1}{hex_bits_12_matches_prefix}{\sailRISCVfnhexBitsOneTwoMatchesPrefix}{}%
  \ifstrequal{#1}{hex\_bits\_12\_matches\_prefix}{\sailRISCVfnhexBitsOneTwoMatchesPrefix}{}%
  \ifstrequal{#1}{hex_bits_13_backwards}{\sailRISCVfnhexBitsOneThreeBackwards}{}%
  \ifstrequal{#1}{hex\_bits\_13\_backwards}{\sailRISCVfnhexBitsOneThreeBackwards}{}%
  \ifstrequal{#1}{hex_bits_13_backwards_matches}{\sailRISCVfnhexBitsOneThreeBackwardsMatches}{}%
  \ifstrequal{#1}{hex\_bits\_13\_backwards\_matches}{\sailRISCVfnhexBitsOneThreeBackwardsMatches}{}%
  \ifstrequal{#1}{hex_bits_13_forwards_matches}{\sailRISCVfnhexBitsOneThreeForwardsMatches}{}%
  \ifstrequal{#1}{hex\_bits\_13\_forwards\_matches}{\sailRISCVfnhexBitsOneThreeForwardsMatches}{}%
  \ifstrequal{#1}{hex_bits_14_backwards}{\sailRISCVfnhexBitsOneFourBackwards}{}%
  \ifstrequal{#1}{hex\_bits\_14\_backwards}{\sailRISCVfnhexBitsOneFourBackwards}{}%
  \ifstrequal{#1}{hex_bits_14_backwards_matches}{\sailRISCVfnhexBitsOneFourBackwardsMatches}{}%
  \ifstrequal{#1}{hex\_bits\_14\_backwards\_matches}{\sailRISCVfnhexBitsOneFourBackwardsMatches}{}%
  \ifstrequal{#1}{hex_bits_14_forwards_matches}{\sailRISCVfnhexBitsOneFourForwardsMatches}{}%
  \ifstrequal{#1}{hex\_bits\_14\_forwards\_matches}{\sailRISCVfnhexBitsOneFourForwardsMatches}{}%
  \ifstrequal{#1}{hex_bits_15_backwards}{\sailRISCVfnhexBitsOneFiveBackwards}{}%
  \ifstrequal{#1}{hex\_bits\_15\_backwards}{\sailRISCVfnhexBitsOneFiveBackwards}{}%
  \ifstrequal{#1}{hex_bits_15_backwards_matches}{\sailRISCVfnhexBitsOneFiveBackwardsMatches}{}%
  \ifstrequal{#1}{hex\_bits\_15\_backwards\_matches}{\sailRISCVfnhexBitsOneFiveBackwardsMatches}{}%
  \ifstrequal{#1}{hex_bits_15_forwards_matches}{\sailRISCVfnhexBitsOneFiveForwardsMatches}{}%
  \ifstrequal{#1}{hex\_bits\_15\_forwards\_matches}{\sailRISCVfnhexBitsOneFiveForwardsMatches}{}%
  \ifstrequal{#1}{hex_bits_16_backwards}{\sailRISCVfnhexBitsOneSixBackwards}{}%
  \ifstrequal{#1}{hex\_bits\_16\_backwards}{\sailRISCVfnhexBitsOneSixBackwards}{}%
  \ifstrequal{#1}{hex_bits_16_backwards_matches}{\sailRISCVfnhexBitsOneSixBackwardsMatches}{}%
  \ifstrequal{#1}{hex\_bits\_16\_backwards\_matches}{\sailRISCVfnhexBitsOneSixBackwardsMatches}{}%
  \ifstrequal{#1}{hex_bits_16_forwards_matches}{\sailRISCVfnhexBitsOneSixForwardsMatches}{}%
  \ifstrequal{#1}{hex\_bits\_16\_forwards\_matches}{\sailRISCVfnhexBitsOneSixForwardsMatches}{}%
  \ifstrequal{#1}{hex_bits_17_backwards}{\sailRISCVfnhexBitsOneSevenBackwards}{}%
  \ifstrequal{#1}{hex\_bits\_17\_backwards}{\sailRISCVfnhexBitsOneSevenBackwards}{}%
  \ifstrequal{#1}{hex_bits_17_backwards_matches}{\sailRISCVfnhexBitsOneSevenBackwardsMatches}{}%
  \ifstrequal{#1}{hex\_bits\_17\_backwards\_matches}{\sailRISCVfnhexBitsOneSevenBackwardsMatches}{}%
  \ifstrequal{#1}{hex_bits_17_forwards_matches}{\sailRISCVfnhexBitsOneSevenForwardsMatches}{}%
  \ifstrequal{#1}{hex\_bits\_17\_forwards\_matches}{\sailRISCVfnhexBitsOneSevenForwardsMatches}{}%
  \ifstrequal{#1}{hex_bits_18_backwards}{\sailRISCVfnhexBitsOneEightBackwards}{}%
  \ifstrequal{#1}{hex\_bits\_18\_backwards}{\sailRISCVfnhexBitsOneEightBackwards}{}%
  \ifstrequal{#1}{hex_bits_18_backwards_matches}{\sailRISCVfnhexBitsOneEightBackwardsMatches}{}%
  \ifstrequal{#1}{hex\_bits\_18\_backwards\_matches}{\sailRISCVfnhexBitsOneEightBackwardsMatches}{}%
  \ifstrequal{#1}{hex_bits_18_forwards_matches}{\sailRISCVfnhexBitsOneEightForwardsMatches}{}%
  \ifstrequal{#1}{hex\_bits\_18\_forwards\_matches}{\sailRISCVfnhexBitsOneEightForwardsMatches}{}%
  \ifstrequal{#1}{hex_bits_19_backwards}{\sailRISCVfnhexBitsOneNineBackwards}{}%
  \ifstrequal{#1}{hex\_bits\_19\_backwards}{\sailRISCVfnhexBitsOneNineBackwards}{}%
  \ifstrequal{#1}{hex_bits_19_backwards_matches}{\sailRISCVfnhexBitsOneNineBackwardsMatches}{}%
  \ifstrequal{#1}{hex\_bits\_19\_backwards\_matches}{\sailRISCVfnhexBitsOneNineBackwardsMatches}{}%
  \ifstrequal{#1}{hex_bits_19_forwards_matches}{\sailRISCVfnhexBitsOneNineForwardsMatches}{}%
  \ifstrequal{#1}{hex\_bits\_19\_forwards\_matches}{\sailRISCVfnhexBitsOneNineForwardsMatches}{}%
  \ifstrequal{#1}{hex_bits_1_backwards}{\sailRISCVfnhexBitsOneBackwards}{}%
  \ifstrequal{#1}{hex\_bits\_1\_backwards}{\sailRISCVfnhexBitsOneBackwards}{}%
  \ifstrequal{#1}{hex_bits_1_backwards_matches}{\sailRISCVfnhexBitsOneBackwardsMatches}{}%
  \ifstrequal{#1}{hex\_bits\_1\_backwards\_matches}{\sailRISCVfnhexBitsOneBackwardsMatches}{}%
  \ifstrequal{#1}{hex_bits_1_forwards_matches}{\sailRISCVfnhexBitsOneForwardsMatches}{}%
  \ifstrequal{#1}{hex\_bits\_1\_forwards\_matches}{\sailRISCVfnhexBitsOneForwardsMatches}{}%
  \ifstrequal{#1}{hex_bits_20_backwards}{\sailRISCVfnhexBitsTwoZeroBackwards}{}%
  \ifstrequal{#1}{hex\_bits\_20\_backwards}{\sailRISCVfnhexBitsTwoZeroBackwards}{}%
  \ifstrequal{#1}{hex_bits_20_backwards_matches}{\sailRISCVfnhexBitsTwoZeroBackwardsMatches}{}%
  \ifstrequal{#1}{hex\_bits\_20\_backwards\_matches}{\sailRISCVfnhexBitsTwoZeroBackwardsMatches}{}%
  \ifstrequal{#1}{hex_bits_20_forwards_matches}{\sailRISCVfnhexBitsTwoZeroForwardsMatches}{}%
  \ifstrequal{#1}{hex\_bits\_20\_forwards\_matches}{\sailRISCVfnhexBitsTwoZeroForwardsMatches}{}%
  \ifstrequal{#1}{hex_bits_21_backwards}{\sailRISCVfnhexBitsTwoOneBackwards}{}%
  \ifstrequal{#1}{hex\_bits\_21\_backwards}{\sailRISCVfnhexBitsTwoOneBackwards}{}%
  \ifstrequal{#1}{hex_bits_21_backwards_matches}{\sailRISCVfnhexBitsTwoOneBackwardsMatches}{}%
  \ifstrequal{#1}{hex\_bits\_21\_backwards\_matches}{\sailRISCVfnhexBitsTwoOneBackwardsMatches}{}%
  \ifstrequal{#1}{hex_bits_21_forwards_matches}{\sailRISCVfnhexBitsTwoOneForwardsMatches}{}%
  \ifstrequal{#1}{hex\_bits\_21\_forwards\_matches}{\sailRISCVfnhexBitsTwoOneForwardsMatches}{}%
  \ifstrequal{#1}{hex_bits_22_backwards}{\sailRISCVfnhexBitsTwoTwoBackwards}{}%
  \ifstrequal{#1}{hex\_bits\_22\_backwards}{\sailRISCVfnhexBitsTwoTwoBackwards}{}%
  \ifstrequal{#1}{hex_bits_22_backwards_matches}{\sailRISCVfnhexBitsTwoTwoBackwardsMatches}{}%
  \ifstrequal{#1}{hex\_bits\_22\_backwards\_matches}{\sailRISCVfnhexBitsTwoTwoBackwardsMatches}{}%
  \ifstrequal{#1}{hex_bits_22_forwards_matches}{\sailRISCVfnhexBitsTwoTwoForwardsMatches}{}%
  \ifstrequal{#1}{hex\_bits\_22\_forwards\_matches}{\sailRISCVfnhexBitsTwoTwoForwardsMatches}{}%
  \ifstrequal{#1}{hex_bits_23_backwards}{\sailRISCVfnhexBitsTwoThreeBackwards}{}%
  \ifstrequal{#1}{hex\_bits\_23\_backwards}{\sailRISCVfnhexBitsTwoThreeBackwards}{}%
  \ifstrequal{#1}{hex_bits_23_backwards_matches}{\sailRISCVfnhexBitsTwoThreeBackwardsMatches}{}%
  \ifstrequal{#1}{hex\_bits\_23\_backwards\_matches}{\sailRISCVfnhexBitsTwoThreeBackwardsMatches}{}%
  \ifstrequal{#1}{hex_bits_23_forwards_matches}{\sailRISCVfnhexBitsTwoThreeForwardsMatches}{}%
  \ifstrequal{#1}{hex\_bits\_23\_forwards\_matches}{\sailRISCVfnhexBitsTwoThreeForwardsMatches}{}%
  \ifstrequal{#1}{hex_bits_24_backwards}{\sailRISCVfnhexBitsTwoFourBackwards}{}%
  \ifstrequal{#1}{hex\_bits\_24\_backwards}{\sailRISCVfnhexBitsTwoFourBackwards}{}%
  \ifstrequal{#1}{hex_bits_24_backwards_matches}{\sailRISCVfnhexBitsTwoFourBackwardsMatches}{}%
  \ifstrequal{#1}{hex\_bits\_24\_backwards\_matches}{\sailRISCVfnhexBitsTwoFourBackwardsMatches}{}%
  \ifstrequal{#1}{hex_bits_24_forwards_matches}{\sailRISCVfnhexBitsTwoFourForwardsMatches}{}%
  \ifstrequal{#1}{hex\_bits\_24\_forwards\_matches}{\sailRISCVfnhexBitsTwoFourForwardsMatches}{}%
  \ifstrequal{#1}{hex_bits_25_backwards}{\sailRISCVfnhexBitsTwoFiveBackwards}{}%
  \ifstrequal{#1}{hex\_bits\_25\_backwards}{\sailRISCVfnhexBitsTwoFiveBackwards}{}%
  \ifstrequal{#1}{hex_bits_25_backwards_matches}{\sailRISCVfnhexBitsTwoFiveBackwardsMatches}{}%
  \ifstrequal{#1}{hex\_bits\_25\_backwards\_matches}{\sailRISCVfnhexBitsTwoFiveBackwardsMatches}{}%
  \ifstrequal{#1}{hex_bits_25_forwards_matches}{\sailRISCVfnhexBitsTwoFiveForwardsMatches}{}%
  \ifstrequal{#1}{hex\_bits\_25\_forwards\_matches}{\sailRISCVfnhexBitsTwoFiveForwardsMatches}{}%
  \ifstrequal{#1}{hex_bits_26_backwards}{\sailRISCVfnhexBitsTwoSixBackwards}{}%
  \ifstrequal{#1}{hex\_bits\_26\_backwards}{\sailRISCVfnhexBitsTwoSixBackwards}{}%
  \ifstrequal{#1}{hex_bits_26_backwards_matches}{\sailRISCVfnhexBitsTwoSixBackwardsMatches}{}%
  \ifstrequal{#1}{hex\_bits\_26\_backwards\_matches}{\sailRISCVfnhexBitsTwoSixBackwardsMatches}{}%
  \ifstrequal{#1}{hex_bits_26_forwards_matches}{\sailRISCVfnhexBitsTwoSixForwardsMatches}{}%
  \ifstrequal{#1}{hex\_bits\_26\_forwards\_matches}{\sailRISCVfnhexBitsTwoSixForwardsMatches}{}%
  \ifstrequal{#1}{hex_bits_27_backwards}{\sailRISCVfnhexBitsTwoSevenBackwards}{}%
  \ifstrequal{#1}{hex\_bits\_27\_backwards}{\sailRISCVfnhexBitsTwoSevenBackwards}{}%
  \ifstrequal{#1}{hex_bits_27_backwards_matches}{\sailRISCVfnhexBitsTwoSevenBackwardsMatches}{}%
  \ifstrequal{#1}{hex\_bits\_27\_backwards\_matches}{\sailRISCVfnhexBitsTwoSevenBackwardsMatches}{}%
  \ifstrequal{#1}{hex_bits_27_forwards_matches}{\sailRISCVfnhexBitsTwoSevenForwardsMatches}{}%
  \ifstrequal{#1}{hex\_bits\_27\_forwards\_matches}{\sailRISCVfnhexBitsTwoSevenForwardsMatches}{}%
  \ifstrequal{#1}{hex_bits_28_backwards}{\sailRISCVfnhexBitsTwoEightBackwards}{}%
  \ifstrequal{#1}{hex\_bits\_28\_backwards}{\sailRISCVfnhexBitsTwoEightBackwards}{}%
  \ifstrequal{#1}{hex_bits_28_backwards_matches}{\sailRISCVfnhexBitsTwoEightBackwardsMatches}{}%
  \ifstrequal{#1}{hex\_bits\_28\_backwards\_matches}{\sailRISCVfnhexBitsTwoEightBackwardsMatches}{}%
  \ifstrequal{#1}{hex_bits_28_forwards_matches}{\sailRISCVfnhexBitsTwoEightForwardsMatches}{}%
  \ifstrequal{#1}{hex\_bits\_28\_forwards\_matches}{\sailRISCVfnhexBitsTwoEightForwardsMatches}{}%
  \ifstrequal{#1}{hex_bits_29_backwards}{\sailRISCVfnhexBitsTwoNineBackwards}{}%
  \ifstrequal{#1}{hex\_bits\_29\_backwards}{\sailRISCVfnhexBitsTwoNineBackwards}{}%
  \ifstrequal{#1}{hex_bits_29_backwards_matches}{\sailRISCVfnhexBitsTwoNineBackwardsMatches}{}%
  \ifstrequal{#1}{hex\_bits\_29\_backwards\_matches}{\sailRISCVfnhexBitsTwoNineBackwardsMatches}{}%
  \ifstrequal{#1}{hex_bits_29_forwards_matches}{\sailRISCVfnhexBitsTwoNineForwardsMatches}{}%
  \ifstrequal{#1}{hex\_bits\_29\_forwards\_matches}{\sailRISCVfnhexBitsTwoNineForwardsMatches}{}%
  \ifstrequal{#1}{hex_bits_2_backwards}{\sailRISCVfnhexBitsTwoBackwards}{}%
  \ifstrequal{#1}{hex\_bits\_2\_backwards}{\sailRISCVfnhexBitsTwoBackwards}{}%
  \ifstrequal{#1}{hex_bits_2_backwards_matches}{\sailRISCVfnhexBitsTwoBackwardsMatches}{}%
  \ifstrequal{#1}{hex\_bits\_2\_backwards\_matches}{\sailRISCVfnhexBitsTwoBackwardsMatches}{}%
  \ifstrequal{#1}{hex_bits_2_forwards_matches}{\sailRISCVfnhexBitsTwoForwardsMatches}{}%
  \ifstrequal{#1}{hex\_bits\_2\_forwards\_matches}{\sailRISCVfnhexBitsTwoForwardsMatches}{}%
  \ifstrequal{#1}{hex_bits_30_backwards}{\sailRISCVfnhexBitsThreeZeroBackwards}{}%
  \ifstrequal{#1}{hex\_bits\_30\_backwards}{\sailRISCVfnhexBitsThreeZeroBackwards}{}%
  \ifstrequal{#1}{hex_bits_30_backwards_matches}{\sailRISCVfnhexBitsThreeZeroBackwardsMatches}{}%
  \ifstrequal{#1}{hex\_bits\_30\_backwards\_matches}{\sailRISCVfnhexBitsThreeZeroBackwardsMatches}{}%
  \ifstrequal{#1}{hex_bits_30_forwards_matches}{\sailRISCVfnhexBitsThreeZeroForwardsMatches}{}%
  \ifstrequal{#1}{hex\_bits\_30\_forwards\_matches}{\sailRISCVfnhexBitsThreeZeroForwardsMatches}{}%
  \ifstrequal{#1}{hex_bits_31_backwards}{\sailRISCVfnhexBitsThreeOneBackwards}{}%
  \ifstrequal{#1}{hex\_bits\_31\_backwards}{\sailRISCVfnhexBitsThreeOneBackwards}{}%
  \ifstrequal{#1}{hex_bits_31_backwards_matches}{\sailRISCVfnhexBitsThreeOneBackwardsMatches}{}%
  \ifstrequal{#1}{hex\_bits\_31\_backwards\_matches}{\sailRISCVfnhexBitsThreeOneBackwardsMatches}{}%
  \ifstrequal{#1}{hex_bits_31_forwards_matches}{\sailRISCVfnhexBitsThreeOneForwardsMatches}{}%
  \ifstrequal{#1}{hex\_bits\_31\_forwards\_matches}{\sailRISCVfnhexBitsThreeOneForwardsMatches}{}%
  \ifstrequal{#1}{hex_bits_32_backwards}{\sailRISCVfnhexBitsThreeTwoBackwards}{}%
  \ifstrequal{#1}{hex\_bits\_32\_backwards}{\sailRISCVfnhexBitsThreeTwoBackwards}{}%
  \ifstrequal{#1}{hex_bits_32_backwards_matches}{\sailRISCVfnhexBitsThreeTwoBackwardsMatches}{}%
  \ifstrequal{#1}{hex\_bits\_32\_backwards\_matches}{\sailRISCVfnhexBitsThreeTwoBackwardsMatches}{}%
  \ifstrequal{#1}{hex_bits_32_forwards_matches}{\sailRISCVfnhexBitsThreeTwoForwardsMatches}{}%
  \ifstrequal{#1}{hex\_bits\_32\_forwards\_matches}{\sailRISCVfnhexBitsThreeTwoForwardsMatches}{}%
  \ifstrequal{#1}{hex_bits_33_backwards}{\sailRISCVfnhexBitsThreeThreeBackwards}{}%
  \ifstrequal{#1}{hex\_bits\_33\_backwards}{\sailRISCVfnhexBitsThreeThreeBackwards}{}%
  \ifstrequal{#1}{hex_bits_33_backwards_matches}{\sailRISCVfnhexBitsThreeThreeBackwardsMatches}{}%
  \ifstrequal{#1}{hex\_bits\_33\_backwards\_matches}{\sailRISCVfnhexBitsThreeThreeBackwardsMatches}{}%
  \ifstrequal{#1}{hex_bits_33_forwards_matches}{\sailRISCVfnhexBitsThreeThreeForwardsMatches}{}%
  \ifstrequal{#1}{hex\_bits\_33\_forwards\_matches}{\sailRISCVfnhexBitsThreeThreeForwardsMatches}{}%
  \ifstrequal{#1}{hex_bits_3_backwards}{\sailRISCVfnhexBitsThreeBackwards}{}%
  \ifstrequal{#1}{hex\_bits\_3\_backwards}{\sailRISCVfnhexBitsThreeBackwards}{}%
  \ifstrequal{#1}{hex_bits_3_backwards_matches}{\sailRISCVfnhexBitsThreeBackwardsMatches}{}%
  \ifstrequal{#1}{hex\_bits\_3\_backwards\_matches}{\sailRISCVfnhexBitsThreeBackwardsMatches}{}%
  \ifstrequal{#1}{hex_bits_3_forwards_matches}{\sailRISCVfnhexBitsThreeForwardsMatches}{}%
  \ifstrequal{#1}{hex\_bits\_3\_forwards\_matches}{\sailRISCVfnhexBitsThreeForwardsMatches}{}%
  \ifstrequal{#1}{hex_bits_48_backwards}{\sailRISCVfnhexBitsFourEightBackwards}{}%
  \ifstrequal{#1}{hex\_bits\_48\_backwards}{\sailRISCVfnhexBitsFourEightBackwards}{}%
  \ifstrequal{#1}{hex_bits_48_backwards_matches}{\sailRISCVfnhexBitsFourEightBackwardsMatches}{}%
  \ifstrequal{#1}{hex\_bits\_48\_backwards\_matches}{\sailRISCVfnhexBitsFourEightBackwardsMatches}{}%
  \ifstrequal{#1}{hex_bits_48_forwards_matches}{\sailRISCVfnhexBitsFourEightForwardsMatches}{}%
  \ifstrequal{#1}{hex\_bits\_48\_forwards\_matches}{\sailRISCVfnhexBitsFourEightForwardsMatches}{}%
  \ifstrequal{#1}{hex_bits_4_backwards}{\sailRISCVfnhexBitsFourBackwards}{}%
  \ifstrequal{#1}{hex\_bits\_4\_backwards}{\sailRISCVfnhexBitsFourBackwards}{}%
  \ifstrequal{#1}{hex_bits_4_backwards_matches}{\sailRISCVfnhexBitsFourBackwardsMatches}{}%
  \ifstrequal{#1}{hex\_bits\_4\_backwards\_matches}{\sailRISCVfnhexBitsFourBackwardsMatches}{}%
  \ifstrequal{#1}{hex_bits_4_forwards_matches}{\sailRISCVfnhexBitsFourForwardsMatches}{}%
  \ifstrequal{#1}{hex\_bits\_4\_forwards\_matches}{\sailRISCVfnhexBitsFourForwardsMatches}{}%
  \ifstrequal{#1}{hex_bits_5_backwards}{\sailRISCVfnhexBitsFiveBackwards}{}%
  \ifstrequal{#1}{hex\_bits\_5\_backwards}{\sailRISCVfnhexBitsFiveBackwards}{}%
  \ifstrequal{#1}{hex_bits_5_backwards_matches}{\sailRISCVfnhexBitsFiveBackwardsMatches}{}%
  \ifstrequal{#1}{hex\_bits\_5\_backwards\_matches}{\sailRISCVfnhexBitsFiveBackwardsMatches}{}%
  \ifstrequal{#1}{hex_bits_5_forwards_matches}{\sailRISCVfnhexBitsFiveForwardsMatches}{}%
  \ifstrequal{#1}{hex\_bits\_5\_forwards\_matches}{\sailRISCVfnhexBitsFiveForwardsMatches}{}%
  \ifstrequal{#1}{hex_bits_64_backwards}{\sailRISCVfnhexBitsSixFourBackwards}{}%
  \ifstrequal{#1}{hex\_bits\_64\_backwards}{\sailRISCVfnhexBitsSixFourBackwards}{}%
  \ifstrequal{#1}{hex_bits_64_backwards_matches}{\sailRISCVfnhexBitsSixFourBackwardsMatches}{}%
  \ifstrequal{#1}{hex\_bits\_64\_backwards\_matches}{\sailRISCVfnhexBitsSixFourBackwardsMatches}{}%
  \ifstrequal{#1}{hex_bits_64_forwards_matches}{\sailRISCVfnhexBitsSixFourForwardsMatches}{}%
  \ifstrequal{#1}{hex\_bits\_64\_forwards\_matches}{\sailRISCVfnhexBitsSixFourForwardsMatches}{}%
  \ifstrequal{#1}{hex_bits_6_backwards}{\sailRISCVfnhexBitsSixBackwards}{}%
  \ifstrequal{#1}{hex\_bits\_6\_backwards}{\sailRISCVfnhexBitsSixBackwards}{}%
  \ifstrequal{#1}{hex_bits_6_backwards_matches}{\sailRISCVfnhexBitsSixBackwardsMatches}{}%
  \ifstrequal{#1}{hex\_bits\_6\_backwards\_matches}{\sailRISCVfnhexBitsSixBackwardsMatches}{}%
  \ifstrequal{#1}{hex_bits_6_forwards_matches}{\sailRISCVfnhexBitsSixForwardsMatches}{}%
  \ifstrequal{#1}{hex\_bits\_6\_forwards\_matches}{\sailRISCVfnhexBitsSixForwardsMatches}{}%
  \ifstrequal{#1}{hex_bits_7_backwards}{\sailRISCVfnhexBitsSevenBackwards}{}%
  \ifstrequal{#1}{hex\_bits\_7\_backwards}{\sailRISCVfnhexBitsSevenBackwards}{}%
  \ifstrequal{#1}{hex_bits_7_backwards_matches}{\sailRISCVfnhexBitsSevenBackwardsMatches}{}%
  \ifstrequal{#1}{hex\_bits\_7\_backwards\_matches}{\sailRISCVfnhexBitsSevenBackwardsMatches}{}%
  \ifstrequal{#1}{hex_bits_7_forwards_matches}{\sailRISCVfnhexBitsSevenForwardsMatches}{}%
  \ifstrequal{#1}{hex\_bits\_7\_forwards\_matches}{\sailRISCVfnhexBitsSevenForwardsMatches}{}%
  \ifstrequal{#1}{hex_bits_8_backwards}{\sailRISCVfnhexBitsEightBackwards}{}%
  \ifstrequal{#1}{hex\_bits\_8\_backwards}{\sailRISCVfnhexBitsEightBackwards}{}%
  \ifstrequal{#1}{hex_bits_8_backwards_matches}{\sailRISCVfnhexBitsEightBackwardsMatches}{}%
  \ifstrequal{#1}{hex\_bits\_8\_backwards\_matches}{\sailRISCVfnhexBitsEightBackwardsMatches}{}%
  \ifstrequal{#1}{hex_bits_8_forwards_matches}{\sailRISCVfnhexBitsEightForwardsMatches}{}%
  \ifstrequal{#1}{hex\_bits\_8\_forwards\_matches}{\sailRISCVfnhexBitsEightForwardsMatches}{}%
  \ifstrequal{#1}{hex_bits_9_backwards}{\sailRISCVfnhexBitsNineBackwards}{}%
  \ifstrequal{#1}{hex\_bits\_9\_backwards}{\sailRISCVfnhexBitsNineBackwards}{}%
  \ifstrequal{#1}{hex_bits_9_backwards_matches}{\sailRISCVfnhexBitsNineBackwardsMatches}{}%
  \ifstrequal{#1}{hex\_bits\_9\_backwards\_matches}{\sailRISCVfnhexBitsNineBackwardsMatches}{}%
  \ifstrequal{#1}{hex_bits_9_forwards_matches}{\sailRISCVfnhexBitsNineForwardsMatches}{}%
  \ifstrequal{#1}{hex\_bits\_9\_forwards\_matches}{\sailRISCVfnhexBitsNineForwardsMatches}{}%
  \ifstrequal{#1}{htif_load}{\sailRISCVfnhtifLoad}{}%
  \ifstrequal{#1}{htif\_load}{\sailRISCVfnhtifLoad}{}%
  \ifstrequal{#1}{htif_store}{\sailRISCVfnhtifStore}{}%
  \ifstrequal{#1}{htif\_store}{\sailRISCVfnhtifStore}{}%
  \ifstrequal{#1}{htif_tick}{\sailRISCVfnhtifTick}{}%
  \ifstrequal{#1}{htif\_tick}{\sailRISCVfnhtifTick}{}%
  \ifstrequal{#1}{in32BitMode}{\sailRISCVfninThreeTwoBitMode}{}%
  \ifstrequal{#1}{inCapBounds}{\sailRISCVfninCapBounds}{}%
  \ifstrequal{#1}{incCapOffset}{\sailRISCVfnincCapOffset}{}%
  \ifstrequal{#1}{init_base_regs}{\sailRISCVfninitBaseRegs}{}%
  \ifstrequal{#1}{init\_base\_regs}{\sailRISCVfninitBaseRegs}{}%
  \ifstrequal{#1}{init_fdext_regs}{\sailRISCVfninitFdextRegs}{}%
  \ifstrequal{#1}{init\_fdext\_regs}{\sailRISCVfninitFdextRegs}{}%
  \ifstrequal{#1}{init_model}{\sailRISCVfninitModel}{}%
  \ifstrequal{#1}{init\_model}{\sailRISCVfninitModel}{}%
  \ifstrequal{#1}{init_platform}{\sailRISCVfninitPlatform}{}%
  \ifstrequal{#1}{init\_platform}{\sailRISCVfninitPlatform}{}%
  \ifstrequal{#1}{init_pmp}{\sailRISCVfninitPmp}{}%
  \ifstrequal{#1}{init\_pmp}{\sailRISCVfninitPmp}{}%
  \ifstrequal{#1}{init_sys}{\sailRISCVfninitSys}{}%
  \ifstrequal{#1}{init\_sys}{\sailRISCVfninitSys}{}%
  \ifstrequal{#1}{init_vmem}{\sailRISCVfninitVmem}{}%
  \ifstrequal{#1}{init\_vmem}{\sailRISCVfninitVmem}{}%
  \ifstrequal{#1}{init_vmem_sv39}{\sailRISCVfninitVmemSvThreeNine}{}%
  \ifstrequal{#1}{init\_vmem\_sv39}{\sailRISCVfninitVmemSvThreeNine}{}%
  \ifstrequal{#1}{init_vmem_sv48}{\sailRISCVfninitVmemSvFourEight}{}%
  \ifstrequal{#1}{init\_vmem\_sv48}{\sailRISCVfninitVmemSvFourEight}{}%
  \ifstrequal{#1}{initial_analysis}{\sailRISCVfninitialAnalysis}{}%
  \ifstrequal{#1}{initial\_analysis}{\sailRISCVfninitialAnalysis}{}%
  \ifstrequal{#1}{int_to_cap}{\sailRISCVfnintToCap}{}%
  \ifstrequal{#1}{int\_to\_cap}{\sailRISCVfnintToCap}{}%
  \ifstrequal{#1}{internal_error}{\sailRISCVfninternalError}{}%
  \ifstrequal{#1}{internal\_error}{\sailRISCVfninternalError}{}%
  \ifstrequal{#1}{interruptType_to_bits}{\sailRISCVfninterruptTypeToBits}{}%
  \ifstrequal{#1}{interruptType\_to\_bits}{\sailRISCVfninterruptTypeToBits}{}%
  \ifstrequal{#1}{iop_of_num}{\sailRISCVfniopOfNum}{}%
  \ifstrequal{#1}{iop\_of\_num}{\sailRISCVfniopOfNum}{}%
  \ifstrequal{#1}{isCapSealed}{\sailRISCVfnisCapSealed}{}%
  \ifstrequal{#1}{isInvalidPTE}{\sailRISCVfnisInvalidPTE}{}%
  \ifstrequal{#1}{isPTEPtr}{\sailRISCVfnisPTEPtr}{}%
  \ifstrequal{#1}{isRVC}{\sailRISCVfnisRVC}{}%
  \ifstrequal{#1}{isValidSv39Addr}{\sailRISCVfnisValidSvThreeNineAddr}{}%
  \ifstrequal{#1}{isValidSv48Addr}{\sailRISCVfnisValidSvFourEightAddr}{}%
  \ifstrequal{#1}{is_CSR_defined}{\sailRISCVfnisCSRDefined}{}%
  \ifstrequal{#1}{is\_CSR\_defined}{\sailRISCVfnisCSRDefined}{}%
  \ifstrequal{#1}{is_aligned_addr}{\sailRISCVfnisAlignedAddr}{}%
  \ifstrequal{#1}{is\_aligned\_addr}{\sailRISCVfnisAlignedAddr}{}%
  \ifstrequal{#1}{is_none}{\sailRISCVfnisNone}{}%
  \ifstrequal{#1}{is\_none}{\sailRISCVfnisNone}{}%
  \ifstrequal{#1}{is_some}{\sailRISCVfnisSome}{}%
  \ifstrequal{#1}{is\_some}{\sailRISCVfnisSome}{}%
  \ifstrequal{#1}{legalize_ccsr}{\sailRISCVfnlegalizzeCcsr}{}%
  \ifstrequal{#1}{legalize\_ccsr}{\sailRISCVfnlegalizzeCcsr}{}%
  \ifstrequal{#1}{legalize_epcc}{\sailRISCVfnlegalizzeEpcc}{}%
  \ifstrequal{#1}{legalize\_epcc}{\sailRISCVfnlegalizzeEpcc}{}%
  \ifstrequal{#1}{legalize_mcounteren}{\sailRISCVfnlegalizzeMcounteren}{}%
  \ifstrequal{#1}{legalize\_mcounteren}{\sailRISCVfnlegalizzeMcounteren}{}%
  \ifstrequal{#1}{legalize_mcountinhibit}{\sailRISCVfnlegalizzeMcountinhibit}{}%
  \ifstrequal{#1}{legalize\_mcountinhibit}{\sailRISCVfnlegalizzeMcountinhibit}{}%
  \ifstrequal{#1}{legalize_medeleg}{\sailRISCVfnlegalizzeMedeleg}{}%
  \ifstrequal{#1}{legalize\_medeleg}{\sailRISCVfnlegalizzeMedeleg}{}%
  \ifstrequal{#1}{legalize_mideleg}{\sailRISCVfnlegalizzeMideleg}{}%
  \ifstrequal{#1}{legalize\_mideleg}{\sailRISCVfnlegalizzeMideleg}{}%
  \ifstrequal{#1}{legalize_mie}{\sailRISCVfnlegalizzeMie}{}%
  \ifstrequal{#1}{legalize\_mie}{\sailRISCVfnlegalizzeMie}{}%
  \ifstrequal{#1}{legalize_mip}{\sailRISCVfnlegalizzeMip}{}%
  \ifstrequal{#1}{legalize\_mip}{\sailRISCVfnlegalizzeMip}{}%
  \ifstrequal{#1}{legalize_misa}{\sailRISCVfnlegalizzeMisa}{}%
  \ifstrequal{#1}{legalize\_misa}{\sailRISCVfnlegalizzeMisa}{}%
  \ifstrequal{#1}{legalize_mstatus}{\sailRISCVfnlegalizzeMstatus}{}%
  \ifstrequal{#1}{legalize\_mstatus}{\sailRISCVfnlegalizzeMstatus}{}%
  \ifstrequal{#1}{legalize_satp}{\sailRISCVfnlegalizzeSatp}{}%
  \ifstrequal{#1}{legalize\_satp}{\sailRISCVfnlegalizzeSatp}{}%
  \ifstrequal{#1}{legalize_satp32}{\sailRISCVfnlegalizzeSatpThreeTwo}{}%
  \ifstrequal{#1}{legalize\_satp32}{\sailRISCVfnlegalizzeSatpThreeTwo}{}%
  \ifstrequal{#1}{legalize_satp64}{\sailRISCVfnlegalizzeSatpSixFour}{}%
  \ifstrequal{#1}{legalize\_satp64}{\sailRISCVfnlegalizzeSatpSixFour}{}%
  \ifstrequal{#1}{legalize_scounteren}{\sailRISCVfnlegalizzeScounteren}{}%
  \ifstrequal{#1}{legalize\_scounteren}{\sailRISCVfnlegalizzeScounteren}{}%
  \ifstrequal{#1}{legalize_sedeleg}{\sailRISCVfnlegalizzeSedeleg}{}%
  \ifstrequal{#1}{legalize\_sedeleg}{\sailRISCVfnlegalizzeSedeleg}{}%
  \ifstrequal{#1}{legalize_sie}{\sailRISCVfnlegalizzeSie}{}%
  \ifstrequal{#1}{legalize\_sie}{\sailRISCVfnlegalizzeSie}{}%
  \ifstrequal{#1}{legalize_sip}{\sailRISCVfnlegalizzeSip}{}%
  \ifstrequal{#1}{legalize\_sip}{\sailRISCVfnlegalizzeSip}{}%
  \ifstrequal{#1}{legalize_sstatus}{\sailRISCVfnlegalizzeSstatus}{}%
  \ifstrequal{#1}{legalize\_sstatus}{\sailRISCVfnlegalizzeSstatus}{}%
  \ifstrequal{#1}{legalize_tcc}{\sailRISCVfnlegalizzeTcc}{}%
  \ifstrequal{#1}{legalize\_tcc}{\sailRISCVfnlegalizzeTcc}{}%
  \ifstrequal{#1}{legalize_tvec}{\sailRISCVfnlegalizzeTvec}{}%
  \ifstrequal{#1}{legalize\_tvec}{\sailRISCVfnlegalizzeTvec}{}%
  \ifstrequal{#1}{legalize_uie}{\sailRISCVfnlegalizzeUie}{}%
  \ifstrequal{#1}{legalize\_uie}{\sailRISCVfnlegalizzeUie}{}%
  \ifstrequal{#1}{legalize_uip}{\sailRISCVfnlegalizzeUip}{}%
  \ifstrequal{#1}{legalize\_uip}{\sailRISCVfnlegalizzeUip}{}%
  \ifstrequal{#1}{legalize_ustatus}{\sailRISCVfnlegalizzeUstatus}{}%
  \ifstrequal{#1}{legalize\_ustatus}{\sailRISCVfnlegalizzeUstatus}{}%
  \ifstrequal{#1}{legalize_xepc}{\sailRISCVfnlegalizzeXepc}{}%
  \ifstrequal{#1}{legalize\_xepc}{\sailRISCVfnlegalizzeXepc}{}%
  \ifstrequal{#1}{lift_sie}{\sailRISCVfnliftSie}{}%
  \ifstrequal{#1}{lift\_sie}{\sailRISCVfnliftSie}{}%
  \ifstrequal{#1}{lift_sip}{\sailRISCVfnliftSip}{}%
  \ifstrequal{#1}{lift\_sip}{\sailRISCVfnliftSip}{}%
  \ifstrequal{#1}{lift_sstatus}{\sailRISCVfnliftSstatus}{}%
  \ifstrequal{#1}{lift\_sstatus}{\sailRISCVfnliftSstatus}{}%
  \ifstrequal{#1}{lift_uie}{\sailRISCVfnliftUie}{}%
  \ifstrequal{#1}{lift\_uie}{\sailRISCVfnliftUie}{}%
  \ifstrequal{#1}{lift_uip}{\sailRISCVfnliftUip}{}%
  \ifstrequal{#1}{lift\_uip}{\sailRISCVfnliftUip}{}%
  \ifstrequal{#1}{lift_ustatus}{\sailRISCVfnliftUstatus}{}%
  \ifstrequal{#1}{lift\_ustatus}{\sailRISCVfnliftUstatus}{}%
  \ifstrequal{#1}{lookup_TLB39}{\sailRISCVfnlookupTLBThreeNine}{}%
  \ifstrequal{#1}{lookup\_TLB39}{\sailRISCVfnlookupTLBThreeNine}{}%
  \ifstrequal{#1}{lookup_TLB48}{\sailRISCVfnlookupTLBFourEight}{}%
  \ifstrequal{#1}{lookup\_TLB48}{\sailRISCVfnlookupTLBFourEight}{}%
  \ifstrequal{#1}{loop}{\sailRISCVfnloop}{}%
  \ifstrequal{#1}{lower_mie}{\sailRISCVfnlowerMie}{}%
  \ifstrequal{#1}{lower\_mie}{\sailRISCVfnlowerMie}{}%
  \ifstrequal{#1}{lower_mip}{\sailRISCVfnlowerMip}{}%
  \ifstrequal{#1}{lower\_mip}{\sailRISCVfnlowerMip}{}%
  \ifstrequal{#1}{lower_mstatus}{\sailRISCVfnlowerMstatus}{}%
  \ifstrequal{#1}{lower\_mstatus}{\sailRISCVfnlowerMstatus}{}%
  \ifstrequal{#1}{lower_sie}{\sailRISCVfnlowerSie}{}%
  \ifstrequal{#1}{lower\_sie}{\sailRISCVfnlowerSie}{}%
  \ifstrequal{#1}{lower_sip}{\sailRISCVfnlowerSip}{}%
  \ifstrequal{#1}{lower\_sip}{\sailRISCVfnlowerSip}{}%
  \ifstrequal{#1}{lower_sstatus}{\sailRISCVfnlowerSstatus}{}%
  \ifstrequal{#1}{lower\_sstatus}{\sailRISCVfnlowerSstatus}{}%
  \ifstrequal{#1}{lrsc_width_str}{\sailRISCVfnlrscWidthStr}{}%
  \ifstrequal{#1}{lrsc\_width\_str}{\sailRISCVfnlrscWidthStr}{}%
  \ifstrequal{#1}{make_TLB_Entry}{\sailRISCVfnmakeTLBEntry}{}%
  \ifstrequal{#1}{make\_TLB\_Entry}{\sailRISCVfnmakeTLBEntry}{}%
  \ifstrequal{#1}{match_TLB_Entry}{\sailRISCVfnmatchTLBEntry}{}%
  \ifstrequal{#1}{match\_TLB\_Entry}{\sailRISCVfnmatchTLBEntry}{}%
  \ifstrequal{#1}{memBitsToCapability}{\sailRISCVfnmemBitsToCapability}{}%
  \ifstrequal{#1}{mem_read}{\sailRISCVfnmemRead}{}%
  \ifstrequal{#1}{mem\_read}{\sailRISCVfnmemRead}{}%
  \ifstrequal{#1}{mem_read_cap}{\sailRISCVfnmemReadCap}{}%
  \ifstrequal{#1}{mem\_read\_cap}{\sailRISCVfnmemReadCap}{}%
  \ifstrequal{#1}{mem_read_meta}{\sailRISCVfnmemReadMeta}{}%
  \ifstrequal{#1}{mem\_read\_meta}{\sailRISCVfnmemReadMeta}{}%
  \ifstrequal{#1}{mem_read_priv}{\sailRISCVfnmemReadPriv}{}%
  \ifstrequal{#1}{mem\_read\_priv}{\sailRISCVfnmemReadPriv}{}%
  \ifstrequal{#1}{mem_read_priv_meta}{\sailRISCVfnmemReadPrivMeta}{}%
  \ifstrequal{#1}{mem\_read\_priv\_meta}{\sailRISCVfnmemReadPrivMeta}{}%
  \ifstrequal{#1}{mem_write_cap}{\sailRISCVfnmemWriteCap}{}%
  \ifstrequal{#1}{mem\_write\_cap}{\sailRISCVfnmemWriteCap}{}%
  \ifstrequal{#1}{mem_write_ea}{\sailRISCVfnmemWriteEa}{}%
  \ifstrequal{#1}{mem\_write\_ea}{\sailRISCVfnmemWriteEa}{}%
  \ifstrequal{#1}{mem_write_ea_cap}{\sailRISCVfnmemWriteEaCap}{}%
  \ifstrequal{#1}{mem\_write\_ea\_cap}{\sailRISCVfnmemWriteEaCap}{}%
  \ifstrequal{#1}{mem_write_value}{\sailRISCVfnmemWriteValue}{}%
  \ifstrequal{#1}{mem\_write\_value}{\sailRISCVfnmemWriteValue}{}%
  \ifstrequal{#1}{mem_write_value_meta}{\sailRISCVfnmemWriteValueMeta}{}%
  \ifstrequal{#1}{mem\_write\_value\_meta}{\sailRISCVfnmemWriteValueMeta}{}%
  \ifstrequal{#1}{mem_write_value_priv}{\sailRISCVfnmemWriteValuePriv}{}%
  \ifstrequal{#1}{mem\_write\_value\_priv}{\sailRISCVfnmemWriteValuePriv}{}%
  \ifstrequal{#1}{mem_write_value_priv_meta}{\sailRISCVfnmemWriteValuePrivMeta}{}%
  \ifstrequal{#1}{mem\_write\_value\_priv\_meta}{\sailRISCVfnmemWriteValuePrivMeta}{}%
  \ifstrequal{#1}{min_instruction_bytes}{\sailRISCVfnminInstructionBytes}{}%
  \ifstrequal{#1}{min\_instruction\_bytes}{\sailRISCVfnminInstructionBytes}{}%
  \ifstrequal{#1}{mmio_read}{\sailRISCVfnmmioRead}{}%
  \ifstrequal{#1}{mmio\_read}{\sailRISCVfnmmioRead}{}%
  \ifstrequal{#1}{mmio_write}{\sailRISCVfnmmioWrite}{}%
  \ifstrequal{#1}{mmio\_write}{\sailRISCVfnmmioWrite}{}%
  \ifstrequal{#1}{n_leading_spaces}{\sailRISCVfnnLeadingSpaces}{}%
  \ifstrequal{#1}{n\_leading\_spaces}{\sailRISCVfnnLeadingSpaces}{}%
  \ifstrequal{#1}{nan_box_H}{\sailRISCVfnnanBoxH}{}%
  \ifstrequal{#1}{nan\_box\_H}{\sailRISCVfnnanBoxH}{}%
  \ifstrequal{#1}{nan_box_S}{\sailRISCVfnnanBoxS}{}%
  \ifstrequal{#1}{nan\_box\_S}{\sailRISCVfnnanBoxS}{}%
  \ifstrequal{#1}{nan_unbox_H}{\sailRISCVfnnanUnboxH}{}%
  \ifstrequal{#1}{nan\_unbox\_H}{\sailRISCVfnnanUnboxH}{}%
  \ifstrequal{#1}{nan_unbox_S}{\sailRISCVfnnanUnboxS}{}%
  \ifstrequal{#1}{nan\_unbox\_S}{\sailRISCVfnnanUnboxS}{}%
  \ifstrequal{#1}{negate_D}{\sailRISCVfnnegateD}{}%
  \ifstrequal{#1}{negate\_D}{\sailRISCVfnnegateD}{}%
  \ifstrequal{#1}{negate_S}{\sailRISCVfnnegateS}{}%
  \ifstrequal{#1}{negate\_S}{\sailRISCVfnnegateS}{}%
  \ifstrequal{#1}{neq_anything}{\sailRISCVfnneqAnything}{}%
  \ifstrequal{#1}{neq\_anything}{\sailRISCVfnneqAnything}{}%
  \ifstrequal{#1}{neq_bits}{\sailRISCVfnneqBits}{}%
  \ifstrequal{#1}{neq\_bits}{\sailRISCVfnneqBits}{}%
  \ifstrequal{#1}{neq_bool}{\sailRISCVfnneqBool}{}%
  \ifstrequal{#1}{neq\_bool}{\sailRISCVfnneqBool}{}%
  \ifstrequal{#1}{neq_int}{\sailRISCVfnneqInt}{}%
  \ifstrequal{#1}{neq\_int}{\sailRISCVfnneqInt}{}%
  \ifstrequal{#1}{neq_vec}{\sailRISCVfnneqVec}{}%
  \ifstrequal{#1}{neq\_vec}{\sailRISCVfnneqVec}{}%
  \ifstrequal{#1}{not_bit}{\sailRISCVfnnotBit}{}%
  \ifstrequal{#1}{not\_bit}{\sailRISCVfnnotBit}{}%
  \ifstrequal{#1}{not_implemented}{\sailRISCVfnnotImplemented}{}%
  \ifstrequal{#1}{not\_implemented}{\sailRISCVfnnotImplemented}{}%
  \ifstrequal{#1}{num_of_Architecture}{\sailRISCVfnnumOfArchitecture}{}%
  \ifstrequal{#1}{num\_of\_Architecture}{\sailRISCVfnnumOfArchitecture}{}%
  \ifstrequal{#1}{num_of_CPtrCmpOp}{\sailRISCVfnnumOfCPtrCmpOp}{}%
  \ifstrequal{#1}{num\_of\_CPtrCmpOp}{\sailRISCVfnnumOfCPtrCmpOp}{}%
  \ifstrequal{#1}{num_of_CapEx}{\sailRISCVfnnumOfCapEx}{}%
  \ifstrequal{#1}{num\_of\_CapEx}{\sailRISCVfnnumOfCapEx}{}%
  \ifstrequal{#1}{num_of_ClearRegSet}{\sailRISCVfnnumOfClearRegSet}{}%
  \ifstrequal{#1}{num\_of\_ClearRegSet}{\sailRISCVfnnumOfClearRegSet}{}%
  \ifstrequal{#1}{num_of_ExceptionType}{\sailRISCVfnnumOfExceptionType}{}%
  \ifstrequal{#1}{num\_of\_ExceptionType}{\sailRISCVfnnumOfExceptionType}{}%
  \ifstrequal{#1}{num_of_ExtStatus}{\sailRISCVfnnumOfExtStatus}{}%
  \ifstrequal{#1}{num\_of\_ExtStatus}{\sailRISCVfnnumOfExtStatus}{}%
  \ifstrequal{#1}{num_of_InterruptType}{\sailRISCVfnnumOfInterruptType}{}%
  \ifstrequal{#1}{num\_of\_InterruptType}{\sailRISCVfnnumOfInterruptType}{}%
  \ifstrequal{#1}{num_of_PmpAddrMatchType}{\sailRISCVfnnumOfPmpAddrMatchType}{}%
  \ifstrequal{#1}{num\_of\_PmpAddrMatchType}{\sailRISCVfnnumOfPmpAddrMatchType}{}%
  \ifstrequal{#1}{num_of_Privilege}{\sailRISCVfnnumOfPrivilege}{}%
  \ifstrequal{#1}{num\_of\_Privilege}{\sailRISCVfnnumOfPrivilege}{}%
  \ifstrequal{#1}{num_of_Retired}{\sailRISCVfnnumOfRetired}{}%
  \ifstrequal{#1}{num\_of\_Retired}{\sailRISCVfnnumOfRetired}{}%
  \ifstrequal{#1}{num_of_SATPMode}{\sailRISCVfnnumOfSATPMode}{}%
  \ifstrequal{#1}{num\_of\_SATPMode}{\sailRISCVfnnumOfSATPMode}{}%
  \ifstrequal{#1}{num_of_TrapVectorMode}{\sailRISCVfnnumOfTrapVectorMode}{}%
  \ifstrequal{#1}{num\_of\_TrapVectorMode}{\sailRISCVfnnumOfTrapVectorMode}{}%
  \ifstrequal{#1}{num_of_a64_barrier_domain}{\sailRISCVfnnumOfASixFourBarrierDomain}{}%
  \ifstrequal{#1}{num\_of\_a64\_barrier\_domain}{\sailRISCVfnnumOfASixFourBarrierDomain}{}%
  \ifstrequal{#1}{num_of_a64_barrier_type}{\sailRISCVfnnumOfASixFourBarrierType}{}%
  \ifstrequal{#1}{num\_of\_a64\_barrier\_type}{\sailRISCVfnnumOfASixFourBarrierType}{}%
  \ifstrequal{#1}{num_of_amoop}{\sailRISCVfnnumOfAmoop}{}%
  \ifstrequal{#1}{num\_of\_amoop}{\sailRISCVfnnumOfAmoop}{}%
  \ifstrequal{#1}{num_of_biop_zbs}{\sailRISCVfnnumOfBiopZbs}{}%
  \ifstrequal{#1}{num\_of\_biop\_zbs}{\sailRISCVfnnumOfBiopZbs}{}%
  \ifstrequal{#1}{num_of_bop}{\sailRISCVfnnumOfBop}{}%
  \ifstrequal{#1}{num\_of\_bop}{\sailRISCVfnnumOfBop}{}%
  \ifstrequal{#1}{num_of_brop_zba}{\sailRISCVfnnumOfBropZba}{}%
  \ifstrequal{#1}{num\_of\_brop\_zba}{\sailRISCVfnnumOfBropZba}{}%
  \ifstrequal{#1}{num_of_brop_zbb}{\sailRISCVfnnumOfBropZbb}{}%
  \ifstrequal{#1}{num\_of\_brop\_zbb}{\sailRISCVfnnumOfBropZbb}{}%
  \ifstrequal{#1}{num_of_brop_zbkb}{\sailRISCVfnnumOfBropZbkb}{}%
  \ifstrequal{#1}{num\_of\_brop\_zbkb}{\sailRISCVfnnumOfBropZbkb}{}%
  \ifstrequal{#1}{num_of_brop_zbs}{\sailRISCVfnnumOfBropZbs}{}%
  \ifstrequal{#1}{num\_of\_brop\_zbs}{\sailRISCVfnnumOfBropZbs}{}%
  \ifstrequal{#1}{num_of_bropw_zba}{\sailRISCVfnnumOfBropwZba}{}%
  \ifstrequal{#1}{num\_of\_bropw\_zba}{\sailRISCVfnnumOfBropwZba}{}%
  \ifstrequal{#1}{num_of_bropw_zbb}{\sailRISCVfnnumOfBropwZbb}{}%
  \ifstrequal{#1}{num\_of\_bropw\_zbb}{\sailRISCVfnnumOfBropwZbb}{}%
  \ifstrequal{#1}{num_of_cache_op_kind}{\sailRISCVfnnumOfCacheOpKind}{}%
  \ifstrequal{#1}{num\_of\_cache\_op\_kind}{\sailRISCVfnnumOfCacheOpKind}{}%
  \ifstrequal{#1}{num_of_csrop}{\sailRISCVfnnumOfCsrop}{}%
  \ifstrequal{#1}{num\_of\_csrop}{\sailRISCVfnnumOfCsrop}{}%
  \ifstrequal{#1}{num_of_ext_access_type}{\sailRISCVfnnumOfExtAccessType}{}%
  \ifstrequal{#1}{num\_of\_ext\_access\_type}{\sailRISCVfnnumOfExtAccessType}{}%
  \ifstrequal{#1}{num_of_ext_exc_type}{\sailRISCVfnnumOfExtExcType}{}%
  \ifstrequal{#1}{num\_of\_ext\_exc\_type}{\sailRISCVfnnumOfExtExcType}{}%
  \ifstrequal{#1}{num_of_ext_ptw_error}{\sailRISCVfnnumOfExtPtwError}{}%
  \ifstrequal{#1}{num\_of\_ext\_ptw\_error}{\sailRISCVfnnumOfExtPtwError}{}%
  \ifstrequal{#1}{num_of_ext_ptw_fail}{\sailRISCVfnnumOfExtPtwFail}{}%
  \ifstrequal{#1}{num\_of\_ext\_ptw\_fail}{\sailRISCVfnnumOfExtPtwFail}{}%
  \ifstrequal{#1}{num_of_ext_ptw_lc}{\sailRISCVfnnumOfExtPtwLc}{}%
  \ifstrequal{#1}{num\_of\_ext\_ptw\_lc}{\sailRISCVfnnumOfExtPtwLc}{}%
  \ifstrequal{#1}{num_of_ext_ptw_sc}{\sailRISCVfnnumOfExtPtwSc}{}%
  \ifstrequal{#1}{num\_of\_ext\_ptw\_sc}{\sailRISCVfnnumOfExtPtwSc}{}%
  \ifstrequal{#1}{num_of_extop_zbb}{\sailRISCVfnnumOfExtopZbb}{}%
  \ifstrequal{#1}{num\_of\_extop\_zbb}{\sailRISCVfnnumOfExtopZbb}{}%
  \ifstrequal{#1}{num_of_f_bin_op_D}{\sailRISCVfnnumOfFBinOpD}{}%
  \ifstrequal{#1}{num\_of\_f\_bin\_op\_D}{\sailRISCVfnnumOfFBinOpD}{}%
  \ifstrequal{#1}{num_of_f_bin_op_H}{\sailRISCVfnnumOfFBinOpH}{}%
  \ifstrequal{#1}{num\_of\_f\_bin\_op\_H}{\sailRISCVfnnumOfFBinOpH}{}%
  \ifstrequal{#1}{num_of_f_bin_op_S}{\sailRISCVfnnumOfFBinOpS}{}%
  \ifstrequal{#1}{num\_of\_f\_bin\_op\_S}{\sailRISCVfnnumOfFBinOpS}{}%
  \ifstrequal{#1}{num_of_f_bin_rm_op_D}{\sailRISCVfnnumOfFBinRmOpD}{}%
  \ifstrequal{#1}{num\_of\_f\_bin\_rm\_op\_D}{\sailRISCVfnnumOfFBinRmOpD}{}%
  \ifstrequal{#1}{num_of_f_bin_rm_op_H}{\sailRISCVfnnumOfFBinRmOpH}{}%
  \ifstrequal{#1}{num\_of\_f\_bin\_rm\_op\_H}{\sailRISCVfnnumOfFBinRmOpH}{}%
  \ifstrequal{#1}{num_of_f_bin_rm_op_S}{\sailRISCVfnnumOfFBinRmOpS}{}%
  \ifstrequal{#1}{num\_of\_f\_bin\_rm\_op\_S}{\sailRISCVfnnumOfFBinRmOpS}{}%
  \ifstrequal{#1}{num_of_f_madd_op_D}{\sailRISCVfnnumOfFMaddOpD}{}%
  \ifstrequal{#1}{num\_of\_f\_madd\_op\_D}{\sailRISCVfnnumOfFMaddOpD}{}%
  \ifstrequal{#1}{num_of_f_madd_op_H}{\sailRISCVfnnumOfFMaddOpH}{}%
  \ifstrequal{#1}{num\_of\_f\_madd\_op\_H}{\sailRISCVfnnumOfFMaddOpH}{}%
  \ifstrequal{#1}{num_of_f_madd_op_S}{\sailRISCVfnnumOfFMaddOpS}{}%
  \ifstrequal{#1}{num\_of\_f\_madd\_op\_S}{\sailRISCVfnnumOfFMaddOpS}{}%
  \ifstrequal{#1}{num_of_f_un_op_D}{\sailRISCVfnnumOfFUnOpD}{}%
  \ifstrequal{#1}{num\_of\_f\_un\_op\_D}{\sailRISCVfnnumOfFUnOpD}{}%
  \ifstrequal{#1}{num_of_f_un_op_H}{\sailRISCVfnnumOfFUnOpH}{}%
  \ifstrequal{#1}{num\_of\_f\_un\_op\_H}{\sailRISCVfnnumOfFUnOpH}{}%
  \ifstrequal{#1}{num_of_f_un_op_S}{\sailRISCVfnnumOfFUnOpS}{}%
  \ifstrequal{#1}{num\_of\_f\_un\_op\_S}{\sailRISCVfnnumOfFUnOpS}{}%
  \ifstrequal{#1}{num_of_f_un_rm_op_D}{\sailRISCVfnnumOfFUnRmOpD}{}%
  \ifstrequal{#1}{num\_of\_f\_un\_rm\_op\_D}{\sailRISCVfnnumOfFUnRmOpD}{}%
  \ifstrequal{#1}{num_of_f_un_rm_op_H}{\sailRISCVfnnumOfFUnRmOpH}{}%
  \ifstrequal{#1}{num\_of\_f\_un\_rm\_op\_H}{\sailRISCVfnnumOfFUnRmOpH}{}%
  \ifstrequal{#1}{num_of_f_un_rm_op_S}{\sailRISCVfnnumOfFUnRmOpS}{}%
  \ifstrequal{#1}{num\_of\_f\_un\_rm\_op\_S}{\sailRISCVfnnumOfFUnRmOpS}{}%
  \ifstrequal{#1}{num_of_iop}{\sailRISCVfnnumOfIop}{}%
  \ifstrequal{#1}{num\_of\_iop}{\sailRISCVfnnumOfIop}{}%
  \ifstrequal{#1}{num_of_pmpAddrMatch}{\sailRISCVfnnumOfPmpAddrMatch}{}%
  \ifstrequal{#1}{num\_of\_pmpAddrMatch}{\sailRISCVfnnumOfPmpAddrMatch}{}%
  \ifstrequal{#1}{num_of_pmpMatch}{\sailRISCVfnnumOfPmpMatch}{}%
  \ifstrequal{#1}{num\_of\_pmpMatch}{\sailRISCVfnnumOfPmpMatch}{}%
  \ifstrequal{#1}{num_of_read_kind}{\sailRISCVfnnumOfReadKind}{}%
  \ifstrequal{#1}{num\_of\_read\_kind}{\sailRISCVfnnumOfReadKind}{}%
  \ifstrequal{#1}{num_of_rop}{\sailRISCVfnnumOfRop}{}%
  \ifstrequal{#1}{num\_of\_rop}{\sailRISCVfnnumOfRop}{}%
  \ifstrequal{#1}{num_of_ropw}{\sailRISCVfnnumOfRopw}{}%
  \ifstrequal{#1}{num\_of\_ropw}{\sailRISCVfnnumOfRopw}{}%
  \ifstrequal{#1}{num_of_rounding_mode}{\sailRISCVfnnumOfRoundingMode}{}%
  \ifstrequal{#1}{num\_of\_rounding\_mode}{\sailRISCVfnnumOfRoundingMode}{}%
  \ifstrequal{#1}{num_of_seed_opst}{\sailRISCVfnnumOfSeedOpst}{}%
  \ifstrequal{#1}{num\_of\_seed\_opst}{\sailRISCVfnnumOfSeedOpst}{}%
  \ifstrequal{#1}{num_of_sop}{\sailRISCVfnnumOfSop}{}%
  \ifstrequal{#1}{num\_of\_sop}{\sailRISCVfnnumOfSop}{}%
  \ifstrequal{#1}{num_of_sopw}{\sailRISCVfnnumOfSopw}{}%
  \ifstrequal{#1}{num\_of\_sopw}{\sailRISCVfnnumOfSopw}{}%
  \ifstrequal{#1}{num_of_trans_kind}{\sailRISCVfnnumOfTransKind}{}%
  \ifstrequal{#1}{num\_of\_trans\_kind}{\sailRISCVfnnumOfTransKind}{}%
  \ifstrequal{#1}{num_of_uop}{\sailRISCVfnnumOfUop}{}%
  \ifstrequal{#1}{num\_of\_uop}{\sailRISCVfnnumOfUop}{}%
  \ifstrequal{#1}{num_of_word_width}{\sailRISCVfnnumOfWordWidth}{}%
  \ifstrequal{#1}{num\_of\_word\_width}{\sailRISCVfnnumOfWordWidth}{}%
  \ifstrequal{#1}{num_of_write_kind}{\sailRISCVfnnumOfWriteKind}{}%
  \ifstrequal{#1}{num\_of\_write\_kind}{\sailRISCVfnnumOfWriteKind}{}%
  \ifstrequal{#1}{nvFlag}{\sailRISCVfnnvFlag}{}%
  \ifstrequal{#1}{nxFlag}{\sailRISCVfnnxFlag}{}%
  \ifstrequal{#1}{ofFlag}{\sailRISCVfnofFlag}{}%
  \ifstrequal{#1}{ones}{\sailRISCVfnones}{}%
  \ifstrequal{#1}{opt_spc_backwards}{\sailRISCVfnoptSpcBackwards}{}%
  \ifstrequal{#1}{opt\_spc\_backwards}{\sailRISCVfnoptSpcBackwards}{}%
  \ifstrequal{#1}{opt_spc_forwards}{\sailRISCVfnoptSpcForwards}{}%
  \ifstrequal{#1}{opt\_spc\_forwards}{\sailRISCVfnoptSpcForwards}{}%
  \ifstrequal{#1}{opt_spc_matches_prefix}{\sailRISCVfnoptSpcMatchesPrefix}{}%
  \ifstrequal{#1}{opt\_spc\_matches\_prefix}{\sailRISCVfnoptSpcMatchesPrefix}{}%
  \ifstrequal{#1}{pc_alignment_mask}{\sailRISCVfnpcAlignmentMask}{}%
  \ifstrequal{#1}{pc\_alignment\_mask}{\sailRISCVfnpcAlignmentMask}{}%
  \ifstrequal{#1}{pcc_access_system_regs}{\sailRISCVfnpccAccessSystemRegs}{}%
  \ifstrequal{#1}{pcc\_access\_system\_regs}{\sailRISCVfnpccAccessSystemRegs}{}%
  \ifstrequal{#1}{phys_mem_read}{\sailRISCVfnphysMemRead}{}%
  \ifstrequal{#1}{phys\_mem\_read}{\sailRISCVfnphysMemRead}{}%
  \ifstrequal{#1}{phys_mem_segments}{\sailRISCVfnphysMemSegments}{}%
  \ifstrequal{#1}{phys\_mem\_segments}{\sailRISCVfnphysMemSegments}{}%
  \ifstrequal{#1}{phys_mem_write}{\sailRISCVfnphysMemWrite}{}%
  \ifstrequal{#1}{phys\_mem\_write}{\sailRISCVfnphysMemWrite}{}%
  \ifstrequal{#1}{plat_htif_tohost}{\sailRISCVfnplatHtifTohost}{}%
  \ifstrequal{#1}{plat\_htif\_tohost}{\sailRISCVfnplatHtifTohost}{}%
  \ifstrequal{#1}{platform_wfi}{\sailRISCVfnplatformWfi}{}%
  \ifstrequal{#1}{platform\_wfi}{\sailRISCVfnplatformWfi}{}%
  \ifstrequal{#1}{pmpAddrMatchType_of_bits}{\sailRISCVfnpmpAddrMatchTypeOfBits}{}%
  \ifstrequal{#1}{pmpAddrMatchType\_of\_bits}{\sailRISCVfnpmpAddrMatchTypeOfBits}{}%
  \ifstrequal{#1}{pmpAddrMatchType_to_bits}{\sailRISCVfnpmpAddrMatchTypeToBits}{}%
  \ifstrequal{#1}{pmpAddrMatchType\_to\_bits}{\sailRISCVfnpmpAddrMatchTypeToBits}{}%
  \ifstrequal{#1}{pmpAddrMatch_of_num}{\sailRISCVfnpmpAddrMatchOfNum}{}%
  \ifstrequal{#1}{pmpAddrMatch\_of\_num}{\sailRISCVfnpmpAddrMatchOfNum}{}%
  \ifstrequal{#1}{pmpAddrRange}{\sailRISCVfnpmpAddrRangeA}{}%
  \ifstrequal{#1}{pmpCheck}{\sailRISCVfnpmpCheck}{}%
  \ifstrequal{#1}{pmpCheckPerms}{\sailRISCVfnpmpCheckPerms}{}%
  \ifstrequal{#1}{pmpCheckRWX}{\sailRISCVfnpmpCheckRWX}{}%
  \ifstrequal{#1}{pmpLocked}{\sailRISCVfnpmpLocked}{}%
  \ifstrequal{#1}{pmpMatchAddr}{\sailRISCVfnpmpMatchAddr}{}%
  \ifstrequal{#1}{pmpMatchEntry}{\sailRISCVfnpmpMatchEntry}{}%
  \ifstrequal{#1}{pmpMatch_of_num}{\sailRISCVfnpmpMatchOfNum}{}%
  \ifstrequal{#1}{pmpMatch\_of\_num}{\sailRISCVfnpmpMatchOfNum}{}%
  \ifstrequal{#1}{pmpReadCfgReg}{\sailRISCVfnpmpReadCfgReg}{}%
  \ifstrequal{#1}{pmpTORLocked}{\sailRISCVfnpmpTORLocked}{}%
  \ifstrequal{#1}{pmpWriteAddr}{\sailRISCVfnpmpWriteAddr}{}%
  \ifstrequal{#1}{pmpWriteCfg}{\sailRISCVfnpmpWriteCfg}{}%
  \ifstrequal{#1}{pmpWriteCfgReg}{\sailRISCVfnpmpWriteCfgReg}{}%
  \ifstrequal{#1}{pmp_mem_read}{\sailRISCVfnpmpMemRead}{}%
  \ifstrequal{#1}{pmp\_mem\_read}{\sailRISCVfnpmpMemRead}{}%
  \ifstrequal{#1}{pmp_mem_write}{\sailRISCVfnpmpMemWrite}{}%
  \ifstrequal{#1}{pmp\_mem\_write}{\sailRISCVfnpmpMemWrite}{}%
  \ifstrequal{#1}{prepare_trap_vector}{\sailRISCVfnprepareTrapVector}{}%
  \ifstrequal{#1}{prepare\_trap\_vector}{\sailRISCVfnprepareTrapVector}{}%
  \ifstrequal{#1}{prepare_xret_target}{\sailRISCVfnprepareXretTarget}{}%
  \ifstrequal{#1}{prepare\_xret\_target}{\sailRISCVfnprepareXretTarget}{}%
  \ifstrequal{#1}{print_insn}{\sailRISCVfnprintInsn}{}%
  \ifstrequal{#1}{print\_insn}{\sailRISCVfnprintInsn}{}%
  \ifstrequal{#1}{privLevel_of_bits}{\sailRISCVfnprivLevelOfBits}{}%
  \ifstrequal{#1}{privLevel\_of\_bits}{\sailRISCVfnprivLevelOfBits}{}%
  \ifstrequal{#1}{privLevel_to_bits}{\sailRISCVfnprivLevelToBits}{}%
  \ifstrequal{#1}{privLevel\_to\_bits}{\sailRISCVfnprivLevelToBits}{}%
  \ifstrequal{#1}{privLevel_to_str}{\sailRISCVfnprivLevelToStr}{}%
  \ifstrequal{#1}{privLevel\_to\_str}{\sailRISCVfnprivLevelToStr}{}%
  \ifstrequal{#1}{processPending}{\sailRISCVfnprocessPending}{}%
  \ifstrequal{#1}{process_fload16}{\sailRISCVfnprocessFloadOneSix}{}%
  \ifstrequal{#1}{process\_fload16}{\sailRISCVfnprocessFloadOneSix}{}%
  \ifstrequal{#1}{process_fload32}{\sailRISCVfnprocessFloadThreeTwo}{}%
  \ifstrequal{#1}{process\_fload32}{\sailRISCVfnprocessFloadThreeTwo}{}%
  \ifstrequal{#1}{process_fload64}{\sailRISCVfnprocessFloadSixFour}{}%
  \ifstrequal{#1}{process\_fload64}{\sailRISCVfnprocessFloadSixFour}{}%
  \ifstrequal{#1}{process_fstore}{\sailRISCVfnprocessFstore}{}%
  \ifstrequal{#1}{process\_fstore}{\sailRISCVfnprocessFstore}{}%
  \ifstrequal{#1}{process_load}{\sailRISCVfnprocessLoad}{}%
  \ifstrequal{#1}{process\_load}{\sailRISCVfnprocessLoad}{}%
  \ifstrequal{#1}{process_loadres}{\sailRISCVfnprocessLoadres}{}%
  \ifstrequal{#1}{process\_loadres}{\sailRISCVfnprocessLoadres}{}%
  \ifstrequal{#1}{ptw_error_to_str}{\sailRISCVfnptwErrorToStr}{}%
  \ifstrequal{#1}{ptw\_error\_to\_str}{\sailRISCVfnptwErrorToStr}{}%
  \ifstrequal{#1}{rC}{\sailRISCVfnrC}{}%
  \ifstrequal{#1}{rC_bits}{\sailRISCVfnrCBits}{}%
  \ifstrequal{#1}{rC\_bits}{\sailRISCVfnrCBits}{}%
  \ifstrequal{#1}{rF}{\sailRISCVfnrF}{}%
  \ifstrequal{#1}{rF_bits}{\sailRISCVfnrFBits}{}%
  \ifstrequal{#1}{rF\_bits}{\sailRISCVfnrFBits}{}%
  \ifstrequal{#1}{rF_or_X_D}{\sailRISCVfnrFOrXD}{}%
  \ifstrequal{#1}{rF\_or\_X\_D}{\sailRISCVfnrFOrXD}{}%
  \ifstrequal{#1}{rF_or_X_H}{\sailRISCVfnrFOrXH}{}%
  \ifstrequal{#1}{rF\_or\_X\_H}{\sailRISCVfnrFOrXH}{}%
  \ifstrequal{#1}{rF_or_X_S}{\sailRISCVfnrFOrXS}{}%
  \ifstrequal{#1}{rF\_or\_X\_S}{\sailRISCVfnrFOrXS}{}%
  \ifstrequal{#1}{rX}{\sailRISCVfnrX}{}%
  \ifstrequal{#1}{rX_bits}{\sailRISCVfnrXBits}{}%
  \ifstrequal{#1}{rX\_bits}{\sailRISCVfnrXBits}{}%
  \ifstrequal{#1}{readCSR}{\sailRISCVfnreadCSR}{}%
  \ifstrequal{#1}{read_kind_of_flags}{\sailRISCVfnreadKindOfFlags}{}%
  \ifstrequal{#1}{read\_kind\_of\_flags}{\sailRISCVfnreadKindOfFlags}{}%
  \ifstrequal{#1}{read_kind_of_num}{\sailRISCVfnreadKindOfNum}{}%
  \ifstrequal{#1}{read\_kind\_of\_num}{\sailRISCVfnreadKindOfNum}{}%
  \ifstrequal{#1}{read_ram}{\sailRISCVfnreadRam}{}%
  \ifstrequal{#1}{read\_ram}{\sailRISCVfnreadRam}{}%
  \ifstrequal{#1}{read_seed_csr}{\sailRISCVfnreadSeedCsr}{}%
  \ifstrequal{#1}{read\_seed\_csr}{\sailRISCVfnreadSeedCsr}{}%
  \ifstrequal{#1}{reg_name_abi}{\sailRISCVfnregNameAbi}{}%
  \ifstrequal{#1}{reg\_name\_abi}{\sailRISCVfnregNameAbi}{}%
  \ifstrequal{#1}{regidx_to_regno}{\sailRISCVfnregidxToRegno}{}%
  \ifstrequal{#1}{regidx\_to\_regno}{\sailRISCVfnregidxToRegno}{}%
  \ifstrequal{#1}{regval_from_reg}{\sailRISCVfnregvalFromReg}{}%
  \ifstrequal{#1}{regval\_from\_reg}{\sailRISCVfnregvalFromReg}{}%
  \ifstrequal{#1}{regval_into_reg}{\sailRISCVfnregvalIntoReg}{}%
  \ifstrequal{#1}{regval\_into\_reg}{\sailRISCVfnregvalIntoReg}{}%
  \ifstrequal{#1}{reset_htif}{\sailRISCVfnresetHtif}{}%
  \ifstrequal{#1}{reset\_htif}{\sailRISCVfnresetHtif}{}%
  \ifstrequal{#1}{retire_instruction}{\sailRISCVfnretireInstruction}{}%
  \ifstrequal{#1}{retire\_instruction}{\sailRISCVfnretireInstruction}{}%
  \ifstrequal{#1}{reverse_bits_in_byte}{\sailRISCVfnreverseBitsInByte}{}%
  \ifstrequal{#1}{reverse\_bits\_in\_byte}{\sailRISCVfnreverseBitsInByte}{}%
  \ifstrequal{#1}{riscv_f16Add}{\sailRISCVfnriscvFOneSixAdd}{}%
  \ifstrequal{#1}{riscv\_f16Add}{\sailRISCVfnriscvFOneSixAdd}{}%
  \ifstrequal{#1}{riscv_f16Div}{\sailRISCVfnriscvFOneSixDiv}{}%
  \ifstrequal{#1}{riscv\_f16Div}{\sailRISCVfnriscvFOneSixDiv}{}%
  \ifstrequal{#1}{riscv_f16Eq}{\sailRISCVfnriscvFOneSixEq}{}%
  \ifstrequal{#1}{riscv\_f16Eq}{\sailRISCVfnriscvFOneSixEq}{}%
  \ifstrequal{#1}{riscv_f16Le}{\sailRISCVfnriscvFOneSixLe}{}%
  \ifstrequal{#1}{riscv\_f16Le}{\sailRISCVfnriscvFOneSixLe}{}%
  \ifstrequal{#1}{riscv_f16Lt}{\sailRISCVfnriscvFOneSixLt}{}%
  \ifstrequal{#1}{riscv\_f16Lt}{\sailRISCVfnriscvFOneSixLt}{}%
  \ifstrequal{#1}{riscv_f16Mul}{\sailRISCVfnriscvFOneSixMul}{}%
  \ifstrequal{#1}{riscv\_f16Mul}{\sailRISCVfnriscvFOneSixMul}{}%
  \ifstrequal{#1}{riscv_f16MulAdd}{\sailRISCVfnriscvFOneSixMulAdd}{}%
  \ifstrequal{#1}{riscv\_f16MulAdd}{\sailRISCVfnriscvFOneSixMulAdd}{}%
  \ifstrequal{#1}{riscv_f16Sqrt}{\sailRISCVfnriscvFOneSixSqrt}{}%
  \ifstrequal{#1}{riscv\_f16Sqrt}{\sailRISCVfnriscvFOneSixSqrt}{}%
  \ifstrequal{#1}{riscv_f16Sub}{\sailRISCVfnriscvFOneSixSub}{}%
  \ifstrequal{#1}{riscv\_f16Sub}{\sailRISCVfnriscvFOneSixSub}{}%
  \ifstrequal{#1}{riscv_f16ToF32}{\sailRISCVfnriscvFOneSixToFThreeTwo}{}%
  \ifstrequal{#1}{riscv\_f16ToF32}{\sailRISCVfnriscvFOneSixToFThreeTwo}{}%
  \ifstrequal{#1}{riscv_f16ToF64}{\sailRISCVfnriscvFOneSixToFSixFour}{}%
  \ifstrequal{#1}{riscv\_f16ToF64}{\sailRISCVfnriscvFOneSixToFSixFour}{}%
  \ifstrequal{#1}{riscv_f16ToI32}{\sailRISCVfnriscvFOneSixToIThreeTwo}{}%
  \ifstrequal{#1}{riscv\_f16ToI32}{\sailRISCVfnriscvFOneSixToIThreeTwo}{}%
  \ifstrequal{#1}{riscv_f16ToI64}{\sailRISCVfnriscvFOneSixToISixFour}{}%
  \ifstrequal{#1}{riscv\_f16ToI64}{\sailRISCVfnriscvFOneSixToISixFour}{}%
  \ifstrequal{#1}{riscv_f16ToUi32}{\sailRISCVfnriscvFOneSixToUiThreeTwo}{}%
  \ifstrequal{#1}{riscv\_f16ToUi32}{\sailRISCVfnriscvFOneSixToUiThreeTwo}{}%
  \ifstrequal{#1}{riscv_f16ToUi64}{\sailRISCVfnriscvFOneSixToUiSixFour}{}%
  \ifstrequal{#1}{riscv\_f16ToUi64}{\sailRISCVfnriscvFOneSixToUiSixFour}{}%
  \ifstrequal{#1}{riscv_f32Add}{\sailRISCVfnriscvFThreeTwoAdd}{}%
  \ifstrequal{#1}{riscv\_f32Add}{\sailRISCVfnriscvFThreeTwoAdd}{}%
  \ifstrequal{#1}{riscv_f32Div}{\sailRISCVfnriscvFThreeTwoDiv}{}%
  \ifstrequal{#1}{riscv\_f32Div}{\sailRISCVfnriscvFThreeTwoDiv}{}%
  \ifstrequal{#1}{riscv_f32Eq}{\sailRISCVfnriscvFThreeTwoEq}{}%
  \ifstrequal{#1}{riscv\_f32Eq}{\sailRISCVfnriscvFThreeTwoEq}{}%
  \ifstrequal{#1}{riscv_f32Le}{\sailRISCVfnriscvFThreeTwoLe}{}%
  \ifstrequal{#1}{riscv\_f32Le}{\sailRISCVfnriscvFThreeTwoLe}{}%
  \ifstrequal{#1}{riscv_f32Lt}{\sailRISCVfnriscvFThreeTwoLt}{}%
  \ifstrequal{#1}{riscv\_f32Lt}{\sailRISCVfnriscvFThreeTwoLt}{}%
  \ifstrequal{#1}{riscv_f32Mul}{\sailRISCVfnriscvFThreeTwoMul}{}%
  \ifstrequal{#1}{riscv\_f32Mul}{\sailRISCVfnriscvFThreeTwoMul}{}%
  \ifstrequal{#1}{riscv_f32MulAdd}{\sailRISCVfnriscvFThreeTwoMulAdd}{}%
  \ifstrequal{#1}{riscv\_f32MulAdd}{\sailRISCVfnriscvFThreeTwoMulAdd}{}%
  \ifstrequal{#1}{riscv_f32Sqrt}{\sailRISCVfnriscvFThreeTwoSqrt}{}%
  \ifstrequal{#1}{riscv\_f32Sqrt}{\sailRISCVfnriscvFThreeTwoSqrt}{}%
  \ifstrequal{#1}{riscv_f32Sub}{\sailRISCVfnriscvFThreeTwoSub}{}%
  \ifstrequal{#1}{riscv\_f32Sub}{\sailRISCVfnriscvFThreeTwoSub}{}%
  \ifstrequal{#1}{riscv_f32ToF16}{\sailRISCVfnriscvFThreeTwoToFOneSix}{}%
  \ifstrequal{#1}{riscv\_f32ToF16}{\sailRISCVfnriscvFThreeTwoToFOneSix}{}%
  \ifstrequal{#1}{riscv_f32ToF64}{\sailRISCVfnriscvFThreeTwoToFSixFour}{}%
  \ifstrequal{#1}{riscv\_f32ToF64}{\sailRISCVfnriscvFThreeTwoToFSixFour}{}%
  \ifstrequal{#1}{riscv_f32ToI32}{\sailRISCVfnriscvFThreeTwoToIThreeTwo}{}%
  \ifstrequal{#1}{riscv\_f32ToI32}{\sailRISCVfnriscvFThreeTwoToIThreeTwo}{}%
  \ifstrequal{#1}{riscv_f32ToI64}{\sailRISCVfnriscvFThreeTwoToISixFour}{}%
  \ifstrequal{#1}{riscv\_f32ToI64}{\sailRISCVfnriscvFThreeTwoToISixFour}{}%
  \ifstrequal{#1}{riscv_f32ToUi32}{\sailRISCVfnriscvFThreeTwoToUiThreeTwo}{}%
  \ifstrequal{#1}{riscv\_f32ToUi32}{\sailRISCVfnriscvFThreeTwoToUiThreeTwo}{}%
  \ifstrequal{#1}{riscv_f32ToUi64}{\sailRISCVfnriscvFThreeTwoToUiSixFour}{}%
  \ifstrequal{#1}{riscv\_f32ToUi64}{\sailRISCVfnriscvFThreeTwoToUiSixFour}{}%
  \ifstrequal{#1}{riscv_f64Add}{\sailRISCVfnriscvFSixFourAdd}{}%
  \ifstrequal{#1}{riscv\_f64Add}{\sailRISCVfnriscvFSixFourAdd}{}%
  \ifstrequal{#1}{riscv_f64Div}{\sailRISCVfnriscvFSixFourDiv}{}%
  \ifstrequal{#1}{riscv\_f64Div}{\sailRISCVfnriscvFSixFourDiv}{}%
  \ifstrequal{#1}{riscv_f64Eq}{\sailRISCVfnriscvFSixFourEq}{}%
  \ifstrequal{#1}{riscv\_f64Eq}{\sailRISCVfnriscvFSixFourEq}{}%
  \ifstrequal{#1}{riscv_f64Le}{\sailRISCVfnriscvFSixFourLe}{}%
  \ifstrequal{#1}{riscv\_f64Le}{\sailRISCVfnriscvFSixFourLe}{}%
  \ifstrequal{#1}{riscv_f64Lt}{\sailRISCVfnriscvFSixFourLt}{}%
  \ifstrequal{#1}{riscv\_f64Lt}{\sailRISCVfnriscvFSixFourLt}{}%
  \ifstrequal{#1}{riscv_f64Mul}{\sailRISCVfnriscvFSixFourMul}{}%
  \ifstrequal{#1}{riscv\_f64Mul}{\sailRISCVfnriscvFSixFourMul}{}%
  \ifstrequal{#1}{riscv_f64MulAdd}{\sailRISCVfnriscvFSixFourMulAdd}{}%
  \ifstrequal{#1}{riscv\_f64MulAdd}{\sailRISCVfnriscvFSixFourMulAdd}{}%
  \ifstrequal{#1}{riscv_f64Sqrt}{\sailRISCVfnriscvFSixFourSqrt}{}%
  \ifstrequal{#1}{riscv\_f64Sqrt}{\sailRISCVfnriscvFSixFourSqrt}{}%
  \ifstrequal{#1}{riscv_f64Sub}{\sailRISCVfnriscvFSixFourSub}{}%
  \ifstrequal{#1}{riscv\_f64Sub}{\sailRISCVfnriscvFSixFourSub}{}%
  \ifstrequal{#1}{riscv_f64ToF16}{\sailRISCVfnriscvFSixFourToFOneSix}{}%
  \ifstrequal{#1}{riscv\_f64ToF16}{\sailRISCVfnriscvFSixFourToFOneSix}{}%
  \ifstrequal{#1}{riscv_f64ToF32}{\sailRISCVfnriscvFSixFourToFThreeTwo}{}%
  \ifstrequal{#1}{riscv\_f64ToF32}{\sailRISCVfnriscvFSixFourToFThreeTwo}{}%
  \ifstrequal{#1}{riscv_f64ToI32}{\sailRISCVfnriscvFSixFourToIThreeTwo}{}%
  \ifstrequal{#1}{riscv\_f64ToI32}{\sailRISCVfnriscvFSixFourToIThreeTwo}{}%
  \ifstrequal{#1}{riscv_f64ToI64}{\sailRISCVfnriscvFSixFourToISixFour}{}%
  \ifstrequal{#1}{riscv\_f64ToI64}{\sailRISCVfnriscvFSixFourToISixFour}{}%
  \ifstrequal{#1}{riscv_f64ToUi32}{\sailRISCVfnriscvFSixFourToUiThreeTwo}{}%
  \ifstrequal{#1}{riscv\_f64ToUi32}{\sailRISCVfnriscvFSixFourToUiThreeTwo}{}%
  \ifstrequal{#1}{riscv_f64ToUi64}{\sailRISCVfnriscvFSixFourToUiSixFour}{}%
  \ifstrequal{#1}{riscv\_f64ToUi64}{\sailRISCVfnriscvFSixFourToUiSixFour}{}%
  \ifstrequal{#1}{riscv_i32ToF16}{\sailRISCVfnriscvIThreeTwoToFOneSix}{}%
  \ifstrequal{#1}{riscv\_i32ToF16}{\sailRISCVfnriscvIThreeTwoToFOneSix}{}%
  \ifstrequal{#1}{riscv_i32ToF32}{\sailRISCVfnriscvIThreeTwoToFThreeTwo}{}%
  \ifstrequal{#1}{riscv\_i32ToF32}{\sailRISCVfnriscvIThreeTwoToFThreeTwo}{}%
  \ifstrequal{#1}{riscv_i32ToF64}{\sailRISCVfnriscvIThreeTwoToFSixFour}{}%
  \ifstrequal{#1}{riscv\_i32ToF64}{\sailRISCVfnriscvIThreeTwoToFSixFour}{}%
  \ifstrequal{#1}{riscv_i64ToF16}{\sailRISCVfnriscvISixFourToFOneSix}{}%
  \ifstrequal{#1}{riscv\_i64ToF16}{\sailRISCVfnriscvISixFourToFOneSix}{}%
  \ifstrequal{#1}{riscv_i64ToF32}{\sailRISCVfnriscvISixFourToFThreeTwo}{}%
  \ifstrequal{#1}{riscv\_i64ToF32}{\sailRISCVfnriscvISixFourToFThreeTwo}{}%
  \ifstrequal{#1}{riscv_i64ToF64}{\sailRISCVfnriscvISixFourToFSixFour}{}%
  \ifstrequal{#1}{riscv\_i64ToF64}{\sailRISCVfnriscvISixFourToFSixFour}{}%
  \ifstrequal{#1}{riscv_ui32ToF16}{\sailRISCVfnriscvUiThreeTwoToFOneSix}{}%
  \ifstrequal{#1}{riscv\_ui32ToF16}{\sailRISCVfnriscvUiThreeTwoToFOneSix}{}%
  \ifstrequal{#1}{riscv_ui32ToF32}{\sailRISCVfnriscvUiThreeTwoToFThreeTwo}{}%
  \ifstrequal{#1}{riscv\_ui32ToF32}{\sailRISCVfnriscvUiThreeTwoToFThreeTwo}{}%
  \ifstrequal{#1}{riscv_ui32ToF64}{\sailRISCVfnriscvUiThreeTwoToFSixFour}{}%
  \ifstrequal{#1}{riscv\_ui32ToF64}{\sailRISCVfnriscvUiThreeTwoToFSixFour}{}%
  \ifstrequal{#1}{riscv_ui64ToF16}{\sailRISCVfnriscvUiSixFourToFOneSix}{}%
  \ifstrequal{#1}{riscv\_ui64ToF16}{\sailRISCVfnriscvUiSixFourToFOneSix}{}%
  \ifstrequal{#1}{riscv_ui64ToF32}{\sailRISCVfnriscvUiSixFourToFThreeTwo}{}%
  \ifstrequal{#1}{riscv\_ui64ToF32}{\sailRISCVfnriscvUiSixFourToFThreeTwo}{}%
  \ifstrequal{#1}{riscv_ui64ToF64}{\sailRISCVfnriscvUiSixFourToFSixFour}{}%
  \ifstrequal{#1}{riscv\_ui64ToF64}{\sailRISCVfnriscvUiSixFourToFSixFour}{}%
  \ifstrequal{#1}{rop_of_num}{\sailRISCVfnropOfNum}{}%
  \ifstrequal{#1}{rop\_of\_num}{\sailRISCVfnropOfNum}{}%
  \ifstrequal{#1}{ropw_of_num}{\sailRISCVfnropwOfNum}{}%
  \ifstrequal{#1}{ropw\_of\_num}{\sailRISCVfnropwOfNum}{}%
  \ifstrequal{#1}{rotate_bits_left}{\sailRISCVfnrotateBitsLeft}{}%
  \ifstrequal{#1}{rotate\_bits\_left}{\sailRISCVfnrotateBitsLeft}{}%
  \ifstrequal{#1}{rotate_bits_right}{\sailRISCVfnrotateBitsRight}{}%
  \ifstrequal{#1}{rotate\_bits\_right}{\sailRISCVfnrotateBitsRight}{}%
  \ifstrequal{#1}{rotatel}{\sailRISCVfnrotatel}{}%
  \ifstrequal{#1}{rotater}{\sailRISCVfnrotater}{}%
  \ifstrequal{#1}{rounding_mode_of_num}{\sailRISCVfnroundingModeOfNum}{}%
  \ifstrequal{#1}{rounding\_mode\_of\_num}{\sailRISCVfnroundingModeOfNum}{}%
  \ifstrequal{#1}{rvfi_read}{\sailRISCVfnrvfiRead}{}%
  \ifstrequal{#1}{rvfi\_read}{\sailRISCVfnrvfiRead}{}%
  \ifstrequal{#1}{rvfi_trap}{\sailRISCVfnrvfiTrap}{}%
  \ifstrequal{#1}{rvfi\_trap}{\sailRISCVfnrvfiTrap}{}%
  \ifstrequal{#1}{rvfi_wX}{\sailRISCVfnrvfiWX}{}%
  \ifstrequal{#1}{rvfi\_wX}{\sailRISCVfnrvfiWX}{}%
  \ifstrequal{#1}{rvfi_write}{\sailRISCVfnrvfiWrite}{}%
  \ifstrequal{#1}{rvfi\_write}{\sailRISCVfnrvfiWrite}{}%
  \ifstrequal{#1}{sail_mask}{\sailRISCVfnsailMask}{}%
  \ifstrequal{#1}{sail\_mask}{\sailRISCVfnsailMask}{}%
  \ifstrequal{#1}{sail_ones}{\sailRISCVfnsailOnes}{}%
  \ifstrequal{#1}{sail\_ones}{\sailRISCVfnsailOnes}{}%
  \ifstrequal{#1}{satp64Mode_of_bits}{\sailRISCVfnsatpSixFourModeOfBits}{}%
  \ifstrequal{#1}{satp64Mode\_of\_bits}{\sailRISCVfnsatpSixFourModeOfBits}{}%
  \ifstrequal{#1}{sealCap}{\sailRISCVfnsealCap}{}%
  \ifstrequal{#1}{seed_opst_of_num}{\sailRISCVfnseedOpstOfNum}{}%
  \ifstrequal{#1}{seed\_opst\_of\_num}{\sailRISCVfnseedOpstOfNum}{}%
  \ifstrequal{#1}{select_instr_or_fcsr_rm}{\sailRISCVfnselectInstrOrFcsrRm}{}%
  \ifstrequal{#1}{select\_instr\_or\_fcsr\_rm}{\sailRISCVfnselectInstrOrFcsrRm}{}%
  \ifstrequal{#1}{setCapAddr}{\sailRISCVfnsetCapAddr}{}%
  \ifstrequal{#1}{setCapBounds}{\sailRISCVfnsetCapBounds}{}%
  \ifstrequal{#1}{setCapFlags}{\sailRISCVfnsetCapFlags}{}%
  \ifstrequal{#1}{setCapOffset}{\sailRISCVfnsetCapOffset}{}%
  \ifstrequal{#1}{setCapOffsetChecked}{\sailRISCVfnsetCapOffsetChecked}{}%
  \ifstrequal{#1}{setCapPerms}{\sailRISCVfnsetCapPerms}{}%
  \ifstrequal{#1}{set_mstatus_SXL}{\sailRISCVfnsetMstatusSXL}{}%
  \ifstrequal{#1}{set\_mstatus\_SXL}{\sailRISCVfnsetMstatusSXL}{}%
  \ifstrequal{#1}{set_mstatus_UXL}{\sailRISCVfnsetMstatusUXL}{}%
  \ifstrequal{#1}{set\_mstatus\_UXL}{\sailRISCVfnsetMstatusUXL}{}%
  \ifstrequal{#1}{set_mtvec}{\sailRISCVfnsetMtvec}{}%
  \ifstrequal{#1}{set\_mtvec}{\sailRISCVfnsetMtvec}{}%
  \ifstrequal{#1}{set_next_pc}{\sailRISCVfnsetNextPc}{}%
  \ifstrequal{#1}{set\_next\_pc}{\sailRISCVfnsetNextPc}{}%
  \ifstrequal{#1}{set_sstatus_UXL}{\sailRISCVfnsetSstatusUXL}{}%
  \ifstrequal{#1}{set\_sstatus\_UXL}{\sailRISCVfnsetSstatusUXL}{}%
  \ifstrequal{#1}{set_stvec}{\sailRISCVfnsetStvec}{}%
  \ifstrequal{#1}{set\_stvec}{\sailRISCVfnsetStvec}{}%
  \ifstrequal{#1}{set_utvec}{\sailRISCVfnsetUtvec}{}%
  \ifstrequal{#1}{set\_utvec}{\sailRISCVfnsetUtvec}{}%
  \ifstrequal{#1}{set_xret_target}{\sailRISCVfnsetXretTarget}{}%
  \ifstrequal{#1}{set\_xret\_target}{\sailRISCVfnsetXretTarget}{}%
  \ifstrequal{#1}{shift_right_arith32}{\sailRISCVfnshiftRightArithThreeTwo}{}%
  \ifstrequal{#1}{shift\_right\_arith32}{\sailRISCVfnshiftRightArithThreeTwo}{}%
  \ifstrequal{#1}{shift_right_arith64}{\sailRISCVfnshiftRightArithSixFour}{}%
  \ifstrequal{#1}{shift\_right\_arith64}{\sailRISCVfnshiftRightArithSixFour}{}%
  \ifstrequal{#1}{slice_mask}{\sailRISCVfnsliceMask}{}%
  \ifstrequal{#1}{slice\_mask}{\sailRISCVfnsliceMask}{}%
  \ifstrequal{#1}{sop_of_num}{\sailRISCVfnsopOfNum}{}%
  \ifstrequal{#1}{sop\_of\_num}{\sailRISCVfnsopOfNum}{}%
  \ifstrequal{#1}{sopw_of_num}{\sailRISCVfnsopwOfNum}{}%
  \ifstrequal{#1}{sopw\_of\_num}{\sailRISCVfnsopwOfNum}{}%
  \ifstrequal{#1}{spc_backwards}{\sailRISCVfnspcBackwards}{}%
  \ifstrequal{#1}{spc\_backwards}{\sailRISCVfnspcBackwards}{}%
  \ifstrequal{#1}{spc_forwards}{\sailRISCVfnspcForwards}{}%
  \ifstrequal{#1}{spc\_forwards}{\sailRISCVfnspcForwards}{}%
  \ifstrequal{#1}{spc_matches_prefix}{\sailRISCVfnspcMatchesPrefix}{}%
  \ifstrequal{#1}{spc\_matches\_prefix}{\sailRISCVfnspcMatchesPrefix}{}%
  \ifstrequal{#1}{step}{\sailRISCVfnstep}{}%
  \ifstrequal{#1}{string_of_bit}{\sailRISCVfnstringOfBit}{}%
  \ifstrequal{#1}{string\_of\_bit}{\sailRISCVfnstringOfBit}{}%
  \ifstrequal{#1}{string_of_capex}{\sailRISCVfnstringOfCapex}{}%
  \ifstrequal{#1}{string\_of\_capex}{\sailRISCVfnstringOfCapex}{}%
  \ifstrequal{#1}{tag_addr_to_addr}{\sailRISCVfntagAddrToAddr}{}%
  \ifstrequal{#1}{tag\_addr\_to\_addr}{\sailRISCVfntagAddrToAddr}{}%
  \ifstrequal{#1}{tick_clock}{\sailRISCVfntickClock}{}%
  \ifstrequal{#1}{tick\_clock}{\sailRISCVfntickClock}{}%
  \ifstrequal{#1}{tick_pc}{\sailRISCVfntickPc}{}%
  \ifstrequal{#1}{tick\_pc}{\sailRISCVfntickPc}{}%
  \ifstrequal{#1}{tick_platform}{\sailRISCVfntickPlatform}{}%
  \ifstrequal{#1}{tick\_platform}{\sailRISCVfntickPlatform}{}%
  \ifstrequal{#1}{to_bits}{\sailRISCVfntoBits}{}%
  \ifstrequal{#1}{to\_bits}{\sailRISCVfntoBits}{}%
  \ifstrequal{#1}{trans_kind_of_num}{\sailRISCVfntransKindOfNum}{}%
  \ifstrequal{#1}{trans\_kind\_of\_num}{\sailRISCVfntransKindOfNum}{}%
  \ifstrequal{#1}{translate39}{\sailRISCVfntranslateThreeNine}{}%
  \ifstrequal{#1}{translate48}{\sailRISCVfntranslateFourEight}{}%
  \ifstrequal{#1}{translateAddr}{\sailRISCVfntranslateAddr}{}%
  \ifstrequal{#1}{translateAddr_priv}{\sailRISCVfntranslateAddrPriv}{}%
  \ifstrequal{#1}{translateAddr\_priv}{\sailRISCVfntranslateAddrPriv}{}%
  \ifstrequal{#1}{translationException}{\sailRISCVfntranslationException}{}%
  \ifstrequal{#1}{translationMode}{\sailRISCVfntranslationMode}{}%
  \ifstrequal{#1}{trapVectorMode_of_bits}{\sailRISCVfntrapVectorModeOfBits}{}%
  \ifstrequal{#1}{trapVectorMode\_of\_bits}{\sailRISCVfntrapVectorModeOfBits}{}%
  \ifstrequal{#1}{trap_handler}{\sailRISCVfntrapHandler}{}%
  \ifstrequal{#1}{trap\_handler}{\sailRISCVfntrapHandler}{}%
  \ifstrequal{#1}{tval}{\sailRISCVfntval}{}%
  \ifstrequal{#1}{tvec_addr}{\sailRISCVfntvecAddr}{}%
  \ifstrequal{#1}{tvec\_addr}{\sailRISCVfntvecAddr}{}%
  \ifstrequal{#1}{ufFlag}{\sailRISCVfnufFlag}{}%
  \ifstrequal{#1}{unsealCap}{\sailRISCVfnunsealCap}{}%
  \ifstrequal{#1}{uop_of_num}{\sailRISCVfnuopOfNum}{}%
  \ifstrequal{#1}{uop\_of\_num}{\sailRISCVfnuopOfNum}{}%
  \ifstrequal{#1}{update_PTE_Bits}{\sailRISCVfnupdatePTEBits}{}%
  \ifstrequal{#1}{update\_PTE\_Bits}{\sailRISCVfnupdatePTEBits}{}%
  \ifstrequal{#1}{update_softfloat_fflags}{\sailRISCVfnupdateSoftfloatFflags}{}%
  \ifstrequal{#1}{update\_softfloat\_fflags}{\sailRISCVfnupdateSoftfloatFflags}{}%
  \ifstrequal{#1}{validDoubleRegs}{\sailRISCVfnvalidDoubleRegs}{}%
  \ifstrequal{#1}{valid_rounding_mode}{\sailRISCVfnvalidRoundingMode}{}%
  \ifstrequal{#1}{valid\_rounding\_mode}{\sailRISCVfnvalidRoundingMode}{}%
  \ifstrequal{#1}{wC}{\sailRISCVfnwC}{}%
  \ifstrequal{#1}{wC_bits}{\sailRISCVfnwCBits}{}%
  \ifstrequal{#1}{wC\_bits}{\sailRISCVfnwCBits}{}%
  \ifstrequal{#1}{wF}{\sailRISCVfnwF}{}%
  \ifstrequal{#1}{wF_bits}{\sailRISCVfnwFBits}{}%
  \ifstrequal{#1}{wF\_bits}{\sailRISCVfnwFBits}{}%
  \ifstrequal{#1}{wF_or_X_D}{\sailRISCVfnwFOrXD}{}%
  \ifstrequal{#1}{wF\_or\_X\_D}{\sailRISCVfnwFOrXD}{}%
  \ifstrequal{#1}{wF_or_X_H}{\sailRISCVfnwFOrXH}{}%
  \ifstrequal{#1}{wF\_or\_X\_H}{\sailRISCVfnwFOrXH}{}%
  \ifstrequal{#1}{wF_or_X_S}{\sailRISCVfnwFOrXS}{}%
  \ifstrequal{#1}{wF\_or\_X\_S}{\sailRISCVfnwFOrXS}{}%
  \ifstrequal{#1}{wX}{\sailRISCVfnwX}{}%
  \ifstrequal{#1}{wX_bits}{\sailRISCVfnwXBits}{}%
  \ifstrequal{#1}{wX\_bits}{\sailRISCVfnwXBits}{}%
  \ifstrequal{#1}{walk39}{\sailRISCVfnwalkThreeNine}{}%
  \ifstrequal{#1}{walk48}{\sailRISCVfnwalkFourEight}{}%
  \ifstrequal{#1}{within_clint}{\sailRISCVfnwithinClint}{}%
  \ifstrequal{#1}{within\_clint}{\sailRISCVfnwithinClint}{}%
  \ifstrequal{#1}{within_htif_readable}{\sailRISCVfnwithinHtifReadable}{}%
  \ifstrequal{#1}{within\_htif\_readable}{\sailRISCVfnwithinHtifReadable}{}%
  \ifstrequal{#1}{within_htif_writable}{\sailRISCVfnwithinHtifWritable}{}%
  \ifstrequal{#1}{within\_htif\_writable}{\sailRISCVfnwithinHtifWritable}{}%
  \ifstrequal{#1}{within_mmio_readable}{\sailRISCVfnwithinMmioReadable}{}%
  \ifstrequal{#1}{within\_mmio\_readable}{\sailRISCVfnwithinMmioReadable}{}%
  \ifstrequal{#1}{within_mmio_writable}{\sailRISCVfnwithinMmioWritable}{}%
  \ifstrequal{#1}{within\_mmio\_writable}{\sailRISCVfnwithinMmioWritable}{}%
  \ifstrequal{#1}{within_phys_mem}{\sailRISCVfnwithinPhysMem}{}%
  \ifstrequal{#1}{within\_phys\_mem}{\sailRISCVfnwithinPhysMem}{}%
  \ifstrequal{#1}{word_width_bytes}{\sailRISCVfnwordWidthBytes}{}%
  \ifstrequal{#1}{word\_width\_bytes}{\sailRISCVfnwordWidthBytes}{}%
  \ifstrequal{#1}{word_width_of_num}{\sailRISCVfnwordWidthOfNum}{}%
  \ifstrequal{#1}{word\_width\_of\_num}{\sailRISCVfnwordWidthOfNum}{}%
  \ifstrequal{#1}{writeCSR}{\sailRISCVfnwriteCSR}{}%
  \ifstrequal{#1}{write_TLB39}{\sailRISCVfnwriteTLBThreeNine}{}%
  \ifstrequal{#1}{write\_TLB39}{\sailRISCVfnwriteTLBThreeNine}{}%
  \ifstrequal{#1}{write_TLB48}{\sailRISCVfnwriteTLBFourEight}{}%
  \ifstrequal{#1}{write\_TLB48}{\sailRISCVfnwriteTLBFourEight}{}%
  \ifstrequal{#1}{write_fflags}{\sailRISCVfnwriteFflags}{}%
  \ifstrequal{#1}{write\_fflags}{\sailRISCVfnwriteFflags}{}%
  \ifstrequal{#1}{write_kind_of_num}{\sailRISCVfnwriteKindOfNum}{}%
  \ifstrequal{#1}{write\_kind\_of\_num}{\sailRISCVfnwriteKindOfNum}{}%
  \ifstrequal{#1}{write_ram}{\sailRISCVfnwriteRam}{}%
  \ifstrequal{#1}{write\_ram}{\sailRISCVfnwriteRam}{}%
  \ifstrequal{#1}{write_ram_ea}{\sailRISCVfnwriteRamEa}{}%
  \ifstrequal{#1}{write\_ram\_ea}{\sailRISCVfnwriteRamEa}{}%
  \ifstrequal{#1}{write_sc_cap_result}{\sailRISCVfnwriteScCapResult}{}%
  \ifstrequal{#1}{write\_sc\_cap\_result}{\sailRISCVfnwriteScCapResult}{}%
  \ifstrequal{#1}{write_seed_csr}{\sailRISCVfnwriteSeedCsr}{}%
  \ifstrequal{#1}{write\_seed\_csr}{\sailRISCVfnwriteSeedCsr}{}%
  \ifstrequal{#1}{zeros_implicit}{\sailRISCVfnzzerosImplicit}{}%
  \ifstrequal{#1}{zeros\_implicit}{\sailRISCVfnzzerosImplicit}{}%
  \ifstrequal{#1}{(operator <=_u)}{\sailRISCVfnzEightoperatorzZerozIzJUzNine}{}%
  \ifstrequal{#1}{(operator $>$=\_u)}{\sailRISCVfnzEightoperatorzZerozIzJUzNine}{}%
  \ifstrequal{#1}{(operator <_s)}{\sailRISCVfnzEightoperatorzZerozISzNine}{}%
  \ifstrequal{#1}{(operator $>$\_s)}{\sailRISCVfnzEightoperatorzZerozISzNine}{}%
  \ifstrequal{#1}{(operator <_u)}{\sailRISCVfnzEightoperatorzZerozIUzNine}{}%
  \ifstrequal{#1}{(operator $>$\_u)}{\sailRISCVfnzEightoperatorzZerozIUzNine}{}%
  \ifstrequal{#1}{(operator >=_s)}{\sailRISCVfnzEightoperatorzZerozKzJSzNine}{}%
  \ifstrequal{#1}{(operator $$>$$=\_s)}{\sailRISCVfnzEightoperatorzZerozKzJSzNine}{}%
  \ifstrequal{#1}{(operator >=_u)}{\sailRISCVfnzEightoperatorzZerozKzJUzNine}{}%
  \ifstrequal{#1}{(operator $$>$$=\_u)}{\sailRISCVfnzEightoperatorzZerozKzJUzNine}{}}

\newcommand{\sailRISCVreffn}[2]{
  \ifstrequal{#1}{Architecture_of_num}{\hyperref[sailRISCVfnzArchitecturezyofzynum]{#2}}{}%
  \ifstrequal{#1}{Architecture\_of\_num}{\hyperref[sailRISCVfnzArchitecturezyofzynum]{#2}}{}%
  \ifstrequal{#1}{CPtrCmpOp_of_num}{\hyperref[sailRISCVfnzCPtrCmpOpzyofzynum]{#2}}{}%
  \ifstrequal{#1}{CPtrCmpOp\_of\_num}{\hyperref[sailRISCVfnzCPtrCmpOpzyofzynum]{#2}}{}%
  \ifstrequal{#1}{CapExCode}{\hyperref[sailRISCVfnzCapExCode]{#2}}{}%
  \ifstrequal{#1}{CapEx_of_num}{\hyperref[sailRISCVfnzCapExzyofzynum]{#2}}{}%
  \ifstrequal{#1}{CapEx\_of\_num}{\hyperref[sailRISCVfnzCapExzyofzynum]{#2}}{}%
  \ifstrequal{#1}{ClearRegSet_of_num}{\hyperref[sailRISCVfnzClearRegSetzyofzynum]{#2}}{}%
  \ifstrequal{#1}{ClearRegSet\_of\_num}{\hyperref[sailRISCVfnzClearRegSetzyofzynum]{#2}}{}%
  \ifstrequal{#1}{EXTS}{\hyperref[sailRISCVfnzEXTS]{#2}}{}%
  \ifstrequal{#1}{EXTZ}{\hyperref[sailRISCVfnzEXTZ]{#2}}{}%
  \ifstrequal{#1}{ExtStatus_of_num}{\hyperref[sailRISCVfnzExtStatuszyofzynum]{#2}}{}%
  \ifstrequal{#1}{ExtStatus\_of\_num}{\hyperref[sailRISCVfnzExtStatuszyofzynum]{#2}}{}%
  \ifstrequal{#1}{FRegStr}{\hyperref[sailRISCVfnzFRegStr]{#2}}{}%
  \ifstrequal{#1}{GPRstr}{\hyperref[sailRISCVfnzGPRstr]{#2}}{}%
  \ifstrequal{#1}{InterruptType_of_num}{\hyperref[sailRISCVfnzInterruptTypezyofzynum]{#2}}{}%
  \ifstrequal{#1}{InterruptType\_of\_num}{\hyperref[sailRISCVfnzInterruptTypezyofzynum]{#2}}{}%
  \ifstrequal{#1}{MAX}{\hyperref[sailRISCVfnzMAX]{#2}}{}%
  \ifstrequal{#1}{MemoryOpResult_add_meta}{\hyperref[sailRISCVfnzMemoryOpResultzyaddzymeta]{#2}}{}%
  \ifstrequal{#1}{MemoryOpResult\_add\_meta}{\hyperref[sailRISCVfnzMemoryOpResultzyaddzymeta]{#2}}{}%
  \ifstrequal{#1}{MemoryOpResult_drop_meta}{\hyperref[sailRISCVfnzMemoryOpResultzydropzymeta]{#2}}{}%
  \ifstrequal{#1}{MemoryOpResult\_drop\_meta}{\hyperref[sailRISCVfnzMemoryOpResultzydropzymeta]{#2}}{}%
  \ifstrequal{#1}{PmpAddrMatchType_of_num}{\hyperref[sailRISCVfnzPmpAddrMatchTypezyofzynum]{#2}}{}%
  \ifstrequal{#1}{PmpAddrMatchType\_of\_num}{\hyperref[sailRISCVfnzPmpAddrMatchTypezyofzynum]{#2}}{}%
  \ifstrequal{#1}{Privilege_of_num}{\hyperref[sailRISCVfnzPrivilegezyofzynum]{#2}}{}%
  \ifstrequal{#1}{Privilege\_of\_num}{\hyperref[sailRISCVfnzPrivilegezyofzynum]{#2}}{}%
  \ifstrequal{#1}{RegStr}{\hyperref[sailRISCVfnzRegStr]{#2}}{}%
  \ifstrequal{#1}{Retired_of_num}{\hyperref[sailRISCVfnzRetiredzyofzynum]{#2}}{}%
  \ifstrequal{#1}{Retired\_of\_num}{\hyperref[sailRISCVfnzRetiredzyofzynum]{#2}}{}%
  \ifstrequal{#1}{SATPMode_of_num}{\hyperref[sailRISCVfnzSATPModezyofzynum]{#2}}{}%
  \ifstrequal{#1}{SATPMode\_of\_num}{\hyperref[sailRISCVfnzSATPModezyofzynum]{#2}}{}%
  \ifstrequal{#1}{TrapVectorMode_of_num}{\hyperref[sailRISCVfnzTrapVectorModezyofzynum]{#2}}{}%
  \ifstrequal{#1}{TrapVectorMode\_of\_num}{\hyperref[sailRISCVfnzTrapVectorModezyofzynum]{#2}}{}%
  \ifstrequal{#1}{__ReadRAM_Meta}{\hyperref[sailRISCVfnzzyzyReadRAMzyMeta]{#2}}{}%
  \ifstrequal{#1}{\_\_ReadRAM\_Meta}{\hyperref[sailRISCVfnzzyzyReadRAMzyMeta]{#2}}{}%
  \ifstrequal{#1}{__WriteRAM_Meta}{\hyperref[sailRISCVfnzzyzyWriteRAMzyMeta]{#2}}{}%
  \ifstrequal{#1}{\_\_WriteRAM\_Meta}{\hyperref[sailRISCVfnzzyzyWriteRAMzyMeta]{#2}}{}%
  \ifstrequal{#1}{__id}{\hyperref[sailRISCVfnzzyzyid]{#2}}{}%
  \ifstrequal{#1}{\_\_id}{\hyperref[sailRISCVfnzzyzyid]{#2}}{}%
  \ifstrequal{#1}{_shl_int_general}{\hyperref[sailRISCVfnzzyshlzyintzygeneral]{#2}}{}%
  \ifstrequal{#1}{\_shl\_int\_general}{\hyperref[sailRISCVfnzzyshlzyintzygeneral]{#2}}{}%
  \ifstrequal{#1}{_shr_int_general}{\hyperref[sailRISCVfnzzyshrzyintzygeneral]{#2}}{}%
  \ifstrequal{#1}{\_shr\_int\_general}{\hyperref[sailRISCVfnzzyshrzyintzygeneral]{#2}}{}%
  \ifstrequal{#1}{a64_barrier_domain_of_num}{\hyperref[sailRISCVfnza64zybarrierzydomainzyofzynum]{#2}}{}%
  \ifstrequal{#1}{a64\_barrier\_domain\_of\_num}{\hyperref[sailRISCVfnza64zybarrierzydomainzyofzynum]{#2}}{}%
  \ifstrequal{#1}{a64_barrier_type_of_num}{\hyperref[sailRISCVfnza64zybarrierzytypezyofzynum]{#2}}{}%
  \ifstrequal{#1}{a64\_barrier\_type\_of\_num}{\hyperref[sailRISCVfnza64zybarrierzytypezyofzynum]{#2}}{}%
  \ifstrequal{#1}{accessType_to_str}{\hyperref[sailRISCVfnzaccessTypezytozystr]{#2}}{}%
  \ifstrequal{#1}{accessType\_to\_str}{\hyperref[sailRISCVfnzaccessTypezytozystr]{#2}}{}%
  \ifstrequal{#1}{accrue_fflags}{\hyperref[sailRISCVfnzaccruezyfflags]{#2}}{}%
  \ifstrequal{#1}{accrue\_fflags}{\hyperref[sailRISCVfnzaccruezyfflags]{#2}}{}%
  \ifstrequal{#1}{add_to_TLB39}{\hyperref[sailRISCVfnzaddzytozyTLB39]{#2}}{}%
  \ifstrequal{#1}{add\_to\_TLB39}{\hyperref[sailRISCVfnzaddzytozyTLB39]{#2}}{}%
  \ifstrequal{#1}{add_to_TLB48}{\hyperref[sailRISCVfnzaddzytozyTLB48]{#2}}{}%
  \ifstrequal{#1}{add\_to\_TLB48}{\hyperref[sailRISCVfnzaddzytozyTLB48]{#2}}{}%
  \ifstrequal{#1}{addr_to_tag_addr}{\hyperref[sailRISCVfnzaddrzytozytagzyaddr]{#2}}{}%
  \ifstrequal{#1}{addr\_to\_tag\_addr}{\hyperref[sailRISCVfnzaddrzytozytagzyaddr]{#2}}{}%
  \ifstrequal{#1}{amo_width_valid}{\hyperref[sailRISCVfnzamozywidthzyvalid]{#2}}{}%
  \ifstrequal{#1}{amo\_width\_valid}{\hyperref[sailRISCVfnzamozywidthzyvalid]{#2}}{}%
  \ifstrequal{#1}{amoop_of_num}{\hyperref[sailRISCVfnzamoopzyofzynum]{#2}}{}%
  \ifstrequal{#1}{amoop\_of\_num}{\hyperref[sailRISCVfnzamoopzyofzynum]{#2}}{}%
  \ifstrequal{#1}{aqrl_str}{\hyperref[sailRISCVfnzaqrlzystr]{#2}}{}%
  \ifstrequal{#1}{aqrl\_str}{\hyperref[sailRISCVfnzaqrlzystr]{#2}}{}%
  \ifstrequal{#1}{arch_to_bits}{\hyperref[sailRISCVfnzarchzytozybits]{#2}}{}%
  \ifstrequal{#1}{arch\_to\_bits}{\hyperref[sailRISCVfnzarchzytozybits]{#2}}{}%
  \ifstrequal{#1}{architecture}{\hyperref[sailRISCVfnzarchitecture]{#2}}{}%
  \ifstrequal{#1}{biop_zbs_of_num}{\hyperref[sailRISCVfnzbiopzyzzbszyofzynum]{#2}}{}%
  \ifstrequal{#1}{biop\_zbs\_of\_num}{\hyperref[sailRISCVfnzbiopzyzzbszyofzynum]{#2}}{}%
  \ifstrequal{#1}{bit_to_bool}{\hyperref[sailRISCVfnzbitzytozybool]{#2}}{}%
  \ifstrequal{#1}{bit\_to\_bool}{\hyperref[sailRISCVfnzbitzytozybool]{#2}}{}%
  \ifstrequal{#1}{bool_to_bit}{\hyperref[sailRISCVfnzboolzytozybit]{#2}}{}%
  \ifstrequal{#1}{bool\_to\_bit}{\hyperref[sailRISCVfnzboolzytozybit]{#2}}{}%
  \ifstrequal{#1}{bool_to_bits}{\hyperref[sailRISCVfnzboolzytozybits]{#2}}{}%
  \ifstrequal{#1}{bool\_to\_bits}{\hyperref[sailRISCVfnzboolzytozybits]{#2}}{}%
  \ifstrequal{#1}{bop_of_num}{\hyperref[sailRISCVfnzbopzyofzynum]{#2}}{}%
  \ifstrequal{#1}{bop\_of\_num}{\hyperref[sailRISCVfnzbopzyofzynum]{#2}}{}%
  \ifstrequal{#1}{brop_zba_of_num}{\hyperref[sailRISCVfnzbropzyzzbazyofzynum]{#2}}{}%
  \ifstrequal{#1}{brop\_zba\_of\_num}{\hyperref[sailRISCVfnzbropzyzzbazyofzynum]{#2}}{}%
  \ifstrequal{#1}{brop_zbb_of_num}{\hyperref[sailRISCVfnzbropzyzzbbzyofzynum]{#2}}{}%
  \ifstrequal{#1}{brop\_zbb\_of\_num}{\hyperref[sailRISCVfnzbropzyzzbbzyofzynum]{#2}}{}%
  \ifstrequal{#1}{brop_zbkb_of_num}{\hyperref[sailRISCVfnzbropzyzzbkbzyofzynum]{#2}}{}%
  \ifstrequal{#1}{brop\_zbkb\_of\_num}{\hyperref[sailRISCVfnzbropzyzzbkbzyofzynum]{#2}}{}%
  \ifstrequal{#1}{brop_zbs_of_num}{\hyperref[sailRISCVfnzbropzyzzbszyofzynum]{#2}}{}%
  \ifstrequal{#1}{brop\_zbs\_of\_num}{\hyperref[sailRISCVfnzbropzyzzbszyofzynum]{#2}}{}%
  \ifstrequal{#1}{bropw_zba_of_num}{\hyperref[sailRISCVfnzbropwzyzzbazyofzynum]{#2}}{}%
  \ifstrequal{#1}{bropw\_zba\_of\_num}{\hyperref[sailRISCVfnzbropwzyzzbazyofzynum]{#2}}{}%
  \ifstrequal{#1}{bropw_zbb_of_num}{\hyperref[sailRISCVfnzbropwzyzzbbzyofzynum]{#2}}{}%
  \ifstrequal{#1}{bropw\_zbb\_of\_num}{\hyperref[sailRISCVfnzbropwzyzzbbzyofzynum]{#2}}{}%
  \ifstrequal{#1}{cache_op_kind_of_num}{\hyperref[sailRISCVfnzcachezyopzykindzyofzynum]{#2}}{}%
  \ifstrequal{#1}{cache\_op\_kind\_of\_num}{\hyperref[sailRISCVfnzcachezyopzykindzyofzynum]{#2}}{}%
  \ifstrequal{#1}{canonical_NaN_D}{\hyperref[sailRISCVfnzcanonicalzyNaNzyD]{#2}}{}%
  \ifstrequal{#1}{canonical\_NaN\_D}{\hyperref[sailRISCVfnzcanonicalzyNaNzyD]{#2}}{}%
  \ifstrequal{#1}{canonical_NaN_H}{\hyperref[sailRISCVfnzcanonicalzyNaNzyH]{#2}}{}%
  \ifstrequal{#1}{canonical\_NaN\_H}{\hyperref[sailRISCVfnzcanonicalzyNaNzyH]{#2}}{}%
  \ifstrequal{#1}{canonical_NaN_S}{\hyperref[sailRISCVfnzcanonicalzyNaNzyS]{#2}}{}%
  \ifstrequal{#1}{canonical\_NaN\_S}{\hyperref[sailRISCVfnzcanonicalzyNaNzyS]{#2}}{}%
  \ifstrequal{#1}{capBitsToCapability}{\hyperref[sailRISCVfnzcapBitsToCapability]{#2}}{}%
  \ifstrequal{#1}{capBitsToEncCapability}{\hyperref[sailRISCVfnzcapBitsToEncCapability]{#2}}{}%
  \ifstrequal{#1}{capBoundsEqual}{\hyperref[sailRISCVfnzcapBoundsEqual]{#2}}{}%
  \ifstrequal{#1}{capToBits}{\hyperref[sailRISCVfnzcapToBits]{#2}}{}%
  \ifstrequal{#1}{capToEncCap}{\hyperref[sailRISCVfnzcapToEncCap]{#2}}{}%
  \ifstrequal{#1}{capToMemBits}{\hyperref[sailRISCVfnzcapToMemBits]{#2}}{}%
  \ifstrequal{#1}{capToString}{\hyperref[sailRISCVfnzcapToString]{#2}}{}%
  \ifstrequal{#1}{cap_reg_name_abi}{\hyperref[sailRISCVfnzcapzyregzynamezyabi]{#2}}{}%
  \ifstrequal{#1}{cap\_reg\_name\_abi}{\hyperref[sailRISCVfnzcapzyregzynamezyabi]{#2}}{}%
  \ifstrequal{#1}{checkPTEPermission}{\hyperref[sailRISCVfnzcheckPTEPermission]{#2}}{}%
  \ifstrequal{#1}{check_CSR}{\hyperref[sailRISCVfnzcheckzyCSR]{#2}}{}%
  \ifstrequal{#1}{check\_CSR}{\hyperref[sailRISCVfnzcheckzyCSR]{#2}}{}%
  \ifstrequal{#1}{check_CSR_access}{\hyperref[sailRISCVfnzcheckzyCSRzyaccess]{#2}}{}%
  \ifstrequal{#1}{check\_CSR\_access}{\hyperref[sailRISCVfnzcheckzyCSRzyaccess]{#2}}{}%
  \ifstrequal{#1}{check_Counteren}{\hyperref[sailRISCVfnzcheckzyCounteren]{#2}}{}%
  \ifstrequal{#1}{check\_Counteren}{\hyperref[sailRISCVfnzcheckzyCounteren]{#2}}{}%
  \ifstrequal{#1}{check_TVM_SATP}{\hyperref[sailRISCVfnzcheckzyTVMzySATP]{#2}}{}%
  \ifstrequal{#1}{check\_TVM\_SATP}{\hyperref[sailRISCVfnzcheckzyTVMzySATP]{#2}}{}%
  \ifstrequal{#1}{check_misaligned}{\hyperref[sailRISCVfnzcheckzymisaligned]{#2}}{}%
  \ifstrequal{#1}{check\_misaligned}{\hyperref[sailRISCVfnzcheckzymisaligned]{#2}}{}%
  \ifstrequal{#1}{check_res_misaligned}{\hyperref[sailRISCVfnzcheckzyreszymisaligned]{#2}}{}%
  \ifstrequal{#1}{check\_res\_misaligned}{\hyperref[sailRISCVfnzcheckzyreszymisaligned]{#2}}{}%
  \ifstrequal{#1}{check_seed_CSR}{\hyperref[sailRISCVfnzcheckzyseedzyCSR]{#2}}{}%
  \ifstrequal{#1}{check\_seed\_CSR}{\hyperref[sailRISCVfnzcheckzyseedzyCSR]{#2}}{}%
  \ifstrequal{#1}{checked_mem_read}{\hyperref[sailRISCVfnzcheckedzymemzyread]{#2}}{}%
  \ifstrequal{#1}{checked\_mem\_read}{\hyperref[sailRISCVfnzcheckedzymemzyread]{#2}}{}%
  \ifstrequal{#1}{checked_mem_write}{\hyperref[sailRISCVfnzcheckedzymemzywrite]{#2}}{}%
  \ifstrequal{#1}{checked\_mem\_write}{\hyperref[sailRISCVfnzcheckedzymemzywrite]{#2}}{}%
  \ifstrequal{#1}{clearTag}{\hyperref[sailRISCVfnzclearTag]{#2}}{}%
  \ifstrequal{#1}{clearTagIf}{\hyperref[sailRISCVfnzclearTagIf]{#2}}{}%
  \ifstrequal{#1}{clearTagIfSealed}{\hyperref[sailRISCVfnzclearTagIfSealed]{#2}}{}%
  \ifstrequal{#1}{clint_dispatch}{\hyperref[sailRISCVfnzclintzydispatch]{#2}}{}%
  \ifstrequal{#1}{clint\_dispatch}{\hyperref[sailRISCVfnzclintzydispatch]{#2}}{}%
  \ifstrequal{#1}{clint_load}{\hyperref[sailRISCVfnzclintzyload]{#2}}{}%
  \ifstrequal{#1}{clint\_load}{\hyperref[sailRISCVfnzclintzyload]{#2}}{}%
  \ifstrequal{#1}{clint_store}{\hyperref[sailRISCVfnzclintzystore]{#2}}{}%
  \ifstrequal{#1}{clint\_store}{\hyperref[sailRISCVfnzclintzystore]{#2}}{}%
  \ifstrequal{#1}{concat_str_bits}{\hyperref[sailRISCVfnzconcatzystrzybits]{#2}}{}%
  \ifstrequal{#1}{concat\_str\_bits}{\hyperref[sailRISCVfnzconcatzystrzybits]{#2}}{}%
  \ifstrequal{#1}{concat_str_dec}{\hyperref[sailRISCVfnzconcatzystrzydec]{#2}}{}%
  \ifstrequal{#1}{concat\_str\_dec}{\hyperref[sailRISCVfnzconcatzystrzydec]{#2}}{}%
  \ifstrequal{#1}{creg2reg_idx}{\hyperref[sailRISCVfnzcreg2regzyidx]{#2}}{}%
  \ifstrequal{#1}{creg2reg\_idx}{\hyperref[sailRISCVfnzcreg2regzyidx]{#2}}{}%
  \ifstrequal{#1}{csrAccess}{\hyperref[sailRISCVfnzcsrAccess]{#2}}{}%
  \ifstrequal{#1}{csrPriv}{\hyperref[sailRISCVfnzcsrPriv]{#2}}{}%
  \ifstrequal{#1}{csr_name}{\hyperref[sailRISCVfnzcsrzyname]{#2}}{}%
  \ifstrequal{#1}{csr\_name}{\hyperref[sailRISCVfnzcsrzyname]{#2}}{}%
  \ifstrequal{#1}{csrop_of_num}{\hyperref[sailRISCVfnzcsropzyofzynum]{#2}}{}%
  \ifstrequal{#1}{csrop\_of\_num}{\hyperref[sailRISCVfnzcsropzyofzynum]{#2}}{}%
  \ifstrequal{#1}{curAsid32}{\hyperref[sailRISCVfnzcurAsid32]{#2}}{}%
  \ifstrequal{#1}{curAsid64}{\hyperref[sailRISCVfnzcurAsid64]{#2}}{}%
  \ifstrequal{#1}{curPTB32}{\hyperref[sailRISCVfnzcurPTB32]{#2}}{}%
  \ifstrequal{#1}{curPTB64}{\hyperref[sailRISCVfnzcurPTB64]{#2}}{}%
  \ifstrequal{#1}{cur_Architecture}{\hyperref[sailRISCVfnzcurzyArchitecture]{#2}}{}%
  \ifstrequal{#1}{cur\_Architecture}{\hyperref[sailRISCVfnzcurzyArchitecture]{#2}}{}%
  \ifstrequal{#1}{decode}{\hyperref[sailRISCVfnzdecode]{#2}}{}%
  \ifstrequal{#1}{decodeCompressed}{\hyperref[sailRISCVfnzdecodeCompressed]{#2}}{}%
  \ifstrequal{#1}{def_spc_backwards}{\hyperref[sailRISCVfnzdefzyspczybackwards]{#2}}{}%
  \ifstrequal{#1}{def\_spc\_backwards}{\hyperref[sailRISCVfnzdefzyspczybackwards]{#2}}{}%
  \ifstrequal{#1}{def_spc_forwards}{\hyperref[sailRISCVfnzdefzyspczyforwards]{#2}}{}%
  \ifstrequal{#1}{def\_spc\_forwards}{\hyperref[sailRISCVfnzdefzyspczyforwards]{#2}}{}%
  \ifstrequal{#1}{def_spc_matches_prefix}{\hyperref[sailRISCVfnzdefzyspczymatcheszyprefix]{#2}}{}%
  \ifstrequal{#1}{def\_spc\_matches\_prefix}{\hyperref[sailRISCVfnzdefzyspczymatcheszyprefix]{#2}}{}%
  \ifstrequal{#1}{dirty_fd_context}{\hyperref[sailRISCVfnzdirtyzyfdzycontext]{#2}}{}%
  \ifstrequal{#1}{dirty\_fd\_context}{\hyperref[sailRISCVfnzdirtyzyfdzycontext]{#2}}{}%
  \ifstrequal{#1}{dirty_fd_context_if_present}{\hyperref[sailRISCVfnzdirtyzyfdzycontextzyifzypresent]{#2}}{}%
  \ifstrequal{#1}{dirty\_fd\_context\_if\_present}{\hyperref[sailRISCVfnzdirtyzyfdzycontextzyifzypresent]{#2}}{}%
  \ifstrequal{#1}{dispatchInterrupt}{\hyperref[sailRISCVfnzdispatchInterrupt]{#2}}{}%
  \ifstrequal{#1}{dzFlag}{\hyperref[sailRISCVfnzdzzFlag]{#2}}{}%
  \ifstrequal{#1}{effectivePrivilege}{\hyperref[sailRISCVfnzeffectivePrivilege]{#2}}{}%
  \ifstrequal{#1}{encCapToBits}{\hyperref[sailRISCVfnzencCapToBits]{#2}}{}%
  \ifstrequal{#1}{encCapabilityToCapability}{\hyperref[sailRISCVfnzencCapabilityToCapability]{#2}}{}%
  \ifstrequal{#1}{eq_unit}{\hyperref[sailRISCVfnzeqzyunit]{#2}}{}%
  \ifstrequal{#1}{eq\_unit}{\hyperref[sailRISCVfnzeqzyunit]{#2}}{}%
  \ifstrequal{#1}{exceptionType_to_bits}{\hyperref[sailRISCVfnzexceptionTypezytozybits]{#2}}{}%
  \ifstrequal{#1}{exceptionType\_to\_bits}{\hyperref[sailRISCVfnzexceptionTypezytozybits]{#2}}{}%
  \ifstrequal{#1}{exceptionType_to_str}{\hyperref[sailRISCVfnzexceptionTypezytozystr]{#2}}{}%
  \ifstrequal{#1}{exceptionType\_to\_str}{\hyperref[sailRISCVfnzexceptionTypezytozystr]{#2}}{}%
  \ifstrequal{#1}{exception_delegatee}{\hyperref[sailRISCVfnzexceptionzydelegatee]{#2}}{}%
  \ifstrequal{#1}{exception\_delegatee}{\hyperref[sailRISCVfnzexceptionzydelegatee]{#2}}{}%
  \ifstrequal{#1}{exception_handler}{\hyperref[sailRISCVfnzexceptionzyhandler]{#2}}{}%
  \ifstrequal{#1}{exception\_handler}{\hyperref[sailRISCVfnzexceptionzyhandler]{#2}}{}%
  \ifstrequal{#1}{extStatus_of_bits}{\hyperref[sailRISCVfnzextStatuszyofzybits]{#2}}{}%
  \ifstrequal{#1}{extStatus\_of\_bits}{\hyperref[sailRISCVfnzextStatuszyofzybits]{#2}}{}%
  \ifstrequal{#1}{extStatus_to_bits}{\hyperref[sailRISCVfnzextStatuszytozybits]{#2}}{}%
  \ifstrequal{#1}{extStatus\_to\_bits}{\hyperref[sailRISCVfnzextStatuszytozybits]{#2}}{}%
  \ifstrequal{#1}{ext_access_type_of_num}{\hyperref[sailRISCVfnzextzyaccesszytypezyofzynum]{#2}}{}%
  \ifstrequal{#1}{ext\_access\_type\_of\_num}{\hyperref[sailRISCVfnzextzyaccesszytypezyofzynum]{#2}}{}%
  \ifstrequal{#1}{ext_check_CSR}{\hyperref[sailRISCVfnzextzycheckzyCSR]{#2}}{}%
  \ifstrequal{#1}{ext\_check\_CSR}{\hyperref[sailRISCVfnzextzycheckzyCSR]{#2}}{}%
  \ifstrequal{#1}{ext_check_CSR_fail}{\hyperref[sailRISCVfnzextzycheckzyCSRzyfail]{#2}}{}%
  \ifstrequal{#1}{ext\_check\_CSR\_fail}{\hyperref[sailRISCVfnzextzycheckzyCSRzyfail]{#2}}{}%
  \ifstrequal{#1}{ext_check_phys_mem_read}{\hyperref[sailRISCVfnzextzycheckzyphyszymemzyread]{#2}}{}%
  \ifstrequal{#1}{ext\_check\_phys\_mem\_read}{\hyperref[sailRISCVfnzextzycheckzyphyszymemzyread]{#2}}{}%
  \ifstrequal{#1}{ext_check_phys_mem_write}{\hyperref[sailRISCVfnzextzycheckzyphyszymemzywrite]{#2}}{}%
  \ifstrequal{#1}{ext\_check\_phys\_mem\_write}{\hyperref[sailRISCVfnzextzycheckzyphyszymemzywrite]{#2}}{}%
  \ifstrequal{#1}{ext_check_xret_priv}{\hyperref[sailRISCVfnzextzycheckzyxretzypriv]{#2}}{}%
  \ifstrequal{#1}{ext\_check\_xret\_priv}{\hyperref[sailRISCVfnzextzycheckzyxretzypriv]{#2}}{}%
  \ifstrequal{#1}{ext_control_check_addr}{\hyperref[sailRISCVfnzextzycontrolzycheckzyaddr]{#2}}{}%
  \ifstrequal{#1}{ext\_control\_check\_addr}{\hyperref[sailRISCVfnzextzycontrolzycheckzyaddr]{#2}}{}%
  \ifstrequal{#1}{ext_control_check_pc}{\hyperref[sailRISCVfnzextzycontrolzycheckzypc]{#2}}{}%
  \ifstrequal{#1}{ext\_control\_check\_pc}{\hyperref[sailRISCVfnzextzycontrolzycheckzypc]{#2}}{}%
  \ifstrequal{#1}{ext_data_get_addr}{\hyperref[sailRISCVfnzextzydatazygetzyaddr]{#2}}{}%
  \ifstrequal{#1}{ext\_data\_get\_addr}{\hyperref[sailRISCVfnzextzydatazygetzyaddr]{#2}}{}%
  \ifstrequal{#1}{ext_exc_type_of_num}{\hyperref[sailRISCVfnzextzyexczytypezyofzynum]{#2}}{}%
  \ifstrequal{#1}{ext\_exc\_type\_of\_num}{\hyperref[sailRISCVfnzextzyexczytypezyofzynum]{#2}}{}%
  \ifstrequal{#1}{ext_exc_type_to_bits}{\hyperref[sailRISCVfnzextzyexczytypezytozybits]{#2}}{}%
  \ifstrequal{#1}{ext\_exc\_type\_to\_bits}{\hyperref[sailRISCVfnzextzyexczytypezytozybits]{#2}}{}%
  \ifstrequal{#1}{ext_exc_type_to_str}{\hyperref[sailRISCVfnzextzyexczytypezytozystr]{#2}}{}%
  \ifstrequal{#1}{ext\_exc\_type\_to\_str}{\hyperref[sailRISCVfnzextzyexczytypezytozystr]{#2}}{}%
  \ifstrequal{#1}{ext_fail_xret_priv}{\hyperref[sailRISCVfnzextzyfailzyxretzypriv]{#2}}{}%
  \ifstrequal{#1}{ext\_fail\_xret\_priv}{\hyperref[sailRISCVfnzextzyfailzyxretzypriv]{#2}}{}%
  \ifstrequal{#1}{ext_fetch_check_pc}{\hyperref[sailRISCVfnzextzyfetchzycheckzypc]{#2}}{}%
  \ifstrequal{#1}{ext\_fetch\_check\_pc}{\hyperref[sailRISCVfnzextzyfetchzycheckzypc]{#2}}{}%
  \ifstrequal{#1}{ext_fetch_hook}{\hyperref[sailRISCVfnzextzyfetchzyhook]{#2}}{}%
  \ifstrequal{#1}{ext\_fetch\_hook}{\hyperref[sailRISCVfnzextzyfetchzyhook]{#2}}{}%
  \ifstrequal{#1}{ext_get_ptw_error}{\hyperref[sailRISCVfnzextzygetzyptwzyerror]{#2}}{}%
  \ifstrequal{#1}{ext\_get\_ptw\_error}{\hyperref[sailRISCVfnzextzygetzyptwzyerror]{#2}}{}%
  \ifstrequal{#1}{ext_handle_control_check_error}{\hyperref[sailRISCVfnzextzyhandlezycontrolzycheckzyerror]{#2}}{}%
  \ifstrequal{#1}{ext\_handle\_control\_check\_error}{\hyperref[sailRISCVfnzextzyhandlezycontrolzycheckzyerror]{#2}}{}%
  \ifstrequal{#1}{ext_handle_data_check_error}{\hyperref[sailRISCVfnzextzyhandlezydatazycheckzyerror]{#2}}{}%
  \ifstrequal{#1}{ext\_handle\_data\_check\_error}{\hyperref[sailRISCVfnzextzyhandlezydatazycheckzyerror]{#2}}{}%
  \ifstrequal{#1}{ext_handle_fetch_check_error}{\hyperref[sailRISCVfnzextzyhandlezyfetchzycheckzyerror]{#2}}{}%
  \ifstrequal{#1}{ext\_handle\_fetch\_check\_error}{\hyperref[sailRISCVfnzextzyhandlezyfetchzycheckzyerror]{#2}}{}%
  \ifstrequal{#1}{ext_init}{\hyperref[sailRISCVfnzextzyinit]{#2}}{}%
  \ifstrequal{#1}{ext\_init}{\hyperref[sailRISCVfnzextzyinit]{#2}}{}%
  \ifstrequal{#1}{ext_init_regs}{\hyperref[sailRISCVfnzextzyinitzyregs]{#2}}{}%
  \ifstrequal{#1}{ext\_init\_regs}{\hyperref[sailRISCVfnzextzyinitzyregs]{#2}}{}%
  \ifstrequal{#1}{ext_post_decode_hook}{\hyperref[sailRISCVfnzextzypostzydecodezyhook]{#2}}{}%
  \ifstrequal{#1}{ext\_post\_decode\_hook}{\hyperref[sailRISCVfnzextzypostzydecodezyhook]{#2}}{}%
  \ifstrequal{#1}{ext_post_step_hook}{\hyperref[sailRISCVfnzextzypostzystepzyhook]{#2}}{}%
  \ifstrequal{#1}{ext\_post\_step\_hook}{\hyperref[sailRISCVfnzextzypostzystepzyhook]{#2}}{}%
  \ifstrequal{#1}{ext_pre_step_hook}{\hyperref[sailRISCVfnzextzyprezystepzyhook]{#2}}{}%
  \ifstrequal{#1}{ext\_pre\_step\_hook}{\hyperref[sailRISCVfnzextzyprezystepzyhook]{#2}}{}%
  \ifstrequal{#1}{ext_ptw_error_of_num}{\hyperref[sailRISCVfnzextzyptwzyerrorzyofzynum]{#2}}{}%
  \ifstrequal{#1}{ext\_ptw\_error\_of\_num}{\hyperref[sailRISCVfnzextzyptwzyerrorzyofzynum]{#2}}{}%
  \ifstrequal{#1}{ext_ptw_fail_of_num}{\hyperref[sailRISCVfnzextzyptwzyfailzyofzynum]{#2}}{}%
  \ifstrequal{#1}{ext\_ptw\_fail\_of\_num}{\hyperref[sailRISCVfnzextzyptwzyfailzyofzynum]{#2}}{}%
  \ifstrequal{#1}{ext_ptw_lc_join}{\hyperref[sailRISCVfnzextzyptwzylczyjoin]{#2}}{}%
  \ifstrequal{#1}{ext\_ptw\_lc\_join}{\hyperref[sailRISCVfnzextzyptwzylczyjoin]{#2}}{}%
  \ifstrequal{#1}{ext_ptw_lc_of_num}{\hyperref[sailRISCVfnzextzyptwzylczyofzynum]{#2}}{}%
  \ifstrequal{#1}{ext\_ptw\_lc\_of\_num}{\hyperref[sailRISCVfnzextzyptwzylczyofzynum]{#2}}{}%
  \ifstrequal{#1}{ext_ptw_sc_join}{\hyperref[sailRISCVfnzextzyptwzysczyjoin]{#2}}{}%
  \ifstrequal{#1}{ext\_ptw\_sc\_join}{\hyperref[sailRISCVfnzextzyptwzysczyjoin]{#2}}{}%
  \ifstrequal{#1}{ext_ptw_sc_of_num}{\hyperref[sailRISCVfnzextzyptwzysczyofzynum]{#2}}{}%
  \ifstrequal{#1}{ext\_ptw\_sc\_of\_num}{\hyperref[sailRISCVfnzextzyptwzysczyofzynum]{#2}}{}%
  \ifstrequal{#1}{ext_rvfi_init}{\hyperref[sailRISCVfnzextzyrvfizyinit]{#2}}{}%
  \ifstrequal{#1}{ext\_rvfi\_init}{\hyperref[sailRISCVfnzextzyrvfizyinit]{#2}}{}%
  \ifstrequal{#1}{ext_veto_disable_C}{\hyperref[sailRISCVfnzextzyvetozydisablezyC]{#2}}{}%
  \ifstrequal{#1}{ext\_veto\_disable\_C}{\hyperref[sailRISCVfnzextzyvetozydisablezyC]{#2}}{}%
  \ifstrequal{#1}{ext_write_fcsr}{\hyperref[sailRISCVfnzextzywritezyfcsr]{#2}}{}%
  \ifstrequal{#1}{ext\_write\_fcsr}{\hyperref[sailRISCVfnzextzywritezyfcsr]{#2}}{}%
  \ifstrequal{#1}{extend_value}{\hyperref[sailRISCVfnzextendzyvalue]{#2}}{}%
  \ifstrequal{#1}{extend\_value}{\hyperref[sailRISCVfnzextendzyvalue]{#2}}{}%
  \ifstrequal{#1}{extop_zbb_of_num}{\hyperref[sailRISCVfnzextopzyzzbbzyofzynum]{#2}}{}%
  \ifstrequal{#1}{extop\_zbb\_of\_num}{\hyperref[sailRISCVfnzextopzyzzbbzyofzynum]{#2}}{}%
  \ifstrequal{#1}{f_bin_op_D_of_num}{\hyperref[sailRISCVfnzfzybinzyopzyDzyofzynum]{#2}}{}%
  \ifstrequal{#1}{f\_bin\_op\_D\_of\_num}{\hyperref[sailRISCVfnzfzybinzyopzyDzyofzynum]{#2}}{}%
  \ifstrequal{#1}{f_bin_op_H_of_num}{\hyperref[sailRISCVfnzfzybinzyopzyHzyofzynum]{#2}}{}%
  \ifstrequal{#1}{f\_bin\_op\_H\_of\_num}{\hyperref[sailRISCVfnzfzybinzyopzyHzyofzynum]{#2}}{}%
  \ifstrequal{#1}{f_bin_op_S_of_num}{\hyperref[sailRISCVfnzfzybinzyopzySzyofzynum]{#2}}{}%
  \ifstrequal{#1}{f\_bin\_op\_S\_of\_num}{\hyperref[sailRISCVfnzfzybinzyopzySzyofzynum]{#2}}{}%
  \ifstrequal{#1}{f_bin_rm_op_D_of_num}{\hyperref[sailRISCVfnzfzybinzyrmzyopzyDzyofzynum]{#2}}{}%
  \ifstrequal{#1}{f\_bin\_rm\_op\_D\_of\_num}{\hyperref[sailRISCVfnzfzybinzyrmzyopzyDzyofzynum]{#2}}{}%
  \ifstrequal{#1}{f_bin_rm_op_H_of_num}{\hyperref[sailRISCVfnzfzybinzyrmzyopzyHzyofzynum]{#2}}{}%
  \ifstrequal{#1}{f\_bin\_rm\_op\_H\_of\_num}{\hyperref[sailRISCVfnzfzybinzyrmzyopzyHzyofzynum]{#2}}{}%
  \ifstrequal{#1}{f_bin_rm_op_S_of_num}{\hyperref[sailRISCVfnzfzybinzyrmzyopzySzyofzynum]{#2}}{}%
  \ifstrequal{#1}{f\_bin\_rm\_op\_S\_of\_num}{\hyperref[sailRISCVfnzfzybinzyrmzyopzySzyofzynum]{#2}}{}%
  \ifstrequal{#1}{f_is_NaN_D}{\hyperref[sailRISCVfnzfzyiszyNaNzyD]{#2}}{}%
  \ifstrequal{#1}{f\_is\_NaN\_D}{\hyperref[sailRISCVfnzfzyiszyNaNzyD]{#2}}{}%
  \ifstrequal{#1}{f_is_NaN_S}{\hyperref[sailRISCVfnzfzyiszyNaNzyS]{#2}}{}%
  \ifstrequal{#1}{f\_is\_NaN\_S}{\hyperref[sailRISCVfnzfzyiszyNaNzyS]{#2}}{}%
  \ifstrequal{#1}{f_is_QNaN_D}{\hyperref[sailRISCVfnzfzyiszyQNaNzyD]{#2}}{}%
  \ifstrequal{#1}{f\_is\_QNaN\_D}{\hyperref[sailRISCVfnzfzyiszyQNaNzyD]{#2}}{}%
  \ifstrequal{#1}{f_is_QNaN_S}{\hyperref[sailRISCVfnzfzyiszyQNaNzyS]{#2}}{}%
  \ifstrequal{#1}{f\_is\_QNaN\_S}{\hyperref[sailRISCVfnzfzyiszyQNaNzyS]{#2}}{}%
  \ifstrequal{#1}{f_is_SNaN_D}{\hyperref[sailRISCVfnzfzyiszySNaNzyD]{#2}}{}%
  \ifstrequal{#1}{f\_is\_SNaN\_D}{\hyperref[sailRISCVfnzfzyiszySNaNzyD]{#2}}{}%
  \ifstrequal{#1}{f_is_SNaN_S}{\hyperref[sailRISCVfnzfzyiszySNaNzyS]{#2}}{}%
  \ifstrequal{#1}{f\_is\_SNaN\_S}{\hyperref[sailRISCVfnzfzyiszySNaNzyS]{#2}}{}%
  \ifstrequal{#1}{f_is_neg_inf_D}{\hyperref[sailRISCVfnzfzyiszynegzyinfzyD]{#2}}{}%
  \ifstrequal{#1}{f\_is\_neg\_inf\_D}{\hyperref[sailRISCVfnzfzyiszynegzyinfzyD]{#2}}{}%
  \ifstrequal{#1}{f_is_neg_inf_S}{\hyperref[sailRISCVfnzfzyiszynegzyinfzyS]{#2}}{}%
  \ifstrequal{#1}{f\_is\_neg\_inf\_S}{\hyperref[sailRISCVfnzfzyiszynegzyinfzyS]{#2}}{}%
  \ifstrequal{#1}{f_is_neg_norm_D}{\hyperref[sailRISCVfnzfzyiszynegzynormzyD]{#2}}{}%
  \ifstrequal{#1}{f\_is\_neg\_norm\_D}{\hyperref[sailRISCVfnzfzyiszynegzynormzyD]{#2}}{}%
  \ifstrequal{#1}{f_is_neg_norm_S}{\hyperref[sailRISCVfnzfzyiszynegzynormzyS]{#2}}{}%
  \ifstrequal{#1}{f\_is\_neg\_norm\_S}{\hyperref[sailRISCVfnzfzyiszynegzynormzyS]{#2}}{}%
  \ifstrequal{#1}{f_is_neg_subnorm_D}{\hyperref[sailRISCVfnzfzyiszynegzysubnormzyD]{#2}}{}%
  \ifstrequal{#1}{f\_is\_neg\_subnorm\_D}{\hyperref[sailRISCVfnzfzyiszynegzysubnormzyD]{#2}}{}%
  \ifstrequal{#1}{f_is_neg_subnorm_S}{\hyperref[sailRISCVfnzfzyiszynegzysubnormzyS]{#2}}{}%
  \ifstrequal{#1}{f\_is\_neg\_subnorm\_S}{\hyperref[sailRISCVfnzfzyiszynegzysubnormzyS]{#2}}{}%
  \ifstrequal{#1}{f_is_neg_zero_D}{\hyperref[sailRISCVfnzfzyiszynegzyzzerozyD]{#2}}{}%
  \ifstrequal{#1}{f\_is\_neg\_zero\_D}{\hyperref[sailRISCVfnzfzyiszynegzyzzerozyD]{#2}}{}%
  \ifstrequal{#1}{f_is_neg_zero_S}{\hyperref[sailRISCVfnzfzyiszynegzyzzerozyS]{#2}}{}%
  \ifstrequal{#1}{f\_is\_neg\_zero\_S}{\hyperref[sailRISCVfnzfzyiszynegzyzzerozyS]{#2}}{}%
  \ifstrequal{#1}{f_is_pos_inf_D}{\hyperref[sailRISCVfnzfzyiszyposzyinfzyD]{#2}}{}%
  \ifstrequal{#1}{f\_is\_pos\_inf\_D}{\hyperref[sailRISCVfnzfzyiszyposzyinfzyD]{#2}}{}%
  \ifstrequal{#1}{f_is_pos_inf_S}{\hyperref[sailRISCVfnzfzyiszyposzyinfzyS]{#2}}{}%
  \ifstrequal{#1}{f\_is\_pos\_inf\_S}{\hyperref[sailRISCVfnzfzyiszyposzyinfzyS]{#2}}{}%
  \ifstrequal{#1}{f_is_pos_norm_D}{\hyperref[sailRISCVfnzfzyiszyposzynormzyD]{#2}}{}%
  \ifstrequal{#1}{f\_is\_pos\_norm\_D}{\hyperref[sailRISCVfnzfzyiszyposzynormzyD]{#2}}{}%
  \ifstrequal{#1}{f_is_pos_norm_S}{\hyperref[sailRISCVfnzfzyiszyposzynormzyS]{#2}}{}%
  \ifstrequal{#1}{f\_is\_pos\_norm\_S}{\hyperref[sailRISCVfnzfzyiszyposzynormzyS]{#2}}{}%
  \ifstrequal{#1}{f_is_pos_subnorm_D}{\hyperref[sailRISCVfnzfzyiszyposzysubnormzyD]{#2}}{}%
  \ifstrequal{#1}{f\_is\_pos\_subnorm\_D}{\hyperref[sailRISCVfnzfzyiszyposzysubnormzyD]{#2}}{}%
  \ifstrequal{#1}{f_is_pos_subnorm_S}{\hyperref[sailRISCVfnzfzyiszyposzysubnormzyS]{#2}}{}%
  \ifstrequal{#1}{f\_is\_pos\_subnorm\_S}{\hyperref[sailRISCVfnzfzyiszyposzysubnormzyS]{#2}}{}%
  \ifstrequal{#1}{f_is_pos_zero_D}{\hyperref[sailRISCVfnzfzyiszyposzyzzerozyD]{#2}}{}%
  \ifstrequal{#1}{f\_is\_pos\_zero\_D}{\hyperref[sailRISCVfnzfzyiszyposzyzzerozyD]{#2}}{}%
  \ifstrequal{#1}{f_is_pos_zero_S}{\hyperref[sailRISCVfnzfzyiszyposzyzzerozyS]{#2}}{}%
  \ifstrequal{#1}{f\_is\_pos\_zero\_S}{\hyperref[sailRISCVfnzfzyiszyposzyzzerozyS]{#2}}{}%
  \ifstrequal{#1}{f_madd_op_D_of_num}{\hyperref[sailRISCVfnzfzymaddzyopzyDzyofzynum]{#2}}{}%
  \ifstrequal{#1}{f\_madd\_op\_D\_of\_num}{\hyperref[sailRISCVfnzfzymaddzyopzyDzyofzynum]{#2}}{}%
  \ifstrequal{#1}{f_madd_op_H_of_num}{\hyperref[sailRISCVfnzfzymaddzyopzyHzyofzynum]{#2}}{}%
  \ifstrequal{#1}{f\_madd\_op\_H\_of\_num}{\hyperref[sailRISCVfnzfzymaddzyopzyHzyofzynum]{#2}}{}%
  \ifstrequal{#1}{f_madd_op_S_of_num}{\hyperref[sailRISCVfnzfzymaddzyopzySzyofzynum]{#2}}{}%
  \ifstrequal{#1}{f\_madd\_op\_S\_of\_num}{\hyperref[sailRISCVfnzfzymaddzyopzySzyofzynum]{#2}}{}%
  \ifstrequal{#1}{f_un_op_D_of_num}{\hyperref[sailRISCVfnzfzyunzyopzyDzyofzynum]{#2}}{}%
  \ifstrequal{#1}{f\_un\_op\_D\_of\_num}{\hyperref[sailRISCVfnzfzyunzyopzyDzyofzynum]{#2}}{}%
  \ifstrequal{#1}{f_un_op_H_of_num}{\hyperref[sailRISCVfnzfzyunzyopzyHzyofzynum]{#2}}{}%
  \ifstrequal{#1}{f\_un\_op\_H\_of\_num}{\hyperref[sailRISCVfnzfzyunzyopzyHzyofzynum]{#2}}{}%
  \ifstrequal{#1}{f_un_op_S_of_num}{\hyperref[sailRISCVfnzfzyunzyopzySzyofzynum]{#2}}{}%
  \ifstrequal{#1}{f\_un\_op\_S\_of\_num}{\hyperref[sailRISCVfnzfzyunzyopzySzyofzynum]{#2}}{}%
  \ifstrequal{#1}{f_un_rm_op_D_of_num}{\hyperref[sailRISCVfnzfzyunzyrmzyopzyDzyofzynum]{#2}}{}%
  \ifstrequal{#1}{f\_un\_rm\_op\_D\_of\_num}{\hyperref[sailRISCVfnzfzyunzyrmzyopzyDzyofzynum]{#2}}{}%
  \ifstrequal{#1}{f_un_rm_op_H_of_num}{\hyperref[sailRISCVfnzfzyunzyrmzyopzyHzyofzynum]{#2}}{}%
  \ifstrequal{#1}{f\_un\_rm\_op\_H\_of\_num}{\hyperref[sailRISCVfnzfzyunzyrmzyopzyHzyofzynum]{#2}}{}%
  \ifstrequal{#1}{f_un_rm_op_S_of_num}{\hyperref[sailRISCVfnzfzyunzyrmzyopzySzyofzynum]{#2}}{}%
  \ifstrequal{#1}{f\_un\_rm\_op\_S\_of\_num}{\hyperref[sailRISCVfnzfzyunzyrmzyopzySzyofzynum]{#2}}{}%
  \ifstrequal{#1}{fastRepCheck}{\hyperref[sailRISCVfnzfastRepCheck]{#2}}{}%
  \ifstrequal{#1}{fdiv_int}{\hyperref[sailRISCVfnzfdivzyint]{#2}}{}%
  \ifstrequal{#1}{fdiv\_int}{\hyperref[sailRISCVfnzfdivzyint]{#2}}{}%
  \ifstrequal{#1}{feq_quiet_D}{\hyperref[sailRISCVfnzfeqzyquietzyD]{#2}}{}%
  \ifstrequal{#1}{feq\_quiet\_D}{\hyperref[sailRISCVfnzfeqzyquietzyD]{#2}}{}%
  \ifstrequal{#1}{feq_quiet_S}{\hyperref[sailRISCVfnzfeqzyquietzyS]{#2}}{}%
  \ifstrequal{#1}{feq\_quiet\_S}{\hyperref[sailRISCVfnzfeqzyquietzyS]{#2}}{}%
  \ifstrequal{#1}{fetch}{\hyperref[sailRISCVfnzfetch]{#2}}{}%
  \ifstrequal{#1}{findPendingInterrupt}{\hyperref[sailRISCVfnzfindPendingInterrupt]{#2}}{}%
  \ifstrequal{#1}{fle_D}{\hyperref[sailRISCVfnzflezyD]{#2}}{}%
  \ifstrequal{#1}{fle\_D}{\hyperref[sailRISCVfnzflezyD]{#2}}{}%
  \ifstrequal{#1}{fle_S}{\hyperref[sailRISCVfnzflezyS]{#2}}{}%
  \ifstrequal{#1}{fle\_S}{\hyperref[sailRISCVfnzflezyS]{#2}}{}%
  \ifstrequal{#1}{flt_D}{\hyperref[sailRISCVfnzfltzyD]{#2}}{}%
  \ifstrequal{#1}{flt\_D}{\hyperref[sailRISCVfnzfltzyD]{#2}}{}%
  \ifstrequal{#1}{flt_S}{\hyperref[sailRISCVfnzfltzyS]{#2}}{}%
  \ifstrequal{#1}{flt\_S}{\hyperref[sailRISCVfnzfltzyS]{#2}}{}%
  \ifstrequal{#1}{flush_TLB}{\hyperref[sailRISCVfnzflushzyTLB]{#2}}{}%
  \ifstrequal{#1}{flush\_TLB}{\hyperref[sailRISCVfnzflushzyTLB]{#2}}{}%
  \ifstrequal{#1}{flush_TLB39}{\hyperref[sailRISCVfnzflushzyTLB39]{#2}}{}%
  \ifstrequal{#1}{flush\_TLB39}{\hyperref[sailRISCVfnzflushzyTLB39]{#2}}{}%
  \ifstrequal{#1}{flush_TLB48}{\hyperref[sailRISCVfnzflushzyTLB48]{#2}}{}%
  \ifstrequal{#1}{flush\_TLB48}{\hyperref[sailRISCVfnzflushzyTLB48]{#2}}{}%
  \ifstrequal{#1}{flush_TLB_Entry}{\hyperref[sailRISCVfnzflushzyTLBzyEntry]{#2}}{}%
  \ifstrequal{#1}{flush\_TLB\_Entry}{\hyperref[sailRISCVfnzflushzyTLBzyEntry]{#2}}{}%
  \ifstrequal{#1}{fmake_D}{\hyperref[sailRISCVfnzfmakezyD]{#2}}{}%
  \ifstrequal{#1}{fmake\_D}{\hyperref[sailRISCVfnzfmakezyD]{#2}}{}%
  \ifstrequal{#1}{fmake_S}{\hyperref[sailRISCVfnzfmakezyS]{#2}}{}%
  \ifstrequal{#1}{fmake\_S}{\hyperref[sailRISCVfnzfmakezyS]{#2}}{}%
  \ifstrequal{#1}{fmod_int}{\hyperref[sailRISCVfnzfmodzyint]{#2}}{}%
  \ifstrequal{#1}{fmod\_int}{\hyperref[sailRISCVfnzfmodzyint]{#2}}{}%
  \ifstrequal{#1}{fregval_from_freg}{\hyperref[sailRISCVfnzfregvalzyfromzyfreg]{#2}}{}%
  \ifstrequal{#1}{fregval\_from\_freg}{\hyperref[sailRISCVfnzfregvalzyfromzyfreg]{#2}}{}%
  \ifstrequal{#1}{fregval_into_freg}{\hyperref[sailRISCVfnzfregvalzyintozyfreg]{#2}}{}%
  \ifstrequal{#1}{fregval\_into\_freg}{\hyperref[sailRISCVfnzfregvalzyintozyfreg]{#2}}{}%
  \ifstrequal{#1}{fsplit_D}{\hyperref[sailRISCVfnzfsplitzyD]{#2}}{}%
  \ifstrequal{#1}{fsplit\_D}{\hyperref[sailRISCVfnzfsplitzyD]{#2}}{}%
  \ifstrequal{#1}{fsplit_S}{\hyperref[sailRISCVfnzfsplitzyS]{#2}}{}%
  \ifstrequal{#1}{fsplit\_S}{\hyperref[sailRISCVfnzfsplitzyS]{#2}}{}%
  \ifstrequal{#1}{getCapBase}{\hyperref[sailRISCVfnzgetCapBase]{#2}}{}%
  \ifstrequal{#1}{getCapBaseBits}{\hyperref[sailRISCVfnzgetCapBaseBits]{#2}}{}%
  \ifstrequal{#1}{getCapBounds}{\hyperref[sailRISCVfnzgetCapBounds]{#2}}{}%
  \ifstrequal{#1}{getCapBoundsBits}{\hyperref[sailRISCVfnzgetCapBoundsBits]{#2}}{}%
  \ifstrequal{#1}{getCapCursor}{\hyperref[sailRISCVfnzgetCapCursor]{#2}}{}%
  \ifstrequal{#1}{getCapFlags}{\hyperref[sailRISCVfnzgetCapFlags]{#2}}{}%
  \ifstrequal{#1}{getCapHardPerms}{\hyperref[sailRISCVfnzgetCapHardPerms]{#2}}{}%
  \ifstrequal{#1}{getCapLength}{\hyperref[sailRISCVfnzgetCapLength]{#2}}{}%
  \ifstrequal{#1}{getCapOffset}{\hyperref[sailRISCVfnzgetCapOffset]{#2}}{}%
  \ifstrequal{#1}{getCapOffsetBits}{\hyperref[sailRISCVfnzgetCapOffsetBits]{#2}}{}%
  \ifstrequal{#1}{getCapPerms}{\hyperref[sailRISCVfnzgetCapPerms]{#2}}{}%
  \ifstrequal{#1}{getCapTop}{\hyperref[sailRISCVfnzgetCapTop]{#2}}{}%
  \ifstrequal{#1}{getCapTopBits}{\hyperref[sailRISCVfnzgetCapTopBits]{#2}}{}%
  \ifstrequal{#1}{getPendingSet}{\hyperref[sailRISCVfnzgetPendingSet]{#2}}{}%
  \ifstrequal{#1}{getRepresentableAlignmentMask}{\hyperref[sailRISCVfnzgetRepresentableAlignmentMask]{#2}}{}%
  \ifstrequal{#1}{getRepresentableLength}{\hyperref[sailRISCVfnzgetRepresentableLength]{#2}}{}%
  \ifstrequal{#1}{get_arch_pc}{\hyperref[sailRISCVfnzgetzyarchzypc]{#2}}{}%
  \ifstrequal{#1}{get\_arch\_pc}{\hyperref[sailRISCVfnzgetzyarchzypc]{#2}}{}%
  \ifstrequal{#1}{get_cheri_mode_cap_addr}{\hyperref[sailRISCVfnzgetzycherizymodezycapzyaddr]{#2}}{}%
  \ifstrequal{#1}{get\_cheri\_mode\_cap\_addr}{\hyperref[sailRISCVfnzgetzycherizymodezycapzyaddr]{#2}}{}%
  \ifstrequal{#1}{get_config_print_instr}{\hyperref[sailRISCVfnzgetzyconfigzyprintzyinstr]{#2}}{}%
  \ifstrequal{#1}{get\_config\_print\_instr}{\hyperref[sailRISCVfnzgetzyconfigzyprintzyinstr]{#2}}{}%
  \ifstrequal{#1}{get_config_print_mem}{\hyperref[sailRISCVfnzgetzyconfigzyprintzymem]{#2}}{}%
  \ifstrequal{#1}{get\_config\_print\_mem}{\hyperref[sailRISCVfnzgetzyconfigzyprintzymem]{#2}}{}%
  \ifstrequal{#1}{get_config_print_platform}{\hyperref[sailRISCVfnzgetzyconfigzyprintzyplatform]{#2}}{}%
  \ifstrequal{#1}{get\_config\_print\_platform}{\hyperref[sailRISCVfnzgetzyconfigzyprintzyplatform]{#2}}{}%
  \ifstrequal{#1}{get_config_print_reg}{\hyperref[sailRISCVfnzgetzyconfigzyprintzyreg]{#2}}{}%
  \ifstrequal{#1}{get\_config\_print\_reg}{\hyperref[sailRISCVfnzgetzyconfigzyprintzyreg]{#2}}{}%
  \ifstrequal{#1}{get_mstatus_SXL}{\hyperref[sailRISCVfnzgetzymstatuszySXL]{#2}}{}%
  \ifstrequal{#1}{get\_mstatus\_SXL}{\hyperref[sailRISCVfnzgetzymstatuszySXL]{#2}}{}%
  \ifstrequal{#1}{get_mstatus_UXL}{\hyperref[sailRISCVfnzgetzymstatuszyUXL]{#2}}{}%
  \ifstrequal{#1}{get\_mstatus\_UXL}{\hyperref[sailRISCVfnzgetzymstatuszyUXL]{#2}}{}%
  \ifstrequal{#1}{get_mtvec}{\hyperref[sailRISCVfnzgetzymtvec]{#2}}{}%
  \ifstrequal{#1}{get\_mtvec}{\hyperref[sailRISCVfnzgetzymtvec]{#2}}{}%
  \ifstrequal{#1}{get_next_pc}{\hyperref[sailRISCVfnzgetzynextzypc]{#2}}{}%
  \ifstrequal{#1}{get\_next\_pc}{\hyperref[sailRISCVfnzgetzynextzypc]{#2}}{}%
  \ifstrequal{#1}{get_sstatus_UXL}{\hyperref[sailRISCVfnzgetzysstatuszyUXL]{#2}}{}%
  \ifstrequal{#1}{get\_sstatus\_UXL}{\hyperref[sailRISCVfnzgetzysstatuszyUXL]{#2}}{}%
  \ifstrequal{#1}{get_stvec}{\hyperref[sailRISCVfnzgetzystvec]{#2}}{}%
  \ifstrequal{#1}{get\_stvec}{\hyperref[sailRISCVfnzgetzystvec]{#2}}{}%
  \ifstrequal{#1}{get_utvec}{\hyperref[sailRISCVfnzgetzyutvec]{#2}}{}%
  \ifstrequal{#1}{get\_utvec}{\hyperref[sailRISCVfnzgetzyutvec]{#2}}{}%
  \ifstrequal{#1}{get_xret_target}{\hyperref[sailRISCVfnzgetzyxretzytarget]{#2}}{}%
  \ifstrequal{#1}{get\_xret\_target}{\hyperref[sailRISCVfnzgetzyxretzytarget]{#2}}{}%
  \ifstrequal{#1}{handle_cheri_cap_exception}{\hyperref[sailRISCVfnzhandlezycherizycapzyexception]{#2}}{}%
  \ifstrequal{#1}{handle\_cheri\_cap\_exception}{\hyperref[sailRISCVfnzhandlezycherizycapzyexception]{#2}}{}%
  \ifstrequal{#1}{handle_cheri_pcc_exception}{\hyperref[sailRISCVfnzhandlezycherizypcczyexception]{#2}}{}%
  \ifstrequal{#1}{handle\_cheri\_pcc\_exception}{\hyperref[sailRISCVfnzhandlezycherizypcczyexception]{#2}}{}%
  \ifstrequal{#1}{handle_cheri_reg_exception}{\hyperref[sailRISCVfnzhandlezycherizyregzyexception]{#2}}{}%
  \ifstrequal{#1}{handle\_cheri\_reg\_exception}{\hyperref[sailRISCVfnzhandlezycherizyregzyexception]{#2}}{}%
  \ifstrequal{#1}{handle_exception}{\hyperref[sailRISCVfnzhandlezyexception]{#2}}{}%
  \ifstrequal{#1}{handle\_exception}{\hyperref[sailRISCVfnzhandlezyexception]{#2}}{}%
  \ifstrequal{#1}{handle_illegal}{\hyperref[sailRISCVfnzhandlezyillegal]{#2}}{}%
  \ifstrequal{#1}{handle\_illegal}{\hyperref[sailRISCVfnzhandlezyillegal]{#2}}{}%
  \ifstrequal{#1}{handle_interrupt}{\hyperref[sailRISCVfnzhandlezyinterrupt]{#2}}{}%
  \ifstrequal{#1}{handle\_interrupt}{\hyperref[sailRISCVfnzhandlezyinterrupt]{#2}}{}%
  \ifstrequal{#1}{handle_load_cap_via_cap}{\hyperref[sailRISCVfnzhandlezyloadzycapzyviazycap]{#2}}{}%
  \ifstrequal{#1}{handle\_load\_cap\_via\_cap}{\hyperref[sailRISCVfnzhandlezyloadzycapzyviazycap]{#2}}{}%
  \ifstrequal{#1}{handle_load_data_via_cap}{\hyperref[sailRISCVfnzhandlezyloadzydatazyviazycap]{#2}}{}%
  \ifstrequal{#1}{handle\_load\_data\_via\_cap}{\hyperref[sailRISCVfnzhandlezyloadzydatazyviazycap]{#2}}{}%
  \ifstrequal{#1}{handle_loadres_cap_via_cap}{\hyperref[sailRISCVfnzhandlezyloadreszycapzyviazycap]{#2}}{}%
  \ifstrequal{#1}{handle\_loadres\_cap\_via\_cap}{\hyperref[sailRISCVfnzhandlezyloadreszycapzyviazycap]{#2}}{}%
  \ifstrequal{#1}{handle_loadres_data_via_cap}{\hyperref[sailRISCVfnzhandlezyloadreszydatazyviazycap]{#2}}{}%
  \ifstrequal{#1}{handle\_loadres\_data\_via\_cap}{\hyperref[sailRISCVfnzhandlezyloadreszydatazyviazycap]{#2}}{}%
  \ifstrequal{#1}{handle_mem_exception}{\hyperref[sailRISCVfnzhandlezymemzyexception]{#2}}{}%
  \ifstrequal{#1}{handle\_mem\_exception}{\hyperref[sailRISCVfnzhandlezymemzyexception]{#2}}{}%
  \ifstrequal{#1}{handle_store_cap_via_cap}{\hyperref[sailRISCVfnzhandlezystorezycapzyviazycap]{#2}}{}%
  \ifstrequal{#1}{handle\_store\_cap\_via\_cap}{\hyperref[sailRISCVfnzhandlezystorezycapzyviazycap]{#2}}{}%
  \ifstrequal{#1}{handle_store_cond_cap_via_cap}{\hyperref[sailRISCVfnzhandlezystorezycondzycapzyviazycap]{#2}}{}%
  \ifstrequal{#1}{handle\_store\_cond\_cap\_via\_cap}{\hyperref[sailRISCVfnzhandlezystorezycondzycapzyviazycap]{#2}}{}%
  \ifstrequal{#1}{handle_store_cond_data_via_cap}{\hyperref[sailRISCVfnzhandlezystorezycondzydatazyviazycap]{#2}}{}%
  \ifstrequal{#1}{handle\_store\_cond\_data\_via\_cap}{\hyperref[sailRISCVfnzhandlezystorezycondzydatazyviazycap]{#2}}{}%
  \ifstrequal{#1}{handle_store_data_via_cap}{\hyperref[sailRISCVfnzhandlezystorezydatazyviazycap]{#2}}{}%
  \ifstrequal{#1}{handle\_store\_data\_via\_cap}{\hyperref[sailRISCVfnzhandlezystorezydatazyviazycap]{#2}}{}%
  \ifstrequal{#1}{handle_trap_extension}{\hyperref[sailRISCVfnzhandlezytrapzyextension]{#2}}{}%
  \ifstrequal{#1}{handle\_trap\_extension}{\hyperref[sailRISCVfnzhandlezytrapzyextension]{#2}}{}%
  \ifstrequal{#1}{hasReservedOType}{\hyperref[sailRISCVfnzhasReservedOType]{#2}}{}%
  \ifstrequal{#1}{haveAtomics}{\hyperref[sailRISCVfnzhaveAtomics]{#2}}{}%
  \ifstrequal{#1}{haveDExt}{\hyperref[sailRISCVfnzhaveDExt]{#2}}{}%
  \ifstrequal{#1}{haveDoubleFPU}{\hyperref[sailRISCVfnzhaveDoubleFPU]{#2}}{}%
  \ifstrequal{#1}{haveFExt}{\hyperref[sailRISCVfnzhaveFExt]{#2}}{}%
  \ifstrequal{#1}{haveMulDiv}{\hyperref[sailRISCVfnzhaveMulDiv]{#2}}{}%
  \ifstrequal{#1}{haveNExt}{\hyperref[sailRISCVfnzhaveNExt]{#2}}{}%
  \ifstrequal{#1}{haveRVC}{\hyperref[sailRISCVfnzhaveRVC]{#2}}{}%
  \ifstrequal{#1}{haveSingleFPU}{\hyperref[sailRISCVfnzhaveSingleFPU]{#2}}{}%
  \ifstrequal{#1}{haveSupMode}{\hyperref[sailRISCVfnzhaveSupMode]{#2}}{}%
  \ifstrequal{#1}{haveUsrMode}{\hyperref[sailRISCVfnzhaveUsrMode]{#2}}{}%
  \ifstrequal{#1}{haveXcheri}{\hyperref[sailRISCVfnzhaveXcheri]{#2}}{}%
  \ifstrequal{#1}{haveZba}{\hyperref[sailRISCVfnzhaveZba]{#2}}{}%
  \ifstrequal{#1}{haveZbb}{\hyperref[sailRISCVfnzhaveZbb]{#2}}{}%
  \ifstrequal{#1}{haveZbc}{\hyperref[sailRISCVfnzhaveZbc]{#2}}{}%
  \ifstrequal{#1}{haveZbkb}{\hyperref[sailRISCVfnzhaveZbkb]{#2}}{}%
  \ifstrequal{#1}{haveZbkc}{\hyperref[sailRISCVfnzhaveZbkc]{#2}}{}%
  \ifstrequal{#1}{haveZbkx}{\hyperref[sailRISCVfnzhaveZbkx]{#2}}{}%
  \ifstrequal{#1}{haveZbs}{\hyperref[sailRISCVfnzhaveZbs]{#2}}{}%
  \ifstrequal{#1}{haveZdinx}{\hyperref[sailRISCVfnzhaveZdinx]{#2}}{}%
  \ifstrequal{#1}{haveZfh}{\hyperref[sailRISCVfnzhaveZfh]{#2}}{}%
  \ifstrequal{#1}{haveZfinx}{\hyperref[sailRISCVfnzhaveZfinx]{#2}}{}%
  \ifstrequal{#1}{haveZhinx}{\hyperref[sailRISCVfnzhaveZhinx]{#2}}{}%
  \ifstrequal{#1}{haveZknd}{\hyperref[sailRISCVfnzhaveZknd]{#2}}{}%
  \ifstrequal{#1}{haveZkne}{\hyperref[sailRISCVfnzhaveZkne]{#2}}{}%
  \ifstrequal{#1}{haveZknh}{\hyperref[sailRISCVfnzhaveZknh]{#2}}{}%
  \ifstrequal{#1}{haveZkr}{\hyperref[sailRISCVfnzhaveZkr]{#2}}{}%
  \ifstrequal{#1}{haveZksed}{\hyperref[sailRISCVfnzhaveZksed]{#2}}{}%
  \ifstrequal{#1}{haveZksh}{\hyperref[sailRISCVfnzhaveZksh]{#2}}{}%
  \ifstrequal{#1}{haveZmmul}{\hyperref[sailRISCVfnzhaveZmmul]{#2}}{}%
  \ifstrequal{#1}{hex_bits_10_backwards}{\hyperref[sailRISCVfnzhexzybitszy10zybackwards]{#2}}{}%
  \ifstrequal{#1}{hex\_bits\_10\_backwards}{\hyperref[sailRISCVfnzhexzybitszy10zybackwards]{#2}}{}%
  \ifstrequal{#1}{hex_bits_10_backwards_matches}{\hyperref[sailRISCVfnzhexzybitszy10zybackwardszymatches]{#2}}{}%
  \ifstrequal{#1}{hex\_bits\_10\_backwards\_matches}{\hyperref[sailRISCVfnzhexzybitszy10zybackwardszymatches]{#2}}{}%
  \ifstrequal{#1}{hex_bits_10_forwards_matches}{\hyperref[sailRISCVfnzhexzybitszy10zyforwardszymatches]{#2}}{}%
  \ifstrequal{#1}{hex\_bits\_10\_forwards\_matches}{\hyperref[sailRISCVfnzhexzybitszy10zyforwardszymatches]{#2}}{}%
  \ifstrequal{#1}{hex_bits_11_backwards}{\hyperref[sailRISCVfnzhexzybitszy11zybackwards]{#2}}{}%
  \ifstrequal{#1}{hex\_bits\_11\_backwards}{\hyperref[sailRISCVfnzhexzybitszy11zybackwards]{#2}}{}%
  \ifstrequal{#1}{hex_bits_11_backwards_matches}{\hyperref[sailRISCVfnzhexzybitszy11zybackwardszymatches]{#2}}{}%
  \ifstrequal{#1}{hex\_bits\_11\_backwards\_matches}{\hyperref[sailRISCVfnzhexzybitszy11zybackwardszymatches]{#2}}{}%
  \ifstrequal{#1}{hex_bits_11_forwards_matches}{\hyperref[sailRISCVfnzhexzybitszy11zyforwardszymatches]{#2}}{}%
  \ifstrequal{#1}{hex\_bits\_11\_forwards\_matches}{\hyperref[sailRISCVfnzhexzybitszy11zyforwardszymatches]{#2}}{}%
  \ifstrequal{#1}{hex_bits_12_backwards}{\hyperref[sailRISCVfnzhexzybitszy12zybackwards]{#2}}{}%
  \ifstrequal{#1}{hex\_bits\_12\_backwards}{\hyperref[sailRISCVfnzhexzybitszy12zybackwards]{#2}}{}%
  \ifstrequal{#1}{hex_bits_12_backwards_matches}{\hyperref[sailRISCVfnzhexzybitszy12zybackwardszymatches]{#2}}{}%
  \ifstrequal{#1}{hex\_bits\_12\_backwards\_matches}{\hyperref[sailRISCVfnzhexzybitszy12zybackwardszymatches]{#2}}{}%
  \ifstrequal{#1}{hex_bits_12_forwards_matches}{\hyperref[sailRISCVfnzhexzybitszy12zyforwardszymatches]{#2}}{}%
  \ifstrequal{#1}{hex\_bits\_12\_forwards\_matches}{\hyperref[sailRISCVfnzhexzybitszy12zyforwardszymatches]{#2}}{}%
  \ifstrequal{#1}{hex_bits_12_matches_prefix}{\hyperref[sailRISCVfnzhexzybitszy12zymatcheszyprefix]{#2}}{}%
  \ifstrequal{#1}{hex\_bits\_12\_matches\_prefix}{\hyperref[sailRISCVfnzhexzybitszy12zymatcheszyprefix]{#2}}{}%
  \ifstrequal{#1}{hex_bits_13_backwards}{\hyperref[sailRISCVfnzhexzybitszy13zybackwards]{#2}}{}%
  \ifstrequal{#1}{hex\_bits\_13\_backwards}{\hyperref[sailRISCVfnzhexzybitszy13zybackwards]{#2}}{}%
  \ifstrequal{#1}{hex_bits_13_backwards_matches}{\hyperref[sailRISCVfnzhexzybitszy13zybackwardszymatches]{#2}}{}%
  \ifstrequal{#1}{hex\_bits\_13\_backwards\_matches}{\hyperref[sailRISCVfnzhexzybitszy13zybackwardszymatches]{#2}}{}%
  \ifstrequal{#1}{hex_bits_13_forwards_matches}{\hyperref[sailRISCVfnzhexzybitszy13zyforwardszymatches]{#2}}{}%
  \ifstrequal{#1}{hex\_bits\_13\_forwards\_matches}{\hyperref[sailRISCVfnzhexzybitszy13zyforwardszymatches]{#2}}{}%
  \ifstrequal{#1}{hex_bits_14_backwards}{\hyperref[sailRISCVfnzhexzybitszy14zybackwards]{#2}}{}%
  \ifstrequal{#1}{hex\_bits\_14\_backwards}{\hyperref[sailRISCVfnzhexzybitszy14zybackwards]{#2}}{}%
  \ifstrequal{#1}{hex_bits_14_backwards_matches}{\hyperref[sailRISCVfnzhexzybitszy14zybackwardszymatches]{#2}}{}%
  \ifstrequal{#1}{hex\_bits\_14\_backwards\_matches}{\hyperref[sailRISCVfnzhexzybitszy14zybackwardszymatches]{#2}}{}%
  \ifstrequal{#1}{hex_bits_14_forwards_matches}{\hyperref[sailRISCVfnzhexzybitszy14zyforwardszymatches]{#2}}{}%
  \ifstrequal{#1}{hex\_bits\_14\_forwards\_matches}{\hyperref[sailRISCVfnzhexzybitszy14zyforwardszymatches]{#2}}{}%
  \ifstrequal{#1}{hex_bits_15_backwards}{\hyperref[sailRISCVfnzhexzybitszy15zybackwards]{#2}}{}%
  \ifstrequal{#1}{hex\_bits\_15\_backwards}{\hyperref[sailRISCVfnzhexzybitszy15zybackwards]{#2}}{}%
  \ifstrequal{#1}{hex_bits_15_backwards_matches}{\hyperref[sailRISCVfnzhexzybitszy15zybackwardszymatches]{#2}}{}%
  \ifstrequal{#1}{hex\_bits\_15\_backwards\_matches}{\hyperref[sailRISCVfnzhexzybitszy15zybackwardszymatches]{#2}}{}%
  \ifstrequal{#1}{hex_bits_15_forwards_matches}{\hyperref[sailRISCVfnzhexzybitszy15zyforwardszymatches]{#2}}{}%
  \ifstrequal{#1}{hex\_bits\_15\_forwards\_matches}{\hyperref[sailRISCVfnzhexzybitszy15zyforwardszymatches]{#2}}{}%
  \ifstrequal{#1}{hex_bits_16_backwards}{\hyperref[sailRISCVfnzhexzybitszy16zybackwards]{#2}}{}%
  \ifstrequal{#1}{hex\_bits\_16\_backwards}{\hyperref[sailRISCVfnzhexzybitszy16zybackwards]{#2}}{}%
  \ifstrequal{#1}{hex_bits_16_backwards_matches}{\hyperref[sailRISCVfnzhexzybitszy16zybackwardszymatches]{#2}}{}%
  \ifstrequal{#1}{hex\_bits\_16\_backwards\_matches}{\hyperref[sailRISCVfnzhexzybitszy16zybackwardszymatches]{#2}}{}%
  \ifstrequal{#1}{hex_bits_16_forwards_matches}{\hyperref[sailRISCVfnzhexzybitszy16zyforwardszymatches]{#2}}{}%
  \ifstrequal{#1}{hex\_bits\_16\_forwards\_matches}{\hyperref[sailRISCVfnzhexzybitszy16zyforwardszymatches]{#2}}{}%
  \ifstrequal{#1}{hex_bits_17_backwards}{\hyperref[sailRISCVfnzhexzybitszy17zybackwards]{#2}}{}%
  \ifstrequal{#1}{hex\_bits\_17\_backwards}{\hyperref[sailRISCVfnzhexzybitszy17zybackwards]{#2}}{}%
  \ifstrequal{#1}{hex_bits_17_backwards_matches}{\hyperref[sailRISCVfnzhexzybitszy17zybackwardszymatches]{#2}}{}%
  \ifstrequal{#1}{hex\_bits\_17\_backwards\_matches}{\hyperref[sailRISCVfnzhexzybitszy17zybackwardszymatches]{#2}}{}%
  \ifstrequal{#1}{hex_bits_17_forwards_matches}{\hyperref[sailRISCVfnzhexzybitszy17zyforwardszymatches]{#2}}{}%
  \ifstrequal{#1}{hex\_bits\_17\_forwards\_matches}{\hyperref[sailRISCVfnzhexzybitszy17zyforwardszymatches]{#2}}{}%
  \ifstrequal{#1}{hex_bits_18_backwards}{\hyperref[sailRISCVfnzhexzybitszy18zybackwards]{#2}}{}%
  \ifstrequal{#1}{hex\_bits\_18\_backwards}{\hyperref[sailRISCVfnzhexzybitszy18zybackwards]{#2}}{}%
  \ifstrequal{#1}{hex_bits_18_backwards_matches}{\hyperref[sailRISCVfnzhexzybitszy18zybackwardszymatches]{#2}}{}%
  \ifstrequal{#1}{hex\_bits\_18\_backwards\_matches}{\hyperref[sailRISCVfnzhexzybitszy18zybackwardszymatches]{#2}}{}%
  \ifstrequal{#1}{hex_bits_18_forwards_matches}{\hyperref[sailRISCVfnzhexzybitszy18zyforwardszymatches]{#2}}{}%
  \ifstrequal{#1}{hex\_bits\_18\_forwards\_matches}{\hyperref[sailRISCVfnzhexzybitszy18zyforwardszymatches]{#2}}{}%
  \ifstrequal{#1}{hex_bits_19_backwards}{\hyperref[sailRISCVfnzhexzybitszy19zybackwards]{#2}}{}%
  \ifstrequal{#1}{hex\_bits\_19\_backwards}{\hyperref[sailRISCVfnzhexzybitszy19zybackwards]{#2}}{}%
  \ifstrequal{#1}{hex_bits_19_backwards_matches}{\hyperref[sailRISCVfnzhexzybitszy19zybackwardszymatches]{#2}}{}%
  \ifstrequal{#1}{hex\_bits\_19\_backwards\_matches}{\hyperref[sailRISCVfnzhexzybitszy19zybackwardszymatches]{#2}}{}%
  \ifstrequal{#1}{hex_bits_19_forwards_matches}{\hyperref[sailRISCVfnzhexzybitszy19zyforwardszymatches]{#2}}{}%
  \ifstrequal{#1}{hex\_bits\_19\_forwards\_matches}{\hyperref[sailRISCVfnzhexzybitszy19zyforwardszymatches]{#2}}{}%
  \ifstrequal{#1}{hex_bits_1_backwards}{\hyperref[sailRISCVfnzhexzybitszy1zybackwards]{#2}}{}%
  \ifstrequal{#1}{hex\_bits\_1\_backwards}{\hyperref[sailRISCVfnzhexzybitszy1zybackwards]{#2}}{}%
  \ifstrequal{#1}{hex_bits_1_backwards_matches}{\hyperref[sailRISCVfnzhexzybitszy1zybackwardszymatches]{#2}}{}%
  \ifstrequal{#1}{hex\_bits\_1\_backwards\_matches}{\hyperref[sailRISCVfnzhexzybitszy1zybackwardszymatches]{#2}}{}%
  \ifstrequal{#1}{hex_bits_1_forwards_matches}{\hyperref[sailRISCVfnzhexzybitszy1zyforwardszymatches]{#2}}{}%
  \ifstrequal{#1}{hex\_bits\_1\_forwards\_matches}{\hyperref[sailRISCVfnzhexzybitszy1zyforwardszymatches]{#2}}{}%
  \ifstrequal{#1}{hex_bits_20_backwards}{\hyperref[sailRISCVfnzhexzybitszy20zybackwards]{#2}}{}%
  \ifstrequal{#1}{hex\_bits\_20\_backwards}{\hyperref[sailRISCVfnzhexzybitszy20zybackwards]{#2}}{}%
  \ifstrequal{#1}{hex_bits_20_backwards_matches}{\hyperref[sailRISCVfnzhexzybitszy20zybackwardszymatches]{#2}}{}%
  \ifstrequal{#1}{hex\_bits\_20\_backwards\_matches}{\hyperref[sailRISCVfnzhexzybitszy20zybackwardszymatches]{#2}}{}%
  \ifstrequal{#1}{hex_bits_20_forwards_matches}{\hyperref[sailRISCVfnzhexzybitszy20zyforwardszymatches]{#2}}{}%
  \ifstrequal{#1}{hex\_bits\_20\_forwards\_matches}{\hyperref[sailRISCVfnzhexzybitszy20zyforwardszymatches]{#2}}{}%
  \ifstrequal{#1}{hex_bits_21_backwards}{\hyperref[sailRISCVfnzhexzybitszy21zybackwards]{#2}}{}%
  \ifstrequal{#1}{hex\_bits\_21\_backwards}{\hyperref[sailRISCVfnzhexzybitszy21zybackwards]{#2}}{}%
  \ifstrequal{#1}{hex_bits_21_backwards_matches}{\hyperref[sailRISCVfnzhexzybitszy21zybackwardszymatches]{#2}}{}%
  \ifstrequal{#1}{hex\_bits\_21\_backwards\_matches}{\hyperref[sailRISCVfnzhexzybitszy21zybackwardszymatches]{#2}}{}%
  \ifstrequal{#1}{hex_bits_21_forwards_matches}{\hyperref[sailRISCVfnzhexzybitszy21zyforwardszymatches]{#2}}{}%
  \ifstrequal{#1}{hex\_bits\_21\_forwards\_matches}{\hyperref[sailRISCVfnzhexzybitszy21zyforwardszymatches]{#2}}{}%
  \ifstrequal{#1}{hex_bits_22_backwards}{\hyperref[sailRISCVfnzhexzybitszy22zybackwards]{#2}}{}%
  \ifstrequal{#1}{hex\_bits\_22\_backwards}{\hyperref[sailRISCVfnzhexzybitszy22zybackwards]{#2}}{}%
  \ifstrequal{#1}{hex_bits_22_backwards_matches}{\hyperref[sailRISCVfnzhexzybitszy22zybackwardszymatches]{#2}}{}%
  \ifstrequal{#1}{hex\_bits\_22\_backwards\_matches}{\hyperref[sailRISCVfnzhexzybitszy22zybackwardszymatches]{#2}}{}%
  \ifstrequal{#1}{hex_bits_22_forwards_matches}{\hyperref[sailRISCVfnzhexzybitszy22zyforwardszymatches]{#2}}{}%
  \ifstrequal{#1}{hex\_bits\_22\_forwards\_matches}{\hyperref[sailRISCVfnzhexzybitszy22zyforwardszymatches]{#2}}{}%
  \ifstrequal{#1}{hex_bits_23_backwards}{\hyperref[sailRISCVfnzhexzybitszy23zybackwards]{#2}}{}%
  \ifstrequal{#1}{hex\_bits\_23\_backwards}{\hyperref[sailRISCVfnzhexzybitszy23zybackwards]{#2}}{}%
  \ifstrequal{#1}{hex_bits_23_backwards_matches}{\hyperref[sailRISCVfnzhexzybitszy23zybackwardszymatches]{#2}}{}%
  \ifstrequal{#1}{hex\_bits\_23\_backwards\_matches}{\hyperref[sailRISCVfnzhexzybitszy23zybackwardszymatches]{#2}}{}%
  \ifstrequal{#1}{hex_bits_23_forwards_matches}{\hyperref[sailRISCVfnzhexzybitszy23zyforwardszymatches]{#2}}{}%
  \ifstrequal{#1}{hex\_bits\_23\_forwards\_matches}{\hyperref[sailRISCVfnzhexzybitszy23zyforwardszymatches]{#2}}{}%
  \ifstrequal{#1}{hex_bits_24_backwards}{\hyperref[sailRISCVfnzhexzybitszy24zybackwards]{#2}}{}%
  \ifstrequal{#1}{hex\_bits\_24\_backwards}{\hyperref[sailRISCVfnzhexzybitszy24zybackwards]{#2}}{}%
  \ifstrequal{#1}{hex_bits_24_backwards_matches}{\hyperref[sailRISCVfnzhexzybitszy24zybackwardszymatches]{#2}}{}%
  \ifstrequal{#1}{hex\_bits\_24\_backwards\_matches}{\hyperref[sailRISCVfnzhexzybitszy24zybackwardszymatches]{#2}}{}%
  \ifstrequal{#1}{hex_bits_24_forwards_matches}{\hyperref[sailRISCVfnzhexzybitszy24zyforwardszymatches]{#2}}{}%
  \ifstrequal{#1}{hex\_bits\_24\_forwards\_matches}{\hyperref[sailRISCVfnzhexzybitszy24zyforwardszymatches]{#2}}{}%
  \ifstrequal{#1}{hex_bits_25_backwards}{\hyperref[sailRISCVfnzhexzybitszy25zybackwards]{#2}}{}%
  \ifstrequal{#1}{hex\_bits\_25\_backwards}{\hyperref[sailRISCVfnzhexzybitszy25zybackwards]{#2}}{}%
  \ifstrequal{#1}{hex_bits_25_backwards_matches}{\hyperref[sailRISCVfnzhexzybitszy25zybackwardszymatches]{#2}}{}%
  \ifstrequal{#1}{hex\_bits\_25\_backwards\_matches}{\hyperref[sailRISCVfnzhexzybitszy25zybackwardszymatches]{#2}}{}%
  \ifstrequal{#1}{hex_bits_25_forwards_matches}{\hyperref[sailRISCVfnzhexzybitszy25zyforwardszymatches]{#2}}{}%
  \ifstrequal{#1}{hex\_bits\_25\_forwards\_matches}{\hyperref[sailRISCVfnzhexzybitszy25zyforwardszymatches]{#2}}{}%
  \ifstrequal{#1}{hex_bits_26_backwards}{\hyperref[sailRISCVfnzhexzybitszy26zybackwards]{#2}}{}%
  \ifstrequal{#1}{hex\_bits\_26\_backwards}{\hyperref[sailRISCVfnzhexzybitszy26zybackwards]{#2}}{}%
  \ifstrequal{#1}{hex_bits_26_backwards_matches}{\hyperref[sailRISCVfnzhexzybitszy26zybackwardszymatches]{#2}}{}%
  \ifstrequal{#1}{hex\_bits\_26\_backwards\_matches}{\hyperref[sailRISCVfnzhexzybitszy26zybackwardszymatches]{#2}}{}%
  \ifstrequal{#1}{hex_bits_26_forwards_matches}{\hyperref[sailRISCVfnzhexzybitszy26zyforwardszymatches]{#2}}{}%
  \ifstrequal{#1}{hex\_bits\_26\_forwards\_matches}{\hyperref[sailRISCVfnzhexzybitszy26zyforwardszymatches]{#2}}{}%
  \ifstrequal{#1}{hex_bits_27_backwards}{\hyperref[sailRISCVfnzhexzybitszy27zybackwards]{#2}}{}%
  \ifstrequal{#1}{hex\_bits\_27\_backwards}{\hyperref[sailRISCVfnzhexzybitszy27zybackwards]{#2}}{}%
  \ifstrequal{#1}{hex_bits_27_backwards_matches}{\hyperref[sailRISCVfnzhexzybitszy27zybackwardszymatches]{#2}}{}%
  \ifstrequal{#1}{hex\_bits\_27\_backwards\_matches}{\hyperref[sailRISCVfnzhexzybitszy27zybackwardszymatches]{#2}}{}%
  \ifstrequal{#1}{hex_bits_27_forwards_matches}{\hyperref[sailRISCVfnzhexzybitszy27zyforwardszymatches]{#2}}{}%
  \ifstrequal{#1}{hex\_bits\_27\_forwards\_matches}{\hyperref[sailRISCVfnzhexzybitszy27zyforwardszymatches]{#2}}{}%
  \ifstrequal{#1}{hex_bits_28_backwards}{\hyperref[sailRISCVfnzhexzybitszy28zybackwards]{#2}}{}%
  \ifstrequal{#1}{hex\_bits\_28\_backwards}{\hyperref[sailRISCVfnzhexzybitszy28zybackwards]{#2}}{}%
  \ifstrequal{#1}{hex_bits_28_backwards_matches}{\hyperref[sailRISCVfnzhexzybitszy28zybackwardszymatches]{#2}}{}%
  \ifstrequal{#1}{hex\_bits\_28\_backwards\_matches}{\hyperref[sailRISCVfnzhexzybitszy28zybackwardszymatches]{#2}}{}%
  \ifstrequal{#1}{hex_bits_28_forwards_matches}{\hyperref[sailRISCVfnzhexzybitszy28zyforwardszymatches]{#2}}{}%
  \ifstrequal{#1}{hex\_bits\_28\_forwards\_matches}{\hyperref[sailRISCVfnzhexzybitszy28zyforwardszymatches]{#2}}{}%
  \ifstrequal{#1}{hex_bits_29_backwards}{\hyperref[sailRISCVfnzhexzybitszy29zybackwards]{#2}}{}%
  \ifstrequal{#1}{hex\_bits\_29\_backwards}{\hyperref[sailRISCVfnzhexzybitszy29zybackwards]{#2}}{}%
  \ifstrequal{#1}{hex_bits_29_backwards_matches}{\hyperref[sailRISCVfnzhexzybitszy29zybackwardszymatches]{#2}}{}%
  \ifstrequal{#1}{hex\_bits\_29\_backwards\_matches}{\hyperref[sailRISCVfnzhexzybitszy29zybackwardszymatches]{#2}}{}%
  \ifstrequal{#1}{hex_bits_29_forwards_matches}{\hyperref[sailRISCVfnzhexzybitszy29zyforwardszymatches]{#2}}{}%
  \ifstrequal{#1}{hex\_bits\_29\_forwards\_matches}{\hyperref[sailRISCVfnzhexzybitszy29zyforwardszymatches]{#2}}{}%
  \ifstrequal{#1}{hex_bits_2_backwards}{\hyperref[sailRISCVfnzhexzybitszy2zybackwards]{#2}}{}%
  \ifstrequal{#1}{hex\_bits\_2\_backwards}{\hyperref[sailRISCVfnzhexzybitszy2zybackwards]{#2}}{}%
  \ifstrequal{#1}{hex_bits_2_backwards_matches}{\hyperref[sailRISCVfnzhexzybitszy2zybackwardszymatches]{#2}}{}%
  \ifstrequal{#1}{hex\_bits\_2\_backwards\_matches}{\hyperref[sailRISCVfnzhexzybitszy2zybackwardszymatches]{#2}}{}%
  \ifstrequal{#1}{hex_bits_2_forwards_matches}{\hyperref[sailRISCVfnzhexzybitszy2zyforwardszymatches]{#2}}{}%
  \ifstrequal{#1}{hex\_bits\_2\_forwards\_matches}{\hyperref[sailRISCVfnzhexzybitszy2zyforwardszymatches]{#2}}{}%
  \ifstrequal{#1}{hex_bits_30_backwards}{\hyperref[sailRISCVfnzhexzybitszy30zybackwards]{#2}}{}%
  \ifstrequal{#1}{hex\_bits\_30\_backwards}{\hyperref[sailRISCVfnzhexzybitszy30zybackwards]{#2}}{}%
  \ifstrequal{#1}{hex_bits_30_backwards_matches}{\hyperref[sailRISCVfnzhexzybitszy30zybackwardszymatches]{#2}}{}%
  \ifstrequal{#1}{hex\_bits\_30\_backwards\_matches}{\hyperref[sailRISCVfnzhexzybitszy30zybackwardszymatches]{#2}}{}%
  \ifstrequal{#1}{hex_bits_30_forwards_matches}{\hyperref[sailRISCVfnzhexzybitszy30zyforwardszymatches]{#2}}{}%
  \ifstrequal{#1}{hex\_bits\_30\_forwards\_matches}{\hyperref[sailRISCVfnzhexzybitszy30zyforwardszymatches]{#2}}{}%
  \ifstrequal{#1}{hex_bits_31_backwards}{\hyperref[sailRISCVfnzhexzybitszy31zybackwards]{#2}}{}%
  \ifstrequal{#1}{hex\_bits\_31\_backwards}{\hyperref[sailRISCVfnzhexzybitszy31zybackwards]{#2}}{}%
  \ifstrequal{#1}{hex_bits_31_backwards_matches}{\hyperref[sailRISCVfnzhexzybitszy31zybackwardszymatches]{#2}}{}%
  \ifstrequal{#1}{hex\_bits\_31\_backwards\_matches}{\hyperref[sailRISCVfnzhexzybitszy31zybackwardszymatches]{#2}}{}%
  \ifstrequal{#1}{hex_bits_31_forwards_matches}{\hyperref[sailRISCVfnzhexzybitszy31zyforwardszymatches]{#2}}{}%
  \ifstrequal{#1}{hex\_bits\_31\_forwards\_matches}{\hyperref[sailRISCVfnzhexzybitszy31zyforwardszymatches]{#2}}{}%
  \ifstrequal{#1}{hex_bits_32_backwards}{\hyperref[sailRISCVfnzhexzybitszy32zybackwards]{#2}}{}%
  \ifstrequal{#1}{hex\_bits\_32\_backwards}{\hyperref[sailRISCVfnzhexzybitszy32zybackwards]{#2}}{}%
  \ifstrequal{#1}{hex_bits_32_backwards_matches}{\hyperref[sailRISCVfnzhexzybitszy32zybackwardszymatches]{#2}}{}%
  \ifstrequal{#1}{hex\_bits\_32\_backwards\_matches}{\hyperref[sailRISCVfnzhexzybitszy32zybackwardszymatches]{#2}}{}%
  \ifstrequal{#1}{hex_bits_32_forwards_matches}{\hyperref[sailRISCVfnzhexzybitszy32zyforwardszymatches]{#2}}{}%
  \ifstrequal{#1}{hex\_bits\_32\_forwards\_matches}{\hyperref[sailRISCVfnzhexzybitszy32zyforwardszymatches]{#2}}{}%
  \ifstrequal{#1}{hex_bits_33_backwards}{\hyperref[sailRISCVfnzhexzybitszy33zybackwards]{#2}}{}%
  \ifstrequal{#1}{hex\_bits\_33\_backwards}{\hyperref[sailRISCVfnzhexzybitszy33zybackwards]{#2}}{}%
  \ifstrequal{#1}{hex_bits_33_backwards_matches}{\hyperref[sailRISCVfnzhexzybitszy33zybackwardszymatches]{#2}}{}%
  \ifstrequal{#1}{hex\_bits\_33\_backwards\_matches}{\hyperref[sailRISCVfnzhexzybitszy33zybackwardszymatches]{#2}}{}%
  \ifstrequal{#1}{hex_bits_33_forwards_matches}{\hyperref[sailRISCVfnzhexzybitszy33zyforwardszymatches]{#2}}{}%
  \ifstrequal{#1}{hex\_bits\_33\_forwards\_matches}{\hyperref[sailRISCVfnzhexzybitszy33zyforwardszymatches]{#2}}{}%
  \ifstrequal{#1}{hex_bits_3_backwards}{\hyperref[sailRISCVfnzhexzybitszy3zybackwards]{#2}}{}%
  \ifstrequal{#1}{hex\_bits\_3\_backwards}{\hyperref[sailRISCVfnzhexzybitszy3zybackwards]{#2}}{}%
  \ifstrequal{#1}{hex_bits_3_backwards_matches}{\hyperref[sailRISCVfnzhexzybitszy3zybackwardszymatches]{#2}}{}%
  \ifstrequal{#1}{hex\_bits\_3\_backwards\_matches}{\hyperref[sailRISCVfnzhexzybitszy3zybackwardszymatches]{#2}}{}%
  \ifstrequal{#1}{hex_bits_3_forwards_matches}{\hyperref[sailRISCVfnzhexzybitszy3zyforwardszymatches]{#2}}{}%
  \ifstrequal{#1}{hex\_bits\_3\_forwards\_matches}{\hyperref[sailRISCVfnzhexzybitszy3zyforwardszymatches]{#2}}{}%
  \ifstrequal{#1}{hex_bits_48_backwards}{\hyperref[sailRISCVfnzhexzybitszy48zybackwards]{#2}}{}%
  \ifstrequal{#1}{hex\_bits\_48\_backwards}{\hyperref[sailRISCVfnzhexzybitszy48zybackwards]{#2}}{}%
  \ifstrequal{#1}{hex_bits_48_backwards_matches}{\hyperref[sailRISCVfnzhexzybitszy48zybackwardszymatches]{#2}}{}%
  \ifstrequal{#1}{hex\_bits\_48\_backwards\_matches}{\hyperref[sailRISCVfnzhexzybitszy48zybackwardszymatches]{#2}}{}%
  \ifstrequal{#1}{hex_bits_48_forwards_matches}{\hyperref[sailRISCVfnzhexzybitszy48zyforwardszymatches]{#2}}{}%
  \ifstrequal{#1}{hex\_bits\_48\_forwards\_matches}{\hyperref[sailRISCVfnzhexzybitszy48zyforwardszymatches]{#2}}{}%
  \ifstrequal{#1}{hex_bits_4_backwards}{\hyperref[sailRISCVfnzhexzybitszy4zybackwards]{#2}}{}%
  \ifstrequal{#1}{hex\_bits\_4\_backwards}{\hyperref[sailRISCVfnzhexzybitszy4zybackwards]{#2}}{}%
  \ifstrequal{#1}{hex_bits_4_backwards_matches}{\hyperref[sailRISCVfnzhexzybitszy4zybackwardszymatches]{#2}}{}%
  \ifstrequal{#1}{hex\_bits\_4\_backwards\_matches}{\hyperref[sailRISCVfnzhexzybitszy4zybackwardszymatches]{#2}}{}%
  \ifstrequal{#1}{hex_bits_4_forwards_matches}{\hyperref[sailRISCVfnzhexzybitszy4zyforwardszymatches]{#2}}{}%
  \ifstrequal{#1}{hex\_bits\_4\_forwards\_matches}{\hyperref[sailRISCVfnzhexzybitszy4zyforwardszymatches]{#2}}{}%
  \ifstrequal{#1}{hex_bits_5_backwards}{\hyperref[sailRISCVfnzhexzybitszy5zybackwards]{#2}}{}%
  \ifstrequal{#1}{hex\_bits\_5\_backwards}{\hyperref[sailRISCVfnzhexzybitszy5zybackwards]{#2}}{}%
  \ifstrequal{#1}{hex_bits_5_backwards_matches}{\hyperref[sailRISCVfnzhexzybitszy5zybackwardszymatches]{#2}}{}%
  \ifstrequal{#1}{hex\_bits\_5\_backwards\_matches}{\hyperref[sailRISCVfnzhexzybitszy5zybackwardszymatches]{#2}}{}%
  \ifstrequal{#1}{hex_bits_5_forwards_matches}{\hyperref[sailRISCVfnzhexzybitszy5zyforwardszymatches]{#2}}{}%
  \ifstrequal{#1}{hex\_bits\_5\_forwards\_matches}{\hyperref[sailRISCVfnzhexzybitszy5zyforwardszymatches]{#2}}{}%
  \ifstrequal{#1}{hex_bits_64_backwards}{\hyperref[sailRISCVfnzhexzybitszy64zybackwards]{#2}}{}%
  \ifstrequal{#1}{hex\_bits\_64\_backwards}{\hyperref[sailRISCVfnzhexzybitszy64zybackwards]{#2}}{}%
  \ifstrequal{#1}{hex_bits_64_backwards_matches}{\hyperref[sailRISCVfnzhexzybitszy64zybackwardszymatches]{#2}}{}%
  \ifstrequal{#1}{hex\_bits\_64\_backwards\_matches}{\hyperref[sailRISCVfnzhexzybitszy64zybackwardszymatches]{#2}}{}%
  \ifstrequal{#1}{hex_bits_64_forwards_matches}{\hyperref[sailRISCVfnzhexzybitszy64zyforwardszymatches]{#2}}{}%
  \ifstrequal{#1}{hex\_bits\_64\_forwards\_matches}{\hyperref[sailRISCVfnzhexzybitszy64zyforwardszymatches]{#2}}{}%
  \ifstrequal{#1}{hex_bits_6_backwards}{\hyperref[sailRISCVfnzhexzybitszy6zybackwards]{#2}}{}%
  \ifstrequal{#1}{hex\_bits\_6\_backwards}{\hyperref[sailRISCVfnzhexzybitszy6zybackwards]{#2}}{}%
  \ifstrequal{#1}{hex_bits_6_backwards_matches}{\hyperref[sailRISCVfnzhexzybitszy6zybackwardszymatches]{#2}}{}%
  \ifstrequal{#1}{hex\_bits\_6\_backwards\_matches}{\hyperref[sailRISCVfnzhexzybitszy6zybackwardszymatches]{#2}}{}%
  \ifstrequal{#1}{hex_bits_6_forwards_matches}{\hyperref[sailRISCVfnzhexzybitszy6zyforwardszymatches]{#2}}{}%
  \ifstrequal{#1}{hex\_bits\_6\_forwards\_matches}{\hyperref[sailRISCVfnzhexzybitszy6zyforwardszymatches]{#2}}{}%
  \ifstrequal{#1}{hex_bits_7_backwards}{\hyperref[sailRISCVfnzhexzybitszy7zybackwards]{#2}}{}%
  \ifstrequal{#1}{hex\_bits\_7\_backwards}{\hyperref[sailRISCVfnzhexzybitszy7zybackwards]{#2}}{}%
  \ifstrequal{#1}{hex_bits_7_backwards_matches}{\hyperref[sailRISCVfnzhexzybitszy7zybackwardszymatches]{#2}}{}%
  \ifstrequal{#1}{hex\_bits\_7\_backwards\_matches}{\hyperref[sailRISCVfnzhexzybitszy7zybackwardszymatches]{#2}}{}%
  \ifstrequal{#1}{hex_bits_7_forwards_matches}{\hyperref[sailRISCVfnzhexzybitszy7zyforwardszymatches]{#2}}{}%
  \ifstrequal{#1}{hex\_bits\_7\_forwards\_matches}{\hyperref[sailRISCVfnzhexzybitszy7zyforwardszymatches]{#2}}{}%
  \ifstrequal{#1}{hex_bits_8_backwards}{\hyperref[sailRISCVfnzhexzybitszy8zybackwards]{#2}}{}%
  \ifstrequal{#1}{hex\_bits\_8\_backwards}{\hyperref[sailRISCVfnzhexzybitszy8zybackwards]{#2}}{}%
  \ifstrequal{#1}{hex_bits_8_backwards_matches}{\hyperref[sailRISCVfnzhexzybitszy8zybackwardszymatches]{#2}}{}%
  \ifstrequal{#1}{hex\_bits\_8\_backwards\_matches}{\hyperref[sailRISCVfnzhexzybitszy8zybackwardszymatches]{#2}}{}%
  \ifstrequal{#1}{hex_bits_8_forwards_matches}{\hyperref[sailRISCVfnzhexzybitszy8zyforwardszymatches]{#2}}{}%
  \ifstrequal{#1}{hex\_bits\_8\_forwards\_matches}{\hyperref[sailRISCVfnzhexzybitszy8zyforwardszymatches]{#2}}{}%
  \ifstrequal{#1}{hex_bits_9_backwards}{\hyperref[sailRISCVfnzhexzybitszy9zybackwards]{#2}}{}%
  \ifstrequal{#1}{hex\_bits\_9\_backwards}{\hyperref[sailRISCVfnzhexzybitszy9zybackwards]{#2}}{}%
  \ifstrequal{#1}{hex_bits_9_backwards_matches}{\hyperref[sailRISCVfnzhexzybitszy9zybackwardszymatches]{#2}}{}%
  \ifstrequal{#1}{hex\_bits\_9\_backwards\_matches}{\hyperref[sailRISCVfnzhexzybitszy9zybackwardszymatches]{#2}}{}%
  \ifstrequal{#1}{hex_bits_9_forwards_matches}{\hyperref[sailRISCVfnzhexzybitszy9zyforwardszymatches]{#2}}{}%
  \ifstrequal{#1}{hex\_bits\_9\_forwards\_matches}{\hyperref[sailRISCVfnzhexzybitszy9zyforwardszymatches]{#2}}{}%
  \ifstrequal{#1}{htif_load}{\hyperref[sailRISCVfnzhtifzyload]{#2}}{}%
  \ifstrequal{#1}{htif\_load}{\hyperref[sailRISCVfnzhtifzyload]{#2}}{}%
  \ifstrequal{#1}{htif_store}{\hyperref[sailRISCVfnzhtifzystore]{#2}}{}%
  \ifstrequal{#1}{htif\_store}{\hyperref[sailRISCVfnzhtifzystore]{#2}}{}%
  \ifstrequal{#1}{htif_tick}{\hyperref[sailRISCVfnzhtifzytick]{#2}}{}%
  \ifstrequal{#1}{htif\_tick}{\hyperref[sailRISCVfnzhtifzytick]{#2}}{}%
  \ifstrequal{#1}{in32BitMode}{\hyperref[sailRISCVfnzin32BitMode]{#2}}{}%
  \ifstrequal{#1}{inCapBounds}{\hyperref[sailRISCVfnzinCapBounds]{#2}}{}%
  \ifstrequal{#1}{incCapOffset}{\hyperref[sailRISCVfnzincCapOffset]{#2}}{}%
  \ifstrequal{#1}{init_base_regs}{\hyperref[sailRISCVfnzinitzybasezyregs]{#2}}{}%
  \ifstrequal{#1}{init\_base\_regs}{\hyperref[sailRISCVfnzinitzybasezyregs]{#2}}{}%
  \ifstrequal{#1}{init_fdext_regs}{\hyperref[sailRISCVfnzinitzyfdextzyregs]{#2}}{}%
  \ifstrequal{#1}{init\_fdext\_regs}{\hyperref[sailRISCVfnzinitzyfdextzyregs]{#2}}{}%
  \ifstrequal{#1}{init_model}{\hyperref[sailRISCVfnzinitzymodel]{#2}}{}%
  \ifstrequal{#1}{init\_model}{\hyperref[sailRISCVfnzinitzymodel]{#2}}{}%
  \ifstrequal{#1}{init_platform}{\hyperref[sailRISCVfnzinitzyplatform]{#2}}{}%
  \ifstrequal{#1}{init\_platform}{\hyperref[sailRISCVfnzinitzyplatform]{#2}}{}%
  \ifstrequal{#1}{init_pmp}{\hyperref[sailRISCVfnzinitzypmp]{#2}}{}%
  \ifstrequal{#1}{init\_pmp}{\hyperref[sailRISCVfnzinitzypmp]{#2}}{}%
  \ifstrequal{#1}{init_sys}{\hyperref[sailRISCVfnzinitzysys]{#2}}{}%
  \ifstrequal{#1}{init\_sys}{\hyperref[sailRISCVfnzinitzysys]{#2}}{}%
  \ifstrequal{#1}{init_vmem}{\hyperref[sailRISCVfnzinitzyvmem]{#2}}{}%
  \ifstrequal{#1}{init\_vmem}{\hyperref[sailRISCVfnzinitzyvmem]{#2}}{}%
  \ifstrequal{#1}{init_vmem_sv39}{\hyperref[sailRISCVfnzinitzyvmemzysv39]{#2}}{}%
  \ifstrequal{#1}{init\_vmem\_sv39}{\hyperref[sailRISCVfnzinitzyvmemzysv39]{#2}}{}%
  \ifstrequal{#1}{init_vmem_sv48}{\hyperref[sailRISCVfnzinitzyvmemzysv48]{#2}}{}%
  \ifstrequal{#1}{init\_vmem\_sv48}{\hyperref[sailRISCVfnzinitzyvmemzysv48]{#2}}{}%
  \ifstrequal{#1}{initial_analysis}{\hyperref[sailRISCVfnzinitialzyanalysis]{#2}}{}%
  \ifstrequal{#1}{initial\_analysis}{\hyperref[sailRISCVfnzinitialzyanalysis]{#2}}{}%
  \ifstrequal{#1}{int_to_cap}{\hyperref[sailRISCVfnzintzytozycap]{#2}}{}%
  \ifstrequal{#1}{int\_to\_cap}{\hyperref[sailRISCVfnzintzytozycap]{#2}}{}%
  \ifstrequal{#1}{internal_error}{\hyperref[sailRISCVfnzinternalzyerror]{#2}}{}%
  \ifstrequal{#1}{internal\_error}{\hyperref[sailRISCVfnzinternalzyerror]{#2}}{}%
  \ifstrequal{#1}{interruptType_to_bits}{\hyperref[sailRISCVfnzinterruptTypezytozybits]{#2}}{}%
  \ifstrequal{#1}{interruptType\_to\_bits}{\hyperref[sailRISCVfnzinterruptTypezytozybits]{#2}}{}%
  \ifstrequal{#1}{iop_of_num}{\hyperref[sailRISCVfnziopzyofzynum]{#2}}{}%
  \ifstrequal{#1}{iop\_of\_num}{\hyperref[sailRISCVfnziopzyofzynum]{#2}}{}%
  \ifstrequal{#1}{isCapSealed}{\hyperref[sailRISCVfnzisCapSealed]{#2}}{}%
  \ifstrequal{#1}{isInvalidPTE}{\hyperref[sailRISCVfnzisInvalidPTE]{#2}}{}%
  \ifstrequal{#1}{isPTEPtr}{\hyperref[sailRISCVfnzisPTEPtr]{#2}}{}%
  \ifstrequal{#1}{isRVC}{\hyperref[sailRISCVfnzisRVC]{#2}}{}%
  \ifstrequal{#1}{isValidSv39Addr}{\hyperref[sailRISCVfnzisValidSv39Addr]{#2}}{}%
  \ifstrequal{#1}{isValidSv48Addr}{\hyperref[sailRISCVfnzisValidSv48Addr]{#2}}{}%
  \ifstrequal{#1}{is_CSR_defined}{\hyperref[sailRISCVfnziszyCSRzydefined]{#2}}{}%
  \ifstrequal{#1}{is\_CSR\_defined}{\hyperref[sailRISCVfnziszyCSRzydefined]{#2}}{}%
  \ifstrequal{#1}{is_aligned_addr}{\hyperref[sailRISCVfnziszyalignedzyaddr]{#2}}{}%
  \ifstrequal{#1}{is\_aligned\_addr}{\hyperref[sailRISCVfnziszyalignedzyaddr]{#2}}{}%
  \ifstrequal{#1}{is_none}{\hyperref[sailRISCVfnziszynone]{#2}}{}%
  \ifstrequal{#1}{is\_none}{\hyperref[sailRISCVfnziszynone]{#2}}{}%
  \ifstrequal{#1}{is_some}{\hyperref[sailRISCVfnziszysome]{#2}}{}%
  \ifstrequal{#1}{is\_some}{\hyperref[sailRISCVfnziszysome]{#2}}{}%
  \ifstrequal{#1}{legalize_ccsr}{\hyperref[sailRISCVfnzlegalizzezyccsr]{#2}}{}%
  \ifstrequal{#1}{legalize\_ccsr}{\hyperref[sailRISCVfnzlegalizzezyccsr]{#2}}{}%
  \ifstrequal{#1}{legalize_epcc}{\hyperref[sailRISCVfnzlegalizzezyepcc]{#2}}{}%
  \ifstrequal{#1}{legalize\_epcc}{\hyperref[sailRISCVfnzlegalizzezyepcc]{#2}}{}%
  \ifstrequal{#1}{legalize_mcounteren}{\hyperref[sailRISCVfnzlegalizzezymcounteren]{#2}}{}%
  \ifstrequal{#1}{legalize\_mcounteren}{\hyperref[sailRISCVfnzlegalizzezymcounteren]{#2}}{}%
  \ifstrequal{#1}{legalize_mcountinhibit}{\hyperref[sailRISCVfnzlegalizzezymcountinhibit]{#2}}{}%
  \ifstrequal{#1}{legalize\_mcountinhibit}{\hyperref[sailRISCVfnzlegalizzezymcountinhibit]{#2}}{}%
  \ifstrequal{#1}{legalize_medeleg}{\hyperref[sailRISCVfnzlegalizzezymedeleg]{#2}}{}%
  \ifstrequal{#1}{legalize\_medeleg}{\hyperref[sailRISCVfnzlegalizzezymedeleg]{#2}}{}%
  \ifstrequal{#1}{legalize_mideleg}{\hyperref[sailRISCVfnzlegalizzezymideleg]{#2}}{}%
  \ifstrequal{#1}{legalize\_mideleg}{\hyperref[sailRISCVfnzlegalizzezymideleg]{#2}}{}%
  \ifstrequal{#1}{legalize_mie}{\hyperref[sailRISCVfnzlegalizzezymie]{#2}}{}%
  \ifstrequal{#1}{legalize\_mie}{\hyperref[sailRISCVfnzlegalizzezymie]{#2}}{}%
  \ifstrequal{#1}{legalize_mip}{\hyperref[sailRISCVfnzlegalizzezymip]{#2}}{}%
  \ifstrequal{#1}{legalize\_mip}{\hyperref[sailRISCVfnzlegalizzezymip]{#2}}{}%
  \ifstrequal{#1}{legalize_misa}{\hyperref[sailRISCVfnzlegalizzezymisa]{#2}}{}%
  \ifstrequal{#1}{legalize\_misa}{\hyperref[sailRISCVfnzlegalizzezymisa]{#2}}{}%
  \ifstrequal{#1}{legalize_mstatus}{\hyperref[sailRISCVfnzlegalizzezymstatus]{#2}}{}%
  \ifstrequal{#1}{legalize\_mstatus}{\hyperref[sailRISCVfnzlegalizzezymstatus]{#2}}{}%
  \ifstrequal{#1}{legalize_satp}{\hyperref[sailRISCVfnzlegalizzezysatp]{#2}}{}%
  \ifstrequal{#1}{legalize\_satp}{\hyperref[sailRISCVfnzlegalizzezysatp]{#2}}{}%
  \ifstrequal{#1}{legalize_satp32}{\hyperref[sailRISCVfnzlegalizzezysatp32]{#2}}{}%
  \ifstrequal{#1}{legalize\_satp32}{\hyperref[sailRISCVfnzlegalizzezysatp32]{#2}}{}%
  \ifstrequal{#1}{legalize_satp64}{\hyperref[sailRISCVfnzlegalizzezysatp64]{#2}}{}%
  \ifstrequal{#1}{legalize\_satp64}{\hyperref[sailRISCVfnzlegalizzezysatp64]{#2}}{}%
  \ifstrequal{#1}{legalize_scounteren}{\hyperref[sailRISCVfnzlegalizzezyscounteren]{#2}}{}%
  \ifstrequal{#1}{legalize\_scounteren}{\hyperref[sailRISCVfnzlegalizzezyscounteren]{#2}}{}%
  \ifstrequal{#1}{legalize_sedeleg}{\hyperref[sailRISCVfnzlegalizzezysedeleg]{#2}}{}%
  \ifstrequal{#1}{legalize\_sedeleg}{\hyperref[sailRISCVfnzlegalizzezysedeleg]{#2}}{}%
  \ifstrequal{#1}{legalize_sie}{\hyperref[sailRISCVfnzlegalizzezysie]{#2}}{}%
  \ifstrequal{#1}{legalize\_sie}{\hyperref[sailRISCVfnzlegalizzezysie]{#2}}{}%
  \ifstrequal{#1}{legalize_sip}{\hyperref[sailRISCVfnzlegalizzezysip]{#2}}{}%
  \ifstrequal{#1}{legalize\_sip}{\hyperref[sailRISCVfnzlegalizzezysip]{#2}}{}%
  \ifstrequal{#1}{legalize_sstatus}{\hyperref[sailRISCVfnzlegalizzezysstatus]{#2}}{}%
  \ifstrequal{#1}{legalize\_sstatus}{\hyperref[sailRISCVfnzlegalizzezysstatus]{#2}}{}%
  \ifstrequal{#1}{legalize_tcc}{\hyperref[sailRISCVfnzlegalizzezytcc]{#2}}{}%
  \ifstrequal{#1}{legalize\_tcc}{\hyperref[sailRISCVfnzlegalizzezytcc]{#2}}{}%
  \ifstrequal{#1}{legalize_tvec}{\hyperref[sailRISCVfnzlegalizzezytvec]{#2}}{}%
  \ifstrequal{#1}{legalize\_tvec}{\hyperref[sailRISCVfnzlegalizzezytvec]{#2}}{}%
  \ifstrequal{#1}{legalize_uie}{\hyperref[sailRISCVfnzlegalizzezyuie]{#2}}{}%
  \ifstrequal{#1}{legalize\_uie}{\hyperref[sailRISCVfnzlegalizzezyuie]{#2}}{}%
  \ifstrequal{#1}{legalize_uip}{\hyperref[sailRISCVfnzlegalizzezyuip]{#2}}{}%
  \ifstrequal{#1}{legalize\_uip}{\hyperref[sailRISCVfnzlegalizzezyuip]{#2}}{}%
  \ifstrequal{#1}{legalize_ustatus}{\hyperref[sailRISCVfnzlegalizzezyustatus]{#2}}{}%
  \ifstrequal{#1}{legalize\_ustatus}{\hyperref[sailRISCVfnzlegalizzezyustatus]{#2}}{}%
  \ifstrequal{#1}{legalize_xepc}{\hyperref[sailRISCVfnzlegalizzezyxepc]{#2}}{}%
  \ifstrequal{#1}{legalize\_xepc}{\hyperref[sailRISCVfnzlegalizzezyxepc]{#2}}{}%
  \ifstrequal{#1}{lift_sie}{\hyperref[sailRISCVfnzliftzysie]{#2}}{}%
  \ifstrequal{#1}{lift\_sie}{\hyperref[sailRISCVfnzliftzysie]{#2}}{}%
  \ifstrequal{#1}{lift_sip}{\hyperref[sailRISCVfnzliftzysip]{#2}}{}%
  \ifstrequal{#1}{lift\_sip}{\hyperref[sailRISCVfnzliftzysip]{#2}}{}%
  \ifstrequal{#1}{lift_sstatus}{\hyperref[sailRISCVfnzliftzysstatus]{#2}}{}%
  \ifstrequal{#1}{lift\_sstatus}{\hyperref[sailRISCVfnzliftzysstatus]{#2}}{}%
  \ifstrequal{#1}{lift_uie}{\hyperref[sailRISCVfnzliftzyuie]{#2}}{}%
  \ifstrequal{#1}{lift\_uie}{\hyperref[sailRISCVfnzliftzyuie]{#2}}{}%
  \ifstrequal{#1}{lift_uip}{\hyperref[sailRISCVfnzliftzyuip]{#2}}{}%
  \ifstrequal{#1}{lift\_uip}{\hyperref[sailRISCVfnzliftzyuip]{#2}}{}%
  \ifstrequal{#1}{lift_ustatus}{\hyperref[sailRISCVfnzliftzyustatus]{#2}}{}%
  \ifstrequal{#1}{lift\_ustatus}{\hyperref[sailRISCVfnzliftzyustatus]{#2}}{}%
  \ifstrequal{#1}{lookup_TLB39}{\hyperref[sailRISCVfnzlookupzyTLB39]{#2}}{}%
  \ifstrequal{#1}{lookup\_TLB39}{\hyperref[sailRISCVfnzlookupzyTLB39]{#2}}{}%
  \ifstrequal{#1}{lookup_TLB48}{\hyperref[sailRISCVfnzlookupzyTLB48]{#2}}{}%
  \ifstrequal{#1}{lookup\_TLB48}{\hyperref[sailRISCVfnzlookupzyTLB48]{#2}}{}%
  \ifstrequal{#1}{loop}{\hyperref[sailRISCVfnzloop]{#2}}{}%
  \ifstrequal{#1}{lower_mie}{\hyperref[sailRISCVfnzlowerzymie]{#2}}{}%
  \ifstrequal{#1}{lower\_mie}{\hyperref[sailRISCVfnzlowerzymie]{#2}}{}%
  \ifstrequal{#1}{lower_mip}{\hyperref[sailRISCVfnzlowerzymip]{#2}}{}%
  \ifstrequal{#1}{lower\_mip}{\hyperref[sailRISCVfnzlowerzymip]{#2}}{}%
  \ifstrequal{#1}{lower_mstatus}{\hyperref[sailRISCVfnzlowerzymstatus]{#2}}{}%
  \ifstrequal{#1}{lower\_mstatus}{\hyperref[sailRISCVfnzlowerzymstatus]{#2}}{}%
  \ifstrequal{#1}{lower_sie}{\hyperref[sailRISCVfnzlowerzysie]{#2}}{}%
  \ifstrequal{#1}{lower\_sie}{\hyperref[sailRISCVfnzlowerzysie]{#2}}{}%
  \ifstrequal{#1}{lower_sip}{\hyperref[sailRISCVfnzlowerzysip]{#2}}{}%
  \ifstrequal{#1}{lower\_sip}{\hyperref[sailRISCVfnzlowerzysip]{#2}}{}%
  \ifstrequal{#1}{lower_sstatus}{\hyperref[sailRISCVfnzlowerzysstatus]{#2}}{}%
  \ifstrequal{#1}{lower\_sstatus}{\hyperref[sailRISCVfnzlowerzysstatus]{#2}}{}%
  \ifstrequal{#1}{lrsc_width_str}{\hyperref[sailRISCVfnzlrsczywidthzystr]{#2}}{}%
  \ifstrequal{#1}{lrsc\_width\_str}{\hyperref[sailRISCVfnzlrsczywidthzystr]{#2}}{}%
  \ifstrequal{#1}{make_TLB_Entry}{\hyperref[sailRISCVfnzmakezyTLBzyEntry]{#2}}{}%
  \ifstrequal{#1}{make\_TLB\_Entry}{\hyperref[sailRISCVfnzmakezyTLBzyEntry]{#2}}{}%
  \ifstrequal{#1}{match_TLB_Entry}{\hyperref[sailRISCVfnzmatchzyTLBzyEntry]{#2}}{}%
  \ifstrequal{#1}{match\_TLB\_Entry}{\hyperref[sailRISCVfnzmatchzyTLBzyEntry]{#2}}{}%
  \ifstrequal{#1}{memBitsToCapability}{\hyperref[sailRISCVfnzmemBitsToCapability]{#2}}{}%
  \ifstrequal{#1}{mem_read}{\hyperref[sailRISCVfnzmemzyread]{#2}}{}%
  \ifstrequal{#1}{mem\_read}{\hyperref[sailRISCVfnzmemzyread]{#2}}{}%
  \ifstrequal{#1}{mem_read_cap}{\hyperref[sailRISCVfnzmemzyreadzycap]{#2}}{}%
  \ifstrequal{#1}{mem\_read\_cap}{\hyperref[sailRISCVfnzmemzyreadzycap]{#2}}{}%
  \ifstrequal{#1}{mem_read_meta}{\hyperref[sailRISCVfnzmemzyreadzymeta]{#2}}{}%
  \ifstrequal{#1}{mem\_read\_meta}{\hyperref[sailRISCVfnzmemzyreadzymeta]{#2}}{}%
  \ifstrequal{#1}{mem_read_priv}{\hyperref[sailRISCVfnzmemzyreadzypriv]{#2}}{}%
  \ifstrequal{#1}{mem\_read\_priv}{\hyperref[sailRISCVfnzmemzyreadzypriv]{#2}}{}%
  \ifstrequal{#1}{mem_read_priv_meta}{\hyperref[sailRISCVfnzmemzyreadzyprivzymeta]{#2}}{}%
  \ifstrequal{#1}{mem\_read\_priv\_meta}{\hyperref[sailRISCVfnzmemzyreadzyprivzymeta]{#2}}{}%
  \ifstrequal{#1}{mem_write_cap}{\hyperref[sailRISCVfnzmemzywritezycap]{#2}}{}%
  \ifstrequal{#1}{mem\_write\_cap}{\hyperref[sailRISCVfnzmemzywritezycap]{#2}}{}%
  \ifstrequal{#1}{mem_write_ea}{\hyperref[sailRISCVfnzmemzywritezyea]{#2}}{}%
  \ifstrequal{#1}{mem\_write\_ea}{\hyperref[sailRISCVfnzmemzywritezyea]{#2}}{}%
  \ifstrequal{#1}{mem_write_ea_cap}{\hyperref[sailRISCVfnzmemzywritezyeazycap]{#2}}{}%
  \ifstrequal{#1}{mem\_write\_ea\_cap}{\hyperref[sailRISCVfnzmemzywritezyeazycap]{#2}}{}%
  \ifstrequal{#1}{mem_write_value}{\hyperref[sailRISCVfnzmemzywritezyvalue]{#2}}{}%
  \ifstrequal{#1}{mem\_write\_value}{\hyperref[sailRISCVfnzmemzywritezyvalue]{#2}}{}%
  \ifstrequal{#1}{mem_write_value_meta}{\hyperref[sailRISCVfnzmemzywritezyvaluezymeta]{#2}}{}%
  \ifstrequal{#1}{mem\_write\_value\_meta}{\hyperref[sailRISCVfnzmemzywritezyvaluezymeta]{#2}}{}%
  \ifstrequal{#1}{mem_write_value_priv}{\hyperref[sailRISCVfnzmemzywritezyvaluezypriv]{#2}}{}%
  \ifstrequal{#1}{mem\_write\_value\_priv}{\hyperref[sailRISCVfnzmemzywritezyvaluezypriv]{#2}}{}%
  \ifstrequal{#1}{mem_write_value_priv_meta}{\hyperref[sailRISCVfnzmemzywritezyvaluezyprivzymeta]{#2}}{}%
  \ifstrequal{#1}{mem\_write\_value\_priv\_meta}{\hyperref[sailRISCVfnzmemzywritezyvaluezyprivzymeta]{#2}}{}%
  \ifstrequal{#1}{min_instruction_bytes}{\hyperref[sailRISCVfnzminzyinstructionzybytes]{#2}}{}%
  \ifstrequal{#1}{min\_instruction\_bytes}{\hyperref[sailRISCVfnzminzyinstructionzybytes]{#2}}{}%
  \ifstrequal{#1}{mmio_read}{\hyperref[sailRISCVfnzmmiozyread]{#2}}{}%
  \ifstrequal{#1}{mmio\_read}{\hyperref[sailRISCVfnzmmiozyread]{#2}}{}%
  \ifstrequal{#1}{mmio_write}{\hyperref[sailRISCVfnzmmiozywrite]{#2}}{}%
  \ifstrequal{#1}{mmio\_write}{\hyperref[sailRISCVfnzmmiozywrite]{#2}}{}%
  \ifstrequal{#1}{n_leading_spaces}{\hyperref[sailRISCVfnznzyleadingzyspaces]{#2}}{}%
  \ifstrequal{#1}{n\_leading\_spaces}{\hyperref[sailRISCVfnznzyleadingzyspaces]{#2}}{}%
  \ifstrequal{#1}{nan_box_H}{\hyperref[sailRISCVfnznanzyboxzyH]{#2}}{}%
  \ifstrequal{#1}{nan\_box\_H}{\hyperref[sailRISCVfnznanzyboxzyH]{#2}}{}%
  \ifstrequal{#1}{nan_box_S}{\hyperref[sailRISCVfnznanzyboxzyS]{#2}}{}%
  \ifstrequal{#1}{nan\_box\_S}{\hyperref[sailRISCVfnznanzyboxzyS]{#2}}{}%
  \ifstrequal{#1}{nan_unbox_H}{\hyperref[sailRISCVfnznanzyunboxzyH]{#2}}{}%
  \ifstrequal{#1}{nan\_unbox\_H}{\hyperref[sailRISCVfnznanzyunboxzyH]{#2}}{}%
  \ifstrequal{#1}{nan_unbox_S}{\hyperref[sailRISCVfnznanzyunboxzyS]{#2}}{}%
  \ifstrequal{#1}{nan\_unbox\_S}{\hyperref[sailRISCVfnznanzyunboxzyS]{#2}}{}%
  \ifstrequal{#1}{negate_D}{\hyperref[sailRISCVfnznegatezyD]{#2}}{}%
  \ifstrequal{#1}{negate\_D}{\hyperref[sailRISCVfnznegatezyD]{#2}}{}%
  \ifstrequal{#1}{negate_S}{\hyperref[sailRISCVfnznegatezyS]{#2}}{}%
  \ifstrequal{#1}{negate\_S}{\hyperref[sailRISCVfnznegatezyS]{#2}}{}%
  \ifstrequal{#1}{neq_anything}{\hyperref[sailRISCVfnzneqzyanything]{#2}}{}%
  \ifstrequal{#1}{neq\_anything}{\hyperref[sailRISCVfnzneqzyanything]{#2}}{}%
  \ifstrequal{#1}{neq_bits}{\hyperref[sailRISCVfnzneqzybits]{#2}}{}%
  \ifstrequal{#1}{neq\_bits}{\hyperref[sailRISCVfnzneqzybits]{#2}}{}%
  \ifstrequal{#1}{neq_bool}{\hyperref[sailRISCVfnzneqzybool]{#2}}{}%
  \ifstrequal{#1}{neq\_bool}{\hyperref[sailRISCVfnzneqzybool]{#2}}{}%
  \ifstrequal{#1}{neq_int}{\hyperref[sailRISCVfnzneqzyint]{#2}}{}%
  \ifstrequal{#1}{neq\_int}{\hyperref[sailRISCVfnzneqzyint]{#2}}{}%
  \ifstrequal{#1}{neq_vec}{\hyperref[sailRISCVfnzneqzyvec]{#2}}{}%
  \ifstrequal{#1}{neq\_vec}{\hyperref[sailRISCVfnzneqzyvec]{#2}}{}%
  \ifstrequal{#1}{not_bit}{\hyperref[sailRISCVfnznotzybit]{#2}}{}%
  \ifstrequal{#1}{not\_bit}{\hyperref[sailRISCVfnznotzybit]{#2}}{}%
  \ifstrequal{#1}{not_implemented}{\hyperref[sailRISCVfnznotzyimplemented]{#2}}{}%
  \ifstrequal{#1}{not\_implemented}{\hyperref[sailRISCVfnznotzyimplemented]{#2}}{}%
  \ifstrequal{#1}{num_of_Architecture}{\hyperref[sailRISCVfnznumzyofzyArchitecture]{#2}}{}%
  \ifstrequal{#1}{num\_of\_Architecture}{\hyperref[sailRISCVfnznumzyofzyArchitecture]{#2}}{}%
  \ifstrequal{#1}{num_of_CPtrCmpOp}{\hyperref[sailRISCVfnznumzyofzyCPtrCmpOp]{#2}}{}%
  \ifstrequal{#1}{num\_of\_CPtrCmpOp}{\hyperref[sailRISCVfnznumzyofzyCPtrCmpOp]{#2}}{}%
  \ifstrequal{#1}{num_of_CapEx}{\hyperref[sailRISCVfnznumzyofzyCapEx]{#2}}{}%
  \ifstrequal{#1}{num\_of\_CapEx}{\hyperref[sailRISCVfnznumzyofzyCapEx]{#2}}{}%
  \ifstrequal{#1}{num_of_ClearRegSet}{\hyperref[sailRISCVfnznumzyofzyClearRegSet]{#2}}{}%
  \ifstrequal{#1}{num\_of\_ClearRegSet}{\hyperref[sailRISCVfnznumzyofzyClearRegSet]{#2}}{}%
  \ifstrequal{#1}{num_of_ExceptionType}{\hyperref[sailRISCVfnznumzyofzyExceptionType]{#2}}{}%
  \ifstrequal{#1}{num\_of\_ExceptionType}{\hyperref[sailRISCVfnznumzyofzyExceptionType]{#2}}{}%
  \ifstrequal{#1}{num_of_ExtStatus}{\hyperref[sailRISCVfnznumzyofzyExtStatus]{#2}}{}%
  \ifstrequal{#1}{num\_of\_ExtStatus}{\hyperref[sailRISCVfnznumzyofzyExtStatus]{#2}}{}%
  \ifstrequal{#1}{num_of_InterruptType}{\hyperref[sailRISCVfnznumzyofzyInterruptType]{#2}}{}%
  \ifstrequal{#1}{num\_of\_InterruptType}{\hyperref[sailRISCVfnznumzyofzyInterruptType]{#2}}{}%
  \ifstrequal{#1}{num_of_PmpAddrMatchType}{\hyperref[sailRISCVfnznumzyofzyPmpAddrMatchType]{#2}}{}%
  \ifstrequal{#1}{num\_of\_PmpAddrMatchType}{\hyperref[sailRISCVfnznumzyofzyPmpAddrMatchType]{#2}}{}%
  \ifstrequal{#1}{num_of_Privilege}{\hyperref[sailRISCVfnznumzyofzyPrivilege]{#2}}{}%
  \ifstrequal{#1}{num\_of\_Privilege}{\hyperref[sailRISCVfnznumzyofzyPrivilege]{#2}}{}%
  \ifstrequal{#1}{num_of_Retired}{\hyperref[sailRISCVfnznumzyofzyRetired]{#2}}{}%
  \ifstrequal{#1}{num\_of\_Retired}{\hyperref[sailRISCVfnznumzyofzyRetired]{#2}}{}%
  \ifstrequal{#1}{num_of_SATPMode}{\hyperref[sailRISCVfnznumzyofzySATPMode]{#2}}{}%
  \ifstrequal{#1}{num\_of\_SATPMode}{\hyperref[sailRISCVfnznumzyofzySATPMode]{#2}}{}%
  \ifstrequal{#1}{num_of_TrapVectorMode}{\hyperref[sailRISCVfnznumzyofzyTrapVectorMode]{#2}}{}%
  \ifstrequal{#1}{num\_of\_TrapVectorMode}{\hyperref[sailRISCVfnznumzyofzyTrapVectorMode]{#2}}{}%
  \ifstrequal{#1}{num_of_a64_barrier_domain}{\hyperref[sailRISCVfnznumzyofzya64zybarrierzydomain]{#2}}{}%
  \ifstrequal{#1}{num\_of\_a64\_barrier\_domain}{\hyperref[sailRISCVfnznumzyofzya64zybarrierzydomain]{#2}}{}%
  \ifstrequal{#1}{num_of_a64_barrier_type}{\hyperref[sailRISCVfnznumzyofzya64zybarrierzytype]{#2}}{}%
  \ifstrequal{#1}{num\_of\_a64\_barrier\_type}{\hyperref[sailRISCVfnznumzyofzya64zybarrierzytype]{#2}}{}%
  \ifstrequal{#1}{num_of_amoop}{\hyperref[sailRISCVfnznumzyofzyamoop]{#2}}{}%
  \ifstrequal{#1}{num\_of\_amoop}{\hyperref[sailRISCVfnznumzyofzyamoop]{#2}}{}%
  \ifstrequal{#1}{num_of_biop_zbs}{\hyperref[sailRISCVfnznumzyofzybiopzyzzbs]{#2}}{}%
  \ifstrequal{#1}{num\_of\_biop\_zbs}{\hyperref[sailRISCVfnznumzyofzybiopzyzzbs]{#2}}{}%
  \ifstrequal{#1}{num_of_bop}{\hyperref[sailRISCVfnznumzyofzybop]{#2}}{}%
  \ifstrequal{#1}{num\_of\_bop}{\hyperref[sailRISCVfnznumzyofzybop]{#2}}{}%
  \ifstrequal{#1}{num_of_brop_zba}{\hyperref[sailRISCVfnznumzyofzybropzyzzba]{#2}}{}%
  \ifstrequal{#1}{num\_of\_brop\_zba}{\hyperref[sailRISCVfnznumzyofzybropzyzzba]{#2}}{}%
  \ifstrequal{#1}{num_of_brop_zbb}{\hyperref[sailRISCVfnznumzyofzybropzyzzbb]{#2}}{}%
  \ifstrequal{#1}{num\_of\_brop\_zbb}{\hyperref[sailRISCVfnznumzyofzybropzyzzbb]{#2}}{}%
  \ifstrequal{#1}{num_of_brop_zbkb}{\hyperref[sailRISCVfnznumzyofzybropzyzzbkb]{#2}}{}%
  \ifstrequal{#1}{num\_of\_brop\_zbkb}{\hyperref[sailRISCVfnznumzyofzybropzyzzbkb]{#2}}{}%
  \ifstrequal{#1}{num_of_brop_zbs}{\hyperref[sailRISCVfnznumzyofzybropzyzzbs]{#2}}{}%
  \ifstrequal{#1}{num\_of\_brop\_zbs}{\hyperref[sailRISCVfnznumzyofzybropzyzzbs]{#2}}{}%
  \ifstrequal{#1}{num_of_bropw_zba}{\hyperref[sailRISCVfnznumzyofzybropwzyzzba]{#2}}{}%
  \ifstrequal{#1}{num\_of\_bropw\_zba}{\hyperref[sailRISCVfnznumzyofzybropwzyzzba]{#2}}{}%
  \ifstrequal{#1}{num_of_bropw_zbb}{\hyperref[sailRISCVfnznumzyofzybropwzyzzbb]{#2}}{}%
  \ifstrequal{#1}{num\_of\_bropw\_zbb}{\hyperref[sailRISCVfnznumzyofzybropwzyzzbb]{#2}}{}%
  \ifstrequal{#1}{num_of_cache_op_kind}{\hyperref[sailRISCVfnznumzyofzycachezyopzykind]{#2}}{}%
  \ifstrequal{#1}{num\_of\_cache\_op\_kind}{\hyperref[sailRISCVfnznumzyofzycachezyopzykind]{#2}}{}%
  \ifstrequal{#1}{num_of_csrop}{\hyperref[sailRISCVfnznumzyofzycsrop]{#2}}{}%
  \ifstrequal{#1}{num\_of\_csrop}{\hyperref[sailRISCVfnznumzyofzycsrop]{#2}}{}%
  \ifstrequal{#1}{num_of_ext_access_type}{\hyperref[sailRISCVfnznumzyofzyextzyaccesszytype]{#2}}{}%
  \ifstrequal{#1}{num\_of\_ext\_access\_type}{\hyperref[sailRISCVfnznumzyofzyextzyaccesszytype]{#2}}{}%
  \ifstrequal{#1}{num_of_ext_exc_type}{\hyperref[sailRISCVfnznumzyofzyextzyexczytype]{#2}}{}%
  \ifstrequal{#1}{num\_of\_ext\_exc\_type}{\hyperref[sailRISCVfnznumzyofzyextzyexczytype]{#2}}{}%
  \ifstrequal{#1}{num_of_ext_ptw_error}{\hyperref[sailRISCVfnznumzyofzyextzyptwzyerror]{#2}}{}%
  \ifstrequal{#1}{num\_of\_ext\_ptw\_error}{\hyperref[sailRISCVfnznumzyofzyextzyptwzyerror]{#2}}{}%
  \ifstrequal{#1}{num_of_ext_ptw_fail}{\hyperref[sailRISCVfnznumzyofzyextzyptwzyfail]{#2}}{}%
  \ifstrequal{#1}{num\_of\_ext\_ptw\_fail}{\hyperref[sailRISCVfnznumzyofzyextzyptwzyfail]{#2}}{}%
  \ifstrequal{#1}{num_of_ext_ptw_lc}{\hyperref[sailRISCVfnznumzyofzyextzyptwzylc]{#2}}{}%
  \ifstrequal{#1}{num\_of\_ext\_ptw\_lc}{\hyperref[sailRISCVfnznumzyofzyextzyptwzylc]{#2}}{}%
  \ifstrequal{#1}{num_of_ext_ptw_sc}{\hyperref[sailRISCVfnznumzyofzyextzyptwzysc]{#2}}{}%
  \ifstrequal{#1}{num\_of\_ext\_ptw\_sc}{\hyperref[sailRISCVfnznumzyofzyextzyptwzysc]{#2}}{}%
  \ifstrequal{#1}{num_of_extop_zbb}{\hyperref[sailRISCVfnznumzyofzyextopzyzzbb]{#2}}{}%
  \ifstrequal{#1}{num\_of\_extop\_zbb}{\hyperref[sailRISCVfnznumzyofzyextopzyzzbb]{#2}}{}%
  \ifstrequal{#1}{num_of_f_bin_op_D}{\hyperref[sailRISCVfnznumzyofzyfzybinzyopzyD]{#2}}{}%
  \ifstrequal{#1}{num\_of\_f\_bin\_op\_D}{\hyperref[sailRISCVfnznumzyofzyfzybinzyopzyD]{#2}}{}%
  \ifstrequal{#1}{num_of_f_bin_op_H}{\hyperref[sailRISCVfnznumzyofzyfzybinzyopzyH]{#2}}{}%
  \ifstrequal{#1}{num\_of\_f\_bin\_op\_H}{\hyperref[sailRISCVfnznumzyofzyfzybinzyopzyH]{#2}}{}%
  \ifstrequal{#1}{num_of_f_bin_op_S}{\hyperref[sailRISCVfnznumzyofzyfzybinzyopzyS]{#2}}{}%
  \ifstrequal{#1}{num\_of\_f\_bin\_op\_S}{\hyperref[sailRISCVfnznumzyofzyfzybinzyopzyS]{#2}}{}%
  \ifstrequal{#1}{num_of_f_bin_rm_op_D}{\hyperref[sailRISCVfnznumzyofzyfzybinzyrmzyopzyD]{#2}}{}%
  \ifstrequal{#1}{num\_of\_f\_bin\_rm\_op\_D}{\hyperref[sailRISCVfnznumzyofzyfzybinzyrmzyopzyD]{#2}}{}%
  \ifstrequal{#1}{num_of_f_bin_rm_op_H}{\hyperref[sailRISCVfnznumzyofzyfzybinzyrmzyopzyH]{#2}}{}%
  \ifstrequal{#1}{num\_of\_f\_bin\_rm\_op\_H}{\hyperref[sailRISCVfnznumzyofzyfzybinzyrmzyopzyH]{#2}}{}%
  \ifstrequal{#1}{num_of_f_bin_rm_op_S}{\hyperref[sailRISCVfnznumzyofzyfzybinzyrmzyopzyS]{#2}}{}%
  \ifstrequal{#1}{num\_of\_f\_bin\_rm\_op\_S}{\hyperref[sailRISCVfnznumzyofzyfzybinzyrmzyopzyS]{#2}}{}%
  \ifstrequal{#1}{num_of_f_madd_op_D}{\hyperref[sailRISCVfnznumzyofzyfzymaddzyopzyD]{#2}}{}%
  \ifstrequal{#1}{num\_of\_f\_madd\_op\_D}{\hyperref[sailRISCVfnznumzyofzyfzymaddzyopzyD]{#2}}{}%
  \ifstrequal{#1}{num_of_f_madd_op_H}{\hyperref[sailRISCVfnznumzyofzyfzymaddzyopzyH]{#2}}{}%
  \ifstrequal{#1}{num\_of\_f\_madd\_op\_H}{\hyperref[sailRISCVfnznumzyofzyfzymaddzyopzyH]{#2}}{}%
  \ifstrequal{#1}{num_of_f_madd_op_S}{\hyperref[sailRISCVfnznumzyofzyfzymaddzyopzyS]{#2}}{}%
  \ifstrequal{#1}{num\_of\_f\_madd\_op\_S}{\hyperref[sailRISCVfnznumzyofzyfzymaddzyopzyS]{#2}}{}%
  \ifstrequal{#1}{num_of_f_un_op_D}{\hyperref[sailRISCVfnznumzyofzyfzyunzyopzyD]{#2}}{}%
  \ifstrequal{#1}{num\_of\_f\_un\_op\_D}{\hyperref[sailRISCVfnznumzyofzyfzyunzyopzyD]{#2}}{}%
  \ifstrequal{#1}{num_of_f_un_op_H}{\hyperref[sailRISCVfnznumzyofzyfzyunzyopzyH]{#2}}{}%
  \ifstrequal{#1}{num\_of\_f\_un\_op\_H}{\hyperref[sailRISCVfnznumzyofzyfzyunzyopzyH]{#2}}{}%
  \ifstrequal{#1}{num_of_f_un_op_S}{\hyperref[sailRISCVfnznumzyofzyfzyunzyopzyS]{#2}}{}%
  \ifstrequal{#1}{num\_of\_f\_un\_op\_S}{\hyperref[sailRISCVfnznumzyofzyfzyunzyopzyS]{#2}}{}%
  \ifstrequal{#1}{num_of_f_un_rm_op_D}{\hyperref[sailRISCVfnznumzyofzyfzyunzyrmzyopzyD]{#2}}{}%
  \ifstrequal{#1}{num\_of\_f\_un\_rm\_op\_D}{\hyperref[sailRISCVfnznumzyofzyfzyunzyrmzyopzyD]{#2}}{}%
  \ifstrequal{#1}{num_of_f_un_rm_op_H}{\hyperref[sailRISCVfnznumzyofzyfzyunzyrmzyopzyH]{#2}}{}%
  \ifstrequal{#1}{num\_of\_f\_un\_rm\_op\_H}{\hyperref[sailRISCVfnznumzyofzyfzyunzyrmzyopzyH]{#2}}{}%
  \ifstrequal{#1}{num_of_f_un_rm_op_S}{\hyperref[sailRISCVfnznumzyofzyfzyunzyrmzyopzyS]{#2}}{}%
  \ifstrequal{#1}{num\_of\_f\_un\_rm\_op\_S}{\hyperref[sailRISCVfnznumzyofzyfzyunzyrmzyopzyS]{#2}}{}%
  \ifstrequal{#1}{num_of_iop}{\hyperref[sailRISCVfnznumzyofzyiop]{#2}}{}%
  \ifstrequal{#1}{num\_of\_iop}{\hyperref[sailRISCVfnznumzyofzyiop]{#2}}{}%
  \ifstrequal{#1}{num_of_pmpAddrMatch}{\hyperref[sailRISCVfnznumzyofzypmpAddrMatch]{#2}}{}%
  \ifstrequal{#1}{num\_of\_pmpAddrMatch}{\hyperref[sailRISCVfnznumzyofzypmpAddrMatch]{#2}}{}%
  \ifstrequal{#1}{num_of_pmpMatch}{\hyperref[sailRISCVfnznumzyofzypmpMatch]{#2}}{}%
  \ifstrequal{#1}{num\_of\_pmpMatch}{\hyperref[sailRISCVfnznumzyofzypmpMatch]{#2}}{}%
  \ifstrequal{#1}{num_of_read_kind}{\hyperref[sailRISCVfnznumzyofzyreadzykind]{#2}}{}%
  \ifstrequal{#1}{num\_of\_read\_kind}{\hyperref[sailRISCVfnznumzyofzyreadzykind]{#2}}{}%
  \ifstrequal{#1}{num_of_rop}{\hyperref[sailRISCVfnznumzyofzyrop]{#2}}{}%
  \ifstrequal{#1}{num\_of\_rop}{\hyperref[sailRISCVfnznumzyofzyrop]{#2}}{}%
  \ifstrequal{#1}{num_of_ropw}{\hyperref[sailRISCVfnznumzyofzyropw]{#2}}{}%
  \ifstrequal{#1}{num\_of\_ropw}{\hyperref[sailRISCVfnznumzyofzyropw]{#2}}{}%
  \ifstrequal{#1}{num_of_rounding_mode}{\hyperref[sailRISCVfnznumzyofzyroundingzymode]{#2}}{}%
  \ifstrequal{#1}{num\_of\_rounding\_mode}{\hyperref[sailRISCVfnznumzyofzyroundingzymode]{#2}}{}%
  \ifstrequal{#1}{num_of_seed_opst}{\hyperref[sailRISCVfnznumzyofzyseedzyopst]{#2}}{}%
  \ifstrequal{#1}{num\_of\_seed\_opst}{\hyperref[sailRISCVfnznumzyofzyseedzyopst]{#2}}{}%
  \ifstrequal{#1}{num_of_sop}{\hyperref[sailRISCVfnznumzyofzysop]{#2}}{}%
  \ifstrequal{#1}{num\_of\_sop}{\hyperref[sailRISCVfnznumzyofzysop]{#2}}{}%
  \ifstrequal{#1}{num_of_sopw}{\hyperref[sailRISCVfnznumzyofzysopw]{#2}}{}%
  \ifstrequal{#1}{num\_of\_sopw}{\hyperref[sailRISCVfnznumzyofzysopw]{#2}}{}%
  \ifstrequal{#1}{num_of_trans_kind}{\hyperref[sailRISCVfnznumzyofzytranszykind]{#2}}{}%
  \ifstrequal{#1}{num\_of\_trans\_kind}{\hyperref[sailRISCVfnznumzyofzytranszykind]{#2}}{}%
  \ifstrequal{#1}{num_of_uop}{\hyperref[sailRISCVfnznumzyofzyuop]{#2}}{}%
  \ifstrequal{#1}{num\_of\_uop}{\hyperref[sailRISCVfnznumzyofzyuop]{#2}}{}%
  \ifstrequal{#1}{num_of_word_width}{\hyperref[sailRISCVfnznumzyofzywordzywidth]{#2}}{}%
  \ifstrequal{#1}{num\_of\_word\_width}{\hyperref[sailRISCVfnznumzyofzywordzywidth]{#2}}{}%
  \ifstrequal{#1}{num_of_write_kind}{\hyperref[sailRISCVfnznumzyofzywritezykind]{#2}}{}%
  \ifstrequal{#1}{num\_of\_write\_kind}{\hyperref[sailRISCVfnznumzyofzywritezykind]{#2}}{}%
  \ifstrequal{#1}{nvFlag}{\hyperref[sailRISCVfnznvFlag]{#2}}{}%
  \ifstrequal{#1}{nxFlag}{\hyperref[sailRISCVfnznxFlag]{#2}}{}%
  \ifstrequal{#1}{ofFlag}{\hyperref[sailRISCVfnzofFlag]{#2}}{}%
  \ifstrequal{#1}{ones}{\hyperref[sailRISCVfnzones]{#2}}{}%
  \ifstrequal{#1}{opt_spc_backwards}{\hyperref[sailRISCVfnzoptzyspczybackwards]{#2}}{}%
  \ifstrequal{#1}{opt\_spc\_backwards}{\hyperref[sailRISCVfnzoptzyspczybackwards]{#2}}{}%
  \ifstrequal{#1}{opt_spc_forwards}{\hyperref[sailRISCVfnzoptzyspczyforwards]{#2}}{}%
  \ifstrequal{#1}{opt\_spc\_forwards}{\hyperref[sailRISCVfnzoptzyspczyforwards]{#2}}{}%
  \ifstrequal{#1}{opt_spc_matches_prefix}{\hyperref[sailRISCVfnzoptzyspczymatcheszyprefix]{#2}}{}%
  \ifstrequal{#1}{opt\_spc\_matches\_prefix}{\hyperref[sailRISCVfnzoptzyspczymatcheszyprefix]{#2}}{}%
  \ifstrequal{#1}{pc_alignment_mask}{\hyperref[sailRISCVfnzpczyalignmentzymask]{#2}}{}%
  \ifstrequal{#1}{pc\_alignment\_mask}{\hyperref[sailRISCVfnzpczyalignmentzymask]{#2}}{}%
  \ifstrequal{#1}{pcc_access_system_regs}{\hyperref[sailRISCVfnzpcczyaccesszysystemzyregs]{#2}}{}%
  \ifstrequal{#1}{pcc\_access\_system\_regs}{\hyperref[sailRISCVfnzpcczyaccesszysystemzyregs]{#2}}{}%
  \ifstrequal{#1}{phys_mem_read}{\hyperref[sailRISCVfnzphyszymemzyread]{#2}}{}%
  \ifstrequal{#1}{phys\_mem\_read}{\hyperref[sailRISCVfnzphyszymemzyread]{#2}}{}%
  \ifstrequal{#1}{phys_mem_segments}{\hyperref[sailRISCVfnzphyszymemzysegments]{#2}}{}%
  \ifstrequal{#1}{phys\_mem\_segments}{\hyperref[sailRISCVfnzphyszymemzysegments]{#2}}{}%
  \ifstrequal{#1}{phys_mem_write}{\hyperref[sailRISCVfnzphyszymemzywrite]{#2}}{}%
  \ifstrequal{#1}{phys\_mem\_write}{\hyperref[sailRISCVfnzphyszymemzywrite]{#2}}{}%
  \ifstrequal{#1}{plat_htif_tohost}{\hyperref[sailRISCVfnzplatzyhtifzytohost]{#2}}{}%
  \ifstrequal{#1}{plat\_htif\_tohost}{\hyperref[sailRISCVfnzplatzyhtifzytohost]{#2}}{}%
  \ifstrequal{#1}{platform_wfi}{\hyperref[sailRISCVfnzplatformzywfi]{#2}}{}%
  \ifstrequal{#1}{platform\_wfi}{\hyperref[sailRISCVfnzplatformzywfi]{#2}}{}%
  \ifstrequal{#1}{pmpAddrMatchType_of_bits}{\hyperref[sailRISCVfnzpmpAddrMatchTypezyofzybits]{#2}}{}%
  \ifstrequal{#1}{pmpAddrMatchType\_of\_bits}{\hyperref[sailRISCVfnzpmpAddrMatchTypezyofzybits]{#2}}{}%
  \ifstrequal{#1}{pmpAddrMatchType_to_bits}{\hyperref[sailRISCVfnzpmpAddrMatchTypezytozybits]{#2}}{}%
  \ifstrequal{#1}{pmpAddrMatchType\_to\_bits}{\hyperref[sailRISCVfnzpmpAddrMatchTypezytozybits]{#2}}{}%
  \ifstrequal{#1}{pmpAddrMatch_of_num}{\hyperref[sailRISCVfnzpmpAddrMatchzyofzynum]{#2}}{}%
  \ifstrequal{#1}{pmpAddrMatch\_of\_num}{\hyperref[sailRISCVfnzpmpAddrMatchzyofzynum]{#2}}{}%
  \ifstrequal{#1}{pmpAddrRange}{\hyperref[sailRISCVfnzpmpAddrRange]{#2}}{}%
  \ifstrequal{#1}{pmpCheck}{\hyperref[sailRISCVfnzpmpCheck]{#2}}{}%
  \ifstrequal{#1}{pmpCheckPerms}{\hyperref[sailRISCVfnzpmpCheckPerms]{#2}}{}%
  \ifstrequal{#1}{pmpCheckRWX}{\hyperref[sailRISCVfnzpmpCheckRWX]{#2}}{}%
  \ifstrequal{#1}{pmpLocked}{\hyperref[sailRISCVfnzpmpLocked]{#2}}{}%
  \ifstrequal{#1}{pmpMatchAddr}{\hyperref[sailRISCVfnzpmpMatchAddr]{#2}}{}%
  \ifstrequal{#1}{pmpMatchEntry}{\hyperref[sailRISCVfnzpmpMatchEntry]{#2}}{}%
  \ifstrequal{#1}{pmpMatch_of_num}{\hyperref[sailRISCVfnzpmpMatchzyofzynum]{#2}}{}%
  \ifstrequal{#1}{pmpMatch\_of\_num}{\hyperref[sailRISCVfnzpmpMatchzyofzynum]{#2}}{}%
  \ifstrequal{#1}{pmpReadCfgReg}{\hyperref[sailRISCVfnzpmpReadCfgReg]{#2}}{}%
  \ifstrequal{#1}{pmpTORLocked}{\hyperref[sailRISCVfnzpmpTORLocked]{#2}}{}%
  \ifstrequal{#1}{pmpWriteAddr}{\hyperref[sailRISCVfnzpmpWriteAddr]{#2}}{}%
  \ifstrequal{#1}{pmpWriteCfg}{\hyperref[sailRISCVfnzpmpWriteCfg]{#2}}{}%
  \ifstrequal{#1}{pmpWriteCfgReg}{\hyperref[sailRISCVfnzpmpWriteCfgReg]{#2}}{}%
  \ifstrequal{#1}{pmp_mem_read}{\hyperref[sailRISCVfnzpmpzymemzyread]{#2}}{}%
  \ifstrequal{#1}{pmp\_mem\_read}{\hyperref[sailRISCVfnzpmpzymemzyread]{#2}}{}%
  \ifstrequal{#1}{pmp_mem_write}{\hyperref[sailRISCVfnzpmpzymemzywrite]{#2}}{}%
  \ifstrequal{#1}{pmp\_mem\_write}{\hyperref[sailRISCVfnzpmpzymemzywrite]{#2}}{}%
  \ifstrequal{#1}{prepare_trap_vector}{\hyperref[sailRISCVfnzpreparezytrapzyvector]{#2}}{}%
  \ifstrequal{#1}{prepare\_trap\_vector}{\hyperref[sailRISCVfnzpreparezytrapzyvector]{#2}}{}%
  \ifstrequal{#1}{prepare_xret_target}{\hyperref[sailRISCVfnzpreparezyxretzytarget]{#2}}{}%
  \ifstrequal{#1}{prepare\_xret\_target}{\hyperref[sailRISCVfnzpreparezyxretzytarget]{#2}}{}%
  \ifstrequal{#1}{print_insn}{\hyperref[sailRISCVfnzprintzyinsn]{#2}}{}%
  \ifstrequal{#1}{print\_insn}{\hyperref[sailRISCVfnzprintzyinsn]{#2}}{}%
  \ifstrequal{#1}{privLevel_of_bits}{\hyperref[sailRISCVfnzprivLevelzyofzybits]{#2}}{}%
  \ifstrequal{#1}{privLevel\_of\_bits}{\hyperref[sailRISCVfnzprivLevelzyofzybits]{#2}}{}%
  \ifstrequal{#1}{privLevel_to_bits}{\hyperref[sailRISCVfnzprivLevelzytozybits]{#2}}{}%
  \ifstrequal{#1}{privLevel\_to\_bits}{\hyperref[sailRISCVfnzprivLevelzytozybits]{#2}}{}%
  \ifstrequal{#1}{privLevel_to_str}{\hyperref[sailRISCVfnzprivLevelzytozystr]{#2}}{}%
  \ifstrequal{#1}{privLevel\_to\_str}{\hyperref[sailRISCVfnzprivLevelzytozystr]{#2}}{}%
  \ifstrequal{#1}{processPending}{\hyperref[sailRISCVfnzprocessPending]{#2}}{}%
  \ifstrequal{#1}{process_fload16}{\hyperref[sailRISCVfnzprocesszyfload16]{#2}}{}%
  \ifstrequal{#1}{process\_fload16}{\hyperref[sailRISCVfnzprocesszyfload16]{#2}}{}%
  \ifstrequal{#1}{process_fload32}{\hyperref[sailRISCVfnzprocesszyfload32]{#2}}{}%
  \ifstrequal{#1}{process\_fload32}{\hyperref[sailRISCVfnzprocesszyfload32]{#2}}{}%
  \ifstrequal{#1}{process_fload64}{\hyperref[sailRISCVfnzprocesszyfload64]{#2}}{}%
  \ifstrequal{#1}{process\_fload64}{\hyperref[sailRISCVfnzprocesszyfload64]{#2}}{}%
  \ifstrequal{#1}{process_fstore}{\hyperref[sailRISCVfnzprocesszyfstore]{#2}}{}%
  \ifstrequal{#1}{process\_fstore}{\hyperref[sailRISCVfnzprocesszyfstore]{#2}}{}%
  \ifstrequal{#1}{process_load}{\hyperref[sailRISCVfnzprocesszyload]{#2}}{}%
  \ifstrequal{#1}{process\_load}{\hyperref[sailRISCVfnzprocesszyload]{#2}}{}%
  \ifstrequal{#1}{process_loadres}{\hyperref[sailRISCVfnzprocesszyloadres]{#2}}{}%
  \ifstrequal{#1}{process\_loadres}{\hyperref[sailRISCVfnzprocesszyloadres]{#2}}{}%
  \ifstrequal{#1}{ptw_error_to_str}{\hyperref[sailRISCVfnzptwzyerrorzytozystr]{#2}}{}%
  \ifstrequal{#1}{ptw\_error\_to\_str}{\hyperref[sailRISCVfnzptwzyerrorzytozystr]{#2}}{}%
  \ifstrequal{#1}{rC}{\hyperref[sailRISCVfnzrC]{#2}}{}%
  \ifstrequal{#1}{rC_bits}{\hyperref[sailRISCVfnzrCzybits]{#2}}{}%
  \ifstrequal{#1}{rC\_bits}{\hyperref[sailRISCVfnzrCzybits]{#2}}{}%
  \ifstrequal{#1}{rF}{\hyperref[sailRISCVfnzrF]{#2}}{}%
  \ifstrequal{#1}{rF_bits}{\hyperref[sailRISCVfnzrFzybits]{#2}}{}%
  \ifstrequal{#1}{rF\_bits}{\hyperref[sailRISCVfnzrFzybits]{#2}}{}%
  \ifstrequal{#1}{rF_or_X_D}{\hyperref[sailRISCVfnzrFzyorzyXzyD]{#2}}{}%
  \ifstrequal{#1}{rF\_or\_X\_D}{\hyperref[sailRISCVfnzrFzyorzyXzyD]{#2}}{}%
  \ifstrequal{#1}{rF_or_X_H}{\hyperref[sailRISCVfnzrFzyorzyXzyH]{#2}}{}%
  \ifstrequal{#1}{rF\_or\_X\_H}{\hyperref[sailRISCVfnzrFzyorzyXzyH]{#2}}{}%
  \ifstrequal{#1}{rF_or_X_S}{\hyperref[sailRISCVfnzrFzyorzyXzyS]{#2}}{}%
  \ifstrequal{#1}{rF\_or\_X\_S}{\hyperref[sailRISCVfnzrFzyorzyXzyS]{#2}}{}%
  \ifstrequal{#1}{rX}{\hyperref[sailRISCVfnzrX]{#2}}{}%
  \ifstrequal{#1}{rX_bits}{\hyperref[sailRISCVfnzrXzybits]{#2}}{}%
  \ifstrequal{#1}{rX\_bits}{\hyperref[sailRISCVfnzrXzybits]{#2}}{}%
  \ifstrequal{#1}{readCSR}{\hyperref[sailRISCVfnzreadCSR]{#2}}{}%
  \ifstrequal{#1}{read_kind_of_flags}{\hyperref[sailRISCVfnzreadzykindzyofzyflags]{#2}}{}%
  \ifstrequal{#1}{read\_kind\_of\_flags}{\hyperref[sailRISCVfnzreadzykindzyofzyflags]{#2}}{}%
  \ifstrequal{#1}{read_kind_of_num}{\hyperref[sailRISCVfnzreadzykindzyofzynum]{#2}}{}%
  \ifstrequal{#1}{read\_kind\_of\_num}{\hyperref[sailRISCVfnzreadzykindzyofzynum]{#2}}{}%
  \ifstrequal{#1}{read_ram}{\hyperref[sailRISCVfnzreadzyram]{#2}}{}%
  \ifstrequal{#1}{read\_ram}{\hyperref[sailRISCVfnzreadzyram]{#2}}{}%
  \ifstrequal{#1}{read_seed_csr}{\hyperref[sailRISCVfnzreadzyseedzycsr]{#2}}{}%
  \ifstrequal{#1}{read\_seed\_csr}{\hyperref[sailRISCVfnzreadzyseedzycsr]{#2}}{}%
  \ifstrequal{#1}{reg_name_abi}{\hyperref[sailRISCVfnzregzynamezyabi]{#2}}{}%
  \ifstrequal{#1}{reg\_name\_abi}{\hyperref[sailRISCVfnzregzynamezyabi]{#2}}{}%
  \ifstrequal{#1}{regidx_to_regno}{\hyperref[sailRISCVfnzregidxzytozyregno]{#2}}{}%
  \ifstrequal{#1}{regidx\_to\_regno}{\hyperref[sailRISCVfnzregidxzytozyregno]{#2}}{}%
  \ifstrequal{#1}{regval_from_reg}{\hyperref[sailRISCVfnzregvalzyfromzyreg]{#2}}{}%
  \ifstrequal{#1}{regval\_from\_reg}{\hyperref[sailRISCVfnzregvalzyfromzyreg]{#2}}{}%
  \ifstrequal{#1}{regval_into_reg}{\hyperref[sailRISCVfnzregvalzyintozyreg]{#2}}{}%
  \ifstrequal{#1}{regval\_into\_reg}{\hyperref[sailRISCVfnzregvalzyintozyreg]{#2}}{}%
  \ifstrequal{#1}{reset_htif}{\hyperref[sailRISCVfnzresetzyhtif]{#2}}{}%
  \ifstrequal{#1}{reset\_htif}{\hyperref[sailRISCVfnzresetzyhtif]{#2}}{}%
  \ifstrequal{#1}{retire_instruction}{\hyperref[sailRISCVfnzretirezyinstruction]{#2}}{}%
  \ifstrequal{#1}{retire\_instruction}{\hyperref[sailRISCVfnzretirezyinstruction]{#2}}{}%
  \ifstrequal{#1}{reverse_bits_in_byte}{\hyperref[sailRISCVfnzreversezybitszyinzybyte]{#2}}{}%
  \ifstrequal{#1}{reverse\_bits\_in\_byte}{\hyperref[sailRISCVfnzreversezybitszyinzybyte]{#2}}{}%
  \ifstrequal{#1}{riscv_f16Add}{\hyperref[sailRISCVfnzriscvzyf16Add]{#2}}{}%
  \ifstrequal{#1}{riscv\_f16Add}{\hyperref[sailRISCVfnzriscvzyf16Add]{#2}}{}%
  \ifstrequal{#1}{riscv_f16Div}{\hyperref[sailRISCVfnzriscvzyf16Div]{#2}}{}%
  \ifstrequal{#1}{riscv\_f16Div}{\hyperref[sailRISCVfnzriscvzyf16Div]{#2}}{}%
  \ifstrequal{#1}{riscv_f16Eq}{\hyperref[sailRISCVfnzriscvzyf16Eq]{#2}}{}%
  \ifstrequal{#1}{riscv\_f16Eq}{\hyperref[sailRISCVfnzriscvzyf16Eq]{#2}}{}%
  \ifstrequal{#1}{riscv_f16Le}{\hyperref[sailRISCVfnzriscvzyf16Le]{#2}}{}%
  \ifstrequal{#1}{riscv\_f16Le}{\hyperref[sailRISCVfnzriscvzyf16Le]{#2}}{}%
  \ifstrequal{#1}{riscv_f16Lt}{\hyperref[sailRISCVfnzriscvzyf16Lt]{#2}}{}%
  \ifstrequal{#1}{riscv\_f16Lt}{\hyperref[sailRISCVfnzriscvzyf16Lt]{#2}}{}%
  \ifstrequal{#1}{riscv_f16Mul}{\hyperref[sailRISCVfnzriscvzyf16Mul]{#2}}{}%
  \ifstrequal{#1}{riscv\_f16Mul}{\hyperref[sailRISCVfnzriscvzyf16Mul]{#2}}{}%
  \ifstrequal{#1}{riscv_f16MulAdd}{\hyperref[sailRISCVfnzriscvzyf16MulAdd]{#2}}{}%
  \ifstrequal{#1}{riscv\_f16MulAdd}{\hyperref[sailRISCVfnzriscvzyf16MulAdd]{#2}}{}%
  \ifstrequal{#1}{riscv_f16Sqrt}{\hyperref[sailRISCVfnzriscvzyf16Sqrt]{#2}}{}%
  \ifstrequal{#1}{riscv\_f16Sqrt}{\hyperref[sailRISCVfnzriscvzyf16Sqrt]{#2}}{}%
  \ifstrequal{#1}{riscv_f16Sub}{\hyperref[sailRISCVfnzriscvzyf16Sub]{#2}}{}%
  \ifstrequal{#1}{riscv\_f16Sub}{\hyperref[sailRISCVfnzriscvzyf16Sub]{#2}}{}%
  \ifstrequal{#1}{riscv_f16ToF32}{\hyperref[sailRISCVfnzriscvzyf16ToF32]{#2}}{}%
  \ifstrequal{#1}{riscv\_f16ToF32}{\hyperref[sailRISCVfnzriscvzyf16ToF32]{#2}}{}%
  \ifstrequal{#1}{riscv_f16ToF64}{\hyperref[sailRISCVfnzriscvzyf16ToF64]{#2}}{}%
  \ifstrequal{#1}{riscv\_f16ToF64}{\hyperref[sailRISCVfnzriscvzyf16ToF64]{#2}}{}%
  \ifstrequal{#1}{riscv_f16ToI32}{\hyperref[sailRISCVfnzriscvzyf16ToI32]{#2}}{}%
  \ifstrequal{#1}{riscv\_f16ToI32}{\hyperref[sailRISCVfnzriscvzyf16ToI32]{#2}}{}%
  \ifstrequal{#1}{riscv_f16ToI64}{\hyperref[sailRISCVfnzriscvzyf16ToI64]{#2}}{}%
  \ifstrequal{#1}{riscv\_f16ToI64}{\hyperref[sailRISCVfnzriscvzyf16ToI64]{#2}}{}%
  \ifstrequal{#1}{riscv_f16ToUi32}{\hyperref[sailRISCVfnzriscvzyf16ToUi32]{#2}}{}%
  \ifstrequal{#1}{riscv\_f16ToUi32}{\hyperref[sailRISCVfnzriscvzyf16ToUi32]{#2}}{}%
  \ifstrequal{#1}{riscv_f16ToUi64}{\hyperref[sailRISCVfnzriscvzyf16ToUi64]{#2}}{}%
  \ifstrequal{#1}{riscv\_f16ToUi64}{\hyperref[sailRISCVfnzriscvzyf16ToUi64]{#2}}{}%
  \ifstrequal{#1}{riscv_f32Add}{\hyperref[sailRISCVfnzriscvzyf32Add]{#2}}{}%
  \ifstrequal{#1}{riscv\_f32Add}{\hyperref[sailRISCVfnzriscvzyf32Add]{#2}}{}%
  \ifstrequal{#1}{riscv_f32Div}{\hyperref[sailRISCVfnzriscvzyf32Div]{#2}}{}%
  \ifstrequal{#1}{riscv\_f32Div}{\hyperref[sailRISCVfnzriscvzyf32Div]{#2}}{}%
  \ifstrequal{#1}{riscv_f32Eq}{\hyperref[sailRISCVfnzriscvzyf32Eq]{#2}}{}%
  \ifstrequal{#1}{riscv\_f32Eq}{\hyperref[sailRISCVfnzriscvzyf32Eq]{#2}}{}%
  \ifstrequal{#1}{riscv_f32Le}{\hyperref[sailRISCVfnzriscvzyf32Le]{#2}}{}%
  \ifstrequal{#1}{riscv\_f32Le}{\hyperref[sailRISCVfnzriscvzyf32Le]{#2}}{}%
  \ifstrequal{#1}{riscv_f32Lt}{\hyperref[sailRISCVfnzriscvzyf32Lt]{#2}}{}%
  \ifstrequal{#1}{riscv\_f32Lt}{\hyperref[sailRISCVfnzriscvzyf32Lt]{#2}}{}%
  \ifstrequal{#1}{riscv_f32Mul}{\hyperref[sailRISCVfnzriscvzyf32Mul]{#2}}{}%
  \ifstrequal{#1}{riscv\_f32Mul}{\hyperref[sailRISCVfnzriscvzyf32Mul]{#2}}{}%
  \ifstrequal{#1}{riscv_f32MulAdd}{\hyperref[sailRISCVfnzriscvzyf32MulAdd]{#2}}{}%
  \ifstrequal{#1}{riscv\_f32MulAdd}{\hyperref[sailRISCVfnzriscvzyf32MulAdd]{#2}}{}%
  \ifstrequal{#1}{riscv_f32Sqrt}{\hyperref[sailRISCVfnzriscvzyf32Sqrt]{#2}}{}%
  \ifstrequal{#1}{riscv\_f32Sqrt}{\hyperref[sailRISCVfnzriscvzyf32Sqrt]{#2}}{}%
  \ifstrequal{#1}{riscv_f32Sub}{\hyperref[sailRISCVfnzriscvzyf32Sub]{#2}}{}%
  \ifstrequal{#1}{riscv\_f32Sub}{\hyperref[sailRISCVfnzriscvzyf32Sub]{#2}}{}%
  \ifstrequal{#1}{riscv_f32ToF16}{\hyperref[sailRISCVfnzriscvzyf32ToF16]{#2}}{}%
  \ifstrequal{#1}{riscv\_f32ToF16}{\hyperref[sailRISCVfnzriscvzyf32ToF16]{#2}}{}%
  \ifstrequal{#1}{riscv_f32ToF64}{\hyperref[sailRISCVfnzriscvzyf32ToF64]{#2}}{}%
  \ifstrequal{#1}{riscv\_f32ToF64}{\hyperref[sailRISCVfnzriscvzyf32ToF64]{#2}}{}%
  \ifstrequal{#1}{riscv_f32ToI32}{\hyperref[sailRISCVfnzriscvzyf32ToI32]{#2}}{}%
  \ifstrequal{#1}{riscv\_f32ToI32}{\hyperref[sailRISCVfnzriscvzyf32ToI32]{#2}}{}%
  \ifstrequal{#1}{riscv_f32ToI64}{\hyperref[sailRISCVfnzriscvzyf32ToI64]{#2}}{}%
  \ifstrequal{#1}{riscv\_f32ToI64}{\hyperref[sailRISCVfnzriscvzyf32ToI64]{#2}}{}%
  \ifstrequal{#1}{riscv_f32ToUi32}{\hyperref[sailRISCVfnzriscvzyf32ToUi32]{#2}}{}%
  \ifstrequal{#1}{riscv\_f32ToUi32}{\hyperref[sailRISCVfnzriscvzyf32ToUi32]{#2}}{}%
  \ifstrequal{#1}{riscv_f32ToUi64}{\hyperref[sailRISCVfnzriscvzyf32ToUi64]{#2}}{}%
  \ifstrequal{#1}{riscv\_f32ToUi64}{\hyperref[sailRISCVfnzriscvzyf32ToUi64]{#2}}{}%
  \ifstrequal{#1}{riscv_f64Add}{\hyperref[sailRISCVfnzriscvzyf64Add]{#2}}{}%
  \ifstrequal{#1}{riscv\_f64Add}{\hyperref[sailRISCVfnzriscvzyf64Add]{#2}}{}%
  \ifstrequal{#1}{riscv_f64Div}{\hyperref[sailRISCVfnzriscvzyf64Div]{#2}}{}%
  \ifstrequal{#1}{riscv\_f64Div}{\hyperref[sailRISCVfnzriscvzyf64Div]{#2}}{}%
  \ifstrequal{#1}{riscv_f64Eq}{\hyperref[sailRISCVfnzriscvzyf64Eq]{#2}}{}%
  \ifstrequal{#1}{riscv\_f64Eq}{\hyperref[sailRISCVfnzriscvzyf64Eq]{#2}}{}%
  \ifstrequal{#1}{riscv_f64Le}{\hyperref[sailRISCVfnzriscvzyf64Le]{#2}}{}%
  \ifstrequal{#1}{riscv\_f64Le}{\hyperref[sailRISCVfnzriscvzyf64Le]{#2}}{}%
  \ifstrequal{#1}{riscv_f64Lt}{\hyperref[sailRISCVfnzriscvzyf64Lt]{#2}}{}%
  \ifstrequal{#1}{riscv\_f64Lt}{\hyperref[sailRISCVfnzriscvzyf64Lt]{#2}}{}%
  \ifstrequal{#1}{riscv_f64Mul}{\hyperref[sailRISCVfnzriscvzyf64Mul]{#2}}{}%
  \ifstrequal{#1}{riscv\_f64Mul}{\hyperref[sailRISCVfnzriscvzyf64Mul]{#2}}{}%
  \ifstrequal{#1}{riscv_f64MulAdd}{\hyperref[sailRISCVfnzriscvzyf64MulAdd]{#2}}{}%
  \ifstrequal{#1}{riscv\_f64MulAdd}{\hyperref[sailRISCVfnzriscvzyf64MulAdd]{#2}}{}%
  \ifstrequal{#1}{riscv_f64Sqrt}{\hyperref[sailRISCVfnzriscvzyf64Sqrt]{#2}}{}%
  \ifstrequal{#1}{riscv\_f64Sqrt}{\hyperref[sailRISCVfnzriscvzyf64Sqrt]{#2}}{}%
  \ifstrequal{#1}{riscv_f64Sub}{\hyperref[sailRISCVfnzriscvzyf64Sub]{#2}}{}%
  \ifstrequal{#1}{riscv\_f64Sub}{\hyperref[sailRISCVfnzriscvzyf64Sub]{#2}}{}%
  \ifstrequal{#1}{riscv_f64ToF16}{\hyperref[sailRISCVfnzriscvzyf64ToF16]{#2}}{}%
  \ifstrequal{#1}{riscv\_f64ToF16}{\hyperref[sailRISCVfnzriscvzyf64ToF16]{#2}}{}%
  \ifstrequal{#1}{riscv_f64ToF32}{\hyperref[sailRISCVfnzriscvzyf64ToF32]{#2}}{}%
  \ifstrequal{#1}{riscv\_f64ToF32}{\hyperref[sailRISCVfnzriscvzyf64ToF32]{#2}}{}%
  \ifstrequal{#1}{riscv_f64ToI32}{\hyperref[sailRISCVfnzriscvzyf64ToI32]{#2}}{}%
  \ifstrequal{#1}{riscv\_f64ToI32}{\hyperref[sailRISCVfnzriscvzyf64ToI32]{#2}}{}%
  \ifstrequal{#1}{riscv_f64ToI64}{\hyperref[sailRISCVfnzriscvzyf64ToI64]{#2}}{}%
  \ifstrequal{#1}{riscv\_f64ToI64}{\hyperref[sailRISCVfnzriscvzyf64ToI64]{#2}}{}%
  \ifstrequal{#1}{riscv_f64ToUi32}{\hyperref[sailRISCVfnzriscvzyf64ToUi32]{#2}}{}%
  \ifstrequal{#1}{riscv\_f64ToUi32}{\hyperref[sailRISCVfnzriscvzyf64ToUi32]{#2}}{}%
  \ifstrequal{#1}{riscv_f64ToUi64}{\hyperref[sailRISCVfnzriscvzyf64ToUi64]{#2}}{}%
  \ifstrequal{#1}{riscv\_f64ToUi64}{\hyperref[sailRISCVfnzriscvzyf64ToUi64]{#2}}{}%
  \ifstrequal{#1}{riscv_i32ToF16}{\hyperref[sailRISCVfnzriscvzyi32ToF16]{#2}}{}%
  \ifstrequal{#1}{riscv\_i32ToF16}{\hyperref[sailRISCVfnzriscvzyi32ToF16]{#2}}{}%
  \ifstrequal{#1}{riscv_i32ToF32}{\hyperref[sailRISCVfnzriscvzyi32ToF32]{#2}}{}%
  \ifstrequal{#1}{riscv\_i32ToF32}{\hyperref[sailRISCVfnzriscvzyi32ToF32]{#2}}{}%
  \ifstrequal{#1}{riscv_i32ToF64}{\hyperref[sailRISCVfnzriscvzyi32ToF64]{#2}}{}%
  \ifstrequal{#1}{riscv\_i32ToF64}{\hyperref[sailRISCVfnzriscvzyi32ToF64]{#2}}{}%
  \ifstrequal{#1}{riscv_i64ToF16}{\hyperref[sailRISCVfnzriscvzyi64ToF16]{#2}}{}%
  \ifstrequal{#1}{riscv\_i64ToF16}{\hyperref[sailRISCVfnzriscvzyi64ToF16]{#2}}{}%
  \ifstrequal{#1}{riscv_i64ToF32}{\hyperref[sailRISCVfnzriscvzyi64ToF32]{#2}}{}%
  \ifstrequal{#1}{riscv\_i64ToF32}{\hyperref[sailRISCVfnzriscvzyi64ToF32]{#2}}{}%
  \ifstrequal{#1}{riscv_i64ToF64}{\hyperref[sailRISCVfnzriscvzyi64ToF64]{#2}}{}%
  \ifstrequal{#1}{riscv\_i64ToF64}{\hyperref[sailRISCVfnzriscvzyi64ToF64]{#2}}{}%
  \ifstrequal{#1}{riscv_ui32ToF16}{\hyperref[sailRISCVfnzriscvzyui32ToF16]{#2}}{}%
  \ifstrequal{#1}{riscv\_ui32ToF16}{\hyperref[sailRISCVfnzriscvzyui32ToF16]{#2}}{}%
  \ifstrequal{#1}{riscv_ui32ToF32}{\hyperref[sailRISCVfnzriscvzyui32ToF32]{#2}}{}%
  \ifstrequal{#1}{riscv\_ui32ToF32}{\hyperref[sailRISCVfnzriscvzyui32ToF32]{#2}}{}%
  \ifstrequal{#1}{riscv_ui32ToF64}{\hyperref[sailRISCVfnzriscvzyui32ToF64]{#2}}{}%
  \ifstrequal{#1}{riscv\_ui32ToF64}{\hyperref[sailRISCVfnzriscvzyui32ToF64]{#2}}{}%
  \ifstrequal{#1}{riscv_ui64ToF16}{\hyperref[sailRISCVfnzriscvzyui64ToF16]{#2}}{}%
  \ifstrequal{#1}{riscv\_ui64ToF16}{\hyperref[sailRISCVfnzriscvzyui64ToF16]{#2}}{}%
  \ifstrequal{#1}{riscv_ui64ToF32}{\hyperref[sailRISCVfnzriscvzyui64ToF32]{#2}}{}%
  \ifstrequal{#1}{riscv\_ui64ToF32}{\hyperref[sailRISCVfnzriscvzyui64ToF32]{#2}}{}%
  \ifstrequal{#1}{riscv_ui64ToF64}{\hyperref[sailRISCVfnzriscvzyui64ToF64]{#2}}{}%
  \ifstrequal{#1}{riscv\_ui64ToF64}{\hyperref[sailRISCVfnzriscvzyui64ToF64]{#2}}{}%
  \ifstrequal{#1}{rop_of_num}{\hyperref[sailRISCVfnzropzyofzynum]{#2}}{}%
  \ifstrequal{#1}{rop\_of\_num}{\hyperref[sailRISCVfnzropzyofzynum]{#2}}{}%
  \ifstrequal{#1}{ropw_of_num}{\hyperref[sailRISCVfnzropwzyofzynum]{#2}}{}%
  \ifstrequal{#1}{ropw\_of\_num}{\hyperref[sailRISCVfnzropwzyofzynum]{#2}}{}%
  \ifstrequal{#1}{rotate_bits_left}{\hyperref[sailRISCVfnzrotatezybitszyleft]{#2}}{}%
  \ifstrequal{#1}{rotate\_bits\_left}{\hyperref[sailRISCVfnzrotatezybitszyleft]{#2}}{}%
  \ifstrequal{#1}{rotate_bits_right}{\hyperref[sailRISCVfnzrotatezybitszyright]{#2}}{}%
  \ifstrequal{#1}{rotate\_bits\_right}{\hyperref[sailRISCVfnzrotatezybitszyright]{#2}}{}%
  \ifstrequal{#1}{rotatel}{\hyperref[sailRISCVfnzrotatel]{#2}}{}%
  \ifstrequal{#1}{rotater}{\hyperref[sailRISCVfnzrotater]{#2}}{}%
  \ifstrequal{#1}{rounding_mode_of_num}{\hyperref[sailRISCVfnzroundingzymodezyofzynum]{#2}}{}%
  \ifstrequal{#1}{rounding\_mode\_of\_num}{\hyperref[sailRISCVfnzroundingzymodezyofzynum]{#2}}{}%
  \ifstrequal{#1}{rvfi_read}{\hyperref[sailRISCVfnzrvfizyread]{#2}}{}%
  \ifstrequal{#1}{rvfi\_read}{\hyperref[sailRISCVfnzrvfizyread]{#2}}{}%
  \ifstrequal{#1}{rvfi_trap}{\hyperref[sailRISCVfnzrvfizytrap]{#2}}{}%
  \ifstrequal{#1}{rvfi\_trap}{\hyperref[sailRISCVfnzrvfizytrap]{#2}}{}%
  \ifstrequal{#1}{rvfi_wX}{\hyperref[sailRISCVfnzrvfizywX]{#2}}{}%
  \ifstrequal{#1}{rvfi\_wX}{\hyperref[sailRISCVfnzrvfizywX]{#2}}{}%
  \ifstrequal{#1}{rvfi_write}{\hyperref[sailRISCVfnzrvfizywrite]{#2}}{}%
  \ifstrequal{#1}{rvfi\_write}{\hyperref[sailRISCVfnzrvfizywrite]{#2}}{}%
  \ifstrequal{#1}{sail_mask}{\hyperref[sailRISCVfnzsailzymask]{#2}}{}%
  \ifstrequal{#1}{sail\_mask}{\hyperref[sailRISCVfnzsailzymask]{#2}}{}%
  \ifstrequal{#1}{sail_ones}{\hyperref[sailRISCVfnzsailzyones]{#2}}{}%
  \ifstrequal{#1}{sail\_ones}{\hyperref[sailRISCVfnzsailzyones]{#2}}{}%
  \ifstrequal{#1}{satp64Mode_of_bits}{\hyperref[sailRISCVfnzsatp64Modezyofzybits]{#2}}{}%
  \ifstrequal{#1}{satp64Mode\_of\_bits}{\hyperref[sailRISCVfnzsatp64Modezyofzybits]{#2}}{}%
  \ifstrequal{#1}{sealCap}{\hyperref[sailRISCVfnzsealCap]{#2}}{}%
  \ifstrequal{#1}{seed_opst_of_num}{\hyperref[sailRISCVfnzseedzyopstzyofzynum]{#2}}{}%
  \ifstrequal{#1}{seed\_opst\_of\_num}{\hyperref[sailRISCVfnzseedzyopstzyofzynum]{#2}}{}%
  \ifstrequal{#1}{select_instr_or_fcsr_rm}{\hyperref[sailRISCVfnzselectzyinstrzyorzyfcsrzyrm]{#2}}{}%
  \ifstrequal{#1}{select\_instr\_or\_fcsr\_rm}{\hyperref[sailRISCVfnzselectzyinstrzyorzyfcsrzyrm]{#2}}{}%
  \ifstrequal{#1}{setCapAddr}{\hyperref[sailRISCVfnzsetCapAddr]{#2}}{}%
  \ifstrequal{#1}{setCapBounds}{\hyperref[sailRISCVfnzsetCapBounds]{#2}}{}%
  \ifstrequal{#1}{setCapFlags}{\hyperref[sailRISCVfnzsetCapFlags]{#2}}{}%
  \ifstrequal{#1}{setCapOffset}{\hyperref[sailRISCVfnzsetCapOffset]{#2}}{}%
  \ifstrequal{#1}{setCapOffsetChecked}{\hyperref[sailRISCVfnzsetCapOffsetChecked]{#2}}{}%
  \ifstrequal{#1}{setCapPerms}{\hyperref[sailRISCVfnzsetCapPerms]{#2}}{}%
  \ifstrequal{#1}{set_mstatus_SXL}{\hyperref[sailRISCVfnzsetzymstatuszySXL]{#2}}{}%
  \ifstrequal{#1}{set\_mstatus\_SXL}{\hyperref[sailRISCVfnzsetzymstatuszySXL]{#2}}{}%
  \ifstrequal{#1}{set_mstatus_UXL}{\hyperref[sailRISCVfnzsetzymstatuszyUXL]{#2}}{}%
  \ifstrequal{#1}{set\_mstatus\_UXL}{\hyperref[sailRISCVfnzsetzymstatuszyUXL]{#2}}{}%
  \ifstrequal{#1}{set_mtvec}{\hyperref[sailRISCVfnzsetzymtvec]{#2}}{}%
  \ifstrequal{#1}{set\_mtvec}{\hyperref[sailRISCVfnzsetzymtvec]{#2}}{}%
  \ifstrequal{#1}{set_next_pc}{\hyperref[sailRISCVfnzsetzynextzypc]{#2}}{}%
  \ifstrequal{#1}{set\_next\_pc}{\hyperref[sailRISCVfnzsetzynextzypc]{#2}}{}%
  \ifstrequal{#1}{set_sstatus_UXL}{\hyperref[sailRISCVfnzsetzysstatuszyUXL]{#2}}{}%
  \ifstrequal{#1}{set\_sstatus\_UXL}{\hyperref[sailRISCVfnzsetzysstatuszyUXL]{#2}}{}%
  \ifstrequal{#1}{set_stvec}{\hyperref[sailRISCVfnzsetzystvec]{#2}}{}%
  \ifstrequal{#1}{set\_stvec}{\hyperref[sailRISCVfnzsetzystvec]{#2}}{}%
  \ifstrequal{#1}{set_utvec}{\hyperref[sailRISCVfnzsetzyutvec]{#2}}{}%
  \ifstrequal{#1}{set\_utvec}{\hyperref[sailRISCVfnzsetzyutvec]{#2}}{}%
  \ifstrequal{#1}{set_xret_target}{\hyperref[sailRISCVfnzsetzyxretzytarget]{#2}}{}%
  \ifstrequal{#1}{set\_xret\_target}{\hyperref[sailRISCVfnzsetzyxretzytarget]{#2}}{}%
  \ifstrequal{#1}{shift_right_arith32}{\hyperref[sailRISCVfnzshiftzyrightzyarith32]{#2}}{}%
  \ifstrequal{#1}{shift\_right\_arith32}{\hyperref[sailRISCVfnzshiftzyrightzyarith32]{#2}}{}%
  \ifstrequal{#1}{shift_right_arith64}{\hyperref[sailRISCVfnzshiftzyrightzyarith64]{#2}}{}%
  \ifstrequal{#1}{shift\_right\_arith64}{\hyperref[sailRISCVfnzshiftzyrightzyarith64]{#2}}{}%
  \ifstrequal{#1}{slice_mask}{\hyperref[sailRISCVfnzslicezymask]{#2}}{}%
  \ifstrequal{#1}{slice\_mask}{\hyperref[sailRISCVfnzslicezymask]{#2}}{}%
  \ifstrequal{#1}{sop_of_num}{\hyperref[sailRISCVfnzsopzyofzynum]{#2}}{}%
  \ifstrequal{#1}{sop\_of\_num}{\hyperref[sailRISCVfnzsopzyofzynum]{#2}}{}%
  \ifstrequal{#1}{sopw_of_num}{\hyperref[sailRISCVfnzsopwzyofzynum]{#2}}{}%
  \ifstrequal{#1}{sopw\_of\_num}{\hyperref[sailRISCVfnzsopwzyofzynum]{#2}}{}%
  \ifstrequal{#1}{spc_backwards}{\hyperref[sailRISCVfnzspczybackwards]{#2}}{}%
  \ifstrequal{#1}{spc\_backwards}{\hyperref[sailRISCVfnzspczybackwards]{#2}}{}%
  \ifstrequal{#1}{spc_forwards}{\hyperref[sailRISCVfnzspczyforwards]{#2}}{}%
  \ifstrequal{#1}{spc\_forwards}{\hyperref[sailRISCVfnzspczyforwards]{#2}}{}%
  \ifstrequal{#1}{spc_matches_prefix}{\hyperref[sailRISCVfnzspczymatcheszyprefix]{#2}}{}%
  \ifstrequal{#1}{spc\_matches\_prefix}{\hyperref[sailRISCVfnzspczymatcheszyprefix]{#2}}{}%
  \ifstrequal{#1}{step}{\hyperref[sailRISCVfnzstep]{#2}}{}%
  \ifstrequal{#1}{string_of_bit}{\hyperref[sailRISCVfnzstringzyofzybit]{#2}}{}%
  \ifstrequal{#1}{string\_of\_bit}{\hyperref[sailRISCVfnzstringzyofzybit]{#2}}{}%
  \ifstrequal{#1}{string_of_capex}{\hyperref[sailRISCVfnzstringzyofzycapex]{#2}}{}%
  \ifstrequal{#1}{string\_of\_capex}{\hyperref[sailRISCVfnzstringzyofzycapex]{#2}}{}%
  \ifstrequal{#1}{tag_addr_to_addr}{\hyperref[sailRISCVfnztagzyaddrzytozyaddr]{#2}}{}%
  \ifstrequal{#1}{tag\_addr\_to\_addr}{\hyperref[sailRISCVfnztagzyaddrzytozyaddr]{#2}}{}%
  \ifstrequal{#1}{tick_clock}{\hyperref[sailRISCVfnztickzyclock]{#2}}{}%
  \ifstrequal{#1}{tick\_clock}{\hyperref[sailRISCVfnztickzyclock]{#2}}{}%
  \ifstrequal{#1}{tick_pc}{\hyperref[sailRISCVfnztickzypc]{#2}}{}%
  \ifstrequal{#1}{tick\_pc}{\hyperref[sailRISCVfnztickzypc]{#2}}{}%
  \ifstrequal{#1}{tick_platform}{\hyperref[sailRISCVfnztickzyplatform]{#2}}{}%
  \ifstrequal{#1}{tick\_platform}{\hyperref[sailRISCVfnztickzyplatform]{#2}}{}%
  \ifstrequal{#1}{to_bits}{\hyperref[sailRISCVfnztozybits]{#2}}{}%
  \ifstrequal{#1}{to\_bits}{\hyperref[sailRISCVfnztozybits]{#2}}{}%
  \ifstrequal{#1}{trans_kind_of_num}{\hyperref[sailRISCVfnztranszykindzyofzynum]{#2}}{}%
  \ifstrequal{#1}{trans\_kind\_of\_num}{\hyperref[sailRISCVfnztranszykindzyofzynum]{#2}}{}%
  \ifstrequal{#1}{translate39}{\hyperref[sailRISCVfnztranslate39]{#2}}{}%
  \ifstrequal{#1}{translate48}{\hyperref[sailRISCVfnztranslate48]{#2}}{}%
  \ifstrequal{#1}{translateAddr}{\hyperref[sailRISCVfnztranslateAddr]{#2}}{}%
  \ifstrequal{#1}{translateAddr_priv}{\hyperref[sailRISCVfnztranslateAddrzypriv]{#2}}{}%
  \ifstrequal{#1}{translateAddr\_priv}{\hyperref[sailRISCVfnztranslateAddrzypriv]{#2}}{}%
  \ifstrequal{#1}{translationException}{\hyperref[sailRISCVfnztranslationException]{#2}}{}%
  \ifstrequal{#1}{translationMode}{\hyperref[sailRISCVfnztranslationMode]{#2}}{}%
  \ifstrequal{#1}{trapVectorMode_of_bits}{\hyperref[sailRISCVfnztrapVectorModezyofzybits]{#2}}{}%
  \ifstrequal{#1}{trapVectorMode\_of\_bits}{\hyperref[sailRISCVfnztrapVectorModezyofzybits]{#2}}{}%
  \ifstrequal{#1}{trap_handler}{\hyperref[sailRISCVfnztrapzyhandler]{#2}}{}%
  \ifstrequal{#1}{trap\_handler}{\hyperref[sailRISCVfnztrapzyhandler]{#2}}{}%
  \ifstrequal{#1}{tval}{\hyperref[sailRISCVfnztval]{#2}}{}%
  \ifstrequal{#1}{tvec_addr}{\hyperref[sailRISCVfnztveczyaddr]{#2}}{}%
  \ifstrequal{#1}{tvec\_addr}{\hyperref[sailRISCVfnztveczyaddr]{#2}}{}%
  \ifstrequal{#1}{ufFlag}{\hyperref[sailRISCVfnzufFlag]{#2}}{}%
  \ifstrequal{#1}{unsealCap}{\hyperref[sailRISCVfnzunsealCap]{#2}}{}%
  \ifstrequal{#1}{uop_of_num}{\hyperref[sailRISCVfnzuopzyofzynum]{#2}}{}%
  \ifstrequal{#1}{uop\_of\_num}{\hyperref[sailRISCVfnzuopzyofzynum]{#2}}{}%
  \ifstrequal{#1}{update_PTE_Bits}{\hyperref[sailRISCVfnzupdatezyPTEzyBits]{#2}}{}%
  \ifstrequal{#1}{update\_PTE\_Bits}{\hyperref[sailRISCVfnzupdatezyPTEzyBits]{#2}}{}%
  \ifstrequal{#1}{update_softfloat_fflags}{\hyperref[sailRISCVfnzupdatezysoftfloatzyfflags]{#2}}{}%
  \ifstrequal{#1}{update\_softfloat\_fflags}{\hyperref[sailRISCVfnzupdatezysoftfloatzyfflags]{#2}}{}%
  \ifstrequal{#1}{validDoubleRegs}{\hyperref[sailRISCVfnzvalidDoubleRegs]{#2}}{}%
  \ifstrequal{#1}{valid_rounding_mode}{\hyperref[sailRISCVfnzvalidzyroundingzymode]{#2}}{}%
  \ifstrequal{#1}{valid\_rounding\_mode}{\hyperref[sailRISCVfnzvalidzyroundingzymode]{#2}}{}%
  \ifstrequal{#1}{wC}{\hyperref[sailRISCVfnzwC]{#2}}{}%
  \ifstrequal{#1}{wC_bits}{\hyperref[sailRISCVfnzwCzybits]{#2}}{}%
  \ifstrequal{#1}{wC\_bits}{\hyperref[sailRISCVfnzwCzybits]{#2}}{}%
  \ifstrequal{#1}{wF}{\hyperref[sailRISCVfnzwF]{#2}}{}%
  \ifstrequal{#1}{wF_bits}{\hyperref[sailRISCVfnzwFzybits]{#2}}{}%
  \ifstrequal{#1}{wF\_bits}{\hyperref[sailRISCVfnzwFzybits]{#2}}{}%
  \ifstrequal{#1}{wF_or_X_D}{\hyperref[sailRISCVfnzwFzyorzyXzyD]{#2}}{}%
  \ifstrequal{#1}{wF\_or\_X\_D}{\hyperref[sailRISCVfnzwFzyorzyXzyD]{#2}}{}%
  \ifstrequal{#1}{wF_or_X_H}{\hyperref[sailRISCVfnzwFzyorzyXzyH]{#2}}{}%
  \ifstrequal{#1}{wF\_or\_X\_H}{\hyperref[sailRISCVfnzwFzyorzyXzyH]{#2}}{}%
  \ifstrequal{#1}{wF_or_X_S}{\hyperref[sailRISCVfnzwFzyorzyXzyS]{#2}}{}%
  \ifstrequal{#1}{wF\_or\_X\_S}{\hyperref[sailRISCVfnzwFzyorzyXzyS]{#2}}{}%
  \ifstrequal{#1}{wX}{\hyperref[sailRISCVfnzwX]{#2}}{}%
  \ifstrequal{#1}{wX_bits}{\hyperref[sailRISCVfnzwXzybits]{#2}}{}%
  \ifstrequal{#1}{wX\_bits}{\hyperref[sailRISCVfnzwXzybits]{#2}}{}%
  \ifstrequal{#1}{walk39}{\hyperref[sailRISCVfnzwalk39]{#2}}{}%
  \ifstrequal{#1}{walk48}{\hyperref[sailRISCVfnzwalk48]{#2}}{}%
  \ifstrequal{#1}{within_clint}{\hyperref[sailRISCVfnzwithinzyclint]{#2}}{}%
  \ifstrequal{#1}{within\_clint}{\hyperref[sailRISCVfnzwithinzyclint]{#2}}{}%
  \ifstrequal{#1}{within_htif_readable}{\hyperref[sailRISCVfnzwithinzyhtifzyreadable]{#2}}{}%
  \ifstrequal{#1}{within\_htif\_readable}{\hyperref[sailRISCVfnzwithinzyhtifzyreadable]{#2}}{}%
  \ifstrequal{#1}{within_htif_writable}{\hyperref[sailRISCVfnzwithinzyhtifzywritable]{#2}}{}%
  \ifstrequal{#1}{within\_htif\_writable}{\hyperref[sailRISCVfnzwithinzyhtifzywritable]{#2}}{}%
  \ifstrequal{#1}{within_mmio_readable}{\hyperref[sailRISCVfnzwithinzymmiozyreadable]{#2}}{}%
  \ifstrequal{#1}{within\_mmio\_readable}{\hyperref[sailRISCVfnzwithinzymmiozyreadable]{#2}}{}%
  \ifstrequal{#1}{within_mmio_writable}{\hyperref[sailRISCVfnzwithinzymmiozywritable]{#2}}{}%
  \ifstrequal{#1}{within\_mmio\_writable}{\hyperref[sailRISCVfnzwithinzymmiozywritable]{#2}}{}%
  \ifstrequal{#1}{within_phys_mem}{\hyperref[sailRISCVfnzwithinzyphyszymem]{#2}}{}%
  \ifstrequal{#1}{within\_phys\_mem}{\hyperref[sailRISCVfnzwithinzyphyszymem]{#2}}{}%
  \ifstrequal{#1}{word_width_bytes}{\hyperref[sailRISCVfnzwordzywidthzybytes]{#2}}{}%
  \ifstrequal{#1}{word\_width\_bytes}{\hyperref[sailRISCVfnzwordzywidthzybytes]{#2}}{}%
  \ifstrequal{#1}{word_width_of_num}{\hyperref[sailRISCVfnzwordzywidthzyofzynum]{#2}}{}%
  \ifstrequal{#1}{word\_width\_of\_num}{\hyperref[sailRISCVfnzwordzywidthzyofzynum]{#2}}{}%
  \ifstrequal{#1}{writeCSR}{\hyperref[sailRISCVfnzwriteCSR]{#2}}{}%
  \ifstrequal{#1}{write_TLB39}{\hyperref[sailRISCVfnzwritezyTLB39]{#2}}{}%
  \ifstrequal{#1}{write\_TLB39}{\hyperref[sailRISCVfnzwritezyTLB39]{#2}}{}%
  \ifstrequal{#1}{write_TLB48}{\hyperref[sailRISCVfnzwritezyTLB48]{#2}}{}%
  \ifstrequal{#1}{write\_TLB48}{\hyperref[sailRISCVfnzwritezyTLB48]{#2}}{}%
  \ifstrequal{#1}{write_fflags}{\hyperref[sailRISCVfnzwritezyfflags]{#2}}{}%
  \ifstrequal{#1}{write\_fflags}{\hyperref[sailRISCVfnzwritezyfflags]{#2}}{}%
  \ifstrequal{#1}{write_kind_of_num}{\hyperref[sailRISCVfnzwritezykindzyofzynum]{#2}}{}%
  \ifstrequal{#1}{write\_kind\_of\_num}{\hyperref[sailRISCVfnzwritezykindzyofzynum]{#2}}{}%
  \ifstrequal{#1}{write_ram}{\hyperref[sailRISCVfnzwritezyram]{#2}}{}%
  \ifstrequal{#1}{write\_ram}{\hyperref[sailRISCVfnzwritezyram]{#2}}{}%
  \ifstrequal{#1}{write_ram_ea}{\hyperref[sailRISCVfnzwritezyramzyea]{#2}}{}%
  \ifstrequal{#1}{write\_ram\_ea}{\hyperref[sailRISCVfnzwritezyramzyea]{#2}}{}%
  \ifstrequal{#1}{write_sc_cap_result}{\hyperref[sailRISCVfnzwritezysczycapzyresult]{#2}}{}%
  \ifstrequal{#1}{write\_sc\_cap\_result}{\hyperref[sailRISCVfnzwritezysczycapzyresult]{#2}}{}%
  \ifstrequal{#1}{write_seed_csr}{\hyperref[sailRISCVfnzwritezyseedzycsr]{#2}}{}%
  \ifstrequal{#1}{write\_seed\_csr}{\hyperref[sailRISCVfnzwritezyseedzycsr]{#2}}{}%
  \ifstrequal{#1}{zeros_implicit}{\hyperref[sailRISCVfnzzzeroszyimplicit]{#2}}{}%
  \ifstrequal{#1}{zeros\_implicit}{\hyperref[sailRISCVfnzzzeroszyimplicit]{#2}}{}%
  \ifstrequal{#1}{(operator <=_u)}{\hyperref[sailRISCVfnzz8operatorz0zIzJzyuz9]{#2}}{}%
  \ifstrequal{#1}{(operator $>$=\_u)}{\hyperref[sailRISCVfnzz8operatorz0zIzJzyuz9]{#2}}{}%
  \ifstrequal{#1}{(operator <_s)}{\hyperref[sailRISCVfnzz8operatorz0zIzysz9]{#2}}{}%
  \ifstrequal{#1}{(operator $>$\_s)}{\hyperref[sailRISCVfnzz8operatorz0zIzysz9]{#2}}{}%
  \ifstrequal{#1}{(operator <_u)}{\hyperref[sailRISCVfnzz8operatorz0zIzyuz9]{#2}}{}%
  \ifstrequal{#1}{(operator $>$\_u)}{\hyperref[sailRISCVfnzz8operatorz0zIzyuz9]{#2}}{}%
  \ifstrequal{#1}{(operator >=_s)}{\hyperref[sailRISCVfnzz8operatorz0zKzJzysz9]{#2}}{}%
  \ifstrequal{#1}{(operator $$>$$=\_s)}{\hyperref[sailRISCVfnzz8operatorz0zKzJzysz9]{#2}}{}%
  \ifstrequal{#1}{(operator >=_u)}{\hyperref[sailRISCVfnzz8operatorz0zKzJzyuz9]{#2}}{}%
  \ifstrequal{#1}{(operator $$>$$=\_u)}{\hyperref[sailRISCVfnzz8operatorz0zKzJzyuz9]{#2}}{}}

\newcommand{\sailRISCVtype}[1]{
  \ifstrequal{#1}{AccessType}{\sailRISCVtypeAccessType}{}%
  \ifstrequal{#1}{Architecture}{\sailRISCVtypeArchitecture}{}%
  \ifstrequal{#1}{CPtrCmpOp}{\sailRISCVtypeCPtrCmpOp}{}%
  \ifstrequal{#1}{CapAddrBits}{\sailRISCVtypeCapAddrBits}{}%
  \ifstrequal{#1}{CapAddrInt}{\sailRISCVtypeCapAddrInt}{}%
  \ifstrequal{#1}{CapBits}{\sailRISCVtypeCapBits}{}%
  \ifstrequal{#1}{CapEx}{\sailRISCVtypeCapEx}{}%
  \ifstrequal{#1}{CapFlagsBits}{\sailRISCVtypeCapFlagsBits}{}%
  \ifstrequal{#1}{CapLen}{\sailRISCVtypeCapLen}{}%
  \ifstrequal{#1}{CapLenBits}{\sailRISCVtypeCapLenBits}{}%
  \ifstrequal{#1}{CapPermsBits}{\sailRISCVtypeCapPermsBits}{}%
  \ifstrequal{#1}{Capability}{\sailRISCVtypeCapability}{}%
  \ifstrequal{#1}{ClearRegSet}{\sailRISCVtypeClearRegSet}{}%
  \ifstrequal{#1}{Counteren}{\sailRISCVtypeCounteren}{}%
  \ifstrequal{#1}{Counterin}{\sailRISCVtypeCounterin}{}%
  \ifstrequal{#1}{EncCapability}{\sailRISCVtypeEncCapability}{}%
  \ifstrequal{#1}{ExceptionType}{\sailRISCVtypeExceptionType}{}%
  \ifstrequal{#1}{ExtStatus}{\sailRISCVtypeExtStatus}{}%
  \ifstrequal{#1}{Ext_ControlAddr_Check}{\sailRISCVtypeExtControlAddrCheck}{}%
  \ifstrequal{#1}{Ext\_ControlAddr\_Check}{\sailRISCVtypeExtControlAddrCheck}{}%
  \ifstrequal{#1}{Ext_DataAddr_Check}{\sailRISCVtypeExtDataAddrCheck}{}%
  \ifstrequal{#1}{Ext\_DataAddr\_Check}{\sailRISCVtypeExtDataAddrCheck}{}%
  \ifstrequal{#1}{Ext_FetchAddr_Check}{\sailRISCVtypeExtFetchAddrCheck}{}%
  \ifstrequal{#1}{Ext\_FetchAddr\_Check}{\sailRISCVtypeExtFetchAddrCheck}{}%
  \ifstrequal{#1}{Ext_PTE_Bits}{\sailRISCVtypeExtPTEBits}{}%
  \ifstrequal{#1}{Ext\_PTE\_Bits}{\sailRISCVtypeExtPTEBits}{}%
  \ifstrequal{#1}{Ext_PhysAddr_Check}{\sailRISCVtypeExtPhysAddrCheck}{}%
  \ifstrequal{#1}{Ext\_PhysAddr\_Check}{\sailRISCVtypeExtPhysAddrCheck}{}%
  \ifstrequal{#1}{Fcsr}{\sailRISCVtypeFcsr}{}%
  \ifstrequal{#1}{FetchResult}{\sailRISCVtypeFetchResult}{}%
  \ifstrequal{#1}{InterruptType}{\sailRISCVtypeInterruptType}{}%
  \ifstrequal{#1}{Mcause}{\sailRISCVtypeMcause}{}%
  \ifstrequal{#1}{Medeleg}{\sailRISCVtypeMedeleg}{}%
  \ifstrequal{#1}{MemoryOpResult}{\sailRISCVtypeMemoryOpResult}{}%
  \ifstrequal{#1}{Minterrupts}{\sailRISCVtypeMinterrupts}{}%
  \ifstrequal{#1}{Misa}{\sailRISCVtypeMisa}{}%
  \ifstrequal{#1}{Mstatus}{\sailRISCVtypeMstatus}{}%
  \ifstrequal{#1}{Mstatush}{\sailRISCVtypeMstatush}{}%
  \ifstrequal{#1}{Mtvec}{\sailRISCVtypeMtvec}{}%
  \ifstrequal{#1}{PTE_Bits}{\sailRISCVtypePTEBits}{}%
  \ifstrequal{#1}{PTE\_Bits}{\sailRISCVtypePTEBits}{}%
  \ifstrequal{#1}{PTE_Check}{\sailRISCVtypePTECheck}{}%
  \ifstrequal{#1}{PTE\_Check}{\sailRISCVtypePTECheck}{}%
  \ifstrequal{#1}{PTW_Error}{\sailRISCVtypePTWError}{}%
  \ifstrequal{#1}{PTW\_Error}{\sailRISCVtypePTWError}{}%
  \ifstrequal{#1}{PTW_Result}{\sailRISCVtypePTWResult}{}%
  \ifstrequal{#1}{PTW\_Result}{\sailRISCVtypePTWResult}{}%
  \ifstrequal{#1}{PmpAddrMatchType}{\sailRISCVtypePmpAddrMatchType}{}%
  \ifstrequal{#1}{Pmpcfg_ent}{\sailRISCVtypePmpcfgEnt}{}%
  \ifstrequal{#1}{Pmpcfg\_ent}{\sailRISCVtypePmpcfgEnt}{}%
  \ifstrequal{#1}{Privilege}{\sailRISCVtypePrivilege}{}%
  \ifstrequal{#1}{Retired}{\sailRISCVtypeRetired}{}%
  \ifstrequal{#1}{SATPMode}{\sailRISCVtypeSATPMode}{}%
  \ifstrequal{#1}{SV32_PTE}{\sailRISCVtypeSVThreeTwoPTE}{}%
  \ifstrequal{#1}{SV32\_PTE}{\sailRISCVtypeSVThreeTwoPTE}{}%
  \ifstrequal{#1}{SV32_Paddr}{\sailRISCVtypeSVThreeTwoPaddr}{}%
  \ifstrequal{#1}{SV32\_Paddr}{\sailRISCVtypeSVThreeTwoPaddr}{}%
  \ifstrequal{#1}{SV32_Vaddr}{\sailRISCVtypeSVThreeTwoVaddr}{}%
  \ifstrequal{#1}{SV32\_Vaddr}{\sailRISCVtypeSVThreeTwoVaddr}{}%
  \ifstrequal{#1}{SV39_PTE}{\sailRISCVtypeSVThreeNinePTE}{}%
  \ifstrequal{#1}{SV39\_PTE}{\sailRISCVtypeSVThreeNinePTE}{}%
  \ifstrequal{#1}{SV39_Paddr}{\sailRISCVtypeSVThreeNinePaddr}{}%
  \ifstrequal{#1}{SV39\_Paddr}{\sailRISCVtypeSVThreeNinePaddr}{}%
  \ifstrequal{#1}{SV39_Vaddr}{\sailRISCVtypeSVThreeNineVaddr}{}%
  \ifstrequal{#1}{SV39\_Vaddr}{\sailRISCVtypeSVThreeNineVaddr}{}%
  \ifstrequal{#1}{SV48_PTE}{\sailRISCVtypeSVFourEightPTE}{}%
  \ifstrequal{#1}{SV48\_PTE}{\sailRISCVtypeSVFourEightPTE}{}%
  \ifstrequal{#1}{SV48_Paddr}{\sailRISCVtypeSVFourEightPaddr}{}%
  \ifstrequal{#1}{SV48\_Paddr}{\sailRISCVtypeSVFourEightPaddr}{}%
  \ifstrequal{#1}{SV48_Vaddr}{\sailRISCVtypeSVFourEightVaddr}{}%
  \ifstrequal{#1}{SV48\_Vaddr}{\sailRISCVtypeSVFourEightVaddr}{}%
  \ifstrequal{#1}{Satp32}{\sailRISCVtypeSatpThreeTwo}{}%
  \ifstrequal{#1}{Satp64}{\sailRISCVtypeSatpSixFour}{}%
  \ifstrequal{#1}{Sedeleg}{\sailRISCVtypeSedeleg}{}%
  \ifstrequal{#1}{Sinterrupts}{\sailRISCVtypeSinterrupts}{}%
  \ifstrequal{#1}{Sstatus}{\sailRISCVtypeSstatus}{}%
  \ifstrequal{#1}{TLB39_Entry}{\sailRISCVtypeTLBThreeNineEntry}{}%
  \ifstrequal{#1}{TLB39\_Entry}{\sailRISCVtypeTLBThreeNineEntry}{}%
  \ifstrequal{#1}{TLB48_Entry}{\sailRISCVtypeTLBFourEightEntry}{}%
  \ifstrequal{#1}{TLB48\_Entry}{\sailRISCVtypeTLBFourEightEntry}{}%
  \ifstrequal{#1}{TLB_Entry}{\sailRISCVtypeTLBEntry}{}%
  \ifstrequal{#1}{TLB\_Entry}{\sailRISCVtypeTLBEntry}{}%
  \ifstrequal{#1}{TR_Result}{\sailRISCVtypeTRResult}{}%
  \ifstrequal{#1}{TR\_Result}{\sailRISCVtypeTRResult}{}%
  \ifstrequal{#1}{TrapVectorMode}{\sailRISCVtypeTrapVectorMode}{}%
  \ifstrequal{#1}{Uinterrupts}{\sailRISCVtypeUinterrupts}{}%
  \ifstrequal{#1}{Ustatus}{\sailRISCVtypeUstatus}{}%
  \ifstrequal{#1}{a64_barrier_domain}{\sailRISCVtypeaSixFourBarrierDomain}{}%
  \ifstrequal{#1}{a64\_barrier\_domain}{\sailRISCVtypeaSixFourBarrierDomain}{}%
  \ifstrequal{#1}{a64_barrier_type}{\sailRISCVtypeaSixFourBarrierType}{}%
  \ifstrequal{#1}{a64\_barrier\_type}{\sailRISCVtypeaSixFourBarrierType}{}%
  \ifstrequal{#1}{amo}{\sailRISCVtypeamo}{}%
  \ifstrequal{#1}{amoop}{\sailRISCVtypeamoop}{}%
  \ifstrequal{#1}{arch_xlen}{\sailRISCVtypearchXlen}{}%
  \ifstrequal{#1}{arch\_xlen}{\sailRISCVtypearchXlen}{}%
  \ifstrequal{#1}{asid32}{\sailRISCVtypeasidThreeTwo}{}%
  \ifstrequal{#1}{asid64}{\sailRISCVtypeasidSixFour}{}%
  \ifstrequal{#1}{ast}{\sailRISCVtypeast}{}%
  \ifstrequal{#1}{barrier_kind}{\sailRISCVtypebarrierKind}{}%
  \ifstrequal{#1}{barrier\_kind}{\sailRISCVtypebarrierKind}{}%
  \ifstrequal{#1}{biop_zbs}{\sailRISCVtypebiopZbs}{}%
  \ifstrequal{#1}{biop\_zbs}{\sailRISCVtypebiopZbs}{}%
  \ifstrequal{#1}{bits}{\sailRISCVtypebits}{}%
  \ifstrequal{#1}{bits_D}{\sailRISCVtypebitsD}{}%
  \ifstrequal{#1}{bits\_D}{\sailRISCVtypebitsD}{}%
  \ifstrequal{#1}{bits_H}{\sailRISCVtypebitsH}{}%
  \ifstrequal{#1}{bits\_H}{\sailRISCVtypebitsH}{}%
  \ifstrequal{#1}{bits_L}{\sailRISCVtypebitsL}{}%
  \ifstrequal{#1}{bits\_L}{\sailRISCVtypebitsL}{}%
  \ifstrequal{#1}{bits_LU}{\sailRISCVtypebitsLU}{}%
  \ifstrequal{#1}{bits\_LU}{\sailRISCVtypebitsLU}{}%
  \ifstrequal{#1}{bits_S}{\sailRISCVtypebitsS}{}%
  \ifstrequal{#1}{bits\_S}{\sailRISCVtypebitsS}{}%
  \ifstrequal{#1}{bits_W}{\sailRISCVtypebitsW}{}%
  \ifstrequal{#1}{bits\_W}{\sailRISCVtypebitsW}{}%
  \ifstrequal{#1}{bits_WU}{\sailRISCVtypebitsWU}{}%
  \ifstrequal{#1}{bits\_WU}{\sailRISCVtypebitsWU}{}%
  \ifstrequal{#1}{bits_fflags}{\sailRISCVtypebitsFflags}{}%
  \ifstrequal{#1}{bits\_fflags}{\sailRISCVtypebitsFflags}{}%
  \ifstrequal{#1}{bits_rm}{\sailRISCVtypebitsRm}{}%
  \ifstrequal{#1}{bits\_rm}{\sailRISCVtypebitsRm}{}%
  \ifstrequal{#1}{bop}{\sailRISCVtypebop}{}%
  \ifstrequal{#1}{brop_zba}{\sailRISCVtypebropZba}{}%
  \ifstrequal{#1}{brop\_zba}{\sailRISCVtypebropZba}{}%
  \ifstrequal{#1}{brop_zbb}{\sailRISCVtypebropZbb}{}%
  \ifstrequal{#1}{brop\_zbb}{\sailRISCVtypebropZbb}{}%
  \ifstrequal{#1}{brop_zbkb}{\sailRISCVtypebropZbkb}{}%
  \ifstrequal{#1}{brop\_zbkb}{\sailRISCVtypebropZbkb}{}%
  \ifstrequal{#1}{brop_zbs}{\sailRISCVtypebropZbs}{}%
  \ifstrequal{#1}{brop\_zbs}{\sailRISCVtypebropZbs}{}%
  \ifstrequal{#1}{bropw_zba}{\sailRISCVtypebropwZba}{}%
  \ifstrequal{#1}{bropw\_zba}{\sailRISCVtypebropwZba}{}%
  \ifstrequal{#1}{bropw_zbb}{\sailRISCVtypebropwZbb}{}%
  \ifstrequal{#1}{bropw\_zbb}{\sailRISCVtypebropwZbb}{}%
  \ifstrequal{#1}{cache_op_kind}{\sailRISCVtypecacheOpKind}{}%
  \ifstrequal{#1}{cache\_op\_kind}{\sailRISCVtypecacheOpKind}{}%
  \ifstrequal{#1}{cap_E_width}{\sailRISCVtypecapEWidth}{}%
  \ifstrequal{#1}{cap\_E\_width}{\sailRISCVtypecapEWidth}{}%
  \ifstrequal{#1}{cap_addr_width}{\sailRISCVtypecapAddrWidth}{}%
  \ifstrequal{#1}{cap\_addr\_width}{\sailRISCVtypecapAddrWidth}{}%
  \ifstrequal{#1}{cap_flags_width}{\sailRISCVtypecapFlagsWidth}{}%
  \ifstrequal{#1}{cap\_flags\_width}{\sailRISCVtypecapFlagsWidth}{}%
  \ifstrequal{#1}{cap_hperms_width}{\sailRISCVtypecapHpermsWidth}{}%
  \ifstrequal{#1}{cap\_hperms\_width}{\sailRISCVtypecapHpermsWidth}{}%
  \ifstrequal{#1}{cap_len_width}{\sailRISCVtypecapLenWidth}{}%
  \ifstrequal{#1}{cap\_len\_width}{\sailRISCVtypecapLenWidth}{}%
  \ifstrequal{#1}{cap_mantissa_width}{\sailRISCVtypecapMantissaWidth}{}%
  \ifstrequal{#1}{cap\_mantissa\_width}{\sailRISCVtypecapMantissaWidth}{}%
  \ifstrequal{#1}{cap_otype_width}{\sailRISCVtypecapOtypeWidth}{}%
  \ifstrequal{#1}{cap\_otype\_width}{\sailRISCVtypecapOtypeWidth}{}%
  \ifstrequal{#1}{cap_perms_width}{\sailRISCVtypecapPermsWidth}{}%
  \ifstrequal{#1}{cap\_perms\_width}{\sailRISCVtypecapPermsWidth}{}%
  \ifstrequal{#1}{cap_reserved_width}{\sailRISCVtypecapReservedWidth}{}%
  \ifstrequal{#1}{cap\_reserved\_width}{\sailRISCVtypecapReservedWidth}{}%
  \ifstrequal{#1}{cap_size}{\sailRISCVtypecapSizze}{}%
  \ifstrequal{#1}{cap\_size}{\sailRISCVtypecapSizze}{}%
  \ifstrequal{#1}{cap_uperms_shift}{\sailRISCVtypecapUpermsShift}{}%
  \ifstrequal{#1}{cap\_uperms\_shift}{\sailRISCVtypecapUpermsShift}{}%
  \ifstrequal{#1}{cap_uperms_width}{\sailRISCVtypecapUpermsWidth}{}%
  \ifstrequal{#1}{cap\_uperms\_width}{\sailRISCVtypecapUpermsWidth}{}%
  \ifstrequal{#1}{capreg_idx}{\sailRISCVtypecapregIdx}{}%
  \ifstrequal{#1}{capreg\_idx}{\sailRISCVtypecapregIdx}{}%
  \ifstrequal{#1}{caps_per_cache_line}{\sailRISCVtypecapsPerCacheLine}{}%
  \ifstrequal{#1}{caps\_per\_cache\_line}{\sailRISCVtypecapsPerCacheLine}{}%
  \ifstrequal{#1}{ccsr}{\sailRISCVtypeccsr}{}%
  \ifstrequal{#1}{cregidx}{\sailRISCVtypecregidx}{}%
  \ifstrequal{#1}{csrRW}{\sailRISCVtypecsrRW}{}%
  \ifstrequal{#1}{csreg}{\sailRISCVtypecsreg}{}%
  \ifstrequal{#1}{csrop}{\sailRISCVtypecsrop}{}%
  \ifstrequal{#1}{ctl_result}{\sailRISCVtypectlResult}{}%
  \ifstrequal{#1}{ctl\_result}{\sailRISCVtypectlResult}{}%
  \ifstrequal{#1}{diafp}{\sailRISCVtypediafp}{}%
  \ifstrequal{#1}{exc_code}{\sailRISCVtypeexcCode}{}%
  \ifstrequal{#1}{exc\_code}{\sailRISCVtypeexcCode}{}%
  \ifstrequal{#1}{exception}{\sailRISCVtypeexception}{}%
  \ifstrequal{#1}{extPte}{\sailRISCVtypeextPte}{}%
  \ifstrequal{#1}{ext_access_type}{\sailRISCVtypeextAccessType}{}%
  \ifstrequal{#1}{ext\_access\_type}{\sailRISCVtypeextAccessType}{}%
  \ifstrequal{#1}{ext_control_addr_error}{\sailRISCVtypeextControlAddrError}{}%
  \ifstrequal{#1}{ext\_control\_addr\_error}{\sailRISCVtypeextControlAddrError}{}%
  \ifstrequal{#1}{ext_data_addr_error}{\sailRISCVtypeextDataAddrError}{}%
  \ifstrequal{#1}{ext\_data\_addr\_error}{\sailRISCVtypeextDataAddrError}{}%
  \ifstrequal{#1}{ext_exc_type}{\sailRISCVtypeextExcType}{}%
  \ifstrequal{#1}{ext\_exc\_type}{\sailRISCVtypeextExcType}{}%
  \ifstrequal{#1}{ext_exception}{\sailRISCVtypeextException}{}%
  \ifstrequal{#1}{ext\_exception}{\sailRISCVtypeextException}{}%
  \ifstrequal{#1}{ext_fetch_addr_error}{\sailRISCVtypeextFetchAddrError}{}%
  \ifstrequal{#1}{ext\_fetch\_addr\_error}{\sailRISCVtypeextFetchAddrError}{}%
  \ifstrequal{#1}{ext_ptw}{\sailRISCVtypeextPtw}{}%
  \ifstrequal{#1}{ext\_ptw}{\sailRISCVtypeextPtw}{}%
  \ifstrequal{#1}{ext_ptw_error}{\sailRISCVtypeextPtwError}{}%
  \ifstrequal{#1}{ext\_ptw\_error}{\sailRISCVtypeextPtwError}{}%
  \ifstrequal{#1}{ext_ptw_fail}{\sailRISCVtypeextPtwFail}{}%
  \ifstrequal{#1}{ext\_ptw\_fail}{\sailRISCVtypeextPtwFail}{}%
  \ifstrequal{#1}{ext_ptw_lc}{\sailRISCVtypeextPtwLc}{}%
  \ifstrequal{#1}{ext\_ptw\_lc}{\sailRISCVtypeextPtwLc}{}%
  \ifstrequal{#1}{ext_ptw_sc}{\sailRISCVtypeextPtwSc}{}%
  \ifstrequal{#1}{ext\_ptw\_sc}{\sailRISCVtypeextPtwSc}{}%
  \ifstrequal{#1}{ext_status}{\sailRISCVtypeextStatus}{}%
  \ifstrequal{#1}{ext\_status}{\sailRISCVtypeextStatus}{}%
  \ifstrequal{#1}{extop_zbb}{\sailRISCVtypeextopZbb}{}%
  \ifstrequal{#1}{extop\_zbb}{\sailRISCVtypeextopZbb}{}%
  \ifstrequal{#1}{f_bin_op_D}{\sailRISCVtypefBinOpD}{}%
  \ifstrequal{#1}{f\_bin\_op\_D}{\sailRISCVtypefBinOpD}{}%
  \ifstrequal{#1}{f_bin_op_H}{\sailRISCVtypefBinOpH}{}%
  \ifstrequal{#1}{f\_bin\_op\_H}{\sailRISCVtypefBinOpH}{}%
  \ifstrequal{#1}{f_bin_op_S}{\sailRISCVtypefBinOpS}{}%
  \ifstrequal{#1}{f\_bin\_op\_S}{\sailRISCVtypefBinOpS}{}%
  \ifstrequal{#1}{f_bin_rm_op_D}{\sailRISCVtypefBinRmOpD}{}%
  \ifstrequal{#1}{f\_bin\_rm\_op\_D}{\sailRISCVtypefBinRmOpD}{}%
  \ifstrequal{#1}{f_bin_rm_op_H}{\sailRISCVtypefBinRmOpH}{}%
  \ifstrequal{#1}{f\_bin\_rm\_op\_H}{\sailRISCVtypefBinRmOpH}{}%
  \ifstrequal{#1}{f_bin_rm_op_S}{\sailRISCVtypefBinRmOpS}{}%
  \ifstrequal{#1}{f\_bin\_rm\_op\_S}{\sailRISCVtypefBinRmOpS}{}%
  \ifstrequal{#1}{f_madd_op_D}{\sailRISCVtypefMaddOpD}{}%
  \ifstrequal{#1}{f\_madd\_op\_D}{\sailRISCVtypefMaddOpD}{}%
  \ifstrequal{#1}{f_madd_op_H}{\sailRISCVtypefMaddOpH}{}%
  \ifstrequal{#1}{f\_madd\_op\_H}{\sailRISCVtypefMaddOpH}{}%
  \ifstrequal{#1}{f_madd_op_S}{\sailRISCVtypefMaddOpS}{}%
  \ifstrequal{#1}{f\_madd\_op\_S}{\sailRISCVtypefMaddOpS}{}%
  \ifstrequal{#1}{f_un_op_D}{\sailRISCVtypefUnOpD}{}%
  \ifstrequal{#1}{f\_un\_op\_D}{\sailRISCVtypefUnOpD}{}%
  \ifstrequal{#1}{f_un_op_H}{\sailRISCVtypefUnOpH}{}%
  \ifstrequal{#1}{f\_un\_op\_H}{\sailRISCVtypefUnOpH}{}%
  \ifstrequal{#1}{f_un_op_S}{\sailRISCVtypefUnOpS}{}%
  \ifstrequal{#1}{f\_un\_op\_S}{\sailRISCVtypefUnOpS}{}%
  \ifstrequal{#1}{f_un_rm_op_D}{\sailRISCVtypefUnRmOpD}{}%
  \ifstrequal{#1}{f\_un\_rm\_op\_D}{\sailRISCVtypefUnRmOpD}{}%
  \ifstrequal{#1}{f_un_rm_op_H}{\sailRISCVtypefUnRmOpH}{}%
  \ifstrequal{#1}{f\_un\_rm\_op\_H}{\sailRISCVtypefUnRmOpH}{}%
  \ifstrequal{#1}{f_un_rm_op_S}{\sailRISCVtypefUnRmOpS}{}%
  \ifstrequal{#1}{f\_un\_rm\_op\_S}{\sailRISCVtypefUnRmOpS}{}%
  \ifstrequal{#1}{flen}{\sailRISCVtypeflen}{}%
  \ifstrequal{#1}{flen_bytes}{\sailRISCVtypeflenBytes}{}%
  \ifstrequal{#1}{flen\_bytes}{\sailRISCVtypeflenBytes}{}%
  \ifstrequal{#1}{flenbits}{\sailRISCVtypeflenbits}{}%
  \ifstrequal{#1}{fregtype}{\sailRISCVtypefregtype}{}%
  \ifstrequal{#1}{half}{\sailRISCVtypehalf}{}%
  \ifstrequal{#1}{htif_cmd}{\sailRISCVtypehtifCmd}{}%
  \ifstrequal{#1}{htif\_cmd}{\sailRISCVtypehtifCmd}{}%
  \ifstrequal{#1}{imm12}{\sailRISCVtypeimmOneTwo}{}%
  \ifstrequal{#1}{imm20}{\sailRISCVtypeimmTwoZero}{}%
  \ifstrequal{#1}{instruction_kind}{\sailRISCVtypeinstructionKind}{}%
  \ifstrequal{#1}{instruction\_kind}{\sailRISCVtypeinstructionKind}{}%
  \ifstrequal{#1}{internal_E_take_bits}{\sailRISCVtypeinternalETakeBits}{}%
  \ifstrequal{#1}{internal\_E\_take\_bits}{\sailRISCVtypeinternalETakeBits}{}%
  \ifstrequal{#1}{interrupt_set}{\sailRISCVtypeinterruptSet}{}%
  \ifstrequal{#1}{interrupt\_set}{\sailRISCVtypeinterruptSet}{}%
  \ifstrequal{#1}{iop}{\sailRISCVtypeiop}{}%
  \ifstrequal{#1}{log2_cap_size}{\sailRISCVtypelogTwoCapSizze}{}%
  \ifstrequal{#1}{log2\_cap\_size}{\sailRISCVtypelogTwoCapSizze}{}%
  \ifstrequal{#1}{max_mem_access}{\sailRISCVtypemaxMemAccess}{}%
  \ifstrequal{#1}{max\_mem\_access}{\sailRISCVtypemaxMemAccess}{}%
  \ifstrequal{#1}{mem_meta}{\sailRISCVtypememMeta}{}%
  \ifstrequal{#1}{mem\_meta}{\sailRISCVtypememMeta}{}%
  \ifstrequal{#1}{niafp}{\sailRISCVtypeniafp}{}%
  \ifstrequal{#1}{niafps}{\sailRISCVtypeniafps}{}%
  \ifstrequal{#1}{opcode}{\sailRISCVtypeopcode}{}%
  \ifstrequal{#1}{option}{\sailRISCVtypeoption}{}%
  \ifstrequal{#1}{paddr32}{\sailRISCVtypepaddrThreeTwo}{}%
  \ifstrequal{#1}{paddr64}{\sailRISCVtypepaddrSixFour}{}%
  \ifstrequal{#1}{pmpAddrMatch}{\sailRISCVtypepmpAddrMatch}{}%
  \ifstrequal{#1}{pmpMatch}{\sailRISCVtypepmpMatch}{}%
  \ifstrequal{#1}{pmp_addr_range}{\sailRISCVtypepmpAddrRange}{}%
  \ifstrequal{#1}{pmp\_addr\_range}{\sailRISCVtypepmpAddrRange}{}%
  \ifstrequal{#1}{priv_level}{\sailRISCVtypeprivLevel}{}%
  \ifstrequal{#1}{priv\_level}{\sailRISCVtypeprivLevel}{}%
  \ifstrequal{#1}{pte32}{\sailRISCVtypepteThreeTwo}{}%
  \ifstrequal{#1}{pte48}{\sailRISCVtypepteFourEight}{}%
  \ifstrequal{#1}{pte64}{\sailRISCVtypepteSixFour}{}%
  \ifstrequal{#1}{pteAttribs}{\sailRISCVtypepteAttribs}{}%
  \ifstrequal{#1}{read_kind}{\sailRISCVtypereadKind}{}%
  \ifstrequal{#1}{read\_kind}{\sailRISCVtypereadKind}{}%
  \ifstrequal{#1}{regfp}{\sailRISCVtyperegfp}{}%
  \ifstrequal{#1}{regfps}{\sailRISCVtyperegfps}{}%
  \ifstrequal{#1}{regidx}{\sailRISCVtyperegidx}{}%
  \ifstrequal{#1}{regno}{\sailRISCVtyperegno}{}%
  \ifstrequal{#1}{regtype}{\sailRISCVtyperegtype}{}%
  \ifstrequal{#1}{rop}{\sailRISCVtyperop}{}%
  \ifstrequal{#1}{ropw}{\sailRISCVtyperopw}{}%
  \ifstrequal{#1}{rounding_mode}{\sailRISCVtyperoundingMode}{}%
  \ifstrequal{#1}{rounding\_mode}{\sailRISCVtyperoundingMode}{}%
  \ifstrequal{#1}{satp_mode}{\sailRISCVtypesatpMode}{}%
  \ifstrequal{#1}{satp\_mode}{\sailRISCVtypesatpMode}{}%
  \ifstrequal{#1}{screg}{\sailRISCVtypescreg}{}%
  \ifstrequal{#1}{seed_opst}{\sailRISCVtypeseedOpst}{}%
  \ifstrequal{#1}{seed\_opst}{\sailRISCVtypeseedOpst}{}%
  \ifstrequal{#1}{sop}{\sailRISCVtypesop}{}%
  \ifstrequal{#1}{sopw}{\sailRISCVtypesopw}{}%
  \ifstrequal{#1}{sync_exception}{\sailRISCVtypesyncException}{}%
  \ifstrequal{#1}{sync\_exception}{\sailRISCVtypesyncException}{}%
  \ifstrequal{#1}{tagaddrbits}{\sailRISCVtypetagaddrbits}{}%
  \ifstrequal{#1}{trans_kind}{\sailRISCVtypetransKind}{}%
  \ifstrequal{#1}{trans\_kind}{\sailRISCVtypetransKind}{}%
  \ifstrequal{#1}{tv_mode}{\sailRISCVtypetvMode}{}%
  \ifstrequal{#1}{tv\_mode}{\sailRISCVtypetvMode}{}%
  \ifstrequal{#1}{uop}{\sailRISCVtypeuop}{}%
  \ifstrequal{#1}{vaddr32}{\sailRISCVtypevaddrThreeTwo}{}%
  \ifstrequal{#1}{vaddr39}{\sailRISCVtypevaddrThreeNine}{}%
  \ifstrequal{#1}{vaddr48}{\sailRISCVtypevaddrFourEight}{}%
  \ifstrequal{#1}{word}{\sailRISCVtypeword}{}%
  \ifstrequal{#1}{word_width}{\sailRISCVtypewordWidth}{}%
  \ifstrequal{#1}{word\_width}{\sailRISCVtypewordWidth}{}%
  \ifstrequal{#1}{write_kind}{\sailRISCVtypewriteKind}{}%
  \ifstrequal{#1}{write\_kind}{\sailRISCVtypewriteKind}{}%
  \ifstrequal{#1}{xlen}{\sailRISCVtypexlen}{}%
  \ifstrequal{#1}{xlen_bytes}{\sailRISCVtypexlenBytes}{}%
  \ifstrequal{#1}{xlen\_bytes}{\sailRISCVtypexlenBytes}{}%
  \ifstrequal{#1}{xlenbits}{\sailRISCVtypexlenbits}{}}

\newcommand{\sailRISCVreftype}[2]{
  \ifstrequal{#1}{AccessType}{\hyperref[sailRISCVtypezAccessType]{#2}}{}%
  \ifstrequal{#1}{Architecture}{\hyperref[sailRISCVtypezArchitecture]{#2}}{}%
  \ifstrequal{#1}{CPtrCmpOp}{\hyperref[sailRISCVtypezCPtrCmpOp]{#2}}{}%
  \ifstrequal{#1}{CapAddrBits}{\hyperref[sailRISCVtypezCapAddrBits]{#2}}{}%
  \ifstrequal{#1}{CapAddrInt}{\hyperref[sailRISCVtypezCapAddrInt]{#2}}{}%
  \ifstrequal{#1}{CapBits}{\hyperref[sailRISCVtypezCapBits]{#2}}{}%
  \ifstrequal{#1}{CapEx}{\hyperref[sailRISCVtypezCapEx]{#2}}{}%
  \ifstrequal{#1}{CapFlagsBits}{\hyperref[sailRISCVtypezCapFlagsBits]{#2}}{}%
  \ifstrequal{#1}{CapLen}{\hyperref[sailRISCVtypezCapLen]{#2}}{}%
  \ifstrequal{#1}{CapLenBits}{\hyperref[sailRISCVtypezCapLenBits]{#2}}{}%
  \ifstrequal{#1}{CapPermsBits}{\hyperref[sailRISCVtypezCapPermsBits]{#2}}{}%
  \ifstrequal{#1}{Capability}{\hyperref[sailRISCVtypezCapability]{#2}}{}%
  \ifstrequal{#1}{ClearRegSet}{\hyperref[sailRISCVtypezClearRegSet]{#2}}{}%
  \ifstrequal{#1}{Counteren}{\hyperref[sailRISCVtypezCounteren]{#2}}{}%
  \ifstrequal{#1}{Counterin}{\hyperref[sailRISCVtypezCounterin]{#2}}{}%
  \ifstrequal{#1}{EncCapability}{\hyperref[sailRISCVtypezEncCapability]{#2}}{}%
  \ifstrequal{#1}{ExceptionType}{\hyperref[sailRISCVtypezExceptionType]{#2}}{}%
  \ifstrequal{#1}{ExtStatus}{\hyperref[sailRISCVtypezExtStatus]{#2}}{}%
  \ifstrequal{#1}{Ext_ControlAddr_Check}{\hyperref[sailRISCVtypezExtzyControlAddrzyCheck]{#2}}{}%
  \ifstrequal{#1}{Ext\_ControlAddr\_Check}{\hyperref[sailRISCVtypezExtzyControlAddrzyCheck]{#2}}{}%
  \ifstrequal{#1}{Ext_DataAddr_Check}{\hyperref[sailRISCVtypezExtzyDataAddrzyCheck]{#2}}{}%
  \ifstrequal{#1}{Ext\_DataAddr\_Check}{\hyperref[sailRISCVtypezExtzyDataAddrzyCheck]{#2}}{}%
  \ifstrequal{#1}{Ext_FetchAddr_Check}{\hyperref[sailRISCVtypezExtzyFetchAddrzyCheck]{#2}}{}%
  \ifstrequal{#1}{Ext\_FetchAddr\_Check}{\hyperref[sailRISCVtypezExtzyFetchAddrzyCheck]{#2}}{}%
  \ifstrequal{#1}{Ext_PTE_Bits}{\hyperref[sailRISCVtypezExtzyPTEzyBits]{#2}}{}%
  \ifstrequal{#1}{Ext\_PTE\_Bits}{\hyperref[sailRISCVtypezExtzyPTEzyBits]{#2}}{}%
  \ifstrequal{#1}{Ext_PhysAddr_Check}{\hyperref[sailRISCVtypezExtzyPhysAddrzyCheck]{#2}}{}%
  \ifstrequal{#1}{Ext\_PhysAddr\_Check}{\hyperref[sailRISCVtypezExtzyPhysAddrzyCheck]{#2}}{}%
  \ifstrequal{#1}{Fcsr}{\hyperref[sailRISCVtypezFcsr]{#2}}{}%
  \ifstrequal{#1}{FetchResult}{\hyperref[sailRISCVtypezFetchResult]{#2}}{}%
  \ifstrequal{#1}{InterruptType}{\hyperref[sailRISCVtypezInterruptType]{#2}}{}%
  \ifstrequal{#1}{Mcause}{\hyperref[sailRISCVtypezMcause]{#2}}{}%
  \ifstrequal{#1}{Medeleg}{\hyperref[sailRISCVtypezMedeleg]{#2}}{}%
  \ifstrequal{#1}{MemoryOpResult}{\hyperref[sailRISCVtypezMemoryOpResult]{#2}}{}%
  \ifstrequal{#1}{Minterrupts}{\hyperref[sailRISCVtypezMinterrupts]{#2}}{}%
  \ifstrequal{#1}{Misa}{\hyperref[sailRISCVtypezMisa]{#2}}{}%
  \ifstrequal{#1}{Mstatus}{\hyperref[sailRISCVtypezMstatus]{#2}}{}%
  \ifstrequal{#1}{Mstatush}{\hyperref[sailRISCVtypezMstatush]{#2}}{}%
  \ifstrequal{#1}{Mtvec}{\hyperref[sailRISCVtypezMtvec]{#2}}{}%
  \ifstrequal{#1}{PTE_Bits}{\hyperref[sailRISCVtypezPTEzyBits]{#2}}{}%
  \ifstrequal{#1}{PTE\_Bits}{\hyperref[sailRISCVtypezPTEzyBits]{#2}}{}%
  \ifstrequal{#1}{PTE_Check}{\hyperref[sailRISCVtypezPTEzyCheck]{#2}}{}%
  \ifstrequal{#1}{PTE\_Check}{\hyperref[sailRISCVtypezPTEzyCheck]{#2}}{}%
  \ifstrequal{#1}{PTW_Error}{\hyperref[sailRISCVtypezPTWzyError]{#2}}{}%
  \ifstrequal{#1}{PTW\_Error}{\hyperref[sailRISCVtypezPTWzyError]{#2}}{}%
  \ifstrequal{#1}{PTW_Result}{\hyperref[sailRISCVtypezPTWzyResult]{#2}}{}%
  \ifstrequal{#1}{PTW\_Result}{\hyperref[sailRISCVtypezPTWzyResult]{#2}}{}%
  \ifstrequal{#1}{PmpAddrMatchType}{\hyperref[sailRISCVtypezPmpAddrMatchType]{#2}}{}%
  \ifstrequal{#1}{Pmpcfg_ent}{\hyperref[sailRISCVtypezPmpcfgzyent]{#2}}{}%
  \ifstrequal{#1}{Pmpcfg\_ent}{\hyperref[sailRISCVtypezPmpcfgzyent]{#2}}{}%
  \ifstrequal{#1}{Privilege}{\hyperref[sailRISCVtypezPrivilege]{#2}}{}%
  \ifstrequal{#1}{Retired}{\hyperref[sailRISCVtypezRetired]{#2}}{}%
  \ifstrequal{#1}{SATPMode}{\hyperref[sailRISCVtypezSATPMode]{#2}}{}%
  \ifstrequal{#1}{SV32_PTE}{\hyperref[sailRISCVtypezSV32zyPTE]{#2}}{}%
  \ifstrequal{#1}{SV32\_PTE}{\hyperref[sailRISCVtypezSV32zyPTE]{#2}}{}%
  \ifstrequal{#1}{SV32_Paddr}{\hyperref[sailRISCVtypezSV32zyPaddr]{#2}}{}%
  \ifstrequal{#1}{SV32\_Paddr}{\hyperref[sailRISCVtypezSV32zyPaddr]{#2}}{}%
  \ifstrequal{#1}{SV32_Vaddr}{\hyperref[sailRISCVtypezSV32zyVaddr]{#2}}{}%
  \ifstrequal{#1}{SV32\_Vaddr}{\hyperref[sailRISCVtypezSV32zyVaddr]{#2}}{}%
  \ifstrequal{#1}{SV39_PTE}{\hyperref[sailRISCVtypezSV39zyPTE]{#2}}{}%
  \ifstrequal{#1}{SV39\_PTE}{\hyperref[sailRISCVtypezSV39zyPTE]{#2}}{}%
  \ifstrequal{#1}{SV39_Paddr}{\hyperref[sailRISCVtypezSV39zyPaddr]{#2}}{}%
  \ifstrequal{#1}{SV39\_Paddr}{\hyperref[sailRISCVtypezSV39zyPaddr]{#2}}{}%
  \ifstrequal{#1}{SV39_Vaddr}{\hyperref[sailRISCVtypezSV39zyVaddr]{#2}}{}%
  \ifstrequal{#1}{SV39\_Vaddr}{\hyperref[sailRISCVtypezSV39zyVaddr]{#2}}{}%
  \ifstrequal{#1}{SV48_PTE}{\hyperref[sailRISCVtypezSV48zyPTE]{#2}}{}%
  \ifstrequal{#1}{SV48\_PTE}{\hyperref[sailRISCVtypezSV48zyPTE]{#2}}{}%
  \ifstrequal{#1}{SV48_Paddr}{\hyperref[sailRISCVtypezSV48zyPaddr]{#2}}{}%
  \ifstrequal{#1}{SV48\_Paddr}{\hyperref[sailRISCVtypezSV48zyPaddr]{#2}}{}%
  \ifstrequal{#1}{SV48_Vaddr}{\hyperref[sailRISCVtypezSV48zyVaddr]{#2}}{}%
  \ifstrequal{#1}{SV48\_Vaddr}{\hyperref[sailRISCVtypezSV48zyVaddr]{#2}}{}%
  \ifstrequal{#1}{Satp32}{\hyperref[sailRISCVtypezSatp32]{#2}}{}%
  \ifstrequal{#1}{Satp64}{\hyperref[sailRISCVtypezSatp64]{#2}}{}%
  \ifstrequal{#1}{Sedeleg}{\hyperref[sailRISCVtypezSedeleg]{#2}}{}%
  \ifstrequal{#1}{Sinterrupts}{\hyperref[sailRISCVtypezSinterrupts]{#2}}{}%
  \ifstrequal{#1}{Sstatus}{\hyperref[sailRISCVtypezSstatus]{#2}}{}%
  \ifstrequal{#1}{TLB39_Entry}{\hyperref[sailRISCVtypezTLB39zyEntry]{#2}}{}%
  \ifstrequal{#1}{TLB39\_Entry}{\hyperref[sailRISCVtypezTLB39zyEntry]{#2}}{}%
  \ifstrequal{#1}{TLB48_Entry}{\hyperref[sailRISCVtypezTLB48zyEntry]{#2}}{}%
  \ifstrequal{#1}{TLB48\_Entry}{\hyperref[sailRISCVtypezTLB48zyEntry]{#2}}{}%
  \ifstrequal{#1}{TLB_Entry}{\hyperref[sailRISCVtypezTLBzyEntry]{#2}}{}%
  \ifstrequal{#1}{TLB\_Entry}{\hyperref[sailRISCVtypezTLBzyEntry]{#2}}{}%
  \ifstrequal{#1}{TR_Result}{\hyperref[sailRISCVtypezTRzyResult]{#2}}{}%
  \ifstrequal{#1}{TR\_Result}{\hyperref[sailRISCVtypezTRzyResult]{#2}}{}%
  \ifstrequal{#1}{TrapVectorMode}{\hyperref[sailRISCVtypezTrapVectorMode]{#2}}{}%
  \ifstrequal{#1}{Uinterrupts}{\hyperref[sailRISCVtypezUinterrupts]{#2}}{}%
  \ifstrequal{#1}{Ustatus}{\hyperref[sailRISCVtypezUstatus]{#2}}{}%
  \ifstrequal{#1}{a64_barrier_domain}{\hyperref[sailRISCVtypeza64zybarrierzydomain]{#2}}{}%
  \ifstrequal{#1}{a64\_barrier\_domain}{\hyperref[sailRISCVtypeza64zybarrierzydomain]{#2}}{}%
  \ifstrequal{#1}{a64_barrier_type}{\hyperref[sailRISCVtypeza64zybarrierzytype]{#2}}{}%
  \ifstrequal{#1}{a64\_barrier\_type}{\hyperref[sailRISCVtypeza64zybarrierzytype]{#2}}{}%
  \ifstrequal{#1}{amo}{\hyperref[sailRISCVtypezamo]{#2}}{}%
  \ifstrequal{#1}{amoop}{\hyperref[sailRISCVtypezamoop]{#2}}{}%
  \ifstrequal{#1}{arch_xlen}{\hyperref[sailRISCVtypezarchzyxlen]{#2}}{}%
  \ifstrequal{#1}{arch\_xlen}{\hyperref[sailRISCVtypezarchzyxlen]{#2}}{}%
  \ifstrequal{#1}{asid32}{\hyperref[sailRISCVtypezasid32]{#2}}{}%
  \ifstrequal{#1}{asid64}{\hyperref[sailRISCVtypezasid64]{#2}}{}%
  \ifstrequal{#1}{ast}{\hyperref[sailRISCVtypezast]{#2}}{}%
  \ifstrequal{#1}{barrier_kind}{\hyperref[sailRISCVtypezbarrierzykind]{#2}}{}%
  \ifstrequal{#1}{barrier\_kind}{\hyperref[sailRISCVtypezbarrierzykind]{#2}}{}%
  \ifstrequal{#1}{biop_zbs}{\hyperref[sailRISCVtypezbiopzyzzbs]{#2}}{}%
  \ifstrequal{#1}{biop\_zbs}{\hyperref[sailRISCVtypezbiopzyzzbs]{#2}}{}%
  \ifstrequal{#1}{bits}{\hyperref[sailRISCVtypezbits]{#2}}{}%
  \ifstrequal{#1}{bits_D}{\hyperref[sailRISCVtypezbitszyD]{#2}}{}%
  \ifstrequal{#1}{bits\_D}{\hyperref[sailRISCVtypezbitszyD]{#2}}{}%
  \ifstrequal{#1}{bits_H}{\hyperref[sailRISCVtypezbitszyH]{#2}}{}%
  \ifstrequal{#1}{bits\_H}{\hyperref[sailRISCVtypezbitszyH]{#2}}{}%
  \ifstrequal{#1}{bits_L}{\hyperref[sailRISCVtypezbitszyL]{#2}}{}%
  \ifstrequal{#1}{bits\_L}{\hyperref[sailRISCVtypezbitszyL]{#2}}{}%
  \ifstrequal{#1}{bits_LU}{\hyperref[sailRISCVtypezbitszyLU]{#2}}{}%
  \ifstrequal{#1}{bits\_LU}{\hyperref[sailRISCVtypezbitszyLU]{#2}}{}%
  \ifstrequal{#1}{bits_S}{\hyperref[sailRISCVtypezbitszyS]{#2}}{}%
  \ifstrequal{#1}{bits\_S}{\hyperref[sailRISCVtypezbitszyS]{#2}}{}%
  \ifstrequal{#1}{bits_W}{\hyperref[sailRISCVtypezbitszyW]{#2}}{}%
  \ifstrequal{#1}{bits\_W}{\hyperref[sailRISCVtypezbitszyW]{#2}}{}%
  \ifstrequal{#1}{bits_WU}{\hyperref[sailRISCVtypezbitszyWU]{#2}}{}%
  \ifstrequal{#1}{bits\_WU}{\hyperref[sailRISCVtypezbitszyWU]{#2}}{}%
  \ifstrequal{#1}{bits_fflags}{\hyperref[sailRISCVtypezbitszyfflags]{#2}}{}%
  \ifstrequal{#1}{bits\_fflags}{\hyperref[sailRISCVtypezbitszyfflags]{#2}}{}%
  \ifstrequal{#1}{bits_rm}{\hyperref[sailRISCVtypezbitszyrm]{#2}}{}%
  \ifstrequal{#1}{bits\_rm}{\hyperref[sailRISCVtypezbitszyrm]{#2}}{}%
  \ifstrequal{#1}{bop}{\hyperref[sailRISCVtypezbop]{#2}}{}%
  \ifstrequal{#1}{brop_zba}{\hyperref[sailRISCVtypezbropzyzzba]{#2}}{}%
  \ifstrequal{#1}{brop\_zba}{\hyperref[sailRISCVtypezbropzyzzba]{#2}}{}%
  \ifstrequal{#1}{brop_zbb}{\hyperref[sailRISCVtypezbropzyzzbb]{#2}}{}%
  \ifstrequal{#1}{brop\_zbb}{\hyperref[sailRISCVtypezbropzyzzbb]{#2}}{}%
  \ifstrequal{#1}{brop_zbkb}{\hyperref[sailRISCVtypezbropzyzzbkb]{#2}}{}%
  \ifstrequal{#1}{brop\_zbkb}{\hyperref[sailRISCVtypezbropzyzzbkb]{#2}}{}%
  \ifstrequal{#1}{brop_zbs}{\hyperref[sailRISCVtypezbropzyzzbs]{#2}}{}%
  \ifstrequal{#1}{brop\_zbs}{\hyperref[sailRISCVtypezbropzyzzbs]{#2}}{}%
  \ifstrequal{#1}{bropw_zba}{\hyperref[sailRISCVtypezbropwzyzzba]{#2}}{}%
  \ifstrequal{#1}{bropw\_zba}{\hyperref[sailRISCVtypezbropwzyzzba]{#2}}{}%
  \ifstrequal{#1}{bropw_zbb}{\hyperref[sailRISCVtypezbropwzyzzbb]{#2}}{}%
  \ifstrequal{#1}{bropw\_zbb}{\hyperref[sailRISCVtypezbropwzyzzbb]{#2}}{}%
  \ifstrequal{#1}{cache_op_kind}{\hyperref[sailRISCVtypezcachezyopzykind]{#2}}{}%
  \ifstrequal{#1}{cache\_op\_kind}{\hyperref[sailRISCVtypezcachezyopzykind]{#2}}{}%
  \ifstrequal{#1}{cap_E_width}{\hyperref[sailRISCVtypezcapzyEzywidth]{#2}}{}%
  \ifstrequal{#1}{cap\_E\_width}{\hyperref[sailRISCVtypezcapzyEzywidth]{#2}}{}%
  \ifstrequal{#1}{cap_addr_width}{\hyperref[sailRISCVtypezcapzyaddrzywidth]{#2}}{}%
  \ifstrequal{#1}{cap\_addr\_width}{\hyperref[sailRISCVtypezcapzyaddrzywidth]{#2}}{}%
  \ifstrequal{#1}{cap_flags_width}{\hyperref[sailRISCVtypezcapzyflagszywidth]{#2}}{}%
  \ifstrequal{#1}{cap\_flags\_width}{\hyperref[sailRISCVtypezcapzyflagszywidth]{#2}}{}%
  \ifstrequal{#1}{cap_hperms_width}{\hyperref[sailRISCVtypezcapzyhpermszywidth]{#2}}{}%
  \ifstrequal{#1}{cap\_hperms\_width}{\hyperref[sailRISCVtypezcapzyhpermszywidth]{#2}}{}%
  \ifstrequal{#1}{cap_len_width}{\hyperref[sailRISCVtypezcapzylenzywidth]{#2}}{}%
  \ifstrequal{#1}{cap\_len\_width}{\hyperref[sailRISCVtypezcapzylenzywidth]{#2}}{}%
  \ifstrequal{#1}{cap_mantissa_width}{\hyperref[sailRISCVtypezcapzymantissazywidth]{#2}}{}%
  \ifstrequal{#1}{cap\_mantissa\_width}{\hyperref[sailRISCVtypezcapzymantissazywidth]{#2}}{}%
  \ifstrequal{#1}{cap_otype_width}{\hyperref[sailRISCVtypezcapzyotypezywidth]{#2}}{}%
  \ifstrequal{#1}{cap\_otype\_width}{\hyperref[sailRISCVtypezcapzyotypezywidth]{#2}}{}%
  \ifstrequal{#1}{cap_perms_width}{\hyperref[sailRISCVtypezcapzypermszywidth]{#2}}{}%
  \ifstrequal{#1}{cap\_perms\_width}{\hyperref[sailRISCVtypezcapzypermszywidth]{#2}}{}%
  \ifstrequal{#1}{cap_reserved_width}{\hyperref[sailRISCVtypezcapzyreservedzywidth]{#2}}{}%
  \ifstrequal{#1}{cap\_reserved\_width}{\hyperref[sailRISCVtypezcapzyreservedzywidth]{#2}}{}%
  \ifstrequal{#1}{cap_size}{\hyperref[sailRISCVtypezcapzysizze]{#2}}{}%
  \ifstrequal{#1}{cap\_size}{\hyperref[sailRISCVtypezcapzysizze]{#2}}{}%
  \ifstrequal{#1}{cap_uperms_shift}{\hyperref[sailRISCVtypezcapzyupermszyshift]{#2}}{}%
  \ifstrequal{#1}{cap\_uperms\_shift}{\hyperref[sailRISCVtypezcapzyupermszyshift]{#2}}{}%
  \ifstrequal{#1}{cap_uperms_width}{\hyperref[sailRISCVtypezcapzyupermszywidth]{#2}}{}%
  \ifstrequal{#1}{cap\_uperms\_width}{\hyperref[sailRISCVtypezcapzyupermszywidth]{#2}}{}%
  \ifstrequal{#1}{capreg_idx}{\hyperref[sailRISCVtypezcapregzyidx]{#2}}{}%
  \ifstrequal{#1}{capreg\_idx}{\hyperref[sailRISCVtypezcapregzyidx]{#2}}{}%
  \ifstrequal{#1}{caps_per_cache_line}{\hyperref[sailRISCVtypezcapszyperzycachezyline]{#2}}{}%
  \ifstrequal{#1}{caps\_per\_cache\_line}{\hyperref[sailRISCVtypezcapszyperzycachezyline]{#2}}{}%
  \ifstrequal{#1}{ccsr}{\hyperref[sailRISCVtypezccsr]{#2}}{}%
  \ifstrequal{#1}{cregidx}{\hyperref[sailRISCVtypezcregidx]{#2}}{}%
  \ifstrequal{#1}{csrRW}{\hyperref[sailRISCVtypezcsrRW]{#2}}{}%
  \ifstrequal{#1}{csreg}{\hyperref[sailRISCVtypezcsreg]{#2}}{}%
  \ifstrequal{#1}{csrop}{\hyperref[sailRISCVtypezcsrop]{#2}}{}%
  \ifstrequal{#1}{ctl_result}{\hyperref[sailRISCVtypezctlzyresult]{#2}}{}%
  \ifstrequal{#1}{ctl\_result}{\hyperref[sailRISCVtypezctlzyresult]{#2}}{}%
  \ifstrequal{#1}{diafp}{\hyperref[sailRISCVtypezdiafp]{#2}}{}%
  \ifstrequal{#1}{exc_code}{\hyperref[sailRISCVtypezexczycode]{#2}}{}%
  \ifstrequal{#1}{exc\_code}{\hyperref[sailRISCVtypezexczycode]{#2}}{}%
  \ifstrequal{#1}{exception}{\hyperref[sailRISCVtypezexception]{#2}}{}%
  \ifstrequal{#1}{extPte}{\hyperref[sailRISCVtypezextPte]{#2}}{}%
  \ifstrequal{#1}{ext_access_type}{\hyperref[sailRISCVtypezextzyaccesszytype]{#2}}{}%
  \ifstrequal{#1}{ext\_access\_type}{\hyperref[sailRISCVtypezextzyaccesszytype]{#2}}{}%
  \ifstrequal{#1}{ext_control_addr_error}{\hyperref[sailRISCVtypezextzycontrolzyaddrzyerror]{#2}}{}%
  \ifstrequal{#1}{ext\_control\_addr\_error}{\hyperref[sailRISCVtypezextzycontrolzyaddrzyerror]{#2}}{}%
  \ifstrequal{#1}{ext_data_addr_error}{\hyperref[sailRISCVtypezextzydatazyaddrzyerror]{#2}}{}%
  \ifstrequal{#1}{ext\_data\_addr\_error}{\hyperref[sailRISCVtypezextzydatazyaddrzyerror]{#2}}{}%
  \ifstrequal{#1}{ext_exc_type}{\hyperref[sailRISCVtypezextzyexczytype]{#2}}{}%
  \ifstrequal{#1}{ext\_exc\_type}{\hyperref[sailRISCVtypezextzyexczytype]{#2}}{}%
  \ifstrequal{#1}{ext_exception}{\hyperref[sailRISCVtypezextzyexception]{#2}}{}%
  \ifstrequal{#1}{ext\_exception}{\hyperref[sailRISCVtypezextzyexception]{#2}}{}%
  \ifstrequal{#1}{ext_fetch_addr_error}{\hyperref[sailRISCVtypezextzyfetchzyaddrzyerror]{#2}}{}%
  \ifstrequal{#1}{ext\_fetch\_addr\_error}{\hyperref[sailRISCVtypezextzyfetchzyaddrzyerror]{#2}}{}%
  \ifstrequal{#1}{ext_ptw}{\hyperref[sailRISCVtypezextzyptw]{#2}}{}%
  \ifstrequal{#1}{ext\_ptw}{\hyperref[sailRISCVtypezextzyptw]{#2}}{}%
  \ifstrequal{#1}{ext_ptw_error}{\hyperref[sailRISCVtypezextzyptwzyerror]{#2}}{}%
  \ifstrequal{#1}{ext\_ptw\_error}{\hyperref[sailRISCVtypezextzyptwzyerror]{#2}}{}%
  \ifstrequal{#1}{ext_ptw_fail}{\hyperref[sailRISCVtypezextzyptwzyfail]{#2}}{}%
  \ifstrequal{#1}{ext\_ptw\_fail}{\hyperref[sailRISCVtypezextzyptwzyfail]{#2}}{}%
  \ifstrequal{#1}{ext_ptw_lc}{\hyperref[sailRISCVtypezextzyptwzylc]{#2}}{}%
  \ifstrequal{#1}{ext\_ptw\_lc}{\hyperref[sailRISCVtypezextzyptwzylc]{#2}}{}%
  \ifstrequal{#1}{ext_ptw_sc}{\hyperref[sailRISCVtypezextzyptwzysc]{#2}}{}%
  \ifstrequal{#1}{ext\_ptw\_sc}{\hyperref[sailRISCVtypezextzyptwzysc]{#2}}{}%
  \ifstrequal{#1}{ext_status}{\hyperref[sailRISCVtypezextzystatus]{#2}}{}%
  \ifstrequal{#1}{ext\_status}{\hyperref[sailRISCVtypezextzystatus]{#2}}{}%
  \ifstrequal{#1}{extop_zbb}{\hyperref[sailRISCVtypezextopzyzzbb]{#2}}{}%
  \ifstrequal{#1}{extop\_zbb}{\hyperref[sailRISCVtypezextopzyzzbb]{#2}}{}%
  \ifstrequal{#1}{f_bin_op_D}{\hyperref[sailRISCVtypezfzybinzyopzyD]{#2}}{}%
  \ifstrequal{#1}{f\_bin\_op\_D}{\hyperref[sailRISCVtypezfzybinzyopzyD]{#2}}{}%
  \ifstrequal{#1}{f_bin_op_H}{\hyperref[sailRISCVtypezfzybinzyopzyH]{#2}}{}%
  \ifstrequal{#1}{f\_bin\_op\_H}{\hyperref[sailRISCVtypezfzybinzyopzyH]{#2}}{}%
  \ifstrequal{#1}{f_bin_op_S}{\hyperref[sailRISCVtypezfzybinzyopzyS]{#2}}{}%
  \ifstrequal{#1}{f\_bin\_op\_S}{\hyperref[sailRISCVtypezfzybinzyopzyS]{#2}}{}%
  \ifstrequal{#1}{f_bin_rm_op_D}{\hyperref[sailRISCVtypezfzybinzyrmzyopzyD]{#2}}{}%
  \ifstrequal{#1}{f\_bin\_rm\_op\_D}{\hyperref[sailRISCVtypezfzybinzyrmzyopzyD]{#2}}{}%
  \ifstrequal{#1}{f_bin_rm_op_H}{\hyperref[sailRISCVtypezfzybinzyrmzyopzyH]{#2}}{}%
  \ifstrequal{#1}{f\_bin\_rm\_op\_H}{\hyperref[sailRISCVtypezfzybinzyrmzyopzyH]{#2}}{}%
  \ifstrequal{#1}{f_bin_rm_op_S}{\hyperref[sailRISCVtypezfzybinzyrmzyopzyS]{#2}}{}%
  \ifstrequal{#1}{f\_bin\_rm\_op\_S}{\hyperref[sailRISCVtypezfzybinzyrmzyopzyS]{#2}}{}%
  \ifstrequal{#1}{f_madd_op_D}{\hyperref[sailRISCVtypezfzymaddzyopzyD]{#2}}{}%
  \ifstrequal{#1}{f\_madd\_op\_D}{\hyperref[sailRISCVtypezfzymaddzyopzyD]{#2}}{}%
  \ifstrequal{#1}{f_madd_op_H}{\hyperref[sailRISCVtypezfzymaddzyopzyH]{#2}}{}%
  \ifstrequal{#1}{f\_madd\_op\_H}{\hyperref[sailRISCVtypezfzymaddzyopzyH]{#2}}{}%
  \ifstrequal{#1}{f_madd_op_S}{\hyperref[sailRISCVtypezfzymaddzyopzyS]{#2}}{}%
  \ifstrequal{#1}{f\_madd\_op\_S}{\hyperref[sailRISCVtypezfzymaddzyopzyS]{#2}}{}%
  \ifstrequal{#1}{f_un_op_D}{\hyperref[sailRISCVtypezfzyunzyopzyD]{#2}}{}%
  \ifstrequal{#1}{f\_un\_op\_D}{\hyperref[sailRISCVtypezfzyunzyopzyD]{#2}}{}%
  \ifstrequal{#1}{f_un_op_H}{\hyperref[sailRISCVtypezfzyunzyopzyH]{#2}}{}%
  \ifstrequal{#1}{f\_un\_op\_H}{\hyperref[sailRISCVtypezfzyunzyopzyH]{#2}}{}%
  \ifstrequal{#1}{f_un_op_S}{\hyperref[sailRISCVtypezfzyunzyopzyS]{#2}}{}%
  \ifstrequal{#1}{f\_un\_op\_S}{\hyperref[sailRISCVtypezfzyunzyopzyS]{#2}}{}%
  \ifstrequal{#1}{f_un_rm_op_D}{\hyperref[sailRISCVtypezfzyunzyrmzyopzyD]{#2}}{}%
  \ifstrequal{#1}{f\_un\_rm\_op\_D}{\hyperref[sailRISCVtypezfzyunzyrmzyopzyD]{#2}}{}%
  \ifstrequal{#1}{f_un_rm_op_H}{\hyperref[sailRISCVtypezfzyunzyrmzyopzyH]{#2}}{}%
  \ifstrequal{#1}{f\_un\_rm\_op\_H}{\hyperref[sailRISCVtypezfzyunzyrmzyopzyH]{#2}}{}%
  \ifstrequal{#1}{f_un_rm_op_S}{\hyperref[sailRISCVtypezfzyunzyrmzyopzyS]{#2}}{}%
  \ifstrequal{#1}{f\_un\_rm\_op\_S}{\hyperref[sailRISCVtypezfzyunzyrmzyopzyS]{#2}}{}%
  \ifstrequal{#1}{flen}{\hyperref[sailRISCVtypezflen]{#2}}{}%
  \ifstrequal{#1}{flen_bytes}{\hyperref[sailRISCVtypezflenzybytes]{#2}}{}%
  \ifstrequal{#1}{flen\_bytes}{\hyperref[sailRISCVtypezflenzybytes]{#2}}{}%
  \ifstrequal{#1}{flenbits}{\hyperref[sailRISCVtypezflenbits]{#2}}{}%
  \ifstrequal{#1}{fregtype}{\hyperref[sailRISCVtypezfregtype]{#2}}{}%
  \ifstrequal{#1}{half}{\hyperref[sailRISCVtypezhalf]{#2}}{}%
  \ifstrequal{#1}{htif_cmd}{\hyperref[sailRISCVtypezhtifzycmd]{#2}}{}%
  \ifstrequal{#1}{htif\_cmd}{\hyperref[sailRISCVtypezhtifzycmd]{#2}}{}%
  \ifstrequal{#1}{imm12}{\hyperref[sailRISCVtypezimm12]{#2}}{}%
  \ifstrequal{#1}{imm20}{\hyperref[sailRISCVtypezimm20]{#2}}{}%
  \ifstrequal{#1}{instruction_kind}{\hyperref[sailRISCVtypezinstructionzykind]{#2}}{}%
  \ifstrequal{#1}{instruction\_kind}{\hyperref[sailRISCVtypezinstructionzykind]{#2}}{}%
  \ifstrequal{#1}{internal_E_take_bits}{\hyperref[sailRISCVtypezinternalzyEzytakezybits]{#2}}{}%
  \ifstrequal{#1}{internal\_E\_take\_bits}{\hyperref[sailRISCVtypezinternalzyEzytakezybits]{#2}}{}%
  \ifstrequal{#1}{interrupt_set}{\hyperref[sailRISCVtypezinterruptzyset]{#2}}{}%
  \ifstrequal{#1}{interrupt\_set}{\hyperref[sailRISCVtypezinterruptzyset]{#2}}{}%
  \ifstrequal{#1}{iop}{\hyperref[sailRISCVtypeziop]{#2}}{}%
  \ifstrequal{#1}{log2_cap_size}{\hyperref[sailRISCVtypezlog2zycapzysizze]{#2}}{}%
  \ifstrequal{#1}{log2\_cap\_size}{\hyperref[sailRISCVtypezlog2zycapzysizze]{#2}}{}%
  \ifstrequal{#1}{max_mem_access}{\hyperref[sailRISCVtypezmaxzymemzyaccess]{#2}}{}%
  \ifstrequal{#1}{max\_mem\_access}{\hyperref[sailRISCVtypezmaxzymemzyaccess]{#2}}{}%
  \ifstrequal{#1}{mem_meta}{\hyperref[sailRISCVtypezmemzymeta]{#2}}{}%
  \ifstrequal{#1}{mem\_meta}{\hyperref[sailRISCVtypezmemzymeta]{#2}}{}%
  \ifstrequal{#1}{niafp}{\hyperref[sailRISCVtypezniafp]{#2}}{}%
  \ifstrequal{#1}{niafps}{\hyperref[sailRISCVtypezniafps]{#2}}{}%
  \ifstrequal{#1}{opcode}{\hyperref[sailRISCVtypezopcode]{#2}}{}%
  \ifstrequal{#1}{option}{\hyperref[sailRISCVtypezoption]{#2}}{}%
  \ifstrequal{#1}{paddr32}{\hyperref[sailRISCVtypezpaddr32]{#2}}{}%
  \ifstrequal{#1}{paddr64}{\hyperref[sailRISCVtypezpaddr64]{#2}}{}%
  \ifstrequal{#1}{pmpAddrMatch}{\hyperref[sailRISCVtypezpmpAddrMatch]{#2}}{}%
  \ifstrequal{#1}{pmpMatch}{\hyperref[sailRISCVtypezpmpMatch]{#2}}{}%
  \ifstrequal{#1}{pmp_addr_range}{\hyperref[sailRISCVtypezpmpzyaddrzyrange]{#2}}{}%
  \ifstrequal{#1}{pmp\_addr\_range}{\hyperref[sailRISCVtypezpmpzyaddrzyrange]{#2}}{}%
  \ifstrequal{#1}{priv_level}{\hyperref[sailRISCVtypezprivzylevel]{#2}}{}%
  \ifstrequal{#1}{priv\_level}{\hyperref[sailRISCVtypezprivzylevel]{#2}}{}%
  \ifstrequal{#1}{pte32}{\hyperref[sailRISCVtypezpte32]{#2}}{}%
  \ifstrequal{#1}{pte48}{\hyperref[sailRISCVtypezpte48]{#2}}{}%
  \ifstrequal{#1}{pte64}{\hyperref[sailRISCVtypezpte64]{#2}}{}%
  \ifstrequal{#1}{pteAttribs}{\hyperref[sailRISCVtypezpteAttribs]{#2}}{}%
  \ifstrequal{#1}{read_kind}{\hyperref[sailRISCVtypezreadzykind]{#2}}{}%
  \ifstrequal{#1}{read\_kind}{\hyperref[sailRISCVtypezreadzykind]{#2}}{}%
  \ifstrequal{#1}{regfp}{\hyperref[sailRISCVtypezregfp]{#2}}{}%
  \ifstrequal{#1}{regfps}{\hyperref[sailRISCVtypezregfps]{#2}}{}%
  \ifstrequal{#1}{regidx}{\hyperref[sailRISCVtypezregidx]{#2}}{}%
  \ifstrequal{#1}{regno}{\hyperref[sailRISCVtypezregno]{#2}}{}%
  \ifstrequal{#1}{regtype}{\hyperref[sailRISCVtypezregtype]{#2}}{}%
  \ifstrequal{#1}{rop}{\hyperref[sailRISCVtypezrop]{#2}}{}%
  \ifstrequal{#1}{ropw}{\hyperref[sailRISCVtypezropw]{#2}}{}%
  \ifstrequal{#1}{rounding_mode}{\hyperref[sailRISCVtypezroundingzymode]{#2}}{}%
  \ifstrequal{#1}{rounding\_mode}{\hyperref[sailRISCVtypezroundingzymode]{#2}}{}%
  \ifstrequal{#1}{satp_mode}{\hyperref[sailRISCVtypezsatpzymode]{#2}}{}%
  \ifstrequal{#1}{satp\_mode}{\hyperref[sailRISCVtypezsatpzymode]{#2}}{}%
  \ifstrequal{#1}{screg}{\hyperref[sailRISCVtypezscreg]{#2}}{}%
  \ifstrequal{#1}{seed_opst}{\hyperref[sailRISCVtypezseedzyopst]{#2}}{}%
  \ifstrequal{#1}{seed\_opst}{\hyperref[sailRISCVtypezseedzyopst]{#2}}{}%
  \ifstrequal{#1}{sop}{\hyperref[sailRISCVtypezsop]{#2}}{}%
  \ifstrequal{#1}{sopw}{\hyperref[sailRISCVtypezsopw]{#2}}{}%
  \ifstrequal{#1}{sync_exception}{\hyperref[sailRISCVtypezsynczyexception]{#2}}{}%
  \ifstrequal{#1}{sync\_exception}{\hyperref[sailRISCVtypezsynczyexception]{#2}}{}%
  \ifstrequal{#1}{tagaddrbits}{\hyperref[sailRISCVtypeztagaddrbits]{#2}}{}%
  \ifstrequal{#1}{trans_kind}{\hyperref[sailRISCVtypeztranszykind]{#2}}{}%
  \ifstrequal{#1}{trans\_kind}{\hyperref[sailRISCVtypeztranszykind]{#2}}{}%
  \ifstrequal{#1}{tv_mode}{\hyperref[sailRISCVtypeztvzymode]{#2}}{}%
  \ifstrequal{#1}{tv\_mode}{\hyperref[sailRISCVtypeztvzymode]{#2}}{}%
  \ifstrequal{#1}{uop}{\hyperref[sailRISCVtypezuop]{#2}}{}%
  \ifstrequal{#1}{vaddr32}{\hyperref[sailRISCVtypezvaddr32]{#2}}{}%
  \ifstrequal{#1}{vaddr39}{\hyperref[sailRISCVtypezvaddr39]{#2}}{}%
  \ifstrequal{#1}{vaddr48}{\hyperref[sailRISCVtypezvaddr48]{#2}}{}%
  \ifstrequal{#1}{word}{\hyperref[sailRISCVtypezword]{#2}}{}%
  \ifstrequal{#1}{word_width}{\hyperref[sailRISCVtypezwordzywidth]{#2}}{}%
  \ifstrequal{#1}{word\_width}{\hyperref[sailRISCVtypezwordzywidth]{#2}}{}%
  \ifstrequal{#1}{write_kind}{\hyperref[sailRISCVtypezwritezykind]{#2}}{}%
  \ifstrequal{#1}{write\_kind}{\hyperref[sailRISCVtypezwritezykind]{#2}}{}%
  \ifstrequal{#1}{xlen}{\hyperref[sailRISCVtypezxlen]{#2}}{}%
  \ifstrequal{#1}{xlen_bytes}{\hyperref[sailRISCVtypezxlenzybytes]{#2}}{}%
  \ifstrequal{#1}{xlen\_bytes}{\hyperref[sailRISCVtypezxlenzybytes]{#2}}{}%
  \ifstrequal{#1}{xlenbits}{\hyperref[sailRISCVtypezxlenbits]{#2}}{}}

\newcommand{\sailRISCVlet}[1]{
  \ifstrequal{#1}{CIA_fp}{\sailRISCVletCIAFp}{}%
  \ifstrequal{#1}{CIA\_fp}{\sailRISCVletCIAFp}{}%
  \ifstrequal{#1}{DDC_IDX}{\sailRISCVletDDCIDX}{}%
  \ifstrequal{#1}{DDC\_IDX}{\sailRISCVletDDCIDX}{}%
  \ifstrequal{#1}{GPRstrs}{\sailRISCVletGPRstrs}{}%
  \ifstrequal{#1}{MSIP_BASE}{\sailRISCVletMSIPBASE}{}%
  \ifstrequal{#1}{MSIP\_BASE}{\sailRISCVletMSIPBASE}{}%
  \ifstrequal{#1}{MTIMECMP_BASE}{\sailRISCVletMTIMECMPBASE}{}%
  \ifstrequal{#1}{MTIMECMP\_BASE}{\sailRISCVletMTIMECMPBASE}{}%
  \ifstrequal{#1}{MTIMECMP_BASE_HI}{\sailRISCVletMTIMECMPBASEHI}{}%
  \ifstrequal{#1}{MTIMECMP\_BASE\_HI}{\sailRISCVletMTIMECMPBASEHI}{}%
  \ifstrequal{#1}{MTIME_BASE}{\sailRISCVletMTIMEBASE}{}%
  \ifstrequal{#1}{MTIME\_BASE}{\sailRISCVletMTIMEBASE}{}%
  \ifstrequal{#1}{MTIME_BASE_HI}{\sailRISCVletMTIMEBASEHI}{}%
  \ifstrequal{#1}{MTIME\_BASE\_HI}{\sailRISCVletMTIMEBASEHI}{}%
  \ifstrequal{#1}{NIA_fp}{\sailRISCVletNIAFp}{}%
  \ifstrequal{#1}{NIA\_fp}{\sailRISCVletNIAFp}{}%
  \ifstrequal{#1}{PAGESIZE_BITS}{\sailRISCVletPAGESIZEBITS}{}%
  \ifstrequal{#1}{PAGESIZE\_BITS}{\sailRISCVletPAGESIZEBITS}{}%
  \ifstrequal{#1}{PCC_IDX}{\sailRISCVletPCCIDX}{}%
  \ifstrequal{#1}{PCC\_IDX}{\sailRISCVletPCCIDX}{}%
  \ifstrequal{#1}{PTE32_LOG_SIZE}{\sailRISCVletPTEThreeTwoLOGSIZE}{}%
  \ifstrequal{#1}{PTE32\_LOG\_SIZE}{\sailRISCVletPTEThreeTwoLOGSIZE}{}%
  \ifstrequal{#1}{PTE32_SIZE}{\sailRISCVletPTEThreeTwoSIZE}{}%
  \ifstrequal{#1}{PTE32\_SIZE}{\sailRISCVletPTEThreeTwoSIZE}{}%
  \ifstrequal{#1}{PTE39_LOG_SIZE}{\sailRISCVletPTEThreeNineLOGSIZE}{}%
  \ifstrequal{#1}{PTE39\_LOG\_SIZE}{\sailRISCVletPTEThreeNineLOGSIZE}{}%
  \ifstrequal{#1}{PTE39_SIZE}{\sailRISCVletPTEThreeNineSIZE}{}%
  \ifstrequal{#1}{PTE39\_SIZE}{\sailRISCVletPTEThreeNineSIZE}{}%
  \ifstrequal{#1}{PTE48_LOG_SIZE}{\sailRISCVletPTEFourEightLOGSIZE}{}%
  \ifstrequal{#1}{PTE48\_LOG\_SIZE}{\sailRISCVletPTEFourEightLOGSIZE}{}%
  \ifstrequal{#1}{PTE48_SIZE}{\sailRISCVletPTEFourEightSIZE}{}%
  \ifstrequal{#1}{PTE48\_SIZE}{\sailRISCVletPTEFourEightSIZE}{}%
  \ifstrequal{#1}{SV32_LEVELS}{\sailRISCVletSVThreeTwoLEVELS}{}%
  \ifstrequal{#1}{SV32\_LEVELS}{\sailRISCVletSVThreeTwoLEVELS}{}%
  \ifstrequal{#1}{SV32_LEVEL_BITS}{\sailRISCVletSVThreeTwoLEVELBITS}{}%
  \ifstrequal{#1}{SV32\_LEVEL\_BITS}{\sailRISCVletSVThreeTwoLEVELBITS}{}%
  \ifstrequal{#1}{SV39_LEVELS}{\sailRISCVletSVThreeNineLEVELS}{}%
  \ifstrequal{#1}{SV39\_LEVELS}{\sailRISCVletSVThreeNineLEVELS}{}%
  \ifstrequal{#1}{SV39_LEVEL_BITS}{\sailRISCVletSVThreeNineLEVELBITS}{}%
  \ifstrequal{#1}{SV39\_LEVEL\_BITS}{\sailRISCVletSVThreeNineLEVELBITS}{}%
  \ifstrequal{#1}{SV48_LEVELS}{\sailRISCVletSVFourEightLEVELS}{}%
  \ifstrequal{#1}{SV48\_LEVELS}{\sailRISCVletSVFourEightLEVELS}{}%
  \ifstrequal{#1}{SV48_LEVEL_BITS}{\sailRISCVletSVFourEightLEVELBITS}{}%
  \ifstrequal{#1}{SV48\_LEVEL\_BITS}{\sailRISCVletSVFourEightLEVELBITS}{}%
  \ifstrequal{#1}{cap_E_width}{\sailRISCVletcapEWidth}{}%
  \ifstrequal{#1}{cap\_E\_width}{\sailRISCVletcapEWidth}{}%
  \ifstrequal{#1}{cap_addr_width}{\sailRISCVletcapAddrWidth}{}%
  \ifstrequal{#1}{cap\_addr\_width}{\sailRISCVletcapAddrWidth}{}%
  \ifstrequal{#1}{cap_flags_width}{\sailRISCVletcapFlagsWidth}{}%
  \ifstrequal{#1}{cap\_flags\_width}{\sailRISCVletcapFlagsWidth}{}%
  \ifstrequal{#1}{cap_hperms_width}{\sailRISCVletcapHpermsWidth}{}%
  \ifstrequal{#1}{cap\_hperms\_width}{\sailRISCVletcapHpermsWidth}{}%
  \ifstrequal{#1}{cap_len_width}{\sailRISCVletcapLenWidth}{}%
  \ifstrequal{#1}{cap\_len\_width}{\sailRISCVletcapLenWidth}{}%
  \ifstrequal{#1}{cap_mantissa_width}{\sailRISCVletcapMantissaWidth}{}%
  \ifstrequal{#1}{cap\_mantissa\_width}{\sailRISCVletcapMantissaWidth}{}%
  \ifstrequal{#1}{cap_max_E}{\sailRISCVletcapMaxE}{}%
  \ifstrequal{#1}{cap\_max\_E}{\sailRISCVletcapMaxE}{}%
  \ifstrequal{#1}{cap_max_addr}{\sailRISCVletcapMaxAddr}{}%
  \ifstrequal{#1}{cap\_max\_addr}{\sailRISCVletcapMaxAddr}{}%
  \ifstrequal{#1}{cap_max_otype}{\sailRISCVletcapMaxOtype}{}%
  \ifstrequal{#1}{cap\_max\_otype}{\sailRISCVletcapMaxOtype}{}%
  \ifstrequal{#1}{cap_otype_width}{\sailRISCVletcapOtypeWidth}{}%
  \ifstrequal{#1}{cap\_otype\_width}{\sailRISCVletcapOtypeWidth}{}%
  \ifstrequal{#1}{cap_perms_width}{\sailRISCVletcapPermsWidth}{}%
  \ifstrequal{#1}{cap\_perms\_width}{\sailRISCVletcapPermsWidth}{}%
  \ifstrequal{#1}{cap_reserved_width}{\sailRISCVletcapReservedWidth}{}%
  \ifstrequal{#1}{cap\_reserved\_width}{\sailRISCVletcapReservedWidth}{}%
  \ifstrequal{#1}{cap_reset_E}{\sailRISCVletcapResetE}{}%
  \ifstrequal{#1}{cap\_reset\_E}{\sailRISCVletcapResetE}{}%
  \ifstrequal{#1}{cap_reset_T}{\sailRISCVletcapResetT}{}%
  \ifstrequal{#1}{cap\_reset\_T}{\sailRISCVletcapResetT}{}%
  \ifstrequal{#1}{cap_size}{\sailRISCVletcapSizze}{}%
  \ifstrequal{#1}{cap\_size}{\sailRISCVletcapSizze}{}%
  \ifstrequal{#1}{cap_uperms_shift}{\sailRISCVletcapUpermsShift}{}%
  \ifstrequal{#1}{cap\_uperms\_shift}{\sailRISCVletcapUpermsShift}{}%
  \ifstrequal{#1}{cap_uperms_width}{\sailRISCVletcapUpermsWidth}{}%
  \ifstrequal{#1}{cap\_uperms\_width}{\sailRISCVletcapUpermsWidth}{}%
  \ifstrequal{#1}{caps_per_cache_line}{\sailRISCVletcapsPerCacheLine}{}%
  \ifstrequal{#1}{caps\_per\_cache\_line}{\sailRISCVletcapsPerCacheLine}{}%
  \ifstrequal{#1}{default_cap}{\sailRISCVletdefaultCap}{}%
  \ifstrequal{#1}{default\_cap}{\sailRISCVletdefaultCap}{}%
  \ifstrequal{#1}{default_meta}{\sailRISCVletdefaultMeta}{}%
  \ifstrequal{#1}{default\_meta}{\sailRISCVletdefaultMeta}{}%
  \ifstrequal{#1}{default_sv32_ext_pte}{\sailRISCVletdefaultSvThreeTwoExtPte}{}%
  \ifstrequal{#1}{default\_sv32\_ext\_pte}{\sailRISCVletdefaultSvThreeTwoExtPte}{}%
  \ifstrequal{#1}{default_write_acc}{\sailRISCVletdefaultWriteAcc}{}%
  \ifstrequal{#1}{default\_write\_acc}{\sailRISCVletdefaultWriteAcc}{}%
  \ifstrequal{#1}{haveRV128}{\sailRISCVlethaveRVOneTwoEight}{}%
  \ifstrequal{#1}{haveRV64}{\sailRISCVlethaveRVSixFour}{}%
  \ifstrequal{#1}{haveSplitRegFile}{\sailRISCVlethaveSplitRegFile}{}%
  \ifstrequal{#1}{init_ext_ptw}{\sailRISCVletinitExtPtw}{}%
  \ifstrequal{#1}{init\_ext\_ptw}{\sailRISCVletinitExtPtw}{}%
  \ifstrequal{#1}{internal_E_take_bits}{\sailRISCVletinternalETakeBits}{}%
  \ifstrequal{#1}{internal\_E\_take\_bits}{\sailRISCVletinternalETakeBits}{}%
  \ifstrequal{#1}{log2_cap_size}{\sailRISCVletlogTwoCapSizze}{}%
  \ifstrequal{#1}{log2\_cap\_size}{\sailRISCVletlogTwoCapSizze}{}%
  \ifstrequal{#1}{null_cap}{\sailRISCVletnullCap}{}%
  \ifstrequal{#1}{null\_cap}{\sailRISCVletnullCap}{}%
  \ifstrequal{#1}{null_cap_bits}{\sailRISCVletnullCapBits}{}%
  \ifstrequal{#1}{null\_cap\_bits}{\sailRISCVletnullCapBits}{}%
  \ifstrequal{#1}{otype_sentry}{\sailRISCVletotypeSentry}{}%
  \ifstrequal{#1}{otype\_sentry}{\sailRISCVletotypeSentry}{}%
  \ifstrequal{#1}{otype_unsealed}{\sailRISCVletotypeUnsealed}{}%
  \ifstrequal{#1}{otype\_unsealed}{\sailRISCVletotypeUnsealed}{}%
  \ifstrequal{#1}{ra}{\sailRISCVletra}{}%
  \ifstrequal{#1}{reserved_otypes}{\sailRISCVletreservedOtypes}{}%
  \ifstrequal{#1}{reserved\_otypes}{\sailRISCVletreservedOtypes}{}%
  \ifstrequal{#1}{sp}{\sailRISCVletsp}{}%
  \ifstrequal{#1}{xlen_max_signed}{\sailRISCVletxlenMaxSigned}{}%
  \ifstrequal{#1}{xlen\_max\_signed}{\sailRISCVletxlenMaxSigned}{}%
  \ifstrequal{#1}{xlen_max_unsigned}{\sailRISCVletxlenMaxUnsigned}{}%
  \ifstrequal{#1}{xlen\_max\_unsigned}{\sailRISCVletxlenMaxUnsigned}{}%
  \ifstrequal{#1}{xlen_min_signed}{\sailRISCVletxlenMinSigned}{}%
  \ifstrequal{#1}{xlen\_min\_signed}{\sailRISCVletxlenMinSigned}{}%
  \ifstrequal{#1}{xlen_val}{\sailRISCVletxlenVal}{}%
  \ifstrequal{#1}{xlen\_val}{\sailRISCVletxlenVal}{}%
  \ifstrequal{#1}{zero_freg}{\sailRISCVletzzeroFreg}{}%
  \ifstrequal{#1}{zero\_freg}{\sailRISCVletzzeroFreg}{}%
  \ifstrequal{#1}{zero_reg}{\sailRISCVletzzeroReg}{}%
  \ifstrequal{#1}{zero\_reg}{\sailRISCVletzzeroReg}{}%
  \ifstrequal{#1}{zreg}{\sailRISCVletzzreg}{}}

\newcommand{\sailRISCVreflet}[2]{
  \ifstrequal{#1}{CIA_fp}{\hyperref[sailRISCVletzCIAzyfp]{#2}}{}%
  \ifstrequal{#1}{CIA\_fp}{\hyperref[sailRISCVletzCIAzyfp]{#2}}{}%
  \ifstrequal{#1}{DDC_IDX}{\hyperref[sailRISCVletzDDCzyIDX]{#2}}{}%
  \ifstrequal{#1}{DDC\_IDX}{\hyperref[sailRISCVletzDDCzyIDX]{#2}}{}%
  \ifstrequal{#1}{GPRstrs}{\hyperref[sailRISCVletzGPRstrs]{#2}}{}%
  \ifstrequal{#1}{MSIP_BASE}{\hyperref[sailRISCVletzMSIPzyBASE]{#2}}{}%
  \ifstrequal{#1}{MSIP\_BASE}{\hyperref[sailRISCVletzMSIPzyBASE]{#2}}{}%
  \ifstrequal{#1}{MTIMECMP_BASE}{\hyperref[sailRISCVletzMTIMECMPzyBASE]{#2}}{}%
  \ifstrequal{#1}{MTIMECMP\_BASE}{\hyperref[sailRISCVletzMTIMECMPzyBASE]{#2}}{}%
  \ifstrequal{#1}{MTIMECMP_BASE_HI}{\hyperref[sailRISCVletzMTIMECMPzyBASEzyHI]{#2}}{}%
  \ifstrequal{#1}{MTIMECMP\_BASE\_HI}{\hyperref[sailRISCVletzMTIMECMPzyBASEzyHI]{#2}}{}%
  \ifstrequal{#1}{MTIME_BASE}{\hyperref[sailRISCVletzMTIMEzyBASE]{#2}}{}%
  \ifstrequal{#1}{MTIME\_BASE}{\hyperref[sailRISCVletzMTIMEzyBASE]{#2}}{}%
  \ifstrequal{#1}{MTIME_BASE_HI}{\hyperref[sailRISCVletzMTIMEzyBASEzyHI]{#2}}{}%
  \ifstrequal{#1}{MTIME\_BASE\_HI}{\hyperref[sailRISCVletzMTIMEzyBASEzyHI]{#2}}{}%
  \ifstrequal{#1}{NIA_fp}{\hyperref[sailRISCVletzNIAzyfp]{#2}}{}%
  \ifstrequal{#1}{NIA\_fp}{\hyperref[sailRISCVletzNIAzyfp]{#2}}{}%
  \ifstrequal{#1}{PAGESIZE_BITS}{\hyperref[sailRISCVletzPAGESIZEzyBITS]{#2}}{}%
  \ifstrequal{#1}{PAGESIZE\_BITS}{\hyperref[sailRISCVletzPAGESIZEzyBITS]{#2}}{}%
  \ifstrequal{#1}{PCC_IDX}{\hyperref[sailRISCVletzPCCzyIDX]{#2}}{}%
  \ifstrequal{#1}{PCC\_IDX}{\hyperref[sailRISCVletzPCCzyIDX]{#2}}{}%
  \ifstrequal{#1}{PTE32_LOG_SIZE}{\hyperref[sailRISCVletzPTE32zyLOGzySIZE]{#2}}{}%
  \ifstrequal{#1}{PTE32\_LOG\_SIZE}{\hyperref[sailRISCVletzPTE32zyLOGzySIZE]{#2}}{}%
  \ifstrequal{#1}{PTE32_SIZE}{\hyperref[sailRISCVletzPTE32zySIZE]{#2}}{}%
  \ifstrequal{#1}{PTE32\_SIZE}{\hyperref[sailRISCVletzPTE32zySIZE]{#2}}{}%
  \ifstrequal{#1}{PTE39_LOG_SIZE}{\hyperref[sailRISCVletzPTE39zyLOGzySIZE]{#2}}{}%
  \ifstrequal{#1}{PTE39\_LOG\_SIZE}{\hyperref[sailRISCVletzPTE39zyLOGzySIZE]{#2}}{}%
  \ifstrequal{#1}{PTE39_SIZE}{\hyperref[sailRISCVletzPTE39zySIZE]{#2}}{}%
  \ifstrequal{#1}{PTE39\_SIZE}{\hyperref[sailRISCVletzPTE39zySIZE]{#2}}{}%
  \ifstrequal{#1}{PTE48_LOG_SIZE}{\hyperref[sailRISCVletzPTE48zyLOGzySIZE]{#2}}{}%
  \ifstrequal{#1}{PTE48\_LOG\_SIZE}{\hyperref[sailRISCVletzPTE48zyLOGzySIZE]{#2}}{}%
  \ifstrequal{#1}{PTE48_SIZE}{\hyperref[sailRISCVletzPTE48zySIZE]{#2}}{}%
  \ifstrequal{#1}{PTE48\_SIZE}{\hyperref[sailRISCVletzPTE48zySIZE]{#2}}{}%
  \ifstrequal{#1}{SV32_LEVELS}{\hyperref[sailRISCVletzSV32zyLEVELS]{#2}}{}%
  \ifstrequal{#1}{SV32\_LEVELS}{\hyperref[sailRISCVletzSV32zyLEVELS]{#2}}{}%
  \ifstrequal{#1}{SV32_LEVEL_BITS}{\hyperref[sailRISCVletzSV32zyLEVELzyBITS]{#2}}{}%
  \ifstrequal{#1}{SV32\_LEVEL\_BITS}{\hyperref[sailRISCVletzSV32zyLEVELzyBITS]{#2}}{}%
  \ifstrequal{#1}{SV39_LEVELS}{\hyperref[sailRISCVletzSV39zyLEVELS]{#2}}{}%
  \ifstrequal{#1}{SV39\_LEVELS}{\hyperref[sailRISCVletzSV39zyLEVELS]{#2}}{}%
  \ifstrequal{#1}{SV39_LEVEL_BITS}{\hyperref[sailRISCVletzSV39zyLEVELzyBITS]{#2}}{}%
  \ifstrequal{#1}{SV39\_LEVEL\_BITS}{\hyperref[sailRISCVletzSV39zyLEVELzyBITS]{#2}}{}%
  \ifstrequal{#1}{SV48_LEVELS}{\hyperref[sailRISCVletzSV48zyLEVELS]{#2}}{}%
  \ifstrequal{#1}{SV48\_LEVELS}{\hyperref[sailRISCVletzSV48zyLEVELS]{#2}}{}%
  \ifstrequal{#1}{SV48_LEVEL_BITS}{\hyperref[sailRISCVletzSV48zyLEVELzyBITS]{#2}}{}%
  \ifstrequal{#1}{SV48\_LEVEL\_BITS}{\hyperref[sailRISCVletzSV48zyLEVELzyBITS]{#2}}{}%
  \ifstrequal{#1}{cap_E_width}{\hyperref[sailRISCVletzcapzyEzywidth]{#2}}{}%
  \ifstrequal{#1}{cap\_E\_width}{\hyperref[sailRISCVletzcapzyEzywidth]{#2}}{}%
  \ifstrequal{#1}{cap_addr_width}{\hyperref[sailRISCVletzcapzyaddrzywidth]{#2}}{}%
  \ifstrequal{#1}{cap\_addr\_width}{\hyperref[sailRISCVletzcapzyaddrzywidth]{#2}}{}%
  \ifstrequal{#1}{cap_flags_width}{\hyperref[sailRISCVletzcapzyflagszywidth]{#2}}{}%
  \ifstrequal{#1}{cap\_flags\_width}{\hyperref[sailRISCVletzcapzyflagszywidth]{#2}}{}%
  \ifstrequal{#1}{cap_hperms_width}{\hyperref[sailRISCVletzcapzyhpermszywidth]{#2}}{}%
  \ifstrequal{#1}{cap\_hperms\_width}{\hyperref[sailRISCVletzcapzyhpermszywidth]{#2}}{}%
  \ifstrequal{#1}{cap_len_width}{\hyperref[sailRISCVletzcapzylenzywidth]{#2}}{}%
  \ifstrequal{#1}{cap\_len\_width}{\hyperref[sailRISCVletzcapzylenzywidth]{#2}}{}%
  \ifstrequal{#1}{cap_mantissa_width}{\hyperref[sailRISCVletzcapzymantissazywidth]{#2}}{}%
  \ifstrequal{#1}{cap\_mantissa\_width}{\hyperref[sailRISCVletzcapzymantissazywidth]{#2}}{}%
  \ifstrequal{#1}{cap_max_E}{\hyperref[sailRISCVletzcapzymaxzyE]{#2}}{}%
  \ifstrequal{#1}{cap\_max\_E}{\hyperref[sailRISCVletzcapzymaxzyE]{#2}}{}%
  \ifstrequal{#1}{cap_max_addr}{\hyperref[sailRISCVletzcapzymaxzyaddr]{#2}}{}%
  \ifstrequal{#1}{cap\_max\_addr}{\hyperref[sailRISCVletzcapzymaxzyaddr]{#2}}{}%
  \ifstrequal{#1}{cap_max_otype}{\hyperref[sailRISCVletzcapzymaxzyotype]{#2}}{}%
  \ifstrequal{#1}{cap\_max\_otype}{\hyperref[sailRISCVletzcapzymaxzyotype]{#2}}{}%
  \ifstrequal{#1}{cap_otype_width}{\hyperref[sailRISCVletzcapzyotypezywidth]{#2}}{}%
  \ifstrequal{#1}{cap\_otype\_width}{\hyperref[sailRISCVletzcapzyotypezywidth]{#2}}{}%
  \ifstrequal{#1}{cap_perms_width}{\hyperref[sailRISCVletzcapzypermszywidth]{#2}}{}%
  \ifstrequal{#1}{cap\_perms\_width}{\hyperref[sailRISCVletzcapzypermszywidth]{#2}}{}%
  \ifstrequal{#1}{cap_reserved_width}{\hyperref[sailRISCVletzcapzyreservedzywidth]{#2}}{}%
  \ifstrequal{#1}{cap\_reserved\_width}{\hyperref[sailRISCVletzcapzyreservedzywidth]{#2}}{}%
  \ifstrequal{#1}{cap_reset_E}{\hyperref[sailRISCVletzcapzyresetzyE]{#2}}{}%
  \ifstrequal{#1}{cap\_reset\_E}{\hyperref[sailRISCVletzcapzyresetzyE]{#2}}{}%
  \ifstrequal{#1}{cap_reset_T}{\hyperref[sailRISCVletzcapzyresetzyT]{#2}}{}%
  \ifstrequal{#1}{cap\_reset\_T}{\hyperref[sailRISCVletzcapzyresetzyT]{#2}}{}%
  \ifstrequal{#1}{cap_size}{\hyperref[sailRISCVletzcapzysizze]{#2}}{}%
  \ifstrequal{#1}{cap\_size}{\hyperref[sailRISCVletzcapzysizze]{#2}}{}%
  \ifstrequal{#1}{cap_uperms_shift}{\hyperref[sailRISCVletzcapzyupermszyshift]{#2}}{}%
  \ifstrequal{#1}{cap\_uperms\_shift}{\hyperref[sailRISCVletzcapzyupermszyshift]{#2}}{}%
  \ifstrequal{#1}{cap_uperms_width}{\hyperref[sailRISCVletzcapzyupermszywidth]{#2}}{}%
  \ifstrequal{#1}{cap\_uperms\_width}{\hyperref[sailRISCVletzcapzyupermszywidth]{#2}}{}%
  \ifstrequal{#1}{caps_per_cache_line}{\hyperref[sailRISCVletzcapszyperzycachezyline]{#2}}{}%
  \ifstrequal{#1}{caps\_per\_cache\_line}{\hyperref[sailRISCVletzcapszyperzycachezyline]{#2}}{}%
  \ifstrequal{#1}{default_cap}{\hyperref[sailRISCVletzdefaultzycap]{#2}}{}%
  \ifstrequal{#1}{default\_cap}{\hyperref[sailRISCVletzdefaultzycap]{#2}}{}%
  \ifstrequal{#1}{default_meta}{\hyperref[sailRISCVletzdefaultzymeta]{#2}}{}%
  \ifstrequal{#1}{default\_meta}{\hyperref[sailRISCVletzdefaultzymeta]{#2}}{}%
  \ifstrequal{#1}{default_sv32_ext_pte}{\hyperref[sailRISCVletzdefaultzysv32zyextzypte]{#2}}{}%
  \ifstrequal{#1}{default\_sv32\_ext\_pte}{\hyperref[sailRISCVletzdefaultzysv32zyextzypte]{#2}}{}%
  \ifstrequal{#1}{default_write_acc}{\hyperref[sailRISCVletzdefaultzywritezyacc]{#2}}{}%
  \ifstrequal{#1}{default\_write\_acc}{\hyperref[sailRISCVletzdefaultzywritezyacc]{#2}}{}%
  \ifstrequal{#1}{haveRV128}{\hyperref[sailRISCVletzhaveRV128]{#2}}{}%
  \ifstrequal{#1}{haveRV64}{\hyperref[sailRISCVletzhaveRV64]{#2}}{}%
  \ifstrequal{#1}{haveSplitRegFile}{\hyperref[sailRISCVletzhaveSplitRegFile]{#2}}{}%
  \ifstrequal{#1}{init_ext_ptw}{\hyperref[sailRISCVletzinitzyextzyptw]{#2}}{}%
  \ifstrequal{#1}{init\_ext\_ptw}{\hyperref[sailRISCVletzinitzyextzyptw]{#2}}{}%
  \ifstrequal{#1}{internal_E_take_bits}{\hyperref[sailRISCVletzinternalzyEzytakezybits]{#2}}{}%
  \ifstrequal{#1}{internal\_E\_take\_bits}{\hyperref[sailRISCVletzinternalzyEzytakezybits]{#2}}{}%
  \ifstrequal{#1}{log2_cap_size}{\hyperref[sailRISCVletzlog2zycapzysizze]{#2}}{}%
  \ifstrequal{#1}{log2\_cap\_size}{\hyperref[sailRISCVletzlog2zycapzysizze]{#2}}{}%
  \ifstrequal{#1}{null_cap}{\hyperref[sailRISCVletznullzycap]{#2}}{}%
  \ifstrequal{#1}{null\_cap}{\hyperref[sailRISCVletznullzycap]{#2}}{}%
  \ifstrequal{#1}{null_cap_bits}{\hyperref[sailRISCVletznullzycapzybits]{#2}}{}%
  \ifstrequal{#1}{null\_cap\_bits}{\hyperref[sailRISCVletznullzycapzybits]{#2}}{}%
  \ifstrequal{#1}{otype_sentry}{\hyperref[sailRISCVletzotypezysentry]{#2}}{}%
  \ifstrequal{#1}{otype\_sentry}{\hyperref[sailRISCVletzotypezysentry]{#2}}{}%
  \ifstrequal{#1}{otype_unsealed}{\hyperref[sailRISCVletzotypezyunsealed]{#2}}{}%
  \ifstrequal{#1}{otype\_unsealed}{\hyperref[sailRISCVletzotypezyunsealed]{#2}}{}%
  \ifstrequal{#1}{ra}{\hyperref[sailRISCVletzra]{#2}}{}%
  \ifstrequal{#1}{reserved_otypes}{\hyperref[sailRISCVletzreservedzyotypes]{#2}}{}%
  \ifstrequal{#1}{reserved\_otypes}{\hyperref[sailRISCVletzreservedzyotypes]{#2}}{}%
  \ifstrequal{#1}{sp}{\hyperref[sailRISCVletzsp]{#2}}{}%
  \ifstrequal{#1}{xlen_max_signed}{\hyperref[sailRISCVletzxlenzymaxzysigned]{#2}}{}%
  \ifstrequal{#1}{xlen\_max\_signed}{\hyperref[sailRISCVletzxlenzymaxzysigned]{#2}}{}%
  \ifstrequal{#1}{xlen_max_unsigned}{\hyperref[sailRISCVletzxlenzymaxzyunsigned]{#2}}{}%
  \ifstrequal{#1}{xlen\_max\_unsigned}{\hyperref[sailRISCVletzxlenzymaxzyunsigned]{#2}}{}%
  \ifstrequal{#1}{xlen_min_signed}{\hyperref[sailRISCVletzxlenzyminzysigned]{#2}}{}%
  \ifstrequal{#1}{xlen\_min\_signed}{\hyperref[sailRISCVletzxlenzyminzysigned]{#2}}{}%
  \ifstrequal{#1}{xlen_val}{\hyperref[sailRISCVletzxlenzyval]{#2}}{}%
  \ifstrequal{#1}{xlen\_val}{\hyperref[sailRISCVletzxlenzyval]{#2}}{}%
  \ifstrequal{#1}{zero_freg}{\hyperref[sailRISCVletzzzerozyfreg]{#2}}{}%
  \ifstrequal{#1}{zero\_freg}{\hyperref[sailRISCVletzzzerozyfreg]{#2}}{}%
  \ifstrequal{#1}{zero_reg}{\hyperref[sailRISCVletzzzerozyreg]{#2}}{}%
  \ifstrequal{#1}{zero\_reg}{\hyperref[sailRISCVletzzzerozyreg]{#2}}{}%
  \ifstrequal{#1}{zreg}{\hyperref[sailRISCVletzzzreg]{#2}}{}}

\newcommand{\sailRISCVregister}[1]{
  \ifstrequal{#1}{DDC}{\sailRISCVregisterDDC}{}%
  \ifstrequal{#1}{MEPCC}{\sailRISCVregisterMEPCC}{}%
  \ifstrequal{#1}{MScratchC}{\sailRISCVregisterMScratchC}{}%
  \ifstrequal{#1}{MTCC}{\sailRISCVregisterMTCC}{}%
  \ifstrequal{#1}{MTDC}{\sailRISCVregisterMTDC}{}%
  \ifstrequal{#1}{PC}{\sailRISCVregisterPC}{}%
  \ifstrequal{#1}{PCC}{\sailRISCVregisterPCC}{}%
  \ifstrequal{#1}{SEPCC}{\sailRISCVregisterSEPCC}{}%
  \ifstrequal{#1}{SScratchC}{\sailRISCVregisterSScratchC}{}%
  \ifstrequal{#1}{STCC}{\sailRISCVregisterSTCC}{}%
  \ifstrequal{#1}{STDC}{\sailRISCVregisterSTDC}{}%
  \ifstrequal{#1}{UEPCC}{\sailRISCVregisterUEPCC}{}%
  \ifstrequal{#1}{UScratchC}{\sailRISCVregisterUScratchC}{}%
  \ifstrequal{#1}{UTCC}{\sailRISCVregisterUTCC}{}%
  \ifstrequal{#1}{UTDC}{\sailRISCVregisterUTDC}{}%
  \ifstrequal{#1}{cur_inst}{\sailRISCVregistercurInst}{}%
  \ifstrequal{#1}{cur\_inst}{\sailRISCVregistercurInst}{}%
  \ifstrequal{#1}{cur_privilege}{\sailRISCVregistercurPrivilege}{}%
  \ifstrequal{#1}{cur\_privilege}{\sailRISCVregistercurPrivilege}{}%
  \ifstrequal{#1}{f0}{\sailRISCVregisterfZero}{}%
  \ifstrequal{#1}{f1}{\sailRISCVregisterfOne}{}%
  \ifstrequal{#1}{f10}{\sailRISCVregisterfOneZero}{}%
  \ifstrequal{#1}{f11}{\sailRISCVregisterfOneOne}{}%
  \ifstrequal{#1}{f12}{\sailRISCVregisterfOneTwo}{}%
  \ifstrequal{#1}{f13}{\sailRISCVregisterfOneThree}{}%
  \ifstrequal{#1}{f14}{\sailRISCVregisterfOneFour}{}%
  \ifstrequal{#1}{f15}{\sailRISCVregisterfOneFive}{}%
  \ifstrequal{#1}{f16}{\sailRISCVregisterfOneSix}{}%
  \ifstrequal{#1}{f17}{\sailRISCVregisterfOneSeven}{}%
  \ifstrequal{#1}{f18}{\sailRISCVregisterfOneEight}{}%
  \ifstrequal{#1}{f19}{\sailRISCVregisterfOneNine}{}%
  \ifstrequal{#1}{f2}{\sailRISCVregisterfTwo}{}%
  \ifstrequal{#1}{f20}{\sailRISCVregisterfTwoZero}{}%
  \ifstrequal{#1}{f21}{\sailRISCVregisterfTwoOne}{}%
  \ifstrequal{#1}{f22}{\sailRISCVregisterfTwoTwo}{}%
  \ifstrequal{#1}{f23}{\sailRISCVregisterfTwoThree}{}%
  \ifstrequal{#1}{f24}{\sailRISCVregisterfTwoFour}{}%
  \ifstrequal{#1}{f25}{\sailRISCVregisterfTwoFive}{}%
  \ifstrequal{#1}{f26}{\sailRISCVregisterfTwoSix}{}%
  \ifstrequal{#1}{f27}{\sailRISCVregisterfTwoSeven}{}%
  \ifstrequal{#1}{f28}{\sailRISCVregisterfTwoEight}{}%
  \ifstrequal{#1}{f29}{\sailRISCVregisterfTwoNine}{}%
  \ifstrequal{#1}{f3}{\sailRISCVregisterfThree}{}%
  \ifstrequal{#1}{f30}{\sailRISCVregisterfThreeZero}{}%
  \ifstrequal{#1}{f31}{\sailRISCVregisterfThreeOne}{}%
  \ifstrequal{#1}{f4}{\sailRISCVregisterfFour}{}%
  \ifstrequal{#1}{f5}{\sailRISCVregisterfFive}{}%
  \ifstrequal{#1}{f6}{\sailRISCVregisterfSix}{}%
  \ifstrequal{#1}{f7}{\sailRISCVregisterfSeven}{}%
  \ifstrequal{#1}{f8}{\sailRISCVregisterfEight}{}%
  \ifstrequal{#1}{f9}{\sailRISCVregisterfNine}{}%
  \ifstrequal{#1}{fcsr}{\sailRISCVregisterfcsr}{}%
  \ifstrequal{#1}{float_fflags}{\sailRISCVregisterfloatFflags}{}%
  \ifstrequal{#1}{float\_fflags}{\sailRISCVregisterfloatFflags}{}%
  \ifstrequal{#1}{float_result}{\sailRISCVregisterfloatResult}{}%
  \ifstrequal{#1}{float\_result}{\sailRISCVregisterfloatResult}{}%
  \ifstrequal{#1}{htif_cmd_write}{\sailRISCVregisterhtifCmdWrite}{}%
  \ifstrequal{#1}{htif\_cmd\_write}{\sailRISCVregisterhtifCmdWrite}{}%
  \ifstrequal{#1}{htif_done}{\sailRISCVregisterhtifDone}{}%
  \ifstrequal{#1}{htif\_done}{\sailRISCVregisterhtifDone}{}%
  \ifstrequal{#1}{htif_exit_code}{\sailRISCVregisterhtifExitCode}{}%
  \ifstrequal{#1}{htif\_exit\_code}{\sailRISCVregisterhtifExitCode}{}%
  \ifstrequal{#1}{htif_payload_writes}{\sailRISCVregisterhtifPayloadWrites}{}%
  \ifstrequal{#1}{htif\_payload\_writes}{\sailRISCVregisterhtifPayloadWrites}{}%
  \ifstrequal{#1}{htif_tohost}{\sailRISCVregisterhtifTohost}{}%
  \ifstrequal{#1}{htif\_tohost}{\sailRISCVregisterhtifTohost}{}%
  \ifstrequal{#1}{instbits}{\sailRISCVregisterinstbits}{}%
  \ifstrequal{#1}{marchid}{\sailRISCVregistermarchid}{}%
  \ifstrequal{#1}{mcause}{\sailRISCVregistermcause}{}%
  \ifstrequal{#1}{mccsr}{\sailRISCVregistermccsr}{}%
  \ifstrequal{#1}{mcounteren}{\sailRISCVregistermcounteren}{}%
  \ifstrequal{#1}{mcountinhibit}{\sailRISCVregistermcountinhibit}{}%
  \ifstrequal{#1}{mcycle}{\sailRISCVregistermcycle}{}%
  \ifstrequal{#1}{medeleg}{\sailRISCVregistermedeleg}{}%
  \ifstrequal{#1}{mepc}{\sailRISCVregistermepc}{}%
  \ifstrequal{#1}{mhartid}{\sailRISCVregistermhartid}{}%
  \ifstrequal{#1}{mideleg}{\sailRISCVregistermideleg}{}%
  \ifstrequal{#1}{mie}{\sailRISCVregistermie}{}%
  \ifstrequal{#1}{mimpid}{\sailRISCVregistermimpid}{}%
  \ifstrequal{#1}{minstret}{\sailRISCVregisterminstret}{}%
  \ifstrequal{#1}{minstret_written}{\sailRISCVregisterminstretWritten}{}%
  \ifstrequal{#1}{minstret\_written}{\sailRISCVregisterminstretWritten}{}%
  \ifstrequal{#1}{mip}{\sailRISCVregistermip}{}%
  \ifstrequal{#1}{misa}{\sailRISCVregistermisa}{}%
  \ifstrequal{#1}{mscratch}{\sailRISCVregistermscratch}{}%
  \ifstrequal{#1}{mstatus}{\sailRISCVregistermstatus}{}%
  \ifstrequal{#1}{mstatush}{\sailRISCVregistermstatush}{}%
  \ifstrequal{#1}{mtime}{\sailRISCVregistermtime}{}%
  \ifstrequal{#1}{mtimecmp}{\sailRISCVregistermtimecmp}{}%
  \ifstrequal{#1}{mtval}{\sailRISCVregistermtval}{}%
  \ifstrequal{#1}{mtvec}{\sailRISCVregistermtvec}{}%
  \ifstrequal{#1}{mvendorid}{\sailRISCVregistermvendorid}{}%
  \ifstrequal{#1}{nextPC}{\sailRISCVregisternextPC}{}%
  \ifstrequal{#1}{nextPCC}{\sailRISCVregisternextPCC}{}%
  \ifstrequal{#1}{pmp0cfg}{\sailRISCVregisterpmpZerocfg}{}%
  \ifstrequal{#1}{pmp10cfg}{\sailRISCVregisterpmpOneZerocfg}{}%
  \ifstrequal{#1}{pmp11cfg}{\sailRISCVregisterpmpOneOnecfg}{}%
  \ifstrequal{#1}{pmp12cfg}{\sailRISCVregisterpmpOneTwocfg}{}%
  \ifstrequal{#1}{pmp13cfg}{\sailRISCVregisterpmpOneThreecfg}{}%
  \ifstrequal{#1}{pmp14cfg}{\sailRISCVregisterpmpOneFourcfg}{}%
  \ifstrequal{#1}{pmp15cfg}{\sailRISCVregisterpmpOneFivecfg}{}%
  \ifstrequal{#1}{pmp1cfg}{\sailRISCVregisterpmpOnecfg}{}%
  \ifstrequal{#1}{pmp2cfg}{\sailRISCVregisterpmpTwocfg}{}%
  \ifstrequal{#1}{pmp3cfg}{\sailRISCVregisterpmpThreecfg}{}%
  \ifstrequal{#1}{pmp4cfg}{\sailRISCVregisterpmpFourcfg}{}%
  \ifstrequal{#1}{pmp5cfg}{\sailRISCVregisterpmpFivecfg}{}%
  \ifstrequal{#1}{pmp6cfg}{\sailRISCVregisterpmpSixcfg}{}%
  \ifstrequal{#1}{pmp7cfg}{\sailRISCVregisterpmpSevencfg}{}%
  \ifstrequal{#1}{pmp8cfg}{\sailRISCVregisterpmpEightcfg}{}%
  \ifstrequal{#1}{pmp9cfg}{\sailRISCVregisterpmpNinecfg}{}%
  \ifstrequal{#1}{pmpaddr0}{\sailRISCVregisterpmpaddrZero}{}%
  \ifstrequal{#1}{pmpaddr1}{\sailRISCVregisterpmpaddrOne}{}%
  \ifstrequal{#1}{pmpaddr10}{\sailRISCVregisterpmpaddrOneZero}{}%
  \ifstrequal{#1}{pmpaddr11}{\sailRISCVregisterpmpaddrOneOne}{}%
  \ifstrequal{#1}{pmpaddr12}{\sailRISCVregisterpmpaddrOneTwo}{}%
  \ifstrequal{#1}{pmpaddr13}{\sailRISCVregisterpmpaddrOneThree}{}%
  \ifstrequal{#1}{pmpaddr14}{\sailRISCVregisterpmpaddrOneFour}{}%
  \ifstrequal{#1}{pmpaddr15}{\sailRISCVregisterpmpaddrOneFive}{}%
  \ifstrequal{#1}{pmpaddr2}{\sailRISCVregisterpmpaddrTwo}{}%
  \ifstrequal{#1}{pmpaddr3}{\sailRISCVregisterpmpaddrThree}{}%
  \ifstrequal{#1}{pmpaddr4}{\sailRISCVregisterpmpaddrFour}{}%
  \ifstrequal{#1}{pmpaddr5}{\sailRISCVregisterpmpaddrFive}{}%
  \ifstrequal{#1}{pmpaddr6}{\sailRISCVregisterpmpaddrSix}{}%
  \ifstrequal{#1}{pmpaddr7}{\sailRISCVregisterpmpaddrSeven}{}%
  \ifstrequal{#1}{pmpaddr8}{\sailRISCVregisterpmpaddrEight}{}%
  \ifstrequal{#1}{pmpaddr9}{\sailRISCVregisterpmpaddrNine}{}%
  \ifstrequal{#1}{satp}{\sailRISCVregistersatp}{}%
  \ifstrequal{#1}{scause}{\sailRISCVregisterscause}{}%
  \ifstrequal{#1}{sccsr}{\sailRISCVregistersccsr}{}%
  \ifstrequal{#1}{scounteren}{\sailRISCVregisterscounteren}{}%
  \ifstrequal{#1}{sedeleg}{\sailRISCVregistersedeleg}{}%
  \ifstrequal{#1}{sepc}{\sailRISCVregistersepc}{}%
  \ifstrequal{#1}{sideleg}{\sailRISCVregistersideleg}{}%
  \ifstrequal{#1}{sscratch}{\sailRISCVregistersscratch}{}%
  \ifstrequal{#1}{stval}{\sailRISCVregisterstval}{}%
  \ifstrequal{#1}{stvec}{\sailRISCVregisterstvec}{}%
  \ifstrequal{#1}{tlb39}{\sailRISCVregistertlbThreeNine}{}%
  \ifstrequal{#1}{tlb48}{\sailRISCVregistertlbFourEight}{}%
  \ifstrequal{#1}{tselect}{\sailRISCVregistertselect}{}%
  \ifstrequal{#1}{ucause}{\sailRISCVregisterucause}{}%
  \ifstrequal{#1}{uccsr}{\sailRISCVregisteruccsr}{}%
  \ifstrequal{#1}{uepc}{\sailRISCVregisteruepc}{}%
  \ifstrequal{#1}{uscratch}{\sailRISCVregisteruscratch}{}%
  \ifstrequal{#1}{utval}{\sailRISCVregisterutval}{}%
  \ifstrequal{#1}{utvec}{\sailRISCVregisterutvec}{}%
  \ifstrequal{#1}{x1}{\sailRISCVregisterxOne}{}%
  \ifstrequal{#1}{x10}{\sailRISCVregisterxOneZero}{}%
  \ifstrequal{#1}{x11}{\sailRISCVregisterxOneOne}{}%
  \ifstrequal{#1}{x12}{\sailRISCVregisterxOneTwo}{}%
  \ifstrequal{#1}{x13}{\sailRISCVregisterxOneThree}{}%
  \ifstrequal{#1}{x14}{\sailRISCVregisterxOneFour}{}%
  \ifstrequal{#1}{x15}{\sailRISCVregisterxOneFive}{}%
  \ifstrequal{#1}{x16}{\sailRISCVregisterxOneSix}{}%
  \ifstrequal{#1}{x17}{\sailRISCVregisterxOneSeven}{}%
  \ifstrequal{#1}{x18}{\sailRISCVregisterxOneEight}{}%
  \ifstrequal{#1}{x19}{\sailRISCVregisterxOneNine}{}%
  \ifstrequal{#1}{x2}{\sailRISCVregisterxTwo}{}%
  \ifstrequal{#1}{x20}{\sailRISCVregisterxTwoZero}{}%
  \ifstrequal{#1}{x21}{\sailRISCVregisterxTwoOne}{}%
  \ifstrequal{#1}{x22}{\sailRISCVregisterxTwoTwo}{}%
  \ifstrequal{#1}{x23}{\sailRISCVregisterxTwoThree}{}%
  \ifstrequal{#1}{x24}{\sailRISCVregisterxTwoFour}{}%
  \ifstrequal{#1}{x25}{\sailRISCVregisterxTwoFive}{}%
  \ifstrequal{#1}{x26}{\sailRISCVregisterxTwoSix}{}%
  \ifstrequal{#1}{x27}{\sailRISCVregisterxTwoSeven}{}%
  \ifstrequal{#1}{x28}{\sailRISCVregisterxTwoEight}{}%
  \ifstrequal{#1}{x29}{\sailRISCVregisterxTwoNine}{}%
  \ifstrequal{#1}{x3}{\sailRISCVregisterxThree}{}%
  \ifstrequal{#1}{x30}{\sailRISCVregisterxThreeZero}{}%
  \ifstrequal{#1}{x31}{\sailRISCVregisterxThreeOne}{}%
  \ifstrequal{#1}{x4}{\sailRISCVregisterxFour}{}%
  \ifstrequal{#1}{x5}{\sailRISCVregisterxFive}{}%
  \ifstrequal{#1}{x6}{\sailRISCVregisterxSix}{}%
  \ifstrequal{#1}{x7}{\sailRISCVregisterxSeven}{}%
  \ifstrequal{#1}{x8}{\sailRISCVregisterxEight}{}%
  \ifstrequal{#1}{x9}{\sailRISCVregisterxNine}{}}

\newcommand{\sailRISCVrefregister}[2]{
  \ifstrequal{#1}{DDC}{\hyperref[sailRISCVregisterzDDC]{#2}}{}%
  \ifstrequal{#1}{MEPCC}{\hyperref[sailRISCVregisterzMEPCC]{#2}}{}%
  \ifstrequal{#1}{MScratchC}{\hyperref[sailRISCVregisterzMScratchC]{#2}}{}%
  \ifstrequal{#1}{MTCC}{\hyperref[sailRISCVregisterzMTCC]{#2}}{}%
  \ifstrequal{#1}{MTDC}{\hyperref[sailRISCVregisterzMTDC]{#2}}{}%
  \ifstrequal{#1}{PC}{\hyperref[sailRISCVregisterzPC]{#2}}{}%
  \ifstrequal{#1}{PCC}{\hyperref[sailRISCVregisterzPCC]{#2}}{}%
  \ifstrequal{#1}{SEPCC}{\hyperref[sailRISCVregisterzSEPCC]{#2}}{}%
  \ifstrequal{#1}{SScratchC}{\hyperref[sailRISCVregisterzSScratchC]{#2}}{}%
  \ifstrequal{#1}{STCC}{\hyperref[sailRISCVregisterzSTCC]{#2}}{}%
  \ifstrequal{#1}{STDC}{\hyperref[sailRISCVregisterzSTDC]{#2}}{}%
  \ifstrequal{#1}{UEPCC}{\hyperref[sailRISCVregisterzUEPCC]{#2}}{}%
  \ifstrequal{#1}{UScratchC}{\hyperref[sailRISCVregisterzUScratchC]{#2}}{}%
  \ifstrequal{#1}{UTCC}{\hyperref[sailRISCVregisterzUTCC]{#2}}{}%
  \ifstrequal{#1}{UTDC}{\hyperref[sailRISCVregisterzUTDC]{#2}}{}%
  \ifstrequal{#1}{cur_inst}{\hyperref[sailRISCVregisterzcurzyinst]{#2}}{}%
  \ifstrequal{#1}{cur\_inst}{\hyperref[sailRISCVregisterzcurzyinst]{#2}}{}%
  \ifstrequal{#1}{cur_privilege}{\hyperref[sailRISCVregisterzcurzyprivilege]{#2}}{}%
  \ifstrequal{#1}{cur\_privilege}{\hyperref[sailRISCVregisterzcurzyprivilege]{#2}}{}%
  \ifstrequal{#1}{f0}{\hyperref[sailRISCVregisterzf0]{#2}}{}%
  \ifstrequal{#1}{f1}{\hyperref[sailRISCVregisterzf1]{#2}}{}%
  \ifstrequal{#1}{f10}{\hyperref[sailRISCVregisterzf10]{#2}}{}%
  \ifstrequal{#1}{f11}{\hyperref[sailRISCVregisterzf11]{#2}}{}%
  \ifstrequal{#1}{f12}{\hyperref[sailRISCVregisterzf12]{#2}}{}%
  \ifstrequal{#1}{f13}{\hyperref[sailRISCVregisterzf13]{#2}}{}%
  \ifstrequal{#1}{f14}{\hyperref[sailRISCVregisterzf14]{#2}}{}%
  \ifstrequal{#1}{f15}{\hyperref[sailRISCVregisterzf15]{#2}}{}%
  \ifstrequal{#1}{f16}{\hyperref[sailRISCVregisterzf16]{#2}}{}%
  \ifstrequal{#1}{f17}{\hyperref[sailRISCVregisterzf17]{#2}}{}%
  \ifstrequal{#1}{f18}{\hyperref[sailRISCVregisterzf18]{#2}}{}%
  \ifstrequal{#1}{f19}{\hyperref[sailRISCVregisterzf19]{#2}}{}%
  \ifstrequal{#1}{f2}{\hyperref[sailRISCVregisterzf2]{#2}}{}%
  \ifstrequal{#1}{f20}{\hyperref[sailRISCVregisterzf20]{#2}}{}%
  \ifstrequal{#1}{f21}{\hyperref[sailRISCVregisterzf21]{#2}}{}%
  \ifstrequal{#1}{f22}{\hyperref[sailRISCVregisterzf22]{#2}}{}%
  \ifstrequal{#1}{f23}{\hyperref[sailRISCVregisterzf23]{#2}}{}%
  \ifstrequal{#1}{f24}{\hyperref[sailRISCVregisterzf24]{#2}}{}%
  \ifstrequal{#1}{f25}{\hyperref[sailRISCVregisterzf25]{#2}}{}%
  \ifstrequal{#1}{f26}{\hyperref[sailRISCVregisterzf26]{#2}}{}%
  \ifstrequal{#1}{f27}{\hyperref[sailRISCVregisterzf27]{#2}}{}%
  \ifstrequal{#1}{f28}{\hyperref[sailRISCVregisterzf28]{#2}}{}%
  \ifstrequal{#1}{f29}{\hyperref[sailRISCVregisterzf29]{#2}}{}%
  \ifstrequal{#1}{f3}{\hyperref[sailRISCVregisterzf3]{#2}}{}%
  \ifstrequal{#1}{f30}{\hyperref[sailRISCVregisterzf30]{#2}}{}%
  \ifstrequal{#1}{f31}{\hyperref[sailRISCVregisterzf31]{#2}}{}%
  \ifstrequal{#1}{f4}{\hyperref[sailRISCVregisterzf4]{#2}}{}%
  \ifstrequal{#1}{f5}{\hyperref[sailRISCVregisterzf5]{#2}}{}%
  \ifstrequal{#1}{f6}{\hyperref[sailRISCVregisterzf6]{#2}}{}%
  \ifstrequal{#1}{f7}{\hyperref[sailRISCVregisterzf7]{#2}}{}%
  \ifstrequal{#1}{f8}{\hyperref[sailRISCVregisterzf8]{#2}}{}%
  \ifstrequal{#1}{f9}{\hyperref[sailRISCVregisterzf9]{#2}}{}%
  \ifstrequal{#1}{fcsr}{\hyperref[sailRISCVregisterzfcsr]{#2}}{}%
  \ifstrequal{#1}{float_fflags}{\hyperref[sailRISCVregisterzfloatzyfflags]{#2}}{}%
  \ifstrequal{#1}{float\_fflags}{\hyperref[sailRISCVregisterzfloatzyfflags]{#2}}{}%
  \ifstrequal{#1}{float_result}{\hyperref[sailRISCVregisterzfloatzyresult]{#2}}{}%
  \ifstrequal{#1}{float\_result}{\hyperref[sailRISCVregisterzfloatzyresult]{#2}}{}%
  \ifstrequal{#1}{htif_cmd_write}{\hyperref[sailRISCVregisterzhtifzycmdzywrite]{#2}}{}%
  \ifstrequal{#1}{htif\_cmd\_write}{\hyperref[sailRISCVregisterzhtifzycmdzywrite]{#2}}{}%
  \ifstrequal{#1}{htif_done}{\hyperref[sailRISCVregisterzhtifzydone]{#2}}{}%
  \ifstrequal{#1}{htif\_done}{\hyperref[sailRISCVregisterzhtifzydone]{#2}}{}%
  \ifstrequal{#1}{htif_exit_code}{\hyperref[sailRISCVregisterzhtifzyexitzycode]{#2}}{}%
  \ifstrequal{#1}{htif\_exit\_code}{\hyperref[sailRISCVregisterzhtifzyexitzycode]{#2}}{}%
  \ifstrequal{#1}{htif_payload_writes}{\hyperref[sailRISCVregisterzhtifzypayloadzywrites]{#2}}{}%
  \ifstrequal{#1}{htif\_payload\_writes}{\hyperref[sailRISCVregisterzhtifzypayloadzywrites]{#2}}{}%
  \ifstrequal{#1}{htif_tohost}{\hyperref[sailRISCVregisterzhtifzytohost]{#2}}{}%
  \ifstrequal{#1}{htif\_tohost}{\hyperref[sailRISCVregisterzhtifzytohost]{#2}}{}%
  \ifstrequal{#1}{instbits}{\hyperref[sailRISCVregisterzinstbits]{#2}}{}%
  \ifstrequal{#1}{marchid}{\hyperref[sailRISCVregisterzmarchid]{#2}}{}%
  \ifstrequal{#1}{mcause}{\hyperref[sailRISCVregisterzmcause]{#2}}{}%
  \ifstrequal{#1}{mccsr}{\hyperref[sailRISCVregisterzmccsr]{#2}}{}%
  \ifstrequal{#1}{mcounteren}{\hyperref[sailRISCVregisterzmcounteren]{#2}}{}%
  \ifstrequal{#1}{mcountinhibit}{\hyperref[sailRISCVregisterzmcountinhibit]{#2}}{}%
  \ifstrequal{#1}{mcycle}{\hyperref[sailRISCVregisterzmcycle]{#2}}{}%
  \ifstrequal{#1}{medeleg}{\hyperref[sailRISCVregisterzmedeleg]{#2}}{}%
  \ifstrequal{#1}{mepc}{\hyperref[sailRISCVregisterzmepc]{#2}}{}%
  \ifstrequal{#1}{mhartid}{\hyperref[sailRISCVregisterzmhartid]{#2}}{}%
  \ifstrequal{#1}{mideleg}{\hyperref[sailRISCVregisterzmideleg]{#2}}{}%
  \ifstrequal{#1}{mie}{\hyperref[sailRISCVregisterzmie]{#2}}{}%
  \ifstrequal{#1}{mimpid}{\hyperref[sailRISCVregisterzmimpid]{#2}}{}%
  \ifstrequal{#1}{minstret}{\hyperref[sailRISCVregisterzminstret]{#2}}{}%
  \ifstrequal{#1}{minstret_written}{\hyperref[sailRISCVregisterzminstretzywritten]{#2}}{}%
  \ifstrequal{#1}{minstret\_written}{\hyperref[sailRISCVregisterzminstretzywritten]{#2}}{}%
  \ifstrequal{#1}{mip}{\hyperref[sailRISCVregisterzmip]{#2}}{}%
  \ifstrequal{#1}{misa}{\hyperref[sailRISCVregisterzmisa]{#2}}{}%
  \ifstrequal{#1}{mscratch}{\hyperref[sailRISCVregisterzmscratch]{#2}}{}%
  \ifstrequal{#1}{mstatus}{\hyperref[sailRISCVregisterzmstatus]{#2}}{}%
  \ifstrequal{#1}{mstatush}{\hyperref[sailRISCVregisterzmstatush]{#2}}{}%
  \ifstrequal{#1}{mtime}{\hyperref[sailRISCVregisterzmtime]{#2}}{}%
  \ifstrequal{#1}{mtimecmp}{\hyperref[sailRISCVregisterzmtimecmp]{#2}}{}%
  \ifstrequal{#1}{mtval}{\hyperref[sailRISCVregisterzmtval]{#2}}{}%
  \ifstrequal{#1}{mtvec}{\hyperref[sailRISCVregisterzmtvec]{#2}}{}%
  \ifstrequal{#1}{mvendorid}{\hyperref[sailRISCVregisterzmvendorid]{#2}}{}%
  \ifstrequal{#1}{nextPC}{\hyperref[sailRISCVregisterznextPC]{#2}}{}%
  \ifstrequal{#1}{nextPCC}{\hyperref[sailRISCVregisterznextPCC]{#2}}{}%
  \ifstrequal{#1}{pmp0cfg}{\hyperref[sailRISCVregisterzpmp0cfg]{#2}}{}%
  \ifstrequal{#1}{pmp10cfg}{\hyperref[sailRISCVregisterzpmp10cfg]{#2}}{}%
  \ifstrequal{#1}{pmp11cfg}{\hyperref[sailRISCVregisterzpmp11cfg]{#2}}{}%
  \ifstrequal{#1}{pmp12cfg}{\hyperref[sailRISCVregisterzpmp12cfg]{#2}}{}%
  \ifstrequal{#1}{pmp13cfg}{\hyperref[sailRISCVregisterzpmp13cfg]{#2}}{}%
  \ifstrequal{#1}{pmp14cfg}{\hyperref[sailRISCVregisterzpmp14cfg]{#2}}{}%
  \ifstrequal{#1}{pmp15cfg}{\hyperref[sailRISCVregisterzpmp15cfg]{#2}}{}%
  \ifstrequal{#1}{pmp1cfg}{\hyperref[sailRISCVregisterzpmp1cfg]{#2}}{}%
  \ifstrequal{#1}{pmp2cfg}{\hyperref[sailRISCVregisterzpmp2cfg]{#2}}{}%
  \ifstrequal{#1}{pmp3cfg}{\hyperref[sailRISCVregisterzpmp3cfg]{#2}}{}%
  \ifstrequal{#1}{pmp4cfg}{\hyperref[sailRISCVregisterzpmp4cfg]{#2}}{}%
  \ifstrequal{#1}{pmp5cfg}{\hyperref[sailRISCVregisterzpmp5cfg]{#2}}{}%
  \ifstrequal{#1}{pmp6cfg}{\hyperref[sailRISCVregisterzpmp6cfg]{#2}}{}%
  \ifstrequal{#1}{pmp7cfg}{\hyperref[sailRISCVregisterzpmp7cfg]{#2}}{}%
  \ifstrequal{#1}{pmp8cfg}{\hyperref[sailRISCVregisterzpmp8cfg]{#2}}{}%
  \ifstrequal{#1}{pmp9cfg}{\hyperref[sailRISCVregisterzpmp9cfg]{#2}}{}%
  \ifstrequal{#1}{pmpaddr0}{\hyperref[sailRISCVregisterzpmpaddr0]{#2}}{}%
  \ifstrequal{#1}{pmpaddr1}{\hyperref[sailRISCVregisterzpmpaddr1]{#2}}{}%
  \ifstrequal{#1}{pmpaddr10}{\hyperref[sailRISCVregisterzpmpaddr10]{#2}}{}%
  \ifstrequal{#1}{pmpaddr11}{\hyperref[sailRISCVregisterzpmpaddr11]{#2}}{}%
  \ifstrequal{#1}{pmpaddr12}{\hyperref[sailRISCVregisterzpmpaddr12]{#2}}{}%
  \ifstrequal{#1}{pmpaddr13}{\hyperref[sailRISCVregisterzpmpaddr13]{#2}}{}%
  \ifstrequal{#1}{pmpaddr14}{\hyperref[sailRISCVregisterzpmpaddr14]{#2}}{}%
  \ifstrequal{#1}{pmpaddr15}{\hyperref[sailRISCVregisterzpmpaddr15]{#2}}{}%
  \ifstrequal{#1}{pmpaddr2}{\hyperref[sailRISCVregisterzpmpaddr2]{#2}}{}%
  \ifstrequal{#1}{pmpaddr3}{\hyperref[sailRISCVregisterzpmpaddr3]{#2}}{}%
  \ifstrequal{#1}{pmpaddr4}{\hyperref[sailRISCVregisterzpmpaddr4]{#2}}{}%
  \ifstrequal{#1}{pmpaddr5}{\hyperref[sailRISCVregisterzpmpaddr5]{#2}}{}%
  \ifstrequal{#1}{pmpaddr6}{\hyperref[sailRISCVregisterzpmpaddr6]{#2}}{}%
  \ifstrequal{#1}{pmpaddr7}{\hyperref[sailRISCVregisterzpmpaddr7]{#2}}{}%
  \ifstrequal{#1}{pmpaddr8}{\hyperref[sailRISCVregisterzpmpaddr8]{#2}}{}%
  \ifstrequal{#1}{pmpaddr9}{\hyperref[sailRISCVregisterzpmpaddr9]{#2}}{}%
  \ifstrequal{#1}{satp}{\hyperref[sailRISCVregisterzsatp]{#2}}{}%
  \ifstrequal{#1}{scause}{\hyperref[sailRISCVregisterzscause]{#2}}{}%
  \ifstrequal{#1}{sccsr}{\hyperref[sailRISCVregisterzsccsr]{#2}}{}%
  \ifstrequal{#1}{scounteren}{\hyperref[sailRISCVregisterzscounteren]{#2}}{}%
  \ifstrequal{#1}{sedeleg}{\hyperref[sailRISCVregisterzsedeleg]{#2}}{}%
  \ifstrequal{#1}{sepc}{\hyperref[sailRISCVregisterzsepc]{#2}}{}%
  \ifstrequal{#1}{sideleg}{\hyperref[sailRISCVregisterzsideleg]{#2}}{}%
  \ifstrequal{#1}{sscratch}{\hyperref[sailRISCVregisterzsscratch]{#2}}{}%
  \ifstrequal{#1}{stval}{\hyperref[sailRISCVregisterzstval]{#2}}{}%
  \ifstrequal{#1}{stvec}{\hyperref[sailRISCVregisterzstvec]{#2}}{}%
  \ifstrequal{#1}{tlb39}{\hyperref[sailRISCVregisterztlb39]{#2}}{}%
  \ifstrequal{#1}{tlb48}{\hyperref[sailRISCVregisterztlb48]{#2}}{}%
  \ifstrequal{#1}{tselect}{\hyperref[sailRISCVregisterztselect]{#2}}{}%
  \ifstrequal{#1}{ucause}{\hyperref[sailRISCVregisterzucause]{#2}}{}%
  \ifstrequal{#1}{uccsr}{\hyperref[sailRISCVregisterzuccsr]{#2}}{}%
  \ifstrequal{#1}{uepc}{\hyperref[sailRISCVregisterzuepc]{#2}}{}%
  \ifstrequal{#1}{uscratch}{\hyperref[sailRISCVregisterzuscratch]{#2}}{}%
  \ifstrequal{#1}{utval}{\hyperref[sailRISCVregisterzutval]{#2}}{}%
  \ifstrequal{#1}{utvec}{\hyperref[sailRISCVregisterzutvec]{#2}}{}%
  \ifstrequal{#1}{x1}{\hyperref[sailRISCVregisterzx1]{#2}}{}%
  \ifstrequal{#1}{x10}{\hyperref[sailRISCVregisterzx10]{#2}}{}%
  \ifstrequal{#1}{x11}{\hyperref[sailRISCVregisterzx11]{#2}}{}%
  \ifstrequal{#1}{x12}{\hyperref[sailRISCVregisterzx12]{#2}}{}%
  \ifstrequal{#1}{x13}{\hyperref[sailRISCVregisterzx13]{#2}}{}%
  \ifstrequal{#1}{x14}{\hyperref[sailRISCVregisterzx14]{#2}}{}%
  \ifstrequal{#1}{x15}{\hyperref[sailRISCVregisterzx15]{#2}}{}%
  \ifstrequal{#1}{x16}{\hyperref[sailRISCVregisterzx16]{#2}}{}%
  \ifstrequal{#1}{x17}{\hyperref[sailRISCVregisterzx17]{#2}}{}%
  \ifstrequal{#1}{x18}{\hyperref[sailRISCVregisterzx18]{#2}}{}%
  \ifstrequal{#1}{x19}{\hyperref[sailRISCVregisterzx19]{#2}}{}%
  \ifstrequal{#1}{x2}{\hyperref[sailRISCVregisterzx2]{#2}}{}%
  \ifstrequal{#1}{x20}{\hyperref[sailRISCVregisterzx20]{#2}}{}%
  \ifstrequal{#1}{x21}{\hyperref[sailRISCVregisterzx21]{#2}}{}%
  \ifstrequal{#1}{x22}{\hyperref[sailRISCVregisterzx22]{#2}}{}%
  \ifstrequal{#1}{x23}{\hyperref[sailRISCVregisterzx23]{#2}}{}%
  \ifstrequal{#1}{x24}{\hyperref[sailRISCVregisterzx24]{#2}}{}%
  \ifstrequal{#1}{x25}{\hyperref[sailRISCVregisterzx25]{#2}}{}%
  \ifstrequal{#1}{x26}{\hyperref[sailRISCVregisterzx26]{#2}}{}%
  \ifstrequal{#1}{x27}{\hyperref[sailRISCVregisterzx27]{#2}}{}%
  \ifstrequal{#1}{x28}{\hyperref[sailRISCVregisterzx28]{#2}}{}%
  \ifstrequal{#1}{x29}{\hyperref[sailRISCVregisterzx29]{#2}}{}%
  \ifstrequal{#1}{x3}{\hyperref[sailRISCVregisterzx3]{#2}}{}%
  \ifstrequal{#1}{x30}{\hyperref[sailRISCVregisterzx30]{#2}}{}%
  \ifstrequal{#1}{x31}{\hyperref[sailRISCVregisterzx31]{#2}}{}%
  \ifstrequal{#1}{x4}{\hyperref[sailRISCVregisterzx4]{#2}}{}%
  \ifstrequal{#1}{x5}{\hyperref[sailRISCVregisterzx5]{#2}}{}%
  \ifstrequal{#1}{x6}{\hyperref[sailRISCVregisterzx6]{#2}}{}%
  \ifstrequal{#1}{x7}{\hyperref[sailRISCVregisterzx7]{#2}}{}%
  \ifstrequal{#1}{x8}{\hyperref[sailRISCVregisterzx8]{#2}}{}%
  \ifstrequal{#1}{x9}{\hyperref[sailRISCVregisterzx9]{#2}}{}}

\newcommand{\sailRISCVoutcome}[1]{
  }

\newcommand{\sailRISCVrefoutcome}[2]{
  }

  \ea\newcommand\csname #1code\endcsname[1]{%
    \csname #1fcl##1execute\endcsname%
    \bigskip%
  }%
  \ea\WithSuffix\ea\newcommand\csname #1code\endcsname*[1]{%
    \csname #1fcl##1execute\endcsname%
  }%
  \ea\newcommand\csname #1valandfun\endcsname[1]{%
    \csname #1##1\endcsname \csname #1fn#1\endcsname%
  }%
}

% We could have sail macro call us back for more formatting flexibility.
%\newcommand{\saildescribe}[2]{
%  \lstinputlisting[language=sail]{#2}
%
%  \hangindent=\parindent #1
%}

% The following macros define how we would like sail code to be documented.
% There is one per category of sail top-level (val spec, typedef, function, function clause etc)
% currently we only use val and fcl.
% They are called by latex generated by sail with
% #1 the latex for any doc-comment from the sail
% #2 a lstinputlisting invocation that
\newcommand{\saildocval}[2]{%
#2%
\par%
\hangindent=\parindent #1%
\medskip%
}
\newcommand{\saildocfcl}[2]{%
#1 #2%
}
\newcommand{\saildocfn}[2]{%
#1 #2%
}
\newcommand{\saildoctype}[2]{%
#1 #2%
}

\newcommand{\@saildoclabelled@capture}[2]{%
  \global\def\@saildoclabelled@name{#1}%
  \global\def\@saildoclabelled@body{#2}%
}

\newcommand{\@saildocfcl@capture}[2]{%
  \global\def\@saildocfcl@doc{#1}%
  \global\def\@saildocfcl@fcl{#2}%
}

\newcommand{\@saildoc@makeerrcmd}[1]{%
  \ea\def\csname #1@error\endcsname{%
    \GenericError{[saildoc] }{Missing definition}{%
      [saildoc] \@backslashchar#1 should have been defined.\MessageBreak%
      Check your Sail version if you re-generated the LaTeX.%
    }{}%
  }%
}

\@saildoc@makeerrcmd{@saildoclabelled@name}
\@saildoc@makeerrcmd{@saildoclabelled@body}
\@saildoc@makeerrcmd{@saildocfcl@doc}
\@saildoc@makeerrcmd{@saildocfcl@fcl}

\newcommand{\@saildoc@makeforbiddenseccmd}[1]{%
  \ea\def\csname @saildoc@#1\ea\endcsname{%
    \GenericError{[saildoc] }{Forbidden command}{%
      [saildoc] \@backslashchar#1 is not allowed.%
    }{}%
  }%
}

% Always use starred variant
\newcommand{\@saildoc@makenestedseccmd}[2]{%
  \ea\def\csname @saildoc@#1\endcsname{%
    \@ifstar{}{}\csname #2\endcsname*%
  }%
}

\@saildoc@makeforbiddenseccmd{part}
\@saildoc@makeforbiddenseccmd{chapter}
\@saildoc@makeforbiddenseccmd{section}
% subsection is special (but still maps to subsubsection); see below
\@saildoc@makenestedseccmd{subsubsection}{paragraph}
\@saildoc@makenestedseccmd{paragraph}{subparagraph}
\@saildoc@makeforbiddenseccmd{subparagraph}

\let\@saildoc@subsection@allowed\@empty
\newcommand{\@saildoc@subsection@allow}[1]{%
  \ea\def\ea\@saildoc@subsection@allowed\ea{%
    \@saildoc@subsection@allowed%
    \@saildoc@subsection@allowed@iter{#1}%
  }%
}
\@saildoc@subsection@allow{Description}
\@saildoc@subsection@allow{Exceptions}
\@saildoc@subsection@allow{Notes}

\newcommand{\@saildoc@subsection@valid}[1]{}

\newcommand{\@saildoc@subsection@invalid}[1]{%
  \GenericError{[saildoc] }{Invalid subsection}{%
    [saildoc] `#1' is not a valid subsection.%
  }{}%
}

\newcommand{\@saildoc@subsection@duplicate}[1]{%
  \GenericError{[saildoc] }{Duplicate subsection}{%
    [saildoc] `#1' is defined more than once.%
  }{}%
}

\newcommand{\@saildoc@subsection@validate}[1]{{%
  \def\@saildoc@subsection@allowed@iter##1{%
    \ifthenelse{\equal{#1}{##1}}{%
      \let\@saildoc@subsection@validate@action\@saildoc@subsection@valid%
    }{%
    }%
  }%
  \ea\ifx\csname @saildoc@subsection@body@#1\endcsname\relax%
    \let\@saildoc@subsection@validate@action\@saildoc@subsection@invalid%
    \@saildoc@subsection@allowed%
  \else%
    \let\@saildoc@subsection@validate@action\@saildoc@subsection@duplicate%
  \fi%
  \@saildoc@subsection@validate@action{#1}%
}}

\NewEnviron{@saildoc@subsection}[1]{%
  \@saildoc@subsection@validate{#1}%
  \ea\ea\ea\global\ea\ea\ea\def\ea\csname @saildoc@subsection@body@#1\ea\endcsname\ea{\BODY}%
}

\newcommand{\@saildoc@subsection@print}[1]{%
  \ea\ifx\csname @saildoc@subsection@body@#1\endcsname\relax%
  \else%
    \ea\ifx\csname @saildoc@subsection@body@#1\endcsname\@empty%
    \else%
      \subsubsection*{#1}%
      \csname @saildoc@subsection@body@#1\endcsname%
    \fi%
  \fi%
}

\newcommand{\@saildoc@subsection@clear}{{%
  \def\@saildoc@subsection@allowed@iter##1{%
    \ea\global\ea\let\csname @saildoc@subsection@body@##1\endcsname\@undefined%
  }%
  \@saildoc@subsection@allowed%
}}

\newcommand{\@saildoc@xpatchcmd@repeat}[3]{%
  \xpatchcmd{#1}{#2}{#3}{\@saildoc@xpatchcmd@repeat{#1}{#2}{#3}}{}%
}

\newcommand{\@saildoc@environ@guard}[2]{%
  % See \makesailcmds for why this space is needed
  #1{#2} %
}

\newcommand{\@saildoc@textbf}[1]{%
  \ifcsname @capperm@\detokenize{#1}\endcsname%
    \csname @capperm@\detokenize{#1}\endcsname%
  \else%
    \textbf{#1}%
  \fi%
}

\newcommand{\makesailcmds}[2]{%
  \makesailcmds@core{#1}{#2}%
  \ea\newcommand\csname #1isarefbody\endcsname[1]{{%
    %
    % Given:
    %
    %   \saildoclabelled{foo}{\saildocfcl{bar}{baz}}
    %
    % we expand to capture foo, and expand the second argument again to capture
    % bar and baz.
    %
    \global\let\@saildoclabelled@name\@saildoclabelled@name@error%
    \global\let\@saildoclabelled@body\@saildoclabelled@body@error%
    \global\let\@saildocfcl@doc\@saildocfcl@doc@error%
    \global\let\@saildocfcl@fcl\@saildocfcl@fcl@error%
    %
    \let\saildoclabelled\@saildoclabelled@capture%
    \let\saildocfcl\@saildocfcl@capture%
    %
    \csname #1code\endcsname*{##1}%
    \@saildoclabelled@body%
    %
    \ea\ea\ea\ifx\ea\ea\ea\relax\ea\detokenize\ea{\@saildocfcl@doc}\relax%
      \GenericWarning{}{#1 Warning: `##1` is not documented}%
    \fi%
    %
    % Now for the fcl body, rewrite:
    %
    %   Foo
    %   \subsection*{Exceptions}
    %   Bar
    %   \subsection*{Notes}
    %   Baz
    %
    % to:
    %
    %   \@saildoc@environ@guard{\begin{@saildoc@subsection}}{Description}
    %   Foo
    %   \end{@saildoc@subsection}
    %   \@saildoc@environ@guard{\begin{@saildoc@subsection}}{Exceptions}
    %   Bar
    %   \end{@saildoc@subsection}
    %   \@saildoc@environ@guard{\begin{@saildoc@subsection}}{Notes}
    %   Baz
    %   \end{@saildoc@subsection}
    %
    % as well as using \@saildoc@subsubsection etc for all the other section
    % commands. We allow the non-starred \subsection too.
    %
    % The extra space inserted by \@saildoc@environ@guard is required to avoid:
    %
    %   \begin{@saildoc@subsection}{Foo}\end{@saildoc@subsection}
    %
    % as the lack of a token before \end confuses environ and makes it split
    % Foo into argument "F" and body "oo". The space gets stripped away so
    % \BODY will be empty.
    %
    \xpretocmd{\@saildocfcl@doc}{\@saildoc@environ@guard{\begin{@saildoc@subsection}}{Description}}{}{}%
    \@saildoc@xpatchcmd@repeat{\@saildocfcl@doc}{\part}{\@saildoc@part}%
    \@saildoc@xpatchcmd@repeat{\@saildocfcl@doc}{\chapter}{\@saildoc@chapter}%
    \@saildoc@xpatchcmd@repeat{\@saildocfcl@doc}{\section}{\@saildoc@section}%
    \@saildoc@xpatchcmd@repeat{\@saildocfcl@doc}{\subsection*}{\end{@saildoc@subsection}\@saildoc@environ@guard{\begin{@saildoc@subsection}}}%
    \@saildoc@xpatchcmd@repeat{\@saildocfcl@doc}{\subsection}{\end{@saildoc@subsection}\@saildoc@environ@guard{\begin{@saildoc@subsection}}}%
    \@saildoc@xpatchcmd@repeat{\@saildocfcl@doc}{\subsubsection}{\@saildoc@subsubsection}%
    \@saildoc@xpatchcmd@repeat{\@saildocfcl@doc}{\paragraph}{\@saildoc@paragraph}%
    \@saildoc@xpatchcmd@repeat{\@saildocfcl@doc}{\subparagraph}{\@saildoc@subparagraph}%
    \xapptocmd{\@saildocfcl@doc}{\end{@saildoc@subsection}}{}{}%
    %
    % We also want to format various special names in our own way, all of which
    % currently use \textbf in the saildoc output.
    %
    \@saildoc@xpatchcmd@repeat{\@saildocfcl@doc}{\textbf}{\@saildoc@textbf}%
    %
    % Now we have the right \begin and \end macros, with the latter directly
    % visible to environ without any expansion, we can capture their contents
    % by expanding again.
    %
    \@saildocfcl@doc%
    %
    % Finally reassemble the documentation in the right order with the Sail in
    % the right place. We use \csuse to avoid having to pre-initialise
    % everything to \@empty.
    %
    % Also add a label so that instruction references from saildoc resolve
    % correctly. This label is not added by the saildoc generator so we insert
    % it manually here using the sail mangling: <prefix>z<insnname>. This is
    % really a valspec mangling, which allows us to link to the description
    % rather than the function body and so saildoc's inability to reference
    % function clauses in markdown turns out to be useful.
    %
    \label{#1z##1}%
    \@saildoc@subsection@print{Description}%
    %
    \subsubsection*{Semantics}%
    \phantomsection%
    \label{\@saildoclabelled@name}%
    \noindent\@saildocfcl@fcl%
    %
    \@saildoc@subsection@print{Exceptions}%
    %
    \@saildoc@subsection@print{Notes}%
    %
    % Reset state for next time
    %
    \@saildoc@subsection@clear%
  }}%
}
\makeatother


\makesailcmds{sailRISCV}{sail_latex_riscv}

% Must be included later than setspace, otherwise all footnote hyperlinks
% point to the title page.
%   PS HACK
%\usepackage[hidelinks]{hyperref}
\ifdefined\trformat
% page labels end up off-by-2 in the tech-report so disable them
\usepackage[colorlinks,pdfpagelabels=false]{hyperref}
\else
\usepackage[colorlinks]{hyperref}
\fi
% Glossaries must be included after hyperref.
\usepackage[toc,nonumberlist]{glossaries}
\usepackage[nottoc]{tocbibind}
\usepackage[capitalise]{cleveref}
  \Crefname{appendix}{Appendix}{Appendices}
  \Crefname{figure}{Figure}{Figures}
\usepackage{footnote}
\usepackage{threeparttable}
\definecolor{CodeColour}{rgb}{0.9,0.9,0.9} %Light grey
\lstset{basicstyle=\small\ttfamily,
        stringstyle=\textit, %italic strings
        keywordstyle=\textbf, %Bold keywords
        commentstyle=,
        breaklines=true, % Wrap long lines
        numbers=left, % Line numbers on the left
        frame=l, %Border on the left
        framerule=0.8pt, % Thick border
        backgroundcolor=\color{CodeColour}, %Coloured code listings
        numberstyle={\small \oldstylenums},  %tiny, old style line numbers
		%stepnumber=5, % Number every fifth line
        numbersep=5pt, % Five points between the line numbers and the text
        tabsize=4
}
\lstdefinelanguage{llvm}
{
	morekeywords={private, constant, i8, i32, define, icmp, label, i64, call, void, ret, getelementptr, br, load, align, nounwind},
	morekeywords={addrspace, inttoptr, ptrtoint, tail},
	morecomment=[l];
}%

\lstdefinelanguage{sail}
  { morekeywords={val,function,cast,type,forall,foreach,from,to,overload,operator,enum,union,undefined,exit,and,assert,sizeof,
      scattered,register,inc,dec,if,then,else,effect,let,as,@,in,end,Type,Int,Order,match,clause,struct},
    morestring=[b]",
    stringstyle={\ttfamily\color{red}},
    showstringspaces=false,
    morecomment=[l][\itshape\color{DarkGreen}]{//},
    morecomment=[s][\itshape\color{DarkGreen}]{/*}{*/},
    deletestring=[bd]{'},
    escapechar=\#,
    emphstyle={\it},
    numbers=none,
    frame=none,
    backgroundcolor=\color{White},
    aboveskip=0em,
    belowskip=0em,
  }

\lstdefinelanguage{bluespec}
{ morekeywords={function,endfunction,for,struct,typedef,Integer,Bit,Bool,TSub,TAdd,return,if,method},
  morestring=[b]"'=’-<>,
  stringstyle={\ttfamily\color{red}},
  morecomment=[l][\itshape\color{DarkGreen}]{//},
  morecomment=[s][\itshape\color{DarkGreen}]{/*}{*/},
  emphstyle={\it},
}

\lstnewenvironment{ccodelisting}{\lstset{language=C}}{}
\lstnewenvironment{llvmlisting}{\lstset{language={llvm}}}{}
\newcommand{\ccode}[1]{\lstinline[backgroundcolor=\color{white},language=C]|#1|}
\newcommand{\llvmir}[1]{\lstinline[backgroundcolor=\color{white},language={llvm}]|#1|}
\newcommand{\asm}[1]{\lstinline[backgroundcolor=\color{white},language={}]|#1|}
\lstnewenvironment{asmcode}{\lstset{language=}}{}
\newcommand{\regname}[1]{{\small\ttfamily\$#1}}

\newcommand{\baselineboxformatting}[1]{%
  % Measure size of contents
  \sbox0{#1}%
  % Use the difference between the contents' height and the bitbox's height,
  % clamped to [-.44\baselineskip, 0], as our minimum depth.
  \setlength{\skip0}{\ht0 - \height}%
  \ifdim\skip0>0pt%
    \setlength{\skip0}{0}%
  \else%
    \ifdim\skip0<-.44\baselineskip%
      \setlength{\skip0}{-.44\baselineskip}%
    \fi%
  \fi%
  \centering\rule[\skip0]{0pt}{\height}#1%
}
\bytefieldsetup{boxformatting=\baselineboxformatting}

% Well this is gross, but it lets us align baselines between labels and
% bytefields in tabular environments... by pretending that the label is
% a "bytefield" of one bit of the right width, with no bounding lines.
\newcommand{\raiseforbf}[1]{%
  {\begin{bytefield}[bitwidth=\widthof{#1}]{1} \bitbox[]{1}{#1} \end{bytefield}}%
}

\makeatletter
\newdimen\rotateinbitbox@height
\newcommand{\rotateinbitbox}[1]{%
  \rotateinbitbox@height=\height%
  \rotatebox{90}{\makebox[\rotateinbitbox@height][c]{#1}}%
}
\makeatother

\hyphenation{CheriBSD}
\hyphenation{FreeBSD}
\hyphenation{CTSRD}
\hyphenation{CheriRTOS}

\reversemarginpar
\setlength{\marginparwidth}{1.2in}
\let\oldmarginpar\marginpar
\renewcommand\marginpar[1]{\-\oldmarginpar[\raggedright\footnotesize #1]%
{\raggedright\footnotesize #1}}

\newcommand{\pathname}[1]{\tt \small #1}
\newcommand{\literal}[1]{{\tt \small #1}}
\newcommand{\function}[1]{{\tt \small #1}}

% Register names
\newcommand{\reg}[1]{{\bf R#1}}		% MIPS register numbers
\newcommand{\creg}[1]{{\bf C#1}}	% Capability register numbers
\newcommand{\PC}{{\bf PC}}
\newcommand{\SP}{{\bf SP}}
\newcommand{\EPC}{{\bf EPC}}
\newcommand{\PCC}{{\bf PCC}}
\newcommand{\DDC}{{\bf DDC}}
\newcommand{\CNULL}{{\bf CNULL}}
\newcommand{\IDC}{{\bf IDC}}
\newcommand{\TSC}{{\bf TSC}}
\newcommand{\KRC}{{\bf KR1C}}
\newcommand{\KQC}{{\bf KR2C}}
\newcommand{\KCC}{{\bf KCC}}
\newcommand{\KDC}{{\bf KDC}}
\newcommand{\ErrorEPCC}{{\bf ErrorEPCC}}
\newcommand{\EPCC}{{\bf EPCC}}
\newcommand{\CULR}{{\bf CULR}}
\newcommand{\CPLR}{{\bf CPLR}}
\newcommand{\EXL}{{\bf EXL}}
\newcommand{\KSU}{{\bf KSU}}
\newcommand{\ErrorEPC}{{\bf ErrorEPC}}
\newcommand{\causereg}{{\bf cause}}
\newcommand{\capcausereg}{{\bf capcause}}

% RISC-V new register names
\newcommand{\UTCC}{{\bf UTCC}}
\newcommand{\UTDC}{{\bf UTDC}}
\newcommand{\UScratchC}{{\bf UScratchC}}
\newcommand{\UEPCC}{{\bf UEPCC}}
\newcommand{\STCC}{{\bf STCC}}
\newcommand{\STDC}{{\bf STDC}}
\newcommand{\SScratchC}{{\bf SScratchC}}
\newcommand{\SEPCC}{{\bf SEPCC}}
\newcommand{\MTCC}{{\bf MTCC}}
\newcommand{\MTDC}{{\bf MTDC}}
\newcommand{\MScratchC}{{\bf MScratchC}}
\newcommand{\MEPCC}{{\bf MEPCC}}
\newcommand{\xTCC}{{\bf {\it x}TCC}}
\newcommand{\xTDC}{{\bf {\it x}TDC}}
\newcommand{\xScratchC}{{\bf {\it x}ScratchC}}
\newcommand{\xEPCC}{{\bf {\it x}EPCC}}
\newcommand{\xccsr}{\texttt{{\it x}ccsr}}
\newcommand{\mccsr}{\texttt{mccsr}}
\newcommand{\sccsr}{\texttt{sccsr}}
\newcommand{\uccsr}{\texttt{uccsr}}
% RISC-V existing registers
\newcommand{\xtdc}{\texttt{{\it x}tdc}}
\newcommand{\xbadaddr}{\texttt{{\it x}badaddr}}
\newcommand{\xtval}{\texttt{{\it x}tval}}
\newcommand{\xtvec}{\texttt{{\it x}tvec}}
\newcommand{\mtvec}{\texttt{mtvec}}
\newcommand{\stvec}{\texttt{stvec}}
\newcommand{\utvec}{\texttt{utvec}}
\newcommand{\xepc}{\texttt{{\it x}epc}}
\newcommand{\mepc}{\texttt{mepc}}
\newcommand{\sepc}{\texttt{sepc}}
\newcommand{\uepc}{\texttt{uepc}}
\newcommand{\xcause}{\texttt{{\it x}cause}}
\newcommand{\mcause}{\texttt{mcause}}
\newcommand{\scause}{\texttt{scause}}
\newcommand{\ucause}{\texttt{ucause}}
\newcommand{\xscratch}{\texttt{{\it x}scratch}}
\newcommand{\mscratch}{\texttt{mscratch}}
\newcommand{\sscratch}{\texttt{sscratch}}
\newcommand{\uscratch}{\texttt{uscratch}}
\newcommand{\menvcfg}{\texttt{menvcfg}}
\newcommand{\senvcfg}{\texttt{senvcfg}}
\newcommand{\xRET}{\insnnoref{{\it x}RET}}

\newcommand{\AL}{{\bf AL}}
\newcommand{\AX}{{\bf AX}}
\newcommand{\DS}{{\bf DS}}
\newcommand{\ES}{{\bf ES}}
\newcommand{\FS}{{\bf FS}}
\newcommand{\GS}{{\bf GS}}
% \SS is an existing LaTeX command
\newcommand{\SSreg}{{\bf SS}}
\newcommand{\EAX}{{\bf EAX}}
\newcommand{\CAX}{{\bf CAX}}
\newcommand{\CBP}{{\bf CBP}}
\newcommand{\CBX}{{\bf CBX}}
\newcommand{\CCX}{{\bf CCX}}
\newcommand{\CDX}{{\bf CDX}}
\newcommand{\CFS}{{\bf CFS}}
\newcommand{\CGS}{{\bf CGS}}
\newcommand{\CDI}{{\bf CDI}}
\newcommand{\CIP}{{\bf CIP}}
\newcommand{\KGS}{{\bf KGS}}
\newcommand{\CRTWO}{{\bf CR2}}
\newcommand{\CRFOUR}{{\bf CR4}}
\newcommand{\CRFIVE}{{\bf CR5}}
\newcommand{\CRTWELVE}{{\bf CR12}}
\newcommand{\CS}{{\bf CS}}
\newcommand{\CSI}{{\bf CSI}}
\newcommand{\CSP}{{\bf CSP}}
\newcommand{\IDT}{{\bf IDT}}
\newcommand{\IST}{{\bf IST}}
\newcommand{\KSC}{{\bf KSC}}
\newcommand{\RAX}{{\bf RAX}}
\newcommand{\RBP}{{\bf RBP}}
\newcommand{\RBX}{{\bf RBX}}
\newcommand{\RCX}{{\bf RCX}}
\newcommand{\RDI}{{\bf RDI}}
\newcommand{\REX}{{\bf REX}}
\newcommand{\RIP}{{\bf RIP}}
\newcommand{\RSI}{{\bf RSI}}
\newcommand{\RSP}{{\bf RSP}}
\newcommand{\RFLAGS}{{\bf RFLAGS}}
\newcommand{\TSS}{{\bf TSS}}
\newcommand{\CSTAR}{{\bf CSTAR}}
\newcommand{\STAR}{{\bf IA32\_STAR}}
\newcommand{\LSTAR}{{\bf IA32\_LSTAR}}
\newcommand{\FSBASE}{{\bf IA32\_FS\_BASE}}
\newcommand{\GSBASE}{{\bf IA32\_GS\_BASE}}
\newcommand{\KGSBASE}{{\bf IA32\_KERNEL\_GS\_BASE}}
\newcommand{\VEX}{{\bf VEX}}
\newcommand{\EVEX}{{\bf EVEX}}

% Capability register fields
\newcommand{\ctag}{{\bf tag}}
\newcommand{\csealed}{{\bf s}}
\newcommand{\cperms}{{\bf perms}}
\newcommand{\cuperms}{{\bf uperms}}
\newcommand{\cflags}{{\bf flags}}
\newcommand{\cotype}{{\bf otype}}
\newcommand{\ccursor}{{\bf cursor}}
\newcommand{\cbase}{{\bf base}}
\newcommand{\clength}{{\bf length}}
\newcommand{\coffset}{{\bf offset}}
\newcommand{\cbound}{{\bf top}}

%  CHERI-128 v1 capability fields
\newcommand{\ctobase}{{\bf toBase}}
\newcommand{\ctobound}{{\bf toBound}}
\newcommand{\cformat}{{\bf FT}}
\newcommand{\cexponent}{{\bf e}}
\newcommand{\csign}{{\bf SN}}

%  CHERI-128 v1 capability fields
\newcommand{\cbasebits}{{\bf baseBits}}
\newcommand{\ctopbits}{{\bf topBits}}
\newcommand{\ccarries}{{\bf C}}

% CHERI-128 candidate 3 fields
\newcommand{\ctop}{{\bf top}}
%\newcommand{\rbase}{\textbf{base\textsubscript{req}}}
\newcommand{\rbase}{\textbf{base\_req}}
\newcommand{\cbasecorrection}{\textbf{c\textsubscript{b}}}
%\newcommand{\cbasecorrection}{\textbf{c\_b}}
%\newcommand{\rlength}{\textbf{length\textsubscript{req}}}
\newcommand{\rlength}{\textbf{rlength}}
\newcommand{\ctopcorrection}{\textbf{c\textsubscript{t}}}
%\newcommand{\ctopcorrection}{\textbf{ctop}} - SWM: why use this version?
\newcommand{\cB}{{\bf B}}
\newcommand{\cT}{{\bf T}}
\newcommand{\caddr}{{\bf a}}

% Architectural parameters
\newcommand{\xlen}{{\texttt{XLEN}}}
\newcommand{\clen}{{\texttt{CLEN}}}

% Field used in several compression formats
\newcommand{\cmuperms}{$\boldsymbol{\mu}\textbf{perms}$}

% Stylized permission bit.  Starred form omits Permit\_ prefix for informal
% references.
\NewDocumentCommand{\capperm}{sm}{\textsc{\small\IfBooleanTF{#1}{}{Permit\_}#2}\xspace}
% Permission bit convenience macros
% We define a short form for normal use and a long @-command form for internal
% use by saildoc.
\makeatletter
\NewDocumentCommand{\makecapperm}{smm}{%
  \def\@make@capperm##1##2{%
    \ea\NewDocumentCommand\csname @capperm@\detokenize{##2#3}\endcsname{s}{%
      \IfBooleanTF{####1}{\capperm*}{\capperm##1}{#3}%
    }%
    \ea\ea\ea\let\ea\csname capperm#2\ea\endcsname%
      \csname @capperm@\detokenize{##2#3}\endcsname%
  }%
  \IfBooleanTF{#1}{\@make@capperm{*}{}}{\@make@capperm{}{Permit\_}}%
  \let\@make@capperm\undefined%
}
\makeatother
\makecapperm{ASR}{Access\_System\_Registers}
\makecapperm{Cid}{Set\_CID}
\makecapperm{Invoke}{Invoke}
\makecapperm{L}{Load}
\makecapperm{LC}{Load\_Capability}
\makecapperm{S}{Store}
\makecapperm{Seal}{Seal}
\makecapperm{SC}{Store\_Capability}
\makecapperm{SLC}{Store\_Local\_Capability}
\makecapperm{Unseal}{Unseal}
\makecapperm{X}{Execute}
% No Permit_, so always use starred form even if not given
\makecapperm*{G}{Global}
\makeatletter
% Legacy until Sail is updated for Perm_Invoke
\ea\ea\ea\let\csname @capperm@\detokenize{Permit\_CInvoke}\ea\endcsname\csname @capperm@\detokenize{Permit\_Invoke}\endcsname
\makeatother

\makeatletter
\newcommand{\@insnlabelname}[2]{insn:#1:#2}
\newcommand{\@insnlabel}[2]{\label{\@insnlabelname{#1}{#2}}}

% If no optional argument is passed (internally, if an empty second argument is
% passed), the lowercased text is the label reference
\newcommand{\@insnrefnofont}[3]{{%
  \if\relax\detokenize{#2}\relax% optional arg not passed
    % NB: \lowercase is not expandable so is outside the \def.
    \lowercase{\def\@insnrefnofont@insnname{#3}}%
  \else% optional arg passed
    \def\@insnrefnofont@insnname{#2}%
  \fi%
  \hyperref[\@insnlabelname{#1}{\@insnrefnofont@insnname}]{#3}%
}}

\NewDocumentCommand{\@insnfmt}{sm}{%
  \IfBooleanTF{#1}{#2}{{\tt \small #2}}%
}

\NewDocumentCommand{\@insnref}{smmm}{%
  \IfBooleanTF{#1}{\@insnfmt*}{\@insnfmt}{\@insnrefnofont{#2}{#3}{#4}}%
}

\newcommand{\@makeinsncmds@explicit}[2]{%
  \ea\newcommand\csname insn#1labelname\endcsname[1]{\@insnlabelname{#2}{##1}}%
  \ea\newcommand\csname insn#1label\endcsname[1]{\@insnlabel{#2}{##1}}%
  \ea\NewDocumentCommand\csname insn#1ref\endcsname{sO{}m}{%
    \IfBooleanTF{##1}{\@insnref*}{\@insnref}{#2}{##2}{##3}%
  }%
}
\newcommand{\@makeinsncmds}[1]{\@makeinsncmds@explicit{#1}{#1}}

\@makeinsncmds{riscv}
% Cannot use more intuitive x86 in command names
\@makeinsncmds{xes}

\newcommand{\definsnarch}[1]{\def\@definsnarch{#1}}
\@makeinsncmds@explicit{}{\@definsnarch}

\let\insnnoref\@insnfmt
\makeatother

% Default is currently RISC-V
\definsnarch{riscv}

\newcommand{\cherithreeop}[5][NOHEADER]{
\begin{bytefield}{32}
	\ifthenelse{\equal{#1}{NOHEADER}}{}
	{\bitheader[endianness=big]{0,5,6,10,11,15,16,20,21,25,26,31}}\\
	\bitbox{6}{{\color{Grey}0x12}}
	\bitbox{5}{0x0}
	\bitbox{5}{#3}
	\bitbox{5}{#4}
	\bitbox{5}{#5}
	\bitbox{6}{#2}
\end{bytefield}%
}
\newcommand{\cheritwoop}[4][NOHEADER]{\cherithreeop[#1]{{\color{Grey}0x3f}}{#3}{#4}{#2}}
\newcommand{\cherioneop}[3][NOHEADER]{\cheritwoop[#1]{{\color{Grey}0x1f}}{#3}{#2}}


\newcommand{\usesDDCinsteadofNULL}[1]{%
\paragraph{Note:}
If the encoded value of \emph{#1} is zero, this instruction will use
\DDC{} as the \emph{#1} operand
}

% When specifying instructions in pseudocode:
\newcommand{\algorithmicnot}{\textbf{not}}
\newcommand{\algorithmicand}{\textbf{and}}
\newcommand{\algorithmicor}{\textbf{or}}
\newcommand{\algorithmictrue}{\textbf{true}}
\newcommand{\algorithmicfalse}{\textbf{false}}
\newcommand{\algorithmicwith}{\textbf{with}}

% Markdown and Sail's LaTeX backend don't do well with literal < and >, so add
% \lt and \gt macros like \le and \ge.
\let\lt<
\let\gt>

\makeatletter
\newcount\@autogrid@col
\newcount\@autogrid@cols
\def\@autogrid@cr{%
  \global\advance\@autogrid@col 1\relax%
  \ifnum\@autogrid@col=\@autogrid@cols%
    \def\@autogrid@cr@body{\cr}%
    \global\@autogrid@col=0\relax%
  \else%
    \def\@autogrid@cr@body{&}%
  \fi%
  \@autogrid@cr@body%
  \let\\\@autogrid@cr%
}
\newenvironment{autogrid}[1]{%
  \let\@autogrid@format\@empty%
  \@autogrid@cols=\numexpr(#1)\relax%
  \@autogrid@col=0\relax%
  \loop\ifnum\@autogrid@col<\@autogrid@cols%
    \ea\def\ea\@autogrid@format\ea{\@autogrid@format l}%
    \advance\@autogrid@col 1\relax%
  \repeat%
  \@autogrid@col=0\relax%
  \def\@autogrid@begintabular{\begin{tabular}}%
  \ea\@autogrid@begintabular\ea{\@autogrid@format}%
  \let\\\@autogrid@cr%
}{%
  \end{tabular}%
}
\makeatother


\makeatletter\@ifclassloaded{standalone}{%
% No need for glossary or bibliography when building tikz figures
}{% else
\renewcommand{\glossarypreamble}{\label{glossary}}
\makeglossaries

\newglossaryentry{abstract capability}
{
  name=abstract capability,
  description={
%     Abstract capabilities maintain the appearance of capability
%     lifespan across operations that violate architectural \gls{capability
%     provenance}.
%     For example, abstract capabilities remain valid despite an OS kernel
%     swapping them to and from disk, which requires that any architectural
%     \gls{capability} in the swapped memory have its \gls{capability tag}
%     restored through re-derivation
    Abstract capabilities are a conceptual abstraction that overlays the
    concrete capabilities of the architecture to describe the intended
    maintenance of capability lifespan across operations that violate
    architectural \gls{capability provenance}.
    For example, if an OS kernel
    swaps a page containing a capability to and from disk, 
    it will have to have its \gls{capability tag}
    restored through re-derivation, so there is no longer an
    architectural provenance relationship between the two, but for
    application-level reasoning it is sometimes useful to regard there
    to be one}
}

\newglossaryentry{address}
{
  name=address,
  description={An integer address suitable for dereference within an address
    space.
    In \gls{CHERI-MIPS}, \glspl{capability} are always interpreted in terms of
    \glspl{virtual address}.
    In \gls{CHERI-RISC-V}, \glspl{capability} may be interpreted as
    \glspl{virtual address} -- or \glspl{physical address} when operating in
    Machine Mode}
}

\longnewglossaryentry{capability}
{
  name=capability,
  plural=capabilities,
}
{
  A capability contains an \gls{address}, \gls{capability bounds}
  describing a range of bytes within which addresses may be
  \glslink{dereference}{dereferenced}, \gls{capability permissions}
  controlling the forms of dereference that may be permitted (e.g., load or
  store), a \gls{capability tag} protecting \gls{capability validity}
  (integrity and \gls{capability provenance}), and a \gls{capability object type}
  indicating whether it is a \gls{sealed capability}
  (and, if so, under which \gls{capability object type} they are sealed)
  or \gls{unsealed capability}.
  The address embedded within a capability may be a \gls{virtual address} or
  a \glspl{physical address} depending on the current addressing mode; when
  used to authorize (un)sealing, the address is instead a
  \gls{capability object type}.

  In CHERI, capabilities are used to implement \glspl{pointer} with additional
  protections in aid of \gls{fine-grained memory protection},
  \gls{control-flow robustness}, and other higher-level protection models such
  as \gls{software compartmentalization}.
  Unlike a \gls{fat pointer}, capabilities are subject to
  \gls{capability provenance}, ensuring that they are derived from a prior
  valid capability only via valid manipulations, and \gls{capability
  monotonicity}, which ensures that manipulation can lead only to
  non-increasing rights.
  CHERI capabilities provide strong compatibility with C-language pointers and
  Memory Management Unit (MMU)-based system-software designs, by virtue of
  its \gls{hybrid capability model}.

  Architecturally, a capability can be viewed as an \gls{address} equal to the
  sum of the \gls{capability base} and \gls{capability offset}, as well as
  associated metadata.
%\psnote{Perhaps this base/offset view should now be de-emphasised?  It's
%arguably  implementation detail in any case}
  Dereferencing a capability is done relative to that address.
%  The implementation may choose to store the pre-computed address
%  combining the base and offset, to avoid an implied addition on each memory
%  access, and to similarly store the base and length as pre-computed
%  addresses.
  The size of an in-memory capability may be smaller than the sum of its
  architectural fields (such as base, offset, and permissions) if a
  \gls{compressed capability} mechanism, such as \gls{CHERI Concentrate}, is
  used.

  In the ISA, capabilities may be used explicitly via \gls{capability-based
  instructions}, an application of the \gls{principle of intentional use},
  but also implicitly using \glslink{legacy instructions}{legacy load
  and store instructions} via the \gls{default data capability}, and
  instruction fetch via the \gls{program-counter capability}.
  A capability is either sealed or unsealed, controlling whether it has
  software-defined or instruction-set-defined behavior, and whether or not its
  fields are immutable.

  Capabilities may be held in a \gls{capability register} in a dedicated
  \gls{capability register file}, a \gls{merged register file}, or a
  suitably aligned \gls{tagged memory}.
}

\newglossaryentry{capability base}
{
  name=capability base,
  description={The lower of the two \gls{capability bounds}, from which
    the \gls{address} of a \gls{capability} can be calculated by using
    the \gls{capability offset}}
}

\newglossaryentry{capability bounds}
{
  name=capability bounds,
  description={Upper and lower bounds, associated with each
    \gls{capability}, describing a range of \glspl{address} that may
    be \glslink{dereference}{dereferenced} via the capability.
    Architecturally, bounds are with respect to the \gls{capability base},
    which provides the lower bound, and \gls{capability length}, which
    provides the upper bound when added to the base.
    The bounds may be empty, connoting no right to dereference at any
    address.
    The address of a capability may float outside of the
    dereferenceable bounds; with a \gls{compressed capability}, it may not
    be possible to represent all possible \glslink{out of
      bounds}{out-of-bounds} addresses.
    Bounds may be manipulated subject to \gls{capability monotonicity}
    using \gls{capability-based instructions}}
}

\newglossaryentry{capability length}
{
  name=capability length,
  description={The distance between the lower and upper \gls{capability
    bounds}}
}

\newglossaryentry{capability monotonicity}
{
  name=capability monotonicity,
  description={Capability monotonicity is a property of the instruction set
    that any requested manipulation of a \gls{capability}, whether in a
    \gls{capability register} or in memory, either leads to strictly
    non-increasing rights, clearing of the \gls{capability tag}, or a
    hardware exception.    
    \knnote{I presume that the ``rights'' of a capability are
    determined by its permissions and its bounds, but not by its
    sealedness. In other words, increasing the permissions or bounds
    of a capability would increase its rights, but unsealing a
    capability would not increase its right. If this is correct,
    perhaps it could help to explicitly state this here.}    
    Controlled violation of monotonicity can be achieved via the exception
    delivery mechanism, which grants rights to additional capability
    register, and also by the \gls{CInvoke} instruction, which may
    unseal (and jump to) suitably checked \glspl{sealed
    capability}.
    \knnote{The exception delivery mechanism and the CCall instruction
    do not violate monotonicity, since they do not increase the rights of
    any capability. They do violate the monotonicity of the set of
    reachable rights (see \ref{sec:model-monotonicity}), because an
    exception makes a capability reachable that might not have been
    reachable before (namely the KCC) and the CCall instruction
    unseals capabilities without needing a capability that has the
    authority to unseal them. Perhaps it would be worth creating a
    glossary entry for the set of reachable rights, and mention that
    this set is monotonic as long as no exceptions are raised or
    CCalls are executed.}}
}

\newglossaryentry{capability object type}
{
  name=capability object type,
  description={In addition to \glslink{fat pointer}{fat-pointer} metadata such
    as \gls{capability bounds} and \gls{capability permissions}, \glspl{capability} also contain an integer object type.
    The object type space is partitioned into a range of non-reserved and
    \gls{reserved capability object type} types.
    The \glspl{reserved capability object type} are hardware-interpreted and
    include \glspl{unsealed capability} or \glspl{sealed entry capability}.
    If the object type is one of the non-\glspl{reserved capability object type},
    the capability is a \gls{sealed capability with an object type}.
    For \glspl{sealed capability with an object type}, the object type is set during a
    sealing operation to the \gls{address} of the \gls{sealing capability}.
    Object types can be used to link a sealed \gls{code capability} and a
    sealed \gls{data capability} when used with \gls{CInvoke} to implement a
    software object model or to implement software-defined tokens of authority}
}

\newglossaryentry{capability offset}
{
  name=capability offset,
  description={The distance between \gls{capability base} and the
    \gls{address} accessed when the \gls{capability} is used as a \gls{pointer}}
}

\newglossaryentry{capability permissions}
{
  name=capability permissions,
  description={A bitmask, associated with each \gls{capability},
    describing a set of ISA- or software-defined operations that may be
    performed via the capability.
    ISA-defined permissions include load data, store data, instruction fetch,
    load capability, and store capability.
    Permissions may be manipulated subject to \gls{capability monotonicity}
    using \gls{capability-based instructions}}
}

\newglossaryentry{capability provenance}
{
  name=capability provenance,
  description={
% The property that, following manipulation, a \gls{capability}
%     remains valid for use only if it is derived from another valid capability
%     using a valid capability operation.
%     Provenance is implemented using a \gls{capability tag} combined with
%     \gls{capability monotonicity}, and will be preserved whether a
%     capability is held in a \gls{capability register} or \gls{tagged memory},
%     subject to suitable use of \gls{capability-based instructions}
The property that a valid-for-use \gls{capability} can only be
    constructed by deriving it from another valid capability
    using a valid capability operation.
% PS: not totally clear what a ``valid capability operation'' is.  An
% execution of a capability instruction that doesn't raise an exception?
    Provenance is implemented using a \gls{capability tag} combined with
    \gls{capability monotonicity}, 
% PS: the text (both previous version and mine) defines provenance as
% a property of the architecture, not ``the provenance of a
% capability'' as the source capability or derivation chain, so we
% can't say ``will be preserved'' like the text did. 
% Maybe we should define that more explicit notion of provenance, and
% replace this glossary entry with one for ``capability provenance
% preservation'', but I've not for now. 
irrespective of 
%
whether a
    capability is held in a \gls{capability register} or \gls{tagged memory}}
% PS: surely ``capability provenance'' should hold universally, not only
%    ``subject to suitable use of \gls{capability-based instructions''?
}

\newglossaryentry{capability register}
{
  name=capability register,
  description={A capability register is an architectural register able to hold
    a \gls{capability} including its \gls{capability tag}, \gls{address},
    other \glslink{fat pointer}{fat-pointer} metadata such as
    its \gls{capability bounds} and \gls{capability permissions}, and optional
    \gls{capability object type}.
    Capability registers may be held in a \gls{capability register file}, a
    \gls{merged register file}, or be a \gls{special capability register}
    accessed by dedicated instructions.
    A capability register might be a dedicated register intended primarily for
    capability-related operations (e.g., the capability registers described
    in \gls{CHERI-MIPS}), or a general-purpose integer
    register that has been extended with capability metadata (such as the
    \gls{program-counter capability}, or the capability registers described in
    \gls{CHERI-RISC-V} when using a merged register file).
    Capability registers must be used to retain tag bits on capabilities
    transiting through memory, as only \gls{capability-based instructions}
    enforce \gls{capability provenance} and \gls{capability monotonicity}}
}

\newglossaryentry{capability register file}
{
  name=capability register file,
  description={The capability register file is a register file dedicated to
    holding general-purpose \glspl{capability}, in contrast to a \gls{merged
    register file}, in which general-purpose integer registers are extended to
    be able to hold tagged capabilities.
    Some general-purpose capability registers have well-known conventions for
    their use in software, including the \gls{return capability} and the
    \gls{stack capability}}
}

\newglossaryentry{capability tag}
{
  name=capability tag,
  description={A capability tag is a 1-bit integrity tag associated with each
    \gls{capability register}, and also with each capability-sized,
    capability-aligned location in memory.
    If the tag is set, the \gls{capability} is valid and can be
    \glslink{dereference}{dereferenced} via the ISA.
    If the tag is clear, then the capability is invalid and cannot be
    dereferenced via the ISA.
    Tags are preserved 
by ISA
operations that conform to \gls{capability
    provenance} and \gls{capability monotonicity} rules -- for example,
    that any attempted modification of \gls{capability bounds} leads to
    non-increasing bounds,
%was ``writes'', not ``bounds'' - presume just a typo?
 and that in-memory capabilities are written only
    via capability stores, not data stores -- otherwise, tags are cleared}
%
%    Subject to these constraints, tags will be preserved by
%    \gls{capability-based instructions}
}

\newglossaryentry{capability validity}
{
  name=capability validity,
  description={A \gls{capability} is valid if its \gls{capability tag}
    is set, which permits use of the capability subject to its
    \gls{capability bounds}, \gls{capability permissions}, and so on.
    Attempts to \gls{dereference} a capability without a tag set will lead
    to a hardware exception}
}

\newglossaryentry{capability-based instructions}
{
  name=capability-based instructions,
  description={These instructions accept capabilities as operands, allowing
    capabilities to be loaded from and stored memory, manipulated subject to
    \gls{capability provenance} and \gls{capability monotonicity} rules,
    and used for a variety of operations such as loading and storing data and
    capabilities, as branch targets, and to retrieve and manipulate capability
    fields -- subject to \gls{capability permissions}}
}

\longnewglossaryentry{CInvoke}
{
  name=CInvoke
}
{
  The \insnref{CInvoke} instruction is a source of controlled
  non-monotonicity in the \gls{CHERI-MIPS} and \gls{CHERI-RISC-V} ISAs.
\psnote{See Kyndylan's note for capability monotonicity}
  It can directly enter any userspace domain described by a pair
  of sealed capabilities with the \emph{Permit\_CInvoke} permission set.
  In particular, it can
  safely enter userspace domain-transition code
  described by the sealed \gls{code capability} while also unsealing
  the sealed \gls{data capability}.
  The sealed operand \glspl{capability register}
  are checked for suitable properties and correspondence, and the userspace
  domain-transition routine can store any return information, perform further error
  checking, and so on.
}

\newglossaryentry{CHERI Concentrate}
{
  name=CHERI Concentrate,
  description={CHERI Concentrate is a specific \gls{compressed capability}
    format that represents a 64-bit \gls{address} with full precision, and
    \gls{capability bounds} relative to that address with reduced precision.
    Bounds have a floating-point representation, requiring that as the size of
    a bounded object increases, greater alignment of its \gls{capability base}
    and \gls{capability length} are required.
    CHERI Concentrate is the successor compression format to \gls{CHERI-128}}
}

\newglossaryentry{CHERI-128}
{
  name=CHERI-128,
  description={CHERI-128 is a specific \gls{compressed capability} format that
    represents a 64-bit \gls{address} with full precision, and
    \gls{capability bounds} relative to that address with reduced precision.
    Bounds have a floating-point representation, requiring that as the size of
    a bounded object increases, greater alignment of its \gls{capability base}
    and \gls{capability length} are required.
    CHERI-128 has been replaced with \gls{CHERI Concentrate}}
}

\newglossaryentry{CHERI-MIPS}
{
  name=CHERI-MIPS,
  description={An application of the CHERI protection model to the 64-bit MIPS
    ISA}
}

\newglossaryentry{CHERI-RISC-V}
{
  name=CHERI-RISC-V,
  description={An application of the CHERI protection model to the RISC-V ISA}
}

\newglossaryentry{CHERI-x86-64}
{
  name=CHERI-x86-64,
  description={An application of the CHERI protection model to the x86-64 ISA}
}

\newglossaryentry{code capability}
{
  name=code capability,
  plural=code capabilities,
  description={A \gls{capability} whose \gls{capability permissions} have been
    configured to permit instruction fetch (i.e., execute) rights; typically,
    write permission will not be granted via an executable capability, in
    contrast to a \gls{data capability}.
    Code capabilities are used to implement \gls{control-flow robustness} by
    constraining the available branch and jump targets}
}

\newglossaryentry{compressed capability}
{
  name=compressed capability,
  plural=compressed capabilities,
  description={A \gls{capability} whose \gls{capability bounds} are
    compressed with respect to its \gls{address}, allowing its
    in-memory footprint to be reduced -- e.g., to 128 bits, rather than the
    roughly
    architectural 256 bits visible to the instruction set when a capability
    is loaded into a register file.
    Certain architecturally valid \glslink{out of bounds}{out-of-bounds}
    addresses may not be \glslink{representable
    capability}{representable} with capability compression; operations leading
     to \glslink{unrepresentable capability}{unrepresentable capabilities}
    will clear the \gls{capability tag} or throw an exception in order to
    ensure continuing \gls{capability monotonicity}.
    \gls{CHERI-128} and \gls{CHERI Concentrate} are specific compressed
    capability models that select particular points in the tradeoff space
    around in-memory capability size, bounds alignment requirements, and
    representability}
}

\newglossaryentry{control-flow robustness}
{
  name=control-flow robustness,
  description={The use of \glspl{code capability} to constrain the set of
    available branch and jump targets for executing code, such that the
    potential for attacker manipulation of the \gls{program-counter
    capability} to simulate injection of arbitrary code is severely
    constrained; a form of \gls{vulnerability mitigation} implemented via
    the \gls{principle of least privilege}}
}

\newglossaryentry{data capability}
{
  name=data capability,
  plural=data capabilities,
  description={A \gls{capability} whose \gls{capability permissions} have been
    configured to permit data load and store, but not instruction fetch (i.e.,
    execute) rights; in contrast to a \gls{code capability}}
}

\newglossaryentry{default data capability}
{
  name=default data capability (\DDC{}),
  description={A \gls{special capability register} constraining
    \glslink{legacy instructions}{legacy} non-\gls{capability-based
    instructions} that load and store data without awareness of the capability
    model.
    Any attempts to load and store will be relocated relative to the default
    data capability's \gls{capability base} and \gls{capability offset}, and
    controlled by its \gls{capability bounds} and \gls{capability
    permissions}.
    Use of the default data capability  violates the \gls{principle of
    intentional use}, but permits compatibility with legacy software.
    A suitably configured default data capability will prevent the use of
    non-capability-based load and store instructions}
}

\newglossaryentry{dereference}
{
  name=dereference,
  description={Dereferencing a \gls{address} means that it is the
    target address for a load, store, or instruction fetch.
    A \gls{capability} may be dereferenced only subject to it being valid
    -- i.e., that its \gls{capability tag} is present --  and is also subject
    to appropriate checks of its \gls{capability bounds}, \gls{capability permissions}, and
    so on.
    Dereference may occur as a result of explicit use of a capability via
    \gls{capability-based instructions}, or implicitly as a result of the
    \gls{program-counter capability} or \gls{default data capability}}
}

\newglossaryentry{exception program-counter capability}
{
  name=exception program-counter capability (\EPCC{}),
  description={A \gls{special capability register} into which the running
    \gls{program-counter capability} will be moved into on an exception, and
    whose value will be moved back into the program-counter capability on
    exception return}
}

\newglossaryentry{fat pointer}
{
  name=fat pointer,
  description={A \gls{pointer} (\gls{address}) that has been extended
    with additional metadata such as \gls{capability bounds} and
    \gls{capability permissions}.
    In conventional fat-pointer designs, fat pointers to not have a notion of
    sealing (i.g., as in \glspl{sealed capability} and \glspl{unsealed
    capability}), nor rules implementing \gls{capability provenance} and
    \gls{capability monotonicity}}
}

\newglossaryentry{fine-grained memory protection}
{
  name=fine-grained memory protection,
  description={The granular description of available code and data in which
    \gls{capability bounds} and \gls{capability permissions} are made as
    small as possible, in order to limit the potential effects of software
    bugs and vulnerabilities.
    This approach applies both to \glspl{code capability} and \glspl{data
    capability}, offering effective \gls{vulnerability mitigation} via
    techniques such as \gls{control-flow robustness}, as well as supporting
    higher-level mitigation techniques such as \gls{software
    compartmentalization}.
    Fine-grained memory protection will typically be driven by the goal of
    implementing the \gls{principle of least privilege}}
}

\newglossaryentry{hybrid capability model}
{
  name=hybrid capability model,
  description={A \gls{capability} model in which not all interfaces to use or
    manipulate capabilities conform to the \gls{principle of intentional
    use}, such that legacy software is able to execute around, or within,
    capability-constrained environments, as well as other features required
    to improve compatibility with conventional software designs permitting
    easier incremental adoption of a capability-system model.
    In CHERI, composition of the capability-system model with the conventional
    Memory Management Unit (MMU), the support for \gls{legacy instructions}
    via the \gls{program-counter capability} and \gls{default data
    capability}, and strong compatibility with the C-language \gls{pointer}
    model, all constitute hybrid aspects of its design, in comparison to a
    more pure capability-system model that might elide those behaviors at a
    cost to compatibility and adoptability}
}

\newglossaryentry{principle of intentional use}
{
  name=principle of intentional use,
  description={A design principle in capability systems in which rights are
    always explicitly, rather than implicitly exercised.
    This arises in the CHERI instruction set through explicit \gls{capability}
    operands to \gls{capability-based instructions}, which contributes to the
    effectiveness of \gls{fine-grained memory protection} and
    \gls{control-flow robustness}.
    When applied, the principle limits not just the rights available in the
    presence of a software vulnerability, but the extent to which software can
    be manipulated into using rights in an unintended (and exploitable)
    manner}
}

\newglossaryentry{invoked data capability}
{
  name=invoked data capability (\IDC{}),
  plural=invoked data capabilities,
  description={A capability register reserved by convention to hold the
    unsealed \gls{data capability} on the callee side of \gls{CInvoke}.
    Typically, for the caller side, this will point at a frame on the caller
    stack sufficient to safely restore any caller state.
    On the callee side, the invoked data capability will be a data capability
    describing the object's internal state}
}

\newglossaryentry{kernel code capability}
{
  name=kernel code capability (\KCC{}),
  description={A \gls{special capability register} reserved to hold a
    privileged \gls{code capability} for use by the kernel during exception
    handling.
    This value will be installed in the \gls{program-counter capability} on
    exception entry, with the previous value of the program-counter
    capability stored in the \gls{exception program-counter capability}}
}

\newglossaryentry{kernel data capability}
{
  name=kernel data capability (\KDC{}),
  description={A \gls{special capability register} reserved to hold a
    privileged \gls{data capability} for use by the kernel during exception
    handling.
    Typically, this will refer either to the data segment for a microkernel
    intended to field exceptions, or for the full kernel.
    Kernels compiled to primarily use \gls{legacy instructions} might install
    this in the \gls{default data capability} for the duration of kernel
    execution.
    Use of this register is controlled by \gls{capability permissions} on
    the currently executing \gls{program-counter capability}}
}

\newglossaryentry{kernel reserved capabilities}
{
  name=kernel reserved capabilities,
  description={These \glspl{capability}, modeled on the MIPS kernel reserved
    registers, are set aside for use by a \gls{CHERI-MIPS} operating-system
    kernel in
    exception handling -- in particular, in allowing userspace registers to
    be saved so that the kernel context can be installed.
    As with the MIPS registers, the userspace ABI is not able to use
    capability registers set aside for kernel use; unlike the MIPS registers,
    the kernel reserved capabilities are available for use in the ISA only
    with a suitably authorized \gls{program-counter capability} installed.
    Due to a different exception-handling model in \gls{CHERI-RISC-V}, that
    ISA does not have kernel reserved capabilities}
}

\newglossaryentry{legacy instructions}
{
  name=legacy instructions,
  description={Legacy instructions are those that accept integer addresses,
    rather than capabilities, as their operands, requiring use of the
    \gls{default data capability} for loads and stores, or that explicitly set
    the program counter to a address, rather than doing setting the
    \gls{program-counter capability}.
    These instructions allow legacy binaries (those compiled without CHERI
    awareness) to execute, but only without the benefits of
    \gls{fine-grained memory protection}, granular \gls{control-flow
    robustness}, or more efficient \gls{software compartmentalization}.
    While still constrained, these instructions do not conform to the
    \gls{principle of intentional use}}
}

\newglossaryentry{merged register file}
{
  name=merged register file,
  description={A single general-purpose register file able to hold both
    integer and tagged \gls{capability} values.
    In \gls{CHERI-MIPS}, a dedicated \gls{capability register file} is used,
    separate from the general-purpose integer register file.
    In \gls{CHERI-RISC-V} and \gls{Morello}, a merged register file is supported, reducing the
    amount of control logic required for a separate register file}
}

\newglossaryentry{Morello}
{
  name=Morello,
  description={An application of the CHERI protection model to the ARMv8-A architecture}
}

\newglossaryentry{out of bounds}
{
  name=out of bounds,
  description={When a \gls{capability}'s \gls{capability offset} falls outside
    of its \gls{capability bounds}, it is out of bounds, and cannot be
    \glslink{dereference}{dereferenced}.
    Even if a capability's offset is in bounds, the width of a data access may
    cause a load, store, or instruction fetch to fall out of bounds, or the
    further offset introduced via a register index or immediate operand to an
    instruction.
%    With 256-bit capabilities, all out-of-bounds pointers are
%    \glspl{representable capability}.
    With \glspl{compressed capability}, if an instruction shifts the offset
    too far out of bounds, this may result in an \gls{unrepresentable
    capability}, leading to the \gls{capability tag} being cleared, or an
    exception being thrown}
}

\newglossaryentry{physical address}
{
  name=physical address,
  plural=physical addresses,
  description={An \gls{address} that is passed directly to the memory
    hierarchy without \glslink{virtual address}{virtual-address} translation.
    In \gls{CHERI-MIPS}, \glspl{capability} contain only virtual addresses.
    In \gls{CHERI-RISC-V}, \glspl{capability} addresses may be interpreted as
    physical addresses in Machine Mode}
}

\newglossaryentry{pointer}
{
  name=pointer,
  description={A pointer is a language-level reference to a memory object.
    In conventional ISAs, a pointer is typically represented as an
    \gls{address}.
    In CHERI, pointers can be represented either as an address
    indirected via the \gls{default data capability} or \gls{program-counter
    capability}, or as a \gls{capability}.
    In the latter cases, its integrity and \gls{capability provenance} are
    protected by the \gls{capability tag}, and its use is limited by
    \gls{capability bounds} and \gls{capability permissions}.
    \Gls{capability-based instructions} preserve the tag as required across
    both \glspl{capability register} and \gls{tagged memory}, and also
    enforce \gls{capability monotonicity}: legitimate operations on the
    pointer cannot broaden the set of rights described by the capability}
}

\newglossaryentry{principle of least privilege}
{
  name=principle of least privilege,
  description={A principle of software design in which the set of rights
    available to running code is minimized to only those required for it to
    function, often with the aim of \gls{vulnerability mitigation}.
    In CHERI, this concept applies via fine-grained memory protection for
    both data and code, and also higher-level \gls{software
    compartmentalization}}
}

\newglossaryentry{program-counter capability}
{
  name=program-counter capability (\PCC{}),
  description={A \gls{special capability register} that extends the existing
    program counter to include
    \gls{capability} metadata such as a \gls{capability tag}, \gls{capability
    bounds}, and \gls{capability permissions}.
    The program-counter capability ensures that instruction fetch occurs only
    subject to capability protections.
    When an exception fires, the value of the program-counter capability will
    be moved to the \gls{exception program-counter capability}, and the value
    of the \gls{kernel data capability} moved into the program-counter
    capability.
    On exception return, the value of the exception program-counter capability
    will be moved into the program-counter capability}
}

\newglossaryentry{representable capability}
{
  name=representable capability,
  plural=representable capabilities,
  description={A \gls{compressed capability} whose \gls{capability offset}
    is representable with respect to its \gls{capability bounds}; this
    does not imply that the offset is ``within bounds'', but does require
    that it be within some broader window around the bounds}
}

\newglossaryentry{reserved capability object type}
{
  name=reserved capability object type,
  plural=reserved capability object types,
  description={Certain \glspl{capability object type} are not available for software use and instead have hardware-defined semantics.
  On \gls{CHERI-MIPS} and \gls{CHERI-RISC-V}, all negative \glspl{capability object type} are
  reserved: \glspl{unsealed capability} use the value $2^{64}-1$ and \glspl{sealed entry capability}
  have an object type of $2^{64}-2$.
  The remaining \glspl{capability object type} are used for \glspl{sealed capability with an object type}}
}

\newglossaryentry{return capability}
{
  name=return capability,
  plural=return capabilities,
  description={A \gls{capability} designated as the destination for the
    return address when using a capability jump-and-link instruction.
    A degree of \gls{control-flow robustness} is provided due to
    \gls{capability bounds}, \gls{capability permissions}, and the
    \gls{capability tag} on the resulting capability, which limits sites that
    may be jumped back to using the return capability}
}

\newglossaryentry{sealed capability}
{
  name=sealed capability,
  plural=sealed capabilities,
  description={A sealed \gls{capability} is one whose \gls{capability object type}
    is not equal to the unsealed object type ($2^{64}-1$ for \gls{CHERI-MIPS} and \gls{CHERI-RISC-V}).
    A sealed capability's \gls{address}, \gls{capability bounds},
    \gls{capability permissions}, and other fields are immutable -- i.e.,
    cannot be modified using \gls{capability-based instructions}.
    A sealed capability cannot be directly \glslink{dereference}{dereferenced}
    using the instruction set, and must be unsealed before it can be used.
    This can be used to implement non-monotonic domain transition, as a
    sealed capability may carry rights not otherwise present in the
    \gls{capability register file}.
    Two types exist: \glspl{sealed capability with an object type} and
    \glspl{sealed entry capability}.
    They have different properties catering to different use cases}
}

\newglossaryentry{sealed capability with an object type}
{
  name=sealed capability with an object type,
  plural=sealed capabilities with object types,
  description={A \gls{sealed capability} whose \gls{capability object type}
    is not one of the \glspl{reserved capability object type}.
    These sealed capability have a \gls{capability object type} derived
    from their \glspl{sealing capability}'s \gls{address}.
    CHERI's sealing feature allows capabilities to be used to describe
    software-defined objects, permitting implementation of encapsulation.
    Unsealing can be performed using the \gls{CInvoke} instruction, or
    using the \insnref{CUnseal} instruction combined with a suitable
    \gls{sealing capability}.
    Sealed capabilities with object types provide the necessary architectural
    encapsulation support to efficiently implement fine-grained
    compartmentalization using an object-oriented model}
}

\newglossaryentry{sealed entry capability}
{
  name=sealed entry capability,
  plural=sealed entry capabilities,
  description={A sealed entry \gls{capability} (also known as
    \gls{sentry capability}) is a \gls{sealed capability}
    whose \gls{capability object type} is set to the sentry \gls{reserved capability object type} ($2^{64}-2$ for \gls{CHERI-MIPS} and \gls{CHERI-RISC-V}).
    Sealed entry capabilities are commonly referred to as \glspl{sentry
    capability}.
    Sealed entry capabilities are do not support linking sealed code and
    data capabilities, unlike \glspl{sealed capability with an object type}.
    A sealed entry capability is unsealed by jumping to it using a regular
    capability jump instruction}
}

\newglossaryentry{sealing capability}
{
  name=sealing capability,
  plural=sealing capabilities,
  description={A sealing capability is one with the \cappermSeal 
    permission, allowing it to be used to create \glspl{sealed capability}
    using a \gls{capability object type} set to the sealing capability's
    \gls{address}, and subject to its bounds}
}

\newglossaryentry{sentry capability}
{
  name=sentry capability,
  plural=sentry capabilities,
  description={Sentry capability is a convenient shorthand for a
    \gls{sealed entry capability}}
}

\newglossaryentry{software compartmentalization}
{
  name=software compartmentalization,
  description={The configuration of \glspl{code capability} and \glspl{data
    capability} available via the \gls{capability register file} or
    \gls{merged register file}, accessible \glspl{special capability
    register}, and \gls{tagged memory} such that software components can be
    isolated from one another, enabling \gls{vulnerability mitigation} via the
    application of the \gls{principle of least privilege} at the application
    layer.
    One approach to implementing software compartmentalization on CHERI is to
    use \gls{CInvoke} to jump into sealed code
    and data capabilities describing a trusted intermediary and destination
    protection domain}
}

\newglossaryentry{stack capability}
{
  name=stack capability,
  plural=stack capabilities,
  description={A \gls{capability} referring to the current stack, whose
    \gls{capability bounds} are suitably configured to allow access only to
    the remaining stack available to allocate at a given point in execution}
}

\newglossaryentry{special capability register}
{
  name=special capability register,
  description={Special capability registers have special architectural
    meanings, and include the \gls{program-counter capability}, the
    \gls{default data capability}, the \gls{exception program-counter
    capability}, the \gls{kernel code capability}, and the \gls{kernel data
    capability}.
    Not all registers are accessible at all times; for example, some may be
    available only in certain rings, or when \PCC{} has the
    Access\_System\_Registers permission set}
}

\newglossaryentry{tagged memory}
{
  name=tagged memory,
  description={Tagged memory associates a 1-bit \gls{capability tag} with
    each \gls{capability}-aligned, capability-sized word in memory.
    \Gls{capability-based instructions} that load and store capabilities
    maintain the tag as the capability transits between memory and the
    \gls{capability register file}, tracking \gls{capability provenance}.
    When data stores (i.e., stores of non-capabilities), the tag on the
    memory location will be atomically cleared, ensuring the integrity of
    in-memory capabilities}
}

\newglossaryentry{trusted computing base}
{
  name=Trusted Computing Base (TCB),
  description={The subset of hardware and software that is critical to the
    security of a system;
    in secure system designs, there is often a goal to minimize the size of
    the TCB in order to minimize the opportunity for exploitable software
    vulnerabilities}
}

\newglossaryentry{trusted stack}
{
  name=trusted stack,
  description={Some software-defined object-capability models offer strong
    call-return semantics -- i.e., that if a return is issued by an invoked
    object, or an uncaught exception is generated, then the appropriate caller
    will be returned to -- exactly once.
    This can be implemented via a trusted stack, maintained by the software
    \gls{trusted computing base} via one or more handlers invoked by \gls{CInvoke}.
    A trusted stack for an object-oriented model will likely maintain at least
    the caller's \gls{program-counter capability} and \gls{invoked data
    capability} to be restored on return}
}

\newglossaryentry{unrepresentable capability}
{
  name=unrepresentable capability,
  plural=unrepresentable capabilities,
  description={A \gls{compressed capability} whose \gls{capability offset} is
    sufficiently outside of its \gls{capability bounds} that the combined
    \gls{pointer} value and bounds cannot be represented in the compressed format;
    constructing an unrepresentable capability will lead to the tag being
    cleared (and information loss) or an exception, rather than a violation
    of \gls{capability provenance} or \gls{capability monotonicity}}
}

\newglossaryentry{unsealed capability}
{
  name=unsealed capability,
  plural=unsealed capabilities,
  description={An unsealed \gls{capability} is one whose \gls{capability object type}
    is the unsealed object type ($2^{64}-1$ for \gls{CHERI-MIPS} and \gls{CHERI-RISC-V}).
    Its remaining capability fields are mutable, subject to \gls{capability
    provenance} and \gls{capability monotonicity} rules.
    These capabilities have hardware-defined behaviors -- i.e., subject to
    \gls{capability bounds}, \gls{capability permissions}, and so on,
    can be \glslink{dereference}{dereferenced}}
}

\newglossaryentry{virtual address}
{
  name=virtual address,
  plural=virtual addresses,
  description={An integer \gls{address} translated by the Memory Management
    Unit (MMU) into a \gls{physical address} for the purposes of load, store,
    and instruction fetch.
    \Glspl{capability} embed an address, represented in the instruction
    set as the sum of the \gls{capability base} and \gls{capability offset},
    as well as \gls{capability bounds} relative to the address.
    The integer addresses passed to \glslink{legacy instructions}{legacy load
    and store instructions} that would previously have been interpreted as
    virtual addresses are, with CHERI, transformed (and checked) using the
    \gls{default data capability}.
    Similarly, the integer addresses passed to legacy branch and jump
    instructions are transformed (and checked) using the \gls{program-counter
    capability}.
    This in effect introduces a further relocation of legacy addresses prior
    to virtual address translation}
}

\newglossaryentry{vulnerability mitigation}
{
  name=vulnerability mitigation,
  description={A set of techniques limiting the effectiveness of the attacker
    to exploit a software vulnerability, typically achieved through use of
    the \gls{principle of least privilege} to constrain injection of
    arbitrary code, control of the \gls{program-counter capability} via
    \gls{control-flow robustness} using \glspl{code capability}, minimization of
    data rights granted via available \glspl{data capability}, and higher-level
    \gls{software compartmentalization}}
}



%% bibliography setup:
% UK date format in bibliography:
\usepackage[british]{babel}
\usepackage{csquotes} % recommended for biblatex
% list up to 99 names instead of the default 3 and set
% giveninits=true to match the abbrv bibtex style.
\usepackage[backend=biber,bibencoding=utf8,style=alphabetic,sortcites,maxnames=99,giveninits=true]{biblatex}
\addbibresource{cheri.bib}
% Note: \citetitle formats the title differently depending on the type of entry,
% whereas this macro always uses \textit{}
\newcommand*{\citetitleit}[1]{\textit{\citefield{#1}{title}}}


% Skip unncessary bibtex fields in the bibliography
\AtEveryBibitem{%
\clearfield{issn}%
\clearfield{urldate}%
\clearfield{urlyear}%
\clearfield{review}%
\clearfield{series}%
\clearfield{note}%
\clearfield{address}%
% avoid printing both isbn and DOI
\iffieldundef{doi}{}{\clearfield{isbn}}%
% we don't want 15 JJ Thomson Avenue, Cambridge for every techreport
% Note: location is a list not a field so we need \clearlist
\clearlist{location}%
}
}\makeatother % end of \@ifclassloaded{standalone}
