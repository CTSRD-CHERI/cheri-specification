This version of the \textit{CHERI Instruction-Set Architecture} is an interim
version distributed for review by DARPA and our collaborators:

\begin{itemize}

\item The CHERI Concentrate capability compression format is now documented,
  with a more detailed rationale section than the prior CHERI-128 section.

\item The \insnref{CLCBI} (Capability Load Capability with Big Immediate)
  instruction, which accelerates position-independent access to global
  variables, is no longer considered experimental.

\item The architecture-neutral description of tagged memory has been
  clarified.

\item The maximum supported lengths for both compressed and uncompressed
  capabilities has been updated: $2^{64}$ for 128-bit +capabilities, and
  $2^{64} - 1$ for 256-bit capabilities.

\item It is clarified that \insnref{CLoadTags} instruction must provide
  cache coherency consistent with other load instructions.
  We recommend ``non-temporal'' behavior, in which unnecessary cache-line
  fills are avoided to limit cache pollution during revocation.

\item We now define the object type for unsealed capabilities, returned by
  the \insnref{CGetType} instruction, as $2^{64}-1$ rather than $0$.

\item An experimental section has been added on how CHERI capabilities might
  compose with memory-versioning schemes such as Sparc ADI and Arm MTE
  (see Section~\ref{app:exp:versioning}).

\item Pseudocode throughout the CHERI ISA specification is now generated from
  our Sail formal model of the CHERI-MIPS ISA~\cite{sail-popl2019}.

\item The \hyperref[glossary]{Glossary} has been updated for CHERI ISAv7
  changes including CHERI-RISC-V, split vs. merged register files,
  capabilities for physical addresses, and special capability registers.

\item Capability exception codes are now shared across architectures.

\item CHERI-RISC-V now includes capability-relative floating-point load and
  store instructions.
  We have clarified that existing RISC-V floating-point load and store
  instructions are constrained by \DDC{}.

\item CHERI-RISC-V now throws exceptions, rather than clearing tags, when
  non-mono\-tonic register-to-register capability operations are attempted.

\item While a specific encoding-mode transition mechanism is not yet specified
  for CHERI-RISC-V, candidate schemes are described and compared in greater
  detail.

\item CHERI-RISC-V's ``capability encoding mode'' now has different impacts
  for uncompressed instructions vs. compressed instructions: In the compressed
  ISA, jump instructions also become capability relative.

\item CHERI-RISC-V page-table entries now contain a ``capability dirty bit''
  to assist with tracking the propagation of capabilities.

\item Throwing an exception on an out-of-bounds capability-relative jump
  rather than on the target fetch is now more clearly explained: This improves
  debuggability by maintaining precise information about context state on
  jump, whereas after the jump, bounds may not be representable due to
  capability compression.
  When an inappropriate \EPCC{} is installed, the exception will still be
  thrown on instruction fetch.

\item A new \ErrorEPCC{} special register has been defined, to assist with
  exceptions thrown within exception handlers; its behavior is modeled on the
  existing MIPS \ErrorEPC{} special register.

\end{itemize}
