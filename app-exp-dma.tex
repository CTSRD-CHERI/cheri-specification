CHERI can be composed with DMA controllers in a number of ways.

\subsection{DMA categorization}
We can survey existing DMA controllers and discover a number of design patterns, which are helpful in understanding how CHERI can be applied to DMA subsystems.
We outline them briefly here and refer to Markettos et al.~\cite{FIXME} for further background.

\subsubsection{Transaction types}
A DMA controller's job is to generate memory transactions based on data sources and sinks.
These come in various forms.

\emph{Memory transactions} involve the DMA controller generating a memory cycle such as a read or a write.

\emph{Streaming transactions} have unidirectional data with an ordering but do not of themselves have a memory address, for example network packets received on the wire.
These are reflected on-chip via interfaces such as AXI streaming~\cite{FIXME}, and historically off-chip via ISA-bus DMA~\cite{FIXME}  (which is still relevant for legacy PC peripherals).

A DMA operation is a combination of a source and a sink, ie we can define memory-to-memory (M2M), a memory read feeds a memory write (a memory copy);
memory-to-stream (M2S), a memory read is sent to an outgoing stream; or
stream-to-memory (S2M), an incoming transaction on a stream generates a memory write.
The final case (stream-to-stream, S2S) is associated with on-chip interconnect and does not concern memory access.

Each case may generate a series of transactions based on data sizes and interconnect widths
For example, a read of a 4KiB disk block from a storage controller with a 64-bit datapath might generate a stream of 512 input transactions to the DMA, which could be expanded to 1024 transactions to a 32-bit DRAM chip.

Additionally between the source and sink some processing may be carried out on the data, for example RGB to YUV conversion of a video stream, making it not strictly a copy.

\subsubsection{DMAs with control information in MMIO}
DMA controllers such as the Raspberry Pi RP2350 microcontroller~\cite{FIXME, section 12.6} have all of their DMA controller state in memory-mapped I/O registers.
There are 16 DMA channels, each one consisting of length, a Read and a Write address, which can be memory or peripheral FIFO registers.
Once a DMA is completed an interrupt can be generated or another DMA channel can be triggered, forming basic DMA chains.
Hence the RP2350 is M2M-only and all the control state is held in the MMIO registers, rather than memory.
The Atmel XMEGA DMA~\cite{FIXME} operates in a similar M2M-only fashion using MMIO registers for control.

%On the RP2350 specifically, four TrustZone-like security levels are provided and a memory protection unit (MPU) holds up to 8 regions to check against them.

\subsubsection{DMAs with control information streamed externally}
The AMD (Xilinx) LogiCORE IP AXI DMA~\cite{FIXME} is an example of an M2S/S2M DMA, comprising a memory access port plus streaming sources and sinks for both data and control transactions.
In its most basic mode it can operate with transaction information in MMIO registers, as the previous category.
However for greated performance transaction information (addresses, lengths) can alternatively be streamed into a control port.
This enables custom logic to control and steer memory traffic.

\subsubsection{Descriptor-based DMA}
\emph{Descriptors} are small data structures typically comprising the address of a data buffer, the size of the transaction (eg number of bytes outstanding to send) and other metadata.
For example, how to increment addresses (allowing 2D/3D-style accesses) and what to do next after this buffer is completed.
Descriptors can be used in roughly two ways.
First is the fetching of descriptors from memory, either main memory or a more local memory.
Second is the streaming of descriptors on another port, for example from a separate descriptor fetching unit (as in the LogiCORE AXI DMA above, and also the Altera Modular Scatter-Gather DMA~\cite{FIXME}).
In-memory descriptors can often be chained or used in arrays in order to form \emph{scatter-gather lists} or \emph{ring buffers} to give a higher order structure to groups of transactions.

\subsection{Composing CHERI and DMA}
DMA controllers operate outside the world of CPU-derived capabilities.
The question arises how to compose them.

\subsubsection{Level 0: DMA clears tags}
The simplest yet safe scenario is that DMA operates outside of the world of a CHERI-enabled CPU.
The DMA hardware is unchanged, using integer address rather than capabilities.
Any data written by the DMA clears tags, so the DMA is unable to forge or unsafely modify valid capabilities.
Any pre-existing peripheral device can safely operate in this manner, although this also means they will not follow the capability protection model; an IOMMU may be required to safely constrain their memory access.

\subsubsection{Level 1: addresses replaced by capabiities}
A `CHERI-aware DMA controller' may choose to replace integer addresses in descriptors or MMIO registers with capabilities.
Memory accesses are checked against the bounds and permissions of the capability during transactions, and invalid transactions are aborted.
Address calculation is based on the address from the capability and is otherwise unchanged.

This avoides any difficulties where the DMA controller's use of addresses and the capability model's use of bounds diverge.
For example, a DMA controller may have separate `offset' and `length' fields to deal with restarting partially-completed transactions (for example, a receive transaction where there was insufficient data received to fill the buffer);
these would not map directly to the bounds of a capability.
