This version of the \textit{CHERI Instruction-Set Architecture} is a full
release of the Version 9 specification:

\begin{itemize}
\item We have shifted to CHERI-RISC-V as our primary reference
  platform instead of CHERI-MIPS.  This included several changes to
  Chapter~\ref{chap:model} and Chapter~\ref{chap:architecture} to
  replace MIPS-specific details with more architectural-neutral
  concepts.
  Section~\ref{section:protection-domain-transition-with-cinvoke} was
  also moved to Chapter~\ref{chap:architecture}.

\item The privileged architecture portions of CHERI-RISC-V are now
  defined as an extension to version 1.11 of the RISC-V privileged
  architecture specification.

\item CHERI-RISC-V reports capability exception details in \xtval{}
  rather than \xccsr{}.

\item The RISC-V \insnnoref{JAL} and \insnnoref{JALR} instructions are
  now mode-dependent meaning that they use capability register
  operands in capability mode rather than always using integer
  registers.  The capability mode version of these instructions are
  named \insnref{CJAL} and \insnref{CJALR}.  The previous
  \insnnoref{CJALR} instruction has been renamed to
  \insnref{JALR.CAP}.  In addition, \insnref{JALR.PCC} has been added
  to permit integer jump and links in capability mode.

\item Section~\ref{subsection:compressed-instructions} has been
  rewritten to reflect an initial implementation of CHERI-RISC-V
  compressed instructions in capability encoding mode.

\item Opcode encodings have been reserved for CHERI-RISC-V memory
  versioning instructions as well as \insnnoref{CRelocate}.

\item CHERI-RISC-V always uses a merged register file and the
  architecture-neutral chapters now assume a merged register file on
  all CHERI architectures.  This included removing the dirty bit from
  \xccsr{} as well as the \insnnoref{CGetAddr}, \insnnoref{Clear}, and
  \insnnoref{CSub} instructions.

\item CHERI-RISC-V clears tags rather than raising exceptions for
  non-monotonic modifications to capabilities.

\item Added \insnref{CGetHigh} and \insnref{CSetHigh} to retrieve and
  modify the upper half of a capability.

\item Added \insnref{CGetTop} to retrieve the upper limit of a
  capability.

\item \DDC{} and \PCC{} no longer relocate legacy memory accesses.
  These registers still constrain legacy memory accesses.  This
  included deprecating \insnref{CFromPtr} and \insnref{CToPtr}.

\item Removed CHERI-MIPS from the specification as it is deprecated
  and no longer actively developed.

\item Added a new section in Chapter~\ref{chap:model} describing
  potential uses of capabilities to protect physical addresses.

\item CHERI-RISC-V now enables/disables CHERI extensions via a bit in
  the \menvcfg{} and \senvcfg{} CSRs rather than \xccsr{}.

\item CHERI-RISC-V \xScratchC{} capability registers now extend the
  existing \xscratch{} registers.

\item We have expanded the CHERI-x86-64 sketch in
  Chapter~\ref{chap:cheri-x86-64} to include details on extensions to
  existing instructions to support operations on capabilities as well
  as details for new instructions in a new ISA reference in
  Chapter~\ref{chap:isaref-x86-64}.  The chapter also contains a new
  section evaluating recent security extensions to x86 and how they
  would compose with CHERI.

\item Added a description of the 64-bit CHERI Concentrate capability
  format.
\end{itemize}
