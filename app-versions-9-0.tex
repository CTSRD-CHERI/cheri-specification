This version of the \textit{CHERI Instruction-Set Architecture} is a full
release of the Version 9 specification:

\begin{itemize}
\item We have shifted to CHERI-RISC-V as our primary reference
  platform instead of CHERI-MIPS.  This included several changes to
  Chapter~\ref{chap:model} and Chapter~\ref{chap:architecture} to
  replace MIPS-specific details with more architectural-neutral
  concepts.
  Section~\ref{section:protection-domain-transition-with-cinvoke} was
  also moved to Chapter~\ref{chap:architecture}.

\item CHERI-RISC-V reports capability exception details in \xtval{}
  rather than \xccsr{}.

\item CHERI-RISC-V always uses a merged register file and the
  architecture-neutral chapters now assume a merged register file on
  all CHERI architectures.  This included removing the dirty bit from
  \xccsr{} as well as the \insnnoref{CLEAR} instruction.

\item CHERI-RISC-V clears tags rather than raising exceptions for
  non-monotonic modifications to capabilities.

\item \DDC{} and \PCC{} no longer relocate legacy memory accesses.
  These registers still constrain legacy memory accesses.

\item Removed CHERI-MIPS from the specification as it is deprecated
  and no longer actively developed.

\item Added a new section in Chapter~\ref{chap:model} describing
  potential uses of capabilities to protect physical addresses.

\item CHERI-RISC-V now enables/disables CHERI extensions via a bit in
  the \menvcfg{} and \senvcfg{} CSRs rather than \xccsr{}.

\item We have expanded the CHERI-x86-64 sketch in
  Chapter~\ref{chap:cheri-x86-64} to include details on extensions to
  existing instructions to support operations on capabilities as well
  as details for new instructions in a new ISA reference in
  Chapter~\ref{chap:isaref-x86-64}.
\end{itemize}
