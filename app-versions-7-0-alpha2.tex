This version of the \textit{CHERI Instruction-Set Architecture} is an interim
version distributed for review by DARPA and our collaborators:

\begin{itemize}
\item We have removed the range check from the \insnref{CToPtr}
  specification, as this has proven microarchitecturally challenging.
  We anticipate that current consumers requiring this range check can use the
  new \insnref{CTestSubset} instruction alongside \insnref{CToPtr}.

\item Use of a branch-delay slot with \insnnoref{CCall} Selector
  1 has been removed.

\item With the addition of \insnnoref{CReadHwr} and \insnnoref{CWriteHwr}
  and shifting of special capability registers out of the
  general-purpose capability register file, we have now removed the check for
  the \cappermASR permission for all registers in the
  general-purpose capability register file.

\item A new \insnnoref{CCheckTag} instruction is added, which
  throws an exception if the tag is not set on the operand capability.
  This instruction could be used by a compiler to shift capability-related
  exception behavior from invalid dereference to calculation of an invalid
  capability via a non-exception-throwing manipulation.

\item We have added a new \insnnoref{CLCBI} instruction that
  allows capability-relative loads of capabilities to be performed using a
  substantially larger immediate (but without a general-purpose
  integer-register operand).
  This substantially accelerates performance in the presence of CHERI-aware
  linkage by avoiding multi-instruction sequences to load capabilities for
  global variables.

\item We have added new discussion relating to microarchitectural side
  channels such as Spectre and Meltdown
  (Section~\ref{section:microarchitectural-sidechannels}).

\item We have added a reference to our ASPLOS 2019 paper, \textit{CheriABI:
  Enforcing Valid Pointer Provenance and Minimizing Pointer Privilege in the
  POSIX C Run-time Environment}, which describes how to adapt a full MMU-based
  OS design to support ubiquitous use of capabilities to implement C and C++
  pointers in userspace~\cite{davis2019:cheriabi}.

\item We have added a reference to our POPL 2019 paper, \textit{ISA Semantics
  for ARMv8-A, RISC-V, and CHERI-MIPS}, which describes a formal modeling
  approach for instruction-set architectures, as well as a formal model of the
  CHERI-MIPS ISA~\cite{sail-popl2019}.

\item We have added a reference to our POPL 2019 paper, \textit{Exploring C
  Semantics and Pointer Provenance}, which describes a formal model for C
  pointer provenance, and is evaluated in part using pure-capability CHERI
  code~\cite{cerberus-popl2019}.

\item We have added a description of an experimental compact capability
  coloring scheme, a possible candidate to replace our Local-Global capability
  flow-control model (Section~\ref{sec:compactcolors}).
  In the proposed scheme, a series of orthogonal ``colors'' can be set or
  cleared on capabilities, authorized by a color space implemented in a
  style similar to the sealed-capability object-type space using a single
  permission.
  For a single color implementing the Local-Global model, two bits are still
  used.
  However, for further colors, only a single bit is used.
  This could make available further colors to use for kernel-user separation,
  inter-process isolation, and so on.

\item An experimental Permit\_Recursive\_Mutable\_Load permission is
  described, which, if not present, causes further capabilities loaded via
  that capability to be loaded without store permissions
  (see Section~\ref{app:exp:recmutload}).

\item We have added a new experimental \insnref{CLoadTags}
  instruction that allows tags to be loaded for a cache line without pulling
  data into the cache.

\item A new experimental \textit{sealed entry capability} feature is
  described, which permits entry via jump but otherwise do not allow
  dereferencing (later editions considered these no longer experimental, and
  so they are described in \cref{sec:arch-sentry}).
  These are similar to enter capabilities from the
  M-Machine~\cite{carter:mmachine94}, and could provide utility in providing
  further constraints on capability use for the purposes of memory protection
  -- e.g., in the implementation of C++ v-tables.

\item A new experimental \textit{memory type token} feature is described,
  which provides an alternative mechanism to object types within pairs of
  sealed capabilities (Section~\ref{app:exp:typetoken}).

\end{itemize}
