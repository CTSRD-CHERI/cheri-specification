\chapter{The CHERI-MIPS Instruction-Set Reference}
\label{chap:isaref-mips}

CHERI-MIPS's instructions express a variety of operations affecting capability
and integer registers as well as memory access and control flow.
A key design concern is {\em guarded manipulation}, which provides
a set of constraints
across all instructions that ensure monotonic non-increase in rights through
capability manipulations.
These instructions also assume, and specify, the presence of {\em tagged
memory}, described in the previous chapter, which protects in-memory
representations of capability values.
Many further behaviors, such as reset state and exception handling (taken for
granted in these instruction descriptions), are also described in the previous
chapter.
A small number of more recently specified experimental instructions are
specified in Appendix~\ref{app:experimental} rather than in this chapter.

The instructions fall into a number of categories: instructions to copy
fields from capability registers into integer registers so that they
can be computed on, instructions for refining fields within capabilities, instructions for memory access via
capabilities, instructions for jumps via capabilities, instructions for sealing capabilities, and instructions for capability invocation.
In this chapter, we specify each instruction via both informal descriptions
and code in the Sail language.
To allow for more succinct code descriptions, we rely on a number of
common function definitions also described in this chapter.

%TODO: Insert this section
%\section{Capability Exceptions}

\newcommand{\cchecktag}{\emph{cb}.\ctag{} is not set.}

\section{Sail language used in instruction descriptions}

The instruction descriptions contained in this chapter are accompanied
by code in the Sail language taken from the Sail CHERI-MIPS
implementation~\cite{sail-cheri-mips}.

\section{Common Constant Definitions}

The constants used in the Sail are show in \cref{table:pseudocode-constants}.
% ; their value depends on the capability format in use and architecture specific features such as the RISC-V capability mode flag.

\begin{figure}
\begin{center}
\begin{tabular}{lrl}
\toprule
 & 128-bit & Description \\
\midrule
\verb+cap_size+ & 16 & Number of bytes used to store a capability. \\
%\verb+caps_per_cacheline+ & - & Number of capabilities in a cacheline (uArch detail) \\
\verb+max_otype+ & $2^{18}-1$ & Maximum \cotype{} allowed by capability format. \\
\verb+last_hperm+ & 11 & Last hardware-defined permission bit. \\
\verb+first_uperm+ & 15 & First software-defined permission bit. \\
\verb+last_uperm+ & 18 & Last software-defined permission bit. \\
\verb+num_flags+ & 0 & Number of capability flags. \\
\bottomrule
\end{tabular}
\end{center}

\caption{Constants in Sail code}
\label{table:pseudocode-constants}
\end{figure}

%The null capability is defined as follows:
%
%null\_capability = int\_to\_cap(0)
%
%\textbf{TO DO}: We should have a table defining the values of the capability
%exception codes.

%\section{Common Variable Definitions}
%
%The following variables are used in the pseudocode:
%
%\begin{tabular}{lll}
%cb & : & Capability \\
%cd & : & Capability \\
%cs & : & Capability \\
%ct & : & Capability \\
%rd & : & Unsigned64 \\
%rs & : & Unsigned64 \\
%rt & : & Unsigned64 \\
%mask & : & Unsigned16 \\
%offset & : & Signed16 \\
%\end{tabular}
%
%\textbf{TO DO: The notation for types should be explained as well. \emph{mem}
%and \emph{tags} are also used as variables.}

\section{Common Function Definitions}

This section contains descriptions of convenience functions used by the Sail code featured in this chapter.

\subsection*{Functions for integer and bit vector manipulation}
\label{sec:sail-int-and-bitvector-functions}

The following functions convert between bit vectors and integers and manipulate bit vectors:

\medskip
\sailMIPSval{unsigned}
\sailMIPSval{signed}
\sailMIPSval{to\_bits}
\sailMIPSval{truncate}
\sailMIPSval{pow2}

% The following are overloads so we can't easily use the generated latex
% Hacky hspace to get rid of unwanted hangindent

\phantomsection
\label{sailMIPSzzzerozyextend}
\saildocval{Adds zeros in most significant bits of vector to obtain a vector of desired length.}{\hspace{-\parindent}\isail{zero_extend}}

\label{sailMIPSzsignzyextend}
\saildocval{Extends the most significant bits of vector preserving the sign bit.}{\hspace{-\parindent}\isail{sign_extend}}

\label{sailMIPSzzzeros}
\saildocval{Produces a bit vector of all zeros}{\hspace{-\parindent}\isail{zeros}}

\label{sailMIPSzzones}
\saildocval{Produces a bit vector of all ones}{\hspace{-\parindent}\isail{ones}}

\subsection*{Functions for ISA exception behavior}

\sailMIPSval{SignalException}
\sailMIPSval{SignalExceptionBadAddr}
\sailMIPSval{raise\_c2\_exception}
\sailMIPSval{raise\_c2\_exception\_noreg}
\sailMIPSval{raise\_c2\_exception\_badaddr}
\sailMIPSval{checkCP2usable}
\sailMIPSval{pcc\_access\_system\_regs}

\subsection*{Functions for control flow}

\sailMIPSvalexecuteBranch{}
\sailMIPSvalexecuteBranchPcc{}
\sailMIPSvalsetNextPcc{}
\sailMIPSvalexecute{}

\subsection*{Functions for reading and writing register and memory}

\sailMIPSval{rGPR}
\sailMIPSval{wGPR}
\sailMIPSval{readCapReg}
\sailMIPSval{readCapRegDDC}
\sailMIPSval{writeCapReg}
\sailMIPSval{memBitsToCapability}
\sailMIPSval{capToMemBits}
\sailMIPSval{MEMr\_wrapper}\note{rmn30}{TODO rename horrible memory functions}
\sailMIPSval{MEMr\_reserve\_wrapper}
\sailMIPSval{MEMr\_tagged}
\sailMIPSval{MEMr\_tagged\_reserve}
\sailMIPSval{MEMr\_tag}
\sailMIPSval{MEMw\_wrapper}
\sailMIPSval{MEMw\_conditional\_wrapper}
\sailMIPSval{MEMw\_tagged}
\sailMIPSval{MEMw\_tagged\_conditional}
\sailMIPSval{TLBTranslate}
\sailMIPSval{TLBTranslateC}
\sailMIPSval{wordWidthBytes}
\sailMIPSval{isAddressAligned}
\sailMIPSval{extendLoad}
\sailMIPSval{getAccessLevel}
\sailMIPSval{grantsAccess}

\subsection*{Functions for manipulating capabilities}

The Sail code abstracts the capability representation using the following functions for getting and setting fields in the capability.
% XXXRW: OK change?
%  The functions have different implementations for 256-bit and 128-bit capability formats.

The base of the capability is the address of the first byte of memory to which it grants access and the top is one greater than the last byte, so the set of dereferenceable addresses is:
\[
\{ a \in \mathbb{N} \mid \mathit{base} \leq a < \mathit{top}\}
\]
Note that for 128-bit capabilities $\mathit{top}$ can be up to $2^{64}$, meaning the entire 64-bit address space can be addressed.
% XXXRW: OK change?
%, but that the 256-bit capability format has a maximum length of $2^{64} - 1$ so the last byte of the address space is inaccessible.  Capability length is defined by the relationship $\mathit{base} + \mathit{length} = \mathit{top}$.
\note{rmn30}{XXX For CHERI 256 $top$ is actually derived using $base + length$ but for 128-bit we derive $length$ using $top - base$.  For valid (tagged) capabilities we have (we hope) the invariant $0 \leq base \leq top \leq 2^{64}$, so the length is always positive and in the range $0 .. 2^{64}$ but for untagged 256-capabilities we may have $base + length \geq 2^{64}$ so we have to be careful about overflow.  For most instructions this is OK because we check the tag before using getCapTop but this is not true of CBuildCap and CTestSubset so sail code may not have correct semantics there -- need to test. For 128-bit more thought is required... }

\medskip
\sailMIPSval{getCapBase}
\sailMIPSval{getCapTop}
\sailMIPSval{getCapLength}

\noindent The capability's address (also known as cursor) and offset (relative to base) are related by:
\[
\mathit{base} + \mathit{offset}\ \mathbf{mod}\ 2^{64} = \mathit{cursor}
\]
The following functions return the cursor and offset of a capability respectively:
\note{rmn30}{Re-name cursor to address here?}

\medskip
\sailMIPSval{getCapCursor}
\sailMIPSval{getCapOffset}
\note{rmn30}{explain what happens when offset is negative? In fact it is computed modulo $2^{64}$ and always converted straight to a 64-bit vector so not important. Should maybe just return vector.}

\noindent The following functions adjust the bounds and offset of capabilities.  When using compressed capabilities not all combinations of bounds and offset are representable, so these functions return a boolean value indicating whether the requested operation was successful.  Even in the case of failure a capability is still returned, although it may not preserve the bounds of the original capability. \note{rmn30}{Need more detail here?}

\medskip
\sailMIPSval{setCapBounds}
\sailMIPSval{setCapAddr}
\sailMIPSval{setCapOffset}
\sailMIPSval{incCapOffset}

\medskip
\sailMIPSval{getRepresentableAlignmentMask}
\sailMIPSval{getRepresentableLength}

\medskip
\sailMIPSval{hasReservedOType}
\sailMIPSval{isSentryCap}
\sailMIPSval{sealCap}
\sailMIPSval{unsealCap}
\sailMIPSval{unrepCap}

\mrnote{We should explain the is\_representable function. What properties is
it required to have? All offsets between 0 and length (inclusive) must
be representable for C semantics. Is a length of zero (for any base) required
to be representable? There is also the fast representability check -- there
are in fact several different representability functions depending on where
its called.}

Capability permissions are accessed using the following functions:

\medskip
\sailMIPSval{getCapPerms}
\sailMIPSval{setCapPerms}
\note{rmn30}{could separate user and hw permissions as per existing pseudocode}
\sailMIPSval{getCapFlags}
\sailMIPSval{setCapFlags}

%\sailMIPSval{CapStruct}
%\sailMIPSval{wordWidthBytes}
%\sailMIPSval{getAccessLevel}
%\sailMIPSval{grantsAccess}
%\sailMIPSval{nullcap}
%\sailMIPSval{uintsixfour}
%\sailMIPSval{CapLen}
%\note{rmn30}{CapLen should really be $0 .. 2^{64}$ but that fails to compile -- will need to tweak sail code a bit.}

\section{Table of CHERI-MIPS Instructions}

\begin{table}
\begin{center}
\begin{tabular}{p{1.3in}p{4.6in}}
\toprule
Mnemonic & Description \\
\midrule
  CGetAddr & \insnmipsref*[cgetaddr]{Move capability address to an integer register} \\
  CGetAndAddr & \insnmipsref*[cgetandaddr]{Move capability address to an integer register, with mask} \textit{(experimental)} \\
  CGetBase & \insnmipsref*[cgetbase]{Move base to an integer register}
    \\
  CGetFlags & \insnmipsref*[cgetflags]{Move flags to an integer register} \\
  CGetLen & \insnmipsref*[cgetlen]{Move length to an integer register}
    \\
  CGetOffset & \insnmipsref*[cgetoffset]{Move offset to an integer register} \\
  CGetPerm & \insnmipsref*[cgetperm]{Move permissions to an integer
    register} \\
  CGetSealed & \insnmipsref*[cgetsealed]{Test if a capability is sealed} \\
  CGetTag & \insnmipsref*[cgettag]{Move tag bit to an integer register} \\
  CGetType & \insnmipsref*[cgettype]{Move object type to an integer
    register} \\
  CPtrCmp & \insnmipsref*[cptrcmp]{Capability pointer compare} \\
  CToPtr & \insnmipsref*[ctoptr]{Capability to integer pointer} \\
\midrule
  CAndAddr & \insnmipsref*[candaddr]{Mask address of capability}
    \textit{(experimental)} \\
  CAndPerm & \insnmipsref*[candperm]{Restrict permissions} \\
  CBuildCap & \insnmipsref*[cbuildcap]{Import a capability}
   \textit{(experimental)} \\
  CClearRegs & \insnmipsref*[cclearregs]{Clear multiple registers} \\
  CClearTag & \insnmipsref*[ccleartag]{Clear the tag bit} \\
  CCopyType & \insnmipsref*[ccopytype]{Import a capability's \cotype{}}
   \textit{(experimental)} \\
  CFromPtr & \insnmipsref*[cfromptr]{Create capability from pointer} \\
  CGetPCC & \insnmipsref*[cgetpcc]{Move PCC to capability register} \\
  CGetPCCSetOffset & \insnmipsref*[cgetpccsetoffset]{Move PCC to capability
    register with new offset} \\
  CGetPCCIncOffset & \insnmipsref*[cgetpccincoffset]{Move PCC to capability
    register and increment offset} \\
  CGetPCCSetAddr & \insnmipsref*[cgetpccsetaddr]{Move PCC to capability
    register with new address} \\
  CIncOffset & \insnmipsref*[cincoffset]{Increment offset} \\
  CIncOffsetImm & \insnmipsref*[cincoffsetimm]{Increment Offset by
    Immediate} \\
  CMove & \insnmipsref*[cmove]{Move capability} \\
  CMOVN & \insnmipsref*[cmovn]{Conditionally move capability on non-zero} \\
  CMOVZ & \insnmipsref*[cmovz]{Conditionally move capability on zero} \\
  CReadHwr & \insnmipsref*[creadhwr]{Read a special-purpose capability register} \\
  CSetAddr & \insnmipsref*[csetaddr]{Set capability address to value from register} \\
  CSetBounds & \insnmipsref*[csetbounds]{Set bounds} \\
  CSetBoundsExact & \insnmipsref*[csetboundsexact]{Set bounds exactly} \\
  CSetBoundsImm & \insnmipsref*[csetboundsimm]{Set bounds (immediate)} \\
  CSetFlags & \insnmipsref*[csetflags]{Set flags}
    \textit{(experimental)} \\
  CSetOffset & \insnmipsref*[csetoffset]{Set cursor to an offset from base} \\
  CSub & \insnmipsref*[csub]{Subtract capabilities} \\
  CWriteHwr & \insnmipsref*[cwritehwr]{Write a special-purpose capability register} \\
\bottomrule
\end{tabular}
\end{center}
\caption{Capability coprocessor instruction summary}
\label{table:capability-instruction-summary}
\end{table}

\begin{table}
\begin{center}
\begin{tabular}{p{1.3in}p{4.6in}}
\toprule
Mnemonic & Description \\
\midrule
  CL[BHWD][U] & \insnmipsref*[clbhwd]{Load integer via capability} \\
  CLC & \insnmipsref*[clc]{Load capability via capability} \\
  CLCBI & \insnmipsref*[clcbi]{Load capability via capability (big
    immediate)} \\
  CLL[BHWD][U] & \insnmipsref*[cllbhwd]{Load linked integer via capability} \\
  CLLC & \insnmipsref*[cllc]{Load linked capability via capability} \\
  CSC & \insnmipsref*[csc]{Store capability via capability} \\
  CS[BHWD] & \insnmipsref*[csbhwd]{Store integer via capability} \\
  CSC[BHWD] & \insnmipsref*[cscbhwd]{Store conditional integer via capability} \\
  CSCC & \insnmipsref*[cscc]{Store conditional capability via capability} \\
\midrule
  CBEZ & \insnmipsref*[cbez]{Branch if capability is NULL} \\
  CBNZ & \insnmipsref*[cbnz]{Branch if capability is not NULL} \\
  CBTS & \insnmipsref*[cbts]{Branch if capability tag is set} \\
  CBTU & \insnmipsref*[cbtu]{Branch if capability tag is unset} \\
  CJALR & \insnmipsref*[cjalr]{Jump and link capability register} \\
  CJR  & \insnmipsref*[cjr]{Jump capability register} \\
\midrule
  CCheckTag & \insnmipsref*[cchecktag]{Raise exception if capability tag is
    unset} \\
\midrule
  CSeal & \insnmipsref*[cseal]{Seal a capability} \\
  CCSeal & \insnmipsref*[ccseal]{Conditionally seal a capability}
   \textit{(experimental)} \\
  CUnseal & \insnmipsref*[cunseal]{Unseal a sealed capability} \\
\midrule
  CInvoke & \insnmipsref*[cinvoke]{Call into another security domain} \\
\midrule
  CGetCause & \insnmipsref*[cgetcause]{Move the capability exception cause
    register to an integer register} \\
  CGetCID & \insnmipsref*[cgetcid]{Move the architectural Compartment ID
    (CID) to an integer register} \\
  CSetCause & \insnmipsref*[csetcause]{Set the capability exception cause
    register} \\
  CSetCID & \insnmipsref*[csetcid]{Set the architectural Compartment ID
    (CID)} \\
\bottomrule
\end{tabular}
\end{center}
\caption{Capability coprocessor instruction summary, continued}
\label{table:capability-instruction-summary-2}
\end{table}

Tables~\ref{table:capability-instruction-summary}
and~\ref{table:capability-instruction-summary-2} list available capability
coprocessor instructions.

% rmn30: personally i think this section is pointless.
%\section{Details of Individual Instructions}
%
%The following sections provide a detailed description of each CHERI ISA
%instructions.
%Each instruction description includes the following information:
%
%\begin{itemize}
%\item Instruction opcode format number
%\item Assembly language syntax
%\item Bitwise figure of the instruction layout
%\item Text description of the instruction
%\item Sail code for the instruction
%\item Enumeration of any exceptions that the instruction can trigger
%\end{itemize}

\clearpage
\phantomsection
\addcontentsline{toc}{subsection}{CAndPerm}
\insnmipslabel{candperm}
\subsection*{CAndPerm: Restrict Permissions}

\subsubsection*{Format}

CAndPerm cd, cb, rt

\begin{center}
\cherithreeop[header]{0xd}{cd}{cs}{rt}
\end{center}

\subsubsection*{Description}

Capability register \emph{cd} is replaced with the contents of capability
register \emph{cb} with the \cperms{} field set to the bitwise and of
its previous value and bits 0 to
\hyperref[table:pseudocode-constants]{\emph{last\_hperm}} of integer register \emph{rt}
and the \cuperms{} field set to the bitwise and of its previous value
and bits \hyperref[table:pseudocode-constants]{\emph{first\_uperm}} to
\hyperref[table:pseudocode-constants]{\emph{last\_uperm}} of \emph{rd}.

\subsubsection*{Semantics}
\sailMIPScode{CAndPerm}

\subsubsection*{Exceptions}

A coprocessor 2 exception is raised if:

\begin{itemize}
\item
\cchecktag{}
\item
\emph{cb} is sealed.
\end{itemize}

\clearpage
\phantomsection
\addcontentsline{toc}{subsection}{CBEZ / CBNZ}
\insnmipslabel{cbez}
\subsection*{CBEZ / CBNZ: Branch if Capability is / is Not NULL}

\subsubsection*{Format}

CBEZ cb, offset

\begin{center}
\begin{bytefield}{32}
\bitheader[endianness=big]{0,15,16,20,21,25,26,31}\\
\bitbox{6}{0x12}
\bitbox{5}{0x11}
\bitbox{5}{cb}
\bitbox{16}{offset}
\end{bytefield}
\end{center}
\phantomsection
\insnmipslabel{cbnz}
CBNZ cb, offset

\begin{center}
\begin{bytefield}{32}
\bitheader[endianness=big]{0,15,16,20,21,25,26,31}\\
\bitbox{6}{0x12}
\bitbox{5}{0x12}
\bitbox{5}{cb}
\bitbox{16}{offset}
\end{bytefield}
\end{center}

\subsubsection*{Description}

Sets the \PC{} to \PC{} $+$ 4*{\em offset} $+$ 4, where {\em offset} is sign
extended, depending on whether \emph{cb} is equal to the NULL capability.

The instruction following the branch, in the delay slot, is executed before
branching.

\subsubsection*{Semantics}

\sailMIPScode{CBZ}

\subsubsection*{Notes}

\begin{itemize}
\item
In the above Sail code {\tt notzero} is false for \insnmipsref{CBEZ} and true
for \insnmipsref{CBNZ} thus inverting the sense of the
comparison (via exclusive-or) for the latter.
\item
Like all MIPS branch instructions, \insnmipsref{CBEZ} and
\insnmipsref{CBNZ} have a branch delay slot.  The instruction after
it will always be executed, regardless of whether the branch is taken
or not.
\item
This instruction is intended to resemble the conditional branch instructions
from the MIPS ISA. In particular, the shift left of the offset by 2 bits and
adding 4 is the same as MIPS conditional branches.
\item
Contrary to previous versions of the CHERI architecture the bounds check on
\PCC{} is performed during execution of the branch so an out-of-bounds target
will result in an exception. In the Sail code this check occurs in the
\lstinline{execute_branch} function.
\end{itemize}

\clearpage
\phantomsection
\addcontentsline{toc}{subsection}{CBTS / CBTU}
\insnmipslabel{cbts}
\subsection*{CBTS / CBTU: Branch if Capability Tag is Set / Unset}

\subsubsection*{Format}

CBTS cb, offset

\begin{center}
\begin{bytefield}{32}
\bitheader[endianness=big]{0,15,16,20,21,25,26,31}\\
\bitbox{6}{0x12}
\bitbox{5}{0x0a}
\bitbox{5}{cb}
\bitbox{16}{offset}
\end{bytefield}
\end{center}
\phantomsection
\insnmipslabel{cbtu}
CBTU cb, offset

\begin{center}
\begin{bytefield}{32}
\bitheader[endianness=big]{0,15,16,20,21,25,26,31}\\
\bitbox{6}{0x12}
\bitbox{5}{0x09}
\bitbox{5}{cb}
\bitbox{16}{offset}
\end{bytefield}
\end{center}

\subsubsection*{Description}

Sets the \PC{} to \PC{} $+$ 4*{\em offset} $+$ 4, where {\em offset} is sign
extended, depending on whether \emph{cb.tag} is set.
The instruction following the branch, in the delay slot, is executed before
branching.

\subsubsection*{Semantics}

\sailMIPScode{CBX}

\subsubsection*{Notes}

\begin{itemize}
\item
In the above Sail code {\tt notset}, is false for \insnmipsref{CBTS} and true for \insnmipsref{CBTU} thus inverting the condition (via exclusive-or) for the latter.
\item
Like all MIPS branch instructions, \insnmipsref{CBTS} and
\insnmipsref{CBTU} have a branch delay slot.  The instruction after
it will always be executed, regardless of whether the branch is taken
or not.
\item
This instruction is intended to resemble the conditional branch instructions
from the MIPS ISA. In particular, the shift left of the offset by 2 bits and
adding 4 is the same as MIPS conditional branches.
\item
Contrary to previous versions of the CHERI architecture the bounds check on
\PCC{} is performed during execution of the branch so an out-of-bounds target
will result in an exception. In the Sail code this check occurs in the
\lstinline{execute_branch} function.
\end{itemize}

%\clearpage
\phantomsection
\addcontentsline{toc}{subsection}{CBuildCap}
\insnmipslabel{cbuildcap}
\subsection*{CBuildCap: Import a Capability}

\subsubsection*{Format}

CBuildCap cd, cb, ct

\begin{center}
\begin{bytefield}{32}
\bitheader[endianness=big]{0,5,6,10,11,15,16,20,21,25,26,31}\\
\bitbox{6}{0x12}
\bitbox{5}{0x0}
\bitbox{5}{cd}
\bitbox{5}{cb}
\bitbox{5}{ct}
\bitbox{6}{0x1d}
\end{bytefield}
\end{center}

\usesDDCinsteadofNULL{cb}

\subsubsection*{Description}

\insnmipsref{CBuildCap} attempts to interpret the contents of \emph{ct}
as if it were a valid capability (even though \emph{ct}.\ctag{} is not
required to be set and so \emph{ct} might contain any bit pattern) and
extracts its \cbase{}, \clength{}, \coffset{}, \cperms{} and \cuperms{}
fields.
If the bounds of \emph{ct} cannot be extracted because the bit
pattern in \emph{ct} does not correspond to a permitted value of the capability
type (e.g. \clength{} is negative), then an exception is raised.

If the extracted bounds of \emph{ct} are within the bounds of \emph{cb}, and
the permissions of \emph{ct} are within the permissions of \emph{cb}, then
\emph{cd} is set equal to \emph{cb} with the \cbase{}, \clength{},
\coffset{}, \cperms{} and \cuperms{} of \emph{ct}.

If \emph{ct} is sealed, this instruction does not copy its \cotype{} into \emph{cd}. With
compressed capabilities, a different representation may be used for the bounds
of sealed and unsealed capabilities. If \emph{ct} is sealed,
\insnmipsref{CBuildCap} will change the representation of the bounds so
that their values are preserved.

Because \emph{ct}.\ctag{} is not required to be set, there is no guarantee
that the bounds of \emph{ct} will be in canonical form.
\insnmipsref{CBuildCap} may convert the bounds into canonical form rather than
simply copying their bit representation.

\insnmipsref{CBuildCap} does not copy the fields of \emph{ct} that are
reserved for future use.

\insnmipsref{CBuildCap} can be used to set the tag bit on a capability (e.g.,
one whose non-tag contents has previously been swapped to disk and than
reloaded into memory, or during dynamic linking as untagged capability values
are relocated and tagged after being loaded from a file).
This provides both improved efficiency relative to manual rederivation of the
tagged capability via a series of instructions, and also provides improved
architectural abstraction by avoiding embedding the rederivation sequence in
code.

\subsubsection*{Semantics}
\sailMIPScode{CBuildCap}

\subsubsection*{Exceptions}

A coprocessor 2 exception is raised if:

\begin{itemize}
\item
\emph{cd}, \emph{cb} or \emph{ct} is a reserved register and \PCC.\cperms{} does
not grant \emph{Permit\_Access\_System\_Registers}.
\item
\emph{cb}.\ctag{} is not set.
\item
\emph{cb} is sealed.
\item
The bounds of \emph{ct} are outside the bounds of \emph{cb}.
\item
The values of \cbase{} and \clength{} found in \emph{ct} are not within the
range permitted for a capability with its \ctag{} bit set.
\item
\emph{ct}.\cperms{} grants a permission that is not granted by
\emph{cb}.\cperms{}.
\item
\emph{ct}.\cuperms{} grants a permission that is not granted by
\emph{cb}.\cuperms{}.
\end{itemize}

\subsubsection*{Notes}

\begin{itemize}
\item
This instruction acts both as an optimization, and to provide architectural
abstraction in the face of future change to the capability model.
A similar effect, albeit with reduced abstraction, could be achieved by
using \insnmipsref{CGetBase}, \insnmipsref{CGetLen} and
\insnmipsref{CGetPerm} to query \emph{ct}, and then using
\insnmipsref{CSetBounds} and \insnmipsref{CAndPerm} to set the bounds
and \cperms{} of \emph{cd}.
\item
Despite the description of its intended use above, \insnmipsref{CBuildCap}
does not actually require that \emph{ct} have an unset tag.
\item
\emph{ct} might be a sealed capability that has had its \ctag{} bit cleared.
In this case (assuming an exception is not raised for another reason),
\emph{cd} will be unsealed and the bit representation of
the \cbase{} and \clength{} fields might be changed to take account of
the differing compressed representations for sealed and unsealed capabilities.
\item
This instruction can not be used to break security properties of the capability
mechanism (such as monotonicity) because \emph{cb} must be a valid capability
and the instruction cannot be used to create a capability that grants rights
that were not granted by \emph{cb}.
\item
As the tag bit on \emph{ct} does not need to be set, there is no guarantee
that the bit pattern in \emph{ct} was created by clearing the tag bit on a
valid capability. It might be an arbitrary bit pattern that was created by
other means. As a result, there is no guarantee that the bit pattern in
\emph{ct} corresponds to the encoding of a valid value of the capability type,
especially when capability compression is in use. Fields might have values outside
of their defined range, and invariants such as \cbase{} $\ge 0$, \cbase{} $+$
\clength{} $\le 2^{64}$ or \clength{} $\ge 0$ might not be true. In addition,
fields might not be in a canonical (normalized) form.
\insnmipsref{CBuildCap} checks that the \cbase{} and \clength{} fields are
within the permitted range for the type and satisfy the above invariants,
raising a length exception if they are not. If the fields are not in normalized
form, \insnmipsref{CBuildCap} may renormalize them rather than simply
copying the bit pattern from \emph{ct} into \emph{cd}.
\item
The type constraint \emph{cd}.\ctag{} $\implies$ \emph{cd}.\cbase{} $\ge 0$ is
guaranteed to be satisfied
because \emph{cb}.\cbase{} $\ge 0$ and an exception would be raised if
\emph{ct}.\cbase{} $\le$ \emph{cb}.\cbase{}.
\item
The type constraint \emph{cd}.\ctag{} $\implies$ \emph{cd}.\cbase{} $+$
\emph{cd}.\clength{} $\le 2^{64}$ is guaranteed to be satisfied because this
constraint is true for \emph{cb}, and an exception would be raised if
\emph{ct}.\cbase $+$ \emph{ct}.\clength{} $>$ \emph{cb}.\cbase{} $+$
\emph{cb}.\clength{}.
\item
Is the value of \emph{cd} guaranteed to be representable?
If \emph{ct} was created by clearing the tag bit on a capability, then its
bounds can be represented exactly and there will be no loss of precision.
If \emph{ct} is sealed, then there is a potential issue that the values
of the bounds that are representable in a sealed capability are not the same as
the range of bounds that are representable in an unsealed capability. We rely
on a property of the existing capability formats that if a value of the bounds
is representable in a sealed capability, then it is also representable in an
unsealed capability.
\item As \insnmipsref{CBuildCap} is not able to restore the seal on a
re-tagged capability, it is intended to be used alongside
\insnmipsref{CCSeal}, which will conditionally seal a capability based on a \cotype{} value
extracted with \insnmipsref{CCopyType}.
These instructions will normally be used in sequence to \textit{(i)} re-tag a
capability with CBuildCap, \textit{(ii)} extract a possible object type from
the untagged value with \insnmipsref{CCopyType}, and \textit{(iii)}
conditionally seal the resulting capability with \insnmipsref{CCSeal}.
\item
The typical use of \insnmipsref{CBuildCap} assumes that there is a single
capability \emph{cb} whose bounds include every capability value that is
expected to be encountered in \emph{ct} (with out of range values being an
error). The following are two examples of situations where this is not the
case, and the sequence of instructions to recreate a capability might need
to decide which capability to use as \emph{cb}:
(a) The operating system has enforced a ``write xor execute'' policy, and
the program attempting to recreate \emph{ct} has a capability with
\emph{Permit\_Write} permission and a capability with \emph{Permit\_Execute}
permission, but does not have a capability with both permissions.
(b) The capability in \emph{ct} might be a capability that authorizes sealing
with the \emph{Permit\_Seal} permission, and the program attempting to
recreate it has a capability for a range of memory addresses and a capability
for a range of \cotype{} values, but does not have a single capability that
includes both ranges.
\end{itemize}

\mrnote{TO DO: We can solve the problem of multiple authorizing capabilities
if CBuildCap does nothing if the source already has the tag set, or if the
authorizing capability does not grant sufficient rights. Then, you can just
try all of the authorizing capabilties in turn and see if any of them worked.}

\clearpage
\phantomsection
\addcontentsline{toc}{subsection}{CCheckTag}
\insnmipslabel{cchecktag}
\subsection*{CCheckTag: Raise Exception if Tag is Unset}

\subsubsection*{Format}

CCheckTag cs

\begin{center}
\cherioneop{0x6}{cs}
\end{center}

\subsubsection*{Description}

An exception is raised (and the capability cause register set to ``tag
violation'') if \emph{cs}.\ctag{} is not set.

\subsubsection*{Semantics}
\sailMIPScode{CCheckTag}

\subsubsection*{Exceptions}

A coprocessor 2 exception is raised if:

\begin{itemize}
\item
\emph{cs}.\ctag{} is not set.
\end{itemize}

\subsubsection{Notes}

\begin{itemize}
\item
This instruction is intended to support debugging modes for compilers where
an untagged capability may result from an attempted non-monotonic operation,
rather than an exception.
\end{itemize}

\clearpage
\phantomsection
\addcontentsline{toc}{subsection}{CClearRegs}
\insnmipslabel{cclearregs}
\insnmipslabel{cclearhi}
\insnmipslabel{cclearlo}
\insnmipslabel{clearhi}
\insnmipslabel{clearlo}
\insnmipslabel{fpclearhi}
\insnmipslabel{fpclearlo}
\subsection*{CClearRegs: Clear Multiple Registers}

\subsubsection*{Format}
ClearLo  mask \\
ClearHi  mask \\
CClearLo mask \\
CClearHi mask \\
FPClearLo mask \\
FPClearHi mask

\begin{center}
\begin{bytefield}{32}
\bitheader[endianness=big]{0,15,16,20,21,25,26,31}\\
\bitbox{6}{0x12}
\bitbox{5}{0x0f}
\bitbox{5}{regset}
\bitbox{16}{mask}
\end{bytefield}
\end{center}

\subsubsection*{Description}

The registers in the target register set, \emph{regset}, corresponding
to the set bits in the immediate \emph{mask} field are cleared.  That
is, if bit 0 of \emph{mask} is set, then the lowest numbered register
in \emph{regset} is cleared, and so on. The following values are
defined for the \emph{regset} field:

\begin{figure}[h]
\begin{center}
\begin{tabular}{l|l|l}
Mnemonic & \emph{regset} & Affected registers   \\
\midrule
ClearLo  & 0 & \reg{0}--\reg{15}    \\
ClearHi  & 1 & \reg{16}--\reg{31}   \\
CClearLo & 2 & \DDC, \creg{1}--\creg{15}  \\
CClearHi & 3 & \creg{16}--\creg{31} \\
FPClearLo & 4 & {\bf F0--F15} \\
FPClearHi & 5 & {\bf F16--F31} \\
\end{tabular}
\end{center}
\end{figure}

For integer registers, clearing means setting to zero.
For capability registers, clearing consists of setting all
capability fields such that the in-memory representation will be all zeroes,
with a cleared tag bit, granting no rights.

\paragraph{Note:} For \insnmipsref{CClearLo} bit 0 in \emph{mask} refers to \DDC{} and not \creg{0} since \creg{0} is the NULL register.

\subsubsection*{Semantics}

\sailMIPScode{ClearRegs}

\subsubsection*{Exceptions}

\begin{itemize}
\item
A Reserved Instruction exception is raised for unknown or
unimplemented values of \emph{regset}.
\item
\insnmipsref{CClearLo} and \insnmipsref{CClearHi} raise a coprocessor
unusable exception if the capability coprocessor is disabled.
\item
\insnmipsref{FPClearLo} and \insnmipsref{FPClearHi} raise a coprocessor
unusable exception if the floating point unit is disabled.
\end{itemize}

\subsubsection*{Notes}

\begin{itemize}
\item
These instructions are designed to accelerate the register clearing
that is required for secure domain transitions.  It is expected that
they can be implemented efficiently in hardware using a single `valid'
bit per register that is cleared by the ClearRegs instruction and set
on any subsequent write to the register.
\item
The mnemonic for the integer-register instruction does not make it
very clear what the instruction does.
It would be preferable to have a more descriptive mnemonic.
\end{itemize}

\clearpage
\phantomsection
\addcontentsline{toc}{subsection}{CClearTag}
\insnmipslabel{ccleartag}
\subsection*{CClearTag: Clear the Tag Bit}

\subsubsection*{Format}

CClearTag cd, cb

\begin{center}
\cheritwoop[header]{0xb}{cd}{cb}
\end{center}

\subsubsection*{Description}

Capability register \emph{cd} is replaced with the contents of \emph{cb}, with
the tag bit cleared.

\subsubsection*{Semantics}

\sailMIPScode{CClearTag}



%\clearpage
\phantomsection
\addcontentsline{toc}{subsection}{CCopyType}
\insnmipslabel{ccopytype}
\subsection*{CCopyType: Import the otype field of a Capability}

\subsubsection*{Format}

CCopyType cd, cb, ct

\begin{center}
\begin{bytefield}{32}
\bitheader[endianness=big]{0,5,6,10,11,15,16,20,21,25,26,31}\\
\bitbox{6}{0x12}
\bitbox{5}{0x0}
\bitbox{5}{cd}
\bitbox{5}{cb}
\bitbox{5}{ct}
\bitbox{6}{0x1e}
\end{bytefield}
\end{center}

\subsubsection*{Description}


\insnmipsref{CCopyType} attempts to interpret the contents of \emph{ct} as
if it were a valid capability (even though \emph{ct}.\ctag{} is not required
to be set, and so might contain any bit pattern),
and extracts its \cotype{} field.
If \emph{ct} is sealed, \emph{cd} is set to \emph{cb} with its \coffset{} field set to
\emph{ct}.\cotype{} $-$ \emph{cb}.\cbase{}. If \emph{ct} is not sealed,
\emph{cd} is set to the NULL capability with its \cbase{} $+$ \coffset{} fields
set to $-1$.

\subsubsection*{Semantics}
\sailMIPScode{CCopyType}

\subsubsection*{Exceptions}

A coprocessor 2 exception is raised if:

\begin{itemize}
\item
\emph{cd}, \emph{cb} or \emph{ct} are reserved registers and \PCC{} does not
grant \emph{Access\_System\_Registers} permission.
\item
\emph{cb}.\ctag{} is not set.
\item
\emph{cb} is sealed.
\item
\emph{ct}.\cotype{} is outside the bounds of \emph{cb}.
\end{itemize}

\subsubsection*{Notes}

\begin{itemize}
\item
The intended use case for this instruction is as part of a routine for
resetting the tag bit on a capability that has had its tag bit cleared
(e.g. by being swapped out to disk and then back into memory).

It is a requirement of this specification that if a capability has its
tag bit cleared (either with \insnmipsref{CClearTag} or by copying it as
data), and \insnmipsref{CCopyType} is used to extract the \cotype{} field
of the result, then \emph{cd}.\cbase{} $+$ \emph{cd}.\coffset{} will be
equal to the \cotype{} of the original capability if it was sealed,
and \emph{cd}.\coffset{} will be -1 if the original capability was not
sealed.
\item
Typical usage of this instruction will be to use \insnmipsref{CBuildCap} to
extract the bounds and permissions of a capability, \insnmipsref{CCopyType}
to extract the \cotype{}, and then use \insnmipsref{CCSeal} to seal the
result of the first step with the correct \cotype{}.
\item
This instruction is an optimization. A similar effect could be achieved by
using \insnmipsref{CGetType} to get \emph{ct}.\cotype{} and then
\insnmipsref{CSetOffset} to set \emph{cd}.\coffset{}.
\item
-1 is not a valid value for the \cotype{} field, so the result distinguishes
between the case when \emph{ct} was sealed and the case when it was not
sealed.
\item
If \emph{ct} is sealed and an exception is not raised, then the result is
guaranteed to be representable, because the bounds checks ensure that
\emph{cd}'s cursor is within its bounds.
\item
If \emph{ct}.\cotype{} is outside of the bounds of \emph{ct}, this is an
error condition (attempting to reconstruct a capability that \emph{cb} does
not give you permission to create). In order to catch this error condition
near to where the problem occurred, we raise an exception. This also has
the benefit of avoiding the case where changing \emph{cb}'s \coffset{}
results in a value that is not representable, as explained in the previous
note.
\end{itemize}

\mrnote{What if \emph{cb} covers only a portion of the \cotype{} space,
and \emph{ct}'s \cotype{} doesn't fall within it? As currently defined,
we raise an exception. But what if we want to detect this error and handle
it somehow? What if we have several capabilities covering different portions
of the type space, and want to try them in turn until one works?}

%\clearpage
\phantomsection
\addcontentsline{toc}{subsection}{CCSeal}
\insnmipslabel{ccseal}
\subsection*{CCSeal: Conditionally Seal a Capability}

\subsubsection*{Format}

CCSeal cd, cs, ct

\begin{center}
\begin{bytefield}{32}
\bitheader[endianness=big]{0,5,6,10,11,15,16,20,21,25,26,31}\\
\bitbox{6}{0x12}
\bitbox{5}{0x0}
\bitbox{5}{cd}
\bitbox{5}{cs}
\bitbox{5}{ct}
\bitbox{6}{0x1f}
\end{bytefield}
\end{center}

\subsubsection*{Description}

If \emph{ct}.\ctag{} is false or \emph{ct}.\cbase{} $+$ \emph{ct}.\coffset{}
$= -1$, \emph{cs} is copied into \emph{cd}.
Otherwise, capability register \emph{cs} is sealed with an \cotype{} of
\emph{ct}.\cbase{} $+$ \emph{ct}.\coffset{}
and the result is placed in \emph{cd} as follows:

\begin{itemize}
\item
\emph{cd} is sealed with \emph{cd}.\cotype{} set to \emph{ct}.\cbase{} + \emph{ct}.\coffset{};
\item
and the other fields of \emph{cd} are copied from \emph{cs}.
\end{itemize}

\emph{ct} must grant \emph{Permit\_Seal} permission, and the new \cotype{}
of \emph{cd} must be between \emph{ct}.\cbase{} and \emph{ct}.\cbase{} $+$
\emph{ct}.\clength{} $-$ 1.

\subsubsection*{Semantics}
\sailMIPScode{CCSeal}

\subsubsection*{Exceptions}

A coprocessor 2 exception is raised if:

\begin{itemize}
\item
\emph{cs}.\ctag{} is not set.
\item
\emph{cs} is sealed.
\item
\emph{ct}.\ctag{} is set and \emph{ct} is sealed.
\item
\emph{ct}.\cperms.\emph{Permit\_Seal} is not set.
% \item
% \emph{ct}.\cperms.\emph{Permit\_Execute} is not set.
\item
\emph{ct}.\ctag{} and \emph{ct}.\coffset{} $\ge$ \emph{ct}.length{}
\item
\emph{ct}.\ctag{} and \emph{ct}.\cbase{} $+$ \emph{ct}.\coffset{} $> \emph{max\_otype}$
\item
The bounds of \emph{cb} cannot be represented exactly in a sealed capability.
\end{itemize}

\subsubsection*{Notes}

\begin{itemize}
\item
If capability compression is in use, the range of possible (\cbase{},
\clength{}, \coffset{}) values might be smaller for sealed capabilities
than for unsealed capabilities. This means that \insnmipsref{CCSeal}
can fail with an exception in the case where the bounds are no longer
precisely representable.
\item
This instruction provides two means of indicating that the capability should
not be sealed: either clearing the tag bit on \emph{ct} or setting \emph{ct}'s
cursor to $-1$. A potential problem with just using a cursor of $-1$ (rather than
clearing the tag bit) to disable sealing is that, depending on \emph{ct}'s
\cbase{} and \coffset{}, setting \emph{ct}'s cursor to $-1$ might have a result
that is not representable. However, the NULL capability has \ctag{} clear
and can always have its cursor set to $-1$. (We implement casts from
\verb+int+ to \verb+int_cap_t+ by setting the cursor of NULL to the value of the
integer, and so this can hold a value of $-1$.) Directly clearing \emph{ct}'s
tag to indicate that sealing should not be performed will work, because it
is always possible to clear the tag bit. Setting \emph{ct}'s cursor to $-1$ with
\insnmipsref{CSetOffset} to indicate that sealing should not be performed will
also work, because this will either set the cursor to $-1$ or (if the result would
not be representable) both clear the tag bit and set the cursor to $-1$. The
latter method may be preferred in a code sequence that extracts the \cotype{}
of a capability with \insnmipsref{CGetType}, getting a value of $-1$ if the
capability is not sealed, setting the cursor of \emph{ct} to the result, and
then using \insnmipsref{CCSeal} to seal a new capability to the same
\cotype{} as the original.
\end{itemize}

\clearpage
\phantomsection
\addcontentsline{toc}{subsection}{CFromPtr}
\insnmipslabel{cfromptr}

\subsection*{CFromPtr: Create Capability from Pointer}

\subsubsection*{Format}

CFromPtr cd, cb, rt

\begin{center}
\cherithreeop[header]{0x13}{cd}{cb}{rt}
\end{center}

\usesDDCinsteadofNULL{cb}

\subsubsection*{Description}

\emph{rt} is a pointer using the C-language convention that a zero value
represents the NULL pointer. If \emph{rt} is zero, then \emph{cd} will be
the NULL capability (tag bit not set, all other bits also not set). If
\emph{rt} is non-zero, then \emph{cd} will be set to \emph{cb} with the
\coffset{} field set to \emph{rt}.

\subsubsection*{Semantics}

\sailMIPScode{CFromPtr}

% {\bf FIXME: Is \cotype{} architecturally defined/meaningful if we don't
% have the set type permission? Also, there is a nasty interaction between the
% semantic defintion of fields and their in-memory representation here: the
% idea is that the in-memory representation is all zero bits, which does not
% sit well with the claim that the in-memory representation might be changed
% in future.}

\subsubsection*{Exceptions}

A coprocessor 2 exception is raised if:

\begin{itemize}
\item
\emph{cb}.\ctag{} is not set and \emph{rt} $\neq$ 0.
\item
\emph{cb} is sealed and \emph{rt} $\neq$ 0.
\end{itemize}

\subsubsection*{Notes}

\begin{itemize}
\item
\insnmipsref{CSetOffset} doesn't raise an exception if the tag bit is unset,
so that it can be used to implement the \emph{intcap\_t} type.
\insnmipsref{CFromPtr} raises an exception if the tag bit is unset: although
it would not be a security violation to to allow it, it is an indication
that the program is in error.
\item
The encodings of the NULL capability are chosen so that zeroing memory will
set a capability variable to NULL.
%This holds true for compressed capabilities
%as well as the 256-bit version.
\end{itemize}

\clearpage
\phantomsection
\addcontentsline{toc}{subsection}{CGetAddr}
\insnmipslabel{cgetaddr}
\subsection*{CGetAddr: Move Capability Address to an Integer Register}

\subsubsection*{Format}

CGetAddr rd, cb

\begin{center}
\cheritwoop[header]{0xf}{rd}{cb}
\end{center}

\subsubsection*{Description}

Integer register \textit{rd} is set equal to the \ccursor{} field (the address) of
capability register \textit{cb}.


\subsubsection*{Semantics}

\sailMIPScode{CGetAddr}

\subsubsection*{Notes}

\begin{itemize}
\item
This differs from \insnmipsref{CToPtr} in that it does not perform any range
checks with respect to an authorizing capability, nor require that \emph{cb}
be a valid and unsealed capability.
\end{itemize}

\clearpage
\phantomsection
\addcontentsline{toc}{subsection}{CGetBase}
\insnmipslabel{cgetbase}
\subsection*{CGetBase: Move Base to an Integer Register}

\subsubsection*{Format}

CGetBase rd, cb

\begin{center}
\cheritwoop[header]{0x2}{rd}{cb}
\end{center}

\subsubsection*{Description}

Integer register \textit{rd} is set equal to the \cbase{} field of capability
register \textit{cb}.

\subsubsection*{Semantics}
\sailMIPScode{CGetBase}



\clearpage
\phantomsection
\addcontentsline{toc}{subsection}{CGetCID}
\insnmipslabel{cgetcid}
\subsection*{CGetCID: Move the Architectural Compartment ID to an Integer
  Register}

\subsubsection*{Format}

CGetCID rd

%
% XXXRW: Format and opcode still TODO.
\begin{center}
% XXXAR: 0x4 is currently the next available one-operand instruction
\cherioneop[header]{0x4}{rd}
\\
\arnote{This encoding is not final -- do not implement}
\end{center}

\subsubsection*{Description}

Move the architectural Compartment ID (CID) to integer register \emph{rd},
retrieving the last value set by \insnmipsref{CSetCID}.

\subsubsection*{Semantics}

\sailMIPScode{CGetCID}

\subsubsection*{Exceptions}

This instruction does not raise any exceptions.

\subsubsection*{Notes}

\begin{itemize}
\item
  We choose not to require any additional privilege to query the CID.
  An argument could be made that this is an information leak.
  If so, one possible design choice would be to require
  Access\_System\_Registers to retrieve the value of the CID; however, this
  would be inconsistent with the access-control model for setting the CID.
\item
  While this instruction has been introduced for debugging purposes, it could
  also have utility in indexing state -- e.g., to implement concepts such as
  compartment-local storage.
\end{itemize}

\clearpage
\phantomsection
\addcontentsline{toc}{subsection}{CGetCause}
\insnmipslabel{cgetcause}
\subsection*{CGetCause: Move the Capability Exception Cause Register to an Integer Register}

\subsubsection*{Format}

CGetCause rd

\begin{center}
\cherioneop[header]{0x1}{rd}
\end{center}

\subsubsection*{Description}

Integer register \textit{rd} is set equal to the capability cause register.

\subsubsection*{Semantics}
\sailMIPScode{CGetCause}

\subsubsection*{Exceptions}

A coprocessor 2 exception is raised if:

\begin{itemize}
\item
\PCC{}.\cperms{}.\emph{Access\_System\_Registers} is not set.
\end{itemize}

\clearpage
\phantomsection
\addcontentsline{toc}{subsection}{CGetFlags}
\insnmipslabel{cgetflags}
\subsection*{CGetFlags: Move Flags to an Integer Register}

\subsubsection*{Format}

CGetFlags rd, cb

\begin{center}
\cheritwoop[header]{0x12}{rd}{cb}
\end{center}

\subsubsection*{Description}

The least significant bits of integer register \emph{rd} are set
equal to the \cflags{} field of capability register \emph{cb}.
The other bits of \emph{rd} are set to zero.

\subsubsection*{Semantics}
\sailMIPScode{CGetFlags}



\clearpage
\phantomsection
\addcontentsline{toc}{subsection}{CGetLen}
\insnmipslabel{cgetlen}
\subsection*{CGetLen: Move Length to an Integer Register}

\subsubsection*{Format}

CGetLen rd, cb

\begin{center}
\cheritwoop[header]{0x3}{rd}{cb}
\end{center}

\subsubsection*{Description}

Integer register \textit{rd} is set equal to the \clength{} field of capability
register \textit{cb}.

\subsubsection*{Semantics}
\sailMIPScode{CGetLen}

\subsubsection*{Notes}

\begin{itemize}
\item
%With the 256-bit representation of capabilities, \clength{} is a 64-bit
%unsigned integer and can never be greater than $2^{64}-1$.
With the 128-bit compressed representation of capabilities, the result of
decompressing the length can be $2^{64}$; \insnmipsref{CGetLen} will return
the maximum value of $2^{64}-1$ in this case.
\end{itemize}

\clearpage
\phantomsection
\addcontentsline{toc}{subsection}{CGetOffset}
\insnmipslabel{cgetoffset}
\subsection*{CGetOffset: Move Offset to an Integer Register}

\subsubsection*{Format}

CGetOffset rd, cb

\begin{center}
\cheritwoop[header]{0x6}{rd}{cb}
\end{center}

\subsubsection*{Description}

Integer register \textit{rd} is set equal to the \coffset{} fields of
capability register \textit{cb}.

\subsubsection*{Semantics}
\sailMIPScode{CGetOffset}



\clearpage
\phantomsection
\addcontentsline{toc}{subsection}{CGetPCC}
\insnmipslabel{cgetpcc}
\subsection*{CGetPCC: Move PCC to Capability Register}

\subsubsection*{Format}

CGetPCC cd


\begin{center}
\cherioneop[header]{0x0}{cd}
\end{center}

\subsubsection*{Description}

Capability register \textit{cd} is set equal to the \PCC{}, with
cd.\coffset{} set equal to \PC{}.
\ajnote{implement as CReadHwr PCC ?}

\subsubsection*{Semantics}

\sailMIPScode{CGetPCC}



\clearpage
\phantomsection
\addcontentsline{toc}{subsection}{CGetPCCSetOffset}
\insnmipslabel{cgetpccsetoffset}
\subsection*{CGetPCCSetOffset: Move PCC to Capability Register with New Offset}

\subsubsection*{Format}

CGetPCCSetOffset cd, rs


\begin{center}
\cheritwoop[header]{0x7}{cd}{rs}
\end{center}

\subsubsection*{Description}

Capability register \emph{cd} is set equal to the \PCC{}, with
cd.\coffset{} set equal to \emph{rs}.

\subsubsection*{Semantics}

\sailMIPScode{CGetPCCSetOffset}

\subsection*{Notes}

\begin{itemize}
\item
This instruction is a performance optimization; a similar effect can be
achieved with \insnmipsref{CGetPCC} followed by \insnmipsref{CSetOffset}.
% \note{rmn30}{is it similar or precisely the same except for performance? If not why not? }
% \note{mr101}{I believe that they are exactly the same, except that using
%two instructions will write to a temporary register, which must be
% accessible, etc.}
\end{itemize}

\clearpage
\phantomsection
\addcontentsline{toc}{subsection}{CGetPCCIncOffset}
\insnmipslabel{cgetpccincoffset}
\subsection*{CGetPCCIncOffset: Move PCC to Capability Register and Increment Offset}

\subsubsection*{Format}

CGetPCCIncOffset cd, rs

\begin{center}
\cheritwoop[header]{0x13}{cd}{rs}
\end{center}

\subsubsection*{Description}

Capability register \emph{cd} is set equal to the \PCC{}, with
cd.\caddr{} set equal to \PCC{}.\caddr{} $+$
\emph{rs}.

\subsubsection*{Semantics}

\sailMIPScode{CGetPCCIncOffset}

\subsection*{Notes}

\begin{itemize}
\item
This instruction is a performance optimization; a similar effect can be
achieved with \insnmipsref{CGetPCC} followed by \insnmipsref{CIncOffset}.
\end{itemize}

\clearpage
\phantomsection
\addcontentsline{toc}{subsection}{CGetPCCSetAddr}
\insnmipslabel{cgetpccsetaddr}
\subsection*{CGetPCCSetAddr: Move PCC to Capability Register with New Address}

\subsubsection*{Format}

CGetPCCSetAddr cd, rs

\begin{center}
\cheritwoop[header]{0x14}{cd}{rs}
\end{center}

\subsubsection*{Description}

Capability register \emph{cd} is set equal to the \PCC{}, with
cd.\caddr{} set equal to \emph{rs}.

\subsubsection*{Semantics}

\sailMIPScode{CGetPCCSetAddr}

\subsection*{Notes}

\begin{itemize}
\item
This instruction is a performance optimization; a similar effect can be
achieved with \insnmipsref{CGetPCC} followed by \insnmipsref{CSetAddr}.
\end{itemize}

\clearpage
\phantomsection
\addcontentsline{toc}{subsection}{CGetPerm}
\insnmipslabel{cgetperm}
\subsection*{CGetPerm: Move Permissions to an Integer Register}

\subsubsection*{Format}

CGetPerm rd, cb

\begin{center}
\cheritwoop[header]{0x0}{rd}{cb}
\end{center}

\subsubsection*{Description}

The least significant \hyperref[table:pseudocode-constants]{\emph{last\_hperm}}$+1$ bits of integer register \emph{rd} are set
equal to the \cperms{} field of capability register \emph{cb}; bits
\hyperref[table:pseudocode-constants]{\emph{first\_uperm}} to
\hyperref[table:pseudocode-constants]{\emph{last\_uperm}} of \emph{rd} are set equal to the
\cuperms{} field of \emph{cb}.  The other bits of \emph{rd} are set to zero.

\subsubsection*{Semantics}
\sailMIPScode{CGetPerm}



\clearpage
\phantomsection
\addcontentsline{toc}{subsection}{CGetSealed}
\insnmipslabel{cgetsealed}
\subsection*{CGetSealed: Test if Capability is Sealed}

\subsubsection*{Format}

CGetSealed rd, cb

\begin{center}
\cheritwoop[header]{0x5}{rd}{cb}
\end{center}

\subsubsection*{Description}

The low-order bit of integer register \emph{rd} is set to
$0$ if \emph{cb}.\cotype{} is $2^{64}-1$ and $1$ otherwise.
All other bits of \emph{rd} are cleared.

\subsubsection*{Semantics}
\sailMIPScode{CGetSealed}



\clearpage
\phantomsection
\addcontentsline{toc}{subsection}{CGetTag}
\insnmipslabel{cgettag}
\subsection*{CGetTag: Move Tag Bit to an Integer Register}

\subsubsection*{Format}

CGetTag rd, cb


\begin{center}
\cheritwoop[header]{0x4}{rd}{cb}
\end{center}

\subsubsection*{Description}

The low bit of integer register \emph{rd} is set to the \ctag{} field of
\emph{cb}.  All other bits are cleared.

\subsubsection*{Semantics}
\sailMIPScode{CGetTag}



\clearpage
\phantomsection
\addcontentsline{toc}{subsection}{CGetType}
\insnmipslabel{cgettype}
\subsection*{CGetType: Move Object Type to an Integer Register}

\subsubsection*{Format}

CGetType rd, cb

\begin{center}
\cheritwoop[header]{0x1}{rd}{cb}
\end{center}

\subsubsection*{Description}

Integer register \textit{rd} is set equal to the \cotype{} field of capability
register \textit{cb}.

\subsubsection*{Semantics}
\sailMIPScode{CGetType}

\subsection*{Notes}

\begin{itemize}
\item
If the capability is unsealed, a value of -1 is returned.
For hardware-interpreted \cotype{}s (see \cref{tab:archotypes}) a sign-extended
(negative) value is returned; for software-defined \cotype{}s
(see \cref{sec:model-sealedcapabilities}) \insnmipsref{CGetType} returns
a zero-extended value.
\end{itemize}

\clearpage
\phantomsection
\addcontentsline{toc}{subsection}{CIncOffset}
\insnmipslabel{cincoffset}
\subsection*{CIncOffset: Increment Offset}

\subsubsection*{Format}

CIncOffset cd, cb, rt

\begin{center}
\cherithreeop[header]{0x11}{cd}{cb}{rt}
\end{center}

\subsubsection*{Description}

Capability register \emph{cd} is set equal to capability register
\emph{cb} with its \coffset{} field replaced with \emph{cb}.\coffset{} $+$
\emph{rt}.

If the requested \cbase{}, \clength{}
and \coffset{} cannot be represented exactly, then \emph{cd}.\ctag{} is
cleared and \emph{cd}.\ccursor{} is set equal to
\emph{cb}.\ccursor{} $+$ \emph{rt}.
\arnote{The values of the other capability fields are unspecified?}

\subsubsection*{Semantics}
\sailMIPScode{CIncOffset}

\subsubsection*{Exceptions}

A coprocessor 2 exception is raised if:

\begin{itemize}
\item
\emph{cb}.\ctag{} is set and \emph{cb} is sealed.
\end{itemize}

\subsubsection*{Notes}

\begin{itemize}
\item
For security reasons, \insnmipsref{CIncOffset} must not change the offset
of a sealed capability.
\item
If the tag bit is not set, and the offset is being used to hold an integer,
then \insnmipsref{CIncOffset} should still increment the offset. This is
so that \insnmipsref{CIncOffset} can be used to implement increment of
a \ccode{intcap\_t} type.
Because the tag is unset, \insnmipsref{CIncOffset}
will ignore the decoded \cotype{} and so will not attempt to enforce
immutability of detagged sealed capabilities.
(The precise effect of \insnmipsref{CIncOffset} on such
non-capability data will depend on which binary representation of
capabilities is being used.)
\item
If the tag bit is not set, and capability compression is in use,
the arbitrary data in \emph{cb} might not decompress to sensible values
of the \cbase{} and \clength{} fields, and there is no guarantee that
retaining these values of \cbase{} and \clength{} while changing
\coffset{} will result in a representable value.

From a software perspective, the requirement is that incrementing \coffset{}
on an untagged capability will work if \cbase{} and \clength{} are zero. (This
is how integers, and pointers that have lost precision, will be represented).
If \cbase{} and \clength{} have non-zero values (or \emph{cb} cannot be
decompressed at all), then the values of \cbase{} and \clength{} after this
instruction are \textbf{UNPREDICTABLE}.
\end{itemize}

\clearpage
\phantomsection
\addcontentsline{toc}{subsection}{CIncOffsetImm}
\insnmipslabel{cincoffsetimm}
\subsection*{CIncOffsetImm: Increment Offset by Immediate}

\subsubsection*{Format}

CIncOffset cd, cb, increment$_{imm}$

\begin{center}
\begin{bytefield}{32}
\bitheader[endianness=big]{0,10,11,15,16,20,21,25,26,31}\\
\bitbox{6}{0x12}
\bitbox{5}{0x13}
\bitbox{5}{cd}
\bitbox{5}{cb}
\bitbox{11}{increment$_{imm}$}
\end{bytefield}
\end{center}

\arnote{The assembler supports both CIncOffset and CIncOffsetImm but I think we should always use CIncOffset}

\subsubsection*{Description}

Capability register \emph{cd} is set equal to capability register
\emph{cb} with its \coffset{} field replaced with \emph{cb}.\coffset{} $+$
\emph{increment$_{imm}$}.

If the requested \cbase{}, \clength{} and \coffset{} cannot be represented exactly,
then \emph{cd}.\ctag{} is cleared and \emph{cd}.\ccursor{} is set equal to
\emph{cb}.\ccursor{} $+$ \emph{increment$_{imm}$}.

\subsubsection*{Semantics}
\sailMIPScode{CIncOffsetImmediate}

\subsubsection*{Exceptions}

A coprocessor 2 exception is raised if:

\begin{itemize}
\item
\emph{cb}.\ctag{} is set and \emph{cb} is sealed.
\end{itemize}

\subsubsection*{Notes}

\begin{itemize}
\item
For security reasons, \insnmipsref{CIncOffsetImm} must not change the offset
of a sealed capability.
\item
If the tag bit is not set, and the offset is being used to hold an integer,
then \insnmipsref{CIncOffsetImm} should still increment the offset. This is
so that \insnmipsref{CIncOffsetImm} can be used to implement increment of
a \ccode{intcap\_t} type.
Because the tag is unset, \insnmipsref{CIncOffset}
will ignore the decoded \cotype{} and so will not attempt to enforce
immutability of detagged sealed capabilities.
(The precise effect of \insnmipsref{CIncOffset} on such
non-capability data will depend on which binary representation of
capabilities is being used.)
\item
If the tag bit is not set, and capability compression is in use,
the arbitrary data in \emph{cb} might not decompress to sensible values
of the \cbase{} and \clength{} fields, and there is no guarantee that
retaining these values of \cbase{} and \clength{} while changing
\coffset{} will result in a representable value.

From a software perspective, the requirement is that increasing \coffset{}
on an untagged capability will work if \cbase{} and \clength{} are zero. (This
is how integers, and pointers that have lost precision, will be represented).
If \cbase{} and \clength{} have non-zero values (or \emph{cb} cannot be
decompressed at all), then the values of \cbase{} and \clength{} after this
instruction are \textbf{UNPREDICTABLE}.
\end{itemize}

\clearpage
\phantomsection
\addcontentsline{toc}{subsection}{CInvoke}
\insnmipslabel{cinvoke}
\subsection*{CInvoke: Call into Another Security Domain}

\subsubsection*{Format}

CInvoke cs, cb

\begin{center}
\begin{bytefield}{32}
\bitheader[endianness=big]{0,10,11,15,16,20,21,25,26,31}\\
\bitbox{6}{0x12}
\bitbox{5}{0x05}
\bitbox{5}{cs}
\bitbox{5}{cb}
\bitbox{11}{0x001}
\end{bytefield}
\end{center}

\subsubsection*{Description}

\insnmipsref{CInvoke} is used to jump between protection domains,
unsealing sealed code and data-capability operands, subject to checks on those
capabilities.
This allows the target protection domain to gain access to a different set of capabilities,
supporting implementation of software encapsulation.
The two operand capabilities must be accessible, be valid capabilities, be
sealed, have matching types, and have suitable permissions and bounds, or an
exception will be thrown.
\emph{cs} contains a sealed code capability for the target subsystem, which
will be unsealed and loaded into \PCC{}.
\emph{cb} contains a sealed data capability for the target subsystem, which
will be unsealed and loaded into \IDC{}.
In the parlance of object-oriented programming, \emph{cb} is a capability for
an \emph{object}'s instance data, and \emph{cs} is a capability for the
methods of the object's class.

With \insnmipsref{CInvoke}, a constrained form of non-monotonicity is supported in
the architecture.
Privilege is escalated by virtue of \insnmipsref{CInvoke}
unsealing sealed operand capability registers during a controlled transfer of
execution to the callee in a jump-style transfer of control.
\insnmipsref{CInvoke} requires \emph{Permit\_CInvoke} to be present on both
 sealed capability operands.

\subsubsection*{Semantics}

\sailMIPScode{CCallA}

\noindent
\note{rmn30}{XXX There is a test that CInvoke in branch delay slot of ordinary branch raises Reserved Instruction exception. This needs to be specified if desired. Does it apply to both version of CInvoke? }
\insnmipsref{CInvoke} executes like a branch in the pipeline, but does
not have a branch delay slot. This is due to the difficulty of allowing one
instruction from the calling domain to execute in the new domain.
See Section~\ref{sec:jump-based-domain-transition}.
% XXXRW: Address note:
%
%\note{jdw57}{What is the formal notation for describing an exception condition
%for the following instruction?}

\subsubsection*{Exceptions}

A coprocessor 2 exception is raised if:

\begin{itemize}
\item
\emph{cs} is not sealed.
\item
\emph{cb} is not sealed.
\item
\emph{cs}.\cotype{} $\ne$ \emph{cb}.\cotype{}
\item
\emph{cs}.\cperms{}.\emph{Permit\_Execute} is not set.
\item
\emph{cb}.\cperms{}.\emph{Permit\_Execute} is set.
\item
\emph{cs}.\coffset{} $\ge$ \emph{cs}.\clength{}.
\item
\emph{cs}.\cperms{}.\emph{Permit\_CInvoke} is not set
\item
\emph{cb}.\cperms{}.\emph{Permit\_CInvoke} is not set
\end{itemize}

\subsubsection*{Notes}

\begin{itemize}
\item
From the point of view of security, \insnmipsref{CInvoke} needs to be an atomic
operation (i.e.  the caller cannot decide to just do some of it, because
partial execution could put the system
into an insecure state).
From a hardware perspective, more complex domain-transition implementations
(e.g., to implement function-call semantics or message passing) may need to
perform multiple memory reads and writes, which might take multiple cycles and
complicate control logic.

\item
Implementations may choose to restrict the register numbers that may be passed as \emph{cs} and \emph{cb} in order to avoid the need to decode the instruction and identify the register arguments.
\jhbnote{Is this still true of CInvoke (selector 1) compared to CCall
  selector 0?}

\item
The unsealing of the capabilities stored to PCC and IDC may have implications
beyond just the object type of these capabilities.
When capability compression is in use, the
microarchitectural bit representation of other fields within a capability
may depend on the value of the \cotype{}, so this assignment may have the
effect of changing the bit representation of the other fields. i.e., a
hardware implementation may need to change the representation of the rest
of the capability.
(In the deprecated CHERI-128 of \cref{app:cheri-128},
for example, which does not have a dedicated \cotype{} field,
\insnmipsref{CInvoke} clears the sealed bits of the capabilities
stored to PCC and IDC but must also zero
the bits that held the \cotype{} values
and are now part of the bounds metadata.)

\end{itemize}

\rwnote{Need to add notes on \insnmipsref{CSetCID} here, as well as examples
  to pseudocode.}

\subsubsection*{Expected Software Use}

Higher-level software protection-domain transitions transform the capability
register file to reduce or expand the set of code and data rights available to
the executing thread of control.
In CHERI-based software, these transitions can be usefully modeled as function
invocation or message passing in which data and capability registers are
passed as arguments or messsages, and in which callers and callees can be
protected from undesired access to internal state from the other party (i.e.,
encapsulation).
Domain transitions may implement symmetric (mutual) or asymmetric distrust
between caller and callee, depending on guarantees about limiting callee
access to caller state, and vice versa.

\insnmipsref{CInvoke} may be used to implement mutual distrust by entering a more
privileged ``trusted intermediary'' able to perform capability and
integer register clearing, saving, and restoring, as well as tracking
properties of communications such as message passing or implementing a trusted
stack for reliable call-return semantics and error recovery.
The \insnmipsref{CInvoke} instruction performs a set of checks on sealed
operand capabilities allowing
domain transition to be more efficient.

A number of use cases can be formulated, depending on trust
model.
To implement mutual distrust, sealed code capabilities must point to an
intermediary that is trusted by the callee to implement escalation to
callee privilege.
With respect to capabilities, the caller can perform its own register
clearing and encapsulation of (optional) return state passed via register
arguments to the callee.
\insnmipsref{CInvoke} does not implement a link register, allowing the
calling convention
to implement semantics not implying a leak of \PCC{} to the callee.
In our CheriOS software prototype, sealed code capabilities refer to one of a
set of message-passing implementations, with the sealed data capability
describing the message ring and target domain's code and data capabilities.
A second \insnmipsref{CJR} out of the message-passing implementation into the
callee, combined with suitable register clearing, is suitable to deescalate
privilege to the callee protection domain without a second use of
\insnmipsref{CInvoke}.

\subsubsection*{Sketch of the CheriBSD CInvoke Model}

The CheriBSD \insnmipsref{CInvoke} model implements domain transition via
privileged call and return handlers.
Modeled on function invocation, the call handler performs various
checks (such as of operand register accessibility, validity, sealing, types,
and permissions) on argument registers.
If the checks pass, the handler unseals the sealed operand capabilities,
installing them in \PCC{} and \IDC{}.
It also clears other non-argument registers to prevent data and capability
leakage from caller to callee.
In addition, CheriBSD implements a trusted stack that tracks caller \PCC{} and
\IDC{} so that the return handler can restore control (and
security state) one instruction after the original call site.
Finally, the call handler also implements a form of capability flow
control by preventing the passing of non-global capabilities between caller
and callee.
The return handler pops an entry from the trusted stack, suitably clears
non-return registers, and performs capability flow-control on non-global return
capabilities.
If the return handler is invoked on an empty trusted stack,
the handler raises an exception.
If the call handler is invoked when the trusted stack is full,
the call fails and returns an error to the caller.

\subsubsection*{Sketch of the CheriOS CInvoke Model}

In the CheriOS model, \insnmipsref{CInvoke} is used to implement an
asynchronous message-passing semantic.
The sealed code capability is directed to a software message-passing
implementation that acts as a ``trusted intermediary'', and the sealed data
capability refers to a description of the destination domain including message
ring.
The message-passing implementation adds argument registers to the ring, and
will then either return control to the sender context, or continue in to the
recipient context.
This is accomplished by suitable register-file manipulation to give up any
unnecessary privilege, and an ordinary capability jump to pass control to an
appropriate unprivileged domain.
As with the CheriBSD model, the message-passing routine must perform any
necessary saving of caller context, checking and clearing of registers, and
installation of callee context to support safe interactions.

\clearpage
\phantomsection
\addcontentsline{toc}{subsection}{CJR / CJALR}
\insnmipslabel{cjalr}
\subsection*{CJR / CJALR: Jump (and Link) Capability Register}

\subsubsection*{Format}

CJALR cb, cd

\begin{center}
\cheritwoop[header]{0xc}{cd}{cb}
\end{center}

\phantomsection
\insnmipslabel{cjr}

CJR cb

\begin{center}
\cherioneop[header]{0x3}{cb}
\end{center}

\note{rmn30}{Now that C0 is NULL we could encode this using CJALR with cd == 0 and save a whole opcode.}
\subsubsection*{Description}

The current \PCC{} (with an offset of the current \PC{} $+$ 8) is
optionally saved in \textit{cd}.
\PCC{} is then loaded from capability register \textit{cb} and \PC{}
is set from its offset.

\subsubsection*{Semantics}
\sailMIPScode{CJALR}

\subsubsection*{Exceptions}

A coprocessor 2 exception will be raised if:

\begin{itemize}
\item
\emph{cb.\ctag{}} is not set.
\item
\emph{cb} is sealed. (But see \cref{sec:arch-sentry}.)
\item
\textit{cb.\cperms.Permit\_Execute} is not set.
\item
\textit{cb.\coffset{}} + 4 is greater than \textit{cb.\clength{}}.
\end{itemize}

An address error exception will be raised if

\begin{itemize}
\item
\textit{cb.\cbase{}} + \textit{cb.\coffset{}} is not 4-byte word aligned.
\end{itemize}

\subsubsection*{Notes}

\begin{itemize}
\item
\insnmipsref{CJALR} has a branch delay slot.
\item
The change to \PCC{} does not take effect until the instruction in the branch
delay slot has been executed.
\item
\note{rmn30}{The following text is copied from cjr but was not present here before I merged cjr and cjalr. What is the importance of this? I think it is just to rule out offsets effectively less than zero but since we actually store the address (not offset) and compare against base in Sail I think it is redundant.}
\textit{cb.\cbase{}} and \textit{cb.\clength{}} are treated as unsigned integers,
and the result of the addition does not wrap around (i.e., an exception is
raised if \emph{cb.\cbase{}}+\emph{cb.\coffset{}} is greater than maxaddr).
\end{itemize}

\clearpage
\phantomsection
\addcontentsline{toc}{subsection}{CL[BHWD][U]}
\insnmipslabel{clbhwd}
\subsection*{Load Integer via Capability Register}

\subsubsection*{Format}

CLB rd, rt, offset(cb)\\
CLH rd, rt, offset(cb)\\
CLW rd, rt, offset(cb)\\
CLD rd, rt, offset(cb)\\
CLBU rd, rt, offset(cb)\\
CLHU rd, rt, offset(cb)\\
CLWU rd, rt, offset(cb)\\
CLBR rd, rt(cb)\\
CLHR rd, rt(cb)\\
CLWR rd, rt(cb)\\
CLDR rd, rt(cb)\\
CLBUR rd, rt(cb)\\
CLHUR rd, rt(cb)\\
CLWUR rd, rt(cb)\\
CLBI rd, offset(cb)\\
CLHI rd, offset(cb)\\
CLWI rd, offset(cb)\\
CLDI rd, offset(cb)\\
CLBUI rd, offset(cb)\\
CLHUI rd, offset(cb)\\
CLWUI rd, offset(cb)

\begin{center}
\begin{bytefield}{32}
\bitheader[endianness=big]{0,1,3,10,11,15,16,20,21,25,26,31}\\
\bitbox{6}{0x32}
\bitbox{5}{rd}
\bitbox{5}{cb}
\bitbox{5}{rt}
\bitbox{8}{offset}
\bitbox{1}{s}
\bitbox{2}{t}
\end{bytefield}
\end{center}

\usesDDCinsteadofNULL{cb}

\subsubsection*{Purpose}

Loads a data value via a capability register, and extends the value to fit the target register.

\subsubsection*{Description}

The lower part of integer register \emph{rd} is loaded from the memory
location specified by \emph{cb.\cbase{}} + \emph{cb.\coffset{}} + \emph{rt} +
$2^t$ $*$  \emph{offset}. Capability register \emph{cb} must
contain a valid capability that grants permission to load data.

The size of the value loaded depends on the value of the \emph{t} field:

\begin{description}
	\item[0] byte (8 bits)
	\item[1] halfword (16 bits)
	\item[2] word (32 bits)
	\item[3] doubleword (64 bits)
\end{description}

The extension behavior depends on the value of the \emph{s} field: 1 indicates sign extend, 0 indicates zero extend.  For example, \insnmipsref[clbhwd]{CLWU} is encoded by setting \emph{s} to 0 and \emph{t} to 2, \insnmipsref[clbhwd]{CLB} is encoded by setting \emph{s} to 1 and
\emph{t} to 0.

\subsubsection*{Semantics}
\sailMIPScode{CLoad}

\subsubsection*{Exceptions}

A coprocessor 2 exception is raised if:

\begin{itemize}
\item
\cchecktag{}
\item
\emph{cb} is sealed.
\item
\emph{cb}.\cperms.\emph{Permit\_Load} is not set.
\item
\emph{addr} $+$ \emph{size} $>$ \emph{cb}.\cbase{} $+$ \emph{cb}.\clength{}

NB: The check depends on the size of the data loaded.
\item
\emph{addr} $<$ \emph{cb}.\cbase{}
\end{itemize}

An AdEL exception is raised if \emph{addr} is not correctly aligned.

\subsubsection*{Notes}

\begin{itemize}
\item
This instruction reuses the opcode from the Load Word to Coprocessor 2
(\insnnoref{LWC2}) instruction in the MIPS Specification.
\item
\emph{rt} is treated as an unsigned integer.\note{rmn30}{since rt is 64-bit and addr is computed modulo $2^{64}$ it doesn't make any difference whether it is treated as signed or unsigned. It actually can make sense to think of it as a signed offset. Maybe we should remove this note? Important to note that immediate offset is signed, however.}
\item
\emph{offset} is treated as a signed integer.
% \item
% The result of the addition does not wrap around (i.e., an exception is
% raised if cb.\cbase{} $+$ cb.\coffset{} $+$ \emph{rt} $+$ \emph{offset}
% is less than zero, or greater than \emph{maxaddr}).
% \item
% The temporary variable \emph{addr} can have values that are less than zero
% or greater than $2^{64}-1$; the computation of \emph{addr} does not wrap
% around modulo $2^{64}$.
\item
BERI1 has a compile-time option to allow unaligned loads and stores. If BERI1
is built with this option, an unaligned load will only raise an exception if
it crosses a cache line boundary.
\end{itemize}

\clearpage
\phantomsection
\addcontentsline{toc}{subsection}{CLC}
\insnmipslabel{clc}
\label{sailMIPSzCLC} % for CLCNT
\subsection*{CLC: Load Capability via Capability}

\subsubsection*{Format}

CLC cd, rt, offset(cb) \\
CLCR cd, rt(cb) \\
CLCI cd, offset(cb)

\begin{center}
\begin{bytefield}{32}
\bitheader[endianness=big]{0,5,6,10,11,15,16,20,21,25,26,31}\\
\bitbox{6}{0x36}
\bitbox{5}{cd}
\bitbox{5}{cb}
\bitbox{5}{rt}
\bitbox{11}{offset}
\end{bytefield}
\end{center}

\usesDDCinsteadofNULL{cb}

\subsubsection*{Description}
Capability register \emph{cd} is loaded from the memory location specified by
\emph{cb.\cbase{}} $+$ \emph{cb.\coffset{}} $+$ \emph{rt} $+$ \emph{offset}.
Capability register
\emph{cb} must
contain a capability that grants permission to load capabilities.  The virtual
address \emph{cb.\cbase{}} $+$ \emph{cb.\coffset{}} $+$ \emph{rt} $+$
\emph{offset} must be \emph{capability\_size} aligned.

The bit in the tag memory corresponding to \emph{cb.\cbase{}} $+$
\emph{cb.\coffset{}} $+$ \emph{rt} $+$ \emph{offset} is
loaded into the tag bit associated with \emph{cd}.
%  The CBTU instruction can be used to check the value of this tag.

\subsubsection*{Semantics}
\sailMIPScode{CLC}

\subsubsection*{Exceptions}


A coprocessor 2 exception is raised if:

\begin{itemize}
\item
\cchecktag{}
\item
\emph{cb} is sealed.
\item
\emph{cb}.\cperms.\emph{Permit\_Load} is not set.
\item
\emph{addr} + \emph{capability\_size} $>$ \emph{cb}.\cbase{} $+$ \emph{cb}.\clength{}.
\item
\emph{addr} $<$ \emph{cb}.\cbase{}.
\end{itemize}


An address error during load (AdEL) exception is raised if:

\begin{itemize}
\item
The virtual address \emph{addr} is not \emph{capability\_size} aligned.
\end{itemize}

\subsubsection*{Notes}

\begin{itemize}
\item
This instruction reuses the opcode from the Load Doubleword to Coprocessor 2
(\insnnoref{LDC2}) instruction in the MIPS Specification.
\item
\emph{offset} is interpreted as a signed integer.
\item
The \insnmipsref[clbhwd]{CLCI} mnemonic is equivalent to \insnmipsref{CLC} with
\emph{cb} being the zero register (\$zero). The \insnmipsref[clbhwd]{CLCR} mnemonic
is equivalent to \insnmipsref{CLC} with \emph{offset} set to
zero.
\item
Although the \emph{capability\_size} can vary, the offset is always in
multiples of 16 bytes (128 bits).
\end{itemize}

\clearpage
\phantomsection
\addcontentsline{toc}{subsection}{CLCBI}
\insnmipslabel{clcbi}
\subsection*{CLCBI: Load Capability via Capability (Big Immediate)}

\subsubsection*{Format}

CLCBI cd, offset(cb)

\begin{center}
\begin{bytefield}{32}
\bitheader[endianness=big]{0,5,6,10,11,15,16,20,21,25,26,31}\\
\bitbox{6}{0x1d}
\bitbox{5}{cd}
\bitbox{5}{cb}
\bitbox{16}{offset}
\end{bytefield}
\end{center}

\usesDDCinsteadofNULL{cb}

\subsubsection*{Description}
Capability register \emph{cd} is loaded from the memory location specified by
\emph{cb.\cbase{}} $+$ \emph{cb.\coffset{}} $+$ \emph{offset}.
Capability register
\emph{cb} must
contain a capability that grants permission to load capabilities.  The virtual
address \emph{cb.\cbase{}} $+$ \emph{cb.\coffset{}} $+$
\emph{offset} must be \emph{capability\_size} aligned.

The bit in the tag memory corresponding to \emph{cb.\cbase{}} $+$
\emph{cb.\coffset{}} $+$ \emph{offset} is
loaded into the tag bit associated with \emph{cd}.
%  The CBTU instruction can be used to check the value of this tag.

\subsubsection*{Semantics}
\sailMIPScode{CLCBI}

\subsubsection*{Exceptions}


A coprocessor 2 exception is raised if:

\begin{itemize}
\item
\cchecktag{}
\item
\emph{cb} is sealed.
\item
\emph{cb}.\cperms.\emph{Permit\_Load} is not set.
\item
\emph{addr} + \emph{capability\_size} $>$ \emph{cb}.\cbase{} $+$ \emph{cb}.\clength{}.
\item
\emph{addr} $<$ \emph{cb}.\cbase{}.
\end{itemize}


An address error during load (AdEL) exception is raised if:

\begin{itemize}
\item
The virtual address \emph{addr} is not \emph{capability\_size} aligned.
\end{itemize}

\subsubsection*{Notes}

\begin{itemize}
\item
This instruction reuses the opcode from the Jump and Link Exchange
(\insnnoref{JALX}) instruction in the MIPS Specification.
Future versions of the architecture may use a different encoding
to avoid reusing an opcode with a delay slot for an instruction without a delay slot.
\item
\emph{offset} is interpreted as a signed integer.
\item
Although the \emph{capability\_size} can vary, the offset is always in
multiples of 16 bytes (128 bits).
\item
%The instruction allows accessing individual members in large arrays of capabilities
%without the need for additional instructions to generate the offset.
The larger immediate of \insnmipsref{CLCBI}) enables more efficient code generation
in pure-capability programs for accesses of global variables.
In many programs and libraries, the 11-bit immediate offset of \insnmipsref{CLC}
is not sufficient reach all entries in the table of global capabilities and
therefore the compiler must to generate a three-instruction sequence instead.
By using of \insnmipsref{CLCBI}) for accessing globals, the code size of most
pure-capability binaries can be reduced by over 10\%.

\item
Architectures with pc-relative loads or instructions to add a large immediate to \PCC{}
(such as \insnnoref{AUIPC}) can use those instead of adding a capability
load with a larger immediate offset.

\end{itemize}

\clearpage
\phantomsection
\addcontentsline{toc}{subsection}{CLL[BHWD][U]}
\insnmipslabel{cllbhwd}
\insnmipslabel{cllb}
\insnmipslabel{cllh}
\insnmipslabel{cllw}
\insnmipslabel{clld}
\subsection*{CLL[BHWD][U]: Load Linked Integer via Capability}

\subsubsection*{Format}

CLLB rd, cb\\
CLLH rd, cb\\
CLLW rd, cb\\
CLLD rd, cb\\
CLLBU rd, cb\\
CLLHU rd, cb\\
CLLWU rd, cb

\begin{center}
\begin{bytefield}{32}
\bitheader[endianness=big]{0,1,2,3,4,10,11,15,16,20,21,25,26,31}\\
\bitbox{6}{0x12}
\bitbox{5}{0x10}
\bitbox{5}{rd}
\bitbox{5}{cb}
\bitbox{7}{\color{lightgray}\rule{\width}{\height}}
\bitbox{1}{1}
\bitbox{1}{s}
\bitbox{2}{t}
\end{bytefield}
\end{center}

\usesDDCinsteadofNULL{cb}

\subsubsection*{Description}

\insnnoref{CLL[BHWD][U]} and \insnnoref{CSC[BHWD]} are used to implement safe access
to data shared between different threads. The typical usage is that
\insnnoref{CLL[BHWD][U]} is followed (an arbitrary number of
instructions later) by \insnnoref{CSC[BHWD]} to the same address; the
\insnnoref{CSC[BHWD]} will only succeed if the memory location that was loaded
by the \insnnoref{CLL[BHWD][U]} has not been modified.

The exact conditions under which \insnnoref{CSC[BHWD]} fails are implementation
dependent, particularly in multicore or multiprocessor implementations). The
following code is intended to represent the security semantics of the
instruction correctly, but should not be taken as a definition of the CPU's
memory coherence model.

\subsubsection*{Semantics}
\sailMIPScode{CLoadLinked}

\subsubsection*{Exceptions}

A coprocessor 2 exception is raised if:

\begin{itemize}
\item
\emph{cb}.\ctag{} is not set.
\item
\emph{cb} is sealed.
\item
\emph{cb}.\cperms{}.\emph{Permit\_Load} is not set.
\item
\emph{addr} $+$ \emph{size} $>$ \emph{cb}.\cbase{} $+$ \emph{cb}.\clength{}
\item
\emph{addr} $<$ \emph{cb}.\cbase{}
\end{itemize}

An AdEL exception is raised if \emph{addr} is not correctly aligned.

\clearpage
\phantomsection
\addcontentsline{toc}{subsection}{CLLC}
\insnmipslabel{cllc}
\subsection*{CLLC: Load Linked Capability via Capability}

\subsubsection*{Format}

CLLC cd, cb \\

\begin{center}
\begin{bytefield}{32}
\bitheader[endianness=big]{0,2,3,10,11,15,16,20,21,25,26,31}\\
\bitbox{6}{0x12}
\bitbox{5}{0x10}
\bitbox{5}{cd}
\bitbox{5}{cb}
\bitbox{7}{\color{lightgray}\rule{\width}{\height}}
\bitbox{4}{1111}
\end{bytefield}
\end{center}

\usesDDCinsteadofNULL{cb}

% \textbf{TO DO: Describe what this instruction does.}

\subsubsection*{Semantics}
\sailMIPScode{CLLC}

\subsubsection*{Exceptions}

A coprocessor 2 exception is raised if:

\begin{itemize}
\item
\emph{cb}.\ctag{} is not set.
\item
\emph{cb} is sealed.
\item
\emph{cb}.\cperms.\emph{Permit\_Load} is not set.
\item
\emph{addr} $+$ \emph{capability\_size} $>$ \emph{cb}.\cbase{} $+$ \emph{cb}.\clength{}
\item
\emph{addr} $<$ \emph{cb}.\cbase{}
\end{itemize}

An AdEL exception is raised if:

\begin{itemize}
\item
\emph{addr} is not capability aligned.
\end{itemize}

\clearpage
\phantomsection
\addcontentsline{toc}{subsection}{CLoadTags}
\insnmipslabel{cloadtags}
\subsection*{CLoadTags: Read Multiple Tags to Integer Register}

\subsubsection*{Format}

CLoadTags rd, cb \\

\begin{center}
\begin{bytefield}{32}
\bitheader[endianness=big]{0,5,6,10,11,15,16,20,21,25,26,31}\\
\bitbox{6}{0x12}
\bitbox{5}{0x00}
\bitbox{5}{rd}
\bitbox{5}{cb}
\bitbox{5}{0x1E}
\bitbox{5}{0x3F}
\end{bytefield}
\end{center}

\usesDDCinsteadofNULL{cb}

\subsubsection*{Description}

The \emph{tags} from memory referenced by \emph{cb} are loaded to \emph{rd},
with bit significance increasing with memory address.  The result of this
instruction must be coherent with other processors, \emph{as if} the
corresponding data memory words had been loaded.  The number of tags loaded is
an implementation-defined constant but is constrained to be a power of two, at
least 1, and no more than the width of \emph{rd}.

Capability register \emph{cb} must contain a capability that grants permission
to load capabilities.  The virtual address \emph{cb.\cbase{}} $+$
\emph{cb.\coffset{}} must be suitably aligned; the precise requirements are,
again, implementation defined, but must equal the width of memory corresponding
to the tags loaded.  If any tag to be loaded corresponds to memory out of
bounds of \emph{cb}, a length violation is indicated.

\subsubsection*{Semantics}

\sailMIPScode{CLoadTags}

\subsubsection*{Exceptions}

A coprocessor 2 exception is raised if:

\begin{itemize}
\item
\cchecktag{}
\item
\emph{cb.\ctag{}} is clear.
\item
\emph{cb} is sealed.
\item
\emph{cb}.\cperms.\emph{Permit\_Load} is not set.
\item
\emph{cb}.\cbase{} $+$ \emph{cb}.\coffset{} + $n$ $*$ \emph{capability\_size} $>$ \emph{cb}.\cbase{} $+$ \emph{cb}.\clength{},
		where $n$ is the number of capabilities to be fetched (\isail{caps_per_cacheline} in the Sail code).
\item
\emph{cb}.\cbase{} $+$ \emph{cb}.\coffset{} $<$ \emph{cb}.\cbase{}.
\end{itemize}

An address error during load (AdEL) exception is raised if:

\begin{itemize}
\item
The virtual address \emph{addr} is not $n$ $*$ \emph{capability\_size} aligned.
\end{itemize}

\subsubsection*{Notes}

\begin{itemize}
%
\item In practice, the number of tags loaded is likely less arbitrary than it
may appear. Usually, the implementation's cache fabric determines the minimum
granularity of coherency and already tracks tag bits along with each cache
line, and so this instruction fetches the tag bits from the cache line
indicated by \emph{cb}.\cbase{} $+$ \emph{cb}.\coffset{}. Of course, additional
complexity in the cache and tag cache fabric could allow this instruction to
retrieve more tags than in a cache line.
Also, the number of tags this instruction loads should be a power-of-two to
avoid alignment issues and to preserve page divisibility in systems with MMUs.
In order to reduce DRAM traffic, it is desirable that this \emph{tag fetch} not
require loading the corresponding data and not necessarily evict other lines
from caches. (However, a non-zero result is probably a reasonable hint that a
capability load is likely to follow.)


\item Software can easily discover the width used by any implementation by
constructing an aligned array of capabilities in memory and observing the
result of \insnmipsref{CLoadTags}.
%; for a CHERI instantiation with 256-bit
%capability representations and 64-bit integer registers, the maximal
%alignment requirement for these probes is 512 bytes.
Such probes need be done only rarely, at system or allocator startup.

\item For multi-core or multi-processor systems with cache fabrics wherein cache
lines are of different sizes, \insnmipsref{CLoadTags} must behave as if all
cores view the memory subsystem through the \emph{smallest} cache line in
the system.

\end{itemize}

\clearpage
\phantomsection
\addcontentsline{toc}{subsection}{CGetAddr}
\insnmipslabel{cmove}
\subsection*{CMove: Move Capability to another Register}

\subsubsection*{Format}

CMove cd, cb

\begin{center}
\cheritwoop[header]{0xa}{cd}{cb}
\end{center}

\subsubsection*{Description}

\insnnoref{CMove} copies \emph{cb} into \emph{cd}.

\subsubsection*{Semantics}

\sailMIPScode{CMove}

\subsubsection*{Notes}

\begin{itemize}
\item This instruction currently has a dedicated encoding but it could also be implemented as an alias for \insnmipsref{CMOVZ} \emph{\$zero}, \emph{cd}, \emph{cb}. \arnote{This is not possible on RISC-V since there is no conditional move. Should we add a note about this?}
\item Originally, \insnmipsref{CMove} was an assembler pseudo for \insnmipsref{CIncOffset} \emph{cd}, \emph{cb}, \emph{\$zero}.
However, this requires that \insnmipsref{CIncOffset} with a sealed capability succeeds if the increment is zero.
A future version of the ISA might no longer support this and require the use of \insnmipsref{CMove} for sealed capabilities.
This would allow for a simpler implementation of \insnmipsref{CIncOffset} where the behavior does not depend on one of the input values. \arnote{Some more rationale about intentionality?}
\end{itemize}

\clearpage
\phantomsection
\addcontentsline{toc}{subsection}{CMOVZ / CMOVN}
\insnmipslabel{cmovn}
\subsection*{CMOVZ / CMOVN: Conditionally Move Capability on Zero / Non-Zero}

\subsubsection*{Format}

CMOVN cd, cb, rt

\begin{center}
\cherithreeop[header]{0x1c}{cd}{cb}{rt}
\end{center}

\phantomsection
\insnmipslabel{cmovz}
CMOVZ cd, cb, rt

\begin{center}
\cherithreeop[header]{0x1b}{cd}{cb}{rt}
\end{center}

\subsubsection*{Description}

CMOVZ copies \emph{cb} into \emph{cd} if \emph{rt} $=$ 0. \\
CMOVN copies \emph{cb} into \emph{cd} if \emph{rt} $\neq$ 0. \\

\subsubsection*{Semantics}

\sailMIPScode{CMOVX}

\subsubsection*{Notes}

\begin{itemize}
\item
In the Sail code {\tt ismovn} is true for \insnmipsref{CMOVN}, thus inverting the condition (via exclusive-or) in that case.
\item
Some implementations of cryptographic algorithms need a constant-time move
operation to avoid revealing secret key material through a timing channel.
(An attacker must not be able to determine whether a condition variable
inside the cryptographic implementation is true or false from observations
of how long the operation took to complete). In the current prototype
implementation of CHERI, no guarantees are made about \insnmipsref{CMOVN}
being constant time.

If CHERI instructions are to be used in high-security cryptographic
processors, consideration should be given to making this operation
constant time.
\end{itemize}

\clearpage
\phantomsection
\addcontentsline{toc}{subsection}{CPtrCmp: CEQ, CNE, CL[TE][U], CEXEQ}
\insnmipslabel{cptrcmp}
\insnmipslabel{ceq}
\insnmipslabel{cne}
\insnmipslabel{clt}
\insnmipslabel{cle}
\insnmipslabel{cltu}
\insnmipslabel{cleu}
\insnmipslabel{cexeq}
\insnmipslabel{cnexeq}
\subsection*{CPtrCmp: CEQ, CNE, CL[TE][U], CEXEQ, CNEXEQ: Capability Pointer Compare}

\subsubsection*{Format}

CEQ rd, cb, ct \\
CNE rd, cb, ct \\
CLT rd, cb, ct \\
CLE rd, cb, ct \\
CLTU rd, cb, ct \\
CLEU rd, cb, ct \\
CEXEQ rd, cb, ct \\
CNEXEQ rd, cb, ct \\
\cherithreeop[header]{\emph{op}}{rd}{cb}{ct}

\subsubsection*{Description}

Capability registers \emph{cb} and \emph{ct} are compared, and the result
of the comparison is placed in integer register \emph{rd}.
The rules for comparison are as follows:

\begin{itemize}
\item For all operations except \insnmipsref{CExEq} and \insnmipsref{CNExEq},
the result of comparison is the result of comparing \emph{cb}.\caddr{} with \emph{ct}.\caddr{}.
Numerical comparison is signed for \insnmipsref{CLT} and \insnmipsref{CLE},
and unsigned for \insnmipsref{CLTU} and \insnmipsref{CLEU}.
\item
\insnmipsref{CExEq} and \insnmipsref{CNExEq} compare all the fields of the two
capabilities, including \ctag{} and the bits that are reserved for future use.
\end{itemize}

This instruction can be used to compare capabilities so that capabilities can
replace pointers in C executables.


\begin{figure}[h]
\begin{center}
\begin{tabular}{l|l|l}
Mnemonic & \emph{op} & Comparison \\ \hline
\insnmipsref{CEQ} & 0x14 & = \\
\insnmipsref{CNE} & 0x15 & $\neq$ \\
\insnmipsref{CLT} & 0x16 & $<$ (signed) \\
\insnmipsref{CLE} & 0x17 & $\le$ (signed) \\
\insnmipsref{CLTU} & 0x18 & $<$ (unsigned) \\
\insnmipsref{CLEU} & 0x19 & $\le$ (unsigned) \\
\insnmipsref{CEXEQ} & 0x1a & all fields are equal \\
\insnmipsref{CNEXEQ} & 0x21 & not all fields are equal \\
\end{tabular}
\end{center}
\end{figure}

\subsubsection*{Semantics}
\sailMIPScode{CPtrCmp}

\subsubsection*{Exceptions}

A reserved instruction exception is raised if

\begin{itemize}
\item
\emph{op} does not correspond to comparison operation whose meaning has been
defined. (All possible values of \emph{op} have now been assigned meanings,
so this exception cannot occur).
\end{itemize}

\subsubsection*{Notes}

\begin{itemize}
\item
\insnmipsref{CLTU} can be used by a C compiler to compile code that
compares two non-NULL pointers (e.g., to detect whether a pointer to a character
within a buffer has reached the end of the buffer). When two pointers to
addresses within the same object (e.g., to different offsets within an array)
are compared, the pointer to the earlier part of the object will be compared
as less. (Signed comparison would also work as long as the object did not span
 address $2^{63}$; the MIPS address space layout makes it unlikely that
objects spanning $2^{63}$ will exist in user-space C code).
\item
Although the ANSI C standard does not specify whether a NULL
pointer is less than or greater than a non-NULL pointer (clearly, they must
not be equal), the comparison instructions have been designed so that when
C pointers are represented by capabilities, NULL will be less than any
non-NULL pointer.
\item
A C compiler may also use these instructions to compare two values of
type \ccode{uintptr\_t} that have been obtained by casting from
an integer value. If the cast is compiled as a \insnmipsref{CFromPtr} of
zero followed by \insnmipsref{CSetOffset} to the integer value, the
result of \insnmipsref{CPtrCmp} will be the same as comparing the original
integer values, because \insnmipsref{CFromPtr} will have set \cbase{} to
zero. Signed and unsigned capability comparison operations are provided so
that both signed and unsigned integer comparisons can be performed on
capability registers.
\item
A program could use pointer comparison to determine the value of
\cbase{}, by setting \coffset{} to different values and testing which values
cause \cbase{} $+$ \coffset{} to wrap around and be less than \cbase{} $+$
a zero offset. This is not an attack against a security property of the ISA,
because \cbase{} is not a secret.
\item
One possible way in which garbage collection could be implemented is for the
garbage collector to move an object and fix up all capabilities that refer
to it. If there are appropriate restrictions on which capabilities the program
has to start with, the garbage collector can be sure that the program does
not have any references to the object stored as integers, and so can know
that it is safe to move the object. With this type of garbage collection,
comparing pointers by extracting their base and offset with
\insnmipsref{CGetBase} and \insnmipsref{CGetOffset} and comparing the
integer values is not guaranteed to work, because the garbage collector
might have moved the object part-way through. \insnmipsref{CPtrCmp} is
atomic, and so will work in this scenario.
\item
Some compilers may make the optimization that if a check for (\emph{a} $=$
\emph{b}) has succeeded, then \emph{b} can be replaced with \emph{a} without
changing the semantics of the program. This optimization is not valid
for the comparison performed by \insnmipsref{CEq}, because two capabilities
can point to the same place in memory but have different bounds, permissions
etc. and so not be interchangeable. The \insnmipsref{CExEq} instruction is
provided for when a test for semantic equivalence of capabilities is
needed; it compares all the fields, even the ones that are reserved for
future use.
\item
Mathematically, \insnmipsref{CEq} divides capabilities into
\emph{equivalence classes}, and the signed or unsigned comparison operators
provide a \emph{total ordering} on these equivalence classes.
\insnmipsref{CExEq} also divides capabilities into equivalence classes,
but these are not totally ordered: two capabilities can be unequal according
to \insnmipsref{CExEq}, and also neither less or greater according to
\emph{CLT} (e.g., if they have the same \cbase{} $+$ \coffset{}, but different
\clength{}).
\item
There is an outstanding issue: when capability compression is in use, does
\insnmipsref{CExEq} compare the compressed representation or the uncompressed
capability? There might be a difference between the two if there are multiple
compressed representations that decompress to the same thing. If
\ctag{} is false, then then capability register might contain non-capability
data (e.g., an integer, or a string) and it might not decompress to anything
sensible. Clearly in this case the in-memory compressed representation should
be compared bit for bit. Is it also acceptable to compare the compressed
representations when \ctag{} is true? This might lead to two capabilities that
are sematically equivalent but have been computed by a different sequence
of operations comparing as not equal. The consequence of this for programs
that use \insnmipsref{CExEq} is for further study.
\item
If a C compiler compiles pointer equality as \insnmipsref{CExEq} (rather than
\insnmipsref{CEq}), it will catch the following example of undefined
behavior. Suppose that \emph{a} and \emph{b} are capabilities for different
objects, but \emph{a} has been incremented until its \cbase{} $+$ \coffset{}
points to the same memory location as \emph{b}. Using \insnmipsref{CExEq},
these pointers will not compare as equal because they have different bounds.
\item
\insnmipsref{CNE} and \insnmipsref{CNExEq} are in principle redundant,
because a compiler could replace \insnmipsref{CNE} and a conditional branch
with \insnmipsref{CEq} and a conditional branch with the opposite condition
(and if the result of the comparison is assigned to a variable, the compiler
could explictly negate the result of \insnmipsref{CEq}, at a small performance
penalty). We provide explicit tests for not equal in order to simplify the
compiler back end.
\end{itemize}

\clearpage
\phantomsection
\addcontentsline{toc}{subsection}{CRepresentableAlignmentMask}
\insnmipslabel{cram}
\insnmipslabel{crepresentablealignmentmask}
\subsection*{CRepresentableAlignmentMask: Retrieve Mask to Align Capabilities to Precisely Representable Address}

\subsubsection*{Format}

CRepresentableAlignmentMask rt, rs

\begin{center}
\cheritwoop[header]{0x11}{rt}{rs}
\end{center}

\subsubsection*{Description}

\emph{rt} is set to a mask that can be used to align down addresses to a value that is sufficiently aligned to set precise bounds for the nearest representable length of \emph{rs} (which may be obtained using the instruction \insnmipsref{CRoundRepresentableLength}).
\arnote{TODO: this description is pretty terrible}

\subsubsection*{Semantics}

\sailMIPScode{CRAM}

\subsubsection*{Notes}

\begin{itemize}
\item The result of \insnmipsref{CRepresentableAlignmentMask} is intended to be used as the mask argument to \insnmipsref{CAndAddr} in order to create  a capability with an address that is sufficiently aligned to perform \insnmipsref{CSetBoundsExact} with the specified length. This \insnmipsref{CSetBoundsExact} is guaranteed to succeed if the size is rounded using \insnmipsref{CRoundRepresentableLength}.
\item The required alignment of an allocation of size \emph{rs} can be computed by negating \emph{rt} and adding one.
\item Combined with \insnmipsref{CRoundRepresentableLength} this instruction can be used in memory allocators to guarantee non-overlapping allocations.
\item This instruction can be useful to adjust the stack pointer to an address that is suitably aligned for dynamic allocations.
\item An alternative to this instruction is the use of count-leading-zeroes and count-trailing-zeros instructions followed by shifting and masking. However, this requires encoding knowledge of the underlying precision in the resulting binary and can therefore result in incompatibilities with future architectures that use a different compression algorithm.
\end{itemize}


\clearpage
\phantomsection
\addcontentsline{toc}{subsection}{CReadHwr}
\insnmipslabel{creadhwr}
\subsection*{CReadHwr: Read a Special-Purpose Capability Register}

\subsubsection*{Format}

CReadHwr cd, selector

\begin{center}
\cheritwoop[header]{0xd}{cd}{selector}
\end{center}

\subsubsection*{Description}

Load the value of special-purpose capability register \emph{selector}
into capability register \emph{cd}.
See \autoref{tab:creadhwr-permissions} for the possible values of
\emph{selector} and the permissions required in order to read the register.

% TODO: move this to the preamble?
\newcommand{\KernelAndAccessSysRegs}{
\begin{tabular}[c]{@{}l@{}}Supervisor Mode and\\
\PCC{}.\cperms{}.\emph{Access\_System\_Registers}
\end{tabular}
}

\begin{table}[h]
\centering
\caption{Access permission required to read special-purpose capability registers}
\label{tab:creadhwr-permissions}
\begin{tabular}{cll@{}}
\toprule
& \textbf{Register}& \textbf{Required for read access} \\
\midrule

\textbf{0}  & Default data capability (\DDC) & $\emptyset$  \\
\textbf{1} & User TLS  (\CULR{}) & $\emptyset$ \\
%\textbf{2} & \ajnote{\PCC - make this CGetPCC} & $\emptyset$ \\[1.5em]

% \textbf{7}  & Program counter capability (\PCC)   & $\emptyset$ \\[1.5em]


\textbf{8}  & Privileged TLS (\CPLR{}) & \PCC{}.\cperms{}.\emph{Access\_System\_Registers} \\

\textbf{22} & Kernel scratch register 1 (\KRC)  & \KernelAndAccessSysRegs \\
\textbf{23} & Kernel scratch register 2 (\KQC)  & \KernelAndAccessSysRegs \\

\textbf{28} & Error exception program counter (\ErrorEPCC)  &  \KernelAndAccessSysRegs \\
\textbf{29} & Kernel code capability (\KCC)     &  \KernelAndAccessSysRegs \\
\textbf{30} & Kernel data capability (\KDC)    & \KernelAndAccessSysRegs \\
\textbf{31} & Exception program counter (\EPCC) & \KernelAndAccessSysRegs \\
\bottomrule
\end{tabular}
\end{table}

\note{rmn30}{currently this table is the same for read and write but need not be (e.g. rdhwr on MIPS has a register that can be written by kernel but only read by user)}

\subsubsection*{Semantics}
\sailMIPScode{CReadHwr}

\subsubsection*{Exceptions}


A reserved Instruction exception is raised for unknown or
unimplemented values of \emph{selector}.

A coprocessor 2 exception is raised if:

\begin{itemize}
\item the permission checks as specified in \autoref{tab:creadhwr-permissions} above were not met for \emph{selector}
\end{itemize}

\subsubsection*{Notes}

\begin{itemize}
\item In the future we may decide to make \PCC{} accessible via this instruction. This would save opcode space since we would no longer required a dedicated \insnmipsref{CGetPCC} instruction.
\end{itemize}

\clearpage
\phantomsection
\addcontentsline{toc}{subsection}{CRoundRepresentableLength}
\insnmipslabel{crrl}
\insnmipslabel{croundrepresentablelength}
\subsection*{CRoundRepresentableLength: Round to Next Precisely Representable Length}

\subsubsection*{Format}

CRoundRepresentableLength rt, rs

\begin{center}
\cheritwoop[header]{0x10}{rt}{rs}
\end{center}

\subsubsection*{Description}

\emph{rt} is set to the smallest value greater or equal to \emph{rs} that can be used by \insnmipsref{CSetBoundsExact} without trapping (assuming a suitably aligned base).

\subsubsection*{Semantics}

\sailMIPScode{CRAP}

\subsubsection*{Notes}

\begin{itemize}
\item This instruction is useful when implementing allocators to round up allocation sizes to a size that can be precisely bounded (and will therefore not overlap with any other allocations). It is also useful when using \ccode{mmap()}, since the requested size must be a precisely representable length.
\item An alternative to this instruction is the use of count-leading-zeroes and count-trailing-zeros instructions followed by shifting and masking. However, this requires encoding knowledge of the underlying precision in the resulting binary and can therefore result in incompatibilities with future architectures that use a different compression algorithm.
\item If the length has to be rounded up to $2^{64}$ then this instruction will return zero. This could happen for very large lengths that span most of the address space. Software must be careful to account for this, especially if the length comes from an untrusted source.
\end{itemize}

\clearpage
\phantomsection
\addcontentsline{toc}{subsection}{CS[BHWD]}
\insnmipslabel{csbhwd}
\subsection*{CS[BHWD]: Store Integer via Capability}

\subsubsection*{Format}

CSB rs, rt, offset(cb)\\
CSH rs, rt, offset(cb)\\
CSW rs, rt, offset(cb)\\
CSD rs, rt, offset(cb)\\
CSBR rs, rt(cb)\\
CSHR rs, rt(cb)\\
CSWR rs, rt(cb)\\
CSDR rs, rt(cb)\\
CSBI rs, offset(cb)\\
CSHI rs, offset(cb)\\
CSWI rs, offset(cb)\\
CSDI rs, offset(cb)\\

\begin{center}
\begin{bytefield}{32}
\bitheader[endianness=big]{0,1,3,10,11,15,16,20,21,25,26,31}\\
\bitbox{6}{0x3A}
\bitbox{5}{rs}
\bitbox{5}{cb}
\bitbox{5}{rt}
\bitbox{8}{offset}
\bitbox{1}{0}
\bitbox{2}{t}
\end{bytefield}
\end{center}

\usesDDCinsteadofNULL{cb}

\subsubsection*{Purpose}

Stores some or all of a register into a memory location.

\subsubsection*{Description}

Part of integer register \emph{rs} is stored to the memory location specified by
\emph{cb.\cbase{}} + \emph{cb.\coffset{}} + \emph{rt} + $2^t$ $*$ \emph{offset}.
Capability register \emph{cb} must contain a capability that grants permission
to store data.

The \emph{t} field determines how many bits of the register are stored to memory:

\begin{description}
	\item[0] byte (8 bits)
	\item[1] halfword (16 bits)
	\item[2] word (32 bits)
	\item[3] doubleword (64 bits)
\end{description}

If less than 64 bits are stored, they are taken from the least-significant
end of the register.

\subsubsection*{Semantics}
\sailMIPScode{CStore}

\subsubsection*{Exceptions}

A coprocessor 2 exception is raised if:

\begin{itemize}
\item
\cchecktag{}
\item
\emph{cb} is sealed.
\item
\emph{cb}.\cperms.\emph{Permit\_Store} is not set.
\item
\emph{addr} $+$ \emph{size} $>$ \emph{cb}.\cbase{} $+$ \emph{cb}.\clength{}.
\item
\emph{addr} $<$ \emph{cb}.\cbase{}
\end{itemize}

An address error during store (AdES) is raised if:

\begin{itemize}
\item
\emph{addr} is not aligned.
\end{itemize}

\subsubsection*{Notes}

\begin{itemize}
\item
This instruction reuses the opcode from the Store Word from Coprocessor 2
(\insnnoref{SWC2}) instruction in the MIPS Specification.
\item
\emph{rt} is treated as an unsigned integer.
\item
\emph{offset} is treated as a signed integer.
% \item
% The computation of \emph{addr} does not wrap around modulo $2^{64}$.
\item
BERI1 has a compile-time option to allow unaligned loads and stores. If BERI1
is built with this option, an unaligned store will only raise an exception if
it crosses a cache line boundary.
\end{itemize}

\clearpage
\phantomsection
\addcontentsline{toc}{subsection}{CSC}
\insnmipslabel{csc}
\subsection*{CSC: Store Capability via Capability}

\subsubsection*{Format}

CSC cs, rt, offset(cb) \\
CSCR cs, rt(cb) \\
CSCI cs, offset(cb)

\begin{center}
\begin{bytefield}{32}
\bitheader[endianness=big]{0,5,6,10,11,15,16,20,21,25,26,31}\\
\bitbox{6}{0x3e}
\bitbox{5}{cs}
\bitbox{5}{cb}
\bitbox{5}{rt}
\bitbox{11}{offset}
\end{bytefield}
\end{center}

\usesDDCinsteadofNULL{cb}

\subsubsection*{Description}

Capability register \emph{cs} is stored at the memory location specified by
\emph{cb.\cbase{}} $+$ \emph{cb.\coffset{}} $+$ \emph{rt} $+$ 16 $*$ \emph{offset},
and the bit in the tag memory associated with \emph{cb.\cbase{}} $+$
\emph{cb.\coffset{}} $+$ \emph{rt} $+$ 16 $*$ \emph{offset} is set to the value of
\emph{cs.tag}.
Capability register \emph{cb} must
contain a capability that grants permission to store capabilities.  The virtual
address \emph{cb.\cbase{}} $+$ \emph{cb.\coffset{}} $+$ \emph{rt} $+$
16 $*$ \emph{offset} must be \emph{capability\_size} aligned.

% When the 256-bit representation of capabilities is in use, the capability
% is stored in memory in the format described in Figure
% \ref{fig:memory-representation-of-a-capability}.
% \cbase{}, \clength{} and \cotype{} are
% stored in memory with the same endian-ness that the CPU uses for double-word
% stores, i.e., big-endian. The bits of \cperms{} are stored with bit zero
% being the least significant bit, so that the least significant bit of the
% eighth byte stored is the \csealed{} bit,
% the next significant bit is the \emph{Global} bit, the next is
% \emph{Permit\_Execute} and so on.

The various capability encoding schemes define bit representations in
memory.  While a given instantiation of CHERI will use a particular scheme,
software should, in general, not attempt to parse capability bit patterns
from memory.  Instructions for capability interrogation (e.g.,
\insnmipsref{CGetAddr}, \insnmipsref{CGetType}) do not require that their
source registers be holding \emph{tagged} capabilities; software wishing to
decode memory bit patterns should rather use \insnmipsref{CLC} and
interrogate the result.

\subsubsection*{Semantics}
\sailMIPScode{CSC}

\subsubsection*{Exceptions}

A coprocessor 2 exception is raised if:

\begin{itemize}
\item
\cchecktag{}
\item
\emph{cb} is sealed.
\item
\emph{cb}.\cperms.\emph{Permit\_Store} is not set.
\item
\emph{cb.\cperms.Permit\_Store\_Capability} is not set.
\item
\emph{cb.\cperms{}.Permit\_Store\_Local} is not set and
\emph{cs.tag} is set and \emph{cs.\cperms{}.Global} is not set.
\item
\emph{addr} $+$ \emph{capability\_size} $>$ \emph{cb}.\cbase{}
$+$ \emph{cb.\clength{}}.
\item
\emph{addr} $<$ \emph{cb}.\cbase{}.
\end{itemize}

\noindent
A TLB Store exception is raised if:

\begin{itemize}
\item
\emph{cs.\ctag{}} is set and the \emph{S} bit in the TLB entry for the page
containing \emph{addr} is not set.
\end{itemize}

\noindent
An address error during store (AdES) exception is raised if:

\begin{itemize}
\item
The virtual address \emph{addr} is not \emph{capability\_size} aligned.
\end{itemize}

\subsubsection*{Notes}

\begin{itemize}
\item
If the address alignment check fails and one of the security checks fails,
a coprocessor 2 exception (and not an address error exception) is raised.
The priority of the exceptions is security-critical, because otherwise a
malicious program could use the type of the exception that is raised to
test the bottom bits of a register that it is not permitted to access.
\item
It is permitted to store a local capability with the tag bit unset even if the permit store local bit is not set in \emph{cb}.
This is because if the tag bit is not set then the permissions have no meaning.
\item
\emph{offset} is interpreted as a signed integer.
\item
This instruction reuses the opcode from the Store Doubleword from Coprocessor 2
(\insnnoref{SDC2}) instruction in the MIPS Specification.
\item
The \insnnoref{CSCI} mnemonic is equivalent to \insnmipsref{CSC} with
\emph{cb} being the zero register (\$zero). The \insnnoref{CSCR} mnemonic
is equivalent to \insnmipsref{CSC} with \emph{offset} set to zero.
\item
BERI1 has a compile-time option to allow unaligned loads and stores.
\insnmipsref{CSC} to an unaligned address will raise an exception even if
BERI1 has been built with this option, because it would be a security
vulnerability if an attacker could construct a corrupted capability with
\ctag{} set by writing it to an unaligned address.
\item
Although the \emph{capability\_size} can vary, the offset is always in
multiples of 16 bytes (128 bits).
\end{itemize}

\clearpage
\phantomsection
\addcontentsline{toc}{subsection}{CSC[BHWD]}
\insnmipslabel{cscbhwd}
\insnmipslabel{cscb}
\insnmipslabel{csch}
\insnmipslabel{cscw}
\insnmipslabel{cscd}
\subsection*{CSC[BHWD]: Store Conditional Integer via Capability}

\subsubsection*{Format}

CSCB rd, rs, cb \\
CSCH rd, rs, cb \\
CSCW rd, rs, cb \\
CSCD rd, rs, cb


\begin{center}
\begin{bytefield}{32}
\bitheader[endianness=big]{0,1,2,3,4,5,6,10,11,15,16,20,21,25,26,31}\\
\bitbox{6}{0x12}
\bitbox{5}{0x10}
\bitbox{5}{rs}
\bitbox{5}{cb}
\bitbox{5}{rd}
\bitbox{2}{\color{lightgray}\rule{\width}{\height}}
\bitbox{2}{00}
\bitbox{2}{t}
\end{bytefield}
\end{center}

\usesDDCinsteadofNULL{cb}

\subsubsection*{Semantics}
\sailMIPScode{CStoreConditional}



\clearpage
\phantomsection
\addcontentsline{toc}{subsection}{CSCC}
\insnmipslabel{cscc}
\subsection*{CSCC: Store Conditional Capability via Capability}

\subsubsection*{Format}

CSCC rd, cs, cb


\begin{center}
\begin{bytefield}{32}
\bitheader[endianness=big]{0,2,3,5,6,10,11,15,16,20,21,25,26,31}\\
\bitbox{6}{0x12}
\bitbox{5}{0x10}
\bitbox{5}{cs}
\bitbox{5}{cb}
\bitbox{5}{rd}
\bitbox{2}{\color{lightgray}\rule{\width}{\height}}
\bitbox{4}{0111}
\end{bytefield}
\end{center}
\usesDDCinsteadofNULL{cb}

% \textbf{TO DO: Describe what this instruction does.}

\subsubsection*{Semantics}
\sailMIPScode{CSCC}

\subsubsection*{Exceptions}

A coprocessor 2 exception is raised if:

\begin{itemize}
\item
\emph{cb}.\ctag{} is not set.
\item
\emph{cb} is sealed.
\item
\emph{cb}.\cperms.\emph{Permit\_Store} is not set.
\item
\emph{cb}.\cperms{}.\emph{Permit\_Store\_Capability} is not set.
\item
\emph{cb}.\cperms{}.\emph{Permit\_Store\_Local\_Capability} is not set
and \emph{cs}.\cperms.\emph{Global} is not set.
\item
\emph{addr} $+$ \emph{capability\_size} $>$
\emph{cb}.\cbase{} $+$ \emph{cb}.\clength{}
\item
\emph{addr} $<$ \emph{cb}.\cbase{}
\end{itemize}

A TLB Store exception is raised if:

\begin{itemize}
\item
The \emph{S} bit in the TLB entry corresponding to virtual address
\emph{addr} is not set.
\end{itemize}

An address error during store (AdES) exception is raised if:

\begin{itemize}
\item
\emph{addr} is not correctly aligned.
\end{itemize}

\clearpage
\phantomsection
\addcontentsline{toc}{subsection}{CSeal}
\insnmipslabel{cseal}
\subsection*{CSeal: Seal a Capability}

\subsubsection*{Format}

CSeal cd, cs, ct

\begin{center}
\cherithreeop[header]{0xb}{cd}{cs}{ct}
\end{center}

\subsubsection*{Description}

Capability register \emph{cs} is sealed
\psnote{it's confusing phrasing to say the \emph{register} itself is sealed, as
opposed to the value the register. Instead, perhaps ``The capability
 in register \emph{cs} is sealed''?  I guess this idiom may occur in
 many places, so I've not just fixed this one. Robert N-W?}
 with an \cotype{} of
\emph{ct}.\cbase{} $+$ \emph{ct}.\coffset{}
and the result is placed in \emph{cd}:

\begin{itemize}
\item
\emph{cd}.\cotype{} is set to \emph{ct}.\cbase{} + \emph{ct}.\coffset{};
\item
\emph{cd} is sealed;
\item
and the other fields of \emph{cd} are copied from \emph{cs}.
\end{itemize}

\emph{ct} must grant \emph{Permit\_Seal} permission, and the new \cotype{}
of \emph{cd} must be between \emph{ct}.\cbase{} and \emph{ct}.\cbase{} $+$
\emph{ct}.\clength{} $-$ 1.

\subsubsection*{Semantics}
\sailMIPScode{CSeal}

\subsubsection*{Exceptions}

A coprocessor 2 exception is raised if:

\begin{itemize}
\item
\emph{cs}.\ctag{} is not set.
\item
\emph{ct}.\ctag{} is not set.
\item
\emph{cs} is sealed.
\item
\emph{ct} is sealed.
\item
\emph{ct}.\cperms.\emph{Permit\_Seal} is not set.
% \item
% \emph{ct}.\cperms.\emph{Permit\_Execute} is not set.
\item
\emph{ct}.\coffset{} $\ge$ \emph{ct}.length{}
\item
\emph{ct}.\cbase{} $+$ \emph{ct}.\coffset{} $> \emph{max\_otype}$
\item
The bounds of \emph{cb} cannot be represented exactly in a sealed capability.
\end{itemize}

\subsubsection*{Notes}

\clearpage
\phantomsection
\addcontentsline{toc}{subsection}{CSealEntry}
\insnmipslabel{csealentry}
\subsection*{CSealEntry: Construct a Sentry Capability}

\subsubsection*{Format}

CSealEntry cd, cs

\begin{center}
\cheritwoop[header]{0x1d}{cd}{cs}
\end{center}

\subsubsection*{Description}

Constructs a sentry capability from the unsealed, valid,
Permit\_Execute-bearing capability in register \emph{cs} and places the
result into capability register \emph{cd}.  Recall \cref{sec:arch-sentry}.

\subsubsection*{Semantics}
\sailMIPScode{CSealEntry}

\subsubsection*{Exceptions}
A coprocessor 2 exception is raised if:

\begin{itemize}
\item
\emph{cs}.\ctag{} is not set.
\item
\emph{cs} is sealed.
\item
\emph{cs}.\cperms.\emph{Permit\_Execute} is not set.
\item
The bounds of \emph{cs} cannot be represented exactly in a sealed capability.
\end{itemize}


\clearpage
\phantomsection
\addcontentsline{toc}{subsection}{CSetAddr}
\insnmipslabel{csetaddr}
\subsection*{CSetAddr: Set the Address of Capability}

\subsubsection*{Format}

CSetAddr cd, cb, rt

\begin{center}
\cherithreeop[header]{0x22}{cd}{cb}{rt}
\end{center}

\subsubsection*{Description}

\emph{cd} is set to \emph{cb} with \emph{cb}.\caddr{} set to \emph{rt}.
If changing the address causes the capability to become unrepresentable, then an untagged capability with the requested address is returned.

\subsubsection*{Semantics}

\sailMIPScode{CSetAddr}

\subsubsection*{Exceptions}

A coprocessor 2 exception is raised if:

\begin{itemize}
\item
\emph{cb}.\ctag{} is set and \emph{cb} is sealed.
\end{itemize}

\subsubsection*{Notes}

\begin{itemize}
\item This instruction may be useful, in combination with \insnmipsref{CGetAddr}, when C is manipulating pointers in ways that require a round trip through integer registers.
\item This instruction is also useful for \ccode{uintptr\_t} arithmetic when using an address interpretation of capabilities. When interpreting \ccode{uintptr\_t} as offsets relative
to the base, the compiler will use \insnmipsref{CGetOffset} and \insnmipsref{CSetOffset} instead.

\end{itemize}

\clearpage
\phantomsection
\addcontentsline{toc}{subsection}{CSetBounds}
\insnmipslabel{csetbounds}
\subsection*{CSetBounds: Set Bounds}

\subsubsection*{Format}

CSetBounds cd, cb, rt

\begin{center}
\cherithreeop[header]{0x8}{cd}{cb}{rt}
\end{center}

\subsubsection*{Description}

Capability register \emph{cd} is replaced with a capability that:

\begin{itemize}
\item
Grants access to a subset of the addresses authorized by \emph{cb}.
That is, \emph{cd}.\cbase{} $\ge$ \emph{cb}.\cbase{} and
\emph{cd}.\cbase{} $+$ \emph{cd}.\clength{} $\le$ \emph{cb}.\cbase{} $+$
\emph{cb}.\clength{}.
\item
Grants access to at least the addresses \emph{cb}.\cbase{} $+$
\emph{cb}.\coffset{} $\ldots$ \emph{cb}.\cbase{} $+$ \emph{cb}.\coffset{}
$+$ \emph{rt} $-$ 1.
That is, \emph{cd}.\cbase{} $\le$ \emph{cb}.\cbase{}
$+$ \emph{cb}.\coffset{} and \emph{cd}.\cbase{} $+$ \emph{cd}.\clength{}
$\ge$ \emph{cb}.\cbase{} $+$ \emph{cb}.\coffset{} $+$ \emph{rt}.
\item
Has an \coffset{} that points to the same memory location as \emph{cb}'s
\coffset{}.
That is, \emph{cd}.\coffset{} = \emph{cb}.\coffset{} + \emph{cb}.\cbase{} -
\emph{cd}.\cbase{}.
\item
Has the same \cperms{} as \emph{cb}, that is, \emph{cd}.\cperms{} = \emph{cb}.\cperms{}.
\end{itemize}

%When the hardware uses a 256-bit representation for capabilities, the bounds
%of the destination capability \emph{cd} are exactly as requested.
With compressed capabilities, not all combinations of \cbase{} and \clength{}
are representable.
\emph{cd} may therefore grant access to a range of memory addresses that is
wider than requested, but is still guaranteed to be within the bounds of
\emph{cb}.
\rwnote{Check that the following statement is true.}
This cannot occur if the requested bounds have been suitably aligned and
padded using the \insnmipsref{CRAM} and \insnmipsref{CRRL} instructions.
If software is not guaranteed to provide suitable alignment and padding, it
may be desirable to use \insnmipsref{CSetBoundsExact} so that an exception will
be thrown the requested bounds cannot be represented.

\subsubsection*{Semantics}

\sailMIPScode{CSetBounds}

\subsubsection*{Exceptions}

A coprocessor 2 exception is raised if:

\begin{itemize}
\item
\cchecktag{}
\item
\emph{cb} is sealed.
\item
\emph{cursor} $<$ \emph{cb}.\cbase{}
\item
\emph{cursor} $+$ \emph{rt} $>$ \emph{cb}.\cbase{} $+$ cb.\clength{}
\end{itemize}

\subsubsection*{Notes}

\begin{itemize}
\item
In the above Sail code, arithmetic is over the mathematical integers and
\emph{rt} is unsigned, so a large value of \emph{rt} cannot cause
\emph{cursor} $+$ \emph{rt} to wrap around and be less than \emph{cb}.\cbase{}.
Implementations (that, for example, will probably use a fixed number of
bits to store values) must handle this overflow case correctly.
\end{itemize}

\clearpage
\phantomsection
\addcontentsline{toc}{subsection}{CSetBoundsExact}
\insnmipslabel{csetboundsexact}
\subsection*{CSetBoundsExact: Set Bounds Exactly}

\subsubsection*{Format}

CSetBoundsExact cd, cb, rt

\begin{center}
\cherithreeop[header]{0x9}{cd}{cb}{rt}
\end{center}

\subsubsection*{Description}

Capability register \emph{cd} is replaced with a capability with its \cbase{}
replaced with \emph{cb}.\cbase{} $+$ \emph{cb}.\coffset{}, \clength{} set to
\emph{rt}, and \coffset{} set to zero.
When capability compression is in use, an exception is thrown if the requested
bounds cannot be represented exactly.

\subsubsection*{Semantics}
\sailMIPScode{CSetBoundsExact}

An exception cannot occur if the requested bounds have been suitably aligned
and padded using the \insnmipsref{CRAM} and \insnmipsref{CRRL} instructions.
If looser bounds, rather than exception, are desired, then it may be
preferable to use \insnmipsref{CSetBounds}.

\subsubsection*{Exceptions}

A coprocessor 2 exception is raised if:

\begin{itemize}
\item
\cchecktag{}
\item
\emph{cb} is sealed.
\item
\emph{cursor} $<$ \emph{cb}.\cbase{}
\item
\emph{cursor} $+$ \emph{rt} $>$ \emph{cb}.\cbase{} $+$ cb.\clength{}
\item
The requested bounds cannot be represented exactly.
\end{itemize}

\subsubsection*{Notes}

\begin{itemize}
\item
In the above Sail code, arithmetic is over the mathematical integers and
\emph{rt} is unsigned, so a large value of \emph{rt} cannot cause
\emph{cursor} $+$ \emph{rt} to wrap around and be less than \emph{cb}.\cbase{}.
Implementations (that, for example, will probably use a fixed number of
bits to store values) must handle this overflow case correctly.
\end{itemize}

\clearpage
\phantomsection
\addcontentsline{toc}{subsection}{CSetBoundsImm}
\insnmipslabel{csetboundsimm}
\subsection*{CSetBoundsImm: Set Bounds (Immediate)}

\subsubsection*{Format}

CSetBounds cd, cb, length$_{imm}$

\begin{center}
\begin{bytefield}{32}
\bitheader[endianness=big]{0,10,11,15,16,20,21,25,26,31}\\
\bitbox{6}{0x12}
\bitbox{5}{0x14}
\bitbox{5}{cd}
\bitbox{5}{cb}
\bitbox{11}{length$_{imm}$}
\end{bytefield}
\end{center}

\arnote{The assembler supports both CSetBounds and CSetBounds but I think we should always use CSetBoundsImm}

\subsubsection*{Description}

Capability register \emph{cd} is replaced with a capability that:

\begin{itemize}
\item
Grants access to a subset of the addresses authorized by \emph{cb}.
That is, \emph{cd}.\cbase{} $\ge$ \emph{cb}.\cbase{} and
\emph{cd}.\cbase{} $+$ \emph{cd}.\clength{} $\le$ \emph{cb}.\cbase{} $+$
\emph{cb}.\clength{}.
\item
Grants access to at least the addresses \emph{cb}.\cbase{} $+$
\emph{cb}.\coffset{} $\ldots$ \emph{cb}.\cbase{} $+$ \emph{cb}.\coffset{}
$+$ \emph{length$_{imm}$} $-$ 1.
That is, \emph{cd}.\cbase{} $\le$ \emph{cb}.\cbase{}
$+$ \emph{cb}.\coffset{} and \emph{cd}.\cbase{} $+$ \emph{cd}.\clength{}
$\ge$ \emph{cb}.\cbase{} $+$ \emph{cb}.\coffset{} $+$ \emph{length$_{imm}$}.
\item
Has an \coffset{} that points to the same memory location as \emph{cb}'s
\coffset{}.
That is, \emph{cd}.\coffset{} = \emph{cb}.\coffset{} + \emph{cb}.\cbase{} -
\emph{cd}.\cbase{}.
\item
Has the same \cperms{} as \emph{cb}, that is, \emph{cd}.\cperms{} = \emph{cb}.\cperms{}.
\end{itemize}

%When the hardware uses a 256-bit representation for capabilities, the bounds
%of the destination capability \emph{cd} are exactly as requested.

With compressed capabilities, not all combinations of \cbase{} and \clength{}
are representable.
\emph{cd} may therefore grant access to a range of memory addresses that is
wider than requested, but is still guaranteed to be within the bounds of
\emph{cb}.

\subsubsection*{Semantics}
\sailMIPScode{CSetBoundsImmediate}

\subsubsection*{Exceptions}

A coprocessor 2 exception is raised if:

\begin{itemize}
\item
\cchecktag{}
\item
\emph{cb} is sealed.
\item
\emph{cursor} $<$ \emph{cb}.\cbase{}
\item
\emph{cursor} $+$ \emph{length$_{imm}$} $>$ \emph{cb}.\cbase{} $+$ cb.\clength{}
\end{itemize}

\subsubsection*{Notes}

\begin{itemize}
\item
In the above Sail code, arithmetic is over the mathematical integers and
\emph{length$_{imm}$} is unsigned, so a large value of \emph{length$_{imm}$} cannot cause
\emph{cursor} $+$ \emph{length$_{imm}$} to wrap around and be less than \emph{cb}.\cbase{}.
Implementations (that, for example, will probably use a fixed number of
bits to store values) must handle this overflow case correctly.
\item If this instruction is used with \creg{0} as the destination register, it can be used to assert that a given capability grants access to at least \emph{length$_{imm}$} bytes. An assembler pseudo instruction \insnmipsref{CAssertInBounds} is supported for this use case.
\end{itemize}

\clearpage
\phantomsection
\addcontentsline{toc}{subsection}{CSetCause}
\insnmipslabel{csetcause}
\subsection*{CSetCause: Set the Capability Exception Cause Register}

\subsubsection*{Format}

CSetCause rt

\begin{center}
\cherioneop[header]{0x2}{rd}
\end{center}

\subsubsection*{Description}

The capability cause register value is set to the low 16 bits of integer
register \textit{rt}.

\jhbnote{Should this instruction be removed?  It was only
  used/relevant for CCall selector 0.}

\subsubsection*{Semantics}

\sailMIPScode{CSetCause}

\subsubsection*{Exceptions}

A coprocessor 2 exception is raised if:

\begin{itemize}
\item
\PCC{}.\cperms{}.\emph{Access\_System\_Registers} is not set.
\end{itemize}

\subsubsection*{Notes}

\begin{itemize}
\item
\insnmipsref{CSetCause} does not cause an exception to be raised (unless
the permission check for \emph{Access\_System\_Registers} fails).
\end{itemize}

\clearpage
\phantomsection
\addcontentsline{toc}{subsection}{CSetCID}
\insnmipslabel{csetcid}
\subsection*{CSetCID: Set the Architectural Compartment ID}

\subsubsection*{Format}

CSetCID cb

%
% XXXRW: Format and opcode still TODO.
\begin{center}
% XXXAR: 0x5 is currently the second available one-operand instruction
\cherioneop[header]{0x5}{cb}
\\
\arnote{This encoding is not final -- do not implement}
\end{center}

\subsubsection*{Description}

Set the architectural Compartment ID (CID) to \emph{cb}.\cbase{} +
\emph{cb}.\coffset{} if \emph{cb} has the Permit\_Set\_CID permission and
\emph{cb}.\coffset{} is in range.
The CID can then be used by the microarchitecture to tag microarchitectural
state.
CIDs can be utilized in a similar style as ASID matching in TLBs to
determine in what context microarchitectural state can be used.
Typical use will be to prevent sharing where it could otherwise be used as a
high-bandwidth microarchitectural side channel between compartments with
confidentiality requirements -- for example, to limit the impact of
Spectre-style attacks~\cite{Kocher2018spectre}.

\subsubsection*{Semantics}

\sailMIPScode{CSetCID}

\subsubsection*{Exceptions}

A coprocessor 2 exception is raised if:

\begin{itemize}
\item
\emph{cb}.\cperms.\emph{Permit\_Set\_CID} is not set.
\item
\emph{addr} + 1 $>$ \emph{cb}.\cbase{} $+$ \emph{cb}.\clength{}.
\item
\emph{addr} $<$ \emph{cb}.\cbase{}.
\end{itemize}

\subsubsection*{Notes}

\begin{itemize}
\item
  The CID can be queried using the \insnmipsref{CGetCID} instruction.
\item
  Although \insnmipsref{CSetCID} has no architectural side effects other than
  setting an integer register with a compartment ID, the intent is that the
  microarchitecture can be made aware of boundaries across which
  microarchitectural side channels are less acceptable.
  A key design goal for \insnmipsref{CSetCID} is to provide flexible
  mechanism above which a range of software policies might be implemented.
\item
  For example, the software supervisor might arrange that all compartments
  have unique CIDs such that branch-predictor state cannot be shared.
  Other policies might use the same CID for compartments between which strong
  confidentiality requirements are not present -- e.g., where only integrity
  or availability protection is required.
\item
  We have chosen not to protect the architectural CID using
  Access\_System\_Registers in order to support virtualizability of the domain
  switcher -- and, in particular, to not require Access\_System\_Registers to
  implement a domain switcher.
  A new permission is used, together with bounds checks, such that ranges of
  CIDs can be delegated when multiple domain switchers are in use.
  For example, a set of CIDs might be reserved for domain-switch
  implementations themselves, and then subranges delegated to individual
  language runtimes or processes within the same address space.
  Note that such a model could obligate two CID operations per domain switch
  involving mutual distrust: one into the domain switcher, and a second out,
  in order to not just protect the two endpoint domains from one another but
  also the switcher.
\item
  How to ensure that \insnmipsref{CSetCID} is not speculated past (e.g., in
  the case of microarchitectural side-channel attacks such as Spectre) is a
  critical question.
  We recommend that \insnmipsref{CSetCID} be considered serialising, and that
  the CID be set immediately on switcher entry, as well as again on switcher
  exit.
\item
  An alternative design choice would accept an integer general-purpose
  register operand, \emph{rt}, as a second argument specifying the CID to
  switch to.
  This might be more consistent with the behavior of \insnmipsref{CGetCID},
  but also consume more opcode space.
\end{itemize}

\clearpage
\phantomsection
\addcontentsline{toc}{subsection}{CSetFlags}
\insnmipslabel{csetflags}
\subsection*{CSetFlags: Set Flags}

\subsubsection*{Format}

CSetFlags cd, cb, rt

\begin{center}
\cherithreeop[header]{0xe}{cd}{cs}{rt}
\end{center}

\subsubsection*{Description}

Capability register \emph{cd} is replaced with the contents of capability
register \emph{cb} with the \cflags{} field set to bits 0 .. \emph{max\_flags} of
integer register \emph{rd}.

\pdrnote{We should define max\_flags (although max\_otype is not defined either)}

\subsubsection*{Semantics}
\sailMIPScode{CSetFlags}

\subsubsection*{Exceptions}

A coprocessor 2 exception is raised if:

\begin{itemize}
\item
\emph{cb} is sealed.
\end{itemize}

\clearpage
\phantomsection
\addcontentsline{toc}{subsection}{CSetOffset}
\insnmipslabel{csetoffset}
\subsection*{CSetOffset: Set Cursor to an Offset from Base}

\subsubsection*{Format}

CSetOffset cd, cs, rt

\begin{center}
\cherithreeop[header]{0xf}{cd}{cs}{rt}
\end{center}

\subsubsection*{Description}

Capability register \emph{cd} is replaced with the contents of capability
register \emph{cs} with the \coffset{} field set to the contents of integer
register \textit{rt}.

If capability compression is in use, and the requested \cbase{}, \clength{}
and \coffset{} cannot be represented exactly, then \emph{cd}.\ctag{} is
cleared, \emph{cd}.\cbase{} and \emph{cd}.\clength{} are set to zero,
\emph{cd}.\cperms{} is cleared and \emph{cd}.\coffset{} is set equal to
\emph{cs}.\cbase $+$ \emph{rt}.

\subsubsection*{Semantics}

\sailMIPScode{CSetOffset}

\subsubsection*{Exceptions}

A coprocessor 2 exception is raised if:

\begin{itemize}
\item
\emph{cs}.\ctag{} is set and \emph{cs} is sealed.
\end{itemize}

\subsubsection*{Notes}

\begin{itemize}
\item
\insnmipsref{CSetOffset} can be used on a capability register whose tag bit
is not set. This can be used to store an integer value in a capability register,
and is useful when implementing a variable that is a union of a capability
and an integer (\ccode{intcap\_t} in C).
% The in-memory representation
% that will be used if the capability register is stored to memory might
% be surprising to some users (with the 256-bit representation of capabilities,
% \cbase{} $+$ \coffset{} is stored in the
% {\bf cursor} field in memory) and may change if the memory representation of
% capabilities changes, so compilers should not rely on it.
\item
With compressed capabilities, the requested offset may not not
representable.
In this case, the result preserves the requested \cbase{} $+$ \coffset{}
(i.e., the cursor) rather than the architectural field \coffset{}. This
field is mainly useful for debugging what went wrong (the capability cannot
be dereferenced, as \ctag{} has been cleared), and for debugging we considered
it more useful to know what the requested capability would have referred to
rather than its \coffset{} relative to a \cbase{} that is no longer available.
This has the disadvantage that it exposes the value of \cbase{} to a program,
but \cbase{} is not a secret and can be accessed by other means. The
main reason for not exposing \cbase{} to programs is so that a garbage
collector can stop the program, move memory, modify the capabilities and
restart the program. A capability with \ctag{} cleared cannot be dereferenced,
and so is not of interest to a garbage collector, and so it doesn't matter
if it exposes \cbase{}.
\end{itemize}

\clearpage
\phantomsection
\addcontentsline{toc}{subsection}{CSub}
\insnmipslabel{csub}
\subsection*{CSub: Subtract Capabilities}

\subsubsection*{Format}

CSub rd, cb, ct

\begin{center}
\cherithreeop[header]{0xa}{rt}{cb}{ct}
\end{center}

\subsubsection*{Description}

Register \emph{rd} is set equal to (\emph{cb}.\ccursor{} $-$ \emph{ct}.\ccursor{}) $\bmod~2^{64}$.

\subsubsection*{Semantics}

\sailMIPScode{CSub}

\subsubsection*{Notes}

\begin{itemize}
\item
\insnmipsref{CSub} can be used to implement C-language pointer subtraction,
or subtraction of \ccode{intcap\_t}.
\item
Like \insnmipsref{CIncOffset}, \insnmipsref{CSub} can be used on either
valid capabilities (\ctag{} set) or on integer values stored in capability
registers (\ctag{} not set).
\item
If a copying garbage collector is in use, pointer subtraction must be
implemented with an atomic operation (such as \insnmipsref{CSub}).
Implementing pointer subtraction with a non-atomic sequence of operations such
as \insnmipsref{CGetOffset} has the risk that the garbage collector will
relocate an object part way through, giving incorrect results for the
pointer difference. If \emph{cb} and \emph{ct} are both pointers into the
same object, then a copying garbage collector will either relocate both of
them or neither of them, leaving the difference the same.
If \emph{cb} and \emph{ct} are pointers into
different objects, the result of the subtraction is not defined by the ANSI
C standard, so it doesn't matter if this difference changes as the garbage
collector moves objects.
\end{itemize}

\clearpage
\phantomsection
\addcontentsline{toc}{subsection}{CToPtr}
\insnmipslabel{ctoptr}
\subsection*{CToPtr: Capability to Integer Pointer}

\subsubsection*{Format}

CToPtr rd, cb, ct

\begin{center}
\cherithreeop[header]{0x12}{rd}{cb}{ct}
\end{center}

\usesDDCinsteadofNULL{ct}

\subsubsection*{Description}

If \emph{cb} has its tag bit unset (i.e. it is either the
NULL capability, or contains some other non-capability data),
then \emph{rd} is set to zero. Otherwise, \emph{rd} is set to
\emph{cb}.\cbase{} $+$ \emph{cb}.\coffset{} - \emph{ct}.\cbase{}

This instruction can be used to convert a capability into a pointer that
uses the C language convention that a zero value represents the NULL pointer.
Note that \emph{rd} will also be zero if \emph{cb}.\cbase{} $+$
\emph{cb}.\coffset{} $=$ \emph{ct}.\cbase{};
this is similar to the C language not being able to distinguish a
NULL pointer from a pointer to a structure at address 0.
\arnote{Should we just return the address for untagged values to handle cases such as \ccode{(void* __capability)-1)}?}

% XXXRW: Address notes:
%
%\note{dc552}{We need to output 0 if the input capability is out of bounds for the
%gsl::span and related bounds checking to work.  I have added the required check
%commented out.}
% \arnote{why can't we just use the capability directly in hybrid mode?}
%
%\note{dc552}{The names cb and ct are confusing for this instruction.  cb is not
%the base register and ct is not the target.}

\arnote{The operand order here is inconsistent with cfromptr/cbuildcap: the authorizing capability is the third operand instead of the second.}

\subsubsection*{Semantics}

\sailMIPScode{CToPtr}

\subsubsection*{Exceptions}

A coprocessor 2 exception will be raised if:

\begin{itemize}
\item
\emph{ct}.\ctag{} is not set.
\end{itemize}

\subsubsection*{Notes}

\begin{itemize}
\item
\emph{ct} being sealed will not cause an exception to be raised.
This is for further study.
\item
This instruction has two different means of returning an error code:
raising an exception (if \emph{ct}.\ctag{} is not set, or the registers
are not accessible) and returning a NULL pointer if \emph{cb}.\ctag{}
is not set.
\item
If the range of \emph{cb} is outside the range of \emph{ct}, a pointer relative
to \emph{ct} can't always be used in place of \emph{cb}: some reads or writes
will fail because they are outside the range of \emph{ct}. To handle this case,
the application can use the \insnmipsref{CTestSubset} instruction
followed by a conditional move.

\item \insnmipsref{CGetAddr} similarly allows access to the sum of the base
and offset of the operand capability, but without the translation relative to the
authorizing capability or validity/sealed checks on the operand.
\end{itemize}

\clearpage
\phantomsection
\addcontentsline{toc}{subsection}{CUnseal}
\insnmipslabel{cunseal}
\subsection*{CUnseal: Unseal a Sealed Capability}

\subsubsection*{Format}

CUnseal cd, cs, ct

\begin{center}
\cherithreeop[header]{0xc}{cd}{cs}{ct}
\end{center}

\subsubsection*{Description}

The sealed capability in $cs$ is unsealed with $ct$ and the result placed
in $cd$. The global bit of $cd$ is the AND of the global bits of
$cs$ and $ct$. $ct$ must be unsealed, have \emph{Permit\_Unseal} permission, and $ct$.\cbase{} + $ct$.\coffset{} must equal $cs$.\cotype{}.

\subsubsection*{Semantics}

\sailMIPScode{CUnseal}

\subsubsection*{Exceptions}

A coprocessor 2 exception is raised if:

\begin{itemize}
\item
\emph{cs}.\ctag{} is not set.
\item
\emph{ct}.\ctag{} is not set.
\item
\emph{cs} is not sealed.
\item
\emph{ct} is sealed.
\item
\emph{ct}.\coffset{} $\ge$ \emph{ct}.\clength{}
\item
\emph{ct}.\cperms{}.\emph{Permit\_Unseal} is not set.
\item
\emph{ct}.\cbase{} $+$ \emph{ct}.\coffset{} $\ne$ \emph{cs}.\cotype{}.
\end{itemize}

\subsubsection*{Notes}

\begin{itemize}
\item
There is no need to check if \emph{ct}.\cbase{} $+$ \emph{ct}.\coffset{}
$>$ \emph{max\_otype}, because this can't happen:
\emph{ct}.\cbase{} $+$ \emph{ct}.\coffset{} must equal \emph{cs}.\cotype{}
for the \cotype{} check to have succeeded, and there is no way
\emph{cs}.\cotype{} could have been set to a value that is out of range.
\end{itemize}

\clearpage
\phantomsection
\addcontentsline{toc}{subsection}{CWriteHwr}
\insnmipslabel{cwritehwr}
\subsection*{CWriteHwr: Write a Special-Purpose Capability Register}

\subsubsection*{Format}

CWriteHwr cb, selector

\begin{center}
\cheritwoop[header]{0xe}{cb}{selector}
\end{center}

\subsubsection*{Description}

The value of the capability register \emph{cb} is stored in the special-purpose
capability register \emph{selector}.
See \autoref{tab:cwritehwr-permissions} for the possible values of
\emph{selector} and the permissions required in order to write to the register.

\begin{table}[h]
\centering
\caption{Access permission required to write special-purpose capability registers}
\label{tab:cwritehwr-permissions}
\begin{tabular}{cll@{}}
\toprule
& \textbf{Register}                 & Required for write access                                                                                         \\
\midrule
\textbf{0}  & Default data capability (\DDC) & $\emptyset$  \\
\textbf{1} & User TLS (\CULR) & $\emptyset$ \\
%\textbf{2} & \ajnote{\PCC - make this CJR} & $\emptyset$ \\
% \textbf{7}  & Program counter capability (\PCC)   & Never writeable \\


\textbf{8}  & Privileged User TLS (\CPLR) & \PCC{}.\cperms{}.\emph{Access\_System\_Registers} \\

\textbf{22} & Kernel scratch register 1 (\KRC)  & \KernelAndAccessSysRegs \\
\textbf{23} & Kernel scratch register 2 (\KQC)  & \KernelAndAccessSysRegs \\

\textbf{28} & Error exception program counter (\ErrorEPCC)  &  \KernelAndAccessSysRegs \\
\textbf{29} & Kernel code capability (\KCC)     &  \KernelAndAccessSysRegs \\
\textbf{30} & Kernel data capability (\KDC)    & \KernelAndAccessSysRegs \\
\textbf{31} & Exception program counter (\EPCC) & \KernelAndAccessSysRegs \\
\bottomrule
\end{tabular}
\end{table}

\note{rmn30}{should we validate the written values, for example requiring that KCC and EPCC have tag set? otherwise we can get into weird states (I think this can happen on current hw).}

\subsubsection*{Semantics}
\sailMIPScode{CWriteHwr}

\subsubsection*{Exceptions}


A reserved Instruction exception is raised for unknown or
unimplemented values of \emph{selector}.

A coprocessor 2 exception is raised if:

\begin{itemize}
\item the permission checks as specified in \autoref{tab:cwritehwr-permissions} above were not met for \emph{selector}
\end{itemize}

\subsubsection*{Notes}

\begin{itemize}
\item In the future we may decide to require \PCC{}.\cperms{}.\emph{Access\_System\_Registers} in order to modify \DDC
\item We may decide to introduce a \insnnoref{CSwapHwr} instruction that
 swaps special-purpose register \emph{selector} and a general-purpose register
\end{itemize}



\clearpage
\section{Assembler Pseudo-Instructions}

For convenience, several pseudo-instructions are accepted by the assembler.
These expand to either single instructions or short sequences of instructions.

\subsection{CGetDefault, CSetDefault}
\insnmipslabel{cgetdefault}
\insnmipslabel{csetdefault}
\subsubsection*{Get/Set Default Capability}

\asm{CGetDefault} and \asm{CSetDefault} get and set the capability register that
is implicitly employed by the legacy MIPS load and store instructions.
In the current version of the ISA, this register is special-purpose capability
register 0.

In previous versions of the ISA, \DDC{} was register \creg{0} in the main
capability register file. In these versions of the architecture, using
\insnnoref{CSetDefault} rather than an capability operation with destination
\creg{0} allowd the Clang/LLVM compiler to know that the semantics of subsequent
MIPS loads and stores will be affected by the change to \DDC{}.

\begin{asmcode}
# The following are equivalent:
  CGetDDC $c1
  CGetDefault $c1
  CReadHWR $c1, $0
\end{asmcode}

\begin{asmcode}
# The following are equivalent:
  CSetDDC $c1
  CSetDefault $c1
  CWriteHWR $c1, $0
\end{asmcode}

\subsection{CGetEPCC, CSetEPCC}
\insnmipslabel{cgetepcc}
\insnmipslabel{csetepcc}
\subsubsection*{Get/Set Exception Program Counter Capability}

Pseudo-operations are provided for getting and setting \EPCC{}. In the current
ISA, EPCC is special-purpose capability register 31.

\begin{asmcode}
# The following are equivalent:
  CGetEPCC $c1
  CReadHWR $c1, $31
\end{asmcode}

\begin{asmcode}
# The following are equivalent:
  CSetEPCC $c1
  CWriteHWR $c1, $31
\end{asmcode}

\subsection{CGetKDC, CSetKDC}
\insnmipslabel{cgetkdc}
\insnmipslabel{csetkdc}
\subsubsection*{Get/Set Kernel Data Capability}

\begin{asmcode}
# The following are equivalent:
  CGetKDC $c1
  CReadHWR $30
\end{asmcode}

\begin{asmcode}
# The following are equivalent:
  CSetKDC $c1
  CWriteHWR $30
\end{asmcode}

\subsection{GGetKCC, CSetKCC}
\insnmipslabel{cgetkcc}
\insnmipslabel{csetkcc}
\subsubsection*{Get/Set Kernel Code Capability}

\begin{asmcode}
# The following are equivalent:
  CGetKCC $c1
  CReadHWR $29
\end{asmcode}

\begin{asmcode}
# The following are equivalent:
  CSetKCC $c1
  CWriteHWR $29
\end{asmcode}

\subsection{CAssertInBounds}
\insnmipslabel{cassertinbounds}
\subsubsection*{Assert that a capability in bounds}
This pseudo operation can be used to assert that a capability grants access to a given number of bytes (if the size argument is omitted, one byte is assumed).
This instruction only checks the bounds of the capability and ignores permissions. Therefore, an access might still be prohibited even if \insnnoref{CAssertInBounds} did not raise an exception.
\begin{asmcode}
# The following are equivalent (check that at least one byte is accessible):
  CAssertInBounds $c1
  CSetBoundsImm $cnull, $c1, 1
\end{asmcode}

\begin{asmcode}
# The following are equivalent (check that at least 10 bytes are accessible):
  CAssertInBounds $c1, 10
  CSetBoundsImm $cnull, $c1, 10
\end{asmcode}


\subsection{Capability Loads and Stores of Floating-Point Values}

The current revision of the CHERI ISA does not have CHERI-MIPS instructions for loading floating point values directly via capabilities.
MIPS does provide instructions for moving values between integer and floating point registers, so
a load or store of a floating point value via a capability
can be implemented in two instructions.

Four pseudo-instructions are defined to implement these patterns.
These are \asm{clwc1} and \asm{cldc1} for loading 32-bit and 64-bit floating point values, and \asm{cswc1} and \asm{csdc1} as the equivalent store operations.
The load operations expand as follows:
\begin{asmcode}
  cldc1    $f7, $zero, 0($c2)
  # Expands to:
  cld    $1, $zero, 0($c2)
  dmtc1    $1, $f7
\end{asmcode}

Note that integer register \regname{1} (\regname{at}) is used;
this pseudo-op is unavailable if the \asm{noat} directive is used.
The 32-bit variant (\asm{clwc1}) has a similar expansion, using \asm{clwu} and \asm{mtc1}.

The store operations are similar:

\begin{asmcode}
  csdc1    $f7, $zero, 0($c2)
  # Expands to:
  dmfc1    $1, $f7
  csd    $1, $zero, 0($c2)
\end{asmcode}

The specified floating point value is moved from the floating point register to \regname{at} and then stored using the correct-sized capability instruction.

\mrnote{If the floating point loads and stores were real instructions rather than
pseudo-ops, we would have to deal with the problem of what happens if a
capability exception or TLB exception is raised in the floating point pipeline.
The floating point and integer pipelines would need to be synchronized so that
the exception occurs at the correct point in program order.}
