\chapter{CHERI ISA Version History}
\label{app:versions}

This appendix contains both a high-level summary of prior CHERI ISA versions
(Section~\ref{sec:detailed-cheri-isa-version-change-history}),
and also a detailed change log for each version
(Section~\ref{sec:detailed-cheri-isa-version-change-history}).
This report was previously made available as the {\em CHERI Architecture
Document}, but is now the {\em CHERI Instruction-Set Architecture}.

\section{CHERI ISA Specification Version Summary}
\label{sec:cheri-isa-specification-version-summary}

A short summary of key ISA versions is presented here:

\begin{description}
\item[CHERI ISAv1 - 1.0--1.4 - 2010--2012]
  Early versions of the CHERI ISA explored the integration of capability
  registers and tagged memory -- first in isolation from, and later in
  composition with, MMU-based virtual memory.
  CHERI-MIPS instructions were targeted only by an extended assembler, with an
  initial microkernel (``Deimos'') able to create compartments on bare metal,
  isolating small programs from one another.
  Key early design choices included:
  \begin{itemize}
  \item to compose with the virtual-memory mechanism by being an
    in-address-space protection feature, supporting complete MMU-based OSes,
  \item to use capabilities to implement code and data pointers for C-language
    TCBs, providing reference-oriented, fine-grained memory protection and
    control-flow integrity,
  \item to impose capability-oriented monotonic non-increase on pointers to
    prevent privilege escalation,
  \item to target capabilities with the compiler using explicit capability
    instructions (including load, store, and jumping/branching),
  \item to derive bounds on capabilities from existing code and data-structure
    properties, OS policy, and the heap and stack allocators,
  \item to have both in-register and in-memory capability storage,
  \item to use a separate capability register file (to be consistent with the
    MIPS coprocessor extension model),
  \item to employ tagged memory to preserve capability integrity and
    provenance outside of capability registers,
  \item to enforce monotonicity through constrained manipulation instructions,
  \item to provide software-defined (sealed) capabilities including a
    ``sealed'' bit, user-defined permissions, and object types,
  \item to support legacy integer pointers via a Default Data Capability
    (\DDC{}),
  \item to extend the program counter (\PC{}) to be the Program-Counter
    Capability (\PCC{}),
  \item to support not just fine-grained memory protection, but also
    higher-level protection models such as software compartmentalization or
    language-based encapsulation.
  \end{itemize}

\item[CHERI ISAv2 - 1.5 - August 2012]
  This version of the CHERI ISA developed a number of aspects of capabilities
  to better support C-language semantics, such as introducing tags on
  capability registers to support capability-oblivious memory copying, as well
  as improvements to support MMU-based operating systems.

\item[UCAM-CL-TR-850 - 1.9 - June 2014]
  This technical report accompanied publication of our ISCA 2014 paper on
  CHERI memory protection.
  Changes from CHERI ISAv2 were significant, supporting a complete
  conventional OS (CheriBSD) and compiler suite (CHERI Clang/LLVM), a defined
  \insnnoref{CCall}/\insnnoref{CReturn} mechanism
  for software-defined
  object capabilities, capability-based load-linked/store-conditional
  instructions to support multi-threaded software, exception-handling
  improvements such as a CP2 cause register, new instructions
  \insnref{CToPtr} and \insnref{CFromPtr} to improve compiler
  efficiency for hybrid compilation, and changes relating to object
  capabilities, such as user-defined permission bits and instructions to check
  permissions/types.

\item[CHERI ISAv3 - 1.10 - September 2014]
  CHERI ISAv3 further converges C-language pointers and capabilities, improves
  exception-handling behavior, and continues to mature support for
  object capabilities.
  A key change is shifting from C-language pointers being represented by the
  base of a capability to having an independent ``offset'' (implemented as a
  ``cursor'') so that monotonicity is imposed only on bounds, and not on the
  pointer itself.
  Pointers are allowed to move outside of their defined bounds, but can be
  dereferenced only within them.
  There is also a new instruction for C-language pointer comparison
  (\insnnoref{CPtrCmp}), and a NULL capability has been defined
  as having
  an in-memory representation of all zeroes without a tag, ensuring that BSS
  (pre-zeroed memory) operates without change.
  The offset behavior is also propagated into code capabilities, changing the
  behavior of \PCC{}, \EPCC{}, \insnref{CJR}, \insnref{CJALR}, and
  several aspects of exception handling.
  The sealed bit was moved out of the permission mask to be a stand-alone bit
  in the capability, and we went from independent \insnnoref{CSealCode}
  and \insnnoref{CSealData} instructions to a single \insnref{CSeal}
  instruction, and the \insnnoref{CSetType} instruction has been removed.
  While the object type originates as a virtual address in an authorizing
  capability, that interpretation is not mandatory due to use of a separate
  hardware-defined permission for sealing.

\item[UCAM-CL-TR-864 - 1.11 - January 2015]
  This technical report refines CHERI ISAv3's convergence of C-language
  pointers and capabilities; for example, it adds a \insnref{CIncOffset}
  instruction that avoids read-modify-write accesses to adjust the offset
  field, as well as exception-handling improvements.
  TLB permission bits relating to capabilities now have modified semantics:
  if the load-capability bit is not present, than capability tags are stripped
  on capability loads from a page, whereas capability stores trigger an
  exception, reflecting the practical semantics found most useful in our
  CheriBSD prototype.

\item[CHERI ISAv4 / UCAM-CL-TR-876 - 1.15 - November 2015]
  This technical report describes \\
  CHERI ISAv4, introducing concepts required
  to support 128-bit compressed capabilities.
  A new \insnref{CSetBounds} instruction is added, allowing adjustments
  to both lower and upper bounds to be simultaneously exposed to the hardware,
  providing more information when making compression choices.
  Various instruction definitions were updated for the potential for
  imprecision in bounds.
  New chapters were added on the protection model, and how CHERI features
  compose to provide stronger overall protection for secure software.
  Fast register-clearing instructions are added to accelerate domain switches.
  A full set of capability-based load-linked, store-conditional instructions
  are added, to better support multi-threaded pure-capability programs.

\item[CHERI ISAv5 / UCAM-CL-TR-891 - 1.18 - June 2016]
  CHERI ISAv5 primarily serves to introduce the CHERI-128 compressed
  capability model, which supersedes prior candidate models.
  A new instruction, \insnnoref{CGetPCCSetOffset}, allows jump targets to
  be more efficiently calculated relative to the current \PCC{}.
  The previous multiple privileged capability permissions authorizing access
  to exception-handling state has been reduced down to a single system
  privilege to reduce bit consumption in capabilities, but also to recognize
  their effective non-independence.
  In order to reduce code-generation overhead, immediates to
  capability-relative loads and stores are now scaled.

\item[CHERI ISAv6 / UCAM-CL-TR-907 - 1.20 - April 2017]
  CHERI ISAv6 introduces support for kernel-mode compartmentalization,
  jump-based rather than exception-based domain transition,
  architecture-abstracted and efficient tag restoration, and more efficient
  generated code.
  A new chapter addresses potential applications of the CHERI protection model
  to the RISC-V and x86-64 ISAs, previously described relative only to the
  64-bit MIPS ISA.
  CHERI ISAv6 better explains our design rationale and research methodology.

\item[CHERI ISAv7 / UCAM-CL-TR-927 - 7.0 - June 2019]
  We more clearly differentiate an archi\-tecture-neutral CHERI protection
  model vs. architecture-specific instantiations in 64-bit MIPS, 64-bit
  RISC-V, and x86-64.
  We have defined a new capability compression scheme, CHERI Concentrate, and
  deprecated the previous CHERI-128 scheme.
  CHERI-MIPS now supports special-purpose capability registers, which have
  been moved out of the numbered general-purpose capability register space.
  New special-purpose capability registers, including those for thread-local
  storage, have been defined.
  CHERI-RISC-V is more substantially elaborated.
  A new compartment-ID register assists in resisting microarchitectural
  side-channel attacks.
  New optimized instructions with immediate fields improve the performance of
  generated code.
  Experimental 64-bit capabilities have been defined for 32-bit architectures,
  as well as instructions to accelerate spatial and temporal memory safety.
  The opcode reencoding begun in prior CHERI ISA specification versions has
  now been completed.

\item[CHERI ISAv8 / UCAM-CL-TR-951 - 8.0 - October 2020]
  Capability compression is now part of the abstract model.
  Both 32-bit and 64-bit architectural address sizes are supported.
  Various previously experimental features, such as sentry capabilities and
  CHERI-RISC-V, are now considered mature. We have defined a number of new
  temporal memory-safety acceleration features including MMU assistance for a
  load-side-barrier revocation model.
  We have added a chapter on practical CHERI microarchitecture.
  CHERI ISAv8 is synchronized with Arm Morello.

\item[CHERI ISAv9 / UCAM-CL-TR-987 - 9.0 - September 2023]
  CHERI-RISC-V has replaced CHERI-MIPS as the primary reference
  platform, and CHERI-MIPS has been removed from the specification.
  CHERI architectures now always use merged register files where
  existing general-purpose registers are extended to support
  capabilities.
  CHERI architectures have adopted two design decisions from Arm
  Morello: 1) CHERI architectures now clear tags rather than raising
  exceptions if an instruction attempts a non-monotonic modification
  of a capability; and 2) \DDC{} and \PCC{} no longer relocate legacy
  memory accesses by default.
  CHERI-RISC-V has received numerous updates to serve as a better
  baseline for an upstream standard proposal including a more mature
  definition of compressed instructions in capability mode.
  CHERI-x86-64 now includes details of extensions to existing x86
  instructions and proposed new instructions in a separate ISA
  reference chapter along with various other updates.

\end{description}

\section{Detailed CHERI ISA Version Change History}
\label{sec:detailed-cheri-isa-version-change-history}

\begin{description}
\item[1.0] This first version of the CHERI architecture document was prepared
  for a six-month deliverable to DARPA.
  It included a high-level architectural description of CHERI, motivations
  for our design choices, and an early version of the capability instruction
  set.

\item[1.1] The second version was prepared in preparation for a meeting of the
  CTSRD External Oversight Group (EOG) in Cambridge during May 2011.
  The update followed a week-long meeting in Cambridge, UK, in which many
  aspects of the CHERI architecture were formalized, including
  details of the capability instruction set.

\item[1.2] The third version of the architecture document came as the first
  annual reports from the CTSRD project were in preparation, including a
  decision to break out formal-methods appendices into their own {\em CHERI
  Formal Methods Report} for the first time.
  With an in-progress prototype of the CHERI capability unit, we
  significantly refined the CHERI ISA with respect to object capabilities, and
  matured notions such as a trusted stack and the role of an
  operating system supervisor.
  The formal methods portions of the document was dramatically
  expanded, with proofs of correctness for many basic security properties.
  Satisfyingly, many `future work' items in earlier versions of the report
  were becoming completed work in this version!

\item[1.3] The fourth version of the architecture document was released
  while
  the first functional CHERI prototype was in testing.  It reflects on
  initial experiences adapting a microkernel to exploit CHERI capability
  features.
  This led to minor architectural refinements, such as improvements to
  instruction opcode layout, some additional instructions (such as allowing
  \insnref{CGetPerm} retrieve the unsealed bit), and automated
  generation of opcode descriptions based on our work in creating a
  CHERI-enhanced MIPS assembler.

\item[1.4] This version updated and clarified a number of aspects of CHERI
  following a prototype implementation used to demonstrate CHERI in November
  2011.
  Changes include updates to the CHERI architecture diagram; replacement of
  the \insnnoref{CDecLen} instruction with \insnnoref{CSetLen},
  addition of a \insnref{CMove} instruction;
  improved descriptions of exception generation; clarification of the
  in-memory representation of capabilities and byte order of permissions;
  modified instruction encodings for \insnref{CGetLen},
  \insnref{CMove}, and \insnnoref{CSetLen};
  specification of reset state for capability registers; and clarification of
  the \insnnoref{CIncBase} instruction.

\item[1.5] This version of the document was produced almost two years
  into the CTSRD project.  It documented a significant revision (version 2) to
  the CHERI ISA, which was motivated by our efforts to introduce
  C-language extensions and compiler support for CHERI, with
  improvements resulting from operating system-level work and
  restructuring the BSV hardware specification to be more
  amenable to formal analysis.  The ISA, programming language, and
  operating system sections were significantly updated.

\item[1.6] This version made incremental refinements to version 2 of the
  CHERI ISA, and also introduced early discussion of the CHERI2 prototype.

\item[1.7] Roughly two and a half years into the project, this version
  clarified and extended documentation of CHERI ISA features such as
  \insnnoref{CCall}/\insnnoref{CReturn} and its software emulation,
  Permit\_Set\_Type, the \insnref{CMove}
  pseudo-op, new load-linked and instructions for store-conditional relative
  to capabilities, and several bug fixes such as corrections to sign extension
  for several instructions.
  A new capability-coprocessor {\pathname cause} register, retrieved using a new
  \insnnoref{CGetCause}, was added to allow querying information on the
  most recent
  CP2 exception (e.g., bounds-check vs type-check violations); priorities were
  provided, and also clarified with respect to coprocessor exceptions vs.
  other MIPS ISA exceptions (e.g., unaligned access).
  This was the first version of the {\em CHERI Architecture Document} released
  to early adopters.

\item[1.8] Less than three and a half years into the project, this version
  refined the CHERI ISA based on experience with compiler, OS, and userspace
  development using the CHERI model.
  To improve C-language compatibility, new instructions \insnref{CToPtr}
  and \insnref{CFromPtr} were defined.
  The capability permissions mask was extended to add user-defined permissions.
  Clarifications were made to the behavior of jump/branch instructions relating
  to branch-delay slots and the program counter.
  \insnref{CClearTag} simply cleared a register's tag, not its value.
  A software-defined capability-cause register range was made available, with a
  new \insnnoref{CSetCause} instruction letting software set the cause for
  testing or control-flow reasons.
  New \insnnoref{CCheckPerm} and \insnnoref{CCheckType} instructions
  were added, letting software
  object methods explicitly test for permissions and the types of arguments.
  TLB permission bits were added to authorize use of loading and storing
  tagged values from pages.
  New \insnnoref{CGetDefault} and \insnnoref{CSetDefault} pseudo-ops
  have become the preferred way to control MIPS ISA memory access.
  \insnnoref{CCall}/\insnnoref{CReturn} calling conventions were
  clarified; \insnnoref{CCall} now pushes the
  incremented version of the program counter, as well as stack pointer, to the
  trusted stack.

\item[1.9 - UCAM-CL-TR-850]
  The document was renamed from the {\em CHERI Architecture Document} to the
  {\em CHERI Instruction-Set Architecture}.
  This version of the document was made available as a University of Cambridge
  Technical Report.
  The high-level ISA description and ISA reference were broken out into
  separate chapters.
  A new rationale chapter was added, along with more detailed explanations
  throughout about design choices.
  Notes were added in a number of places regarding non-MIPS adaptations of
  CHERI and 128-bit variants.
  Potential future directions, such as capability cursors, are discussed in
  more detail.
  Further descriptions of the memory-protection model and its use by operating
  systems and compilers was added.
  Throughout, content has been updated to reflect more recent work on compiler
  and operating-system support for CHERI.
  Bugs have been fixed in the specification of the \insnref{CJR} and
  \insnref{CJALR} instructions.
  Definitions and behavior for user-defined permission bits and OS exception
  handling have been clarified.

\item[1.10]
  This version of the Instruction-Set Architecture is timed for delivery at
  the end of the fourth year of the CTSRD Project.  It reflects a significant
  further revision to the ISA (version 3) focused on C-language compatibility,
  better exception-handling semantics, and reworking of the object-capability
  mechanism.

  The definition of the NULL capability has been revised such that the memory
  representation is now all zeroes, and with a zeroed tag.  This allows
  zeroed memory (e.g., ELF BSS segments) to be interpreted as being filled
  with NULL capabilities.  To this end, the tag is now defined as unset, and
  the Unsealed bit has now been inverted to be a Sealed bit; the
  \insnnoref{CGetUnsealed} instruction has been renamed to
  \insnnoref{CGetSealed}.

  A new \coffset{} field has been added to the capability, which converts CHERI
  from a simple base/length capability to blending capabilities and fat
  pointers that associate a base and bounds with an offset.
  This approach learns from the extensive fat-pointer research literature to
  improve C-language compatibility.
  The offset can take on any 64-bit value, and is added to the base on
  dereference; if the resulting pointer does not fall within the base and
  length, then an exception will be thrown.
  New instructions are added to read (\insnref{CGetOffset}) and write
  (\insnref{CSetOffset}) the
  field, and the semantics of memory access and other CHERI instructions
  (e.g., \insnnoref{CIncBase}) are updated for this new behavior.

  A new \insnnoref{CPtrCmp} instruction has been added, which provides
  C-friendly
  comparison of capabilities; the instruction encoding supports various types
  of comparisons including `equal to', `not equal to', and both signed and
  unsigned `less than' and `less than or equal to' operators.

  \insnnoref{CGetPCC} now returns \PC{} as the \coffset{} field of the
  returned \PCC{} rather than storing it to a general-purpose integer register.
  \insnref{CJR} and \insnref{CJALR} now accept target \PC{} values
  via the offsets of their
  jump-target capability arguments rather than via explicit general-purpose
  integer registers.
  \insnref{CJALR} now allows specification of the return-program-counter
  capability register in a manner similar to return-address arguments to the
  MIPS \insnnoref{JALR} instruction.

  \insnnoref{CCall} and \insnnoref{CReturn} are updated to save and
  restore the saved \PC{} in the
  \coffset{} field of the saved \EPCC{} rather than separately.
  \EPCC{} now incorporates the saved exception \PC{} in its \coffset{} field.
  The behavior of \EPCC{} and expectations about software-supervisor behavior
  are described in greater detail.
  The security implications of exception cause-code precedence as relates to
  alignment and the emulation of unaligned loads and stores are clarified.
  The behavior of \insnnoref{CSetCause} has been clarified to indicate
  that the instruction should not raise an exception unless the check for
  \capperm*{Access\_EPCC} fails.
  When an exception is raised due to the state of an argument register for
  an instruction, it is now defined which register will be named as the source
  of the exception in the capability cause register.

  The object-capability type field is now 24-bit; while a relationship to
  addresses is maintained in order to allow delegation of type allocation,
  that relationship is deemphasized.
  It is assumed that the software type manager will impose any required
  semantics on the field, including any necessary uniqueness for the software
  security model.
  The \insnnoref{CSetType} instruction has been removed, and a single
  \insnnoref{CSeal} instruction
  replaces the previous separate \insnnoref{CSealCode} and
  \insnnoref{CSealData} instructions.

  The validity of capability fields accessed via the ISA is now defined for
  untagged capabilities; the undefinedness of the in-memory representation of
  capabilities is now explicit in order to permit `non-portable'
  micro-architectural optimizations.

  There is now a structured description of the pseudocode language used in
  defining instructions.
  Format numbers have now been removed from instruction descriptions.

  Ephemeral capabilities are renamed to `local capabilities,' and
  non-ephemeral capabilities are renamed to `global capabilities'; the
  semantics are unchanged.

\item[1.11 - UCAM-CL-TR-864]
  This version of the CHERI ISA has been prepared for publication as a
  University of Cambridge technical report.
  It includes a number of refinements to CHERI ISA version 3 based on further
  practical implementation experience with both C-language memory protection
  and software compartmentalization.

  There are a number of updates to the specification reflecting introduction
  of the \coffset{} field, including discussion of its semantics.
  A new \insnref{CIncOffset} instruction has been added, which avoids the
  need to read the offset into a general-purpose integer register for frequent
  arithmetic operations on pointers.

  Interactions between \EPC{} and \EPCC{} are now better specified, including
    that use of untagged capabilities has undefined behavior.
  \insnnoref{CBTS} and \insnnoref{CBTU} are now defined to use
    branch-delay slots, matching other MIPS-ISA branch instructions.
  \insnref{CJALR} is defined as suitably incrementing the returned
    program counter, along with branch-delay slot semantics.
  Additional software-path pseudocode is present for \insnnoref{CCall} and
    \insnnoref{CReturn}.

  \insnref{CAndPerm} and \insnref{CGetPerm} use of argument-register
    or return-register permission bits has been clarified.
  Exception priorities and cause-code register values have been defined,
    clarified, or corrected for \insnref{CClearTag},
  \insnnoref{CGetPCC}, \insnref{CSC}, and \insnref{CSeal}.
  Sign or zero extension for immediates and offsets are now defined
    \insnnoref[clbhwd]{CL}, \insnnoref[clbhwd]{CS},
    and other instructions.

  Exceptions caused due to TLB bits controlling loading and storing of
    capabilities are now CP2 rather than TLB exceptions, reducing code-path
    changes for MIPS exception handlers.
  These TLB bits now have modified semantics: {\bf LC} now discards tag bits
    on the underlying line rather than throwing an exception; {\bf SC} will
    throw an exception only if a tagged store would result, rather than
    whenever a write occurs from a capability register.
  These affect \insnref{CLC} and \insnref{CSC}.

  Pseudocode definitions now appear earlier in the chapter, and have now been
    extended to describe \EPCC{} behavior.
  The ISA reference has been sorted alphabetically by instruction name.

\item[1.12] This is an interim release as we begin to grapple with 128-bit
  capabilities.
  This requires us to better document architectural assumptions, but also
  start to propose changes to the instruction set to reflect differing
  semantics (e.g., exposing more information to potential capability
  compression).
  A new \insnref{CSetBounds} instruction is proposed, which allows both
  the base and length of a capability to be set in a single instruction, which
  may allow the micro-architecture to reduce potential loss of precision.
  Pseudocode is now provided for both the pure-exception version of the
  \insnnoref{CCall} instruction, and also hardware-accelerated permission
  checking.

\item[1.13] This is an interim release as our 128-bit capability format (and
  general awareness of imprecision) evolves; this release also makes early
  infrastructural changes to support an optional converging of capability and
  general-purpose integer register files.

  Named constants, rather than specific sizes (e.g., 256-bit vs. 128-bit) are
  now used throughout the specification.
  Reset state for permissions is now relative to available permissions.
  Two variations on 128-bit capabilities are defined, employing two variations
  on capability compression.
  Throughout the specification, the notion of ``representable'' is now
  explicitly defined, and non-representable values must now be handled.

  The definitions of \insnref{CIncOffset}, \insnref{CSetOffset}, and
  \insnref{CSeal} have been modified to reflect the potential for
  imprecision.
  In the event of a loss of precision, the capability base, rather than
  offset, will be preserved, allowing the underlying memory object to continue
  to be accurately represented.

  Saturating behavior is now defined when a compressed capability's length
  could represent a value greater than the maximum value for a 64-bit MIPS
  integer register.

  EPCC behavior is now defined when a jump or branch target might push the
  offset of PCC outside of the representable range for EPCC.

  \insnnoref{CIncBase} and \insnnoref{CSetLen} are deprecated in favor
  of \insnref{CSetBounds}, which presents changes to base and bounds to
  the hardware atomically.
  The \insnref{CMove} pseudo-operation is now implemented using
  \insnref{CIncOffset} rather than \insnnoref{CIncBase}.
  \insnref{CFromPtr} has been modified to behave more like
  \insnref{CSetOffset}: only the offset, not the base, is modified.
  Bug fixes have been applied to the definitions of \insnref{CSetBounds}
  and \insnref{CUnseal}.

  Several bugs in the specification of \insnref{CLC}, \insnnoref{CLLD},
  \insnref{CSC}, and \insnnoref[csbhwd]{CSD}, relating to omissions
  during the update to capability offsets, have been fixed.
  \insnref{CLC}'s description has been updated to properly reflect its
  immediate argument.

  New instructions \insnnoref{CClearHi} and \insnnoref{CClearLo} have
  been added to accelerate register clearing during protection-domain
  switches.

  New pseudo-ops \insnnoref{CGetEPCC}, \insnnoref{CSetEPCC},
  \insnnoref{CGetKCC}, \insnnoref{CSetKCC}, \insnnoref{CGetKDC}, and
  \insnnoref{CSetKDC} have been defined, in the interests of better
  supporting a migration of `special' registers out of the capability register
  file -- which facilitates a convergence of capability and general-purpose
  integer register files.

\item[1.14]
  Two new chapters have been added, one describing the abstract CHERI
  protection model in greater detail (and independent from concrete ISA
  changes), and the second exploring the composition of CHERI's ISA-level
  features in supporting higher-level software protection models.

  The value of the NULL capability is now centrally defined (all fields zero;
  untagged).

  \insnnoref{ClearLo} and \insnnoref{ClearHi} instructions are now
  defined for clearing general-purpose integer registers, supplementing
  \insnnoref{CClearHi} and \insnnoref{CClearLo}.
  All four instructions are described together under \insnnoref{CClearRegs}.

  A new \insnref{CSetBoundsExact} instruction is defined, allowing an
  exception to be thrown if an attempt to narrow bounds cannot occur
  precisely.
  This is intended for use in memory allocators where it is a software
  invariant that bounds are always exact.
  A new exception code is defined for this case.

  A full range of data widths are now support for capability-relative
  load-linked, store conditional: \insnnoref{CLLB}, \insnnoref{CLLH},
  \insnnoref{CLLW}, \insnnoref{CLLD}, \insnnoref{CSCB},
  \insnnoref{CSCH}, \insnnoref{CSCW}, and \insnnoref{CSCD} (as well as
  unsigned load-linked variations).
  Previously, only a doubleword variation was defined, but cannot be used to
  emulate the narrower widths as fine-grained bounds around a narrow type
  would throw a bounds-check exception.
  Existing load-linked, store-conditional variations for capabilities
  (\insnref{CLLC}, \insnnoref{CSCC}) have been updated, including with
  respect to opcode assignments.

  A new `candidate three' variation on compressed capabilities has been
  defined, which differentiates sealed and unsealed formats.
  The unsealed variation invests greater numbers of bits in bounds accuracy,
  and has a full 64-bit cursor, but does not contain a broader set of
  software-defined permissions or an object-type field.
  The sealed variation also has a full 64-bit cursor, but has reduced bounds
  accuracy in return for a 20-bit object-type field and a set of
  software-defined permissions.

  `Candidate two' of compressed capabilities has been updated to reflect
  changes in the hardware prototype by reducing toBase and toBound precision
  by one bit each.

  Explicit equations have been added explaining how bounds are calculated
  from each of the 128-bit compressed capability candidates, as well as their
  alignment requirements.

  Exception priorities have been documented (or clarified) for a number of
  instructions including \insnref{CJALR}, \insnref{CLC},
  \insnnoref{CLLD}, \insnref{CSC}, \insnnoref{CSCC},
  \insnnoref{CSetLen}, \insnref{CSeal}, \insnref{CUnSeal}, and
  \insnref{CSetBounds}.

  The behavior of \insnnoref{CPtrCmp} is now defined when an undefined
  comparison type is used.

  It is clarified that capability store failures due to TLB-enforced
  limitations on capability stores trigger a TLB, rather than a CP2,
  exception.

  A new capability comparison instruction, \insnnoref{CEXEQ}, checks
  whether all fields in the capability are equal; the previous
  \insnnoref{CEQ} instruction checked only that their offsets pointed at the
  same location.

  A new capability instruction, \insnnoref{CSUB}, allows the implementation
  of C-language pointer subtraction semantics with the atomicity properties
  required for garbage collection.

  The list of BERI- and CHERI-related publications, including peer-reviewed
  conference publications and technical reports, has been updated.

\item[1.15 - UCAM-CL-TR-876]
  This version of the CHERI ISA, \textit{CHERI ISAv4}, has been prepared for
  publication as a University of Cambridge technical report.

  The instructions \insnnoref{CIncBase} and \insnnoref{CSetLen}
  (deprecated in version 1.13 of the CHERI ISA) have now been removed in favor
  of \insnref{CSetBounds} (added in version 1.12 of the CHERI ISA).
  The new instruction was introduced in order to atomically expose changes to
  both upper and lower bounds of a capability, rather than requiring them to
  be updated separately, required to implement compressed capabilities.

  The design rationale has been updated to better describe our ongoing
  exploration of whether special registers (such as \KCC{}) should be in the
  capability register file, and the potential implications of shifting to a
  userspace exception handler for \insnnoref{CCall}/\insnnoref{CReturn}.

\item[1.16] This is an interim update of the instruction-set specification in
  which aspects of the 128-bit capability model are clarified and extended.

  The ``candidate 3'' unsealed 128-bit compressed capability representation
  has been to increase the exponent field (\cexponent{}) to 6 bits from 4, and
  the \cbasebits{} and \ctopbits{} fields have been reduced to 20 bits each
  from the 22 bits.
  \cperms{} has been increased from 11 to 15 to allow for a larger set of
  software-defined permissions.
% XXX-BD: 2 - 4 + 4 != 0.  Presumably we consumed two reserved bits?
  The sealed representation has also been updated similarly, with a total of
  10 bits for \cotype{} (split over {\bf otypeLow} and {\bf otypeHigh}), 10
  bits each for \cbasebits{} and \ctopbits{}, and a 6-bit exponent.
  The algorithm for decompressing a compressed capability has been changed to
  better utilize the encoding space, and to more clearly differentiate
  representable from in-bounds values.
  A variety of improvements and clarifications have been made to the
  compression model and its description.

  Differences between, and representations of, permissions for 128-bit and
  256-bit capability are now better described.

  Capability unrepresentable exceptions will now be thrown in various
  situations where the result of a capability manipulation or operation cannot
  be represented.
  For manipulations such as \insnref{CSeal} and \insnref{CFromPtr},
  an exception will be thrown.
  For operations such as \insnnoref{CBTU} and \insnnoref{CBTS}, the
  exception will be thrown on the first instruction fetch following a branch
  to an unrepresentable target, rather than on the branch instruction itself.
  CHERI1 and CHERI2 no longer differ on how out-of-bounds exceptions are
  thrown for capability branches: it uniformly occurs on fetching the target
  instruction.

  The ISA specification makes it more clear that \insnnoref{CEQ},
  \insnnoref{CNE}, \insnnoref[cptrcmp]{CL[TE]U}, and \insnnoref{CEXEQ} are
  forms of the \insnnoref{CPtrCmp} instruction.

  The ISA todo list has been updated to recommend a capability
  conditional-move (\insnnoref{CCMove}) instruction.

  There is now more explicit discussion of the MIPS n64 ABI, Hybrid ABI,
  and Pure-Capability ABI.
  Conventions for capability-register have been updated and clarified --
  for example, register assignments for the stack capability, jump register,
  and link register.
  The definition that {\bf RCC}, the return code capability, is register
  \creg{24} has been updated to reflect our use of \creg{17} in actual code
  generation.

  Erroneous references to an undefined instruction \insnnoref{CSetBase},
  introduced during removal of the \insnnoref{CIncBase} instruction, have
  been corrected to refer to \insnref{CSetBounds}.

\item[1.17] This is an interim update of the instruction-set architecture
  enhancing (and specifying in more detail) the CHERI-128 ``compressed''
  128-bit capability format, better aligning the 128-bit and 256-bit models,
  and adding capability-related instructions required for more efficient code
  generation.
  This is a draft release of what will be considered \textit{CHERI ISAv5}.

  The chapter on ISA design now includes a section describing ``deep'' versus
  ``surface'' aspects of the CHERI model as mapped into the ISA.
  For example, use of tagged capabilities is a core aspect of the model, but
  the particular choice to have a separate capability register file, rather
  than extending general-purpose integer registers to optionally hold capabilities, is
  a surface design choice in that the operating system and compiler can target
  the same software-visible protection model against both.
  Likewise, although CHERI-128 specifies a concrete compression model, a range
  of compression approaches are accepted by the CHERI model.

  A new chapter has been added describing some of our assumptions about how
  capabilities will be used to build secure systems, for example, that
  untrusted code will not be permitted to modify TLB state -- which permits
  changing the interpretation of capabilities relative to virtual addresses.

  The rationale chapter has been updated to more thoroughly describe our
  capability compression design space.

  A new CHERI ISA quick-reference appendix has been added to the
  specification, documenting both current and proposed instruction
  encodings.

  Sections of the introduction on historical context have been shifted to a
  stand-alone chapter.

  Descriptions in the introduction have been updated relating to
  our hardware and software prototypes.

  References to PhD dissertations on CHERI have been added to the publications
  section of the introduction.

  A clarification has been added: the use of the term ``capability
  coprocessor'' relates to CHERI's utilization of the MIPS ISA coprocessor
  opcode space, and is not intended to suggest substantial decoupling of
  capability-related processing from the processor design.

  Compressed capability ``candidate 3'' is now CHERI-128. The \cbasebits{},
  \ctopbits{} and
  \ccursor{} fields have been renamed respectively \cB{}, \cT{} and \caddr{}
  (following the terminology used in the micro paper). When sealed, only the
  top 8 bits of the \cB{} and \cT{} fields are preserved, and the bottom 12
  bits are zeroes, which implies stronger alignment requirements for sealed
  capabilities. The exponent \cexponent{} field remains a 6-bit field, but its
  bottom 2 bits are ignored, as it is believed that coarser granularity is
  acceptable, and making the hardware simpler. The \cotype{} field benefits
  from the shorter \cB{} and \cT{} fields and is now 24 bits -- which is the same
  as the \cotype{} for 256-bit CHERI. Finally, the representable region
  associated with a capability has changed from being centred around the
  described object to an asymmetric region with more space above the object
  than below. The full description is available in Section~\ref{compression}.

  Alignment requirements for software allocators (such as stack and heap
  allocators) in the presence of capability compression are now more
  concisely described.

  The immediate operands to load and store instructions, including
  \insnnoref{CLC}, \insnnoref{CSC}, \insnnoref[clbhwd]{CL[BHWD][U]}, and
  \insnnoref[csbhwd]{CS[BHWD]} are now ``scaled'' by the width of the data being
  stored (with the exception of capability stores, where scaling is by 16
  bytes regardless of in-memory capability size).
  This extends the range of capability-relative loads and stores, permitting
  a far greater proportion of stack spills to be expressed without additional
  stack-pointer modification.
  This is a binary-incompatible change to the ISA.

  The textual description of the \insnref{CSeal} instruction has been
  updated to match the pseudocode in using $>=$ rather than $>$ in selecting
  an exception code.

  A redundant check has been removed in the definition of the
  \insnref{CUnseal} instruction, and an explanation added.

  Opcodes have now been specified for the \insnref{CSetBoundsExact} and
  \insnnoref{CSub} instructions.

  To improve code generation when constructing a \PCC{}-relative capability as
  a jump target, a new \insnnoref{CGetPCCSetOffset} instruction has been
  added.
  This instruction has the combined effects of performing sequential
  \insnnoref{CGetPCC} and \insnref{CSetOffset} operations.

  A broader set of opcode rationalizations and cleanups have been applied
  across the ISA, to facilitate efficient decoding and future use of the
  opcode space.
  This includes changes to \insnnoref{CGetPCC}.

  \creg{25} is no longer reserved for exception-handler use, as \creg{27} and
  \creg{28} are already reserved for this purpose.
  It is therefore available for ABI use.

  The 256-bit architectural capability model has been updated to use a single
  system permission, \cappermASR, to control access to
  exception-handling and privileged ISA state, rather than splitting it over
  multiple permissions.
  This brings the permission models in 128-bit and 256-bit representations
  back into full alignment from a software perspective.
  This also simplifies permission checking for instructions such as
  \insnnoref{CClearRegs}.
  The permission numbering space has been rationalized as part of this change.
  Similarly, the set of exceptions has been updated to reflect a single system
  permission.
  The descriptions of various instructions (such as \insnnoref{CClearRegs}
  have been updated with respect to revised protections for special registers
  and exception handling.

  The descriptions of \insnnoref{CCall} and \insnnoref{CReturn} now
  include an explanation of additional software-defined behavior such as
  capability control-flow based on the local/global model.

  The common definition of privileged registers (included in the definitions
  of instructions) has been updated to explicitly include \EPCC{}.

  Future ISA additions are proposed to add testing of branch instructions for
  NULL and non-NULL capabilities.

\item[1.18 - UCAM-CL-TR-891] This version of the CHERI ISA,
  \textit{CHERI ISAv5}, has been prepared for publication as a University of
  Cambridge technical report.

  The chapter on the CHERI protection model has been refined and extending,
  including adding more information on sealed capabilities, the link between
  memory allocation and the setting of bounds and permissions, more detailed
  coverage of capability flow control, and interactions with MMU-based models.

  A new chapter has been added exploring assumptions that must be made when
  building high-assurance software for CHERI.

  The detailed ISA version history has shifted from the introduction to a new
  appendix; a summary of key versions is maintained in the introduction, along
  with changes in the current document version.

  A glossary of key terms has been added.

  The term ``coprocessor'' is deemphasized, as, while it refers correctly to
  CHERI's use of the MIPS opcode extension space, some readers found it
  suggestive of an independent hardware unit rather than tight integration into
  the processor pipeline and memory subsystem.

  A reference has been added to Robert Norton's PhD dissertation on optimized
  CHERI domain switching.

  A reference has been added to our PLDI 2016 paper on C-language semantics and
  their interaction with the CHERI model.

  The object-type field in both 128-bit and 256-bit capabilities is now 24 bits,
  with Top and Bottom fields reduced to 8 bits for sealed capabilities.
  This reflects a survey of current object-oriented software systems, suggesting
  that 24 bits is a more reasonable upper bound than 20 bits.

  The assembly arguments to \insnref{CJALR} have been swapped for greater
  consistency with jump-and-link register instructions in the MIPS ISA.

  We have reduced the number of privileged permissions in the 256-bit capability
  model to a single privileged permission, \cappermASR, to match
  128-bit CHERI.
  This is a binary-incompatible change.

  We have improved the description of the CHERI-128 model in a number of ways,
  including a new section on the CHERI-128 representable bounds check.

  The architecture chapter contains a more detailed discussion of potential ways
  to reduce the overhead of CHERI by reducing the number of capability
  registers, converging the general-purpose integer and capability register files,
  capability compression, and so on.

  We have extended our discussion of ``deep'' vs ``shallow'' aspects of the
  CHERI model.

  New sections describe potential non-pointer uses of capabilities, as well as
  possible uses as primitives supporting higher-level languages.

  Instructions that convert from integers to capabilities now share common
  \ccode{int_to_cap} pseudocode.

  The notes on \insnnoref{CBTS} have been synchronized to those on
  \insnnoref{CBTU}.

  Use of language has generally been improved to differentiate the
  architectural 256-bit capability model (e.g., in which its fields are
  64-bit) from the 128-bit and 256-bit in-memory representations.
  This includes consideration of differing representations of capability
  permissions in the architectural interface (via instructions) and the
  microarchitectural implementation.

  A number of descriptions of features of, and motivations for, the CHERI design
  have been clarified, extended, or otherwise improved.

  It is clarified that when combining immediate and register operands with
  the base and offset, 64-bit wrap-around is permitted in capability-relative
  load and store instructions -- rather than throwing an exception.
  This is required to support sound optimizations in frequent
  compiler-generated load/store sequences for C-language programs.

\item[1.19] This release of the \textit{CHERI Instruction-Set Architecture
  (ISA) Specification} is an interim version intended for submission to
  DARPA/AFRL to meet the requirements of CTSRD deliverable A015.

  The behavior of \insnref{CToPtr} in the event that the pointer of one
  capability is to the base of the containing capability has been clarified.

  The \cappermASR permission is extended to cover non-CHERI ISA
  privileges, such as use of MIPS TLB-management, interrupt-control,
  exception-handling, and cache-control instructions available in the kernel
  ring.
  The aim of these in-progress changes is to allow the compartmentalization of
  kernel code.

\item[1.20 - UCAM-CL-TR-907] This version of the CHERI ISA, \textit{CHERI
  ISAv6}, has been prepared for publication as University of Cambridge
  technical report UCAM-CL-TR-907.

  Chapter~\ref{chap:introduction} has been substantially reformulated,
  providing brief introductions to both the CHERI protection model and
  CHERI-MIPS ISA, with much remaining content on our research methodology now
  shifted to its own new chapter, Chapter~\ref{chap:research}.
  Our architectural and application-level least-privilege motivations are now
  more clearly described, as well as hybrid aspects of the CHERI approach.
  Throughout, better distinction is made between the CHERI protection model and
  the CHERI-MIPS ISA, which is a specific instantiation of the model with
  respect to 64-bit MIPS.
  The research methodology chapter now provides a discussion of our overall
  approach, more detailed descriptions of various phases of our research and
  development cycle, and describes major transitions in our approach as the
  project proceeded.

  Chapter~\ref{chap:model} on the software-facing CHERI protection model has
  been improved to provide more clear explanations of our approach as well as
  additional illustrations.
  The chapter now more clearly enunciates two guiding principles
  underlying the CHERI ISA design: the \textit{principle of least privilege},
  and the \textit{principle of intentional use}.
  The former has been widely considered in the security literature, and
  motivates privilege reduction in the CHERI ISA.
  The latter has not previously described, and is supports the use of explicitly
  named rights, rather than implicitly selected ones, wherever possible in order
  to avoid `confused deputy' problems.
  Both contribute to vulnerability mitigation effects.
  New sections have been added on revocation and garbage collection.
  The role and implementation of monotonicity (and also non-monotonicity) in
  the ISA are more clearly described.

  A chapter on architectural sketches has been added, describing how the CHERI
  protection model might be introduced in the RISC-V and x86-64 ISAs.
  In doing so, we identify a number of key aspects of the CHERI model that are
  required regardless of the underlying ISA.
  We argue that the CHERI protection model is a \textit{portable} model that can
  be implemented consistently across a broad range of underlying ISAs and
  concrete integrations with those ISAs.
  One implication of this argument is that portable CHERI-aware software can be
  implemented across underlying architectural implementations.

  Chapter~\ref{chap:architecture} now describes, at a high level, CHERI's
  expectations for tagged memory.

  We in general now prefer the phrase ``control-flow robustness'' to
  ``control-flow integrity'' when talking about capability protection for code
  pointers, in order to avoid confusion with conventional CFI.

  The descriptions of software-defined aspects of the \insnnoref{CCall} and
  \insnnoref{CReturn} instructions have been removed from the description and
  pseudocode of each instruction.
  They are instead part of an expanded set of notes on potential software use
  for these instructions.

  A new \insnnoref{CCall} selector 1 has been added that provides a jump-like
  domain transition without use of an architectural exception.
  In this mode of operation, \insnnoref{CCall} unseals the sealed code and
  data capabilities to enter the new domain, offering a different set of
  hardware and software tradeoffs from the existing selector-0 semantics.
  For example, complex exception-related mechanism is avoided in hardware for
  domain switches, with the potential to substantially improve performance.
  Software would most likely use this mechanism to branch into a trusted
  intermediary capability of supporting safe and controlled switching to a new
  object.

  To support the new \insnnoref{CCall} selector 1, a new permission,
  \emph{Permit\_CCall} is defined authorizing use of the selector on sealed
  capabilities.
  The permission must be present on both sealed code and data capabilities.

  To support the new \insnnoref{CCall} selector 1, a new CP2 exception cause
  code, Permit\_CCall Violation is defined to report a lack of the
  \emph{Permit\_CCall} permission on sealed code or data capabilities passed to
  \insnnoref{CCall}.

  New experimental instructions \insnref{CBuildCap} (import a capability),
  \insnref{CCopyType} (import the \cotype{} field of a capability), and
  \insnref{CCSeal} (conditionally seal a capability) have been added
  to the ISA to be used when re-internalizing capabilities that have been
  written to non-capability-aware memory or storage.
  This instruction is intended to satisfy use cases such as swapping to
  disk, migrating processes, migrating virtual machines, and run-time linking.
  A suitable authorizing capability is required in order to restore the
  tag.
  As these instructions are considered experimental, they are documented in
  Appendix~\ref{app:experimental} rather than the main specification.

  The \insnref{CGetType} instruction now returns $-1$ when used on an
  unsealed capability, in order to allow it to be more easily used with
  \insnref{CCSeal}.

  Two new conditional-move instructions are added to the CHERI-MIPS ISA:
  \insnnoref{CMOVN} (conditionally move capability on non-zero), and
  \insnnoref{CMOVZ} (conditionally move capability on zero).
  These complement existing conditional-move instructions in the 64-bit MIPS
  ISA, allowing more efficient generated code.

  The \insnref{CJR} (capability jump register) and \insnref{CJALR}
  (capability jump and link register) have been changed to accept non-global
  capability jump targets.

  The \insnref{CLC} (capability load capability) and \insnref{CLLC}
  (capability load-linked conditional) instructions will now strip loaded tags,
  rather than throwing an exception, if the Permit\_Load\_Capability permission
  is not present.

  The \insnref{CToPtr} (capability to pointer) instruction now checks that
  the source register is not sealed, and performs comparative range checks of
  the two source capabilities.
  More detailed rationale has been provided for the design of the
  \insnref{CToPtr} instruction in Chapter~\ref{chap:rationale}.

  The pseudocode for the \insnnoref{CCheckType} (capability check
  type) instruction has been corrected to test uperm as well as perm.
  The pseudocode for \insnnoref{CCheckType} has been corrected to test the
  sealed bit on both source capabilities.
  An encoding error for \insnnoref{CCheckType} in the ISA quick reference has
  been corrected.

  The pseudocode for the \insnref{CGetPerm} (capability get permissions)
  instruction has been updated to match syntax used in the
  \insnref{CGetType} and \insnnoref{CGetCause} instructions.

  The pseudocode for the \insnref{CUnseal} (capability unseal) instruction
  has been corrected to avoid an aliasing problem when the source and
  destination register are the same.

  The description of the \insnref{CSeal} (capability seal) instruction has
  been clarified to explain that precision cannot be lost in the case where
  bounds are no longer precisely representable, as an exception will be thrown.

  The description of the fast representability check for compressed capabilities
  has been improved.

  CHERI-related exception handling behavior is now clarified with respect to the
  MIPS EXL status bit, with the aim of ensuring consistent behavior.
  Regardless of bounds set on \KCC{}, a suitable offset is selected so that the
  standard MIPS exception vector will be executed via the exception \PCC{}.

  The section on CHERI control has been
  clarified to more specifically identify 64-bit MIPS privileged instructions,
  KSU bits, and general operation modified by the \cappermASR
  permission.
  The section now also more specifically described privileged behaviors not
  controlled by the permission, such as use of specific exception vectors.
  A corresponding rationale section has been added to
  Chapter~\ref{chap:rationale}.

  A number of potential future instruction-set improvements relating to
  capability compression, control flow, and instruction variants with immediates
  have been added to the future ISA changes list in
  Chapter~\ref{chap:architecture}.

  Opcode-space reservations for the previously removed \insnnoref{CIncBase}
  and \insnnoref{CSetLen} instructions have also been removed.

  \creg{25}, which had its hard-coded ISA use removed in CHERI ISAv5, has now
  been made a caller-save capability register in the ABI.

  Citations to further CHERI research publications have been added.

\item[1.21] This release of the \textit{CHERI Instruction-Set Architecture} is
  an interim version intended for submission to DARPA/AFRL to meet the
  requirements of CTSRD deliverable A001, and contains the following changes
  relative to CHERI ISAv6:

  The ISA encoding reference has been updated for new experimental
  instructions.

  A new \insnnoref{CNExEq} instruction has been added, which provides a
  more efficient implementation of a test for negative exact inequality than
  utilizing \insnnoref{CExEq} and inverting the result.

  Specify that when a TLB exception results from attempting to store a
  tagged capability via a TLB entry that does not authorize tagged store, the
  MIPS EntryHi register will be set correspondingly.

\item[7.0-ALPHA1]
This release of the \textit{CHERI Instruction-Set Architecture} is an
interim version intended for submission to DARPA/AFRL to meet the requirements
of CTSRD deliverable A001:

\begin{itemize}
\item The CHERI ISA specification version numbering scheme has changed to
include the target major version in the draft version number.

\item A significant refactoring of early chapters in the report has taken place:
there is now a more clear distinction between architecture-neutral aspects
of CHERI, and those that are architecture specific.
The CHERI-MIPS ISA is now its own chapter distinct from architecture-neutral
material.
We have aimed to maximize architecture-neutral content -- e.g., capability
semantics and contents, in-memory representation, compression, etc. -- using
the architecture-specific chapters to address only architecture-specific
aspects of the mapping of CHERI into the specific architecture -- e.g., as
relates to register-file integration, exception handling, and the Memory
Management Unit (MMU).
In some areas, content must be split between architecture-neutral and
architecture-specific chapters, such as behavior on reset, handling of the
\cappermASR permission and its role in controlling
architecture-specific behavior, and the integration of CHERI with virtual
memory, where the goals are largely architecture neutral but mechanism is
architecture specific.

\item There are now dedicated chapters for each of our applications of CHERI
to each of three ISAs: 64-bit MIPS, 64-bit
RISC-V (Chapter~\ref{chap:cheri-riscv}), and x86-64
(Chapter~\ref{chap:cheri-x86-64}).

\item Our CHERI-RISC-V prototype has been substantially elaborated, and now
includes an experimental encoding in Appendix~\ref{app:isaquick-riscv}.
We have somewhat further elaborated our x86-64 model, including addressing
topics such as new page-table bits for CHERI, including a hardware-managed
capability dirty bit.
We also consider potential implications for RISC-V compressed instructions.

\item We have completed an opcode renumbering for CHERI-MIPS.
The ``proposed new encoding'' from CHERI ISAv6 has now become the
established encodings; the prior encodings are now documented as
``deprecated encodings''.

\item Substantial improvements have been made to descriptive text around memory
protection, with the concept of ``pointer protection'' -- i.e., as
implemented via tags -- more clearly differentiated from memory protection.

\item We now more clearly describe how terms like ``lower bound'' and ``upper
bound'' relate to the base, offset, and length fields.

\item We now more clearly differentiate language-level capability semantics
from capability use in code generation and the ABI, considering
pure-capability and hybrid C as distinct from pure-capability and hybrid code
generation.
We explain that different language-level integer interpretations of
capabilities are supportable by the architecture, depending on compiler
code-generation choices.

\item Potential software policies for revocation, garbage collection, and
capability flow control based on CHERI primitives are described in greater
detail.

\item Monotonicity is more clearly described, as are the explicit
opportunities for non-monotonicity around exception handling and
\insnnoref{CCall} Selector 1.
Handling of disallowed requests for non-monotonicity or bypass of guarded
manipulation by software is more explicitly discussed, including the
opportunities for both exception throwing and tag stripping to maintain
CHERI's invariants.

\item Further notes have been added regarding the in-memory representation of
capabilities, including the storage of NULL capabilities, virtual addresses
for non-NULL capabilities, and how to store integer values in untagged
capability registers.
These values now appear in the bottom 64 bits of the in-memory
representation.
Topics such as endianness are also considered.

\item NULL capabilities are now defined as having a base of 0x0, the maximum
length supported in a particular representation ($2^{64}$ for 128-bit
capabilities, and $2^{64} - 1$ for 256-bit capabilities), and no granted
permissions.
NULL capabilities continue to have an all zeros in-memory representation.
This allows integers to be stored in the offset of an untagged capability
without concern that they may hold values that are unrepresentable with
respect to capability bounds.

\item New instructions \insnnoref{CReadHwr} and \insnnoref{CWriteHwr} have
been added.
These have allowed us to migrate special capability registers (SCRs) out of
the general-purpose capability register file, including \DDC{}, the new user
TLS register (\CULR{}), the new privileged TLS register (\CPLR{}), \KRC{},
\KQC{}, \KCC{}, \KDC{}, and \EPCC{}.
Access to privileged special registers continues to be authorized by the
\cappermASR permission on \PCC{}.

\item With this migration, \creg{0} is now available to use as a NULL
capability register, which is more consistent with the baseline MIPS ISA in
which \reg{0} is the zero register.
The only exception to this is in capability-relative load and store
instructions, and the \insnref{CTestSubset} instruction, in
which an operand of \creg{0} specifies that \DDC{} should be used.

\item Various instruction pseudo-ops to access special registers, such as
\insnnoref{CGetDefault}, now expand to special capability register access
instructions instead of capability move instructions.

\item With consideration of merged rather than split integer and capability
register files for RISC-V and x86-64, and a separation between
general-purpose capability registers and special capability registers (SCRs) on 64-bit MIPS, we
avoid describing the integer register file as the ``general-purpose register
file''.
We describe a number of tradeoffs around ISA design relating to using a
split vs. merged register file; avoiding the use of specific capability
registers as special registers assists in supporting both register-file
approaches.

\item The CPU reset state of various capability registers is now more clearly
defined.
Most capability registers are initialized to NULL on reset, with the
exception of \DDC{}, \PCC{}, \KCC{}, and \EPCC{}.
These defaults authorize initial access to memory for the boot process, and
are designed to allow CHERI-unaware code to operate oblivious to the
capability-system feature set.

\item We more clearly describe design choices around failure-mode choices,
including throwing exceptions and clearing tag bits.
Here, concerns in conclude stylistic consistency with the host architecture,
potential use cases, and interactions with the compiler and operating
system.

\item In general, we now refer to software-defined permissions rather than
user-defined permissions, as these permissions without an architectural
interpretation may be used in any ring.

\item Permission numbering has been rationalized so that 128-bit and 256-bit
microarchitectural permission numbers consistently start at 15.

\item The existing permission \cappermSeal, which authorized sealing and
explicit unsealing of sealed capabilities, has now been broken out into two
separate permissions: \cappermSeal, which authorizes sealing, and
\cappermUnseal, which authorizes explicit unsealing.
This will allow privilege to be reduced where unsealing is desirable (e.g.,
within object implementations, or in C++ vtable use) by not requiring that
permission to seal for the object type is also granted.

\item The ISA quick reference has been updated to reflect new instructions, as
well as to more correctly reflect endianness.

\item We have added a reference to our recently released technical report, \textit{Capability
Hardware Enhanced RISC Instructions (CHERI): Notes on the Meltdown and
Spectre Attacks}~\cite{UCAM-CL-TR-916}, which considers the potential
interactions between CHERI and the recently announced Spectre and Meltdown
microarchitectural side-channel attacks.
CHERI offers substantial potential to assist in mitigating aspects of these
attacks, as long as the microarchitecture performs required capability
checks before performing any speculative memory accesses.

\item We have added two new instructions, Get the architectural Compartment ID
(\insnref{CGetCID}) and Set the architectural Compartment ID
(\insnref{CSetCID}), which allow information on compartments to
be passed to via architecture to microarchitecture in order to support
mitigation of side-channel attacks.
This could be used to tag branch-predictor entries to control the
compartments in which they can be used, for example.
A new Permit\_Set\_CID permission allows capabilities to delegate use of
ranges of CIDs.

\item Bugs have been fixed in the definitions of various capability-relative
load and store instructions, in which permission checks involving the
Permit\_Load, Permit\_Load\_Cap, Permit\_Store, and Permit\_Store\_Cap
permissions were not properly updated from our shift from an untagged
capability register file to a tagged register file.
All loads now require Permit\_Load.
If Permit\_Load\_Cap is also present, then tags will not be stripped when
loading into a capability register.
All stores now require Permit\_Store.
If Permit\_Store\_Cap is also present, then storing a tagged capability will
not generate an exception.

\item New Capability Set Bounds From Immediate
(\insnref{CSetBoundsImm}) and Capability Increment Offset From Immediate
(\insnref{CIncOffsetImm}) instructions have been added.
These instructions optimize global-variable setup and stack allocations by
reducing the number of instructions and registers required to adjust pointer
values and set bounds.

\item New Capability Branch if Not NULL (\insnnoref{CBNZ}) and
Capability Branch if NULL (\insnnoref{CBEZ}) instructions have
been added, which optimize pointer comparisons to NULL.

\item A new Capability to Address (\insnref{CGetAddr})
instruction allows the direct retrieval of a capability's virtual address,
rather than requiring the base and offset to be separately retrieved and added
together.
This facilitates efficient implementation of a CHERI C variant in which all
casts of capabilities to integers have virtual-address rather than offset
interpretation.
A capability's virtual address is now more directly defined when we specify
capability fields.

\item We more clearly describe \insnnoref{CCall} Selector 1 as
``exception-free domain transition'' rather than ``userspace domain
transition'', as it is also intended to be used in more privileged rings.

\item We have shifted to more consistently throwing an exception at jump
instructions (e.g., \insnref{CJR}) that go out of bounds,
rather than throwing the exception when fetching the first instruction at
the target address.
This provides more debugging information when using compressed
capabilities, as otherwise \EPCC{} might have unrepresentable bounds in the
event that the jump target is outside of the representable region.

\item The exception vectors use during failures of Selector 0 and Selector 1
\insnnoref{CCall} have been clarified.
The general-purpose exception vector is used for all failure modes with
\insnnoref{CCall} Selector 1.

\item We have added a new experimental instruction, Test that Capability is a Subset of Another
(\insnref{CTestSubset}).
This instruction is intended to be used by garbage collectors that need to
rapidly test whether a capability points into the range of another
capability.

\item A new experimental 64-bit capability format for 32-bit virtual addresses
has been added.

\item A description of an experimental {\it linear capability} model has been
added (Section~\ref{section:linear-capabilities}).
This model introduces the concept that a capability may be linear -- i.e.,
that it can only be moved rather copied in memory-to-register,
register-to-register, and register-to-memory operations.
This introduces two new instructions, Linear Load Capability Register
(\insnnoref{LLCR}) and Linear Store Capability Register
(\insnnoref{LSCR}).
This functionality has not yet been fully specified.

\item An experimental appendix considers possible implementations of {\it
indirect capabilities}, in which a capability value points at an actual
capability to utilize, allowing table-based capability lookups
(Section~\ref{section:indirect-capabilities}).

\item An experimental appendix considering potential forms of compression for
capability permissions has been added (Section~\ref{app:exp:compressperm}).

\item We have added a reference to our ICCD 2017 paper, \textit{Efficient
Tagged Memory}, which describes how to efficiently implement tagged memory in
memory subsystems not supporting inline tags directly in
  DRAM~\cite{joannou2017:tagged-memory}.
\end{itemize}


\item[7.0-ALPHA2]
This version of the \textit{CHERI Instruction-Set Architecture} is an interim
version distributed for review by DARPA and our collaborators:

\begin{itemize}
\item We have removed the range check from the \insnref{CToPtr}
  specification, as this has proven microarchitecturally challenging.
  We anticipate that current consumers requiring this range check can use the
  new \insnref{CTestSubset} instruction alongside \insnref{CToPtr}.

\item Use of a branch-delay slot with \insnnoref{CCall} Selector
  1 has been removed.

\item With the addition of \insnref{CReadHwr} and \insnref{CWriteHwr}
  and shifting of special capability registers out of the
  general-purpose capability register file, we have now removed the check for
  the \cappermASR permission for all registers in the
  general-purpose capability register file.

\item A new \insnref{CCheckTag} instruction is added, which
  throws an exception if the tag is not set on the operand capability.
  This instruction could be used by a compiler to shift capability-related
  exception behavior from invalid dereference to calculation of an invalid
  capability via a non-exception-throwing manipulation.

\item We have added a new \insnref{CLCBI} instruction that
  allows capability-relative loads of capabilities to be performed using a
  substantially larger immediate (but without a general-purpose
  integer-register operand).
  This substantially accelerates performance in the presence of CHERI-aware
  linkage by avoiding multi-instruction sequences to load capabilities for
  global variables.

\item We have added new discussion relating to microarchitectural side
  channels such as Spectre and Meltdown
  (Section~\ref{section:microarchitectural-sidechannels}).

\item We have added a reference to our ASPLOS 2019 paper, \textit{CheriABI:
  Enforcing Valid Pointer Provenance and Minimizing Pointer Privilege in the
  POSIX C Run-time Environment}, which describes how to adapt a full MMU-based
  OS design to support ubiquitous use of capabilities to implement C and C++
  pointers in userspace~\cite{davis2019:cheriabi}.

\item We have added a reference to our POPL 2019 paper, \textit{ISA Semantics
  for ARMv8-A, RISC-V, and CHERI-MIPS}, which describes a formal modeling
  approach for instruction-set architectures, as well as a formal model of the
  CHERI-MIPS ISA~\cite{sail-popl2019}.

\item We have added a reference to our POPL 2019 paper, \textit{Exploring C
  Semantics and Pointer Provenance}, which describes a formal model for C
  pointer provenance, and is evaluated in part using pure-capability CHERI
  code~\cite{cerberus-popl2019}.

\item We have added a description of an experimental compact capability
  coloring scheme, a possible candidate to replace our Local-Global capability
  flow-control model (Section~\ref{sec:compactcolors}).
  In the proposed scheme, a series of orthogonal ``colors'' can be set or
  cleared on capabilities, authorized by a color space implemented in a
  style similar to the sealed-capability object-type space using a single
  permission.
  For a single color implementing the Local-Global model, two bits are still
  used.
  However, for further colors, only a single bit is used.
  This could make available further colors to use for kernel-user separation,
  inter-process isolation, and so on.

\item An experimental Permit\_Recursive\_Mutable\_Load permission is
  described, which, if not present, causes further capabilities loaded via
  that capability to be loaded without store permissions
  (see Section~\ref{app:exp:recmutload}).

\item We have added a new experimental \insnref{CLoadTags}
  instruction that allows tags to be loaded for a cache line without pulling
  data into the cache.

\item A new experimental \textit{sealed entry capability} feature is
  described, which permits entry via jump but otherwise do not allow
  dereferencing (later editions considered these no longer experimental, and
  so they are described in \cref{sec:arch-sentry}).
  These are similar to enter capabilities from the
  M-Machine~\cite{carter:mmachine94}, and could provide utility in providing
  further constraints on capability use for the purposes of memory protection
  -- e.g., in the implementation of C++ v-tables.

\item A new experimental \textit{memory type token} feature is described,
  which provides an alternative mechanism to object types within pairs of
  sealed capabilities (Section~\ref{app:exp:typetoken}).

\end{itemize}


\item[7.0-ALPHA3]
This version of the \textit{CHERI Instruction-Set Architecture} is an interim
version distributed for review by DARPA and our collaborators:

\begin{itemize}

\item The CHERI Concentrate capability compression format is now documented,
  with a more detailed rationale section than the prior CHERI-128 section.

\item The \insnref{CLCBI} (Capability Load Capability with Big Immediate)
  instruction, which accelerates position-independent access to global
  variables, is no longer considered experimental.

\item The architecture-neutral description of tagged memory has been
  clarified.

\item The maximum supported lengths for both compressed and uncompressed
  capabilities has been updated: $2^{64}$ for 128-bit +capabilities, and
  $2^{64} - 1$ for 256-bit capabilities.

\item It is clarified that \insnref{CLoadTags} instruction must provide
  cache coherency consistent with other load instructions.
  We recommend ``non-temporal'' behavior, in which unnecessary cache-line
  fills are avoided to limit cache pollution during revocation.

\item We now define the object type for unsealed capabilities, returned by
  the \insnref{CGetType} instruction, as $2^{64}-1$ rather than $0$.

\item An experimental section has been added on how CHERI capabilities might
  compose with memory-versioning schemes such as Sparc ADI and Arm MTE
  (see Section~\ref{app:exp:versioning}).

\item Pseudocode throughout the CHERI ISA specification is now generated from
  our Sail formal model of the CHERI-MIPS ISA~\cite{sail-popl2019}.

\item The \hyperref[glossary]{Glossary} has been updated for CHERI ISAv7
  changes including CHERI-RISC-V, split vs. merged register files,
  capabilities for physical addresses, and special capability registers.

\item Capability exception codes are now shared across architectures.

\item CHERI-RISC-V now includes capability-relative floating-point load and
  store instructions.
  We have clarified that existing RISC-V floating-point load and store
  instructions are constrained by \DDC{}.

\item CHERI-RISC-V now throws exceptions, rather than clearing tags, when
  non-mono\-tonic register-to-register capability operations are attempted.

\item While a specific encoding-mode transition mechanism is not yet specified
  for CHERI-RISC-V, candidate schemes are described and compared in greater
  detail.

\item CHERI-RISC-V's ``capability encoding mode'' now has different impacts
  for uncompressed instructions vs. compressed instructions: In the compressed
  ISA, jump instructions also become capability relative.

\item CHERI-RISC-V page-table entries now contain a ``capability dirty bit''
  to assist with tracking the propagation of capabilities.

\item Throwing an exception on an out-of-bounds capability-relative jump
  rather than on the target fetch is now more clearly explained: This improves
  debuggability by maintaining precise information about context state on
  jump, whereas after the jump, bounds may not be representable due to
  capability compression.
  When an inappropriate \EPCC{} is installed, the exception will still be
  thrown on instruction fetch.

\item A new \ErrorEPCC{} special register has been defined, to assist with
  exceptions thrown within exception handlers; its behavior is modeled on the
  existing MIPS \ErrorEPC{} special register.

\end{itemize}


\item[7.0-ALPHA4]
This version of the \textit{CHERI Instruction-Set Architecture} is an interim
version distributed for review by DARPA and our collaborators:

\begin{itemize}
\item We have added new instructions \insnref{CSetAddr} (Set capability
  address to value from register), \insnref{CAndAddr} (Mask address of
  capability -- experimental), and \insnnoref{CGetAndAddr} (Move capability
  address to an integer register, with mask -- experimental), which optimize
  common virtual-address-related operations in language runtimes such as
  WebKit's Javascript engine.
  These instructions cater better to a language mapping from C's
  \ccode{intptr_t} type to the virtual address, rather than offset, of a
  capability, which has been our focus previously.
  These complement the previously added \insnref{CGetAddr} that allows
  easier compiler access to a capability's virtual address.

\item We have added two new experimental instructions, \insnref{CRAM}
  (Retrieve Mask to Align Capabilities to Precisely Representable Address) and
  \insnref{CRRL} (Round to Next Precisely Representable Value), which
  allow software to retrieve alignment information for the base and length for
  a proposed set of bounds.

\item \insnref{CMove}, which was previously an assembler pseudo-operation
  for \insnref{CIncOffset}, is now a stand-alone instruction.
  This avoids the need to special case sealed capabilities when
  \insnref{CIncOffset} is used solely to move, not to modify, a
  capability.

\item The names of the instructions \insnnoref{CSetBoundsImmediate} and
  \insnnoref{CIncOffsetImmediate} have been shortened to
  \insnref{CSetBoundsImm} and \insnref{CIncOffsetImm}.

\item The instructions \insnnoref{CCheckType} and \insnnoref{CCheckPerm}
  have been deprecated, as they have not proven to be particularly useful in
  implementing multi-protection-domain systems.

\item We have added a new pseudo-operation, \insnnoref{CAssertInBounds},
  described in \cref{\insnmipslabelname{cassertinbounds}}, allows an exception
  to be thrown if the address of a capability is not within bounds.

\item The instruction \insnnoref{CCheckTag} has now been assigned an opcode.

\item We have revised the encodings of many instructions in our draft
  CHERI-RISC-V specification in Appendix~\ref{app:isaquick-riscv}.

\item We more clearly specify that when a special register write occurs to
  \EPC{}, the result is similar to \insnref{CSetOffset} but with the tag
  bit stripped, in the event of a failure, rather than an exception being
  thrown.

\item We have added a reference to our TaPP 2018 paper, \textit{Pointer
  Provenance in a Capability Architecture}, which describes how architectural
  traces of pointer behavior, visible through the CHERI instruction set, can
  be analyzed to understand software and structure.

\item We have added a reference to our ICCD 2018 paper, \textit{CheriRTOS:
  A Capability Model for Embedded Devices}, which describes an embedded
  variant of CHERI using 64-bit capabilities for 32-bit addresses, and how
  embedded real-time operating systems might utilize CHERI features.

\item We have revised our description of conventions for capability values,
  including when used as pointers, to hold integers, and for NULL value, to
  more clearly describe their use.
  We more clearly describe the requirements for the in-memory
  representation of capabilities, such as a zeroed NULL capability so that
  BSS behaves as desired.
  We provide more clear architecture-neutral explanations of pointer
  dereferencing, capability
  permissions and their composition, the namespaces protected by capability
  permissions, exception handling, exception priorities, virtual memory, and
  system reset.
  These definitions appear in Chapter~\ref{chap:architecture}.
  Chapter~\ref{chap:cheri-mips}, which describes CHERI-MIPS, has been
  shortened as a variety of content has been made architectural neutral.

\item More detailed rationale is provided for our composition of CHERI with
  the MIPS exception-handling model.

\item We are more careful to use the term ``pointer'' to refer to the
  C-language type, verses integer or capability values that maybe used by the
  compiler to implement pointers.

\item With the advent of ISA variations utilizing a merged register file, we
  are more careful to differentiate integer registers from general-purpose
  registers, as general-purpose registers may also hold capabilities.

\item We more clearly define the terms ``upper bound'' and ``lower bound''.

\item We now more clearly describe the effects of our \textit{principle of
  intentionality} on capa\-bility-aware instruction design in
  Section~\ref{sec:capability-aware-instructions}.

\item We better describe the rationale for tagged capabilities in registers
  and memory, in contrast to cryptographic and probabilistic protections, in
  Section~\ref{sec:probablistic_capability_protection}.

\item We have made a number of improvements to the CHERI-x86-64 sketch,
  described in Chapter~\ref{chap:cheri-x86-64}, to improve realism around trap
  handling and instruction design.

\item We have rewritten our description of the interaction between CHERI and
  Direct Memory Access (DMA) in Section~\ref{sec:dma}. to more clearly
  describe tag-stripping and capability-aware DMA options.

\end{itemize}


\item[7.0]
This version of the \textit{CHERI Instruction-Set Architecture} is a full
release of the Version 7 specification:

\begin{itemize}
\item We have now deprecated the CHERI-128 capability compression format, in
  favor of CHERI Concentrate.

\item The RISC-V \insnnoref{AUIPC} instruction now returns a
  \PCC{}-relative capability in the capability encoding mode.

\item Capabilities now contain a \cflags{} field (\cref{sec:model-flags}),
  which will hold state that
  can be changed without affecting privilege.
  Corresponding experimental \insnref{CGetFlags} and
  \insnref{CSetFlags} instructions have been added.

\item The capability encoding-mode bit in CHERI-RISC-V is specified as a bit
  in the \cflags{} field of a capability.
  The current mode is defined as the flag bit in the currently installed
  \PCC{}.
  Design considerations and other potential options are described in
  Chapter~\ref{chap:rationale}.

\item We now more explicitly describe the reset states of special- and
  general-purpose capability registers for CHERI-MIPS and CHERI-RISC-V.

\item Compressed capabilities now contain a dedicated \cotype{} field that
  always holds an object type (see
  \cref{sec:model-object-types,section:object-type}), rather than stealing
  bounds bits for object type when sealing.  Now, any representable capability
  may be sealed.  Several object type values are reserved for architectural
  experimentation (see \cref{tab:archotypes}).

\item More detail is provided regarding the integration of CHERI Concentrate
  with special registers, its alignment requirements, and so on.

\item Initial discussion of a disjoint capability tree for physical
  addresses and hardware facilities using these has been added to
  the experimental appendix, in \cref{app:exp:physcap}.

\item Initial discussion of a hybrid 64/128-bit capability design has been
  added to the experimental appendix, in \cref{sec:windowedshortcaps}.

\item We have added formal Sail instruction semantics for CHERI-RISC-V; this
  is currently in Appendix~\ref{app:isaquick-riscv}.

\item We have added a reference to our IEEE TC 2019 paper, \textit{CHERI
  Concentrate: Practical Compressed Capabilities}, which describes our current
  approach to capability compression.

\item We have added a reference to Alexandre Joannou's PhD dissertation,
  \textit{High-perform\-ance memory safety: optimizing the CHERI capability
  machine}, which describes approaches to improving the efficiency of
  capability compression and tagged memory.

\end{itemize}


\item[8.0]
This version of the \textit{CHERI Instruction-Set Architecture} is a full
release of the Version 8 specification:

\begin{itemize}
\item We have performed modest updates to discuss Arm's Morello processor,
  System-on-Chip (SoC), and board.
  The authoritative reference to Morello is Arm's Morello
  specification~\cite{arm-morello}.

\item We have added a new chapter, Chapter~\ref{chap:microarchitecture},
  describing the impact on CHERI on practical microarchitecture at a high
  level.
  It considers topics such as the impact of capabilities on the pipeline and
  register file, efficient implementation of bounds compression and
  decompression, fast bounds checking, tagged memory, and the potential
  interaction with speculative execution.
  This includes insights gained during the building of Arm's Morello
  processor and SoC.

\item Shift away from the idea that the fully precise 256-bit capabilities
  are the essential model for CHERI.
  Instead, describe capabilities as an architectural type made up of a set
  of architectural fields, which may be constrained in terms of precision, and
  that have an in-memory representation.
  The microarchitecture may hold capabilities in another format internally
  (e.g., when loaded in registers).

\item The CHERI Concentrate description has been improved, including adding
  information about the Fast Representability Check.
  A number of constants have been updated.

\item In several chapters, more care is taken when using the words
  ``capability,'' ``pointer,'' and ``address,'' which are not interchangeable.

\item Throughout, be more clear that CHERI applies to 32-bit architectures,
  not just 64-bit architectures.
  In Chapter~\ref{chap:architecture}, introduce \xlen{} and \clen{}
  terminology previously only used for CHERI-RISC-V, to better abstract away
  from specific address lengths.

\item We are now more clear that capabilities may describe physical as well as
  virtual addresses, and that virtual addressing may be implemented either
  using software-managed TLBs (as on MIPS) or via architectural page tables
  (as on RISC-V and ARMv8-A).

\item The \insnnoref{CCall} instruction has been replaced with
  \insnref{CInvoke}, which does not have a selector mechanism and supports
  only exception-free domain transition.
  The prior exception-based mechanism was introduced as scaffolding used
  during domain-transition research, and the exception-free mechanism is our
  preferred approach.
  We have removed capability exception cause codes previously used
  for that purpose, and will likely deprecate \insnnoref{CSetCause}, which was
  used only for exception-based domain transition.

\item The deprecated \insnnoref{CCheckPerm} and \insnnoref{CCheckType}
  instructions have been removed as they were intended to be used with the
  exception-based \insnnoref{CCall} mechanism.

\item We have removed the experimental 64-bit capability format based on our
  older CHERI-128 compression model.
  We now document a 64-bit capability format as part of CHERI Concentrate in
  Section~\ref{section:architectural-capabilities}.

\item We now advocate for a policy of specifying a minimum capability bounds
  precision, even with \insnref{CRRL} and \insnref{CRAM} instructions; see
  Section~\ref{sec:the-value-of-architectural-minimum-precision}.

\item We now more fully explore, and explain, exception throwing around
  capability load and store permissions on capabilities and on page-table /
  TLB entries.
  MMU-originated exceptions are now distinct from CHERI exceptions.
  Whereas a capability load through a capability without \cappermLC will always strip
  the tag, a capability load via a page without load capability permission
  will either strip the tag or throw an exception.
  Similarly, a store via a page without permission to store a capability will
  throw an exception if the tag bit is set on the capability being stored.

\item We now discuss the use of per-page capability load barriers to
  enable efficient capability revocation and garbage collection in
  Section~\ref{section:capability-load-barriers}.  Load
  barriers use additional MMU permissions to selectively trap on
  capability loads.

\item \insnnoref{CPtrCmp} has been changed to compare only the addresses of
  the two capabilities, and no longer consider the tag bit, for non-exact
  comparisons.
  The previous behavior could result in surprising run-time failures of C/C++
  programs.

\item Sentry capabilities are no longer considered experimental.

\item The instruction \insnref{CSealEntry} has been added to the ISA quick
  references, from which it was missing.

\item We now more completely elaborate how sentry capabilities interact with
  exception handling, including automatic unsealing of sentries when installed
  in exception program counters, not just when they are jumped to.

\item Two new instructions (\insnnoref{CGetPCCIncOffset} and
  \insnnoref{CGetPCCSetAddr}) have been added to improve code density when
  generating code to create \PCC{}-derived capabilities for code or data
  access.
  This is of particular use when utilizing sentry capabilities.

\item The \insnref{CRRL} and \insnref{CRAM} instructions, which assist with
  capability bounds alignment and padding, are no longer considered
  experimental.

\item The \insnref{CLoadTags} instruction is no longer considered
  experimental, and has now also been defined for CHERI-RISC-V.

\item We have removed a note on \insnnoref{CSeal}
  regarding representable unsealed capabilities being unrepresentable as
  sealed capabilities: all representable capabilities can now be sealed.

\item We have a removed a note on \insnnoref{CIncOffset} regarding a special
  case in which offset of 0 is permitted to operate on sealed capabilities,
  allowing \insnnoref{CIncOffset} to be used to implement
  \insnnoref{CMove} as a pseudo-instruction.
  This is no longer required, as \insnnoref{CMove} is now its own
  instruction to avoid this special casing.

\item \insnnoref{CIncOffset} is no longer specified to clear the base and
  length if the bounds become unrepresentable, as we guarantee that the cursor
  will hold the arithmetic result rather than the offset.

\item \insnref{CSetFlags} no longer incorrectly notes that an exception
  is thrown if the argument is untagged.

\item We have added a new chapter, Chapter~\ref{chap:isaref-riscv}, listing the
  semantics and encodings of CHERI-RISC-V instructions.

\item Capability flags, and their associated instructions,
  \insnref{CGetFlags} and \insnref{CSetFlags}, are no longer considered
  experimental.
  On CHERI-RISC-V, a capability flag is used to control what instruction
  endcoding mode is active.
  For details, see
  \cref{sec:model-flags,sec:arch-flags,sec:cheri-riscv-encmodes}.

\item Certain CHERI-RISC-V SCRs are now defined to exist only when their
  corresponding extensions (e.g., the N extension) is present.

\item In CHERI-RISC-V, when a special capability register is written using
  \insnriscvref{CSpecialRW}, and some bits in the register are defined as WARL
  (i.e., may be modified during the write), the tag bit will be cleared if the
  capability value is sealed.

\item A new Unaligned Base CHERI exception code has been defined, allowing
  CHERI-RISC-V to throw an exception if the installed \PCC{} value has an
  unaligned base.

\item CHERI-RISC-V encodings are now defined for \insnriscvref{CRRL} and
  \insnriscvref{CRAM}, and encoding space is reserved for a future implementation of
  \insnriscvref{CClear} when using a split register file.

\item CHERI-RISC-V encodings have been added for the
  \insnriscvref{CSetEqualExact}, \insnriscvref{CLoadTags}, and
  \insnriscvref{CClearTags} instructions.

\item The CHERI-RISC-V exception cause code has changed from 0x20 to 0x1c
  due to a collision with encoding space reserved for future use.

\item We now define a set of CHERI-RISC-V atomic instructions corresponding to
  the equivalent base RISC-V atomic instructions.

\item We now document which RISC-V CSRs and FCSRs are white listed to not
  require \cappermASR, such as the cycle counter.

\item Capability exception cause codes are now properly architecture neutral,
  but the mechanism for obtaining them is more accurately architecture
  specific.

\item CHERI-RISC-V now reports capability-related exception information via
  the existing RISC-V Trap Value CSRs, \xtval{}, rather than through a new
  capability cause register (a design inherited from CHERI-MIPS that is less
  consistent with the baseline RISC-V design).

\item We added two new exception codes for CHERI-related MMU faults to
  CHERI-RISC-V.  For these exceptions, \xtval{} holds the address of
  the faulting memory reference as with existing MMU faults on RISC-V.

\item We added additional PTE bits to CHERI-RISC-V to support
  capability revocation.  \texttt{CD} provides a capability dirty bit
  to track pages holding capabilities.  \texttt{CRM} and \texttt{CRG}
  enable per-page capability load barriers.

\item We document the CHERI-RISC-V \insnriscvref{CRET}, \insnriscvref{CJR}, and
  \insnriscvref{CJALR} assembly aliases.

\item We have added a CHERI-RISC-V assembly programming section to the
  CHERI-RISC-V ISA quick reference.

\item There have been numerous updates to our CHERI-x86-64 architectural
  sketch.  The page table permission bits have been adjusted to avoid
  conflicting with the Protection Keys extension.  Violations of the
  CHERI page table permissions now raise a page fault rather than a
  CHERI fault.  The read capability permission now faults if violated
  rather than stripping tags.  We have removed an ambiguous suggestion
  for handling RIP-relative addressing.

\item The interaction of CHERI with existing memory versioning schemes
  (e.g., Arm MTE) in \cref{app:exp:versioning} is now more fully articulated
  and includes support for integer-pointer versioning instructions as well as
  a new atomic instruction to manipulate version values in memory.

\item A new experimental instruction, \insnref{CLCNT}, has been proposed to
  perform a non-temporal (streaming) load of a capability via a capability.
  This instruction may prove useful when scanning memory for capabilities in
  order to implement revocation.
  We have not yet validated this approach through a full-stack implementation.

\item A new experimental instruction, \insnref{CClearTags}, has been proposed
  to perform fast zeroing of multiple tags in memory, and will not allocate
  cache lines if data is not already present in a cache.
  This instruction may prove useful when rapidly clearing capabilities for
  revocation purposes, in the absence of data confidentiality requirements.
  We have not yet validated this approach through a full-stack implementation.

\item A new experimental protection-domain transition mechanism,
  \textit{sealed indirect enter capabilities}, has been proposed to allow
  a sealed entry capability to carry not just a pointer to domain-specific
  code, but also to domain-specific data, by adding an additional level of
  indirection.
  A new instruction, \insnnoref{CInvokeInd}, would be used to invoke a
  sealed indirect enter capability.
  We have not yet validated this approach through a full-stack implementation.
  This is described in Section~\ref{app:exp:indsentry}.

\item A new \emph{ephemeral capability} type is proposed.
  Ephemeral capabilities can be held in registers but not stored to memory,
  and we consider the implications for fast cross-domain calls, and an
  explicitly maintained delegation tree to provide prompt revocation of
  delegation sub-trees.
  We have not yet validated this approach through a full-stack implementation.
  This is described in Section~\ref{app:exp:hierarchal-evocation}.

\item A new \emph{anti-tamper seal} mechanism is proposed, to allow
  validation that a delegated capability that has been returned has not been
  modified (other than its address) while in use.
  The suggested use case is around memory allocators, to ensure that a pointer
  passed to free is consistent with the pointer originally allocated.
  We have not yet validated this approach through a full-stack implementation.
  This is described in Section~\ref{sec:anti-tamper}.

\item We now describe potential capability-related prefetch instructions in
  Section~\ref{sec:caching-and-explicit-prefetch}, with specific
  consideration of side-channel attacks.
  We also explicitly specify \DDC{} bounds checking on the MIPS
  \insnnoref{PREF} prefetch instruction in
  Section~\ref{sec:mips-prefetch}.

\item We now reference our papers \textit{CHERIvoke: Characterising Pointer
  Revocation using CHERI Capabilities for Temporal Memory Safety} (IEEE MICRO
  2019), \textit{Rigorous engineering for hardware security: Formal modelling
  and proof in the CHERI design and implementation process} (IEEE SSP 2020),
  and \textit{Cornucopia: Temporal Safety for CHERI Heaps} (IEEE SSP 2020).

\end{itemize}


\item[9.0]
This version of the \textit{CHERI Instruction-Set Architecture} is a full
release of the Version 9 specification:

\begin{itemize}
\item We have shifted to CHERI-RISC-V as our primary reference
  platform instead of CHERI-MIPS.  This included several changes to
  Chapter~\ref{chap:model} and Chapter~\ref{chap:architecture} to
  replace MIPS-specific details with more architectural-neutral
  concepts.
  Section~\ref{section:protection-domain-transition-with-cinvoke} was
  also moved to Chapter~\ref{chap:architecture}.

\item The privileged architecture portions of CHERI-RISC-V are now
  defined as an extension to version 1.11 of the RISC-V privileged
  architecture specification.

\item CHERI-RISC-V reports capability exception details in \xtval{}
  rather than \xccsr{}.

\item The RISC-V \insnnoref{JAL} and \insnnoref{JALR} instructions are
  now mode-dependent meaning that they use capability register
  operands in capability mode rather than always using integer
  registers.  The capability mode version of these instructions are
  named \insnref{CJAL} and \insnref{CJALR}.  The previous
  \insnnoref{CJALR} instruction has been renamed to
  \insnref{JALR.CAP}.  In addition, \insnref{JALR.PCC} has been added
  to permit integer jump and links in capability mode.

\item Section~\ref{subsection:compressed-instructions} has been
  rewritten to reflect an initial implementation of CHERI-RISC-V
  compressed instructions in capability encoding mode.

\item Opcode encodings have been reserved for CHERI-RISC-V memory
  versioning instructions as well as \insnnoref{CRelocate}.

\item CHERI-RISC-V always uses a merged register file and the
  architecture-neutral chapters now assume a merged register file on
  all CHERI architectures.  This included removing the dirty bit from
  \xccsr{} as well as the \insnnoref{CGetAddr}, \insnnoref{Clear}, and
  \insnnoref{CSub} instructions.

\item CHERI-RISC-V clears tags rather than raising exceptions for
  non-monotonic modifications to capabilities.

\item Added \insnref{CGetHigh} and \insnref{CSetHigh} to retrieve and
  modify the upper half of a capability.

\item Added \insnref{CGetTop} to retrieve the upper limit of a
  capability.

\item \DDC{} and \PCC{} no longer relocate legacy memory accesses.
  These registers still constrain legacy memory accesses.  This
  included deprecating \insnref{CFromPtr} and \insnref{CToPtr}.

\item Removed CHERI-MIPS from the specification as it is deprecated
  and no longer actively developed.

\item Added a new section in Chapter~\ref{chap:model} describing
  potential uses of capabilities to protect physical addresses.

\item CHERI-RISC-V now enables/disables CHERI extensions via a bit in
  the \menvcfg{} and \senvcfg{} CSRs rather than \xccsr{}.

\item CHERI-RISC-V \xScratchC{} capability registers now extend the
  existing \xscratch{} registers.

\item We have expanded the CHERI-x86-64 sketch in
  Chapter~\ref{chap:cheri-x86-64} to include details on extensions to
  existing instructions to support operations on capabilities as well
  as details for new instructions in a new ISA reference in
  Chapter~\ref{chap:isaref-x86-64}.  The chapter also contains a new
  section evaluating recent security extensions to x86 and how they
  would compose with CHERI.

\item Added a description of the 64-bit CHERI Concentrate capability
  format.
\end{itemize}


\end{description}
