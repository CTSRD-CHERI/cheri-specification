This version of the \textit{CHERI Instruction-Set Architecture} is an interim
version distributed for review by DARPA and our collaborators:

\begin{itemize}
\item We have added new instructions \insnref{CSetAddr} (Set capability
  address to value from register), \insnref{CAndAddr} (Mask address of
  capability -- experimental), and \insnnoref{CGetAndAddr} (Move capability
  address to an integer register, with mask -- experimental), which optimize
  common virtual-address-related operations in language runtimes such as
  WebKit's Javascript engine.
  These instructions cater better to a language mapping from C's
  \ccode{intptr_t} type to the virtual address, rather than offset, of a
  capability, which has been our focus previously.
  These complement the previously added \insnref{CGetAddr} that allows
  easier compiler access to a capability's virtual address.

\item We have added two new experimental instructions, \insnref{CRAM}
  (Retrieve Mask to Align Capabilities to Precisely Representable Address) and
  \insnref{CRRL} (Round to Next Precisely Representable Value), which
  allow software to retrieve alignment information for the base and length for
  a proposed set of bounds.

\item \insnref{CMove}, which was previously an assembler pseudo-operation
  for \insnref{CIncOffset}, is now a stand-alone instruction.
  This avoids the need to special case sealed capabilities when
  \insnref{CIncOffset} is used solely to move, not to modify, a
  capability.

\item The names of the instructions \insnnoref{CSetBoundsImmediate} and
  \insnnoref{CIncOffsetImmediate} have been shortened to
  \insnref{CSetBoundsImm} and \insnref{CIncOffsetImm}.

\item The instructions \insnnoref{CCheckType} and \insnnoref{CCheckPerm}
  have been deprecated, as they have not proven to be particularly useful in
  implementing multi-protection-domain systems.

\item We have added a new pseudo-operation, \insnnoref{CAssertInBounds},
  described in \cref{\insnmipslabelname{cassertinbounds}}, allows an exception
  to be thrown if the address of a capability is not within bounds.

\item The instruction \insnnoref{CCheckTag} has now been assigned an opcode.

\item We have revised the encodings of many instructions in our draft
  CHERI-RISC-V specification in Appendix~\ref{app:isaquick-riscv}.

\item We more clearly specify that when a special register write occurs to
  \EPC{}, the result is similar to \insnref{CSetOffset} but with the tag
  bit stripped, in the event of a failure, rather than an exception being
  thrown.

\item We have added a reference to our TaPP 2018 paper, \textit{Pointer
  Provenance in a Capability Architecture}, which describes how architectural
  traces of pointer behavior, visible through the CHERI instruction set, can
  be analyzed to understand software and structure.

\item We have added a reference to our ICCD 2018 paper, \textit{CheriRTOS:
  A Capability Model for Embedded Devices}, which describes an embedded
  variant of CHERI using 64-bit capabilities for 32-bit addresses, and how
  embedded real-time operating systems might utilize CHERI features.

\item We have revised our description of conventions for capability values,
  including when used as pointers, to hold integers, and for NULL value, to
  more clearly describe their use.
  We more clearly describe the requirements for the in-memory
  representation of capabilities, such as a zeroed NULL capability so that
  BSS behaves as desired.
  We provide more clear architecture-neutral explanations of pointer
  dereferencing, capability
  permissions and their composition, the namespaces protected by capability
  permissions, exception handling, exception priorities, virtual memory, and
  system reset.
  These definitions appear in Chapter~\ref{chap:architecture}.
  Chapter~\ref{chap:cheri-mips}, which describes CHERI-MIPS, has been
  shortened as a variety of content has been made architectural neutral.

\item More detailed rationale is provided for our composition of CHERI with
  the MIPS exception-handling model.

\item We are more careful to use the term ``pointer'' to refer to the
  C-language type, verses integer or capability values that maybe used by the
  compiler to implement pointers.

\item With the advent of ISA variations utilizing a merged register file, we
  are more careful to differentiate integer registers from general-purpose
  registers, as general-purpose registers may also hold capabilities.

\item We more clearly define the terms ``upper bound'' and ``lower bound''.

\item We now more clearly describe the effects of our \textit{principle of
  intentionality} on capa\-bility-aware instruction design in
  Section~\ref{sec:capability-aware-instructions}.

\item We better describe the rationale for tagged capabilities in registers
  and memory, in contrast to cryptographic and probabilistic protections, in
  Section~\ref{sec:probablistic_capability_protection}.

\item We have made a number of improvements to the CHERI-x86-64 sketch,
  described in Chapter~\ref{chap:cheri-x86-64}, to improve realism around trap
  handling and instruction design.

\item We have rewritten our description of the interaction between CHERI and
  Direct Memory Access (DMA) in Section~\ref{sec:dma}. to more clearly
  describe tag-stripping and capability-aware DMA options.

\end{itemize}
