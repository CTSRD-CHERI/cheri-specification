\clearpage
\phantomsection
\addcontentsline{toc}{subsection}{CSetAddr}
\insnmipslabel{csetaddr}
\subsection*{CSetAddr: Set the Address of Capability}

\subsubsection*{Format}

CSetAddr cd, cb, rt

\begin{center}
\cherithreeop[header]{0x22}{cd}{cb}{rt}
\end{center}

\subsubsection*{Description}

\emph{cd} is set to \emph{cb} with \emph{cb}.\caddr{} set to \emph{rt}.
If changing the address causes the capability to become unrepresentable, then an untagged capability with the requested address is returned.

\subsubsection*{Semantics}

\sailMIPScode{CSetAddr}

\subsubsection*{Exceptions}

A coprocessor 2 exception is raised if:

\begin{itemize}
\item
\emph{cb}.\ctag{} is set and \emph{cb} is sealed.
\end{itemize}

\subsubsection*{Notes}

\begin{itemize}
\item This instruction may be useful, in combination with \insnmipsref{CGetAddr}, when C is manipulating pointers in ways that require a round trip through integer registers.
\item This instruction is also useful for \ccode{uintptr\_t} arithmetic when using an address interpretation of capabilities. When interpreting \ccode{uintptr\_t} as offsets relative
to the base, the compiler will use \insnmipsref{CGetOffset} and \insnmipsref{CSetOffset} instead.

\end{itemize}
