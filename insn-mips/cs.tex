\clearpage
\phantomsection
\addcontentsline{toc}{subsection}{CS[BHWD]}
\insnmipslabel{csbhwd}
\subsection*{CS[BHWD]: Store Integer via Capability}

\subsubsection*{Format}

CSB rs, rt, offset(cb)\\
CSH rs, rt, offset(cb)\\
CSW rs, rt, offset(cb)\\
CSD rs, rt, offset(cb)\\
CSBR rs, rt(cb)\\
CSHR rs, rt(cb)\\
CSWR rs, rt(cb)\\
CSDR rs, rt(cb)\\
CSBI rs, offset(cb)\\
CSHI rs, offset(cb)\\
CSWI rs, offset(cb)\\
CSDI rs, offset(cb)\\

\begin{center}
\begin{bytefield}{32}
\bitheader[endianness=big]{0,1,3,10,11,15,16,20,21,25,26,31}\\
\bitbox{6}{0x3A}
\bitbox{5}{rs}
\bitbox{5}{cb}
\bitbox{5}{rt}
\bitbox{8}{offset}
\bitbox{1}{0}
\bitbox{2}{t}
\end{bytefield}
\end{center}

\usesDDCinsteadofNULL{cb}

\subsubsection*{Purpose}

Stores some or all of a register into a memory location.

\subsubsection*{Description}

Part of integer register \emph{rs} is stored to the memory location specified by
\emph{cb.\cbase{}} + \emph{cb.\coffset{}} + \emph{rt} + $2^t$ $*$ \emph{offset}.
Capability register \emph{cb} must contain a capability that grants permission
to store data.

The \emph{t} field determines how many bits of the register are stored to memory:

\begin{description}
	\item[0] byte (8 bits)
	\item[1] halfword (16 bits)
	\item[2] word (32 bits)
	\item[3] doubleword (64 bits)
\end{description}

If less than 64 bits are stored, they are taken from the least-significant
end of the register.

\subsubsection*{Semantics}
\sailMIPScode{CStore}

\subsubsection*{Exceptions}

A coprocessor 2 exception is raised if:

\begin{itemize}
\item
\cchecktag{}
\item
\emph{cb} is sealed.
\item
\emph{cb}.\cperms.\emph{Permit\_Store} is not set.
\item
\emph{addr} $+$ \emph{size} $>$ \emph{cb}.\cbase{} $+$ \emph{cb}.\clength{}.
\item
\emph{addr} $<$ \emph{cb}.\cbase{}
\end{itemize}

An address error during store (AdES) is raised if:

\begin{itemize}
\item
\emph{addr} is not aligned.
\end{itemize}

\subsubsection*{Notes}

\begin{itemize}
\item
This instruction reuses the opcode from the Store Word from Coprocessor 2
(\insnnoref{SWC2}) instruction in the MIPS Specification.
\item
\emph{rt} is treated as an unsigned integer.
\item
\emph{offset} is treated as a signed integer.
% \item
% The computation of \emph{addr} does not wrap around modulo $2^{64}$.
\item
BERI1 has a compile-time option to allow unaligned loads and stores. If BERI1
is built with this option, an unaligned store will only raise an exception if
it crosses a cache line boundary.
\end{itemize}
