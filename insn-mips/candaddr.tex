\clearpage
\phantomsection
\addcontentsline{toc}{subsection}{CAndAddr}
\insnmipslabel{candaddr}
\subsection*{CAndAddr: Mask Address of Capability}

\subsubsection*{Format}

CAndAddr cd, cb, rs

\begin{center}
\cherithreeop[header]{0x24}{cd}{cb}{rs}
\end{center}

\subsubsection*{Description}

\emph{cd} is set to \emph{cb} with \emph{cb}.\caddr{} logically ANDed with \emph{rs}.
If the new address causes the capability to become unrepresentable then an untagged capability with address set to the masked address is returned.

\subsubsection*{Semantics}

\sailMIPScode{CAndAddr}

\subsubsection*{Exceptions}

A coprocessor 2 exception is raised if:

\begin{itemize}
\item
\emph{cb}.\ctag{} is set and \emph{cb} is sealed.
\end{itemize}

\subsubsection*{Notes}

\begin{itemize}
\item This instruction may be appropriate for use cases in which C is masking pointers
before use, for example when the lower bits are used to store additional metadata in a pointer with known alignment.
\item This instructions also allows more efficient aligning down (and up) of capabilities. In C source code this will
almost always be a compile time constant so we might discover that we need a version of this instruction that
takes an immediate operand. Using a 5-bit immediate field we could encode align-down/align-up masks from
$2$ to $2^{32}$.
\item This instruction is also useful in order to create precisely representable bounds for heap and stack allocations.
\end{itemize}
