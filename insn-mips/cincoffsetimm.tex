\clearpage
\phantomsection
\addcontentsline{toc}{subsection}{CIncOffsetImm}
\insnmipslabel{cincoffsetimm}
\subsection*{CIncOffsetImm: Increment Offset by Immediate}

\subsubsection*{Format}

CIncOffset cd, cb, increment$_{imm}$

\begin{center}
\begin{bytefield}{32}
\bitheader[endianness=big]{0,10,11,15,16,20,21,25,26,31}\\
\bitbox{6}{0x12}
\bitbox{5}{0x13}
\bitbox{5}{cd}
\bitbox{5}{cb}
\bitbox{11}{increment$_{imm}$}
\end{bytefield}
\end{center}

\arnote{The assembler supports both CIncOffset and CIncOffsetImm but I think we should always use CIncOffset}

\subsubsection*{Description}

Capability register \emph{cd} is set equal to capability register
\emph{cb} with its \coffset{} field replaced with \emph{cb}.\coffset{} $+$
\emph{increment$_{imm}$}.

If the requested \cbase{}, \clength{} and \coffset{} cannot be represented exactly,
then \emph{cd}.\ctag{} is cleared and \emph{cd}.\ccursor{} is set equal to
\emph{cb}.\ccursor{} $+$ \emph{increment$_{imm}$}.

\subsubsection*{Semantics}
\sailMIPScode{CIncOffsetImmediate}

\subsubsection*{Exceptions}

A coprocessor 2 exception is raised if:

\begin{itemize}
\item
\emph{cb}.\ctag{} is set and \emph{cb} is sealed.
\end{itemize}

\subsubsection*{Notes}

\begin{itemize}
\item
For security reasons, \insnmipsref{CIncOffsetImm} must not change the offset
of a sealed capability.
\item
If the tag bit is not set, and the offset is being used to hold an integer,
then \insnmipsref{CIncOffsetImm} should still increment the offset. This is
so that \insnmipsref{CIncOffsetImm} can be used to implement increment of
a \ccode{intcap\_t} type.
Because the tag is unset, \insnmipsref{CIncOffset}
will ignore the decoded \cotype{} and so will not attempt to enforce
immutability of detagged sealed capabilities.
(The precise effect of \insnmipsref{CIncOffset} on such
non-capability data will depend on which binary representation of
capabilities is being used.)
\item
If the tag bit is not set, and capability compression is in use,
the arbitrary data in \emph{cb} might not decompress to sensible values
of the \cbase{} and \clength{} fields, and there is no guarantee that
retaining these values of \cbase{} and \clength{} while changing
\coffset{} will result in a representable value.

From a software perspective, the requirement is that increasing \coffset{}
on an untagged capability will work if \cbase{} and \clength{} are zero. (This
is how integers, and pointers that have lost precision, will be represented).
If \cbase{} and \clength{} have non-zero values (or \emph{cb} cannot be
decompressed at all), then the values of \cbase{} and \clength{} after this
instruction are \textbf{UNPREDICTABLE}.
\end{itemize}
