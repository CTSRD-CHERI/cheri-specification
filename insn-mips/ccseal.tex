\clearpage
\phantomsection
\addcontentsline{toc}{subsection}{CCSeal}
\insnmipslabel{ccseal}
\subsection*{CCSeal: Conditionally Seal a Capability}

\subsubsection*{Format}

CCSeal cd, cs, ct

\begin{center}
\begin{bytefield}{32}
\bitheader[endianness=big]{0,5,6,10,11,15,16,20,21,25,26,31}\\
\bitbox{6}{0x12}
\bitbox{5}{0x0}
\bitbox{5}{cd}
\bitbox{5}{cs}
\bitbox{5}{ct}
\bitbox{6}{0x1f}
\end{bytefield}
\end{center}

\subsubsection*{Description}

If \emph{ct}.\ctag{} is false or \emph{ct}.\cbase{} $+$ \emph{ct}.\coffset{}
$= -1$, \emph{cs} is copied into \emph{cd}.
Otherwise, capability register \emph{cs} is sealed with an \cotype{} of
\emph{ct}.\cbase{} $+$ \emph{ct}.\coffset{}
and the result is placed in \emph{cd} as follows:

\begin{itemize}
\item
\emph{cd} is sealed with \emph{cd}.\cotype{} set to \emph{ct}.\cbase{} + \emph{ct}.\coffset{};
\item
and the other fields of \emph{cd} are copied from \emph{cs}.
\end{itemize}

\emph{ct} must grant \emph{Permit\_Seal} permission, and the new \cotype{}
of \emph{cd} must be between \emph{ct}.\cbase{} and \emph{ct}.\cbase{} $+$
\emph{ct}.\clength{} $-$ 1.

\subsubsection*{Semantics}
\sailMIPScode{CCSeal}

\subsubsection*{Exceptions}

A coprocessor 2 exception is raised if:

\begin{itemize}
\item
\emph{cs}.\ctag{} is not set.
\item
\emph{cs} is sealed.
\item
\emph{ct}.\ctag{} is set and \emph{ct} is sealed.
\item
\emph{ct}.\cperms.\emph{Permit\_Seal} is not set.
% \item
% \emph{ct}.\cperms.\emph{Permit\_Execute} is not set.
\item
\emph{ct}.\ctag{} and \emph{ct}.\coffset{} $\ge$ \emph{ct}.length{}
\item
\emph{ct}.\ctag{} and \emph{ct}.\cbase{} $+$ \emph{ct}.\coffset{} $> \emph{max\_otype}$
\item
The bounds of \emph{cb} cannot be represented exactly in a sealed capability.
\end{itemize}

\subsubsection*{Notes}

\begin{itemize}
\item
If capability compression is in use, the range of possible (\cbase{},
\clength{}, \coffset{}) values might be smaller for sealed capabilities
than for unsealed capabilities. This means that \insnmipsref{CCSeal}
can fail with an exception in the case where the bounds are no longer
precisely representable.
\item
This instruction provides two means of indicating that the capability should
not be sealed: either clearing the tag bit on \emph{ct} or setting \emph{ct}'s
cursor to $-1$. A potential problem with just using a cursor of $-1$ (rather than
clearing the tag bit) to disable sealing is that, depending on \emph{ct}'s
\cbase{} and \coffset{}, setting \emph{ct}'s cursor to $-1$ might have a result
that is not representable. However, the NULL capability has \ctag{} clear
and can always have its cursor set to $-1$. (We implement casts from
\verb+int+ to \verb+int_cap_t+ by setting the cursor of NULL to the value of the
integer, and so this can hold a value of $-1$.) Directly clearing \emph{ct}'s
tag to indicate that sealing should not be performed will work, because it
is always possible to clear the tag bit. Setting \emph{ct}'s cursor to $-1$ with
\insnmipsref{CSetOffset} to indicate that sealing should not be performed will
also work, because this will either set the cursor to $-1$ or (if the result would
not be representable) both clear the tag bit and set the cursor to $-1$. The
latter method may be preferred in a code sequence that extracts the \cotype{}
of a capability with \insnmipsref{CGetType}, getting a value of $-1$ if the
capability is not sealed, setting the cursor of \emph{ct} to the result, and
then using \insnmipsref{CCSeal} to seal a new capability to the same
\cotype{} as the original.
\end{itemize}
