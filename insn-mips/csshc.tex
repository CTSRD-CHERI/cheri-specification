\clearpage
\phantomsection
\addcontentsline{toc}{subsection}{CSShC}
\insnmipslabel{csshc}
\subsection*{CSShC: Store Short Capability via Capability}

\subsubsection*{Format}

CSShC cs, rt, offset(cb) \\
CSShCR cs, rt(cb) \\
CSShCI cs, offset(cb)

\usesDDCinsteadofNULL{cb}

\subsubsection*{Description}

Capability register \emph{cs} is encoded to a short capability form and
stored at the memory location specified by
%
\emph{addr} $\stackrel{\Delta}{=}$ \emph{cb.\cbase{}} $+$ \emph{cb.\coffset{}} $+$ \emph{rt} $+$ 16 $*$
\emph{offset},
%
provided \emph{addr} is suitably aligned for short capabilities.
%
Capability register \emph{cb} must contain a capability that grants
permission to store capabilities.
%
The bits in the tag memory associated with \emph{addr} are updated to
indicate a short capability resides at this location if \emph{cs.\ctag{}} is
set and encoding was successful.  See \cref{sec:windowedshortcaps} for
possible tag representations.

\subsubsection*{Encoding}

As per \cref{sec:windowedshortcaps}, several properties must hold of the
capabilities given to \insnmipsref{CSShC} for a successful store to take
place:

\begin{itemize}
  \item \emph{cs} may be out of bounds by strictly less than 4 GiB.
  \item The bounds of \emph{cs} must be representable in short capabilities.
  \item If \emph{cs} is sealed, its \cotype{} must be representable in short
        capabilities.
\end{itemize}

If any of the above tests fail, the store will take place but will update
the tags corresponding to \emph{addr} to indicate that the memory contains
data.

Further, the following permission bits may not be stored in the short
capability format, and may read back (via \insnmipsref{CLShC}) as false:
Permit\_Seal, Permit\_Unseal, Permit\_SetCID,
Permit\_Access\_System\_Registers, and Permit\_Store\_Local\_Capability.
Software must, however, not depend on \insnmipsref{CSShC} to clear these
bits; an \insnmipsref{CAndPerm} must be used to ensure that these rights
are discarded.

% \subsubsection*{Semantics}

\subsubsection*{Exceptions}

A coprocessor 2 exception is raised if:

\begin{itemize}
\item
\cchecktag{}
\item
\emph{cb} is sealed.
\item
\emph{cb}.\cperms.\emph{Permit\_Store} is not set.
\item
\emph{cb.\cperms.Permit\_Store\_Capability} is not set.
\item
\emph{cb.\cperms{}.Permit\_Store\_Local} is not set and
\emph{cs.tag} is set and \emph{cs.\cperms{}.Global} is not set.
\item
\emph{addr} $+$ \emph{short\_capability\_size} $>$ \emph{cb}.\cbase{} $+$ \emph{cb.\clength{}}.
\item
\emph{addr} $<$ \emph{cb}.\cbase{}.
\item
\emph{cs.\cbase{}}, \emph{cs.\cbase{}} $+$ \emph{cs.\clength{}}, and
the store's target memory location computed above differ in their top 32 bits.
\end{itemize}

\noindent
A TLB Store exception is raised if:

\begin{itemize}
\item
\emph{cs.\ctag{}} is set and the \emph{S} bit in the TLB entry for the page
containing \emph{addr} is not set.
\end{itemize}

\noindent
An address error during store (AdES) exception is raised if:

\begin{itemize}
\item
The virtual address \emph{addr} is not \emph{short\_capability\_size} aligned.
\end{itemize}

\subsubsection*{Notes}

\begin{itemize}
\item
If the address alignment check fails and one of the security checks fails,
a coprocessor 2 exception (and not an address error exception) is raised.
The priority of the exceptions is security-critical, because otherwise a
malicious program could use the type of the exception that is raised to
test the bottom bits of a register that it is not permitted to access.
\item
\emph{offset} is interpreted as a signed integer.
\item
The \insnnoref{CSShCI} mnemonic is equivalent to \insnmipsref{CSShC} with
\emph{cb} being the zero register (\$zero). The \insnnoref{CSShCR} mnemonic
is equivalent to \insnmipsref{CSShC} with \emph{offset} set to zero.
\item
Although the \emph{short\_capability\_size} can vary, the offset is always in
multiples of 8 bytes (64 bits).
\end{itemize}
