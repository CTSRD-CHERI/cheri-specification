\clearpage
\phantomsection
\addcontentsline{toc}{subsection}{CLCBI}
\insnmipslabel{clcbi}
\subsection*{CLCBI: Load Capability via Capability (Big Immediate)}

\subsubsection*{Format}

CLCBI cd, offset(cb)

\begin{center}
\begin{bytefield}{32}
\bitheader[endianness=big]{0,5,6,10,11,15,16,20,21,25,26,31}\\
\bitbox{6}{0x1d}
\bitbox{5}{cd}
\bitbox{5}{cb}
\bitbox{16}{offset}
\end{bytefield}
\end{center}

\usesDDCinsteadofNULL{cb}

\subsubsection*{Description}
Capability register \emph{cd} is loaded from the memory location specified by
\emph{cb.\cbase{}} $+$ \emph{cb.\coffset{}} $+$ \emph{offset}.
Capability register
\emph{cb} must
contain a capability that grants permission to load capabilities.  The virtual
address \emph{cb.\cbase{}} $+$ \emph{cb.\coffset{}} $+$
\emph{offset} must be \emph{capability\_size} aligned.

The bit in the tag memory corresponding to \emph{cb.\cbase{}} $+$
\emph{cb.\coffset{}} $+$ \emph{offset} is
loaded into the tag bit associated with \emph{cd}.
%  The CBTU instruction can be used to check the value of this tag.

\subsubsection*{Semantics}
\sailMIPScode{CLCBI}

\subsubsection*{Exceptions}


A coprocessor 2 exception is raised if:

\begin{itemize}
\item
\cchecktag{}
\item
\emph{cb} is sealed.
\item
\emph{cb}.\cperms.\emph{Permit\_Load} is not set.
\item
\emph{addr} + \emph{capability\_size} $>$ \emph{cb}.\cbase{} $+$ \emph{cb}.\clength{}.
\item
\emph{addr} $<$ \emph{cb}.\cbase{}.
\end{itemize}


An address error during load (AdEL) exception is raised if:

\begin{itemize}
\item
The virtual address \emph{addr} is not \emph{capability\_size} aligned.
\end{itemize}

\subsubsection*{Notes}

\begin{itemize}
\item
This instruction reuses the opcode from the Jump and Link Exchange
(\insnnoref{JALX}) instruction in the MIPS Specification.
Future versions of the architecture may use a different encoding
to avoid reusing an opcode with a delay slot for an instruction without a delay slot.
\item
\emph{offset} is interpreted as a signed integer.
\item
Although the \emph{capability\_size} can vary, the offset is always in
multiples of 16 bytes (128 bits).
\item
%The instruction allows accessing individual members in large arrays of capabilities
%without the need for additional instructions to generate the offset.
The larger immediate of \insnmipsref{CLCBI}) enables more efficient code generation
in pure-capability programs for accesses of global variables.
In many programs and libraries, the 11-bit immediate offset of \insnmipsref{CLC}
is not sufficient reach all entries in the table of global capabilities and
therefore the compiler must to generate a three-instruction sequence instead.
By using of \insnmipsref{CLCBI}) for accessing globals, the code size of most
pure-capability binaries can be reduced by over 10\%.

\item
Architectures with pc-relative loads or instructions to add a large immediate to \PCC{}
(such as \insnnoref{AUIPC}) can use those instead of adding a capability
load with a larger immediate offset.

\end{itemize}
