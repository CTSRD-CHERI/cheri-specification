\clearpage
\phantomsection
\addcontentsline{toc}{subsection}{CSetCID}
\insnmipslabel{csetcid}
\subsection*{CSetCID: Set the Architectural Compartment ID}

\subsubsection*{Format}

CSetCID cb

%
% XXXRW: Format and opcode still TODO.
\begin{center}
% XXXAR: 0x5 is currently the second available one-operand instruction
\cherioneop[header]{0x5}{cb}
\\
\arnote{This encoding is not final -- do not implement}
\end{center}

\subsubsection*{Description}

Set the architectural Compartment ID (CID) to \emph{cb}.\cbase{} +
\emph{cb}.\coffset{} if \emph{cb} has the Permit\_Set\_CID permission and
\emph{cb}.\coffset{} is in range.
The CID can then be used by the microarchitecture to tag microarchitectural
state.
CIDs can be utilized in a similar style as ASID matching in TLBs to
determine in what context microarchitectural state can be used.
Typical use will be to prevent sharing where it could otherwise be used as a
high-bandwidth microarchitectural side channel between compartments with
confidentiality requirements -- for example, to limit the impact of
Spectre-style attacks~\cite{Kocher2018spectre}.

\subsubsection*{Semantics}

\sailMIPScode{CSetCID}

\subsubsection*{Exceptions}

A coprocessor 2 exception is raised if:

\begin{itemize}
\item
\emph{cb}.\cperms.\emph{Permit\_Set\_CID} is not set.
\item
\emph{addr} + 1 $>$ \emph{cb}.\cbase{} $+$ \emph{cb}.\clength{}.
\item
\emph{addr} $<$ \emph{cb}.\cbase{}.
\end{itemize}

\subsubsection*{Notes}

\begin{itemize}
\item
  The CID can be queried using the \insnmipsref{CGetCID} instruction.
\item
  Although \insnmipsref{CSetCID} has no architectural side effects other than
  setting an integer register with a compartment ID, the intent is that the
  microarchitecture can be made aware of boundaries across which
  microarchitectural side channels are less acceptable.
  A key design goal for \insnmipsref{CSetCID} is to provide flexible
  mechanism above which a range of software policies might be implemented.
\item
  For example, the software supervisor might arrange that all compartments
  have unique CIDs such that branch-predictor state cannot be shared.
  Other policies might use the same CID for compartments between which strong
  confidentiality requirements are not present -- e.g., where only integrity
  or availability protection is required.
\item
  We have chosen not to protect the architectural CID using
  Access\_System\_Registers in order to support virtualizability of the domain
  switcher -- and, in particular, to not require Access\_System\_Registers to
  implement a domain switcher.
  A new permission is used, together with bounds checks, such that ranges of
  CIDs can be delegated when multiple domain switchers are in use.
  For example, a set of CIDs might be reserved for domain-switch
  implementations themselves, and then subranges delegated to individual
  language runtimes or processes within the same address space.
  Note that such a model could obligate two CID operations per domain switch
  involving mutual distrust: one into the domain switcher, and a second out,
  in order to not just protect the two endpoint domains from one another but
  also the switcher.
\item
  How to ensure that \insnmipsref{CSetCID} is not speculated past (e.g., in
  the case of microarchitectural side-channel attacks such as Spectre) is a
  critical question.
  We recommend that \insnmipsref{CSetCID} be considered serialising, and that
  the CID be set immediately on switcher entry, as well as again on switcher
  exit.
\item
  An alternative design choice would accept an integer general-purpose
  register operand, \emph{rt}, as a second argument specifying the CID to
  switch to.
  This might be more consistent with the behavior of \insnmipsref{CGetCID},
  but also consume more opcode space.
\end{itemize}
