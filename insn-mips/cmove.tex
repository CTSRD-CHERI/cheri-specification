\clearpage
\phantomsection
\addcontentsline{toc}{subsection}{CGetAddr}
\insnmipslabel{cmove}
\subsection*{CMove: Move Capability to another Register}

\subsubsection*{Format}

CMove cd, cb

\begin{center}
\cheritwoop[header]{0xa}{cd}{cb}
\end{center}

\subsubsection*{Description}

\insnnoref{CMove} copies \emph{cb} into \emph{cd}.

\subsubsection*{Semantics}

\sailMIPScode{CMove}

\subsubsection*{Notes}

\begin{itemize}
\item This instruction currently has a dedicated encoding but it could also be implemented as an alias for \insnmipsref{CMOVZ} \emph{\$zero}, \emph{cd}, \emph{cb}. \arnote{This is not possible on RISC-V since there is no conditional move. Should we add a note about this?}
\item Originally, \insnmipsref{CMove} was an assembler pseudo for \insnmipsref{CIncOffset} \emph{cd}, \emph{cb}, \emph{\$zero}.
However, this requires that \insnmipsref{CIncOffset} with a sealed capability succeeds if the increment is zero.
A future version of the ISA might no longer support this and require the use of \insnmipsref{CMove} for sealed capabilities.
This would allow for a simpler implementation of \insnmipsref{CIncOffset} where the behavior does not depend on one of the input values. \arnote{Some more rationale about intentionality?}
\end{itemize}
