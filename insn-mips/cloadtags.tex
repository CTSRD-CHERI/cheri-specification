\clearpage
\phantomsection
\addcontentsline{toc}{subsection}{CLoadTags}
\insnmipslabel{cloadtags}
\subsection*{CLoadTags: Read Multiple Tags to Integer Register}

\subsubsection*{Format}

CLoadTags rd, cb \\

\begin{center}
\begin{bytefield}{32}
\bitheader[endianness=big]{0,5,6,10,11,15,16,20,21,25,26,31}\\
\bitbox{6}{0x12}
\bitbox{5}{0x00}
\bitbox{5}{rd}
\bitbox{5}{cb}
\bitbox{5}{0x1E}
\bitbox{5}{0x3F}
\end{bytefield}
\end{center}

\usesDDCinsteadofNULL{cb}

\subsubsection*{Description}

The \emph{tags} from memory referenced by \emph{cb} are loaded to \emph{rd},
with bit significance increasing with memory address.  The result of this
instruction must be coherent with other processors, \emph{as if} the
corresponding data memory words had been loaded.  The number of tags loaded is
an implementation-defined constant but is constrained to be a power of two, at
least 1, and no more than the width of \emph{rd}.

Capability register \emph{cb} must contain a capability that grants permission
to load capabilities.  The virtual address \emph{cb.\cbase{}} $+$
\emph{cb.\coffset{}} must be suitably aligned; the precise requirements are,
again, implementation defined, but must equal the width of memory corresponding
to the tags loaded.  If any tag to be loaded corresponds to memory out of
bounds of \emph{cb}, a length violation is indicated.

\subsubsection*{Semantics}

\sailMIPScode{CLoadTags}

\subsubsection*{Exceptions}

A coprocessor 2 exception is raised if:

\begin{itemize}
\item
\cchecktag{}
\item
\emph{cb.\ctag{}} is clear.
\item
\emph{cb} is sealed.
\item
\emph{cb}.\cperms.\emph{Permit\_Load} is not set.
\item
\emph{cb}.\cbase{} $+$ \emph{cb}.\coffset{} + $n$ $*$ \emph{capability\_size} $>$ \emph{cb}.\cbase{} $+$ \emph{cb}.\clength{},
		where $n$ is the number of capabilities to be fetched (\isail{caps_per_cacheline} in the Sail code).
\item
\emph{cb}.\cbase{} $+$ \emph{cb}.\coffset{} $<$ \emph{cb}.\cbase{}.
\end{itemize}

An address error during load (AdEL) exception is raised if:

\begin{itemize}
\item
The virtual address \emph{addr} is not $n$ $*$ \emph{capability\_size} aligned.
\end{itemize}

\subsubsection*{Notes}

\begin{itemize}
%
\item In practice, the number of tags loaded is likely less arbitrary than it
may appear. Usually, the implementation's cache fabric determines the minimum
granularity of coherency and already tracks tag bits along with each cache
line, and so this instruction fetches the tag bits from the cache line
indicated by \emph{cb}.\cbase{} $+$ \emph{cb}.\coffset{}. Of course, additional
complexity in the cache and tag cache fabric could allow this instruction to
retrieve more tags than in a cache line.
Also, the number of tags this instruction loads should be a power-of-two to
avoid alignment issues and to preserve page divisibility in systems with MMUs.
In order to reduce DRAM traffic, it is desirable that this \emph{tag fetch} not
require loading the corresponding data and not necessarily evict other lines
from caches. (However, a non-zero result is probably a reasonable hint that a
capability load is likely to follow.)


\item Software can easily discover the width used by any implementation by
constructing an aligned array of capabilities in memory and observing the
result of \insnmipsref{CLoadTags}.
%; for a CHERI instantiation with 256-bit
%capability representations and 64-bit integer registers, the maximal
%alignment requirement for these probes is 512 bytes.
Such probes need be done only rarely, at system or allocator startup.

\item For multi-core or multi-processor systems with cache fabrics wherein cache
lines are of different sizes, \insnmipsref{CLoadTags} must behave as if all
cores view the memory subsystem through the \emph{smallest} cache line in
the system.

\end{itemize}
