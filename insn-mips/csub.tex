\clearpage
\phantomsection
\addcontentsline{toc}{subsection}{CSub}
\insnmipslabel{csub}
\subsection*{CSub: Subtract Capabilities}

\subsubsection*{Format}

CSub rd, cb, ct

\begin{center}
\cherithreeop[header]{0xa}{rt}{cb}{ct}
\end{center}

\subsubsection*{Description}

Register \emph{rd} is set equal to (\emph{cb}.\ccursor{} $-$ \emph{ct}.\ccursor{}) $\bmod~2^{64}$.

\subsubsection*{Semantics}

\sailMIPScode{CSub}

\subsubsection*{Notes}

\begin{itemize}
\item
\insnmipsref{CSub} can be used to implement C-language pointer subtraction,
or subtraction of \ccode{intcap\_t}.
\item
Like \insnmipsref{CIncOffset}, \insnmipsref{CSub} can be used on either
valid capabilities (\ctag{} set) or on integer values stored in capability
registers (\ctag{} not set).
\item
If a copying garbage collector is in use, pointer subtraction must be
implemented with an atomic operation (such as \insnmipsref{CSub}).
Implementing pointer subtraction with a non-atomic sequence of operations such
as \insnmipsref{CGetOffset} has the risk that the garbage collector will
relocate an object part way through, giving incorrect results for the
pointer difference. If \emph{cb} and \emph{ct} are both pointers into the
same object, then a copying garbage collector will either relocate both of
them or neither of them, leaving the difference the same.
If \emph{cb} and \emph{ct} are pointers into
different objects, the result of the subtraction is not defined by the ANSI
C standard, so it doesn't matter if this difference changes as the garbage
collector moves objects.
\end{itemize}
