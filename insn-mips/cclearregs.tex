\clearpage
\phantomsection
\addcontentsline{toc}{subsection}{CClearRegs}
\insnmipslabel{cclearregs}
\insnmipslabel{cclearhi}
\insnmipslabel{cclearlo}
\insnmipslabel{clearhi}
\insnmipslabel{clearlo}
\insnmipslabel{fpclearhi}
\insnmipslabel{fpclearlo}
\subsection*{CClearRegs: Clear Multiple Registers}

\subsubsection*{Format}
ClearLo  mask \\
ClearHi  mask \\
CClearLo mask \\
CClearHi mask \\
FPClearLo mask \\
FPClearHi mask

\begin{center}
\begin{bytefield}{32}
\bitheader[endianness=big]{0,15,16,20,21,25,26,31}\\
\bitbox{6}{0x12}
\bitbox{5}{0x0f}
\bitbox{5}{regset}
\bitbox{16}{mask}
\end{bytefield}
\end{center}

\subsubsection*{Description}

The registers in the target register set, \emph{regset}, corresponding
to the set bits in the immediate \emph{mask} field are cleared.  That
is, if bit 0 of \emph{mask} is set, then the lowest numbered register
in \emph{regset} is cleared, and so on. The following values are
defined for the \emph{regset} field:

\begin{figure}[h]
\begin{center}
\begin{tabular}{l|l|l}
Mnemonic & \emph{regset} & Affected registers   \\
\midrule
ClearLo  & 0 & \reg{0}--\reg{15}    \\
ClearHi  & 1 & \reg{16}--\reg{31}   \\
CClearLo & 2 & \DDC, \creg{1}--\creg{15}  \\
CClearHi & 3 & \creg{16}--\creg{31} \\
FPClearLo & 4 & {\bf F0--F15} \\
FPClearHi & 5 & {\bf F16--F31} \\
\end{tabular}
\end{center}
\end{figure}

For integer registers, clearing means setting to zero.
For capability registers, clearing consists of setting all
capability fields such that the in-memory representation will be all zeroes,
with a cleared tag bit, granting no rights.

\paragraph{Note:} For \insnmipsref{CClearLo} bit 0 in \emph{mask} refers to \DDC{} and not \creg{0} since \creg{0} is the NULL register.

\subsubsection*{Semantics}

\sailMIPScode{ClearRegs}

\subsubsection*{Exceptions}

\begin{itemize}
\item
A Reserved Instruction exception is raised for unknown or
unimplemented values of \emph{regset}.
\item
\insnmipsref{CClearLo} and \insnmipsref{CClearHi} raise a coprocessor
unusable exception if the capability coprocessor is disabled.
\item
\insnmipsref{FPClearLo} and \insnmipsref{FPClearHi} raise a coprocessor
unusable exception if the floating point unit is disabled.
\end{itemize}

\subsubsection*{Notes}

\begin{itemize}
\item
These instructions are designed to accelerate the register clearing
that is required for secure domain transitions.  It is expected that
they can be implemented efficiently in hardware using a single `valid'
bit per register that is cleared by the ClearRegs instruction and set
on any subsequent write to the register.
\item
The mnemonic for the integer-register instruction does not make it
very clear what the instruction does.
It would be preferable to have a more descriptive mnemonic.
\end{itemize}
