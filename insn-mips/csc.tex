\clearpage
\phantomsection
\addcontentsline{toc}{subsection}{CSC}
\insnmipslabel{csc}
\subsection*{CSC: Store Capability via Capability}

\subsubsection*{Format}

CSC cs, rt, offset(cb) \\
CSCR cs, rt(cb) \\
CSCI cs, offset(cb)

\begin{center}
\begin{bytefield}{32}
\bitheader[endianness=big]{0,5,6,10,11,15,16,20,21,25,26,31}\\
\bitbox{6}{0x3e}
\bitbox{5}{cs}
\bitbox{5}{cb}
\bitbox{5}{rt}
\bitbox{11}{offset}
\end{bytefield}
\end{center}

\usesDDCinsteadofNULL{cb}

\subsubsection*{Description}

Capability register \emph{cs} is stored at the memory location specified by
\emph{cb.\cbase{}} $+$ \emph{cb.\coffset{}} $+$ \emph{rt} $+$ 16 $*$ \emph{offset},
and the bit in the tag memory associated with \emph{cb.\cbase{}} $+$
\emph{cb.\coffset{}} $+$ \emph{rt} $+$ 16 $*$ \emph{offset} is set to the value of
\emph{cs.tag}.
Capability register \emph{cb} must
contain a capability that grants permission to store capabilities.  The virtual
address \emph{cb.\cbase{}} $+$ \emph{cb.\coffset{}} $+$ \emph{rt} $+$
16 $*$ \emph{offset} must be \emph{capability\_size} aligned.

% When the 256-bit representation of capabilities is in use, the capability
% is stored in memory in the format described in Figure
% \ref{fig:memory-representation-of-a-capability}.
% \cbase{}, \clength{} and \cotype{} are
% stored in memory with the same endian-ness that the CPU uses for double-word
% stores, i.e., big-endian. The bits of \cperms{} are stored with bit zero
% being the least significant bit, so that the least significant bit of the
% eighth byte stored is the \csealed{} bit,
% the next significant bit is the \emph{Global} bit, the next is
% \emph{Permit\_Execute} and so on.

The various capability encoding schemes define bit representations in
memory.  While a given instantiation of CHERI will use a particular scheme,
software should, in general, not attempt to parse capability bit patterns
from memory.  Instructions for capability interrogation (e.g.,
\insnmipsref{CGetAddr}, \insnmipsref{CGetType}) do not require that their
source registers be holding \emph{tagged} capabilities; software wishing to
decode memory bit patterns should rather use \insnmipsref{CLC} and
interrogate the result.

\subsubsection*{Semantics}
\sailMIPScode{CSC}

\subsubsection*{Exceptions}

A coprocessor 2 exception is raised if:

\begin{itemize}
\item
\cchecktag{}
\item
\emph{cb} is sealed.
\item
\emph{cb}.\cperms.\emph{Permit\_Store} is not set.
\item
\emph{cb.\cperms.Permit\_Store\_Capability} is not set.
\item
\emph{cb.\cperms{}.Permit\_Store\_Local} is not set and
\emph{cs.tag} is set and \emph{cs.\cperms{}.Global} is not set.
\item
\emph{addr} $+$ \emph{capability\_size} $>$ \emph{cb}.\cbase{}
$+$ \emph{cb.\clength{}}.
\item
\emph{addr} $<$ \emph{cb}.\cbase{}.
\end{itemize}

\noindent
A TLB Store exception is raised if:

\begin{itemize}
\item
\emph{cs.\ctag{}} is set and the \emph{S} bit in the TLB entry for the page
containing \emph{addr} is not set.
\end{itemize}

\noindent
An address error during store (AdES) exception is raised if:

\begin{itemize}
\item
The virtual address \emph{addr} is not \emph{capability\_size} aligned.
\end{itemize}

\subsubsection*{Notes}

\begin{itemize}
\item
If the address alignment check fails and one of the security checks fails,
a coprocessor 2 exception (and not an address error exception) is raised.
The priority of the exceptions is security-critical, because otherwise a
malicious program could use the type of the exception that is raised to
test the bottom bits of a register that it is not permitted to access.
\item
It is permitted to store a local capability with the tag bit unset even if the permit store local bit is not set in \emph{cb}.
This is because if the tag bit is not set then the permissions have no meaning.
\item
\emph{offset} is interpreted as a signed integer.
\item
This instruction reuses the opcode from the Store Doubleword from Coprocessor 2
(\insnnoref{SDC2}) instruction in the MIPS Specification.
\item
The \insnnoref{CSCI} mnemonic is equivalent to \insnmipsref{CSC} with
\emph{cb} being the zero register (\$zero). The \insnnoref{CSCR} mnemonic
is equivalent to \insnmipsref{CSC} with \emph{offset} set to zero.
\item
BERI1 has a compile-time option to allow unaligned loads and stores.
\insnmipsref{CSC} to an unaligned address will raise an exception even if
BERI1 has been built with this option, because it would be a security
vulnerability if an attacker could construct a corrupted capability with
\ctag{} set by writing it to an unaligned address.
\item
Although the \emph{capability\_size} can vary, the offset is always in
multiples of 16 bytes (128 bits).
\end{itemize}
