\clearpage
\phantomsection
\addcontentsline{toc}{subsection}{CLL[BHWD][U]}
\insnmipslabel{cllbhwd}
\insnmipslabel{cllb}
\insnmipslabel{cllh}
\insnmipslabel{cllw}
\insnmipslabel{clld}
\subsection*{CLL[BHWD][U]: Load Linked Integer via Capability}

\subsubsection*{Format}

CLLB rd, cb\\
CLLH rd, cb\\
CLLW rd, cb\\
CLLD rd, cb\\
CLLBU rd, cb\\
CLLHU rd, cb\\
CLLWU rd, cb

\begin{center}
\begin{bytefield}{32}
\bitheader[endianness=big]{0,1,2,3,4,10,11,15,16,20,21,25,26,31}\\
\bitbox{6}{0x12}
\bitbox{5}{0x10}
\bitbox{5}{rd}
\bitbox{5}{cb}
\bitbox{7}{\color{lightgray}\rule{\width}{\height}}
\bitbox{1}{1}
\bitbox{1}{s}
\bitbox{2}{t}
\end{bytefield}
\end{center}

\usesDDCinsteadofNULL{cb}

\subsubsection*{Description}

\insnnoref{CLL[BHWD][U]} and \insnnoref{CSC[BHWD]} are used to implement safe access
to data shared between different threads. The typical usage is that
\insnnoref{CLL[BHWD][U]} is followed (an arbitrary number of
instructions later) by \insnnoref{CSC[BHWD]} to the same address; the
\insnnoref{CSC[BHWD]} will only succeed if the memory location that was loaded
by the \insnnoref{CLL[BHWD][U]} has not been modified.

The exact conditions under which \insnnoref{CSC[BHWD]} fails are implementation
dependent, particularly in multicore or multiprocessor implementations). The
following code is intended to represent the security semantics of the
instruction correctly, but should not be taken as a definition of the CPU's
memory coherence model.

\subsubsection*{Semantics}
\sailMIPScode{CLoadLinked}

\subsubsection*{Exceptions}

A coprocessor 2 exception is raised if:

\begin{itemize}
\item
\emph{cb}.\ctag{} is not set.
\item
\emph{cb} is sealed.
\item
\emph{cb}.\cperms{}.\emph{Permit\_Load} is not set.
\item
\emph{addr} $+$ \emph{size} $>$ \emph{cb}.\cbase{} $+$ \emph{cb}.\clength{}
\item
\emph{addr} $<$ \emph{cb}.\cbase{}
\end{itemize}

An AdEL exception is raised if \emph{addr} is not correctly aligned.
