\clearpage
\phantomsection
\addcontentsline{toc}{subsection}{CFromPtr}
\insnmipslabel{cfromptr}

\subsection*{CFromPtr: Create Capability from Pointer}

\subsubsection*{Format}

CFromPtr cd, cb, rt

\begin{center}
\cherithreeop[header]{0x13}{cd}{cb}{rt}
\end{center}

\usesDDCinsteadofNULL{cb}

\subsubsection*{Description}

\emph{rt} is a pointer using the C-language convention that a zero value
represents the NULL pointer. If \emph{rt} is zero, then \emph{cd} will be
the NULL capability (tag bit not set, all other bits also not set). If
\emph{rt} is non-zero, then \emph{cd} will be set to \emph{cb} with the
\coffset{} field set to \emph{rt}.

\subsubsection*{Semantics}

\sailMIPScode{CFromPtr}

% {\bf FIXME: Is \cotype{} architecturally defined/meaningful if we don't
% have the set type permission? Also, there is a nasty interaction between the
% semantic defintion of fields and their in-memory representation here: the
% idea is that the in-memory representation is all zero bits, which does not
% sit well with the claim that the in-memory representation might be changed
% in future.}

\subsubsection*{Exceptions}

A coprocessor 2 exception is raised if:

\begin{itemize}
\item
\emph{cb}.\ctag{} is not set and \emph{rt} $\neq$ 0.
\item
\emph{cb} is sealed and \emph{rt} $\neq$ 0.
\end{itemize}

\subsubsection*{Notes}

\begin{itemize}
\item
\insnmipsref{CSetOffset} doesn't raise an exception if the tag bit is unset,
so that it can be used to implement the \emph{intcap\_t} type.
\insnmipsref{CFromPtr} raises an exception if the tag bit is unset: although
it would not be a security violation to to allow it, it is an indication
that the program is in error.
\item
The encodings of the NULL capability are chosen so that zeroing memory will
set a capability variable to NULL.
%This holds true for compressed capabilities
%as well as the 256-bit version.
\end{itemize}
