\clearpage
\phantomsection
\addcontentsline{toc}{subsection}{CMOVZ / CMOVN}
\insnmipslabel{cmovn}
\subsection*{CMOVZ / CMOVN: Conditionally Move Capability on Zero / Non-Zero}

\subsubsection*{Format}

CMOVN cd, cb, rt

\begin{center}
\cherithreeop[header]{0x1c}{cd}{cb}{rt}
\end{center}

\phantomsection
\insnmipslabel{cmovz}
CMOVZ cd, cb, rt

\begin{center}
\cherithreeop[header]{0x1b}{cd}{cb}{rt}
\end{center}

\subsubsection*{Description}

CMOVZ copies \emph{cb} into \emph{cd} if \emph{rt} $=$ 0. \\
CMOVN copies \emph{cb} into \emph{cd} if \emph{rt} $\neq$ 0. \\

\subsubsection*{Semantics}

\sailMIPScode{CMOVX}

\subsubsection*{Notes}

\begin{itemize}
\item
In the Sail code {\tt ismovn} is true for \insnmipsref{CMOVN}, thus inverting the condition (via exclusive-or) in that case.
\item
Some implementations of cryptographic algorithms need a constant-time move
operation to avoid revealing secret key material through a timing channel.
(An attacker must not be able to determine whether a condition variable
inside the cryptographic implementation is true or false from observations
of how long the operation took to complete). In the current prototype
implementation of CHERI, no guarantees are made about \insnmipsref{CMOVN}
being constant time.

If CHERI instructions are to be used in high-security cryptographic
processors, consideration should be given to making this operation
constant time.
\end{itemize}
