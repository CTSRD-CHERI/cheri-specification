\clearpage
\phantomsection
\addcontentsline{toc}{subsection}{CBuildCap}
\insnmipslabel{cbuildcap}
\subsection*{CBuildCap: Import a Capability}

\subsubsection*{Format}

CBuildCap cd, cb, ct

\begin{center}
\begin{bytefield}{32}
\bitheader[endianness=big]{0,5,6,10,11,15,16,20,21,25,26,31}\\
\bitbox{6}{0x12}
\bitbox{5}{0x0}
\bitbox{5}{cd}
\bitbox{5}{cb}
\bitbox{5}{ct}
\bitbox{6}{0x1d}
\end{bytefield}
\end{center}

\usesDDCinsteadofNULL{cb}

\subsubsection*{Description}

\insnmipsref{CBuildCap} attempts to interpret the contents of \emph{ct}
as if it were a valid capability (even though \emph{ct}.\ctag{} is not
required to be set and so \emph{ct} might contain any bit pattern) and
extracts its \cbase{}, \clength{}, \coffset{}, \cperms{} and \cuperms{}
fields.
If the bounds of \emph{ct} cannot be extracted because the bit
pattern in \emph{ct} does not correspond to a permitted value of the capability
type (e.g. \clength{} is negative), then an exception is raised.

If the extracted bounds of \emph{ct} are within the bounds of \emph{cb}, and
the permissions of \emph{ct} are within the permissions of \emph{cb}, then
\emph{cd} is set equal to \emph{cb} with the \cbase{}, \clength{},
\coffset{}, \cperms{} and \cuperms{} of \emph{ct}.

If \emph{ct} is sealed, this instruction does not copy its \cotype{} into \emph{cd}. With
compressed capabilities, a different representation may be used for the bounds
of sealed and unsealed capabilities. If \emph{ct} is sealed,
\insnmipsref{CBuildCap} will change the representation of the bounds so
that their values are preserved.

Because \emph{ct}.\ctag{} is not required to be set, there is no guarantee
that the bounds of \emph{ct} will be in canonical form.
\insnmipsref{CBuildCap} may convert the bounds into canonical form rather than
simply copying their bit representation.

\insnmipsref{CBuildCap} does not copy the fields of \emph{ct} that are
reserved for future use.

\insnmipsref{CBuildCap} can be used to set the tag bit on a capability (e.g.,
one whose non-tag contents has previously been swapped to disk and than
reloaded into memory, or during dynamic linking as untagged capability values
are relocated and tagged after being loaded from a file).
This provides both improved efficiency relative to manual rederivation of the
tagged capability via a series of instructions, and also provides improved
architectural abstraction by avoiding embedding the rederivation sequence in
code.

\subsubsection*{Semantics}
\sailMIPScode{CBuildCap}

\subsubsection*{Exceptions}

A coprocessor 2 exception is raised if:

\begin{itemize}
\item
\emph{cd}, \emph{cb} or \emph{ct} is a reserved register and \PCC.\cperms{} does
not grant \emph{Permit\_Access\_System\_Registers}.
\item
\emph{cb}.\ctag{} is not set.
\item
\emph{cb} is sealed.
\item
The bounds of \emph{ct} are outside the bounds of \emph{cb}.
\item
The values of \cbase{} and \clength{} found in \emph{ct} are not within the
range permitted for a capability with its \ctag{} bit set.
\item
\emph{ct}.\cperms{} grants a permission that is not granted by
\emph{cb}.\cperms{}.
\item
\emph{ct}.\cuperms{} grants a permission that is not granted by
\emph{cb}.\cuperms{}.
\end{itemize}

\subsubsection*{Notes}

\begin{itemize}
\item
This instruction acts both as an optimization, and to provide architectural
abstraction in the face of future change to the capability model.
A similar effect, albeit with reduced abstraction, could be achieved by
using \insnmipsref{CGetBase}, \insnmipsref{CGetLen} and
\insnmipsref{CGetPerm} to query \emph{ct}, and then using
\insnmipsref{CSetBounds} and \insnmipsref{CAndPerm} to set the bounds
and \cperms{} of \emph{cd}.
\item
Despite the description of its intended use above, \insnmipsref{CBuildCap}
does not actually require that \emph{ct} have an unset tag.
\item
\emph{ct} might be a sealed capability that has had its \ctag{} bit cleared.
In this case (assuming an exception is not raised for another reason),
\emph{cd} will be unsealed and the bit representation of
the \cbase{} and \clength{} fields might be changed to take account of
the differing compressed representations for sealed and unsealed capabilities.
\item
This instruction can not be used to break security properties of the capability
mechanism (such as monotonicity) because \emph{cb} must be a valid capability
and the instruction cannot be used to create a capability that grants rights
that were not granted by \emph{cb}.
\item
As the tag bit on \emph{ct} does not need to be set, there is no guarantee
that the bit pattern in \emph{ct} was created by clearing the tag bit on a
valid capability. It might be an arbitrary bit pattern that was created by
other means. As a result, there is no guarantee that the bit pattern in
\emph{ct} corresponds to the encoding of a valid value of the capability type,
especially when capability compression is in use. Fields might have values outside
of their defined range, and invariants such as \cbase{} $\ge 0$, \cbase{} $+$
\clength{} $\le 2^{64}$ or \clength{} $\ge 0$ might not be true. In addition,
fields might not be in a canonical (normalized) form.
\insnmipsref{CBuildCap} checks that the \cbase{} and \clength{} fields are
within the permitted range for the type and satisfy the above invariants,
raising a length exception if they are not. If the fields are not in normalized
form, \insnmipsref{CBuildCap} may renormalize them rather than simply
copying the bit pattern from \emph{ct} into \emph{cd}.
\item
The type constraint \emph{cd}.\ctag{} $\implies$ \emph{cd}.\cbase{} $\ge 0$ is
guaranteed to be satisfied
because \emph{cb}.\cbase{} $\ge 0$ and an exception would be raised if
\emph{ct}.\cbase{} $\le$ \emph{cb}.\cbase{}.
\item
The type constraint \emph{cd}.\ctag{} $\implies$ \emph{cd}.\cbase{} $+$
\emph{cd}.\clength{} $\le 2^{64}$ is guaranteed to be satisfied because this
constraint is true for \emph{cb}, and an exception would be raised if
\emph{ct}.\cbase $+$ \emph{ct}.\clength{} $>$ \emph{cb}.\cbase{} $+$
\emph{cb}.\clength{}.
\item
Is the value of \emph{cd} guaranteed to be representable?
If \emph{ct} was created by clearing the tag bit on a capability, then its
bounds can be represented exactly and there will be no loss of precision.
If \emph{ct} is sealed, then there is a potential issue that the values
of the bounds that are representable in a sealed capability are not the same as
the range of bounds that are representable in an unsealed capability. We rely
on a property of the existing capability formats that if a value of the bounds
is representable in a sealed capability, then it is also representable in an
unsealed capability.
\item As \insnmipsref{CBuildCap} is not able to restore the seal on a
re-tagged capability, it is intended to be used alongside
\insnmipsref{CCSeal}, which will conditionally seal a capability based on a \cotype{} value
extracted with \insnmipsref{CCopyType}.
These instructions will normally be used in sequence to \textit{(i)} re-tag a
capability with CBuildCap, \textit{(ii)} extract a possible object type from
the untagged value with \insnmipsref{CCopyType}, and \textit{(iii)}
conditionally seal the resulting capability with \insnmipsref{CCSeal}.
\item
The typical use of \insnmipsref{CBuildCap} assumes that there is a single
capability \emph{cb} whose bounds include every capability value that is
expected to be encountered in \emph{ct} (with out of range values being an
error). The following are two examples of situations where this is not the
case, and the sequence of instructions to recreate a capability might need
to decide which capability to use as \emph{cb}:
(a) The operating system has enforced a ``write xor execute'' policy, and
the program attempting to recreate \emph{ct} has a capability with
\emph{Permit\_Write} permission and a capability with \emph{Permit\_Execute}
permission, but does not have a capability with both permissions.
(b) The capability in \emph{ct} might be a capability that authorizes sealing
with the \emph{Permit\_Seal} permission, and the program attempting to
recreate it has a capability for a range of memory addresses and a capability
for a range of \cotype{} values, but does not have a single capability that
includes both ranges.
\end{itemize}

\mrnote{TO DO: We can solve the problem of multiple authorizing capabilities
if CBuildCap does nothing if the source already has the tag set, or if the
authorizing capability does not grant sufficient rights. Then, you can just
try all of the authorizing capabilties in turn and see if any of them worked.}
