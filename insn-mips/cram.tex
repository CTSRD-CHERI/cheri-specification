\clearpage
\phantomsection
\addcontentsline{toc}{subsection}{CRepresentableAlignmentMask}
\insnmipslabel{cram}
\insnmipslabel{crepresentablealignmentmask}
\subsection*{CRepresentableAlignmentMask: Retrieve Mask to Align Capabilities to Precisely Representable Address}

\subsubsection*{Format}

CRepresentableAlignmentMask rt, rs

\begin{center}
\cheritwoop[header]{0x11}{rt}{rs}
\end{center}

\subsubsection*{Description}

\emph{rt} is set to a mask that can be used to align down addresses to a value that is sufficiently aligned to set precise bounds for the nearest representable length of \emph{rs} (which may be obtained using the instruction \insnmipsref{CRoundRepresentableLength}).
\arnote{TODO: this description is pretty terrible}

\subsubsection*{Semantics}

\sailMIPScode{CRAM}

\subsubsection*{Notes}

\begin{itemize}
\item The result of \insnmipsref{CRepresentableAlignmentMask} is intended to be used as the mask argument to \insnmipsref{CAndAddr} in order to create  a capability with an address that is sufficiently aligned to perform \insnmipsref{CSetBoundsExact} with the specified length. This \insnmipsref{CSetBoundsExact} is guaranteed to succeed if the size is rounded using \insnmipsref{CRoundRepresentableLength}.
\item The required alignment of an allocation of size \emph{rs} can be computed by negating \emph{rt} and adding one.
\item Combined with \insnmipsref{CRoundRepresentableLength} this instruction can be used in memory allocators to guarantee non-overlapping allocations.
\item This instruction can be useful to adjust the stack pointer to an address that is suitably aligned for dynamic allocations.
\item An alternative to this instruction is the use of count-leading-zeroes and count-trailing-zeros instructions followed by shifting and masking. However, this requires encoding knowledge of the underlying precision in the resulting binary and can therefore result in incompatibilities with future architectures that use a different compression algorithm.
\end{itemize}

