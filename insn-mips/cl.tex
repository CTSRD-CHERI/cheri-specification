\clearpage
\phantomsection
\addcontentsline{toc}{subsection}{CL[BHWD][U]}
\insnmipslabel{clbhwd}
\subsection*{Load Integer via Capability Register}

\subsubsection*{Format}

CLB rd, rt, offset(cb)\\
CLH rd, rt, offset(cb)\\
CLW rd, rt, offset(cb)\\
CLD rd, rt, offset(cb)\\
CLBU rd, rt, offset(cb)\\
CLHU rd, rt, offset(cb)\\
CLWU rd, rt, offset(cb)\\
CLBR rd, rt(cb)\\
CLHR rd, rt(cb)\\
CLWR rd, rt(cb)\\
CLDR rd, rt(cb)\\
CLBUR rd, rt(cb)\\
CLHUR rd, rt(cb)\\
CLWUR rd, rt(cb)\\
CLBI rd, offset(cb)\\
CLHI rd, offset(cb)\\
CLWI rd, offset(cb)\\
CLDI rd, offset(cb)\\
CLBUI rd, offset(cb)\\
CLHUI rd, offset(cb)\\
CLWUI rd, offset(cb)

\begin{center}
\begin{bytefield}{32}
\bitheader[endianness=big]{0,1,3,10,11,15,16,20,21,25,26,31}\\
\bitbox{6}{0x32}
\bitbox{5}{rd}
\bitbox{5}{cb}
\bitbox{5}{rt}
\bitbox{8}{offset}
\bitbox{1}{s}
\bitbox{2}{t}
\end{bytefield}
\end{center}

\usesDDCinsteadofNULL{cb}

\subsubsection*{Purpose}

Loads a data value via a capability register, and extends the value to fit the target register.

\subsubsection*{Description}

The lower part of integer register \emph{rd} is loaded from the memory
location specified by \emph{cb.\cbase{}} + \emph{cb.\coffset{}} + \emph{rt} +
$2^t$ $*$  \emph{offset}. Capability register \emph{cb} must
contain a valid capability that grants permission to load data.

The size of the value loaded depends on the value of the \emph{t} field:

\begin{description}
	\item[0] byte (8 bits)
	\item[1] halfword (16 bits)
	\item[2] word (32 bits)
	\item[3] doubleword (64 bits)
\end{description}

The extension behavior depends on the value of the \emph{s} field: 1 indicates sign extend, 0 indicates zero extend.  For example, \insnmipsref[clbhwd]{CLWU} is encoded by setting \emph{s} to 0 and \emph{t} to 2, \insnmipsref[clbhwd]{CLB} is encoded by setting \emph{s} to 1 and
\emph{t} to 0.

\subsubsection*{Semantics}
\sailMIPScode{CLoad}

\subsubsection*{Exceptions}

A coprocessor 2 exception is raised if:

\begin{itemize}
\item
\cchecktag{}
\item
\emph{cb} is sealed.
\item
\emph{cb}.\cperms.\emph{Permit\_Load} is not set.
\item
\emph{addr} $+$ \emph{size} $>$ \emph{cb}.\cbase{} $+$ \emph{cb}.\clength{}

NB: The check depends on the size of the data loaded.
\item
\emph{addr} $<$ \emph{cb}.\cbase{}
\end{itemize}

An AdEL exception is raised if \emph{addr} is not correctly aligned.

\subsubsection*{Notes}

\begin{itemize}
\item
This instruction reuses the opcode from the Load Word to Coprocessor 2
(\insnnoref{LWC2}) instruction in the MIPS Specification.
\item
\emph{rt} is treated as an unsigned integer.\note{rmn30}{since rt is 64-bit and addr is computed modulo $2^{64}$ it doesn't make any difference whether it is treated as signed or unsigned. It actually can make sense to think of it as a signed offset. Maybe we should remove this note? Important to note that immediate offset is signed, however.}
\item
\emph{offset} is treated as a signed integer.
% \item
% The result of the addition does not wrap around (i.e., an exception is
% raised if cb.\cbase{} $+$ cb.\coffset{} $+$ \emph{rt} $+$ \emph{offset}
% is less than zero, or greater than \emph{maxaddr}).
% \item
% The temporary variable \emph{addr} can have values that are less than zero
% or greater than $2^{64}-1$; the computation of \emph{addr} does not wrap
% around modulo $2^{64}$.
\item
BERI1 has a compile-time option to allow unaligned loads and stores. If BERI1
is built with this option, an unaligned load will only raise an exception if
it crosses a cache line boundary.
\end{itemize}
