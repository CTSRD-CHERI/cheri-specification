\clearpage
\phantomsection
\addcontentsline{toc}{subsection}{CSetOffset}
\insnmipslabel{csetoffset}
\subsection*{CSetOffset: Set Cursor to an Offset from Base}

\subsubsection*{Format}

CSetOffset cd, cs, rt

\begin{center}
\cherithreeop[header]{0xf}{cd}{cs}{rt}
\end{center}

\subsubsection*{Description}

Capability register \emph{cd} is replaced with the contents of capability
register \emph{cs} with the \coffset{} field set to the contents of integer
register \textit{rt}.

If capability compression is in use, and the requested \cbase{}, \clength{}
and \coffset{} cannot be represented exactly, then \emph{cd}.\ctag{} is
cleared, \emph{cd}.\cbase{} and \emph{cd}.\clength{} are set to zero,
\emph{cd}.\cperms{} is cleared and \emph{cd}.\coffset{} is set equal to
\emph{cs}.\cbase $+$ \emph{rt}.

\subsubsection*{Semantics}

\sailMIPScode{CSetOffset}

\subsubsection*{Exceptions}

A coprocessor 2 exception is raised if:

\begin{itemize}
\item
\emph{cs}.\ctag{} is set and \emph{cs} is sealed.
\end{itemize}

\subsubsection*{Notes}

\begin{itemize}
\item
\insnmipsref{CSetOffset} can be used on a capability register whose tag bit
is not set. This can be used to store an integer value in a capability register,
and is useful when implementing a variable that is a union of a capability
and an integer (\ccode{intcap\_t} in C).
% The in-memory representation
% that will be used if the capability register is stored to memory might
% be surprising to some users (with the 256-bit representation of capabilities,
% \cbase{} $+$ \coffset{} is stored in the
% {\bf cursor} field in memory) and may change if the memory representation of
% capabilities changes, so compilers should not rely on it.
\item
With compressed capabilities, the requested offset may not not
representable.
In this case, the result preserves the requested \cbase{} $+$ \coffset{}
(i.e., the cursor) rather than the architectural field \coffset{}. This
field is mainly useful for debugging what went wrong (the capability cannot
be dereferenced, as \ctag{} has been cleared), and for debugging we considered
it more useful to know what the requested capability would have referred to
rather than its \coffset{} relative to a \cbase{} that is no longer available.
This has the disadvantage that it exposes the value of \cbase{} to a program,
but \cbase{} is not a secret and can be accessed by other means. The
main reason for not exposing \cbase{} to programs is so that a garbage
collector can stop the program, move memory, modify the capabilities and
restart the program. A capability with \ctag{} cleared cannot be dereferenced,
and so is not of interest to a garbage collector, and so it doesn't matter
if it exposes \cbase{}.
\end{itemize}
