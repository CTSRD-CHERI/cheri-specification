\clearpage
\phantomsection
\addcontentsline{toc}{subsection}{CToPtr}
\insnmipslabel{ctoptr}
\subsection*{CToPtr: Capability to Integer Pointer}

\subsubsection*{Format}

CToPtr rd, cb, ct

\begin{center}
\cherithreeop[header]{0x12}{rd}{cb}{ct}
\end{center}

\usesDDCinsteadofNULL{ct}

\subsubsection*{Description}

If \emph{cb} has its tag bit unset (i.e. it is either the
NULL capability, or contains some other non-capability data),
then \emph{rd} is set to zero. Otherwise, \emph{rd} is set to
\emph{cb}.\cbase{} $+$ \emph{cb}.\coffset{} - \emph{ct}.\cbase{}

This instruction can be used to convert a capability into a pointer that
uses the C language convention that a zero value represents the NULL pointer.
Note that \emph{rd} will also be zero if \emph{cb}.\cbase{} $+$
\emph{cb}.\coffset{} $=$ \emph{ct}.\cbase{};
this is similar to the C language not being able to distinguish a
NULL pointer from a pointer to a structure at address 0.
\arnote{Should we just return the address for untagged values to handle cases such as \ccode{(void* __capability)-1)}?}

% XXXRW: Address notes:
%
%\note{dc552}{We need to output 0 if the input capability is out of bounds for the
%gsl::span and related bounds checking to work.  I have added the required check
%commented out.}
% \arnote{why can't we just use the capability directly in hybrid mode?}
%
%\note{dc552}{The names cb and ct are confusing for this instruction.  cb is not
%the base register and ct is not the target.}

\arnote{The operand order here is inconsistent with cfromptr/cbuildcap: the authorizing capability is the third operand instead of the second.}

\subsubsection*{Semantics}

\sailMIPScode{CToPtr}

\subsubsection*{Exceptions}

A coprocessor 2 exception will be raised if:

\begin{itemize}
\item
\emph{ct}.\ctag{} is not set.
\end{itemize}

\subsubsection*{Notes}

\begin{itemize}
\item
\emph{ct} being sealed will not cause an exception to be raised.
This is for further study.
\item
This instruction has two different means of returning an error code:
raising an exception (if \emph{ct}.\ctag{} is not set, or the registers
are not accessible) and returning a NULL pointer if \emph{cb}.\ctag{}
is not set.
\item
If the range of \emph{cb} is outside the range of \emph{ct}, a pointer relative
to \emph{ct} can't always be used in place of \emph{cb}: some reads or writes
will fail because they are outside the range of \emph{ct}. To handle this case,
the application can use the \insnmipsref{CTestSubset} instruction
followed by a conditional move.

\item \insnmipsref{CGetAddr} similarly allows access to the sum of the base
and offset of the operand capability, but without the translation relative to the
authorizing capability or validity/sealed checks on the operand.
\end{itemize}
