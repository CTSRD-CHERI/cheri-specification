\clearpage
\phantomsection
\addcontentsline{toc}{subsection}{CCopyType}
\insnmipslabel{ccopytype}
\subsection*{CCopyType: Import the otype field of a Capability}

\subsubsection*{Format}

CCopyType cd, cb, ct

\begin{center}
\begin{bytefield}{32}
\bitheader[endianness=big]{0,5,6,10,11,15,16,20,21,25,26,31}\\
\bitbox{6}{0x12}
\bitbox{5}{0x0}
\bitbox{5}{cd}
\bitbox{5}{cb}
\bitbox{5}{ct}
\bitbox{6}{0x1e}
\end{bytefield}
\end{center}

\subsubsection*{Description}


\insnmipsref{CCopyType} attempts to interpret the contents of \emph{ct} as
if it were a valid capability (even though \emph{ct}.\ctag{} is not required
to be set, and so might contain any bit pattern),
and extracts its \cotype{} field.
If \emph{ct} is sealed, \emph{cd} is set to \emph{cb} with its \coffset{} field set to
\emph{ct}.\cotype{} $-$ \emph{cb}.\cbase{}. If \emph{ct} is not sealed,
\emph{cd} is set to the NULL capability with its \cbase{} $+$ \coffset{} fields
set to $-1$.

\subsubsection*{Semantics}
\sailMIPScode{CCopyType}

\subsubsection*{Exceptions}

A coprocessor 2 exception is raised if:

\begin{itemize}
\item
\emph{cd}, \emph{cb} or \emph{ct} are reserved registers and \PCC{} does not
grant \emph{Access\_System\_Registers} permission.
\item
\emph{cb}.\ctag{} is not set.
\item
\emph{cb} is sealed.
\item
\emph{ct}.\cotype{} is outside the bounds of \emph{cb}.
\end{itemize}

\subsubsection*{Notes}

\begin{itemize}
\item
The intended use case for this instruction is as part of a routine for
resetting the tag bit on a capability that has had its tag bit cleared
(e.g. by being swapped out to disk and then back into memory).

It is a requirement of this specification that if a capability has its
tag bit cleared (either with \insnmipsref{CClearTag} or by copying it as
data), and \insnmipsref{CCopyType} is used to extract the \cotype{} field
of the result, then \emph{cd}.\cbase{} $+$ \emph{cd}.\coffset{} will be
equal to the \cotype{} of the original capability if it was sealed,
and \emph{cd}.\coffset{} will be -1 if the original capability was not
sealed.
\item
Typical usage of this instruction will be to use \insnmipsref{CBuildCap} to
extract the bounds and permissions of a capability, \insnmipsref{CCopyType}
to extract the \cotype{}, and then use \insnmipsref{CCSeal} to seal the
result of the first step with the correct \cotype{}.
\item
This instruction is an optimization. A similar effect could be achieved by
using \insnmipsref{CGetType} to get \emph{ct}.\cotype{} and then
\insnmipsref{CSetOffset} to set \emph{cd}.\coffset{}.
\item
-1 is not a valid value for the \cotype{} field, so the result distinguishes
between the case when \emph{ct} was sealed and the case when it was not
sealed.
\item
If \emph{ct} is sealed and an exception is not raised, then the result is
guaranteed to be representable, because the bounds checks ensure that
\emph{cd}'s cursor is within its bounds.
\item
If \emph{ct}.\cotype{} is outside of the bounds of \emph{ct}, this is an
error condition (attempting to reconstruct a capability that \emph{cb} does
not give you permission to create). In order to catch this error condition
near to where the problem occurred, we raise an exception. This also has
the benefit of avoiding the case where changing \emph{cb}'s \coffset{}
results in a value that is not representable, as explained in the previous
note.
\end{itemize}

\mrnote{What if \emph{cb} covers only a portion of the \cotype{} space,
and \emph{ct}'s \cotype{} doesn't fall within it? As currently defined,
we raise an exception. But what if we want to detect this error and handle
it somehow? What if we have several capabilities covering different portions
of the type space, and want to try them in turn until one works?}
