\clearpage
\phantomsection
\addcontentsline{toc}{subsection}{CSetBoundsImm}
\insnmipslabel{csetboundsimm}
\subsection*{CSetBoundsImm: Set Bounds (Immediate)}

\subsubsection*{Format}

CSetBounds cd, cb, length$_{imm}$

\begin{center}
\begin{bytefield}{32}
\bitheader[endianness=big]{0,10,11,15,16,20,21,25,26,31}\\
\bitbox{6}{0x12}
\bitbox{5}{0x14}
\bitbox{5}{cd}
\bitbox{5}{cb}
\bitbox{11}{length$_{imm}$}
\end{bytefield}
\end{center}

\arnote{The assembler supports both CSetBounds and CSetBounds but I think we should always use CSetBoundsImm}

\subsubsection*{Description}

Capability register \emph{cd} is replaced with a capability that:

\begin{itemize}
\item
Grants access to a subset of the addresses authorized by \emph{cb}.
That is, \emph{cd}.\cbase{} $\ge$ \emph{cb}.\cbase{} and
\emph{cd}.\cbase{} $+$ \emph{cd}.\clength{} $\le$ \emph{cb}.\cbase{} $+$
\emph{cb}.\clength{}.
\item
Grants access to at least the addresses \emph{cb}.\cbase{} $+$
\emph{cb}.\coffset{} $\ldots$ \emph{cb}.\cbase{} $+$ \emph{cb}.\coffset{}
$+$ \emph{length$_{imm}$} $-$ 1.
That is, \emph{cd}.\cbase{} $\le$ \emph{cb}.\cbase{}
$+$ \emph{cb}.\coffset{} and \emph{cd}.\cbase{} $+$ \emph{cd}.\clength{}
$\ge$ \emph{cb}.\cbase{} $+$ \emph{cb}.\coffset{} $+$ \emph{length$_{imm}$}.
\item
Has an \coffset{} that points to the same memory location as \emph{cb}'s
\coffset{}.
That is, \emph{cd}.\coffset{} = \emph{cb}.\coffset{} + \emph{cb}.\cbase{} -
\emph{cd}.\cbase{}.
\item
Has the same \cperms{} as \emph{cb}, that is, \emph{cd}.\cperms{} = \emph{cb}.\cperms{}.
\end{itemize}

%When the hardware uses a 256-bit representation for capabilities, the bounds
%of the destination capability \emph{cd} are exactly as requested.

With compressed capabilities, not all combinations of \cbase{} and \clength{}
are representable.
\emph{cd} may therefore grant access to a range of memory addresses that is
wider than requested, but is still guaranteed to be within the bounds of
\emph{cb}.

\subsubsection*{Semantics}
\sailMIPScode{CSetBoundsImmediate}

\subsubsection*{Exceptions}

A coprocessor 2 exception is raised if:

\begin{itemize}
\item
\cchecktag{}
\item
\emph{cb} is sealed.
\item
\emph{cursor} $<$ \emph{cb}.\cbase{}
\item
\emph{cursor} $+$ \emph{length$_{imm}$} $>$ \emph{cb}.\cbase{} $+$ cb.\clength{}
\end{itemize}

\subsubsection*{Notes}

\begin{itemize}
\item
In the above Sail code, arithmetic is over the mathematical integers and
\emph{length$_{imm}$} is unsigned, so a large value of \emph{length$_{imm}$} cannot cause
\emph{cursor} $+$ \emph{length$_{imm}$} to wrap around and be less than \emph{cb}.\cbase{}.
Implementations (that, for example, will probably use a fixed number of
bits to store values) must handle this overflow case correctly.
\item If this instruction is used with \creg{0} as the destination register, it can be used to assert that a given capability grants access to at least \emph{length$_{imm}$} bytes. An assembler pseudo instruction \insnmipsref{CAssertInBounds} is supported for this use case.
\end{itemize}
