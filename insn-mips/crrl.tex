\clearpage
\phantomsection
\addcontentsline{toc}{subsection}{CRoundRepresentableLength}
\insnmipslabel{crrl}
\insnmipslabel{croundrepresentablelength}
\subsection*{CRoundRepresentableLength: Round to Next Precisely Representable Length}

\subsubsection*{Format}

CRoundRepresentableLength rt, rs

\begin{center}
\cheritwoop[header]{0x10}{rt}{rs}
\end{center}

\subsubsection*{Description}

\emph{rt} is set to the smallest value greater or equal to \emph{rs} that can be used by \insnmipsref{CSetBoundsExact} without trapping (assuming a suitably aligned base).

\subsubsection*{Semantics}

\sailMIPScode{CRAP}

\subsubsection*{Notes}

\begin{itemize}
\item This instruction is useful when implementing allocators to round up allocation sizes to a size that can be precisely bounded (and will therefore not overlap with any other allocations). It is also useful when using \ccode{mmap()}, since the requested size must be a precisely representable length.
\item An alternative to this instruction is the use of count-leading-zeroes and count-trailing-zeros instructions followed by shifting and masking. However, this requires encoding knowledge of the underlying precision in the resulting binary and can therefore result in incompatibilities with future architectures that use a different compression algorithm.
\item If the length has to be rounded up to $2^{64}$ then this instruction will return zero. This could happen for very large lengths that span most of the address space. Software must be careful to account for this, especially if the length comes from an untrusted source.
\end{itemize}
