\clearpage
\phantomsection
\addcontentsline{toc}{subsection}{CPtrCmp: CEQ, CNE, CL[TE][U], CEXEQ}
\insnmipslabel{cptrcmp}
\insnmipslabel{ceq}
\insnmipslabel{cne}
\insnmipslabel{clt}
\insnmipslabel{cle}
\insnmipslabel{cltu}
\insnmipslabel{cleu}
\insnmipslabel{cexeq}
\insnmipslabel{cnexeq}
\subsection*{CPtrCmp: CEQ, CNE, CL[TE][U], CEXEQ, CNEXEQ: Capability Pointer Compare}

\subsubsection*{Format}

CEQ rd, cb, ct \\
CNE rd, cb, ct \\
CLT rd, cb, ct \\
CLE rd, cb, ct \\
CLTU rd, cb, ct \\
CLEU rd, cb, ct \\
CEXEQ rd, cb, ct \\
CNEXEQ rd, cb, ct \\
\cherithreeop[header]{\emph{op}}{rd}{cb}{ct}

\subsubsection*{Description}

Capability registers \emph{cb} and \emph{ct} are compared, and the result
of the comparison is placed in integer register \emph{rd}.
The rules for comparison are as follows:

\begin{itemize}
\item For all operations except \insnmipsref{CExEq} and \insnmipsref{CNExEq},
the result of comparison is the result of comparing \emph{cb}.\caddr{} with \emph{ct}.\caddr{}.
Numerical comparison is signed for \insnmipsref{CLT} and \insnmipsref{CLE},
and unsigned for \insnmipsref{CLTU} and \insnmipsref{CLEU}.
\item
\insnmipsref{CExEq} and \insnmipsref{CNExEq} compare all the fields of the two
capabilities, including \ctag{} and the bits that are reserved for future use.
\end{itemize}

This instruction can be used to compare capabilities so that capabilities can
replace pointers in C executables.


\begin{figure}[h]
\begin{center}
\begin{tabular}{l|l|l}
Mnemonic & \emph{op} & Comparison \\ \hline
\insnmipsref{CEQ} & 0x14 & = \\
\insnmipsref{CNE} & 0x15 & $\neq$ \\
\insnmipsref{CLT} & 0x16 & $<$ (signed) \\
\insnmipsref{CLE} & 0x17 & $\le$ (signed) \\
\insnmipsref{CLTU} & 0x18 & $<$ (unsigned) \\
\insnmipsref{CLEU} & 0x19 & $\le$ (unsigned) \\
\insnmipsref{CEXEQ} & 0x1a & all fields are equal \\
\insnmipsref{CNEXEQ} & 0x21 & not all fields are equal \\
\end{tabular}
\end{center}
\end{figure}

\subsubsection*{Semantics}
\sailMIPScode{CPtrCmp}

\subsubsection*{Exceptions}

A reserved instruction exception is raised if

\begin{itemize}
\item
\emph{op} does not correspond to comparison operation whose meaning has been
defined. (All possible values of \emph{op} have now been assigned meanings,
so this exception cannot occur).
\end{itemize}

\subsubsection*{Notes}

\begin{itemize}
\item
\insnmipsref{CLTU} can be used by a C compiler to compile code that
compares two non-NULL pointers (e.g., to detect whether a pointer to a character
within a buffer has reached the end of the buffer). When two pointers to
addresses within the same object (e.g., to different offsets within an array)
are compared, the pointer to the earlier part of the object will be compared
as less. (Signed comparison would also work as long as the object did not span
 address $2^{63}$; the MIPS address space layout makes it unlikely that
objects spanning $2^{63}$ will exist in user-space C code).
\item
Although the ANSI C standard does not specify whether a NULL
pointer is less than or greater than a non-NULL pointer (clearly, they must
not be equal), the comparison instructions have been designed so that when
C pointers are represented by capabilities, NULL will be less than any
non-NULL pointer.
\item
A C compiler may also use these instructions to compare two values of
type \ccode{uintptr\_t} that have been obtained by casting from
an integer value. If the cast is compiled as a \insnmipsref{CFromPtr} of
zero followed by \insnmipsref{CSetOffset} to the integer value, the
result of \insnmipsref{CPtrCmp} will be the same as comparing the original
integer values, because \insnmipsref{CFromPtr} will have set \cbase{} to
zero. Signed and unsigned capability comparison operations are provided so
that both signed and unsigned integer comparisons can be performed on
capability registers.
\item
A program could use pointer comparison to determine the value of
\cbase{}, by setting \coffset{} to different values and testing which values
cause \cbase{} $+$ \coffset{} to wrap around and be less than \cbase{} $+$
a zero offset. This is not an attack against a security property of the ISA,
because \cbase{} is not a secret.
\item
One possible way in which garbage collection could be implemented is for the
garbage collector to move an object and fix up all capabilities that refer
to it. If there are appropriate restrictions on which capabilities the program
has to start with, the garbage collector can be sure that the program does
not have any references to the object stored as integers, and so can know
that it is safe to move the object. With this type of garbage collection,
comparing pointers by extracting their base and offset with
\insnmipsref{CGetBase} and \insnmipsref{CGetOffset} and comparing the
integer values is not guaranteed to work, because the garbage collector
might have moved the object part-way through. \insnmipsref{CPtrCmp} is
atomic, and so will work in this scenario.
\item
Some compilers may make the optimization that if a check for (\emph{a} $=$
\emph{b}) has succeeded, then \emph{b} can be replaced with \emph{a} without
changing the semantics of the program. This optimization is not valid
for the comparison performed by \insnmipsref{CEq}, because two capabilities
can point to the same place in memory but have different bounds, permissions
etc. and so not be interchangeable. The \insnmipsref{CExEq} instruction is
provided for when a test for semantic equivalence of capabilities is
needed; it compares all the fields, even the ones that are reserved for
future use.
\item
Mathematically, \insnmipsref{CEq} divides capabilities into
\emph{equivalence classes}, and the signed or unsigned comparison operators
provide a \emph{total ordering} on these equivalence classes.
\insnmipsref{CExEq} also divides capabilities into equivalence classes,
but these are not totally ordered: two capabilities can be unequal according
to \insnmipsref{CExEq}, and also neither less or greater according to
\emph{CLT} (e.g., if they have the same \cbase{} $+$ \coffset{}, but different
\clength{}).
\item
There is an outstanding issue: when capability compression is in use, does
\insnmipsref{CExEq} compare the compressed representation or the uncompressed
capability? There might be a difference between the two if there are multiple
compressed representations that decompress to the same thing. If
\ctag{} is false, then then capability register might contain non-capability
data (e.g., an integer, or a string) and it might not decompress to anything
sensible. Clearly in this case the in-memory compressed representation should
be compared bit for bit. Is it also acceptable to compare the compressed
representations when \ctag{} is true? This might lead to two capabilities that
are sematically equivalent but have been computed by a different sequence
of operations comparing as not equal. The consequence of this for programs
that use \insnmipsref{CExEq} is for further study.
\item
If a C compiler compiles pointer equality as \insnmipsref{CExEq} (rather than
\insnmipsref{CEq}), it will catch the following example of undefined
behavior. Suppose that \emph{a} and \emph{b} are capabilities for different
objects, but \emph{a} has been incremented until its \cbase{} $+$ \coffset{}
points to the same memory location as \emph{b}. Using \insnmipsref{CExEq},
these pointers will not compare as equal because they have different bounds.
\item
\insnmipsref{CNE} and \insnmipsref{CNExEq} are in principle redundant,
because a compiler could replace \insnmipsref{CNE} and a conditional branch
with \insnmipsref{CEq} and a conditional branch with the opposite condition
(and if the result of the comparison is assigned to a variable, the compiler
could explictly negate the result of \insnmipsref{CEq}, at a small performance
penalty). We provide explicit tests for not equal in order to simplify the
compiler back end.
\end{itemize}
