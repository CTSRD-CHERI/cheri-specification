\clearpage
\phantomsection
\addcontentsline{toc}{subsection}{CInvoke}
\insnmipslabel{cinvoke}
\subsection*{CInvoke: Call into Another Security Domain}

\subsubsection*{Format}

CInvoke cs, cb

\begin{center}
\begin{bytefield}{32}
\bitheader[endianness=big]{0,10,11,15,16,20,21,25,26,31}\\
\bitbox{6}{0x12}
\bitbox{5}{0x05}
\bitbox{5}{cs}
\bitbox{5}{cb}
\bitbox{11}{0x001}
\end{bytefield}
\end{center}

\subsubsection*{Description}

\insnmipsref{CInvoke} is used to jump between protection domains,
unsealing sealed code and data-capability operands, subject to checks on those
capabilities.
This allows the target protection domain to gain access to a different set of capabilities,
supporting implementation of software encapsulation.
The two operand capabilities must be accessible, be valid capabilities, be
sealed, have matching types, and have suitable permissions and bounds, or an
exception will be thrown.
\emph{cs} contains a sealed code capability for the target subsystem, which
will be unsealed and loaded into \PCC{}.
\emph{cb} contains a sealed data capability for the target subsystem, which
will be unsealed and loaded into \IDC{}.
In the parlance of object-oriented programming, \emph{cb} is a capability for
an \emph{object}'s instance data, and \emph{cs} is a capability for the
methods of the object's class.

With \insnmipsref{CInvoke}, a constrained form of non-monotonicity is supported in
the architecture.
Privilege is escalated by virtue of \insnmipsref{CInvoke}
unsealing sealed operand capability registers during a controlled transfer of
execution to the callee in a jump-style transfer of control.
\insnmipsref{CInvoke} requires \emph{Permit\_CInvoke} to be present on both
 sealed capability operands.

\subsubsection*{Semantics}

\sailMIPScode{CCallA}

\noindent
\note{rmn30}{XXX There is a test that CInvoke in branch delay slot of ordinary branch raises Reserved Instruction exception. This needs to be specified if desired. Does it apply to both version of CInvoke? }
\insnmipsref{CInvoke} executes like a branch in the pipeline, but does
not have a branch delay slot. This is due to the difficulty of allowing one
instruction from the calling domain to execute in the new domain.
See Section~\ref{sec:jump-based-domain-transition}.
% XXXRW: Address note:
%
%\note{jdw57}{What is the formal notation for describing an exception condition
%for the following instruction?}

\subsubsection*{Exceptions}

A coprocessor 2 exception is raised if:

\begin{itemize}
\item
\emph{cs} is not sealed.
\item
\emph{cb} is not sealed.
\item
\emph{cs}.\cotype{} $\ne$ \emph{cb}.\cotype{}
\item
\emph{cs}.\cperms{}.\emph{Permit\_Execute} is not set.
\item
\emph{cb}.\cperms{}.\emph{Permit\_Execute} is set.
\item
\emph{cs}.\coffset{} $\ge$ \emph{cs}.\clength{}.
\item
\emph{cs}.\cperms{}.\emph{Permit\_CInvoke} is not set
\item
\emph{cb}.\cperms{}.\emph{Permit\_CInvoke} is not set
\end{itemize}

\subsubsection*{Notes}

\begin{itemize}
\item
From the point of view of security, \insnmipsref{CInvoke} needs to be an atomic
operation (i.e.  the caller cannot decide to just do some of it, because
partial execution could put the system
into an insecure state).
From a hardware perspective, more complex domain-transition implementations
(e.g., to implement function-call semantics or message passing) may need to
perform multiple memory reads and writes, which might take multiple cycles and
complicate control logic.

\item
Implementations may choose to restrict the register numbers that may be passed as \emph{cs} and \emph{cb} in order to avoid the need to decode the instruction and identify the register arguments.
\jhbnote{Is this still true of CInvoke (selector 1) compared to CCall
  selector 0?}

\item
The unsealing of the capabilities stored to PCC and IDC may have implications
beyond just the object type of these capabilities.
When capability compression is in use, the
microarchitectural bit representation of other fields within a capability
may depend on the value of the \cotype{}, so this assignment may have the
effect of changing the bit representation of the other fields. i.e., a
hardware implementation may need to change the representation of the rest
of the capability.
(In the deprecated CHERI-128 of \cref{app:cheri-128},
for example, which does not have a dedicated \cotype{} field,
\insnmipsref{CInvoke} clears the sealed bits of the capabilities
stored to PCC and IDC but must also zero
the bits that held the \cotype{} values
and are now part of the bounds metadata.)

\end{itemize}

\rwnote{Need to add notes on \insnmipsref{CSetCID} here, as well as examples
  to pseudocode.}

\subsubsection*{Expected Software Use}

Higher-level software protection-domain transitions transform the capability
register file to reduce or expand the set of code and data rights available to
the executing thread of control.
In CHERI-based software, these transitions can be usefully modeled as function
invocation or message passing in which data and capability registers are
passed as arguments or messsages, and in which callers and callees can be
protected from undesired access to internal state from the other party (i.e.,
encapsulation).
Domain transitions may implement symmetric (mutual) or asymmetric distrust
between caller and callee, depending on guarantees about limiting callee
access to caller state, and vice versa.

\insnmipsref{CInvoke} may be used to implement mutual distrust by entering a more
privileged ``trusted intermediary'' able to perform capability and
integer register clearing, saving, and restoring, as well as tracking
properties of communications such as message passing or implementing a trusted
stack for reliable call-return semantics and error recovery.
The \insnmipsref{CInvoke} instruction performs a set of checks on sealed
operand capabilities allowing
domain transition to be more efficient.

A number of use cases can be formulated, depending on trust
model.
To implement mutual distrust, sealed code capabilities must point to an
intermediary that is trusted by the callee to implement escalation to
callee privilege.
With respect to capabilities, the caller can perform its own register
clearing and encapsulation of (optional) return state passed via register
arguments to the callee.
\insnmipsref{CInvoke} does not implement a link register, allowing the
calling convention
to implement semantics not implying a leak of \PCC{} to the callee.
In our CheriOS software prototype, sealed code capabilities refer to one of a
set of message-passing implementations, with the sealed data capability
describing the message ring and target domain's code and data capabilities.
A second \insnmipsref{CJR} out of the message-passing implementation into the
callee, combined with suitable register clearing, is suitable to deescalate
privilege to the callee protection domain without a second use of
\insnmipsref{CInvoke}.

\subsubsection*{Sketch of the CheriBSD CInvoke Model}

The CheriBSD \insnmipsref{CInvoke} model implements domain transition via
privileged call and return handlers.
Modeled on function invocation, the call handler performs various
checks (such as of operand register accessibility, validity, sealing, types,
and permissions) on argument registers.
If the checks pass, the handler unseals the sealed operand capabilities,
installing them in \PCC{} and \IDC{}.
It also clears other non-argument registers to prevent data and capability
leakage from caller to callee.
In addition, CheriBSD implements a trusted stack that tracks caller \PCC{} and
\IDC{} so that the return handler can restore control (and
security state) one instruction after the original call site.
Finally, the call handler also implements a form of capability flow
control by preventing the passing of non-global capabilities between caller
and callee.
The return handler pops an entry from the trusted stack, suitably clears
non-return registers, and performs capability flow-control on non-global return
capabilities.
If the return handler is invoked on an empty trusted stack,
the handler raises an exception.
If the call handler is invoked when the trusted stack is full,
the call fails and returns an error to the caller.

\subsubsection*{Sketch of the CheriOS CInvoke Model}

In the CheriOS model, \insnmipsref{CInvoke} is used to implement an
asynchronous message-passing semantic.
The sealed code capability is directed to a software message-passing
implementation that acts as a ``trusted intermediary'', and the sealed data
capability refers to a description of the destination domain including message
ring.
The message-passing implementation adds argument registers to the ring, and
will then either return control to the sender context, or continue in to the
recipient context.
This is accomplished by suitable register-file manipulation to give up any
unnecessary privilege, and an ordinary capability jump to pass control to an
appropriate unprivileged domain.
As with the CheriBSD model, the message-passing routine must perform any
necessary saving of caller context, checking and clearing of registers, and
installation of callee context to support safe interactions.
