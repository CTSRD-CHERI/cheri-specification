\clearpage
\phantomsection
\addcontentsline{toc}{subsection}{CTestSubset}
\insnmipslabel{ctestsubset}
\subsection*{CTestSubset: Test that Capability is a Subset of Another}

\subsubsection*{Format}

CTestSubset rd, cb, ct

\begin{center}
\begin{bytefield}{32}
\bitheader[endianness=big]{0,5,6,10,11,15,16,20,21,25,26,31}\\
\bitbox{6}{0x12}
\bitbox{5}{0x0}
\bitbox{5}{rd}
\bitbox{5}{cb}
\bitbox{5}{ct}
\bitbox{6}{0x20}
\end{bytefield}
\end{center}

\usesDDCinsteadofNULL{cb}

\subsubsection*{Description}

\insnmipsref{CTestSubset} tests if the bounds of \emph{ct} are within the
bounds of \emph{cb}, and the permissions of \emph{ct} are within the permissions
of \emph{cb}, setting \emph{rd} to \emph{1} if so, and \emph{0} if not.

\note{mr101}{What should happen if one of the capabilities is NULL? Is the
NULL capability a subset of a valid capability? Comparing the \cbase{} and
\clength{} fields when the \ctag{} bit is clear seems the wrong thing to do}

\note{mr101}{Is a zero length capability a subset of a valid capability,
even if it's \cbase{} is not within the range. If you view the operation as
a subset of the memory bytes, then it is a subset}

\note{mr101}{The motivating use case is \insnmipsref{CToPtr},
\insnmipsref{CTestSubset}, \insnnoref{MOVZ reg, zero}. In error cases
where \insnmipsref{CToPtr} has returned zero, we don't really care what
\insnmipsref{CTestSubset} returns, because we're going to get the NULL
pointer anyway.}

\subsubsection*{Semantics}
\sailMIPScode{CTestSubset}

\subsubsection*{Exceptions}

A coprocessor 2 exception is raised if:

\begin{itemize}
\item
\emph{cd}, \emph{cb} or \emph{ct} is a reserved register and \PCC.\cperms{} does
not grant \emph{Permit\_Access\_System\_Registers}.
\end{itemize}

\subsubsection*{Notes}

\begin{itemize}
\item
This instruction was originally motivated as an additional check for
\insnmipsref{CToPtr}.
A conversion of a capability to a pointer with respect to a default capability
would normally expect that the entire capability is accessible within the
default capability with (at least) the original permissions.
\insnmipsref{CTestSubset} can perform this assertion, and a \insnmipsref{CMove}
instruction can replace the result of the \insnmipsref{CToPtr} with NULL upon
failure.
\item
Another use case for this instruction is in garbage collection. For this
application, we want to be able to test if one capability is a subset of
the other even if one is sealed and the other is not. (For the purposes of
garbage collection, a sealed reference to a region of memory is still a
reference to that region of memory). With compressed capabilities, the bounds
are represented differently for sealed and unsealed capabilities, but
\insnmipsref{CTestSubset} is still able to perform the subset check.
\end{itemize}
