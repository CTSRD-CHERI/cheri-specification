function #\hyperref[sailRISCVzsetCapBounds]{setCapBounds}#(cap, base, top) : (Capability, CapAddrBits, CapLenBits) -> (bool, Capability) = {
  /* {cap with base=base; length=(#\hyperref[sailRISCVzbits]{bits}#(64)) length; offset=0} */
  let ext_base = 0b0 @ base;
  let length = top - ext_base;
  /* Find an exponent that will put the most significant bit of length
   * second from the top as assumed during decoding. We ignore the bottom
   * MW bits because those are handled by the ie = 0 format.
   */
  let e = maxE - #\hyperref[sailRISCVzcountzyleadingzyzzeros]{count\_leading\_zeros}#(length[cap_addr_width..cap_mantissa_width - 1]);
  // Use use internal exponent if e is non-zero or if e is zero but
  // but the implied bit of length is not #\hyperref[sailRISCVzzzero]{zero}# (denormal vs. normal case)
  let ie = (e != 0) | length[cap_mantissa_width - 2] == bitone;

  /* The non-ie e == 0 case is easy. It is exact so just extract relevant bits. */
  Bbits = #\hyperref[sailRISCVztruncate]{truncate}#(base, cap_mantissa_width);
  Tbits = #\hyperref[sailRISCVztruncate]{truncate}#(top, cap_mantissa_width);
  lostSignificantTop : bool = false;
  lostSignificantBase : bool = false;
  incE : bool = false;

  if ie then {
    /* the internal exponent case is trickier */

    /* Extract B and T #\hyperref[sailRISCVzbits]{bits}# (we lose 3 bits of each to store the exponent) */
    B_ie = #\hyperref[sailRISCVztruncate]{truncate}#(base >> (e + 3), cap_mantissa_width - 3);
    T_ie = #\hyperref[sailRISCVztruncate]{truncate}#(top >> (e + 3), cap_mantissa_width - 3);

    /* Find out whether we have lost significant bits of base and top using a
     * mask of bits that we will #\hyperref[sailRISCVzlose]{lose}# (including 3 extra for exp).
     */
    maskLo : CapLenBits = #\hyperref[sailRISCVzEXTZ]{EXTZ}#(#\hyperref[sailRISCVzones]{ones}#(e+3));
    lostSignificantBase = #\hyperref[sailRISCVzunsigned]{unsigned}#(ext_base & maskLo) != 0;
    lostSignificantTop  = #\hyperref[sailRISCVzunsigned]{unsigned}#(top & maskLo) != 0;

    if lostSignificantTop then {
      /* we must increment T to make sure it is still above top even with lost bits.
         It might wrap around but if that makes B<T then decoding will compensate. */
      T_ie = T_ie + 1;
    };

    /* Has the length overflowed? We chose e so that the top two bits of len would be 0b01,
       but either because of incrementing T or losing bits of base it might have grown. */
    len_ie = T_ie - B_ie;
    if len_ie[cap_mantissa_width - 4] == bitone then {
      /* length overflow -- increment E by one and then recalculate
         T, B etc accordingly */
      incE = true;

      lostSignificantBase = lostSignificantBase | B_ie[0] == bitone;
      lostSignificantTop  = lostSignificantTop | T_ie[0] == bitone;

      B_ie = #\hyperref[sailRISCVztruncate]{truncate}#(base >> (e + 4), cap_mantissa_width - 3);
      let incT : range(0,1) = if lostSignificantTop then 1 else 0;
      T_ie = #\hyperref[sailRISCVztruncate]{truncate}#(top >> (e + 4), cap_mantissa_width - 3) + incT;
    };

    Bbits = B_ie @ 0b000;
    Tbits = T_ie @ 0b000;
  };
  let newCap = {cap with address=base, E=#\hyperref[sailRISCVztozybits]{to\_bits}#(6, if incE then e + 1 else e), B=Bbits, T=Tbits, internal_e=ie};
  let exact = #\hyperref[sailRISCVznot]{not}#(lostSignificantBase | lostSignificantTop);
  (exact, newCap)
}
