hex_bits_28_forwards_matches : bits(28) -> bool