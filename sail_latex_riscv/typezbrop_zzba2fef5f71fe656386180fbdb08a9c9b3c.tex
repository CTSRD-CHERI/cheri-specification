enum brop_zba = {RISCV_SH1ADD, RISCV_SH2ADD, RISCV_SH3ADD}
