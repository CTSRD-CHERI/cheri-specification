\chapter{The CHERI-x86-64 Instruction-Set Reference}
\label{chap:isaref-x86-64}

\newcolumntype{Y}{>{\centering\arraybackslash}X}
\newcolumntype{Z}{>{\raggedright\arraybackslash}X}

\newenvironment{x86opcodetable}{%
  \tabularx{\textwidth}{| l | l | p{2.0em} | p{2.5em} | p{2.5em} | Z |} \hline
    \textbf{Opcode} & \textbf{Instruction} & \textbf{Op/ En} &
    \textbf{Cap Mode} & \textbf{64-bit Mode} & \textbf{Description}\\
    \hline
}{%
  \endtabularx
}

\newcommand{\xopcode}[6]{%
  #1 & #2 & #3 & #4 & #5 & #6\\
  \hline
}

\newenvironment{x86opentable}{%
  \bigskip
  \noindent
  \tabularx{\textwidth}{| c | Y | Y | Y | Y |}
    \multicolumn{5}{c}{\bfseries Instruction Operand Encoding}\\
    \hline
    Op/En & Operand 1 & Operand 2 & Operand 3 & Operand 4\\
    \hline
}{%
  \endtabularx
}

\newcommand{\xopen}[5]{%
  #1 & #2 & #3 & #4 & #5\\
  \hline
}

In this chapter, we specify new CHERI instructions as well as
extensions to existing instructions to support capability-sized
operands.  Instructions are described using similar syntax to Volume 2
of Intel's Software Developer's Manual~\cite{intel-sdm-vol2} with a
few extensions.

An additional symbol is defined to represent object code in the
``Opcode'' column:

\begin{itemize}
  \item \textbf{CAP} { }---{ } Indicates the use of the capability
    operand prefix.
\end{itemize}

Additional symbols are defined to represent operands in the
``Instruction'' column:

\begin{itemize}
  \item \textbf{rc} { }---{ } One of the general-purpose capability
    registers: \CAX{}, \CBX{}, \CCX{}, \CDX{}, \CDI{}, \CSI{}, \CBP{},
    \CSP{}, \creg{8}-\creg{15}.

  \item \textbf{r/mc} { }---{ } A capability operand that is either
    the contents of one of the capability registers for \textbf{rc} or
    a capability in memory.
\end{itemize}

In addition, all of these instructions are either invalid or not
encodable in Compatibility/Legacy mode, so that column is omitted from
opcode tables.  However, a new column is added to describe capability
mode support using one of the following annotations:

\begin{itemize}
  \item \textbf{V} { }---{ } Supported.
  \item \textbf{I} { }---{ } Not supported.
\end{itemize}

\clearpage
\section{Extensions to x86-64 Instructions}

This section contains extensions to existing instructions to support
capability operands.  For each of these instructions, the instruction
description should be treated as an extension to the description of
the existing instruction in Volume 2 of Intel's Software Developer's
Manual.  Many of the instruction descriptions in this section reuse
language from Intel's manual to highlight the similarity in semantics
between the base instructions and their CHERI extensions.

\clearpage
\phantomsection
\addcontentsline{toc}{subsection}{MOV -- Move}
\insnxeslabel{mov}
\subsection*{MOV -- Move}

\begin{x86opcodetable}
  \xopcode{CAP + 89 \emph{/r}}{MOV \emph{r/mc, rc}}
  {MR}{Valid}{Valid}
  {Move \emph{rc} to \emph{r/mc}.}
  \xopcode{CAP + 8B \emph{/r}}{MOV \emph{rc, r/mc}}
  {RM}{Valid}{Valid}
  {Move \emph{r/mc} to \emph{rc}.}
  \xopcode{CAP + C7 \emph{/0 id}}{MOV \emph{r/mc, imm32}}
  {MI}{Valid}{Valid}
  {Move \emph{imm32 sign-extended to 64-bits} to \emph{r/mc}.}
\end{x86opcodetable}

\begin{x86opentable}
  \xopen{MR}{ModRM:r/m (w)}{ModRM:reg (r)}{NA}{NA}
  \xopen{RM}{ModRM:Reg (w)}{ModRM:r/m (r)}{NA}{NA}
  \xopen{MI}{ModRM:r/m (w)}{imm32}{NA}{NA}
\end{x86opentable}

\subsubsection*{Description}

Copies the source operand to the destination operand. The destination
operand can be a register or a memory location. The source operand can
be an immediate, a register, or a memory location.  If the source
operand is an immediate, the value is sign-extended to 64-bits and
used as the address of a NULL-derived capability.

Note that some \insnnoref{MOV} opcodes such as \texttt{B8+ rw} and
\texttt{C6 /0} are not extended to support the \insnnoref{CAP} prefix
as the behavior would be identical.  The \texttt{C7 /0} opcode is
extended primarily to support storing constants such as NULL to
capabilities in memory without requiring an intermediate register.

The \texttt{A1} and \texttt{A3} opcodes are not extended to support
capabilities.

\subsubsection*{Flags Affected}

None

\clearpage
\phantomsection
\addcontentsline{toc}{subsection}{MOVNTI - Store Using Non-Temporal Hint}
\insnxeslabel{movnti}
\subsection*{MOVNTI - Store Using Non-Temporal Hint}

\begin{x86opcodetable}
  \xopcode{NP CAP + 0F C3 \emph{/r}}{MOV \emph{mc, rc}}
  {MR}{Valid}{Valid}
  {Move \emph{rc} to \emph{mc} using non-temporal hint.}
\end{x86opcodetable}

\begin{x86opentable}
  \xopen{MR}{ModRM:r/m (w)}{ModRM:reg (r)}{NA}{NA}
\end{x86opentable}

\subsubsection*{Description}

Moves the capability in the source operand to the destination operand
using a non-temporal hint.

\subsubsection*{Flags Affected}

None


\clearpage
\section{CHERI-x86-64 Instructions}

This section contains new instructions added to support operations on
capabilities.  The opcode assignments in this section are tentative
and subject to change.  Single byte opcodes have been used for
instructions which we believe may either be used frequently or in
frequently-accessed code paths.

\clearpage
\phantomsection
\addcontentsline{toc}{subsection}{MOV - Move to/from Additional Capability Registers}
\insnxeslabel{movcap}
\subsection*{MOV - Move to/from Additional Capability Registers}

\begin{x86opcodetable}
  \xopcode{0F 24 \emph{/r}}{MOV \emph{rc,} CFS/CGS/DDC}
  {MR}{Valid}{Valid}
  {Move additional capability register to \emph{rc}.}
  \xopcode{0F 25 \emph{/r}}{MOV CFS/CGS/DDC\emph{, rc}}
  {RM}{Valid}{Valid}
  {Move \emph{rc} to additional capability register.}
\end{x86opcodetable}

\begin{x86opentable}
  \xopen{MR}{ModRM:r/m (w)}{ModRM:reg (r)}{NA}{NA}
  \xopen{RM}{ModRM:reg (w)}{ModRM:r/m (r)}{NA}{NA}
\end{x86opentable}

\subsubsection*{Description}

Moves the contents of an additional capability register to a
general-purpose capability register or vice versa.

Similar to the \insnnoref{MOV} opcodes for control and debug
registers, the \textbf{reg} field of the ModRM byte always identifies
the additional capability register to read or write.  The \textbf{mod}
field of ModRM is ignored, and the \textbf{r/m} field identifies the
general-purpose capability register.  Attempts to reference invalid
additional capability registers will raise a UD\# exception.

Attempts to access additional capability registers other than \CFS{},
\CGS{}, or \DDC{} from a privilege level other than 0 will raise a
GP\#(0) exception.

\subsubsection*{Flags Affected}

None


\clearpage
\section{Summary of New Opcodes}

The following new opcodes are added in 64-bit mode and are also
available in capability mode.

\bigskip
\noindent
\begin{tabular}{| l | l |} \hline
  \textbf{Opcode} & \textbf{Instruction}\\
  \hline
  0F 24 \emph{/r} & \insnxesref[movcap]{MOV} \emph{rc,} CFS/CGS/DDC\\
  \hline
  0F 25 \emph{/r} & \insnxesref[movcap]{MOV} CFS/CGS/DDC\emph{, rc}\\
  \hline
\end{tabular}
