\chapter{The CHERI-x86-64 Instruction-Set Reference}
\label{chap:isaref-x86-64}

\newcolumntype{Y}{>{\centering\arraybackslash}X}
\newcolumntype{Z}{>{\raggedright\arraybackslash}X}

\newenvironment{x86opcodetable}{%
  \tabularx{\textwidth}{| l | l | p{2.0em} | p{2.5em} | p{2.5em} | Z |} \hline
    \textbf{Opcode} & \textbf{Instruction} & \textbf{Op/ En} &
    \textbf{Cap Mode} & \textbf{64-bit Mode} & \textbf{Description}\\
    \hline
}{%
  \endtabularx
}

\newcommand{\xopcode}[6]{%
  #1 & #2 & #3 & #4 & #5 & #6\\
  \hline
}

\newenvironment{x86opentable}{%
  \bigskip
  \noindent
  \tabularx{\textwidth}{| c | Y | Y | Y | Y |}
    \multicolumn{5}{c}{\bfseries Instruction Operand Encoding}\\
    \hline
    Op/En & Operand 1 & Operand 2 & Operand 3 & Operand 4\\
    \hline
}{%
  \endtabularx
}

\newcommand{\xopen}[5]{%
  #1 & #2 & #3 & #4 & #5\\
  \hline
}

In this chapter, we specify new CHERI instructions as well as
extensions to existing instructions to support capability-sized
operands.  Instructions are described using similar syntax to Volume 2
of Intel's Software Developer's Manual~\cite{intel-sdm-vol2} with a
few extensions.

An additional symbol is defined to represent object code in the
``Opcode'' column:

\begin{itemize}
  \item \textbf{CAP} { }---{ } Indicates the use of the capability
    operand prefix.
\end{itemize}

Additional symbols are defined to represent operands in the
``Instruction'' column:

\begin{itemize}
  \item \textbf{rc} { }---{ } One of the general-purpose capability
    registers: \CAX{}, \CBX{}, \CCX{}, \CDX{}, \CDI{}, \CSI{}, \CBP{},
    \CSP{}, \creg{8}-\creg{15}.

  \item \textbf{r/mc} { }---{ } A capability operand that is either
    the contents of one of the capability registers for \textbf{rc} or
    a capability in memory.
\end{itemize}

In addition, all of these instructions are either invalid or not
encodable in Compatibility/Legacy mode, so that column is omitted from
opcode tables.  However, a new column is added to describe capability
mode support using one of the following annotations:

\begin{itemize}
  \item \textbf{V} { }---{ } Supported.
  \item \textbf{I} { }---{ } Not supported.
\end{itemize}

\clearpage
\section{Extensions to x86-64 Instructions}

This section contains extensions to existing instructions to support
capability operands.  For each of these instructions, the instruction
description should be treated as an extension to the description of
the existing instruction in Volume 2 of Intel's Software Developer's
Manual.  Many of the instruction descriptions in this section reuse
language from Intel's manual to highlight the similarity in semantics
between the base instructions and their CHERI extensions.

\clearpage
\section{CHERI-x86-64 Instructions}

This section contains new instructions added to support operations on
capabilities.  The opcode assignments in this section are tentative
and subject to change.  Single byte opcodes have been used for
instructions which we believe may either be used frequently or in
frequently-accessed code paths.

\clearpage
\section{Summary of New Opcodes}

The following new opcodes are added in 64-bit mode and are also
available in capability mode.

\bigskip
\noindent
\begin{tabular}{| l | l |} \hline
  \textbf{Opcode} & \textbf{Instruction}\\
  \hline
\end{tabular}
