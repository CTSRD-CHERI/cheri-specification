\chapter{The CHERI-x86-64 Instruction-Set Reference}
\label{chap:isaref-x86-64}

\newcolumntype{Y}{>{\centering\arraybackslash}X}
\newcolumntype{Z}{>{\raggedright\arraybackslash}X}

\newenvironment{x86opcodetable}{%
  \tabularx{\textwidth}{| l | l | p{2.0em} | p{2.5em} | p{2.5em} | Z |} \hline
    \textbf{Opcode} & \textbf{Instruction} & \textbf{Op/ En} &
    \textbf{Cap Mode} & \textbf{64-bit Mode} & \textbf{Description}\\
    \hline
}{%
  \endtabularx
}

\newcommand{\xopcode}[6]{%
  #1 & #2 & #3 & #4 & #5 & #6\\
  \hline
}

\newenvironment{x86opentable}{%
  \bigskip
  \noindent
  \tabularx{\textwidth}{| c | Y | Y | Y | Y |}
    \multicolumn{5}{c}{\bfseries Instruction Operand Encoding}\\
    \hline
    Op/En & Operand 1 & Operand 2 & Operand 3 & Operand 4\\
    \hline
}{%
  \endtabularx
}

\newcommand{\xopen}[5]{%
  #1 & #2 & #3 & #4 & #5\\
  \hline
}

In this chapter, we specify new CHERI instructions as well as
extensions to existing instructions to support capability-sized
operands.  Instructions are described using similar syntax to Volume 2
of Intel's Software Developer's Manual~\cite{intel-sdm-vol2} with a
few extensions.

An additional symbol is defined to represent object code in the
``Opcode'' column:

\begin{itemize}
  \item \textbf{CAP} { }---{ } Indicates the use of the capability
    operand prefix.
\end{itemize}

Additional symbols are defined to represent operands in the
``Instruction'' column:

\begin{itemize}
  \item \textbf{rc} { }---{ } One of the general-purpose capability
    registers: \CAX{}, \CBX{}, \CCX{}, \CDX{}, \CDI{}, \CSI{}, \CBP{},
    \CSP{}, \creg{8}-\creg{15}.

  \item \textbf{r/mc} { }---{ } A capability operand that is either
    the contents of one of the capability registers for \textbf{rc} or
    a capability in memory.
\end{itemize}

In addition, all of these instructions are either invalid or not
encodable in Compatibility/Legacy mode, so that column is omitted from
opcode tables.  However, a new column is added to describe capability
mode support using one of the following annotations:

\begin{itemize}
  \item \textbf{V} { }---{ } Supported.
  \item \textbf{I} { }---{ } Not supported.
\end{itemize}

\clearpage
\section{Extensions to x86-64 Instructions}

This section contains extensions to existing instructions to support
capability operands.  For each of these instructions, the instruction
description should be treated as an extension to the description of
the existing instruction in Volume 2 of Intel's Software Developer's
Manual.  Many of the instruction descriptions in this section reuse
language from Intel's manual to highlight the similarity in semantics
between the base instructions and their CHERI extensions.

\clearpage
\phantomsection
\addcontentsline{toc}{subsection}{ADD - Add}
\insnxeslabel{add}
\subsection*{ADD - Add}

\begin{x86opcodetable}
  \xopcode{CAP + 05 \emph{id}}{ADD CAX\emph{, imm32}}
  {I}{Valid}{Valid}
  {Add \emph{imm32 sign-extended to 64-bits} to the address field of
    CAX.}
  \xopcode{CAP + 81 /0 \emph{id}}{ADD \emph{r/mc, imm32}}
  {MI}{Valid}{Valid}
  {Add \emph{imm32 sign-extended to 64-bits} to the address field of
    \emph{r/mc}.}
  \xopcode{CAP + 83 /0 \emph{ib}}{ADD \emph{r/mc, imm8}}
  {MI}{Valid}{Valid}
  {Add \emph{sign}-extended \emph{imm8} to the address field of
    \emph{r/mc}.}
  \xopcode{CAP + 01 \emph{/r}}{ADD \emph{r/mc, r64}}
  {MR}{Valid}{Valid}
  {Add \emph{r64} to the address field of \emph{r/mc}.}
  \xopcode{CAP + 03 \emph{/r}}{ADD \emph{rc, r/m64}}
  {RM}{Valid}{Valid}
  {Add \emph{r/m64} to the address field of \emph{rc}.}
\end{x86opcodetable}

\begin{x86opentable}
  \xopen{RM}{ModRM:Reg (r,w)}{ModRM:r/m (r)}{NA}{NA}
  \xopen{MR}{ModRM:r/m (r,w)}{ModRM:reg (r)}{NA}{NA}
  \xopen{MI}{ModRM:r/m (r,w)}{imm8/32}{NA}{NA}
  \xopen{I}{CAX}{imm32}{NA}{NA}
\end{x86opentable}

\subsubsection*{Description}

Adds the source operand to the \textbf{address} field of the
destination operand and then stores the result in the destination
operand. The destination operand can be a register or a memory
location. The source operand can be an immediate, a register, or a
memory location.

If the new value of the \textbf{address} field makes the resulting
capability unrepresentable, the \textbf{tag} field in the resulting
capability is cleared.

\subsubsection*{Flags Affected}

The OF, SF, ZF, AF, CF, and PF flags are set according to the value of
the resulting \textbf{address} field.

\clearpage
\phantomsection
\addcontentsline{toc}{subsection}{AND - Logical AND}
\insnxeslabel{and}
\subsection*{AND - Logical AND}

\begin{x86opcodetable}
  \xopcode{CAP + 25 \emph{id}}{AND CAX\emph{, imm32}}
  {I}{Valid}{Valid}
  {Bitwise AND of \emph{imm32 sign-extended to 64-bits} with the
    address field of CAX.}
  \xopcode{CAP + 81 /4 \emph{id}}{AND \emph{r/mc, imm32}}
  {MI}{Valid}{Valid}
  {Bitwise AND of \emph{imm32 sign-extended to 64-bits} with the
    address field of \emph{r/mc}.}
  \xopcode{CAP + 83 /4 \emph{ib}}{AND \emph{r/mc, imm8}}
  {MI}{Valid}{Valid}
  {Bitwise AND of \emph{sign}-extended \emph{imm8} with the address
    field of \emph{r/mc}.}
  \xopcode{CAP + 21 \emph{/r}}{AND \emph{r/mc, r64}}
  {MR}{Valid}{Valid}
  {Bitwise AND of \emph{r64} with the address field of \emph{r/mc}.}
  \xopcode{CAP + 23 \emph{/r}}{AND \emph{rc, r/m64}}
  {RM}{Valid}{Valid}
  {Bitwise AND of \emph{r/m64} with the address field of \emph{rc}.}
\end{x86opcodetable}

\begin{x86opentable}
  \xopen{RM}{ModRM:Reg (r,w)}{ModRM:r/m (r)}{NA}{NA}
  \xopen{MR}{ModRM:r/m (r,w)}{ModRM:reg (r)}{NA}{NA}
  \xopen{MI}{ModRM:r/m (r,w)}{imm8/32}{NA}{NA}
  \xopen{I}{CAX}{imm32}{NA}{NA}
\end{x86opentable}

\subsubsection*{Description}

Derives a new capability from the destination operand whose
\textbf{address} field is set to the bitwise AND of the source operand
and the \textbf{address} field of the destination operand and then
stores the result in the destination operand. The destination operand
can be a register or a memory location. The source operand can be an
immediate, a register, or a memory location.

If the new value of the \textbf{address} field makes the resulting
capability unrepresentable, the \textbf{tag} field in the resulting
capability is cleared.

\subsubsection*{Flags Affected}

The OF, SF, ZF, AF, CF, and PF flags are set according to the value of
the resulting \textbf{address} field.

\clearpage
\phantomsection
\addcontentsline{toc}{subsection}{CMOVcc - Conditional Move}
\insnxeslabel{cmov}
\subsection*{CMOV\emph{cc} - Conditional Move}

\begin{x86opcodetable}
  \xopcode{CAP + 0F 40 \emph{/r}}{CMOVO \emph{rc, r/mc}}
  {RM}{Valid}{Valid}
  {Move if overflow (OF=1).}
  \xopcode{CAP + 0F 41 \emph{/r}}{CMOVNO \emph{rc, r/mc}}
  {RM}{Valid}{Valid}
  {Move if not overflow (OF=0).}
  \xopcode{CAP + 0F 42 \emph{/r}}{CMOVC \emph{rc, r/mc}}
  {RM}{Valid}{Valid}
  {Move if carry (CF=1).}
  \xopcode{CAP + 0F 43 \emph{/r}}{CMOVNC \emph{rc, r/mc}}
  {RM}{Valid}{Valid}
  {Move if not carry (CF=0).}
  \xopcode{CAP + 0F 44 \emph{/r}}{CMOVZ \emph{rc, r/mc}}
  {RM}{Valid}{Valid}
  {Move if zero (ZF=1).}
  \xopcode{CAP + 0F 45 \emph{/r}}{CMOVNZ \emph{rc, r/mc}}
  {RM}{Valid}{Valid}
  {Move if not zero (ZF=0).}
  \xopcode{CAP + 0F 46 \emph{/r}}{CMOVBE \emph{rc, r/mc}}
  {RM}{Valid}{Valid}
  {Move if below or equal (CF=1 or ZF=1).}
  \xopcode{CAP + 0F 47 \emph{/r}}{CMOVA \emph{rc, r/mc}}
  {RM}{Valid}{Valid}
  {Move if above (CF=0 and ZF=0).}
  \xopcode{CAP + 0F 48 \emph{/r}}{CMOVS \emph{rc, r/mc}}
  {RM}{Valid}{Valid}
  {Move if sign (SF=1).}
  \xopcode{CAP + 0F 49 \emph{/r}}{CMOVNS \emph{rc, r/mc}}
  {RM}{Valid}{Valid}
  {Move if not sign (SF=0).}
  \xopcode{CAP + 0F 4A \emph{/r}}{CMOVP \emph{rc, r/mc}}
  {RM}{Valid}{Valid}
  {Move if parity (PF=1).}
  \xopcode{CAP + 0F 4B \emph{/r}}{CMOVNP \emph{rc, r/mc}}
  {RM}{Valid}{Valid}
  {Move if no parity (PF=0).}
  \xopcode{CAP + 0F 4C \emph{/r}}{CMOVL \emph{rc, r/mc}}
  {RM}{Valid}{Valid}
  {Move if less (SF$\neq{}$OF).}
  \xopcode{CAP + 0F 4D \emph{/r}}{CMOVGE \emph{rc, r/mc}}
  {RM}{Valid}{Valid}
  {Move if greater or equal (SF=OF).}
  \xopcode{CAP + 0F 4E \emph{/r}}{CMOVLE \emph{rc, r/mc}}
  {RM}{Valid}{Valid}
  {Move if less or equal (ZF=1 or SF$\neq{}$OF).}
  \xopcode{CAP + 0F 4F \emph{/r}}{CMOVG \emph{rc, r/mc}}
  {RM}{Valid}{Valid}
  {Move if greater (ZF=0 and SF=OF).}
\end{x86opcodetable}

\begin{x86opentable}
  \xopen{RM}{ModRM:Reg (w)}{ModRM:r/m (r)}{NA}{NA}
\end{x86opentable}

\subsubsection*{Description}

Copies the source operand to the destination operand if one or more
status flags in the \RFLAGS{} register are in the required state. The
destination operand is a register.  The source operand can be a
register or memory location.

\subsubsection*{Flags Affected}

None

\clearpage
\phantomsection
\addcontentsline{toc}{subsection}{LODS/LODSC -- Load String}
\insnxeslabel{lods}
\subsection*{LODS/LODSC -- Load String}

\begin{x86opcodetable}
  \xopcode{CAP + AD}{LODS \emph{mc}}
  {ZO}{Valid}{Valid}
  {Load capability at address (C|R)SI into CAX.}
  \xopcode{CAP + AD}{LODSC}
  {ZO}{Valid}{Valid}
  {Load capability at address (C|R)SI into CAX.}
\end{x86opcodetable}

\begin{x86opentable}
  \xopen{ZO}{NA}{NA}{NA}{NA}
\end{x86opentable}

\subsubsection*{Description}

Loads a capablity from the source operand into the \CAX{} register.
The source operand is a memory location identified by the \RSI{} or
\CSI{} register (depending on the addressing mode).  After the
capability is loaded from memory, the address register is incremented
or decremented by the size of a capability according to the setting of
\texttt{DF} in \RFLAGS{}.

\subsubsection*{Flags Affected}

None

\clearpage
\phantomsection
\addcontentsline{toc}{subsection}{MOV -- Move}
\insnxeslabel{mov}
\subsection*{MOV -- Move}

\begin{x86opcodetable}
  \xopcode{CAP + 89 \emph{/r}}{MOV \emph{r/mc, rc}}
  {MR}{Valid}{Valid}
  {Move \emph{rc} to \emph{r/mc}.}
  \xopcode{CAP + 8B \emph{/r}}{MOV \emph{rc, r/mc}}
  {RM}{Valid}{Valid}
  {Move \emph{r/mc} to \emph{rc}.}
  \xopcode{CAP + C7 \emph{/0 id}}{MOV \emph{r/mc, imm32}}
  {MI}{Valid}{Valid}
  {Move \emph{imm32 sign-extended to 64-bits} to \emph{r/mc}.}
\end{x86opcodetable}

\begin{x86opentable}
  \xopen{MR}{ModRM:r/m (w)}{ModRM:reg (r)}{NA}{NA}
  \xopen{RM}{ModRM:Reg (w)}{ModRM:r/m (r)}{NA}{NA}
  \xopen{MI}{ModRM:r/m (w)}{imm32}{NA}{NA}
\end{x86opentable}

\subsubsection*{Description}

Copies the source operand to the destination operand. The destination
operand can be a register or a memory location. The source operand can
be an immediate, a register, or a memory location.  If the source
operand is an immediate, the value is sign-extended to 64-bits and
used as the address of a NULL-derived capability.

Note that some \insnnoref{MOV} opcodes such as \texttt{B8+ rw} and
\texttt{C6 /0} are not extended to support the \insnnoref{CAP} prefix
as the behavior would be identical.  The \texttt{C7 /0} opcode is
extended primarily to support storing constants such as NULL to
capabilities in memory without requiring an intermediate register.

The \texttt{A1} and \texttt{A3} opcodes are not extended to support
capabilities.

\subsubsection*{Flags Affected}

None

\clearpage
\phantomsection
\addcontentsline{toc}{subsection}{MOVNTI - Store Using Non-Temporal Hint}
\insnxeslabel{movnti}
\subsection*{MOVNTI - Store Using Non-Temporal Hint}

\begin{x86opcodetable}
  \xopcode{NP CAP + 0F C3 \emph{/r}}{MOV \emph{mc, rc}}
  {MR}{Valid}{Valid}
  {Move \emph{rc} to \emph{mc} using non-temporal hint.}
\end{x86opcodetable}

\begin{x86opentable}
  \xopen{MR}{ModRM:r/m (w)}{ModRM:reg (r)}{NA}{NA}
\end{x86opentable}

\subsubsection*{Description}

Moves the capability in the source operand to the destination operand
using a non-temporal hint.

\subsubsection*{Flags Affected}

None

\clearpage
\phantomsection
\addcontentsline{toc}{subsection}{MOVS/MOVSC -- Move Data from String
  to String}
\insnxeslabel{movs}
\subsection*{MOVS/MOVSC -- Move Data from String to String}

\begin{x86opcodetable}
  \xopcode{CAP + A5}{MOVS \emph{mc, mc}}
  {ZO}{Valid}{Valid}
  {Move capability from address (C|R)SI to (C|R)DI.}
  \xopcode{CAP + A5}{MOVSC}
  {ZO}{Valid}{Valid}
  {Move capability from address (C|R)SI to (C|R)DI.}
\end{x86opcodetable}

\begin{x86opentable}
  \xopen{ZO}{NA}{NA}{NA}{NA}
\end{x86opentable}

\subsubsection*{Description}

Moves a capablity from the source operand to the destination operand.
The source operand is a memory location identified by the \RSI{} or
\CSI{} register (depending on the addressing mode).  The destination
operand is a memory location identified by the \RDI{} or \CDI{}
register (depending on the addressing mode).  After the capability is
copied, the address registers are incremented or decremented by the
size of a capability according to the setting of \texttt{DF} in
\RFLAGS{}.

\subsubsection*{Flags Affected}

None

\clearpage
\phantomsection
\addcontentsline{toc}{subsection}{OR -- Logical Inclusive OR}
\insnxeslabel{or}
\subsection*{OR -- Logical Inclusive OR}

\begin{x86opcodetable}
  \xopcode{CAP + 0D \emph{id}}{OR CAX\emph{, imm32}}
  {I}{Valid}{Valid}
  {Bitwise inclusive OR of \emph{imm32 sign-extended to 64-bits} with
    the address field of CAX.}
  \xopcode{CAP + 81 /1 \emph{id}}{OR \emph{r/mc, imm32}}
  {MI}{Valid}{Valid}
  {Bitwise inclusive OR of \emph{imm32 sign-extended to 64-bits} with
    the address field of \emph{r/mc}.}
  \xopcode{CAP + 83 /1 \emph{ib}}{OR \emph{r/mc, imm8}}
  {MI}{Valid}{Valid}
  {Bitwise inclusive OR of \emph{sign}-extended \emph{imm8} with the
    address field of \emph{r/mc}.}
  \xopcode{CAP + 09 \emph{/r}}{OR \emph{r/mc, r64}}
  {MR}{Valid}{Valid}
  {Bitwise inclusive OR of \emph{r64} with the address field of
    \emph{r/mc}.}
  \xopcode{CAP + 0B \emph{/r}}{OR \emph{rc, r/m64}}
  {RM}{Valid}{Valid}
  {Bitwise inclusive OR of \emph{r/m64} with the address field of
    \emph{rc}.}
\end{x86opcodetable}

\begin{x86opentable}
  \xopen{RM}{ModRM:Reg (r,w)}{ModRM:r/m (r)}{NA}{NA}
  \xopen{MR}{ModRM:r/m (r,w)}{ModRM:reg (r)}{NA}{NA}
  \xopen{MI}{ModRM:r/m (r,w)}{imm8/32}{NA}{NA}
  \xopen{I}{CAX}{imm32}{NA}{NA}
\end{x86opentable}

\subsubsection*{Description}

Derives a new capability from the destination operand whose
\textbf{address} field is set to the bitwise inclusive OR of the
source operand and the \textbf{address} field of the destination
operand and then stores the result in the destination operand. The
destination operand can be a register or a memory location. The source
operand can be an immediate, a register, or a memory location.

If the new value of the \textbf{address} field makes the resulting
capability unrepresentable, the \textbf{tag} field in the resulting
capability is cleared.

\subsubsection*{Flags Affected}

The OF, SF, ZF, AF, CF, and PF flags are set according to the value of
the resulting \textbf{address} field.

\clearpage
\phantomsection
\addcontentsline{toc}{subsection}{STOS/STOSC - Store String}
\insnxeslabel{stos}
\subsection*{STOS/STOSC - Store String}

\begin{x86opcodetable}
  \xopcode{CAP + AB}{STOS \emph{mc}}
  {ZO}{Valid}{Valid}
  {Store CAX at address (C|R)DI.}
  \xopcode{CAP + AB}{STOSC}
  {ZO}{Valid}{Valid}
  {Store CAX at address (C|R)DI.}
\end{x86opcodetable}

\begin{x86opentable}
  \xopen{ZO}{NA}{NA}{NA}{NA}
\end{x86opentable}

\subsubsection*{Description}

Stores capability in the \CAX{} register into the destination operand.
The destination operand is a memory location identified by the \RDI{}
or \CDI{} register (depending on the addressing mode).  After the
capability is stored to memory, the address register is incremented or
decremented by the size of a capability according to the setting of
\texttt{DF} in \RFLAGS{}.

\subsubsection*{Flags Affected}

None

\clearpage
\phantomsection
\addcontentsline{toc}{subsection}{SUB -- Subtract}
\insnxeslabel{sub}
\subsection*{SUB -- Subtract}

\begin{x86opcodetable}
  \xopcode{CAP + 2D \emph{id}}{SUB CAX\emph{, imm32}}
  {I}{Valid}{Valid}
  {Subtract \emph{imm32 sign-extended to 64-bits} from the address field of
    CAX.}
  \xopcode{CAP + 81 /5 \emph{id}}{SUB \emph{r/mc, imm32}}
  {MI}{Valid}{Valid}
  {Subract \emph{imm32 sign-extended to 64-bits} from the address field of
    \emph{r/mc}.}
  \xopcode{CAP + 83 /5 \emph{ib}}{SUB \emph{r/mc, imm8}}
  {MI}{Valid}{Valid}
  {Subtract \emph{sign}-extended \emph{imm8} from the address field of
    \emph{r/mc}.}
  \xopcode{CAP + 29 \emph{/r}}{SUB \emph{r/mc, r64}}
  {MR}{Valid}{Valid}
  {Subtract \emph{r64} from the address field of \emph{r/mc}.}
  \xopcode{CAP + 2B \emph{/r}}{SUB \emph{rc, r/m64}}
  {RM}{Valid}{Valid}
  {Subtract \emph{r/m64} from the address field of \emph{rc}.}
\end{x86opcodetable}

\begin{x86opentable}
  \xopen{RM}{ModRM:Reg (r,w)}{ModRM:r/m (r)}{NA}{NA}
  \xopen{MR}{ModRM:r/m (r,w)}{ModRM:reg (r)}{NA}{NA}
  \xopen{MI}{ModRM:r/m (r,w)}{imm8/32}{NA}{NA}
  \xopen{I}{CAX}{imm32}{NA}{NA}
\end{x86opentable}

\subsubsection*{Description}

Subtracts the source operand from the \textbf{address} field of the
destination operand and then stores the result in the destination
operand. The destination operand can be a register or a memory
location. The source operand can be an immediate, a register, or a
memory location.

If the new value of the \textbf{address} field makes the resulting
capability unrepresentable, the \textbf{tag} field in the resulting
capability is cleared.

\subsubsection*{Flags Affected}

The OF, SF, ZF, AF, CF, and PF flags are set according to the value of
the resulting \textbf{address} field.

\clearpage
\phantomsection
\addcontentsline{toc}{subsection}{XOR -- Logical Exclusive OR}
\insnxeslabel{xor}
\subsection*{XOR -- Logical Exclusive OR}

\begin{x86opcodetable}
  \xopcode{CAP + 35 \emph{id}}{XOR CAX\emph{, imm32}}
  {I}{Valid}{Valid}
  {Bitwise exclusive OR of \emph{imm32 sign-extended to 64-bits} with
    the address field of CAX.}
  \xopcode{CAP + 81 /6 \emph{id}}{XOR \emph{r/mc, imm32}}
  {MI}{Valid}{Valid}
  {Bitwise exclusive OR of \emph{imm32 sign-extended to 64-bits} with
    the address field of \emph{r/mc}.}
  \xopcode{CAP + 83 /6 \emph{ib}}{XOR \emph{r/mc, imm8}}
  {MI}{Valid}{Valid}
  {Bitwise exclusive OR of \emph{sign}-extended \emph{imm8} with the
    address field of \emph{r/mc}.}
  \xopcode{CAP + 31 \emph{/r}}{XOR \emph{r/mc, r64}}
  {MR}{Valid}{Valid}
  {Bitwise exclusive OR of \emph{r64} with the address field of \emph{r/mc}.}
  \xopcode{CAP + 33 \emph{/r}}{XOR \emph{rc, r/m64}}
  {RM}{Valid}{Valid}
  {Bitwise exclusive OR of \emph{r/m64} with the address field of \emph{rc}.}
\end{x86opcodetable}

\begin{x86opentable}
  \xopen{RM}{ModRM:Reg (r,w)}{ModRM:r/m (r)}{NA}{NA}
  \xopen{MR}{ModRM:r/m (r,w)}{ModRM:reg (r)}{NA}{NA}
  \xopen{MI}{ModRM:r/m (r,w)}{imm8/32}{NA}{NA}
  \xopen{I}{CAX}{imm32}{NA}{NA}
\end{x86opentable}

\subsubsection*{Description}

Derives a new capability from the destination operand whose
\textbf{address} field is set to the bitwise exclusive OR of the
source operand and the \textbf{address} field of the destination
operand and then stores the result in the destination operand. The
destination operand can be a register or a memory location. The source
operand can be an immediate, a register, or a memory location.

If the new value of the \textbf{address} field makes the resulting
capability unrepresentable, the \textbf{tag} field in the resulting
capability is cleared.

\subsubsection*{Flags Affected}

The OF, SF, ZF, AF, CF, and PF flags are set according to the value of
the resulting \textbf{address} field.


\clearpage
\section{CHERI-x86-64 Instructions}

This section contains new instructions added to support operations on
capabilities.  The opcode assignments in this section are tentative
and subject to change.  Single byte opcodes have been used for
instructions which we believe may either be used frequently or in
frequently-accessed code paths.

\clearpage
\phantomsection
\addcontentsline{toc}{subsection}{CINVOKE -- Invoke Sealed Capability Pair}
\insnxeslabel{cinvoke}
\subsection*{CINVOKE -- Invoke Sealed Capability Pair}

\begin{x86opcodetable}
  \xopcode{EA \emph{/r}}{CINVOKE \emph{rc, r/mc}}
  {RM}{Valid}{Valid}
  {Set CAX to \emph{r/mc} and jump to \emph{rc}.}
\end{x86opcodetable}

\begin{x86opentable}
  \xopen{RM}{ModRM:reg (r)}{ModRM:r/m (r)}{NA}{NA}
\end{x86opentable}

\subsubsection*{Description}

Jumps to a pair of sealed capabilities.  The first source operand can
be a register; the second source operand can be a register or memory
location.

If both operands are sealed with the same \textbf{otype}, sets \CIP{}
to the unsealed first operand and sets \CAX{} to the unsealed second
operand.  Note that this control transfer is a jump and does not push
any values onto the stack.  If this instruction fails, it raises a
Capability Violation Fault (see
Section~\ref{sec:x86:capability-fault}).

\subsubsection*{Flags Affected}

None

\clearpage
\phantomsection
\addcontentsline{toc}{subsection}{CRAM - Representable Alignment Mask}
\insnxeslabel{cram}
\subsection*{CRAM - Representable Alignment Mask}

\begin{x86opcodetable}
  \xopcode{1F \emph{/r}}{CRAM \emph{r64, r/m64}}
  {RM}{Valid}{Valid}
  {Set \emph{r64} to mask sufficient for aligned bounds spanning the
    length in \emph{r/m64}.}
\end{x86opcodetable}

\begin{x86opentable}
  \xopen{RM}{ModRM:reg (w)}{ModRM:r/m (r)}{NA}{NA}
\end{x86opentable}

\subsubsection*{Description}

Sets the destination operand to a mask that can be used to round
addresses down to a value that is sufficiently aligned to set exact
bounds for the nearest representable length of the source operand.
The source operand can be a register or memory location.

\subsubsection*{Flags Affected}

None

\clearpage
\phantomsection
\addcontentsline{toc}{subsection}{CRRL -- Round Representable Length}
\insnxeslabel{crrl}
\subsection*{CRRL -- Round Representable Length}

\begin{x86opcodetable}
  \xopcode{1E \emph{/r}}{CRRL \emph{r64, r/m64}}
  {RM}{Valid}{Valid}
  {Set \emph{r64} to minimum representable length greater or equal to
    \emph{r/m64}.}
\end{x86opcodetable}

\begin{x86opentable}
  \xopen{RM}{ModRM:reg (w)}{ModRM:r/m (r)}{NA}{NA}
\end{x86opentable}

\subsubsection*{Description}

Sets the destination operand to the smallest value greater or equal to
the source operand that can be used as a length to set exact bounds on
a capability with a suitably aligned base.  The source operand can be
a register or memory location.

\subsubsection*{Flags Affected}

None

\clearpage
\phantomsection
\addcontentsline{toc}{subsection}{GCBASE -- Get Capability Base}
\insnxeslabel{gcbase}
\subsection*{GCBASE -- Get Capability Base}

\begin{x86opcodetable}
  \xopcode{F2 0F 7A \emph{/r}}{GCBASE \emph{r64, r/mc}}
  {RM}{Valid}{Valid}
  {Store the base field of \emph{r/mc} in \emph{r64}.}
\end{x86opcodetable}

\begin{x86opentable}
  \xopen{RM}{ModRM:reg (w)}{ModRM:r/m (r)}{NA}{NA}
\end{x86opentable}

\subsubsection*{Description}

Sets the destination operand to the \textbf{base} field of the source
operand.  The source operand can be a register or memory location.

\subsubsection*{Flags Affected}

None

\clearpage
\phantomsection
\addcontentsline{toc}{subsection}{GCLEN - Get Capability Length}
\insnxeslabel{gclen}
\subsection*{GCLEN - Get Capability Length}

\begin{x86opcodetable}
  \xopcode{F3 0F 7A \emph{/r}}{GCLEN \emph{r64, r/mc}}
  {RM}{Valid}{Valid}
  {Store the bounds length of \emph{r/mc} in \emph{r64}.}
\end{x86opcodetable}

\begin{x86opentable}
  \xopen{RM}{ModRM:reg (w)}{ModRM:r/m (r)}{NA}{NA}
\end{x86opentable}

\subsubsection*{Description}

Sets the destination operand to the \textbf{length} field of the
source operand.  The source operand can be a register or memory
location.

\subsubsection*{Flags Affected}

ZF is set to 1 if the result is zero.  The CF, PF, AF, SF, and OF
flags are undefined.

\clearpage
\phantomsection
\addcontentsline{toc}{subsection}{GCOFF - Get Capability Offset}
\insnxeslabel{gcoff}
\subsection*{GCOFF - Get Capability Offset}

\begin{x86opcodetable}
  \xopcode{66 0F 7B \emph{/r}}{GCOFF \emph{r64, r/mc}}
  {RM}{Valid}{Valid}
  {Store the offset of \emph{r/mc} in \emph{r64}.}
\end{x86opcodetable}

\begin{x86opentable}
  \xopen{RM}{ModRM:reg (w)}{ModRM:r/m (r)}{NA}{NA}
\end{x86opentable}

\subsubsection*{Description}

Sets the destination operand to the \textbf{offset} field of the
source operand.  The source operand can be a register or memory
location.

\subsubsection*{Flags Affected}

ZF is set to 1 if the result is zero.  The CF, PF, AF, SF, and OF
flags are undefined.

\clearpage
\phantomsection
\addcontentsline{toc}{subsection}{GCPERM - Get Capability Permissions}
\insnxeslabel{gcperm}
\subsection*{GCPERM - Get Capability Permissions}

\begin{x86opcodetable}
  \xopcode{NP 0F 7A \emph{/r}}{GCPERM \emph{r64, r/mc}}
  {RM}{Valid}{Valid}
  {Store the permissions mask of \emph{r/mc} in \emph{r64}.}
\end{x86opcodetable}

\begin{x86opentable}
  \xopen{RM}{ModRM:reg (w)}{ModRM:r/m (r)}{NA}{NA}
\end{x86opentable}

\subsubsection*{Description}

Sets the destination operand to the combined \textbf{perms} and
\textbf{uperms} fields of the source operand.  The source operand can
be a register or memory location.

\subsubsection*{Flags Affected}

ZF is set to 1 if the result is zero.  The CF, PF, AF, SF, and OF
flags are undefined.

\clearpage
\phantomsection
\addcontentsline{toc}{subsection}{GCTAG -- Get Capability Tag}
\insnxeslabel{gctag}
\subsection*{GCTAG -- Get Capability Tag}

\begin{x86opcodetable}
  \xopcode{0E /3}{GCTAG \emph{r/mc}}
  {M}{Valid}{Valid}
  {Store the tag of \emph{r/mc} in ZF.}
\end{x86opcodetable}

\begin{x86opentable}
  \xopen{M}{ModRM:r/m (r)}{NA}{NA}{NA}
\end{x86opentable}

\subsubsection*{Description}

Sets ZF in \RFLAGS{} to the \textbf{tag} field of the source
operand.  The source operand can be a register or memory location.

\subsubsection*{Flags Affected}

ZF is set to 1 if the result is zero.  The CF, PF, AF, SF, and OF
flags are undefined.

\clearpage
\phantomsection
\addcontentsline{toc}{subsection}{GCTYPE - Get Capability Object Type}
\insnxeslabel{gctype}
\subsection*{GCTYPE - Get Capability Object Type}

\begin{x86opcodetable}
  \xopcode{66 0F 7A \emph{/r}}{GCTYPE \emph{r64, r/mc}}
  {RM}{Valid}{Valid}
  {Store the object type of \emph{r/mc} in \emph{r64}.}
\end{x86opcodetable}

\begin{x86opentable}
  \xopen{RM}{ModRM:reg (w)}{ModRM:r/m (r)}{NA}{NA}
\end{x86opentable}

\subsubsection*{Description}

Sets the destination operand to the \textbf{otype} field of the source
operand.  The source operand can be a register or memory location.

\subsubsection*{Flags Affected}

ZF is set to 1 if the result is zero.  SF is set to 1 if the result is
less than zero. The CF, PF, AF, and OF flags are undefined.

\clearpage
\phantomsection
\addcontentsline{toc}{subsection}{MOV - Move to/from Additional Capability Registers}
\insnxeslabel{movcap}
\subsection*{MOV - Move to/from Additional Capability Registers}

\begin{x86opcodetable}
  \xopcode{0F 24 \emph{/r}}{MOV \emph{rc,} CFS/CGS/DDC}
  {MR}{Valid}{Valid}
  {Move additional capability register to \emph{rc}.}
  \xopcode{0F 25 \emph{/r}}{MOV CFS/CGS/DDC\emph{, rc}}
  {RM}{Valid}{Valid}
  {Move \emph{rc} to additional capability register.}
\end{x86opcodetable}

\begin{x86opentable}
  \xopen{MR}{ModRM:r/m (w)}{ModRM:reg (r)}{NA}{NA}
  \xopen{RM}{ModRM:reg (w)}{ModRM:r/m (r)}{NA}{NA}
\end{x86opentable}

\subsubsection*{Description}

Moves the contents of an additional capability register to a
general-purpose capability register or vice versa.

Similar to the \insnnoref{MOV} opcodes for control and debug
registers, the \textbf{reg} field of the ModRM byte always identifies
the additional capability register to read or write.  The \textbf{mod}
field of ModRM is ignored, and the \textbf{r/m} field identifies the
general-purpose capability register.  Attempts to reference invalid
additional capability registers will raise a UD\# exception.

Attempts to access additional capability registers other than \CFS{},
\CGS{}, or \DDC{} from a privilege level other than 0 will raise a
GP\#(0) exception.

\subsubsection*{Flags Affected}

None

\clearpage
\phantomsection
\addcontentsline{toc}{subsection}{SCADDR - Set Capability Address}
\insnxeslabel{scaddr}
\subsection*{SCADDR - Set Capability Address}

\begin{x86opcodetable}
  \xopcode{16 \emph{/r}}{SCADDR \emph{r/mc, r64}}
  {MR}{Valid}{Valid}
  {Set the address field of \emph{r/mc} to \emph{r64}.}
\end{x86opcodetable}

\begin{x86opentable}
  \xopen{MR}{ModRM:r/m (r, w)}{ModRM:reg (r)}{NA}{NA}
\end{x86opentable}

\subsubsection*{Description}

Sets the \textbf{address} field of the destination operand to the
source operand and stores the result in the destination operand.  The
destination operand can be a register or memory location; the source
operand can be a register.  If the destination operand is sealed and
tagged, the destination operand is set to its original value with the
\textbf{tag} field cleared.

If the new value of the \textbf{address} field makes the resulting
capability unrepresentable, the \textbf{tag} field in the resulting
capability is cleared.

\subsubsection*{Flags Affected}

None

\clearpage
\phantomsection
\addcontentsline{toc}{subsection}{SCBND -- Set Capability Bounds}
\insnxeslabel{scbnd}
\subsection*{SCBND -- Set Capability Bounds}

\begin{x86opcodetable}
  \xopcode{17 \emph{/r}}{SCBND \emph{r/mc, r64}}
  {MR}{Valid}{Valid}
  {Set bounds of \emph{r/mc} to \emph{r64}.}
  \xopcode{37 /0 \emph{id}}{SCBND \emph{r/mc, imm32}}
  {MI}{Valid}{Valid}
  {Set bounds of \emph{r/mc} to zero-extended \emph{imm32}.}
\end{x86opcodetable}

\begin{x86opentable}
  \xopen{MR}{ModRM:r/m (r, w)}{ModRM:reg (r)}{NA}{NA}
  \xopen{MI}{ModRM:r/m (r, w)}{imm32}{NA}{NA}
\end{x86opentable}

\subsubsection*{Description}

Derives a new capability from the destination operand and source
operand and then store the result in the destination operand.  The
destination operand can be a register or memory location; the source
operand can be an immediate or a register.  When an immediate value is
used as an operand, it is zero-extended.

The new capability's \textbf{base} field is set to the current
\textbf{address} field of the destination operand, and the
\textbf{length} field is set to the source operand.  If the resulting
capability cannot be represented exactly, the \textbf{base} will be
rounded down and the \textbf{length} will be rounded up by the
smallest amount needed to form a representable capability covering the
requested bounds.  If the original capability is sealed or the bounds
of the new capability exceed the original capability, the \textbf{tag}
field in the resulting capability is cleared.

\subsubsection*{Flags Affected}

The ZF flag is set to the value of the \textbf{tag} field in the
resulting capability.  The CF, PF, AF, SF, and OF flags are undefined.

\clearpage
\phantomsection
\addcontentsline{toc}{subsection}{SCBNDE -- Set Exact Capability Bounds}
\insnxeslabel{scbnde}
\subsection*{SCBNDE -- Set Exact Capability Bounds}

\begin{x86opcodetable}
  \xopcode{27 \emph{/r}}{SCBNDE \emph{r/mc, r64}}
  {MR}{Valid}{Valid}
  {Set bounds of \emph{r/mc} to \emph{r64}.}
  \xopcode{37 /1 \emph{id}}{SCBNDE \emph{r/mc, imm32}}
  {MI}{Valid}{Valid}
  {Set bounds of \emph{r/mc} to zero-extended \emph{imm32}.}
\end{x86opcodetable}

\begin{x86opentable}
  \xopen{MR}{ModRM:r/m (r, w)}{ModRM:reg (r)}{NA}{NA}
  \xopen{MI}{ModRM:r/m (r, w)}{imm32}{NA}{NA}
\end{x86opentable}

\subsubsection*{Description}

Derives a new capability from the destination operand and source
operand and then store the result in the destination operand.  The
destination operand can be a register or memory location; the source
operand can be an immediate or a register.  When an immediate value is
used as an operand, it is zero-extended.

The new capability's \textbf{base} field is set to the current
\textbf{address} field of the destination operand, and the
\textbf{length} field is set to the source operand.  If the resulting
capability cannot be represented exactly, the \textbf{tag} field will
be cleared, the \textbf{base} will be rounded down, and the
\textbf{length} will be rounded up by the smallest amount needed to
form a representable capability covering the requested bounds.  If the
original capability is sealed or the bounds of the new capability
exceed the original capability, the \textbf{tag} field in the
resulting capability is cleared.

\subsubsection*{Flags Affected}

The ZF flag is set to the value of the \textbf{tag} field in the
resulting capability.  The CF, PF, AF, SF, and OF flags are undefined.


\clearpage
\section{Summary of New Opcodes}

The following new opcodes are added in 64-bit mode and are also
available in capability mode.

\bigskip
\noindent
\begin{tabular}{| l | l |} \hline
  \textbf{Opcode} & \textbf{Instruction}\\
  \hline
  16 \emph{/r} & \insnxesref{SCADDR} \emph{r/mc, r64}\\
  \hline
  17 \emph{/r} & \insnxesref{SCBND} \emph{r/mc, r64}\\
  \hline
  1E \emph{/r} & \insnxesref{CRRL} \emph{r64, r/m64}\\
  \hline
  1F \emph{/r} & \insnxesref{CRAM} \emph{r64, r/m64}\\
  \hline
  27 \emph{/r} & \insnxesref{SCBNDE} \emph{r/mc, r64}\\
  \hline
  37 /0 \emph{id} & \insnxesref{SCBND} \emph{r/mc, imm32}\\
  \hline
  37 /1 \emph{id} & \insnxesref{SCBNDE} \emph{r/mc, imm32}\\
  \hline
  EA \emph{/r} & \insnxesref{CINVOKE} \emph{rc, r/mc}\\
  \hline
  0F 24 \emph{/r} & \insnxesref[movcap]{MOV} \emph{rc,} CFS/CGS/DDC\\
  \hline
  0F 25 \emph{/r} & \insnxesref[movcap]{MOV} CFS/CGS/DDC\emph{, rc}\\
  \hline
  NP 0F 7A \emph{/r} & \insnxesref{GCPERM} \emph{r64, r/mc}\\
  \hline
  66 0F 7A \emph{/r} & \insnxesref{GCTYPE} \emph{r64, r/mc}\\
  \hline
  F2 0F 7A \emph{/r} & \insnxesref{GCBASE} \emph{r64, r/mc}\\
  \hline
  F3 0F 7A \emph{/r} & \insnxesref{GCLEN} \emph{r64, r/mc}\\
  \hline
  NP 0F 7B \emph{/r} & \insnxesref{GCTAG} \emph{r64, r/mc}\\
  \hline
  66 0F 7B \emph{/r} & \insnxesref{GCOFF} \emph{r64, r/mc}\\
  \hline
\end{tabular}
