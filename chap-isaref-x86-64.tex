\chapter{The CHERI-x86-64 Instruction-Set Reference}
\label{chap:isaref-x86-64}

\newcolumntype{Y}{>{\centering\arraybackslash}X}
\newcolumntype{Z}{>{\raggedright\arraybackslash}X}

\newenvironment{x86opcodetable}{%
  \tabularx{\textwidth}{| l | l | p{2.0em} | p{2.5em} | p{2.5em} | Z |} \hline
    \textbf{Opcode} & \textbf{Instruction} & \textbf{Op/ En} &
    \textbf{Cap Mode} & \textbf{64-bit Mode} & \textbf{Description}\\
    \hline
}{%
  \endtabularx
}

\newcommand{\xopcode}[6]{%
  #1 & #2 & #3 & #4 & #5 & #6\\
  \hline
}

\newenvironment{x86opentable}{%
  \bigskip
  \noindent
  \tabularx{\textwidth}{| c | Y | Y | Y | Y |}
    \multicolumn{5}{c}{\bfseries Instruction Operand Encoding}\\
    \hline
    Op/En & Operand 1 & Operand 2 & Operand 3 & Operand 4\\
    \hline
}{%
  \endtabularx
}

\newcommand{\xopen}[5]{%
  #1 & #2 & #3 & #4 & #5\\
  \hline
}

In this chapter, we specify new CHERI instructions as well as
extensions to existing instructions to support capability-sized
operands.  Instructions are described using similar syntax to Volume 2
of Intel's Software Developer's Manual~\cite{intel-sdm-vol2} with a
few extensions.

An additional symbol is defined to represent object code in the
``Opcode'' column:

\begin{itemize}
  \item \textbf{CAP} { }---{ } Indicates the use of the capability
    operand prefix.

  \item \textbf{+rc} { }---{ } Indicates the lower 3 bits of the
    opcode byte is used to encode the register operand without a
    modR/M byte.
\end{itemize}

Additional symbols are defined to represent operands in the
``Instruction'' column:

\begin{itemize}
  \item \textbf{rc} { }---{ } One of the general-purpose capability
    registers: \CAX{}, \CBX{}, \CCX{}, \CDX{}, \CDI{}, \CSI{}, \CBP{},
    \CSP{}, \creg{8}-\creg{15}.

  \item \textbf{r/mc} { }---{ } A capability operand that is either
    the contents of one of the capability registers for \textbf{rc} or
    a capability in memory.

  \item \textbf{m2c} { }---{ } A pair of adjacent capabilities in
    memory.
\end{itemize}

In addition, all of these instructions are either invalid or not
encodable in Compatibility/Legacy mode, so that column is omitted from
opcode tables.  However, a new column is added to describe capability
mode support using one of the following annotations:

\begin{itemize}
  \item \textbf{V} { }---{ } Supported.
  \item \textbf{I} { }---{ } Not supported.
  \item \textbf{N. E.} { }---{ } Not encodable in capability mode.
\end{itemize}

\clearpage
\section{Extensions to x86-64 Instructions}

This section contains extensions to existing instructions to support
capability operands.  For each of these instructions, the instruction
description should be treated as an extension to the description of
the existing instruction in Volume 2 of Intel's Software Developer's
Manual.  Many of the instruction descriptions in this section reuse
language from Intel's manual to highlight the similarity in semantics
between the base instructions and their CHERI extensions.

\clearpage
\phantomsection
\addcontentsline{toc}{subsection}{ADD - Add}
\insnxeslabel{add}
\subsection*{ADD - Add}

\begin{x86opcodetable}
  \xopcode{CAP + 05 \emph{id}}{ADD CAX\emph{, imm32}}
  {I}{Valid}{Valid}
  {Add \emph{imm32 sign-extended to 64-bits} to the address field of
    CAX.}
  \xopcode{CAP + 81 /0 \emph{id}}{ADD \emph{r/mc, imm32}}
  {MI}{Valid}{Valid}
  {Add \emph{imm32 sign-extended to 64-bits} to the address field of
    \emph{r/mc}.}
  \xopcode{CAP + 83 /0 \emph{ib}}{ADD \emph{r/mc, imm8}}
  {MI}{Valid}{Valid}
  {Add \emph{sign}-extended \emph{imm8} to the address field of
    \emph{r/mc}.}
  \xopcode{CAP + 01 \emph{/r}}{ADD \emph{r/mc, r64}}
  {MR}{Valid}{Valid}
  {Add \emph{r64} to the address field of \emph{r/mc}.}
  \xopcode{CAP + 03 \emph{/r}}{ADD \emph{rc, r/m64}}
  {RM}{Valid}{Valid}
  {Add \emph{r/m64} to the address field of \emph{rc}.}
\end{x86opcodetable}

\begin{x86opentable}
  \xopen{RM}{ModRM:Reg (r,w)}{ModRM:r/m (r)}{NA}{NA}
  \xopen{MR}{ModRM:r/m (r,w)}{ModRM:reg (r)}{NA}{NA}
  \xopen{MI}{ModRM:r/m (r,w)}{imm8/32}{NA}{NA}
  \xopen{I}{CAX}{imm32}{NA}{NA}
\end{x86opentable}

\subsubsection*{Description}

Adds the source operand to the \textbf{address} field of the
destination operand and then stores the result in the destination
operand. The destination operand can be a register or a memory
location. The source operand can be an immediate, a register, or a
memory location.

If the new value of the \textbf{address} field makes the resulting
capability unrepresentable, the \textbf{tag} field in the resulting
capability is cleared.

\subsubsection*{Flags Affected}

The OF, SF, ZF, AF, CF, and PF flags are set according to the value of
the resulting \textbf{address} field.

\clearpage
\phantomsection
\addcontentsline{toc}{subsection}{AND - Logical AND}
\insnxeslabel{and}
\subsection*{AND - Logical AND}

\begin{x86opcodetable}
  \xopcode{CAP + 25 \emph{id}}{AND CAX\emph{, imm32}}
  {I}{Valid}{Valid}
  {Bitwise AND of \emph{imm32 sign-extended to 64-bits} with the
    address field of CAX.}
  \xopcode{CAP + 81 /4 \emph{id}}{AND \emph{r/mc, imm32}}
  {MI}{Valid}{Valid}
  {Bitwise AND of \emph{imm32 sign-extended to 64-bits} with the
    address field of \emph{r/mc}.}
  \xopcode{CAP + 83 /4 \emph{ib}}{AND \emph{r/mc, imm8}}
  {MI}{Valid}{Valid}
  {Bitwise AND of \emph{sign}-extended \emph{imm8} with the address
    field of \emph{r/mc}.}
  \xopcode{CAP + 21 \emph{/r}}{AND \emph{r/mc, r64}}
  {MR}{Valid}{Valid}
  {Bitwise AND of \emph{r64} with the address field of \emph{r/mc}.}
  \xopcode{CAP + 23 \emph{/r}}{AND \emph{rc, r/m64}}
  {RM}{Valid}{Valid}
  {Bitwise AND of \emph{r/m64} with the address field of \emph{rc}.}
\end{x86opcodetable}

\begin{x86opentable}
  \xopen{RM}{ModRM:Reg (r,w)}{ModRM:r/m (r)}{NA}{NA}
  \xopen{MR}{ModRM:r/m (r,w)}{ModRM:reg (r)}{NA}{NA}
  \xopen{MI}{ModRM:r/m (r,w)}{imm8/32}{NA}{NA}
  \xopen{I}{CAX}{imm32}{NA}{NA}
\end{x86opentable}

\subsubsection*{Description}

Derives a new capability from the destination operand whose
\textbf{address} field is set to the bitwise AND of the source operand
and the \textbf{address} field of the destination operand and then
stores the result in the destination operand. The destination operand
can be a register or a memory location. The source operand can be an
immediate, a register, or a memory location.

If the new value of the \textbf{address} field makes the resulting
capability unrepresentable, the \textbf{tag} field in the resulting
capability is cleared.

\subsubsection*{Flags Affected}

The OF, SF, ZF, AF, CF, and PF flags are set according to the value of
the resulting \textbf{address} field.

\clearpage
\phantomsection
\addcontentsline{toc}{subsection}{CALL - Call Procedure}
\insnxeslabel{call}
\subsection*{CALL - Call Procedure}

\begin{x86opcodetable}
  \xopcode{E8 \emph{cw}}{CALL \emph{rel16}}
  {D}{Invalid}{N.S.}
  {Call near, relative displacement.}
  \xopcode{E8 \emph{cd}}{CALL \emph{rel32}}
  {D}{Valid}{Valid}
  {Call near, relative displacement, 32-bit displacement sign-extended
    to 64-bits.}
  \xopcode{FF /2}{CALL \emph{r/m16}}
  {M}{N.E.}{N.E.}
  {Call near, absolute indirect, address given in \emph{r/m16}.}
  \xopcode{FF /2}{CALL \emph{r/m32}}
  {M}{N.E.}{N.E.}
  {Call near, absolute indirect, address given in \emph{r/m32}.}
  \xopcode{FF /2}{CALL \emph{r/m64}}
  {M}{Invalid}{Valid}
  {Call near, absolute indirect, address given in \emph{r/m64}.}
  \xopcode{FF /2}{CALL \emph{r/mc}}
  {M}{Valid}{Valid}
  {Call near, absolute indirect, address given in \emph{r/mc}.}
  \xopcode{9A \emph{cd}}{CALL \emph{ptr16:16}}
  {D}{N.E.}{N.E.}
  {Call far, absolute, address in operand.}
  \xopcode{9A \emph{cp}}{CALL \emph{ptr16:32}}
  {D}{N.E.}{N.E.}
  {Call far, absolute, address in operand.}
  \xopcode{FF /3}{CALL \emph{m16:16}}
  {M}{Invalid}{Valid}
  {Call far, absolute indirect, address given in \emph{m16:16}.}
  \xopcode{FF /3}{CALL \emph{m16:32}}
  {M}{Invalid}{Valid}
  {Call far, absolute indirect, address given in \emph{m16:32}.}
  \xopcode{REX.W FF /3}{CALL \emph{m16:64}}
  {M}{Invalid}{Valid}
  {Call far, absolute indirect, address given in \emph{m16:64}.}
\end{x86opcodetable}

\begin{x86opentable}
  \xopen{D}{Offset}{NA}{NA}{NA}
  \xopen{M}{ModRM:r/m (r)}{NA}{NA}{NA}
\end{x86opentable}

\subsubsection*{Description}

Saves return address on the stack and branches to the location
specified by the first operand.  In 64-bit mode the \insnnoref{CAP}
prefix can be used with opcode \texttt{FF /2} to select the capability
operand size instead of 64-bit.  In capability mode, near calls always
use the capability operand size.

Relative calls always apply the relative displacement to the address
of the next instruction to compute a new value of the \textbf{address}
field of \CIP{}.

Near calls which use a capability operand size always push \CIP{} onto
the current stack before executing the first instruction at the target
rather than pushing the value of \RIP{}.

\subsubsection*{Flags Affected}

None

\clearpage
\phantomsection
\addcontentsline{toc}{subsection}{CMOVcc - Conditional Move}
\insnxeslabel{cmov}
\subsection*{CMOV\emph{cc} - Conditional Move}

\begin{x86opcodetable}
  \xopcode{CAP + 0F 40 \emph{/r}}{CMOVO \emph{rc, r/mc}}
  {RM}{Valid}{Valid}
  {Move if overflow (OF=1).}
  \xopcode{CAP + 0F 41 \emph{/r}}{CMOVNO \emph{rc, r/mc}}
  {RM}{Valid}{Valid}
  {Move if not overflow (OF=0).}
  \xopcode{CAP + 0F 42 \emph{/r}}{CMOVC \emph{rc, r/mc}}
  {RM}{Valid}{Valid}
  {Move if carry (CF=1).}
  \xopcode{CAP + 0F 43 \emph{/r}}{CMOVNC \emph{rc, r/mc}}
  {RM}{Valid}{Valid}
  {Move if not carry (CF=0).}
  \xopcode{CAP + 0F 44 \emph{/r}}{CMOVZ \emph{rc, r/mc}}
  {RM}{Valid}{Valid}
  {Move if zero (ZF=1).}
  \xopcode{CAP + 0F 45 \emph{/r}}{CMOVNZ \emph{rc, r/mc}}
  {RM}{Valid}{Valid}
  {Move if not zero (ZF=0).}
  \xopcode{CAP + 0F 46 \emph{/r}}{CMOVBE \emph{rc, r/mc}}
  {RM}{Valid}{Valid}
  {Move if below or equal (CF=1 or ZF=1).}
  \xopcode{CAP + 0F 47 \emph{/r}}{CMOVA \emph{rc, r/mc}}
  {RM}{Valid}{Valid}
  {Move if above (CF=0 and ZF=0).}
  \xopcode{CAP + 0F 48 \emph{/r}}{CMOVS \emph{rc, r/mc}}
  {RM}{Valid}{Valid}
  {Move if sign (SF=1).}
  \xopcode{CAP + 0F 49 \emph{/r}}{CMOVNS \emph{rc, r/mc}}
  {RM}{Valid}{Valid}
  {Move if not sign (SF=0).}
  \xopcode{CAP + 0F 4A \emph{/r}}{CMOVP \emph{rc, r/mc}}
  {RM}{Valid}{Valid}
  {Move if parity (PF=1).}
  \xopcode{CAP + 0F 4B \emph{/r}}{CMOVNP \emph{rc, r/mc}}
  {RM}{Valid}{Valid}
  {Move if no parity (PF=0).}
  \xopcode{CAP + 0F 4C \emph{/r}}{CMOVL \emph{rc, r/mc}}
  {RM}{Valid}{Valid}
  {Move if less (SF$\neq{}$OF).}
  \xopcode{CAP + 0F 4D \emph{/r}}{CMOVGE \emph{rc, r/mc}}
  {RM}{Valid}{Valid}
  {Move if greater or equal (SF=OF).}
  \xopcode{CAP + 0F 4E \emph{/r}}{CMOVLE \emph{rc, r/mc}}
  {RM}{Valid}{Valid}
  {Move if less or equal (ZF=1 or SF$\neq{}$OF).}
  \xopcode{CAP + 0F 4F \emph{/r}}{CMOVG \emph{rc, r/mc}}
  {RM}{Valid}{Valid}
  {Move if greater (ZF=0 and SF=OF).}
\end{x86opcodetable}

\begin{x86opentable}
  \xopen{RM}{ModRM:Reg (w)}{ModRM:r/m (r)}{NA}{NA}
\end{x86opentable}

\subsubsection*{Description}

Copies the source operand to the destination operand if one or more
status flags in the \RFLAGS{} register are in the required state. The
destination operand is a register.  The source operand can be a
register or memory location.

\subsubsection*{Flags Affected}

None

\clearpage
\phantomsection
\addcontentsline{toc}{subsection}{CMP - Compare Two Operands}
\insnxeslabel{cmp}
\subsection*{CMP - Compare Two Operands}

\begin{x86opcodetable}
  \xopcode{CAP + 39 \emph{/r}}{CMP \emph{r/mc, rc}}
  {MR}{Valid}{Valid}
  {Compare \emph{rc} with \emph{r/mc}.}
  \xopcode{CAP + 3B \emph{/r}}{CMP \emph{rc, r/mc}}
  {RM}{Valid}{Valid}
  {Compare \emph{r/mc} with \emph{rc}.}
\end{x86opcodetable}

\begin{x86opentable}
  \xopen{RM}{ModRM:Reg (r)}{ModRM:r/m (r)}{NA}{NA}
  \xopen{MR}{ModRM:r/m (r)}{ModRM:reg (r)}{NA}{NA}
\end{x86opentable}

\subsubsection*{Description}

Compares the first source operand with the scond source operand and
sets status flags in \RFLAGS{}.  Sets ZF to 1 if the two operands are
equal and 0 otherwise.  Sets SF to 1 if the \textbf{tag} fields of the
two operands are equal and the bounds and permissions of the second
operand are a subset of the first operand.

\subsubsection*{Flags Affected}

The ZF and SF flags are set as described above.  The OF, SF, AF, CF,
and PF flags are undefined.

\clearpage
\phantomsection
\addcontentsline{toc}{subsection}{CMPXCHG -- Compare and Exchange}
\insnxeslabel{cmpxchg}
\subsection*{CMPXCHG -- Compare and Exchange}

\begin{x86opcodetable}
  \xopcode{CAP + 0F B1 \emph{/r}}{CMPXCHG \emph{r/mc, rc}}
  {MR}{Valid}{Valid}
  {Compare CAX with \emph{r/mc}. If equal, store \emph{rc} to
    \emph{r/mc}, else load \emph{r/mc} into CAX.}
\end{x86opcodetable}

\begin{x86opentable}
  \xopen{MR}{ModRM:r/m (r,w)}{ModRM:reg (r)}{NA}{NA}
\end{x86opentable}

\subsubsection*{Description}

Compares \CAX{} with the destination operand.  If the values are
equal, stores the source operand in the destination operand.  If the
values are not equal, loads the destination operand into \CAX{}.

The instruction can be used with a \insnnoref{LOCK} prefix to execute
atomically.

\subsubsection*{Flags Affected}

The OF, SF, ZF, AF, CF, and PF flags are set according to the value of
the result of comparing \CAX{} with the destination operand.

\clearpage
\phantomsection
\addcontentsline{toc}{subsection}{CMPXCHG2C -- Compare and Exchange Pair}
\insnxeslabel{cmpxchg2c}
\subsection*{CMPXCHG2C -- Compare and Exchange Pair}

\begin{x86opcodetable}
  \xopcode{CAP + 0F C7 /1}{CMPXCHG2C \emph{m2c}}
  {MR}{Valid}{Valid}
  {Compare CDX:CAX with \emph{m2c}. If equal, set ZF and CCX:CBX to
    \emph{m2c}, else load \emph{m2c} into CDX:CAX.}
\end{x86opcodetable}

\begin{x86opentable}
  \xopen{MR}{ModRM:r/m (r,w)}{ModRM:reg (r)}{NA}{NA}
\end{x86opentable}

\subsubsection*{Description}

Compares the pair of capabilities in \CDX{}:\CAX{} with the
destination operand.  \CAX{} is compared with the first capability in
memory, and \CDX{} is compared with the second capability in memory.
If the capabilities are equal, stores \CCX{}:\CBX{} in the destination
operand.  If the capabilities are not equal, loads the destination
operand into \CDX{}:\CAX{}.

The instruction can be used with a \insnnoref{LOCK} prefix to execute
atomically.

\subsubsection*{Flags Affected}

The ZF flag is set to the result of the comparison (1 if equal).  The
OF, SF, AF, CF, and PF flags are unaffected.

\clearpage
\phantomsection
\addcontentsline{toc}{subsection}{DEC - Decrement by 1}
\insnxeslabel{dec}
\subsection*{DEC - Decrement by 1}

\begin{x86opcodetable}
  \xopcode{CAP + FF /1}{DEC \emph{r/mc}}
  {M}{Valid}{Valid}
  {Decrement address field of \emph{r/mc} by 1.}
\end{x86opcodetable}

\begin{x86opentable}
  \xopen{M}{ModRM:r/m (r,w)}{NA}{NA}{NA}
\end{x86opentable}

\subsubsection*{Description}

Subtracts 1 from the \textbf{address} field of the destination operand
and then stores the result in the destination operand. The destination
operand can be a register or a memory location.

If the new value of the \textbf{address} field makes the resulting
capability unrepresentable, the \textbf{tag} field in the resulting
capability is cleared.

\subsubsection*{Flags Affected}

The CF flag is not affected.  The OF, SF, ZF, AF, and PF flags are set
according to the value of the resulting \textbf{address} field.

\clearpage
\phantomsection
\addcontentsline{toc}{subsection}{INC -- Increment by 1}
\insnxeslabel{inc}
\subsection*{INC -- Increment by 1}

\begin{x86opcodetable}
  \xopcode{CAP + FF /0}{INC \emph{r/mc}}
  {M}{Valid}{Valid}
  {Increment address field of \emph{r/mc} by 1.}
\end{x86opcodetable}

\begin{x86opentable}
  \xopen{M}{ModRM:r/m (r,w)}{NA}{NA}{NA}
\end{x86opentable}

\subsubsection*{Description}

Adds 1 to the \textbf{address} field of the destination operand and
then stores the result in the destination operand. The destination
operand can be a register or a memory location.

If the new value of the \textbf{address} field makes the resulting
capability unrepresentable, the \textbf{tag} field in the resulting
capability is cleared.

\subsubsection*{Flags Affected}

The CF flag is not affected.  The OF, SF, ZF, AF, and PF flags are set
according to the value of the resulting \textbf{address} field.

\clearpage
\phantomsection
\addcontentsline{toc}{subsection}{JMP -- Jump}
\insnxeslabel{jmp}
\subsection*{JMP -- Jump}

\begin{x86opcodetable}
  \xopcode{E9 \emph{cw}}{JMP \emph{rel16}}
  {D}{Invalid}{N.S.}
  {Jump near, relative displacement.}
  \xopcode{E9 \emph{cd}}{JMP \emph{rel32}}
  {D}{Valid}{Valid}
  {Jump near, relative displacement, 32-bit displacement sign-extended
    to 64-bits.}
  \xopcode{FF /4}{JMP \emph{r/m16}}
  {M}{N.E.}{N.S.}
  {Jump near, absolute indirect, address given in \emph{r/m16}.}
  \xopcode{FF /4}{JMP \emph{r/m32}}
  {M}{N.E.}{N.S.}
  {Jump near, absolute indirect, address given in \emph{r/m32}.}
  \xopcode{FF /4}{JMP \emph{r/m64}}
  {M}{Invalid}{Valid}
  {Jump near, absolute indirect, address given in \emph{r/m64}.}
  \xopcode{FF /4}{JMP \emph{r/mc}}
  {M}{Valid}{Valid}
  {Jump near, absolute indirect, address given in \emph{r/mc}.}
  \xopcode{EA \emph{cd}}{JMP \emph{ptr16:16}}
  {D}{N.E.}{Invalid}
  {Jump far, absolute, address in operand.}
  \xopcode{EA \emph{cp}}{JMP \emph{ptr16:32}}
  {D}{N.E.}{Invalid}
  {Jump far, absolute, address in operand.}
  \xopcode{FF /5}{JMP \emph{m16:16}}
  {M}{Invalid}{Valid}
  {Jump far, absolute indirect, address given in \emph{m16:16}.}
  \xopcode{FF /5}{JMP \emph{m16:32}}
  {M}{Invalid}{Valid}
  {Jump far, absolute indirect, address given in \emph{m16:32}.}
  \xopcode{REX.W FF /5}{JMP \emph{m16:64}}
  {M}{Invalid}{Valid}
  {Jump far, absolute indirect, address given in \emph{m16:64}.}
\end{x86opcodetable}

\begin{x86opentable}
  \xopen{D}{Offset}{NA}{NA}{NA}
  \xopen{M}{ModRM:r/m (r)}{NA}{NA}{NA}
\end{x86opentable}

\subsubsection*{Description}

Branches to the location specified by the first operand.  In 64-bit
mode the \insnnoref{CAP} prefix can be used with opcode \texttt{FF /4}
to select the capability operand size instead of 64-bit.  In
capability mode, \texttt{FF /4} supports only the capability operand
size.

Relative jumps always apply the relative displacement to the address
of the next instruction to compute a new value of the \textbf{address}
field of \CIP{}.

\subsubsection*{Flags Affected}

None

\clearpage
\phantomsection
\addcontentsline{toc}{subsection}{LEA - Load Effective Address}
\insnxeslabel{lea}
\subsection*{LEA - Load Effective Address}

\begin{x86opcodetable}
  \xopcode{CAP + 8D \emph{/r}}{LEA \emph{rc, m}}
  {RM}{Valid}{Valid}
  {Store effective capability address of \emph{m} in \emph{rc}.}
\end{x86opcodetable}

\begin{x86opentable}
  \xopen{RM}{ModRM:reg (w)}{ModRM:r/m (r)}{NA}{NA}
\end{x86opentable}

\subsubsection*{Description}

Computes the effective capability of the second operand and stores the
result in the first operand.  When used with the \insnnoref{CAP}
prefix, the effective address is always computed using
capability-aware addressing.

\subsubsection*{Flags Affected}

None

\clearpage
\phantomsection
\addcontentsline{toc}{subsection}{LODS/LODSC -- Load String}
\insnxeslabel{lods}
\subsection*{LODS/LODSC -- Load String}

\begin{x86opcodetable}
  \xopcode{CAP + AD}{LODS \emph{mc}}
  {ZO}{Valid}{Valid}
  {Load capability at address (C|R)SI into CAX.}
  \xopcode{CAP + AD}{LODSC}
  {ZO}{Valid}{Valid}
  {Load capability at address (C|R)SI into CAX.}
\end{x86opcodetable}

\begin{x86opentable}
  \xopen{ZO}{NA}{NA}{NA}{NA}
\end{x86opentable}

\subsubsection*{Description}

Loads a capablity from the source operand into the \CAX{} register.
The source operand is a memory location identified by the \RSI{} or
\CSI{} register (depending on the addressing mode).  After the
capability is loaded from memory, the address register is incremented
or decremented by the size of a capability according to the setting of
\texttt{DF} in \RFLAGS{}.

\subsubsection*{Flags Affected}

None

\clearpage
\phantomsection
\addcontentsline{toc}{subsection}{MOV -- Move}
\insnxeslabel{mov}
\subsection*{MOV -- Move}

\begin{x86opcodetable}
  \xopcode{CAP + 89 \emph{/r}}{MOV \emph{r/mc, rc}}
  {MR}{Valid}{Valid}
  {Move \emph{rc} to \emph{r/mc}.}
  \xopcode{CAP + 8B \emph{/r}}{MOV \emph{rc, r/mc}}
  {RM}{Valid}{Valid}
  {Move \emph{r/mc} to \emph{rc}.}
  \xopcode{CAP + C7 \emph{/0 id}}{MOV \emph{r/mc, imm32}}
  {MI}{Valid}{Valid}
  {Move \emph{imm32 sign-extended to 64-bits} to \emph{r/mc}.}
\end{x86opcodetable}

\begin{x86opentable}
  \xopen{MR}{ModRM:r/m (w)}{ModRM:reg (r)}{NA}{NA}
  \xopen{RM}{ModRM:Reg (w)}{ModRM:r/m (r)}{NA}{NA}
  \xopen{MI}{ModRM:r/m (w)}{imm32}{NA}{NA}
\end{x86opentable}

\subsubsection*{Description}

Copies the source operand to the destination operand. The destination
operand can be a register or a memory location. The source operand can
be an immediate, a register, or a memory location.  If the source
operand is an immediate, the value is sign-extended to 64-bits and
used as the address of a NULL-derived capability.

Note that some \insnnoref{MOV} opcodes such as \texttt{B8+ rw} and
\texttt{C6 /0} are not extended to support the \insnnoref{CAP} prefix
as the behavior would be identical.  The \texttt{C7 /0} opcode is
extended primarily to support storing constants such as NULL to
capabilities in memory without requiring an intermediate register.

The \texttt{A1} and \texttt{A3} opcodes are not extended to support
capabilities.

\subsubsection*{Flags Affected}

None

\clearpage
\phantomsection
\addcontentsline{toc}{subsection}{MOVNTI - Store Using Non-Temporal Hint}
\insnxeslabel{movnti}
\subsection*{MOVNTI - Store Using Non-Temporal Hint}

\begin{x86opcodetable}
  \xopcode{NP CAP + 0F C3 \emph{/r}}{MOV \emph{mc, rc}}
  {MR}{Valid}{Valid}
  {Move \emph{rc} to \emph{mc} using non-temporal hint.}
\end{x86opcodetable}

\begin{x86opentable}
  \xopen{MR}{ModRM:r/m (w)}{ModRM:reg (r)}{NA}{NA}
\end{x86opentable}

\subsubsection*{Description}

Moves the capability in the source operand to the destination operand
using a non-temporal hint.

\subsubsection*{Flags Affected}

None

\clearpage
\phantomsection
\addcontentsline{toc}{subsection}{MOVS/MOVSC -- Move Data from String
  to String}
\insnxeslabel{movs}
\subsection*{MOVS/MOVSC -- Move Data from String to String}

\begin{x86opcodetable}
  \xopcode{CAP + A5}{MOVS \emph{mc, mc}}
  {ZO}{Valid}{Valid}
  {Move capability from address (C|R)SI to (C|R)DI.}
  \xopcode{CAP + A5}{MOVSC}
  {ZO}{Valid}{Valid}
  {Move capability from address (C|R)SI to (C|R)DI.}
\end{x86opcodetable}

\begin{x86opentable}
  \xopen{ZO}{NA}{NA}{NA}{NA}
\end{x86opentable}

\subsubsection*{Description}

Moves a capablity from the source operand to the destination operand.
The source operand is a memory location identified by the \RSI{} or
\CSI{} register (depending on the addressing mode).  The destination
operand is a memory location identified by the \RDI{} or \CDI{}
register (depending on the addressing mode).  After the capability is
copied, the address registers are incremented or decremented by the
size of a capability according to the setting of \texttt{DF} in
\RFLAGS{}.

\subsubsection*{Flags Affected}

None

\clearpage
\phantomsection
\addcontentsline{toc}{subsection}{OR -- Logical Inclusive OR}
\insnxeslabel{or}
\subsection*{OR -- Logical Inclusive OR}

\begin{x86opcodetable}
  \xopcode{CAP + 0D \emph{id}}{OR CAX\emph{, imm32}}
  {I}{Valid}{Valid}
  {Bitwise inclusive OR of \emph{imm32 sign-extended to 64-bits} with
    the address field of CAX.}
  \xopcode{CAP + 81 /1 \emph{id}}{OR \emph{r/mc, imm32}}
  {MI}{Valid}{Valid}
  {Bitwise inclusive OR of \emph{imm32 sign-extended to 64-bits} with
    the address field of \emph{r/mc}.}
  \xopcode{CAP + 83 /1 \emph{ib}}{OR \emph{r/mc, imm8}}
  {MI}{Valid}{Valid}
  {Bitwise inclusive OR of \emph{sign}-extended \emph{imm8} with the
    address field of \emph{r/mc}.}
  \xopcode{CAP + 09 \emph{/r}}{OR \emph{r/mc, r64}}
  {MR}{Valid}{Valid}
  {Bitwise inclusive OR of \emph{r64} with the address field of
    \emph{r/mc}.}
  \xopcode{CAP + 0B \emph{/r}}{OR \emph{rc, r/m64}}
  {RM}{Valid}{Valid}
  {Bitwise inclusive OR of \emph{r/m64} with the address field of
    \emph{rc}.}
\end{x86opcodetable}

\begin{x86opentable}
  \xopen{RM}{ModRM:Reg (r,w)}{ModRM:r/m (r)}{NA}{NA}
  \xopen{MR}{ModRM:r/m (r,w)}{ModRM:reg (r)}{NA}{NA}
  \xopen{MI}{ModRM:r/m (r,w)}{imm8/32}{NA}{NA}
  \xopen{I}{CAX}{imm32}{NA}{NA}
\end{x86opentable}

\subsubsection*{Description}

Derives a new capability from the destination operand whose
\textbf{address} field is set to the bitwise inclusive OR of the
source operand and the \textbf{address} field of the destination
operand and then stores the result in the destination operand. The
destination operand can be a register or a memory location. The source
operand can be an immediate, a register, or a memory location.

If the new value of the \textbf{address} field makes the resulting
capability unrepresentable, the \textbf{tag} field in the resulting
capability is cleared.

\subsubsection*{Flags Affected}

The OF, SF, ZF, AF, CF, and PF flags are set according to the value of
the resulting \textbf{address} field.

\clearpage
\phantomsection
\addcontentsline{toc}{subsection}{POP -- Pop Value from the Stack}
\insnxeslabel{pop}
\subsection*{POP -- Pop Value from the Stack}

\begin{x86opcodetable}
  \xopcode{8F /0}{POP \emph{r/m16}}
  {M}{N.E.}{Valid}
  {Pop \emph{r/m16}.}
  \xopcode{8F /0}{POP \emph{r/m32}}
  {M}{N.E.}{N.E.}
  {Pop \emph{r/m32}.}
  \xopcode{8F /0}{POP \emph{r/m64}}
  {M}{Valid}{Valid}
  {Pop \emph{r/m64}.}
  \xopcode{8F /0}{POP \emph{r/mc}}
  {M}{Valid}{Valid}
  {Pop \emph{r/mc}.}
  \xopcode{58+\emph{rw}}{POP \emph{r16}}
  {O}{N.E.}{Valid}
  {Pop \emph{r16}.}
  \xopcode{58+\emph{rd}}{POP \emph{r32}}
  {O}{N.E.}{N.E.}
  {Pop \emph{r32}.}
  \xopcode{58+\emph{ro}}{POP \emph{r64}}
  {O}{Valid}{Valid}
  {Pop \emph{r64}.}
  \xopcode{58+\emph{rc}}{POP \emph{rc}}
  {O}{Valid}{Valid}
  {Pop \emph{rc}.}
\end{x86opcodetable}

\begin{x86opentable}
  \xopen{M}{ModRM:r/m (w)}{NA}{NA}{NA}
  \xopen{O}{opcode + rd (w)}{NA}{NA}{NA}
\end{x86opentable}

\subsubsection*{Description}

Stores the value at the top of the stack in the destination operand
and increments the stack pointer.  The following extensions apply to
these instructions in 64-bit mode:

\begin{itemize}
  \item Address size: The 0x07 prefix selects a capability-aware
    address.

  \item Operand size: The \insnnoref{CAP} prefix selects a capability
    operand size.
\end{itemize}

In capability mode, the various sizes are:

\begin{itemize}
  \item Address size: The default addressing mode uses
    capability-aware addressing.  64-bit addresses can be used by
    specifying the 0x07 prefix.

  \item Operand size: The default operand size is a capability.  If
    the \insnnoref{CAP} prefix is specified, the operand size is
    64-bits.  16-bit and 32-bit operands cannot be used in capability
    mode.

  \item Stack-address size: In capability mode the stack pointer is
    always \CSP{}.
\end{itemize}

\subsubsection*{Flags Affected}

None

\clearpage
\phantomsection
\addcontentsline{toc}{subsection}{PUSH -- Push Value Onto the Stack}
\insnxeslabel{push}
\subsection*{PUSH -- Push Value Onto the Stack}

\begin{x86opcodetable}
  \xopcode{FF /6}{PUSH \emph{r/m16}}
  {M}{N.E.}{Valid}
  {Push \emph{r/m16}.}
  \xopcode{FF /6}{PUSH \emph{r/m32}}
  {M}{N.E.}{N.E.}
  {Push \emph{r/m32}.}
  \xopcode{FF /6}{PUSH \emph{r/m64}}
  {M}{Valid}{Valid}
  {Push \emph{r/m64}.}
  \xopcode{FF /6}{PUSH \emph{r/mc}}
  {M}{Valid}{Valid}
  {Push \emph{r/mc}.}
  \xopcode{50+\emph{rw}}{PUSH \emph{r16}}
  {O}{N.E.}{Valid}
  {Push \emph{r16}.}
  \xopcode{50+\emph{rd}}{PUSH \emph{r32}}
  {O}{N.E.}{N.E.}
  {Push \emph{r32}.}
  \xopcode{50+\emph{ro}}{PUSH \emph{r64}}
  {O}{Valid}{Valid}
  {Push \emph{r64}.}
  \xopcode{50+\emph{rc}}{PUSH \emph{rc}}
  {O}{Valid}{Valid}
  {Push \emph{rc}.}
  \xopcode{6A \emph{ib}}{PUSH \emph{imm8}}
  {I}{Valid}{Valid}
  {Push \emph{imm8}.}
  \xopcode{68 \emph{iw}}{PUSH \emph{imm16}}
  {I}{Valid}{Valid}
  {Push \emph{imm16}.}
  \xopcode{68 \emph{id}}{PUSH \emph{imm32}}
  {I}{Valid}{Valid}
  {Push \emph{imm32}.}
\end{x86opcodetable}

\begin{x86opentable}
  \xopen{M}{ModRM:r/m (r)}{NA}{NA}{NA}
  \xopen{O}{opcode + rd (r)}{NA}{NA}{NA}
  \xopen{I}{imm8/16/32}{NA}{NA}{NA}
\end{x86opentable}

\subsubsection*{Description}

Decrements the stack pointer and stores the source operand on the
stack.  The following extensions apply to these instructions in 64-bit
mode:

\begin{itemize}
  \item Address size: The 0x07 prefix selects a capability-aware
    address.

  \item Operand size: The \insnnoref{CAP} prefix selects a capability
    operand size.  When a capability operand size is used, immediate
    operands are sign-extended to 64-bits and the result used as the
    address of a null-derived capability.
\end{itemize}

In capability mode, the various sizes are:

\begin{itemize}
  \item Address size: The default addressing mode uses
    capability-aware addressing.  64-bit addresses can be used by
    specifying the 0x07 prefix.

  \item Operand size: The default operand size is a capability.  If
    the \insnnoref{CAP} prefix is specified, the operand size is
    64-bits.  Immediate operands are sign-extended to 64-bits.  When a
    capability operand size is used, sign-extended immediate operands
    are used as the address of a null-derived capability.

  \item Stack-address size: In capability mode the stack pointer is
    always \CSP{}.
\end{itemize}

\subsubsection*{Flags Affected}

None

\clearpage
\phantomsection
\addcontentsline{toc}{subsection}{RET -- Return from Procedure}
\insnxeslabel{ret}
\subsection*{RET -- Return from Procedure}

\begin{x86opcodetable}
  \xopcode{C3}{RET}
  {ZO}{Valid}{Valid}
  {Near return.}
  \xopcode{CB}{RET}
  {ZO}{Invalid}{Valid}
  {Far return.}
  \xopcode{C2 \emph{iw}}{RET \emph{imm16}}
  {I}{Valid}{Valid}
  {Near return and pop \emph{imm16} bytes from stack.}
  \xopcode{CA \emph{iw}}{RET \emph{imm16}}
  {I}{Invalid}{Valid}
  {Far return and pop \emph{imm16} bytes from stack.}
\end{x86opcodetable}

\begin{x86opentable}
  \xopen{ZO}{NA}{NA}{NA}{NA}
  \xopen{I}{imm16}{NA}{NA}{NA}
\end{x86opentable}

\subsubsection*{Description}

Pops return address from the stack and transfers control to the popped
address.  In 64-bit mode the \insnnoref{CAP} prefix can be used with
near returns to select the capability operand size instead of 64-bit.
In capability mode, near returns always use the capability operand
size.

Near returns which use a capability operand size always pop a
capability off of the stack to load into \CIP{} rather than popping
off the new value of \RIP{}.

\subsubsection*{Flags Affected}

None

\clearpage
\phantomsection
\addcontentsline{toc}{subsection}{STOS/STOSC - Store String}
\insnxeslabel{stos}
\subsection*{STOS/STOSC - Store String}

\begin{x86opcodetable}
  \xopcode{CAP + AB}{STOS \emph{mc}}
  {ZO}{Valid}{Valid}
  {Store CAX at address (C|R)DI.}
  \xopcode{CAP + AB}{STOSC}
  {ZO}{Valid}{Valid}
  {Store CAX at address (C|R)DI.}
\end{x86opcodetable}

\begin{x86opentable}
  \xopen{ZO}{NA}{NA}{NA}{NA}
\end{x86opentable}

\subsubsection*{Description}

Stores capability in the \CAX{} register into the destination operand.
The destination operand is a memory location identified by the \RDI{}
or \CDI{} register (depending on the addressing mode).  After the
capability is stored to memory, the address register is incremented or
decremented by the size of a capability according to the setting of
\texttt{DF} in \RFLAGS{}.

\subsubsection*{Flags Affected}

None

\clearpage
\phantomsection
\addcontentsline{toc}{subsection}{SUB -- Subtract}
\insnxeslabel{sub}
\subsection*{SUB -- Subtract}

\begin{x86opcodetable}
  \xopcode{CAP + 2D \emph{id}}{SUB CAX\emph{, imm32}}
  {I}{Valid}{Valid}
  {Subtract \emph{imm32 sign-extended to 64-bits} from the address field of
    CAX.}
  \xopcode{CAP + 81 /5 \emph{id}}{SUB \emph{r/mc, imm32}}
  {MI}{Valid}{Valid}
  {Subract \emph{imm32 sign-extended to 64-bits} from the address field of
    \emph{r/mc}.}
  \xopcode{CAP + 83 /5 \emph{ib}}{SUB \emph{r/mc, imm8}}
  {MI}{Valid}{Valid}
  {Subtract \emph{sign}-extended \emph{imm8} from the address field of
    \emph{r/mc}.}
  \xopcode{CAP + 29 \emph{/r}}{SUB \emph{r/mc, r64}}
  {MR}{Valid}{Valid}
  {Subtract \emph{r64} from the address field of \emph{r/mc}.}
  \xopcode{CAP + 2B \emph{/r}}{SUB \emph{rc, r/m64}}
  {RM}{Valid}{Valid}
  {Subtract \emph{r/m64} from the address field of \emph{rc}.}
\end{x86opcodetable}

\begin{x86opentable}
  \xopen{RM}{ModRM:Reg (r,w)}{ModRM:r/m (r)}{NA}{NA}
  \xopen{MR}{ModRM:r/m (r,w)}{ModRM:reg (r)}{NA}{NA}
  \xopen{MI}{ModRM:r/m (r,w)}{imm8/32}{NA}{NA}
  \xopen{I}{CAX}{imm32}{NA}{NA}
\end{x86opentable}

\subsubsection*{Description}

Subtracts the source operand from the \textbf{address} field of the
destination operand and then stores the result in the destination
operand. The destination operand can be a register or a memory
location. The source operand can be an immediate, a register, or a
memory location.

If the new value of the \textbf{address} field makes the resulting
capability unrepresentable, the \textbf{tag} field in the resulting
capability is cleared.

\subsubsection*{Flags Affected}

The OF, SF, ZF, AF, CF, and PF flags are set according to the value of
the resulting \textbf{address} field.

\clearpage
\phantomsection
\addcontentsline{toc}{subsection}{XADD - Exchange and Add}
\insnxeslabel{xadd}
\subsection*{XADD - Exchange and Add}

\begin{x86opcodetable}
  \xopcode{CAP + 0F C1 \emph{/r}}{XADD \emph{r/mc, rc, r64}}
  {MRR}{Valid}{Valid}
  {Load original value of \emph{r/mc} into \emph{rc}.  Add \emph{r64}
    to the address field of \emph{r/mc}.}
\end{x86opcodetable}

\begin{x86opentable}
  \xopen{MRR}{ModRM:r/m (r,w)}{ModRM:reg (w)}{ModRM:reg (r)}{NA}
\end{x86opentable}

\subsubsection*{Description}

Stores the original value of the destination (first) operand in the second
operand.  Derives a new capability value by adding the third operand
to the \textbf{address} field of the destination operand and stores
the result in the destination operand.  Note that the third operand
must be the 64-bit register which aliases the low 64-bits of the
second operand.

If the new value of the \textbf{address} field makes the resulting
capability unrepresentable, the \textbf{tag} field in the resulting
capability is cleared.

The instruction can be used with a \insnnoref{LOCK} prefix to execute
atomically.

\subsubsection*{Flags Affected}

The OF, SF, ZF, AF, CF, and PF flags are set according to the value of
the resulting \textbf{address} field.

\clearpage
\phantomsection
\addcontentsline{toc}{subsection}{XCHG -- Exchange}
\insnxeslabel{xchg}
\subsection*{XCHG -- Exchange}

\begin{x86opcodetable}
  \xopcode{CAP + 90+\emph{rc}}{XCHG CAX, \emph{rc}}
  {O}{Valid}{Valid}
  {Exchange \emph{rc} with CAX.}
  \xopcode{CAP + 90+\emph{rc}}{XCHG \emph{rc}, CAX}
  {O}{Valid}{Valid}
  {Exchange CAX with \emph{rc}.}
  \xopcode{CAP + 87 \emph{/r}}{XCHG \emph{r/mc, rc}}
  {MR}{Valid}{Valid}
  {Exchange \emph{rc} with \emph{r/mc}.}
  \xopcode{CAP + 87 \emph{/r}}{XCHG \emph{rc, r/mc}}
  {RM}{Valid}{Valid}
  {Exchange \emph{r/mc} with \emph{rc}.}
\end{x86opcodetable}

\begin{x86opentable}
  \xopen{O}{CAX (r,w)}{opcode + rc (r, w)}{NA}{NA}
  \xopen{O}{opcode + rc (r, w)}{CAX (r,w)}{NA}{NA}
  \xopen{MR}{ModRM:r/m (r,w)}{ModRM:reg (r, w)}{NA}{NA}
  \xopen{RM}{ModRM:reg (r,w)}{ModRM:r/m (r, w)}{NA}{NA}
\end{x86opentable}

\subsubsection*{Description}

Exchanges the contents of the source and destination operands.  The
operands can be two general-purpose registers or a general-purpose
register and a memory location.

The specific opcode \texttt{90} would remain a single byte
\insnnoref{NOP} that would not alter the value of \CAX{}.

\subsubsection*{Flags Affected}

None

\clearpage
\phantomsection
\addcontentsline{toc}{subsection}{XOR -- Logical Exclusive OR}
\insnxeslabel{xor}
\subsection*{XOR -- Logical Exclusive OR}

\begin{x86opcodetable}
  \xopcode{CAP + 35 \emph{id}}{XOR CAX\emph{, imm32}}
  {I}{Valid}{Valid}
  {Bitwise exclusive OR of \emph{imm32 sign-extended to 64-bits} with
    the address field of CAX.}
  \xopcode{CAP + 81 /6 \emph{id}}{XOR \emph{r/mc, imm32}}
  {MI}{Valid}{Valid}
  {Bitwise exclusive OR of \emph{imm32 sign-extended to 64-bits} with
    the address field of \emph{r/mc}.}
  \xopcode{CAP + 83 /6 \emph{ib}}{XOR \emph{r/mc, imm8}}
  {MI}{Valid}{Valid}
  {Bitwise exclusive OR of \emph{sign}-extended \emph{imm8} with the
    address field of \emph{r/mc}.}
  \xopcode{CAP + 31 \emph{/r}}{XOR \emph{r/mc, r64}}
  {MR}{Valid}{Valid}
  {Bitwise exclusive OR of \emph{r64} with the address field of \emph{r/mc}.}
  \xopcode{CAP + 33 \emph{/r}}{XOR \emph{rc, r/m64}}
  {RM}{Valid}{Valid}
  {Bitwise exclusive OR of \emph{r/m64} with the address field of \emph{rc}.}
\end{x86opcodetable}

\begin{x86opentable}
  \xopen{RM}{ModRM:Reg (r,w)}{ModRM:r/m (r)}{NA}{NA}
  \xopen{MR}{ModRM:r/m (r,w)}{ModRM:reg (r)}{NA}{NA}
  \xopen{MI}{ModRM:r/m (r,w)}{imm8/32}{NA}{NA}
  \xopen{I}{CAX}{imm32}{NA}{NA}
\end{x86opentable}

\subsubsection*{Description}

Derives a new capability from the destination operand whose
\textbf{address} field is set to the bitwise exclusive OR of the
source operand and the \textbf{address} field of the destination
operand and then stores the result in the destination operand. The
destination operand can be a register or a memory location. The source
operand can be an immediate, a register, or a memory location.

If the new value of the \textbf{address} field makes the resulting
capability unrepresentable, the \textbf{tag} field in the resulting
capability is cleared.

\subsubsection*{Flags Affected}

The OF, SF, ZF, AF, CF, and PF flags are set according to the value of
the resulting \textbf{address} field.


\clearpage
\section{CHERI-x86-64 Instructions}

This section contains new instructions added to support operations on
capabilities.  The opcode assignments in this section are tentative
and subject to change.  Single byte opcodes have been used for
instructions which we believe may either be used frequently or in
frequently-accessed code paths.

\clearpage
\phantomsection
\addcontentsline{toc}{subsection}{ANDCPERM -- Mask Capability Permissions}
\insnxeslabel{andcperm}
\subsection*{ANDCPERM -- Mask Capability Permissions}

\begin{x86opcodetable}
  \xopcode{NP 0F 0C \emph{/r}}{ANDCPERM \emph{r/mc, r64}}
  {MR}{Valid}{Valid}
  {Mask permissions of \emph{r/mc} by \emph{r64}.}
  \xopcode{37 /2 \emph{id}}{ANDCPERM \emph{r/mc, imm32}}
  {MI}{Valid}{Valid}
  {Mask permissions of \emph{r/mc} by sign-extended \emph{imm32}.}
\end{x86opcodetable}

\begin{x86opentable}
  \xopen{MR}{ModRM:r/m (r, w)}{ModRM:reg (r)}{NA}{NA}
  \xopen{MI}{ModRM:r/m (r,w)}{imm32}{NA}{NA}
\end{x86opentable}

\subsubsection*{Description}

Derives a new capability from the destination operand with the
\textbf{perms} and \textbf{uperms} field bitwise ANDed with the source
operand and stores the result in the destination operand.  The
destination operand can be a register or memory location; the source
operand can be a register or immediate.  If the destination operand is
sealed and tagged, set the destination operand to its original value
with the \textbf{tag} field cleared.

\subsubsection*{Flags Affected}

None

\clearpage
\phantomsection
\addcontentsline{toc}{subsection}{BUILDCAP - Construct Capability}
\insnxeslabel{buildcap}
\subsection*{BUILDCAP - Construct Capability}

\begin{x86opcodetable}
  \xopcode{VEX.LZ.0F.W0 0E \emph{/r}}{BUILDCAP \emph{rca, r/mc, rcb}}
  {RMV}{Valid}{Valid}
  {Construct capability from \emph{r/mc} and \emph{rcb} and store in
    \emph{rca}.}
\end{x86opcodetable}

\begin{x86opentable}
  \xopen{RMV}{ModRM:reg (w)}{ModRM:r/m (r)}{VEX.vvvv (r)}{NA}
\end{x86opentable}

\subsubsection*{Description}

Constructs a new capability equal to the second operand with the
\textbf{base}, \textbf{length}, \textbf{address}, \textbf{perms}, and
\textbf{uperms} fields replaced with the corresponding fields from the
third operand and stores the result in the first (destination) operand.
If the third operand is a sentry then the result is also sealed as a
sentry.  If the resulting capability is not a subset of the second
operand in bounds or permissions, or is not a legally-derivable
capability, or if the second operand did not have its \textbf{tag}
field set, or if the second operand was sealed as a non-sentry, the
resulting capability is set to the third operand with the \textbf{tag}
field cleared.

\subsubsection*{Flags Affected}

ZF is set to the \textbf{tag} field of the resulting capability.  The
CF, PF, AF, and OF flags are undefined.

\clearpage
\phantomsection
\addcontentsline{toc}{subsection}{CINVOKE -- Invoke Sealed Capability Pair}
\insnxeslabel{cinvoke}
\subsection*{CINVOKE -- Invoke Sealed Capability Pair}

\begin{x86opcodetable}
  \xopcode{EA \emph{/r}}{CINVOKE \emph{rc, r/mc}}
  {RM}{Valid}{Valid}
  {Set CAX to \emph{r/mc} and jump to \emph{rc}.}
\end{x86opcodetable}

\begin{x86opentable}
  \xopen{RM}{ModRM:reg (r)}{ModRM:r/m (r)}{NA}{NA}
\end{x86opentable}

\subsubsection*{Description}

Jumps to a pair of sealed capabilities.  The first source operand can
be a register; the second source operand can be a register or memory
location.

If both operands are sealed with the same \textbf{otype}, sets \CIP{}
to the unsealed first operand and sets \CAX{} to the unsealed second
operand.  Note that this control transfer is a jump and does not push
any values onto the stack.  If this instruction fails, it raises a
Capability Violation Fault (see
Section~\ref{sec:x86:capability-fault}).

\subsubsection*{Flags Affected}

None

\clearpage
\phantomsection
\addcontentsline{toc}{subsection}{CLCTAG - Clear Capability Tag}
\insnxeslabel{clctag}
\subsection*{CLCTAG - Clear Capability Tag}

\begin{x86opcodetable}
  \xopcode{0E /1}{CLCTAG \emph{r/mc}}
  {M}{Valid}{Valid}
  {Clear tag of \emph{r/mc}.}
\end{x86opcodetable}

\begin{x86opentable}
  \xopen{M}{ModRM:r/m (r, w)}{NA}{NA}{NA}
\end{x86opentable}

\subsubsection*{Description}

Clears the \textbf{tag} field of the destination operand.  The
destination operand can be a register or memory location.

\subsubsection*{Flags Affected}

None

\clearpage
\phantomsection
\addcontentsline{toc}{subsection}{CLCTAGS -- Clear Capability Tags}
\insnxeslabel{clctags}
\subsection*{CLCTAGS -- Clear Capability Tags}

\begin{x86opcodetable}
  \xopcode{0E /2}{CLCTAGS \emph{mcs}}
  {M}{Valid}{Valid}
  {Clear capability tags of \emph{mcs}.}
\end{x86opcodetable}

\begin{x86opentable}
  \xopen{M}{ModRM:r/m (w)}{NA}{NA}{NA}
\end{x86opentable}

\subsubsection*{Description}

Clears the capability tags for a stride of capabilities in memory
starting at the destination operand.  The authorizing capability for
the destination operand must include \cappermS{} and authorize access
to the entire stride of capabilities.  The address of the destination
operand must also be aligned to a stride of capabilities.

If this instruction fails, it raises either a Capability Violation
Fault (see Section~\ref{sec:x86:capability-fault}) or a Page Fault.

\subsubsection*{Flags Affected}

None

\clearpage
\phantomsection
\addcontentsline{toc}{subsection}{CPYTYPE - Construct Sealing Capability}
\insnxeslabel{cpytype}
\subsection*{CPYTYPE - Construct Sealing Capability}

\begin{x86opcodetable}
  \xopcode{VEX.LZ.66.0F.W0 0E \emph{/r}}{CPYTYPE \emph{rca, r/mc, rcb}}
  {RMV}{Valid}{Valid}
  {Construct a sealing capability from \emph{r/mc} and \emph{rcb} and
    store in \emph{rca}.}
\end{x86opcodetable}

\begin{x86opentable}
  \xopen{RMV}{ModRM:reg (w)}{ModRM:r/m (r)}{VEX.vvvv (r)}{NA}
\end{x86opentable}

\subsubsection*{Description}

Constructs a new capability equal to the second operand with the
\textbf{address} field set to the \textbf{otype} field of the third
operand and stores the result in the first (destination) operand.
If the third operand's \textbf{otype} field is reserved or if the
second operand was sealed, the \textbf{tag} in the resulting
capability is cleared.

\subsubsection*{Flags Affected}

ZF is set to the \textbf{tag} field of the resulting capability.  The
CF, PF, AF, and OF flags are undefined.

\clearpage
\phantomsection
\addcontentsline{toc}{subsection}{CRAM - Representable Alignment Mask}
\insnxeslabel{cram}
\subsection*{CRAM - Representable Alignment Mask}

\begin{x86opcodetable}
  \xopcode{1F \emph{/r}}{CRAM \emph{r64, r/m64}}
  {RM}{Valid}{Valid}
  {Set \emph{r64} to mask sufficient for aligned bounds spanning the
    length in \emph{r/m64}.}
\end{x86opcodetable}

\begin{x86opentable}
  \xopen{RM}{ModRM:reg (w)}{ModRM:r/m (r)}{NA}{NA}
\end{x86opentable}

\subsubsection*{Description}

Sets the destination operand to a mask that can be used to round
addresses down to a value that is sufficiently aligned to set exact
bounds for the nearest representable length of the source operand.
The source operand can be a register or memory location.

\subsubsection*{Flags Affected}

None

\clearpage
\phantomsection
\addcontentsline{toc}{subsection}{CRRL -- Round Representable Length}
\insnxeslabel{crrl}
\subsection*{CRRL -- Round Representable Length}

\begin{x86opcodetable}
  \xopcode{1E \emph{/r}}{CRRL \emph{r64, r/m64}}
  {RM}{Valid}{Valid}
  {Set \emph{r64} to minimum representable length greater or equal to
    \emph{r/m64}.}
\end{x86opcodetable}

\begin{x86opentable}
  \xopen{RM}{ModRM:reg (w)}{ModRM:r/m (r)}{NA}{NA}
\end{x86opentable}

\subsubsection*{Description}

Sets the destination operand to the smallest value greater or equal to
the source operand that can be used as a length to set exact bounds on
a capability with a suitably aligned base.  The source operand can be
a register or memory location.

\subsubsection*{Flags Affected}

None

\clearpage
\phantomsection
\addcontentsline{toc}{subsection}{CSEAL - Conditional Capability Seal}
\insnxeslabel{cseal}
\subsection*{CSEAL - Conditional Capability Seal}

\begin{x86opcodetable}
  \xopcode{VEX.LZ.F2.0F.W0 0E \emph{/r}}{CSEAL \emph{rca, r/mc, rcb}}
  {RMV}{Valid}{Valid}
  {Conditionally seal \emph{r/mc} with type from address field of
    \emph{rcb} and store in \emph{rca}.}
\end{x86opcodetable}

\begin{x86opentable}
  \xopen{RMV}{ModRM:reg (w)}{ModRM:r/m (r)}{VEX.vvvv (r)}{NA}
\end{x86opentable}

\subsubsection*{Description}

Seals the second operand with the \textbf{otype} equal to the
\textbf{address} field of the third operand and stores the result in
the first (destination) operand.  If the sealing operation fails, the
second operand is copied to the first operand.

\subsubsection*{Flags Affected}

The ZF flag is set to 1 if the sealing operation succeeds; otherwise
0.  The CF, PF, AF, SF, and OF flags are undefined.

\clearpage
\phantomsection
\addcontentsline{toc}{subsection}{GCBASE -- Get Capability Base}
\insnxeslabel{gcbase}
\subsection*{GCBASE -- Get Capability Base}

\begin{x86opcodetable}
  \xopcode{F2 0F 7A \emph{/r}}{GCBASE \emph{r64, r/mc}}
  {RM}{Valid}{Valid}
  {Store the base field of \emph{r/mc} in \emph{r64}.}
\end{x86opcodetable}

\begin{x86opentable}
  \xopen{RM}{ModRM:reg (w)}{ModRM:r/m (r)}{NA}{NA}
\end{x86opentable}

\subsubsection*{Description}

Sets the destination operand to the \textbf{base} field of the source
operand.  The source operand can be a register or memory location.

\subsubsection*{Flags Affected}

None

\clearpage
\phantomsection
\addcontentsline{toc}{subsection}{GCHI - Get Capability High Word}
\insnxeslabel{gchi}
\subsection*{GCHI - Get Capability High Word}

\begin{x86opcodetable}
  \xopcode{F2 0F 7B \emph{/r}}{GCHI \emph{r64, r/mc}}
  {RM}{Valid}{Valid}
  {Store the upper half of \emph{r/mc} in \emph{r64}.}
\end{x86opcodetable}

\begin{x86opentable}
  \xopen{RM}{ModRM:reg (w)}{ModRM:r/m (r)}{NA}{NA}
\end{x86opentable}

\subsubsection*{Description}

Sets the destination operand to the \textbf{high half} of the source
operand.  The source operand can be a register or memory location.

\subsubsection*{Flags Affected}

ZF is set to 1 if the result is zero.  The CF, PF, AF, SF, and OF
flags are undefined.

\clearpage
\phantomsection
\addcontentsline{toc}{subsection}{GCLEN - Get Capability Length}
\insnxeslabel{gclen}
\subsection*{GCLEN - Get Capability Length}

\begin{x86opcodetable}
  \xopcode{F3 0F 7A \emph{/r}}{GCLEN \emph{r64, r/mc}}
  {RM}{Valid}{Valid}
  {Store the bounds length of \emph{r/mc} in \emph{r64}.}
\end{x86opcodetable}

\begin{x86opentable}
  \xopen{RM}{ModRM:reg (w)}{ModRM:r/m (r)}{NA}{NA}
\end{x86opentable}

\subsubsection*{Description}

Sets the destination operand to the \textbf{length} field of the
source operand.  The source operand can be a register or memory
location.

\subsubsection*{Flags Affected}

ZF is set to 1 if the result is zero.  The CF, PF, AF, SF, and OF
flags are undefined.

\clearpage
\phantomsection
\addcontentsline{toc}{subsection}{GCLIM -- Get Capability Limit}
\insnxeslabel{gclim}
\subsection*{GCLIM -- Get Capability Limit}

\begin{x86opcodetable}
  \xopcode{F3 0F 7B \emph{/r}}{GCLIM \emph{r64, r/mc}}
  {RM}{Valid}{Valid}
  {Store the limit of \emph{r/mc} in \emph{r64}.}
\end{x86opcodetable}

\begin{x86opentable}
  \xopen{RM}{ModRM:reg (w)}{ModRM:r/m (r)}{NA}{NA}
\end{x86opentable}

\subsubsection*{Description}

Sets the destination operand to the limit (i.e. one past the last
addressable byte) of the source operand.  The source operand can be a
register or memory location.  If the source operand permits accessing
the last byte of the address space, the destination operand is set to
the value $2^{64}-1$.

\subsubsection*{Flags Affected}

None

\clearpage
\phantomsection
\addcontentsline{toc}{subsection}{GCOFF - Get Capability Offset}
\insnxeslabel{gcoff}
\subsection*{GCOFF - Get Capability Offset}

\begin{x86opcodetable}
  \xopcode{66 0F 7B \emph{/r}}{GCOFF \emph{r64, r/mc}}
  {RM}{Valid}{Valid}
  {Store the offset of \emph{r/mc} in \emph{r64}.}
\end{x86opcodetable}

\begin{x86opentable}
  \xopen{RM}{ModRM:reg (w)}{ModRM:r/m (r)}{NA}{NA}
\end{x86opentable}

\subsubsection*{Description}

Sets the destination operand to the \textbf{offset} field of the
source operand.  The source operand can be a register or memory
location.

\subsubsection*{Flags Affected}

ZF is set to 1 if the result is zero.  The CF, PF, AF, SF, and OF
flags are undefined.

\clearpage
\phantomsection
\addcontentsline{toc}{subsection}{GCPERM - Get Capability Permissions}
\insnxeslabel{gcperm}
\subsection*{GCPERM - Get Capability Permissions}

\begin{x86opcodetable}
  \xopcode{NP 0F 7A \emph{/r}}{GCPERM \emph{r64, r/mc}}
  {RM}{Valid}{Valid}
  {Store the permissions mask of \emph{r/mc} in \emph{r64}.}
\end{x86opcodetable}

\begin{x86opentable}
  \xopen{RM}{ModRM:reg (w)}{ModRM:r/m (r)}{NA}{NA}
\end{x86opentable}

\subsubsection*{Description}

Sets the destination operand to the combined \textbf{perms} and
\textbf{uperms} fields of the source operand.  The source operand can
be a register or memory location.

\subsubsection*{Flags Affected}

ZF is set to 1 if the result is zero.  The CF, PF, AF, SF, and OF
flags are undefined.

\clearpage
\phantomsection
\addcontentsline{toc}{subsection}{GCTAG -- Get Capability Tag}
\insnxeslabel{gctag}
\subsection*{GCTAG -- Get Capability Tag}

\begin{x86opcodetable}
  \xopcode{0E /3}{GCTAG \emph{r/mc}}
  {M}{Valid}{Valid}
  {Store the tag of \emph{r/mc} in ZF.}
\end{x86opcodetable}

\begin{x86opentable}
  \xopen{M}{ModRM:r/m (r)}{NA}{NA}{NA}
\end{x86opentable}

\subsubsection*{Description}

Sets ZF in \RFLAGS{} to the \textbf{tag} field of the source
operand.  The source operand can be a register or memory location.

\subsubsection*{Flags Affected}

ZF is set to 1 if the result is zero.  The CF, PF, AF, SF, and OF
flags are undefined.

\clearpage
\phantomsection
\addcontentsline{toc}{subsection}{GCTYPE - Get Capability Object Type}
\insnxeslabel{gctype}
\subsection*{GCTYPE - Get Capability Object Type}

\begin{x86opcodetable}
  \xopcode{66 0F 7A \emph{/r}}{GCTYPE \emph{r64, r/mc}}
  {RM}{Valid}{Valid}
  {Store the object type of \emph{r/mc} in \emph{r64}.}
\end{x86opcodetable}

\begin{x86opentable}
  \xopen{RM}{ModRM:reg (w)}{ModRM:r/m (r)}{NA}{NA}
\end{x86opentable}

\subsubsection*{Description}

Sets the destination operand to the \textbf{otype} field of the source
operand.  The source operand can be a register or memory location.

\subsubsection*{Flags Affected}

ZF is set to 1 if the result is zero.  SF is set to 1 if the result is
less than zero. The CF, PF, AF, and OF flags are undefined.

\clearpage
\phantomsection
\addcontentsline{toc}{subsection}{LCTAGS - Load Capability Tags}
\insnxeslabel{lctags}
\subsection*{LCTAGS - Load Capability Tags}

\begin{x86opcodetable}
  \xopcode{2F \emph{/r}}{LCTAGS \emph{r64, m8}}
  {RM}{Valid}{Valid}
  {Load bitmask of capability tags of \emph{m8} into \emph{r64}.}
\end{x86opcodetable}

\begin{x86opentable}
  \xopen{RM}{ModRM:Reg (w)}{ModRM:r/m (r)}{NA}{NA}
\end{x86opentable}

\subsubsection*{Description}

Loads a packed bitmask of capability tags for a stride of capabilities
in memory starting at the source operand and stores the bitmask in the
destination operand.  Bit 0 corresponds to tag for the capability at
the address of the source operand.  The authorizing capability for the
source operand must include \cappermL{} and \cappermLC{} and authorize
access to the entire stride of capabilities.  The source operand
address must also be aligned to a stride of capabilities.

If this instruction fails, it raises either a Capability Violation
Fault (see Section~\ref{sec:x86:capability-fault}) or a Page Fault.

\subsubsection*{Flags Affected}

ZF is set to 1 if the result is zero.  The CF, PF, AF, SF, and OF
flags are undefined.

\clearpage
\phantomsection
\addcontentsline{toc}{subsection}{MOV - Move to/from Additional Capability Registers}
\insnxeslabel{movcap}
\subsection*{MOV - Move to/from Additional Capability Registers}

\begin{x86opcodetable}
  \xopcode{0F 24 \emph{/r}}{MOV \emph{rc,} CFS/CGS/DDC}
  {MR}{Valid}{Valid}
  {Move additional capability register to \emph{rc}.}
  \xopcode{0F 25 \emph{/r}}{MOV CFS/CGS/DDC\emph{, rc}}
  {RM}{Valid}{Valid}
  {Move \emph{rc} to additional capability register.}
\end{x86opcodetable}

\begin{x86opentable}
  \xopen{MR}{ModRM:r/m (w)}{ModRM:reg (r)}{NA}{NA}
  \xopen{RM}{ModRM:reg (w)}{ModRM:r/m (r)}{NA}{NA}
\end{x86opentable}

\subsubsection*{Description}

Moves the contents of an additional capability register to a
general-purpose capability register or vice versa.

Similar to the \insnnoref{MOV} opcodes for control and debug
registers, the \textbf{reg} field of the ModRM byte always identifies
the additional capability register to read or write.  The \textbf{mod}
field of ModRM is ignored, and the \textbf{r/m} field identifies the
general-purpose capability register.  Attempts to reference invalid
additional capability registers will raise a UD\# exception.

Attempts to access additional capability registers other than \CFS{},
\CGS{}, or \DDC{} from a privilege level other than 0 will raise a
GP\#(0) exception.

\subsubsection*{Flags Affected}

None

\clearpage
\phantomsection
\addcontentsline{toc}{subsection}{SCADDR - Set Capability Address}
\insnxeslabel{scaddr}
\subsection*{SCADDR - Set Capability Address}

\begin{x86opcodetable}
  \xopcode{16 \emph{/r}}{SCADDR \emph{r/mc, r64}}
  {MR}{Valid}{Valid}
  {Set the address field of \emph{r/mc} to \emph{r64}.}
\end{x86opcodetable}

\begin{x86opentable}
  \xopen{MR}{ModRM:r/m (r, w)}{ModRM:reg (r)}{NA}{NA}
\end{x86opentable}

\subsubsection*{Description}

Sets the \textbf{address} field of the destination operand to the
source operand and stores the result in the destination operand.  The
destination operand can be a register or memory location; the source
operand can be a register.  If the destination operand is sealed and
tagged, the destination operand is set to its original value with the
\textbf{tag} field cleared.

If the new value of the \textbf{address} field makes the resulting
capability unrepresentable, the \textbf{tag} field in the resulting
capability is cleared.

\subsubsection*{Flags Affected}

None

\clearpage
\phantomsection
\addcontentsline{toc}{subsection}{SCBND -- Set Capability Bounds}
\insnxeslabel{scbnd}
\subsection*{SCBND -- Set Capability Bounds}

\begin{x86opcodetable}
  \xopcode{17 \emph{/r}}{SCBND \emph{r/mc, r64}}
  {MR}{Valid}{Valid}
  {Set bounds of \emph{r/mc} to \emph{r64}.}
  \xopcode{37 /0 \emph{id}}{SCBND \emph{r/mc, imm32}}
  {MI}{Valid}{Valid}
  {Set bounds of \emph{r/mc} to zero-extended \emph{imm32}.}
\end{x86opcodetable}

\begin{x86opentable}
  \xopen{MR}{ModRM:r/m (r, w)}{ModRM:reg (r)}{NA}{NA}
  \xopen{MI}{ModRM:r/m (r, w)}{imm32}{NA}{NA}
\end{x86opentable}

\subsubsection*{Description}

Derives a new capability from the destination operand and source
operand and then store the result in the destination operand.  The
destination operand can be a register or memory location; the source
operand can be an immediate or a register.  When an immediate value is
used as an operand, it is zero-extended.

The new capability's \textbf{base} field is set to the current
\textbf{address} field of the destination operand, and the
\textbf{length} field is set to the source operand.  If the resulting
capability cannot be represented exactly, the \textbf{base} will be
rounded down and the \textbf{length} will be rounded up by the
smallest amount needed to form a representable capability covering the
requested bounds.  If the original capability is sealed or the bounds
of the new capability exceed the original capability, the \textbf{tag}
field in the resulting capability is cleared.

\subsubsection*{Flags Affected}

The ZF flag is set to the value of the \textbf{tag} field in the
resulting capability.  The CF, PF, AF, SF, and OF flags are undefined.

\clearpage
\phantomsection
\addcontentsline{toc}{subsection}{SCBNDE -- Set Exact Capability Bounds}
\insnxeslabel{scbnde}
\subsection*{SCBNDE -- Set Exact Capability Bounds}

\begin{x86opcodetable}
  \xopcode{27 \emph{/r}}{SCBNDE \emph{r/mc, r64}}
  {MR}{Valid}{Valid}
  {Set bounds of \emph{r/mc} to \emph{r64}.}
  \xopcode{37 /1 \emph{id}}{SCBNDE \emph{r/mc, imm32}}
  {MI}{Valid}{Valid}
  {Set bounds of \emph{r/mc} to zero-extended \emph{imm32}.}
\end{x86opcodetable}

\begin{x86opentable}
  \xopen{MR}{ModRM:r/m (r, w)}{ModRM:reg (r)}{NA}{NA}
  \xopen{MI}{ModRM:r/m (r, w)}{imm32}{NA}{NA}
\end{x86opentable}

\subsubsection*{Description}

Derives a new capability from the destination operand and source
operand and then store the result in the destination operand.  The
destination operand can be a register or memory location; the source
operand can be an immediate or a register.  When an immediate value is
used as an operand, it is zero-extended.

The new capability's \textbf{base} field is set to the current
\textbf{address} field of the destination operand, and the
\textbf{length} field is set to the source operand.  If the resulting
capability cannot be represented exactly, the \textbf{tag} field will
be cleared, the \textbf{base} will be rounded down, and the
\textbf{length} will be rounded up by the smallest amount needed to
form a representable capability covering the requested bounds.  If the
original capability is sealed or the bounds of the new capability
exceed the original capability, the \textbf{tag} field in the
resulting capability is cleared.

\subsubsection*{Flags Affected}

The ZF flag is set to the value of the \textbf{tag} field in the
resulting capability.  The CF, PF, AF, SF, and OF flags are undefined.

\clearpage
\phantomsection
\addcontentsline{toc}{subsection}{SCHI - Set Capability High Half}
\insnxeslabel{schi}
\subsection*{SCHI - Set Capability High Half}

\begin{x86opcodetable}
  \xopcode{NP 0F 0A \emph{/r}}{SCHI \emph{r/mc, r64}}
  {MR}{Valid}{Valid}
  {Set high half of \emph{r/mc} to \emph{r64}.}
\end{x86opcodetable}

\begin{x86opentable}
  \xopen{MR}{ModRM:r/m (r, w)}{ModRM:reg (r)}{NA}{NA}
\end{x86opentable}

\subsubsection*{Description}

Sets the \textbf{high half} of the destination operand to the source
operand and stores the result in the destination operand.  The
destination operand can be a register or memory location; the source
operand can be a register.  The \textbf{tag} field in the destination
operand is cleared.

\subsubsection*{Flags Affected}

None

\clearpage
\phantomsection
\addcontentsline{toc}{subsection}{SCOFF -- Set Capability Offset}
\insnxeslabel{scoff}
\subsection*{SCOFF -- Set Capability Offset}

\begin{x86opcodetable}
  \xopcode{66 0F 0C \emph{/r}}{SCOFF \emph{r/mc, r64}}
  {MR}{Valid}{Valid}
  {Set offset of \emph{r/mc} to \emph{r64}.}
\end{x86opcodetable}

\begin{x86opentable}
  \xopen{MR}{ModRM:r/m (r, w)}{ModRM:reg (r)}{NA}{NA}
\end{x86opentable}

\subsubsection*{Description}

Sets the \textbf{offset} field of the destination operand to the
source operand and stores the result in the destination operand.  The
destination operand can be a register or memory location; the source
operand can be a register.  If the destination operand is sealed and
tagged, the destination operand is set to its original value with the
\textbf{tag} field cleared.

If the new value of the \textbf{offset} field makes the resulting
capability unrepresentable, the \textbf{tag} field in the resulting
capability is cleared.

\subsubsection*{Flags Affected}

None

\clearpage
\phantomsection
\addcontentsline{toc}{subsection}{SEAL - Seal Capability}
\insnxeslabel{seal}
\subsection*{SEAL - Seal Capability}

\begin{x86opcodetable}
  \xopcode{F2 0F 0C \emph{/r}}{SEAL \emph{r/mc, rc}}
  {MR}{Valid}{Valid}
  {Seal \emph{r/mc} with type from the address field of \emph{rc}.}
\end{x86opcodetable}

\begin{x86opentable}
  \xopen{MR}{ModRM:r/m (r, w)}{ModRM:reg (r)}{NA}{NA}
\end{x86opentable}

\subsubsection*{Description}

Seals the destination operand with the \textbf{otype} equal to the
\textbf{address} field of the source operand and stores the result in
the destination operand.  If the sealing operation fails, the
destination operand is set to the original value of the destination
operand with the \textbf{tag} field cleared.

\subsubsection*{Flags Affected}

The ZF flag is set to 1 if the sealing operation succeeds; otherwise
0.  The CF, PF, AF, SF, and OF flags are undefined.

\clearpage
\phantomsection
\addcontentsline{toc}{subsection}{SENTRY -- Seal Capability as a Sentry}
\insnxeslabel{sentry}
\subsection*{SENTRY -- Seal Capability as a Sentry}

\begin{x86opcodetable}
  \xopcode{0E /0}{SENTRY \emph{r/mc}}
  {M}{Valid}{Valid}
  {Seal \emph{r/mc} as a sentry.}
\end{x86opcodetable}

\begin{x86opentable}
  \xopen{M}{ModRM:r/m (r, w)}{NA}{NA}{NA}
\end{x86opentable}

\subsubsection*{Description}

Seals the destination operand as a sentry capability.  If the sealing
operation fails, leave the destination operand unchanged.

\subsubsection*{Flags Affected}

The ZF flag is set to 1 if the sealing operation succeeds; otherwise
0.  The CF, PF, AF, SF, and OF flags are undefined.

\clearpage
\phantomsection
\addcontentsline{toc}{subsection}{UNSEAL - Unseal Capability}
\insnxeslabel{unseal}
\subsection*{UNSEAL - Unseal Capability}

\begin{x86opcodetable}
  \xopcode{F3 0F 0C \emph{/r}}{UNSEAL \emph{r/mc, rc}}
  {MR}{Valid}{Valid}
  {Unseal \emph{r/mc} using \emph{rc} as the authority.}
\end{x86opcodetable}

\begin{x86opentable}
  \xopen{MR}{ModRM:r/m (r, w)}{ModRM:reg (r)}{NA}{NA}
\end{x86opentable}

\subsubsection*{Description}

Unseals the destination operand using the source operand as the
unsealing authority and stores the result in the destination operand.
If the unsealing operation fails, the destination operand is set to
the original value of the destination operand with the \textbf{tag}
field cleared.

\subsubsection*{Flags Affected}

The ZF flag is set to 1 if the unsealing operation succeeds; otherwise
0.  The CF, PF, AF, SF, and OF flags are undefined.


\clearpage
\section{Summary of New Opcodes}

The following new opcodes are added in 64-bit mode and are also
available in capability mode.

\bigskip
\noindent
\begin{tabular}{| l | l |} \hline
  \textbf{Opcode} & \textbf{Instruction}\\
  \hline
  0E /0 & \insnxesref{SENTRY} \emph{r/mc}\\
  \hline
  0E /1 & \insnxesref{CLCTAG} \emph{r/mc}\\
  \hline
  0E /2 & \insnxesref{CLCTAGS} \emph{m8}\\
  \hline
  0E /3 & \insnxesref{GCTAG} \emph{r/mc}\\
  \hline
  16 \emph{/r} & \insnxesref{SCADDR} \emph{r/mc, r64}\\
  \hline
  17 \emph{/r} & \insnxesref{SCBND} \emph{r/mc, r64}\\
  \hline
  1E \emph{/r} & \insnxesref{CRRL} \emph{r64, r/m64}\\
  \hline
  1F \emph{/r} & \insnxesref{CRAM} \emph{r64, r/m64}\\
  \hline
  27 \emph{/r} & \insnxesref{SCBNDE} \emph{r/mc, r64}\\
  \hline
  2F \emph{/r} & \insnxesref{LCTAGS} \emph{r64, m8}\\
  \hline
  37 /0 \emph{id} & \insnxesref{SCBND} \emph{r/mc, imm32}\\
  \hline
  37 /1 \emph{id} & \insnxesref{SCBNDE} \emph{r/mc, imm32}\\
  \hline
  37 /2 \emph{id} & \insnxesref{ANDCPERM} \emph{r/mc, imm32}\\
  \hline
  EA \emph{/r} & \insnxesref{CINVOKE} \emph{rc, r/mc}\\
  \hline
  NP 0F 0A \emph{/r} & \insnxesref{SCHI} \emph{r/mc, r64}\\
  \hline
  NP 0F 0C \emph{/r} & \insnxesref{ANDCPERM} \emph{r/mc, r64}\\
  \hline
  66 0F 0C \emph{/r} & \insnxesref{SCOFF} \emph{r/mc, r64}\\
  \hline
  F2 0F 0C \emph{/r} & \insnxesref{SEAL} \emph{r/mc, rc}\\
  \hline
  F3 0F 0C \emph{/r} & \insnxesref{UNSEAL} \emph{r/mc, rc}\\
  \hline
  VEX.LZ.0F.W0 0E \emph{/r} & \insnxesref{BUILDCAP} \emph{rca, r/mc, rcb}\\
  \hline
  VEX.LZ.66.0F.W0 0E \emph{/r} & \insnxesref{CPYTYPE} \emph{rca, r/mc, rcb}\\
  \hline
  VEX.LZ.F2.0F.W0 0E \emph{/r} & \insnxesref{CSEAL} \emph{rca, r/mc, rcb}\\
  \hline
  0F 24 \emph{/r} & \insnxesref[movcap]{MOV} \emph{rc,} CFS/CGS/DDC\\
  \hline
  0F 25 \emph{/r} & \insnxesref[movcap]{MOV} CFS/CGS/DDC\emph{, rc}\\
  \hline
  NP 0F 7A \emph{/r} & \insnxesref{GCPERM} \emph{r64, r/mc}\\
  \hline
  66 0F 7A \emph{/r} & \insnxesref{GCTYPE} \emph{r64, r/mc}\\
  \hline
  F2 0F 7A \emph{/r} & \insnxesref{GCBASE} \emph{r64, r/mc}\\
  \hline
  F3 0F 7A \emph{/r} & \insnxesref{GCLEN} \emph{r64, r/mc}\\
  \hline
  66 0F 7B \emph{/r} & \insnxesref{GCOFF} \emph{r64, r/mc}\\
  \hline
  F2 0F 7B \emph{/r} & \insnxesref{GCHI} \emph{r64, r/mc}\\
  \hline
  F3 0F 7B \emph{/r} & \insnxesref{GCLIM} \emph{r64, r/mc}\\
  \hline
\end{tabular}

\clearpage
\section{Instructions Deprecated in Capability Mode}

The following instructions are valid in 64-bit mode but invalid in
capability mode.  For conciseness, the placeholder \texttt{XX} in the
table below indicates any valid size for the instruction (i.e. 8, 16, 32,
or 64).

\bigskip
\noindent
\begin{tabularx}{\textwidth}{| l | l | Z |} \hline
  \textbf{Opcode} & \textbf{Instruction} &
  \textbf{Description}\\
  \hline
  A0 & MOV AL,\emph{moffs8} & Move byte at (\emph{offset}) to AL.\\
  \hline
  A1 & MOV [ER]AX,\emph{moffsxx} & Move value at (\emph{offset}) to [ER]AX.\\
  \hline
  A2 & MOV \emph{moffs8},AL & Move AL to  (\emph{offset}).\\
  \hline
  A3 & MOV \emph{moffs8},[ER]AX & Move [ER]AX to  (\emph{offset}).\\
  \hline
  CA & RET \emph{imm16} &
  Far return and pop \emph{imm16} bytes from stack.\\
  \hline
  CB & RET & Far return.\\
  \hline
  FF /3 & CALL \emph{m16:xx} &
  Call far, absolute indirect, address given in \emph{m16:xx}.\\
  \hline
  FF /5 & JMP \emph{m16:xx} &
  Jump far, absolute indirect, address given in \emph{m16:xx}.\\
  \hline
  0F 02 \emph{/r} & LAR \emph{rxx,rxx/m16} &
  Load access rights.\\
  \hline
  0F 03 \emph{/r} & LSL \emph{rxx,rxx/m16} &
  Load segment limit.\\
  \hline
  0F A0 & PUSH FS & Push FS.\\
  \hline
  0F A1 & POP FS & Pop FS.\\
  \hline
  0F A8 & PUSH GS & Push GS.\\
  \hline
  0F A9 & POP GS & Pop GS.\\
  \hline
  0F B2 \emph{/r} & LSS \emph{rxx,m16:xx} &
  Load SS:\emph{rxx} with far pointer.\\
  \hline
  0F B4 \emph{/r} & LFS \emph{rxx,m16:xx} &
  Load FS:\emph{rxx} with far pointer.\\
  \hline
  0F B5 \emph{/r} & LGS \emph{rxx,m16:xx} &
  Load GS:\emph{rxx} with far pointer.\\
  \hline
\end{tabularx}
