\chapter{The CHERI-RISC-V Instruction-Set Reference}
\label{chap:isaref-riscv}

\ifcsname @def@riscv@insns@tex\endcsname
  \ea\endinput
\fi
\ea\gdef\csname @def@riscv@insns@tex\endcsname{1}

\input{def-riscv-insns-macros}

\rvcherisubclass{top}{0}{Two Source \& Dest}

\rvcherisubclass{srcsrcdest}{7C}{Stores}
\rvcherisubclass{srcsrcdest}{7D}{Loads}
\rvcherisubclass{srcsrcdest}{7E}{Two Source}
\rvcherisubclass{srcsrcdest}{7F}{Source \& Dest}

\rvcherisubclass{srcsrc}{1F}{One Source}

\rvcherisubclass{srcdest}{1F}{Dest-Only}

\rvcherisrcdest[name=CGetPerm]{0}{rd}{cs1}
\rvcherisrcdest[name=CGetType]{1}{rd}{cs1}
\rvcherisrcdest[name=CGetBase]{2}{rd}{cs1}
\rvcherisrcdest[name=CGetLen]{3}{rd}{cs1}
\rvcherisrcdest[name=CGetTag]{4}{rd}{cs1}
\rvcherisrcdest[name=CGetSealed]{5}{rd}{cs1}
\rvcherisrcdest[name=CGetOffset]{6}{rd}{cs1}
\rvcherisrcdest[name=CGetFlags]{7}{rd}{cs1}
\rvcherisrcdest[name=CGetAddr,noref,tablesuffix=\rvcheriremovedfootnotemark]{F}{rd}{cs1}
\rvcherisrcdest[name=CGetTop]{18}{rd}{cs1}
\rvcherisrcdest[name=CGetHigh]{17}{rd}{cs1}

\rvcherisrcsrcdest[name=CSeal]{B}{cd}{cs1}{cs2}
\rvcherisrcsrcdest[name=CUnseal]{C}{cd}{cs1}{cs2}
\rvcherisrcsrcdest[name=CAndPerm]{D}{cd}{cs1}{rs2}
\rvcherisrcsrcdest[name=CSetFlags]{E}{cd}{cs1}{rs2}
\rvcherisrcsrcdest[name=CSetOffset]{F}{cd}{cs1}{rs2}
\rvcherisrcsrcdest[name=CSetAddr]{10}{cd}{cs1}{rs2}
\rvcherisrcsrcdest[name=CIncOffset]{11}{cd}{cs1}{rs2}
\rvcherisrcsrcdestimm[name=CIncOffsetImm]{1}{cd}{cs1}{imm}
\rvcherisrcsrcdest[name=CSetBounds]{8}{cd}{cs1}{rs2}
\rvcherisrcsrcdest[name=CSetBoundsExact]{9}{cd}{cs1}{rs2}
\rvcherisrcsrcdestimm[name=CSetBoundsImm]{2}{cd}{cs1}{uimm}
\rvcherisrcsrcdest[name=CSetHigh]{16}{cd}{cs1}{rs2}
\rvcherisrcdest[name=CClearTag]{B}{cd}{cs1}
\rvcherisrcsrcdest[name=CBuildCap]{1D}{cd}{cs1}{cs2}
\rvcherisrcsrcdest[name=CCopyType]{1E}{cd}{cs1}{cs2}
\rvcherisrcsrcdest[name=CCSeal]{1F}{cd}{cs1}{cs2}
\rvcherisrcdest[name=CSealEntry]{11}{cd}{cs1}
\rvcherisrcsrcdest[name=CRelocate,noref,tablesuffix=\rvcherireservedfootnotemark]{15}{cd}{cs1}{rs2}

\rvcherisrcsrcdest[name=CToPtr,tablesuffix=\rvcherideprecatedfootnotemark]{12}{rd}{cs1}{cs2}
\rvcherisrcsrcdest[name=CFromPtr,tablesuffix=\rvcherideprecatedfootnotemark]{13}{cd}{cs1}{rs2}
\rvcherisrcsrcdest[name=CSub,noref,tablesuffix=\rvcheriremovedfootnotemark]{14}{rd}{cs1}{cs2}
\rvcherisrcdest[name=CMove]{A}{cd}{cs1}

\rvcherisrcsrcdest[name=CTestSubset]{20}{rd}{cs1}{cs2}
\rvcherisrcsrcdest[name=CSetEqualExact,shortname=CSEQX]{21}{rd}{cs1}{cs2}

\rvcherisrcdest[name=JALR.CAP]{C}{cd}{cs1}
\rvcherisrcdest[name=JALR.PCC]{14}{rd}{rs1}
\rvcherisrcsrc[name=CInvoke]{1}{cs1}{cs2}

\rvcheriscr[name=CSpecialRW]{1}{cd}{scr}{cs1}

\rvchericlear[name=CClear]{E}
\rvchericlear[name=FPClear]{10}

\rvcherisrcdest[name=CRoundRepresentableLength,shortname=CRRL]{8}{rd}{rs1}
\rvcherisrcdest[name=CRepresentableAlignmentMask,shortname=CRAM]{9}{rd}{rs1}

\rvcherisrcdest[name=CLoadTags]{12}{rd}{cs1}
\rvcherisrc[name=CClearTags,noref,tablesuffix=\rvcherireservedfootnotemark]{0}{cs1}

\rvcherisrcdest[name=CGetVersion,noref,tablesuffix=\rvcherireservedfootnotemark]{13}{rd}{cs1}
\rvcherisrcsrcdest[name=CSetVersion,noref,tablesuffix=\rvcherireservedfootnotemark]{2}{cd}{cs1}{rs2}
\rvcheristorever[name=CStoreVersion,noref,tablesuffix=\rvcherireservedfootnotemark]{2}{rs2}{cs1}
\rvcherisrcdest[name=CLoadVersion,noref,tablesuffix=\rvcherireservedfootnotemark]{16}{rd}{cs1}
\rvcherisrcdest[name=CLoadVersions,noref,tablesuffix=\rvcherireservedfootnotemark]{15}{rd}{cs1}
\rvcherisrcsrcdest[name=CAmoCDecVersion,noref,tablesuffix=\rvcherireservedfootnotemark]{3}{rd}{cs2}{cs1}

\rvcheriexpload[name=LB.DDC]{00}{rd}{rs1}
\rvcheriexpload[name=LH.DDC]{01}{rd}{rs1}
\rvcheriexpload[name=LW.DDC]{02}{rd}{rs1}
\rvcheriexpload[name=LC.DDC,restriction=RV32,notable]{03}{cd}{rs1}
\rvcheriexpload[name=LD.DDC,restriction=RV64/128]{03}{rd}{rs1}
\rvcheriexpload[name=LC.DDC,restriction=RV64,notable]{17}{cd}{rs1}
\rvcheriexpload[name=LQ.DDC,restriction=RV128,noref]{17}{rd}{rs1}
\rvcheriexpload[name=LBU.DDC]{04}{rd}{rs1}
\rvcheriexpload[name=LHU.DDC]{05}{rd}{rs1}
\rvcheriexpload[name=LWU.DDC,restriction=RV64/128]{06}{rd}{rs1}
\rvcheriexpload[name=LDU.DDC,restriction=RV128,noref,tablesuffix=\rvcherildufootnotemark]{07}{rd}{rs1}

\rvcheriexpload[name=LB.CAP]{08}{rd}{cs1}
\rvcheriexpload[name=LH.CAP]{09}{rd}{cs1}
\rvcheriexpload[name=LW.CAP]{0A}{rd}{cs1}
\rvcheriexpload[name=LC.CAP,restriction=RV32,notable]{0B}{cd}{cs1}
\rvcheriexpload[name=LD.CAP,restriction=RV64/128]{0B}{rd}{cs1}
\rvcheriexpload[name=LC.CAP,restriction=RV64,notable]{1F}{cd}{cs1}
\rvcheriexpload[name=LQ.CAP,restriction=RV128,noref]{1F}{rd}{cs1}
\rvcheriexpload[name=LBU.CAP]{0C}{rd}{cs1}
\rvcheriexpload[name=LHU.CAP]{0D}{rd}{cs1}
\rvcheriexpload[name=LWU.CAP,restriction=RV64/128]{0E}{rd}{cs1}
\rvcheriexpload[name=LDU.CAP,restriction=RV128,noref,tablesuffix=\rvcherildufootnotemark]{0F}{rd}{cs1}

\rvcheriexploadres[name=LR.B.DDC,noref,tablesuffix=\rvcheriatomicfootnotemark]{10}{rd}{rs1}
\rvcheriexploadres[name=LR.H.DDC,noref,tablesuffix=\rvcheriatomicfootnotemark]{11}{rd}{rs1}
\rvcheriexploadres[name=LR.W.DDC,noref,tablesuffix=\rvcheriatomicfootnotemark]{12}{rd}{rs1}
\rvcheriexploadres[name=LR.C.DDC,restriction=RV32,noref,tablesuffix=\rvcheriatomicfootnotemark,notable]{13}{cd}{rs1}
\rvcheriexploadres[name=LR.D.DDC,restriction=RV64/128,noref,tablesuffix=\rvcheriatomicfootnotemark]{13}{rd}{rs1}
\rvcheriexploadres[name=LR.C.DDC,restriction=RV64,noref,tablesuffix=\rvcheriatomicfootnotemark,notable]{14}{cd}{rs1}
\rvcheriexploadres[name=LR.Q.DDC,restriction=RV128,noref,tablesuffix=\rvcheriatomicfootnotemark]{14}{rd}{rs1}

\rvcheriexploadres[name=LR.B.CAP,noref,tablesuffix=\rvcheriatomicfootnotemark]{18}{rd}{cs1}
\rvcheriexploadres[name=LR.H.CAP,noref,tablesuffix=\rvcheriatomicfootnotemark]{19}{rd}{cs1}
\rvcheriexploadres[name=LR.W.CAP,noref,tablesuffix=\rvcheriatomicfootnotemark]{1A}{rd}{cs1}
\rvcheriexploadres[name=LR.C.CAP,restriction=RV32,noref,tablesuffix=\rvcheriatomicfootnotemark,notable]{1B}{cd}{cs1}
\rvcheriexploadres[name=LR.D.CAP,restriction=RV64/128,noref,tablesuffix=\rvcheriatomicfootnotemark]{1B}{rd}{cs1}
\rvcheriexploadres[name=LR.C.CAP,restriction=RV64,noref,tablesuffix=\rvcheriatomicfootnotemark,notable]{1C}{cd}{cs1}
\rvcheriexploadres[name=LR.Q.CAP,restriction=RV128,noref,tablesuffix=\rvcheriatomicfootnotemark]{1C}{rd}{cs1}

\rvcheriexpstore[name=SB.DDC]{00}{rs2}{rs1}
\rvcheriexpstore[name=SH.DDC]{01}{rs2}{rs1}
\rvcheriexpstore[name=SW.DDC]{02}{rs2}{rs1}
\rvcheriexpstore[name=SC.DDC,restriction=RV32,notable]{03}{cs2}{rs1}
\rvcheriexpstore[name=SD.DDC,restriction=RV64/128]{03}{rs2}{rs1}
\rvcheriexpstore[name=SC.DDC,restriction=RV64,notable]{04}{cs2}{rs1}
\rvcheriexpstore[name=SQ.DDC,restriction=RV128,noref]{04}{rs2}{rs1}

\rvcheriexpstore[name=SB.CAP]{08}{rs2}{cs1}
\rvcheriexpstore[name=SH.CAP]{09}{rs2}{cs1}
\rvcheriexpstore[name=SW.CAP]{0A}{rs2}{cs1}
\rvcheriexpstore[name=SC.CAP,restriction=RV32,notable]{0B}{cs2}{cs1}
\rvcheriexpstore[name=SD.CAP,restriction=RV64/128]{0B}{rs2}{cs1}
\rvcheriexpstore[name=SC.CAP,restriction=RV64,notable]{0C}{cs2}{cs1}
\rvcheriexpstore[name=SQ.CAP,restriction=RV128,noref]{0C}{rs2}{cs1}

\rvcheriexpstorecond[name=SC.B.DDC,noref,tablesuffix=\rvcheriatomicfootnotemark]{10}{rd}{rs2}{rs1}
\rvcheriexpstorecond[name=SC.H.DDC,noref,tablesuffix=\rvcheriatomicfootnotemark]{11}{rd}{rs2}{rs1}
\rvcheriexpstorecond[name=SC.W.DDC,noref,tablesuffix=\rvcheriatomicfootnotemark]{12}{rd}{rs2}{rs1}
\rvcheriexpstorecond[name=SC.C.DDC,restriction=RV32,noref,tablesuffix=\rvcheriatomicfootnotemark,notable]{13}{cd}{cs2}{rs1}
\rvcheriexpstorecond[name=SC.D.DDC,restriction=RV64/128,noref,tablesuffix=\rvcheriatomicfootnotemark]{13}{rd}{rs2}{rs1}
\rvcheriexpstorecond[name=SC.C.DDC,restriction=RV64,noref,tablesuffix=\rvcheriatomicfootnotemark,notable]{14}{cd}{cs2}{rs1}
\rvcheriexpstorecond[name=SC.Q.DDC,restriction=RV128,noref,tablesuffix=\rvcheriatomicfootnotemark]{14}{rd}{rs2}{rs1}

\rvcheriexpstorecond[name=SC.B.CAP,noref,tablesuffix=\rvcheriatomicfootnotemark]{18}{rd}{rs2}{cs1}
\rvcheriexpstorecond[name=SC.H.CAP,noref,tablesuffix=\rvcheriatomicfootnotemark]{19}{rd}{rs2}{cs1}
\rvcheriexpstorecond[name=SC.W.CAP,noref,tablesuffix=\rvcheriatomicfootnotemark]{1A}{rd}{rs2}{cs1}
\rvcheriexpstorecond[name=SC.C.CAP,restriction=RV32,noref,tablesuffix=\rvcheriatomicfootnotemark,notable]{1B}{cd}{cs2}{cs1}
\rvcheriexpstorecond[name=SC.D.CAP,restriction=RV64/128,noref,tablesuffix=\rvcheriatomicfootnotemark]{1B}{rd}{rs2}{cs1}
\rvcheriexpstorecond[name=SC.C.CAP,restriction=RV64,noref,tablesuffix=\rvcheriatomicfootnotemark,notable]{1C}{cd}{cs2}{cs1}
\rvcheriexpstorecond[name=SC.Q.CAP,restriction=RV128,noref,tablesuffix=\rvcheriatomicfootnotemark]{1C}{rd}{rs2}{cs1}

\def\rvcheriasminsnref#1{#1}
\def\rvcheriasminsnnoref#1{#1}
\providecommand{\rvcheriasmfmt}{}
\renewcommand{\rvcheriasmfmt}[2][]{%
  #2%
  \ifthenelse{\equal{#1}{}}{%
  }{%
    ~{\textit{\footnotesize{(#1)}}}%
  }%
}

In this chapter, we specify each instruction via both informal descriptions
and code in the Sail language.
To allow for more succinct code descriptions, we rely on a number of
common function definitions and constants also described in this chapter.

\section{Sail language used in instruction descriptions}
The instruction descriptions contained in this chapter are accompanied
by code in the Sail language taken from the Sail
CHERI-RISC-V implementation~\cite{sail-cheri-riscv}.
Sail is a domain specific imperative language designed for describing
processor architectures.  It has a compiler that can output executable
code in OCaml or C for building executable models, and can also
translate to various theorem prover languages for automated reasoning
about the ISA.
A brief description of the Sail language features used in this chapter can be
found in \cref{chap:sail}.
For a full description see the Sail language documentation~\cite{sail-url}.

\section{Constant Definitions}
The following constants are used in various type and function definitions throughout the specification.
The concrete values listed here apply to the CHERI-RISC-V ISA with
128-bit capabilities that extends the base 64-bit RISC-V ISA.
The constants for with 64-bit CHERI capabilities (extending the 32-bit RISC-V ISA)
can be found in the CHERI-RISC-V Sail model~\cite{sail-cheri-riscv}.
\arnote{and are not listed here as they may change in the near future?}

\medskip
% Note: we can use \sailRISCVtype{foo\_bar} instead after rems-project/sail#100
\phantomsection\label{sailRISCVzxlen}
\sailRISCVtype{xlen}
\sailRISCVtypecapAddrWidth{}
\sailRISCVtypecapLenWidth{}
\sailRISCVtypecapSizze{}
\sailRISCVtypecapMantissaWidth{}

% FIXME: These extra labels are required to allow markdown saildoc references such as [cap_uperms_width] to work correctly.
\phantomsection\label{sailRISCVzcapzyhpermszywidth}
\sailRISCVtypecapHpermsWidth{}
\phantomsection\label{sailRISCVzcapzyupermszywidth}
\sailRISCVtypecapUpermsWidth{}
\phantomsection\label{sailRISCVzcapzyupermszyshift}
\sailRISCVtypecapUpermsShift{}
\phantomsection\label{sailRISCVzcapzyflagszywidth}
\sailRISCVtypecapFlagsWidth{}
\phantomsection\label{sailRISCVzcapzyotypezywidth}
\sailRISCVtypecapOtypeWidth{}

\phantomsection\label{sailRISCVzcapzymaxzyotype}
\sailRISCVletcapMaxOtype{}

\section{Function Definitions}

This section contains descriptions of convenience functions used by the Sail code featured in this chapter.

\subsection*{Functions for integer and bit vector manipulation}

The following functions convert between bit vectors and integers and manipulate bit vectors:

\medskip
\sailRISCVval{unsigned}
\sailRISCVval{signed}
\sailRISCVval{to\_bits}
% Technically this is not a vector, but it seems appropriate to put these together
\sailRISCVval{bool\_to\_bit}
\sailRISCVval{bool\_to\_bits}
\sailRISCVval{truncate}
\sailRISCVval{pow2}

% The following are overloads so we can't easily use the generated latex
% Hacky hspace to get rid of unwanted hangindent

\phantomsection
\label{sailRISCVzEXTZ}
\saildocval{Adds zeros in most significant bits of vector to obtain a vector of desired length.}{\hspace{-\parindent}\isail{EXTZ}}

\label{sailRISCVzEXTS}
\saildocval{Extends the most significant bits of vector preserving the sign bit.}{\hspace{-\parindent}\isail{EXTS}}

\label{sailRISCVzzzeros}
\saildocval{Produces a bit vector of all zeros}{\hspace{-\parindent}\isail{zeros}}

\label{sailRISCVzzones}
\saildocval{Produces a bit vector of all ones}{\hspace{-\parindent}\isail{ones}}

\subsection*{Types used in function definitions}

\sailRISCVtype{CapBits}
\sailRISCVtype{CapAddrBits}
\sailRISCVtype{CapLenBits}
\sailRISCVtype{CapPermsBits}

% Expanding this looks very ugly \sailRISCVtype{Capability}
\medskip
\noindent
Many functions also use \isail{struct Capability}, a structure holding a
partially-decompressed representation of CHERI capabilities.
%
The following functions can be used to convert between the structure
representation and the raw capability bits:

\medskip%
\sailRISCVval{capBitsToCapability}
\sailRISCVval{capToBits}

\subsection*{Functions for reading and writing register and memory}

\arnote{TODO: document X() and C()}
%\label{sailRISCVzC}
\begin{lstlisting}[language=sail,label=sailRISCVzC]
C(n) : regno -> Capability
C(n) : (regno, Capability) -> unit
\end{lstlisting}
The overloaded function \isail{C(n)} is used to read or write capability register \isail{n}.

\begin{lstlisting}[language=sail,label=sailRISCVzX]
X(n) : regno -> xlenbits
X(n) : (regno, xlenbits) -> unit
\end{lstlisting}
The overloaded function \isail{X(n)} is used to read or write integer register \isail{n}.

\begin{lstlisting}[language=sail,label=sailRISCVzF]
F(n) : regno -> xlenbits
F(n) : (regno, xlenbits) -> unit
\end{lstlisting}
The overloaded function \isail{F(n)} is used to read or write floating-point register \isail{n}.

\medskip
\sailRISCVval{memBitsToCapability}
\sailRISCVval{capToMemBits}
\sailRISCVval{int\_to\_cap}

\medskip
\sailRISCVval{get\_cheri\_mode\_cap\_addr}
\sailRISCVval{handle\_load\_cap\_via\_cap}
\sailRISCVval{handle\_load\_data\_via\_cap}
\sailRISCVval{handle\_store\_cap\_via\_cap}
\sailRISCVval{handle\_store\_data\_via\_cap}



\subsection*{Functions for ISA exception behavior}
\sailRISCVval{handle\_exception}
\sailRISCVval{handle\_illegal}
\sailRISCVval{handle\_mem\_exception}
\sailRISCVval{handle\_cheri\_cap\_exception}
\sailRISCVval{handle\_cheri\_reg\_exception}
\sailRISCVval{handle\_cheri\_pcc\_exception}

\medskip
\sailRISCVval{pcc\_access\_system\_regs}
\sailRISCVval{privLevel\_to\_bits}
\sailRISCVval{min\_instruction\_bytes}

\medskip
\sailRISCVval{legalize\_epcc}
\sailRISCVval{legalize\_tcc}

% TODO: \subsection*{Functions for control flow}


\subsection*{Functions for manipulating capabilities}

The Sail code abstracts the capability representation using the following functions for getting and setting fields in the capability.
The base of the capability is the address of the first byte of memory to which it grants access and the top is one greater than the last byte, so the set of dereferenceable addresses is:
\[
\{ a \in \mathbb{N} \mid \mathit{base} \leq a < \mathit{top}\}
\]
Note that for 128-bit capabilities $\mathit{top}$ can be up to $2^{64}$, meaning the entire 64-bit address space can be addressed.

\medskip
\sailRISCVval{getCapBounds}
\sailRISCVval{getCapBaseBits}
\sailRISCVval{getCapTop}
\sailRISCVval{getCapLength}
\sailRISCVval{inCapBounds}

\noindent The capability's address (also known as cursor) and offset (relative to base) are related by:
\[
\mathit{base} + \mathit{offset}\ \mathbf{mod}\ 2^{64} = \mathit{cursor}
\]
The following functions return the cursor and offset of a capability respectively:
\note{rmn30}{Re-name cursor to address here?}

\medskip
\sailRISCVval{getCapCursor}
\sailRISCVval{getCapOffsetBits}
\note{rmn30}{explain what happens when offset is negative? In fact it is computed modulo $2^{64}$ and always converted straight to a 64-bit vector so not important. Should maybe just return vector.}

\noindent The following functions adjust the bounds and offset of capabilities.
Not all combinations of bounds and offset are representable, so these functions return a boolean value indicating whether the requested operation was successful.
Even in the case of failure a capability is still returned, although it may not preserve the bounds of the original capability.

\medskip
\sailRISCVval{setCapBounds}
\sailRISCVval{setCapAddr}
\sailRISCVval{setCapOffset}
\sailRISCVval{incCapOffset}

\medskip
\sailRISCVval{clearTag}
\sailRISCVval{clearTagIf}
\sailRISCVval{clearTagIfSealed}

\medskip
\sailRISCVval{getRepresentableAlignmentMask}
\sailRISCVval{getRepresentableLength}

\medskip
\arnote{TODO: short description of sealing and unsealing}
\sailRISCVval{sealCap}
\sailRISCVval{unsealCap}
\sailRISCVval{isCapSealed}
\sailRISCVval{hasReservedOType}

\noindent
Capability permissions and flags are accessed using the following functions:

\medskip
\sailRISCVval{getCapPerms}
\sailRISCVval{setCapPerms}
\sailRISCVval{getCapFlags}
\sailRISCVval{setCapFlags}

\subsection*{Checking for availability of ISA features}
\sailRISCVval{haveRVC}
\sailRISCVval{haveFExt}
\sailRISCVval{haveNExt}
\sailRISCVval{haveSupMode}


\section{CHERI-RISC-V Instructions}

\input{insn-riscv/auipcc}
\clearpage
\phantomsection
\addcontentsline{toc}{subsection}{CAndPerm}
\insnmipslabel{candperm}
\subsection*{CAndPerm: Restrict Permissions}

\subsubsection*{Format}

CAndPerm cd, cb, rt

\begin{center}
\cherithreeop[header]{0xd}{cd}{cs}{rt}
\end{center}

\subsubsection*{Description}

Capability register \emph{cd} is replaced with the contents of capability
register \emph{cb} with the \cperms{} field set to the bitwise and of
its previous value and bits 0 to
\hyperref[table:pseudocode-constants]{\emph{last\_hperm}} of integer register \emph{rt}
and the \cuperms{} field set to the bitwise and of its previous value
and bits \hyperref[table:pseudocode-constants]{\emph{first\_uperm}} to
\hyperref[table:pseudocode-constants]{\emph{last\_uperm}} of \emph{rd}.

\subsubsection*{Semantics}
\sailMIPScode{CAndPerm}

\subsubsection*{Exceptions}

A coprocessor 2 exception is raised if:

\begin{itemize}
\item
\cchecktag{}
\item
\emph{cb} is sealed.
\end{itemize}

\clearpage
\phantomsection
\addcontentsline{toc}{subsection}{CBuildCap}
\insnriscvlabel{cbuildcap}
\subsection*{CBuildCap}

\subsubsection*{Format}

\rvcheriasm{CBuildCap}

\begin{center}
\rvcheriheader
\rvcheribitbox{CBuildCap}
\end{center}

\sailRISCVisarefbody{CBuildCap}

\clearpage
\phantomsection
\addcontentsline{toc}{subsection}{CClear}
\insnriscvlabel{cclear}
\subsection*{CClear}

\subsubsection*{Format}

\rvcheriasm{CClear}

\begin{center}
\rvcheriheader
\rvcheribitbox{CClear}
\end{center}

\sailRISCVisarefbody{CClear}

\clearpage
\phantomsection
\addcontentsline{toc}{subsection}{CClearTag}
\insnmipslabel{ccleartag}
\subsection*{CClearTag: Clear the Tag Bit}

\subsubsection*{Format}

CClearTag cd, cb

\begin{center}
\cheritwoop[header]{0xb}{cd}{cb}
\end{center}

\subsubsection*{Description}

Capability register \emph{cd} is replaced with the contents of \emph{cb}, with
the tag bit cleared.

\subsubsection*{Semantics}

\sailMIPScode{CClearTag}



%\clearpage
\phantomsection
\addcontentsline{toc}{subsection}{CClearTags}
\insnriscvlabel{ccleartags}
\subsection*{CClearTags}

\subsubsection*{Format}

\rvcheriasm{CClearTags}

\begin{center}
\rvcheriheader
\rvcheribitbox{CClearTags}
\end{center}

\sailRISCVisarefbody{CClearTags}
 % missing sail
\clearpage
\phantomsection
\addcontentsline{toc}{subsection}{CCopyType}
\insnriscvlabel{ccopytype}
\subsection*{CCopyType}

\subsubsection*{Format}

\rvcheriasm{CCopyType}

\begin{center}
\rvcheriheader
\rvcheribitbox{CCopyType}
\end{center}

\sailRISCVisarefbody{CCopyType}

\clearpage
\phantomsection
\addcontentsline{toc}{subsection}{CCSeal}
\insnriscvlabel{ccseal}
\subsection*{CCSeal}

\subsubsection*{Format}

\rvcheriasm{CCSeal}

\begin{center}
\rvcheriheader
\rvcheribitbox{CCSeal}
\end{center}

\sailRISCVisarefbody{CCSeal}

\clearpage
\phantomsection
\addcontentsline{toc}{subsection}{CFromPtr}
\insnriscvlabel{cfromptr}
\subsection*{CFromPtr}

\subsubsection*{Format}

\rvcheriasm{CFromPtr}

\begin{center}
\rvcheriheader
\rvcheribitbox{CFromPtr}
\end{center}

\sailRISCVisarefbody{CFromPtr}

\subsubsection*{Deprecated}

This instruction is deprecated and may be removed in a future version.

\clearpage
\phantomsection
\addcontentsline{toc}{subsection}{CGetBase}
\insnmipslabel{cgetbase}
\subsection*{CGetBase: Move Base to an Integer Register}

\subsubsection*{Format}

CGetBase rd, cb

\begin{center}
\cheritwoop[header]{0x2}{rd}{cb}
\end{center}

\subsubsection*{Description}

Integer register \textit{rd} is set equal to the \cbase{} field of capability
register \textit{cb}.

\subsubsection*{Semantics}
\sailMIPScode{CGetBase}



\clearpage
\phantomsection
\addcontentsline{toc}{subsection}{CGetFlags}
\insnriscvlabel{cgetflags}
\subsection*{CGetFlags}

\subsubsection*{Format}

\rvcheriasm{CGetFlags}

\begin{center}
\rvcheriheader
\rvcheribitbox{CGetFlags}
\end{center}

\sailRISCVisarefbody{CGetFlags}

\input{insn-riscv/cgethigh}
\clearpage
\phantomsection
\addcontentsline{toc}{subsection}{CGetLen}
\insnmipslabel{cgetlen}
\subsection*{CGetLen: Move Length to an Integer Register}

\subsubsection*{Format}

CGetLen rd, cb

\begin{center}
\cheritwoop[header]{0x3}{rd}{cb}
\end{center}

\subsubsection*{Description}

Integer register \textit{rd} is set equal to the \clength{} field of capability
register \textit{cb}.

\subsubsection*{Semantics}
\sailMIPScode{CGetLen}

\subsubsection*{Notes}

\begin{itemize}
\item
%With the 256-bit representation of capabilities, \clength{} is a 64-bit
%unsigned integer and can never be greater than $2^{64}-1$.
With the 128-bit compressed representation of capabilities, the result of
decompressing the length can be $2^{64}$; \insnmipsref{CGetLen} will return
the maximum value of $2^{64}-1$ in this case.
\end{itemize}

\clearpage
\phantomsection
\addcontentsline{toc}{subsection}{CGetOffset}
\insnriscvlabel{cgetoffset}
\subsection*{CGetOffset}

\subsubsection*{Format}

\rvcheriasm{CGetOffset}

\begin{center}
\rvcheriheader
\rvcheribitbox{CGetOffset}
\end{center}

\sailRISCVisarefbody{CGetOffset}

\clearpage
\phantomsection
\addcontentsline{toc}{subsection}{CGetPerm}
\insnmipslabel{cgetperm}
\subsection*{CGetPerm: Move Permissions to an Integer Register}

\subsubsection*{Format}

CGetPerm rd, cb

\begin{center}
\cheritwoop[header]{0x0}{rd}{cb}
\end{center}

\subsubsection*{Description}

The least significant \hyperref[table:pseudocode-constants]{\emph{last\_hperm}}$+1$ bits of integer register \emph{rd} are set
equal to the \cperms{} field of capability register \emph{cb}; bits
\hyperref[table:pseudocode-constants]{\emph{first\_uperm}} to
\hyperref[table:pseudocode-constants]{\emph{last\_uperm}} of \emph{rd} are set equal to the
\cuperms{} field of \emph{cb}.  The other bits of \emph{rd} are set to zero.

\subsubsection*{Semantics}
\sailMIPScode{CGetPerm}



\clearpage
\phantomsection
\addcontentsline{toc}{subsection}{CGetSealed}
\insnriscvlabel{cgetsealed}
\subsection*{CGetSealed}

\subsubsection*{Format}

\rvcheriasm{CGetSealed}

\begin{center}
\rvcheriheader
\rvcheribitbox{CGetSealed}
\end{center}

\sailRISCVisarefbody{CGetSealed}

\clearpage
\phantomsection
\addcontentsline{toc}{subsection}{CGetTag}
\insnmipslabel{cgettag}
\subsection*{CGetTag: Move Tag Bit to an Integer Register}

\subsubsection*{Format}

CGetTag rd, cb


\begin{center}
\cheritwoop[header]{0x4}{rd}{cb}
\end{center}

\subsubsection*{Description}

The low bit of integer register \emph{rd} is set to the \ctag{} field of
\emph{cb}.  All other bits are cleared.

\subsubsection*{Semantics}
\sailMIPScode{CGetTag}



\input{insn-riscv/cgettop}
\clearpage
\phantomsection
\addcontentsline{toc}{subsection}{CGetType}
\insnmipslabel{cgettype}
\subsection*{CGetType: Move Object Type to an Integer Register}

\subsubsection*{Format}

CGetType rd, cb

\begin{center}
\cheritwoop[header]{0x1}{rd}{cb}
\end{center}

\subsubsection*{Description}

Integer register \textit{rd} is set equal to the \cotype{} field of capability
register \textit{cb}.

\subsubsection*{Semantics}
\sailMIPScode{CGetType}

\subsection*{Notes}

\begin{itemize}
\item
If the capability is unsealed, a value of -1 is returned.
For hardware-interpreted \cotype{}s (see \cref{tab:archotypes}) a sign-extended
(negative) value is returned; for software-defined \cotype{}s
(see \cref{sec:model-sealedcapabilities}) \insnmipsref{CGetType} returns
a zero-extended value.
\end{itemize}

\clearpage
\phantomsection
\addcontentsline{toc}{subsection}{CIncOffset}
\insnriscvlabel{cincoffset}
\subsection*{CIncOffset}

\subsubsection*{Format}

\rvcheriasm{CIncOffset}

\begin{center}
\rvcheriheader
\rvcheribitbox{CIncOffset}
\end{center}

\sailRISCVisarefbody{CIncOffset}

\clearpage
\phantomsection
\addcontentsline{toc}{subsection}{CIncOffsetImm}
\insnriscvlabel{cincoffsetimm}
\subsection*{CIncOffsetImm}

\subsubsection*{Format}

\rvcheriasm{CIncOffsetImm}

\begin{center}
\rvcheriheader
\rvcheribitbox{CIncOffsetImm}
\end{center}

\sailRISCVisarefbody{CIncOffsetImmediate}

\clearpage
\phantomsection
\addcontentsline{toc}{subsection}{CInvoke}
\insnriscvlabel{cinvoke}
\subsection*{CInvoke}

\subsubsection*{Format}

\rvcheriasm{CInvoke}

\begin{center}
\rvcheriheader
\rvcheribitbox{CInvoke}
\end{center}

\sailRISCVisarefbody{CInvoke}

\input{insn-riscv/cjal}
\clearpage
\phantomsection
\addcontentsline{toc}{subsection}{CJALR}
\insnriscvlabel{cjalr}
\subsection*{CJALR}

\subsubsection*{Format}

\rvcheriasm{CJALR}

\begin{center}
\rvcheriheader
\rvcheribitbox{CJALR}
\end{center}

\sailRISCVisarefbody{CJALR}

\clearpage
\phantomsection
\addcontentsline{toc}{subsection}{[C]LC}
\insnriscvlabel{lc}
\insnriscvlabel{clc}
\subsection*{[C]LC}

\subsubsection*{Format}

\noindent\rvcheriasmfmt[RV32, integer mode]{\rvcheriasminsnref{LC} cd, rs1, imm}

\noindent\rvcheriasmfmt[RV32, capability mode]{\rvcheriasminsnref{CLC} cd, cs1, imm}

\begin{center}
\begin{bytefield}{32}
	\bitheader[endianness=big]{0,6,7,11,12,14,15,19,20,31}\\
	\bitbox{12}{imm}
	\bitbox{5}{rs1/cs1}
	\bitbox{3}{0x3}
	\bitbox{5}{cd}
	\bitbox{7}{0x3}
\end{bytefield}
\end{center}

\noindent\rvcheriasmfmt[RV64, integer mode]{\rvcheriasminsnref{LC} cd, rs1, imm}

\noindent\rvcheriasmfmt[RV64, capability mode]{\rvcheriasminsnref{CLC} cd, cs1, imm}

\begin{center}
\begin{bytefield}{32}
	\bitheader[endianness=big]{0,6,7,11,12,14,15,19,20,31}\\
	\bitbox{12}{imm}
	\bitbox{5}{rs1/cs1}
	\bitbox{3}{0x2}
	\bitbox{5}{cd}
	\bitbox{7}{0xf}
\end{bytefield}
\end{center}

% XXX: Ideally we would be able to use [LC](LoadCapImm) in the saildoc but that
% generates a link to the literal LoadCapImm.
\label{sailRISCVzLC}
\sailRISCVisarefbody{LoadCapImm}
 % [c]lc
\clearpage
\phantomsection
\addcontentsline{toc}{subsection}{CLoadTags}
\insnriscvlabel{cloadtags}
\subsection*{CLoadTags}

\subsubsection*{Format}

\rvcheriasm{CLoadTags}

\begin{center}
\rvcheriheader
\rvcheribitbox{CLoadTags}
\end{center}

\sailRISCVisarefbody{CLoadTags}

\clearpage
\phantomsection
\addcontentsline{toc}{subsection}{CGetAddr}
\insnmipslabel{cmove}
\subsection*{CMove: Move Capability to another Register}

\subsubsection*{Format}

CMove cd, cb

\begin{center}
\cheritwoop[header]{0xa}{cd}{cb}
\end{center}

\subsubsection*{Description}

\insnnoref{CMove} copies \emph{cb} into \emph{cd}.

\subsubsection*{Semantics}

\sailMIPScode{CMove}

\subsubsection*{Notes}

\begin{itemize}
\item This instruction currently has a dedicated encoding but it could also be implemented as an alias for \insnmipsref{CMOVZ} \emph{\$zero}, \emph{cd}, \emph{cb}. \arnote{This is not possible on RISC-V since there is no conditional move. Should we add a note about this?}
\item Originally, \insnmipsref{CMove} was an assembler pseudo for \insnmipsref{CIncOffset} \emph{cd}, \emph{cb}, \emph{\$zero}.
However, this requires that \insnmipsref{CIncOffset} with a sealed capability succeeds if the increment is zero.
A future version of the ISA might no longer support this and require the use of \insnmipsref{CMove} for sealed capabilities.
This would allow for a simpler implementation of \insnmipsref{CIncOffset} where the behavior does not depend on one of the input values. \arnote{Some more rationale about intentionality?}
\end{itemize}

\input{insn-riscv/crepresentablealignmentmask}
\input{insn-riscv/croundrepresentablelength}
\clearpage
\phantomsection
\addcontentsline{toc}{subsection}{[C]SC}
\insnriscvlabel{sc}
\insnriscvlabel{csc}
\subsection*{[C]SC}

\subsubsection*{Format}

\noindent\rvcheriasmfmt[RV32, integer mode]{\rvcheriasminsnref{SC} cs2, rs1, imm}

\noindent\rvcheriasmfmt[RV32, capability mode]{\rvcheriasminsnref{CSC} cs2, cs1, imm}

\begin{center}
\begin{bytefield}{32}
	\bitheader[endianness=big]{0,6,7,11,12,14,15,19,20,24,25,31}\\
	\bitbox{7}{imm[11:5]}
	\bitbox{5}{cs2}
	\bitbox{5}{rs1/cs1}
	\bitbox{3}{0x3}
	\bitbox{5}{imm[0:4]}
	\bitbox{7}{0x23}
\end{bytefield}
\end{center}

\noindent\rvcheriasmfmt[RV64, integer mode]{\rvcheriasminsnref{SC} cs2, rs1, imm}

\noindent\rvcheriasmfmt[RV64, capability mode]{\rvcheriasminsnref{CSC} cs2, cs1, imm}

\begin{center}
\begin{bytefield}{32}
	\bitheader[endianness=big]{0,6,7,11,12,14,15,19,20,24,25,31}\\
	\bitbox{7}{imm[11:5]}
	\bitbox{5}{cs2}
	\bitbox{5}{rs1/cs1}
	\bitbox{3}{0x4}
	\bitbox{5}{imm[0:4]}
	\bitbox{7}{0x23}
\end{bytefield}
\end{center}

\sailRISCVisarefbody{StoreCapImm}
 % [c]sc
\clearpage
\phantomsection
\addcontentsline{toc}{subsection}{CSeal}
\insnmipslabel{cseal}
\subsection*{CSeal: Seal a Capability}

\subsubsection*{Format}

CSeal cd, cs, ct

\begin{center}
\cherithreeop[header]{0xb}{cd}{cs}{ct}
\end{center}

\subsubsection*{Description}

Capability register \emph{cs} is sealed
\psnote{it's confusing phrasing to say the \emph{register} itself is sealed, as
opposed to the value the register. Instead, perhaps ``The capability
 in register \emph{cs} is sealed''?  I guess this idiom may occur in
 many places, so I've not just fixed this one. Robert N-W?}
 with an \cotype{} of
\emph{ct}.\cbase{} $+$ \emph{ct}.\coffset{}
and the result is placed in \emph{cd}:

\begin{itemize}
\item
\emph{cd}.\cotype{} is set to \emph{ct}.\cbase{} + \emph{ct}.\coffset{};
\item
\emph{cd} is sealed;
\item
and the other fields of \emph{cd} are copied from \emph{cs}.
\end{itemize}

\emph{ct} must grant \emph{Permit\_Seal} permission, and the new \cotype{}
of \emph{cd} must be between \emph{ct}.\cbase{} and \emph{ct}.\cbase{} $+$
\emph{ct}.\clength{} $-$ 1.

\subsubsection*{Semantics}
\sailMIPScode{CSeal}

\subsubsection*{Exceptions}

A coprocessor 2 exception is raised if:

\begin{itemize}
\item
\emph{cs}.\ctag{} is not set.
\item
\emph{ct}.\ctag{} is not set.
\item
\emph{cs} is sealed.
\item
\emph{ct} is sealed.
\item
\emph{ct}.\cperms.\emph{Permit\_Seal} is not set.
% \item
% \emph{ct}.\cperms.\emph{Permit\_Execute} is not set.
\item
\emph{ct}.\coffset{} $\ge$ \emph{ct}.length{}
\item
\emph{ct}.\cbase{} $+$ \emph{ct}.\coffset{} $> \emph{max\_otype}$
\item
The bounds of \emph{cb} cannot be represented exactly in a sealed capability.
\end{itemize}

\subsubsection*{Notes}

\clearpage
\phantomsection
\addcontentsline{toc}{subsection}{CSealEntry}
\insnriscvlabel{csealentry}
\subsection*{CSealEntry}

\subsubsection*{Format}

\rvcheriasm{CSealEntry}

\begin{center}
\rvcheriheader
\rvcheribitbox{CSealEntry}
\end{center}

\sailRISCVisarefbody{CSealEntry}

\clearpage
\phantomsection
\addcontentsline{toc}{subsection}{CSetAddr}
\insnmipslabel{csetaddr}
\subsection*{CSetAddr: Set the Address of Capability}

\subsubsection*{Format}

CSetAddr cd, cb, rt

\begin{center}
\cherithreeop[header]{0x22}{cd}{cb}{rt}
\end{center}

\subsubsection*{Description}

\emph{cd} is set to \emph{cb} with \emph{cb}.\caddr{} set to \emph{rt}.
If changing the address causes the capability to become unrepresentable, then an untagged capability with the requested address is returned.

\subsubsection*{Semantics}

\sailMIPScode{CSetAddr}

\subsubsection*{Exceptions}

A coprocessor 2 exception is raised if:

\begin{itemize}
\item
\emph{cb}.\ctag{} is set and \emph{cb} is sealed.
\end{itemize}

\subsubsection*{Notes}

\begin{itemize}
\item This instruction may be useful, in combination with \insnmipsref{CGetAddr}, when C is manipulating pointers in ways that require a round trip through integer registers.
\item This instruction is also useful for \ccode{uintptr\_t} arithmetic when using an address interpretation of capabilities. When interpreting \ccode{uintptr\_t} as offsets relative
to the base, the compiler will use \insnmipsref{CGetOffset} and \insnmipsref{CSetOffset} instead.

\end{itemize}

\clearpage
\phantomsection
\addcontentsline{toc}{subsection}{CSetBounds}
\insnmipslabel{csetbounds}
\subsection*{CSetBounds: Set Bounds}

\subsubsection*{Format}

CSetBounds cd, cb, rt

\begin{center}
\cherithreeop[header]{0x8}{cd}{cb}{rt}
\end{center}

\subsubsection*{Description}

Capability register \emph{cd} is replaced with a capability that:

\begin{itemize}
\item
Grants access to a subset of the addresses authorized by \emph{cb}.
That is, \emph{cd}.\cbase{} $\ge$ \emph{cb}.\cbase{} and
\emph{cd}.\cbase{} $+$ \emph{cd}.\clength{} $\le$ \emph{cb}.\cbase{} $+$
\emph{cb}.\clength{}.
\item
Grants access to at least the addresses \emph{cb}.\cbase{} $+$
\emph{cb}.\coffset{} $\ldots$ \emph{cb}.\cbase{} $+$ \emph{cb}.\coffset{}
$+$ \emph{rt} $-$ 1.
That is, \emph{cd}.\cbase{} $\le$ \emph{cb}.\cbase{}
$+$ \emph{cb}.\coffset{} and \emph{cd}.\cbase{} $+$ \emph{cd}.\clength{}
$\ge$ \emph{cb}.\cbase{} $+$ \emph{cb}.\coffset{} $+$ \emph{rt}.
\item
Has an \coffset{} that points to the same memory location as \emph{cb}'s
\coffset{}.
That is, \emph{cd}.\coffset{} = \emph{cb}.\coffset{} + \emph{cb}.\cbase{} -
\emph{cd}.\cbase{}.
\item
Has the same \cperms{} as \emph{cb}, that is, \emph{cd}.\cperms{} = \emph{cb}.\cperms{}.
\end{itemize}

%When the hardware uses a 256-bit representation for capabilities, the bounds
%of the destination capability \emph{cd} are exactly as requested.
With compressed capabilities, not all combinations of \cbase{} and \clength{}
are representable.
\emph{cd} may therefore grant access to a range of memory addresses that is
wider than requested, but is still guaranteed to be within the bounds of
\emph{cb}.
\rwnote{Check that the following statement is true.}
This cannot occur if the requested bounds have been suitably aligned and
padded using the \insnmipsref{CRAM} and \insnmipsref{CRRL} instructions.
If software is not guaranteed to provide suitable alignment and padding, it
may be desirable to use \insnmipsref{CSetBoundsExact} so that an exception will
be thrown the requested bounds cannot be represented.

\subsubsection*{Semantics}

\sailMIPScode{CSetBounds}

\subsubsection*{Exceptions}

A coprocessor 2 exception is raised if:

\begin{itemize}
\item
\cchecktag{}
\item
\emph{cb} is sealed.
\item
\emph{cursor} $<$ \emph{cb}.\cbase{}
\item
\emph{cursor} $+$ \emph{rt} $>$ \emph{cb}.\cbase{} $+$ cb.\clength{}
\end{itemize}

\subsubsection*{Notes}

\begin{itemize}
\item
In the above Sail code, arithmetic is over the mathematical integers and
\emph{rt} is unsigned, so a large value of \emph{rt} cannot cause
\emph{cursor} $+$ \emph{rt} to wrap around and be less than \emph{cb}.\cbase{}.
Implementations (that, for example, will probably use a fixed number of
bits to store values) must handle this overflow case correctly.
\end{itemize}

\clearpage
\phantomsection
\addcontentsline{toc}{subsection}{CSetBoundsExact}
\insnmipslabel{csetboundsexact}
\subsection*{CSetBoundsExact: Set Bounds Exactly}

\subsubsection*{Format}

CSetBoundsExact cd, cb, rt

\begin{center}
\cherithreeop[header]{0x9}{cd}{cb}{rt}
\end{center}

\subsubsection*{Description}

Capability register \emph{cd} is replaced with a capability with its \cbase{}
replaced with \emph{cb}.\cbase{} $+$ \emph{cb}.\coffset{}, \clength{} set to
\emph{rt}, and \coffset{} set to zero.
When capability compression is in use, an exception is thrown if the requested
bounds cannot be represented exactly.

\subsubsection*{Semantics}
\sailMIPScode{CSetBoundsExact}

An exception cannot occur if the requested bounds have been suitably aligned
and padded using the \insnmipsref{CRAM} and \insnmipsref{CRRL} instructions.
If looser bounds, rather than exception, are desired, then it may be
preferable to use \insnmipsref{CSetBounds}.

\subsubsection*{Exceptions}

A coprocessor 2 exception is raised if:

\begin{itemize}
\item
\cchecktag{}
\item
\emph{cb} is sealed.
\item
\emph{cursor} $<$ \emph{cb}.\cbase{}
\item
\emph{cursor} $+$ \emph{rt} $>$ \emph{cb}.\cbase{} $+$ cb.\clength{}
\item
The requested bounds cannot be represented exactly.
\end{itemize}

\subsubsection*{Notes}

\begin{itemize}
\item
In the above Sail code, arithmetic is over the mathematical integers and
\emph{rt} is unsigned, so a large value of \emph{rt} cannot cause
\emph{cursor} $+$ \emph{rt} to wrap around and be less than \emph{cb}.\cbase{}.
Implementations (that, for example, will probably use a fixed number of
bits to store values) must handle this overflow case correctly.
\end{itemize}

\clearpage
\phantomsection
\addcontentsline{toc}{subsection}{CSetBoundsImm}
\insnmipslabel{csetboundsimm}
\subsection*{CSetBoundsImm: Set Bounds (Immediate)}

\subsubsection*{Format}

CSetBounds cd, cb, length$_{imm}$

\begin{center}
\begin{bytefield}{32}
\bitheader[endianness=big]{0,10,11,15,16,20,21,25,26,31}\\
\bitbox{6}{0x12}
\bitbox{5}{0x14}
\bitbox{5}{cd}
\bitbox{5}{cb}
\bitbox{11}{length$_{imm}$}
\end{bytefield}
\end{center}

\arnote{The assembler supports both CSetBounds and CSetBounds but I think we should always use CSetBoundsImm}

\subsubsection*{Description}

Capability register \emph{cd} is replaced with a capability that:

\begin{itemize}
\item
Grants access to a subset of the addresses authorized by \emph{cb}.
That is, \emph{cd}.\cbase{} $\ge$ \emph{cb}.\cbase{} and
\emph{cd}.\cbase{} $+$ \emph{cd}.\clength{} $\le$ \emph{cb}.\cbase{} $+$
\emph{cb}.\clength{}.
\item
Grants access to at least the addresses \emph{cb}.\cbase{} $+$
\emph{cb}.\coffset{} $\ldots$ \emph{cb}.\cbase{} $+$ \emph{cb}.\coffset{}
$+$ \emph{length$_{imm}$} $-$ 1.
That is, \emph{cd}.\cbase{} $\le$ \emph{cb}.\cbase{}
$+$ \emph{cb}.\coffset{} and \emph{cd}.\cbase{} $+$ \emph{cd}.\clength{}
$\ge$ \emph{cb}.\cbase{} $+$ \emph{cb}.\coffset{} $+$ \emph{length$_{imm}$}.
\item
Has an \coffset{} that points to the same memory location as \emph{cb}'s
\coffset{}.
That is, \emph{cd}.\coffset{} = \emph{cb}.\coffset{} + \emph{cb}.\cbase{} -
\emph{cd}.\cbase{}.
\item
Has the same \cperms{} as \emph{cb}, that is, \emph{cd}.\cperms{} = \emph{cb}.\cperms{}.
\end{itemize}

%When the hardware uses a 256-bit representation for capabilities, the bounds
%of the destination capability \emph{cd} are exactly as requested.

With compressed capabilities, not all combinations of \cbase{} and \clength{}
are representable.
\emph{cd} may therefore grant access to a range of memory addresses that is
wider than requested, but is still guaranteed to be within the bounds of
\emph{cb}.

\subsubsection*{Semantics}
\sailMIPScode{CSetBoundsImmediate}

\subsubsection*{Exceptions}

A coprocessor 2 exception is raised if:

\begin{itemize}
\item
\cchecktag{}
\item
\emph{cb} is sealed.
\item
\emph{cursor} $<$ \emph{cb}.\cbase{}
\item
\emph{cursor} $+$ \emph{length$_{imm}$} $>$ \emph{cb}.\cbase{} $+$ cb.\clength{}
\end{itemize}

\subsubsection*{Notes}

\begin{itemize}
\item
In the above Sail code, arithmetic is over the mathematical integers and
\emph{length$_{imm}$} is unsigned, so a large value of \emph{length$_{imm}$} cannot cause
\emph{cursor} $+$ \emph{length$_{imm}$} to wrap around and be less than \emph{cb}.\cbase{}.
Implementations (that, for example, will probably use a fixed number of
bits to store values) must handle this overflow case correctly.
\item If this instruction is used with \creg{0} as the destination register, it can be used to assert that a given capability grants access to at least \emph{length$_{imm}$} bytes. An assembler pseudo instruction \insnmipsref{CAssertInBounds} is supported for this use case.
\end{itemize}

\input{insn-riscv/csetequalexact}
\clearpage
\phantomsection
\addcontentsline{toc}{subsection}{CSetFlags}
\insnriscvlabel{csetflags}
\subsection*{CSetFlags}

\subsubsection*{Format}

\rvcheriasm{CSetFlags}

\begin{center}
\rvcheriheader
\rvcheribitbox{CSetFlags}
\end{center}

\sailRISCVisarefbody{CSetFlags}

\input{insn-riscv/csethigh}
\clearpage
\phantomsection
\addcontentsline{toc}{subsection}{CSetOffset}
\insnriscvlabel{csetoffset}
\subsection*{CSetOffset}

\subsubsection*{Format}

\rvcheriasm{CSetOffset}

\begin{center}
\rvcheriheader
\rvcheribitbox{CSetOffset}
\end{center}

\sailRISCVisarefbody{CSetOffset}

\input{insn-riscv/cspecialrw}
\clearpage
\phantomsection
\addcontentsline{toc}{subsection}{CTestSubset}
\insnmipslabel{ctestsubset}
\subsection*{CTestSubset: Test that Capability is a Subset of Another}

\subsubsection*{Format}

CTestSubset rd, cb, ct

\begin{center}
\begin{bytefield}{32}
\bitheader[endianness=big]{0,5,6,10,11,15,16,20,21,25,26,31}\\
\bitbox{6}{0x12}
\bitbox{5}{0x0}
\bitbox{5}{rd}
\bitbox{5}{cb}
\bitbox{5}{ct}
\bitbox{6}{0x20}
\end{bytefield}
\end{center}

\usesDDCinsteadofNULL{cb}

\subsubsection*{Description}

\insnmipsref{CTestSubset} tests if the bounds of \emph{ct} are within the
bounds of \emph{cb}, and the permissions of \emph{ct} are within the permissions
of \emph{cb}, setting \emph{rd} to \emph{1} if so, and \emph{0} if not.

\note{mr101}{What should happen if one of the capabilities is NULL? Is the
NULL capability a subset of a valid capability? Comparing the \cbase{} and
\clength{} fields when the \ctag{} bit is clear seems the wrong thing to do}

\note{mr101}{Is a zero length capability a subset of a valid capability,
even if it's \cbase{} is not within the range. If you view the operation as
a subset of the memory bytes, then it is a subset}

\note{mr101}{The motivating use case is \insnmipsref{CToPtr},
\insnmipsref{CTestSubset}, \insnnoref{MOVZ reg, zero}. In error cases
where \insnmipsref{CToPtr} has returned zero, we don't really care what
\insnmipsref{CTestSubset} returns, because we're going to get the NULL
pointer anyway.}

\subsubsection*{Semantics}
\sailMIPScode{CTestSubset}

\subsubsection*{Exceptions}

A coprocessor 2 exception is raised if:

\begin{itemize}
\item
\emph{cd}, \emph{cb} or \emph{ct} is a reserved register and \PCC.\cperms{} does
not grant \emph{Permit\_Access\_System\_Registers}.
\end{itemize}

\subsubsection*{Notes}

\begin{itemize}
\item
This instruction was originally motivated as an additional check for
\insnmipsref{CToPtr}.
A conversion of a capability to a pointer with respect to a default capability
would normally expect that the entire capability is accessible within the
default capability with (at least) the original permissions.
\insnmipsref{CTestSubset} can perform this assertion, and a \insnmipsref{CMove}
instruction can replace the result of the \insnmipsref{CToPtr} with NULL upon
failure.
\item
Another use case for this instruction is in garbage collection. For this
application, we want to be able to test if one capability is a subset of
the other even if one is sealed and the other is not. (For the purposes of
garbage collection, a sealed reference to a region of memory is still a
reference to that region of memory). With compressed capabilities, the bounds
are represented differently for sealed and unsealed capabilities, but
\insnmipsref{CTestSubset} is still able to perform the subset check.
\end{itemize}

\clearpage
\phantomsection
\addcontentsline{toc}{subsection}{CToPtr}
\insnriscvlabel{ctoptr}
\subsection*{CToPtr}

\subsubsection*{Format}

\rvcheriasm{CToPtr}

\begin{center}
\rvcheriheader
\rvcheribitbox{CToPtr}
\end{center}

\sailRISCVisarefbody{CToPtr}

\clearpage
\phantomsection
\addcontentsline{toc}{subsection}{CUnseal}
\insnmipslabel{cunseal}
\subsection*{CUnseal: Unseal a Sealed Capability}

\subsubsection*{Format}

CUnseal cd, cs, ct

\begin{center}
\cherithreeop[header]{0xc}{cd}{cs}{ct}
\end{center}

\subsubsection*{Description}

The sealed capability in $cs$ is unsealed with $ct$ and the result placed
in $cd$. The global bit of $cd$ is the AND of the global bits of
$cs$ and $ct$. $ct$ must be unsealed, have \emph{Permit\_Unseal} permission, and $ct$.\cbase{} + $ct$.\coffset{} must equal $cs$.\cotype{}.

\subsubsection*{Semantics}

\sailMIPScode{CUnseal}

\subsubsection*{Exceptions}

A coprocessor 2 exception is raised if:

\begin{itemize}
\item
\emph{cs}.\ctag{} is not set.
\item
\emph{ct}.\ctag{} is not set.
\item
\emph{cs} is not sealed.
\item
\emph{ct} is sealed.
\item
\emph{ct}.\coffset{} $\ge$ \emph{ct}.\clength{}
\item
\emph{ct}.\cperms{}.\emph{Permit\_Unseal} is not set.
\item
\emph{ct}.\cbase{} $+$ \emph{ct}.\coffset{} $\ne$ \emph{cs}.\cotype{}.
\end{itemize}

\subsubsection*{Notes}

\begin{itemize}
\item
There is no need to check if \emph{ct}.\cbase{} $+$ \emph{ct}.\coffset{}
$>$ \emph{max\_otype}, because this can't happen:
\emph{ct}.\cbase{} $+$ \emph{ct}.\coffset{} must equal \emph{cs}.\cotype{}
for the \cotype{} check to have succeeded, and there is no way
\emph{cs}.\cotype{} could have been set to a value that is out of range.
\end{itemize}

\clearpage
\phantomsection
\addcontentsline{toc}{subsection}{FPClear}
\insnriscvlabel{fpclear}
\subsection*{FPClear}

\subsubsection*{Format}

\rvcheriasm{FPClear}

\begin{center}
\rvcheriheader
\rvcheribitbox{FPClear}
\end{center}

\sailRISCVisarefbody{FPClear}

\clearpage
\phantomsection
\addcontentsline{toc}{subsection}{JALR.CAP}
\insnriscvlabel{jalrcap}
\insnriscvlabel{jalr.cap}
\subsection*{JALR.CAP}

\subsubsection*{Format}

\rvcheriasm{JALR.CAP}

\begin{center}
\rvcheriheader
\rvcheribitbox{JALR.CAP}
\end{center}

\sailRISCVisarefbody{JALRUnderscoreCAP}

\clearpage
\phantomsection
\addcontentsline{toc}{subsection}{JALR.PCC}
\insnriscvlabel{jalrpcc}
\insnriscvlabel{jalr.pcc}
\subsection*{JALR.PCC}

\subsubsection*{Format}

\rvcheriasm{JALR.PCC}

\begin{center}
\rvcheriheader
\rvcheribitbox{JALR.PCC}
\end{center}

\sailRISCVisarefbody{JALRUnderscorePCC}

\clearpage
\phantomsection
\addcontentsline{toc}{subsection}{L[BHWD][U].CAP}
\insnriscvlabel{loaddatacap}
\insnriscvlabel{lb.cap}
\insnriscvlabel{lh.cap}
\insnriscvlabel{lw.cap}
\insnriscvlabel{ld.cap}
\insnriscvlabel{lbu.cap}
\insnriscvlabel{lhu.cap}
\insnriscvlabel{lwu.cap}
\subsection*{L[BHWD][U].CAP}

\subsubsection*{Format}

\noindent\rvcheriasm{LB.CAP}

\begin{center}
\rvcheriheader
\rvcheribitbox{LB.CAP}
\end{center}

\noindent\rvcheriasm{LH.CAP}

\begin{center}
\rvcheriheader
\rvcheribitbox{LH.CAP}
\end{center}

\noindent\rvcheriasm{LW.CAP}

\begin{center}
\rvcheriheader
\rvcheribitbox{LW.CAP}
\end{center}

\noindent\rvcheriasm{LD.CAP:RV64/128}

\begin{center}
\rvcheriheader
\rvcheribitbox{LD.CAP:RV64/128}
\end{center}

\noindent\rvcheriasm{LBU.CAP}

\begin{center}
\rvcheriheader
\rvcheribitbox{LBU.CAP}
\end{center}

\noindent\rvcheriasm{LHU.CAP}

\begin{center}
\rvcheriheader
\rvcheribitbox{LHU.CAP}
\end{center}

\noindent\rvcheriasm{LWU.CAP:RV64/128}

\begin{center}
\rvcheriheader
\rvcheribitbox{LWU.CAP:RV64/128}
\end{center}

\sailRISCVisarefbody{LoadDataCap}
 % l[bhwd][u].cap
\clearpage
\phantomsection
\addcontentsline{toc}{subsection}{L[BHWD][U].DDC}
\insnriscvlabel{loaddataddc}
\insnriscvlabel{lb.ddc}
\insnriscvlabel{lh.ddc}
\insnriscvlabel{lw.ddc}
\insnriscvlabel{ld.ddc}
\insnriscvlabel{lbu.ddc}
\insnriscvlabel{lhu.ddc}
\insnriscvlabel{lwu.ddc}
\subsection*{L[BHWD][U].DDC}

\subsubsection*{Format}

\noindent\rvcheriasm{LB.DDC}

\begin{center}
\rvcheriheader
\rvcheribitbox{LB.DDC}
\end{center}

\noindent\rvcheriasm{LH.DDC}

\begin{center}
\rvcheriheader
\rvcheribitbox{LH.DDC}
\end{center}

\noindent\rvcheriasm{LW.DDC}

\begin{center}
\rvcheriheader
\rvcheribitbox{LW.DDC}
\end{center}

\noindent\rvcheriasm{LD.DDC:RV64/128}

\begin{center}
\rvcheriheader
\rvcheribitbox{LD.DDC:RV64/128}
\end{center}

\noindent\rvcheriasm{LBU.DDC}

\begin{center}
\rvcheriheader
\rvcheribitbox{LBU.DDC}
\end{center}

\noindent\rvcheriasm{LHU.DDC}

\begin{center}
\rvcheriheader
\rvcheribitbox{LHU.DDC}
\end{center}

\noindent\rvcheriasm{LWU.DDC:RV64/128}

\begin{center}
\rvcheriheader
\rvcheribitbox{LWU.DDC:RV64/128}
\end{center}

\sailRISCVisarefbody{LoadDataDDC}
 % l[bhwd][u].ddc
\clearpage
\phantomsection
\addcontentsline{toc}{subsection}{LC.CAP}
\insnriscvlabel{loadcapcap}
\insnriscvlabel{lc.cap}
\subsection*{LC.CAP}

\subsubsection*{Format}

\noindent\rvcheriasm{LC.CAP:RV32}

\begin{center}
\rvcheriheader
\rvcheribitbox{LC.CAP:RV32}
\end{center}

\noindent\rvcheriasm{LC.CAP:RV64}

\begin{center}
\rvcheriheader
\rvcheribitbox{LC.CAP:RV64}
\end{center}

\sailRISCVisarefbody{LoadCapCap}
 % lc.cap
\clearpage
\phantomsection
\addcontentsline{toc}{subsection}{LC.DDC}
\insnriscvlabel{loadcapddc}
\insnriscvlabel{lc.ddc}
\subsection*{LC.DDC}

\subsubsection*{Format}

\noindent\rvcheriasm{LC.DDC:RV32}

\begin{center}
\rvcheriheader
\rvcheribitbox{LC.DDC:RV32}
\end{center}

\noindent\rvcheriasm{LC.DDC:RV64}

\begin{center}
\rvcheriheader
\rvcheribitbox{LC.DDC:RV64}
\end{center}

\sailRISCVisarefbody{LoadCapDDC}
 % lc.ddc
\clearpage
\phantomsection
\addcontentsline{toc}{subsection}{S[BHWD].CAP}
\insnriscvlabel{storedatacap}
\insnriscvlabel{sb.cap}
\insnriscvlabel{sh.cap}
\insnriscvlabel{sw.cap}
\insnriscvlabel{sd.cap}
\subsection*{S[BHWD].CAP}

\subsubsection*{Format}

\noindent\rvcheriasm{SB.CAP}

\begin{center}
\rvcheriheader
\rvcheribitbox{SB.CAP}
\end{center}

\noindent\rvcheriasm{SH.CAP}

\begin{center}
\rvcheriheader
\rvcheribitbox{SH.CAP}
\end{center}

\noindent\rvcheriasm{SW.CAP}

\begin{center}
\rvcheriheader
\rvcheribitbox{SW.CAP}
\end{center}

\noindent\rvcheriasm{SD.CAP:RV64/128}

\begin{center}
\rvcheriheader
\rvcheribitbox{SD.CAP:RV64/128}
\end{center}

\sailRISCVisarefbody{StoreDataCap}
 % s[bhwd].cap
\clearpage
\phantomsection
\addcontentsline{toc}{subsection}{S[BHWD].DDC}
\insnriscvlabel{storedataddc}
\insnriscvlabel{sb.ddc}
\insnriscvlabel{sh.ddc}
\insnriscvlabel{sw.ddc}
\insnriscvlabel{sd.ddc}
\subsection*{S[BHWD].DDC}

\subsubsection*{Format}

\noindent\rvcheriasm{SB.DDC}

\begin{center}
\rvcheriheader
\rvcheribitbox{SB.DDC}
\end{center}

\noindent\rvcheriasm{SH.DDC}

\begin{center}
\rvcheriheader
\rvcheribitbox{SH.DDC}
\end{center}

\noindent\rvcheriasm{SW.DDC}

\begin{center}
\rvcheriheader
\rvcheribitbox{SW.DDC}
\end{center}

\noindent\rvcheriasm{SD.DDC:RV64/128}

\begin{center}
\rvcheriheader
\rvcheribitbox{SD.DDC:RV64/128}
\end{center}

\sailRISCVisarefbody{StoreDataDDC}
 % s[bhwd].ddc
\clearpage
\phantomsection
\addcontentsline{toc}{subsection}{SC.CAP}
\insnriscvlabel{storecapcap}
\insnriscvlabel{sc.cap}
\subsection*{SC.CAP}

\subsubsection*{Format}

\noindent\rvcheriasm{SC.CAP:RV32}

\begin{center}
\rvcheriheader
\rvcheribitbox{SC.CAP:RV32}
\end{center}

\noindent\rvcheriasm{SC.CAP:RV64}

\begin{center}
\rvcheriheader
\rvcheribitbox{SC.CAP:RV64}
\end{center}

\sailRISCVisarefbody{StoreCapCap}
 % sc.cap
\clearpage
\phantomsection
\addcontentsline{toc}{subsection}{SC.DDC}
\insnriscvlabel{storecapddc}
\insnriscvlabel{sc.ddc}
\subsection*{SC.DDC}

\subsubsection*{Format}

\noindent\rvcheriasm{SC.DDC:RV32}

\begin{center}
\rvcheriheader
\rvcheribitbox{SC.DDC:RV32}
\end{center}

\noindent\rvcheriasm{SC.DDC:RV64}

\begin{center}
\rvcheriheader
\rvcheribitbox{SC.DDC:RV64}
\end{center}

\sailRISCVisarefbody{StoreCapDDC}
 % sc.ddc
