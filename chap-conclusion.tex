\chapter{Conclusion}
\label{chap:conclusion}
The CHERI project is in its tenth year -- an evolution described in detail in
Chapter~\ref{chap:research}.
Throughout the project, we have utilized hardware-software co-design, with an
increasing focus on also co-designing with formal models.
For the last several years, we have increased our focus on transition,
including through a close collaboration with Arm in the development of their
Morello architecture and hardware prototype~\cite{arm-morello}.
Our contributions include:

\begin{enumerate}
\item We developed the CHERI protection model and reference CHERI-MIPS and
  CHERI-RISC-V Instruction-Set Architectures, which  offer low-overhead
  fine-grained memory protection and support scalable software
  compartmentalization based on a hybrid capability model.
  Over several generations of the ISA, we refined integration with
  conventional
  RISCs ISA, composed the capability-system model with the MMU, pursued strong
  C-language compatibility, developed compartmentalization features based on
  an object-capability model, refined the architecture to improve performance
  and adoptability through features such as compressed 128-bit capabilities,
  introduced support for temporal memory safety,
  and developed the notion of a portable protection model that can be applied
  to further ISAs.
  We also explored the implications of CHERI on 32-bit microcontroller
  architectures
  that do not have MMUs, giving capabilities a physical interpretation, and
  also taking into account common microcontroller microarchitectural choices.

\item Employed increasingly complete formal models of the protection model and
  ISA semantics.
  We began by using PVS/SAL formal models of the ISA to analyze
  expressivity and security.
  Subsequently, and in close collaboration with the University of Cambridge's
  EPSRC-funded Rigorous Engineering of Mainstream Systems (REMS) Project and
  DARPA-funded CHERI Instruction-set Formal Verification (CIFV), we
  developed L3 and SAIL formal models suitable to act as a gold model for
  testing, to use in automated test generation, and as inputs to formal
  verification tools to prove ISA-level security properties.
  We have also used formal modeling to explore how CHERI interacts with
  C-language semantics.
  In the future, we hope to employ these models in support of hardware and
  software verification.

\item Elaborated the ISA feature set in CHERI to support real-world operating
  systems -- primarily this has consisted of developing mature compositions of
  CHERI's concepts with system features such as the MMU and exception model.
  We have also spent considerable time refining successive versions of
  the ISA intended to better support high levels of MMU-based operating-system
  and C-language compatibility, as well as automatic use by compilers.
  This work has incorporated ideas from, but also gone substantially beyond,
  the C-language fat-pointer and software compartmentalization research
  literature.

\item Created our CheriBSD and CheriFreeRTOS operating-system prototypes.
  These reference designs explore how CHERI integration with the OS can
  improve both OS and application security.
  We created prototype memory protection and software compartmentalization
  environments including CheriBSD's CheriABI process environment, and multiple
  approaches to sandboxing.
  We have applied CHERI to protection of key system libraries, but also the
  operating system kernel itself.
  This has included substantial baseline OS infrastructure work, including
  porting FreeBSD to the RISC-V architecture.

\item Prototyped, tested, and refined CHERI ISA extensions across multiple CPU
  architectures.
  We have open sourced reference MIPS and RISC-V processor designs, and the
  QEMU ISA-level emulator, on order to allow reproducible experimentation with
  our approach, as well as to act as an open-source platform for other future
  hardware-software research projects.

\item Adapted the Clang/LLVM compiler suite to be able to generate CHERI ISA
  instructions as directed by C-language annotations, exploring a variety of
  language models, code-generation models, and ABIs.
  We have explored two new C-language models and associated code generation:
  a hybrid in which explicitly annotated or automatically inferred pointers
  are compiled as capabilities; and a pure-capability model in which all
  pointers and implied virtual addresses are compiled as capabilities.
  Similarly, we have begun an exploration of how CHERI affects program
  linkage, with early prototype integration with the compile-time and run-time
  linkers.
  These collectively provide strong spatial and pointer protection for both
  data and code pointers.
  We have upstreamed substantial improvements to Clang/LLVM MIPS support, as
  well as changes making it easier to support ISA features such as extra-wide
  pointers utilized in the CHERI ISA.
  We have also begun to explore how CHERI can support higher-level language
  protection, such as by using it to reinforce memory safety and security for
  native code running under the Java Native Interface (JNI).

\item Began to develop semi-automated techniques to assist software
  developers in compartmentalizing applications using Capsicum and CHERI
  features.
  This is a subproject known as Security-Oriented Analysis of Application
  Programs (SOAAP), and performed in collaboration with Google.

\item Collaborated with Arm on the creation of Morello~\cite{arm-morello}, a
  tight integration of CHERI into the ARMv8-A architecture as well as a
  reference industrial quality microarchitecture.
  Morello boards will ship for experimental use in late 2021 along with our
  complete CHERI software stack.
\end{enumerate}

Collectively, these accomplishments have validated our research hypotheses:
that a hybrid capability-system architecture and viable supporting
microarchitecture can support low-overhead memory protection and fine-grained
software compartmentalization while maintaining strong compatibility with
current RISC, MMU-based, and C/C++-language software stacks, as well as an
incremental software adoption path to additional trustworthiness.
Further, the resulting protection model, co-designed around a specific ISA and
concrete extensions, is in fact a generalizable and portable protection model
that has been applied to other ISAs; it is suitable for a multitude of
implementations in architecture and microarchitecture.
Formal methodology deployed judiciously throughout the design and
implementation process has increased our confidence that the resulting design
can support robust and resilient software designs.

\section{Future Work}
We have made a strong beginning, but clearly there is still much to do in our
remaining CHERI efforts.
Our ongoing key areas of research include:

\begin{itemize}
\item
Continuing to refine performance with respect to both the architecture (e.g.,
models for capability compression) and microarchitecture (e.g., as relates to
efficient implementations of compression and tagged memory).

\item
Exploring how CHERI's features might be scaled up (e.g., to superscalar
processor designs), down (e.g., to 32-bit microcontrollers without MMUs), and
to other compute types (e.g., DMA engines, GPUs, and so on).
Also, looking at how CHERI interacts with other emerging hardware technologies
such as non-volatile memory, where CHERI may support more rapid, robust, and
secure adoption.

\item
Continuing to elaborate how CHERI should affect the design of operating
systems (whether hybrid systems such as CheriBSD, or clean-slate designs),
languages (e.g., C, C++, Java, and so on), and runtimes (e.g., system
libraries, run-time linking, and higher-level language runtimes).

\item
Continuing to explore how CHERI affects software tracing and debugging; for
example, through capability-aware software debuggers.

\item
Continuing to explore potential models for software compartmentalization, such
as clean-slate microkernel-style message passing grounded in CHERI's
object-capability features, but not hybridized with conventional OS designs.
In addition, continuing to investigate potential approaches to semi- or
fully automated software compartmentalization.

\item
Continuing our efforts to develop and utilize formal models of the
microarchitecture, architecture, operating system, linkage model, language
properties, compilation, and higher-level applications.
This will help us understand (and ensure) the protection benefits of CHERI up
and down the hardware-software stack.
\end{itemize}
