This release of the \textit{CHERI Instruction-Set Architecture} is an
interim version intended for submission to DARPA/AFRL to meet the requirements
of CTSRD deliverable A001:

\begin{itemize}
\item The CHERI ISA specification version numbering scheme has changed to
include the target major version in the draft version number.

\item A significant refactoring of early chapters in the report has taken place:
there is now a more clear distinction between architecture-neutral aspects
of CHERI, and those that are architecture specific.
The CHERI-MIPS ISA is now its own chapter distinct from architecture-neutral
material.
We have aimed to maximize architecture-neutral content -- e.g., capability
semantics and contents, in-memory representation, compression, etc. -- using
the architecture-specific chapters to address only architecture-specific
aspects of the mapping of CHERI into the specific architecture -- e.g., as
relates to register-file integration, exception handling, and the Memory
Management Unit (MMU).
In some areas, content must be split between architecture-neutral and
architecture-specific chapters, such as behavior on reset, handling of the
\cappermASR permission and its role in controlling
architecture-specific behavior, and the integration of CHERI with virtual
memory, where the goals are largely architecture neutral but mechanism is
architecture specific.

\item There are now dedicated chapters for each of our applications of CHERI
to each of three ISAs: 64-bit MIPS, 64-bit
RISC-V (Chapter~\ref{chap:cheri-riscv}), and x86-64
(Chapter~\ref{chap:cheri-x86-64}).

\item Our CHERI-RISC-V prototype has been substantially elaborated, and now
includes an experimental encoding in Appendix~\ref{app:isaquick-riscv}.
We have somewhat further elaborated our x86-64 model, including addressing
topics such as new page-table bits for CHERI, including a hardware-managed
capability dirty bit.
We also consider potential implications for RISC-V compressed instructions.

\item We have completed an opcode renumbering for CHERI-MIPS.
The ``proposed new encoding'' from CHERI ISAv6 has now become the
established encodings; the prior encodings are now documented as
``deprecated encodings''.

\item Substantial improvements have been made to descriptive text around memory
protection, with the concept of ``pointer protection'' -- i.e., as
implemented via tags -- more clearly differentiated from memory protection.

\item We now more clearly describe how terms like ``lower bound'' and ``upper
bound'' relate to the base, offset, and length fields.

\item We now more clearly differentiate language-level capability semantics
from capability use in code generation and the ABI, considering
pure-capability and hybrid C as distinct from pure-capability and hybrid code
generation.
We explain that different language-level integer interpretations of
capabilities are supportable by the architecture, depending on compiler
code-generation choices.

\item Potential software policies for revocation, garbage collection, and
capability flow control based on CHERI primitives are described in greater
detail.

\item Monotonicity is more clearly described, as are the explicit
opportunities for non-monotonicity around exception handling and
\insnnoref{CCall} Selector 1.
Handling of disallowed requests for non-monotonicity or bypass of guarded
manipulation by software is more explicitly discussed, including the
opportunities for both exception throwing and tag stripping to maintain
CHERI's invariants.

\item Further notes have been added regarding the in-memory representation of
capabilities, including the storage of NULL capabilities, virtual addresses
for non-NULL capabilities, and how to store integer values in untagged
capability registers.
These values now appear in the bottom 64 bits of the in-memory
representation.
Topics such as endianness are also considered.

\item NULL capabilities are now defined as having a base of 0x0, the maximum
length supported in a particular representation ($2^{64}$ for 128-bit
capabilities, and $2^{64} - 1$ for 256-bit capabilities), and no granted
permissions.
NULL capabilities continue to have an all zeros in-memory representation.
This allows integers to be stored in the offset of an untagged capability
without concern that they may hold values that are unrepresentable with
respect to capability bounds.

\item New instructions \insnnoref{CReadHwr} and \insnnoref{CWriteHwr} have
been added.
These have allowed us to migrate special capability registers (SCRs) out of
the general-purpose capability register file, including \DDC{}, the new user
TLS register (\CULR{}), the new privileged TLS register (\CPLR{}), \KRC{},
\KQC{}, \KCC{}, \KDC{}, and \EPCC{}.
Access to privileged special registers continues to be authorized by the
\cappermASR permission on \PCC{}.

\item With this migration, \creg{0} is now available to use as a NULL
capability register, which is more consistent with the baseline MIPS ISA in
which \reg{0} is the zero register.
The only exception to this is in capability-relative load and store
instructions, and the \insnref{CTestSubset} instruction, in
which an operand of \creg{0} specifies that \DDC{} should be used.

\item Various instruction pseudo-ops to access special registers, such as
\insnnoref{CGetDefault}, now expand to special capability register access
instructions instead of capability move instructions.

\item With consideration of merged rather than split integer and capability
register files for RISC-V and x86-64, and a separation between
general-purpose capability registers and special capability registers (SCRs) on 64-bit MIPS, we
avoid describing the integer register file as the ``general-purpose register
file''.
We describe a number of tradeoffs around ISA design relating to using a
split vs. merged register file; avoiding the use of specific capability
registers as special registers assists in supporting both register-file
approaches.

\item The CPU reset state of various capability registers is now more clearly
defined.
Most capability registers are initialized to NULL on reset, with the
exception of \DDC{}, \PCC{}, \KCC{}, and \EPCC{}.
These defaults authorize initial access to memory for the boot process, and
are designed to allow CHERI-unaware code to operate oblivious to the
capability-system feature set.

\item We more clearly describe design choices around failure-mode choices,
including throwing exceptions and clearing tag bits.
Here, concerns in conclude stylistic consistency with the host architecture,
potential use cases, and interactions with the compiler and operating
system.

\item In general, we now refer to software-defined permissions rather than
user-defined permissions, as these permissions without an architectural
interpretation may be used in any ring.

\item Permission numbering has been rationalized so that 128-bit and 256-bit
microarchitectural permission numbers consistently start at 15.

\item The existing permission \cappermSeal, which authorized sealing and
explicit unsealing of sealed capabilities, has now been broken out into two
separate permissions: \cappermSeal, which authorizes sealing, and
\cappermUnseal, which authorizes explicit unsealing.
This will allow privilege to be reduced where unsealing is desirable (e.g.,
within object implementations, or in C++ vtable use) by not requiring that
permission to seal for the object type is also granted.

\item The ISA quick reference has been updated to reflect new instructions, as
well as to more correctly reflect endianness.

\item We have added a reference to our recently released technical report, \textit{Capability
Hardware Enhanced RISC Instructions (CHERI): Notes on the Meltdown and
Spectre Attacks}~\cite{UCAM-CL-TR-916}, which considers the potential
interactions between CHERI and the recently announced Spectre and Meltdown
microarchitectural side-channel attacks.
CHERI offers substantial potential to assist in mitigating aspects of these
attacks, as long as the microarchitecture performs required capability
checks before performing any speculative memory accesses.

\item We have added two new instructions, Get the architectural Compartment ID
(\insnref{CGetCID}) and Set the architectural Compartment ID
(\insnref{CSetCID}), which allow information on compartments to
be passed to via architecture to microarchitecture in order to support
mitigation of side-channel attacks.
This could be used to tag branch-predictor entries to control the
compartments in which they can be used, for example.
A new Permit\_Set\_CID permission allows capabilities to delegate use of
ranges of CIDs.

\item Bugs have been fixed in the definitions of various capability-relative
load and store instructions, in which permission checks involving the
Permit\_Load, Permit\_Load\_Cap, Permit\_Store, and Permit\_Store\_Cap
permissions were not properly updated from our shift from an untagged
capability register file to a tagged register file.
All loads now require Permit\_Load.
If Permit\_Load\_Cap is also present, then tags will not be stripped when
loading into a capability register.
All stores now require Permit\_Store.
If Permit\_Store\_Cap is also present, then storing a tagged capability will
not generate an exception.

\item New Capability Set Bounds From Immediate
(\insnref{CSetBoundsImm}) and Capability Increment Offset From Immediate
(\insnref{CIncOffsetImm}) instructions have been added.
These instructions optimize global-variable setup and stack allocations by
reducing the number of instructions and registers required to adjust pointer
values and set bounds.

\item New Capability Branch if Not NULL (\insnnoref{CBNZ}) and
Capability Branch if NULL (\insnnoref{CBEZ}) instructions have
been added, which optimize pointer comparisons to NULL.

\item A new Capability to Address (\insnref{CGetAddr})
instruction allows the direct retrieval of a capability's virtual address,
rather than requiring the base and offset to be separately retrieved and added
together.
This facilitates efficient implementation of a CHERI C variant in which all
casts of capabilities to integers have virtual-address rather than offset
interpretation.
A capability's virtual address is now more directly defined when we specify
capability fields.

\item We more clearly describe \insnnoref{CCall} Selector 1 as
``exception-free domain transition'' rather than ``userspace domain
transition'', as it is also intended to be used in more privileged rings.

\item We have shifted to more consistently throwing an exception at jump
instructions (e.g., \insnref{CJR}) that go out of bounds,
rather than throwing the exception when fetching the first instruction at
the target address.
This provides more debugging information when using compressed
capabilities, as otherwise \EPCC{} might have unrepresentable bounds in the
event that the jump target is outside of the representable region.

\item The exception vectors use during failures of Selector 0 and Selector 1
\insnnoref{CCall} have been clarified.
The general-purpose exception vector is used for all failure modes with
\insnnoref{CCall} Selector 1.

\item We have added a new experimental instruction, Test that Capability is a Subset of Another
(\insnref{CTestSubset}).
This instruction is intended to be used by garbage collectors that need to
rapidly test whether a capability points into the range of another
capability.

\item A new experimental 64-bit capability format for 32-bit virtual addresses
has been added.

\item A description of an experimental {\it linear capability} model has been
added (Section~\ref{section:linear-capabilities}).
This model introduces the concept that a capability may be linear -- i.e.,
that it can only be moved rather copied in memory-to-register,
register-to-register, and register-to-memory operations.
This introduces two new instructions, Linear Load Capability Register
(\insnnoref{LLCR}) and Linear Store Capability Register
(\insnnoref{LSCR}).
This functionality has not yet been fully specified.

\item An experimental appendix considers possible implementations of {\it
indirect capabilities}, in which a capability value points at an actual
capability to utilize, allowing table-based capability lookups
(Section~\ref{section:indirect-capabilities}).

\item An experimental appendix considering potential forms of compression for
capability permissions has been added (Section~\ref{app:exp:compressperm}).

\item We have added a reference to our ICCD 2017 paper, \textit{Efficient
Tagged Memory}, which describes how to efficiently implement tagged memory in
memory subsystems not supporting inline tags directly in
  DRAM~\cite{joannou2017:tagged-memory}.
\end{itemize}
