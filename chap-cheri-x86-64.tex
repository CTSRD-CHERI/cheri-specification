\chapter{The CHERI-x86-64 Instruction-Set Architecture (Sketch)}
\label{chap:cheri-x86-64}

\rwnote{New introduction is required, and some change of pitch.}

In this chapter, we explore models for applying CHERI protection to the x86
architecture.
The x86 architecture is a widely deployed CPU architecture used in a
variety of applications ranging from mobile to high-performance computing.
The architecture has evolved over time from 16-bit processors without
MMUs to present-day systems with 64-bit processors supporting virtual
memory via a combination of segmentation and paging.

The x86 architecture has spanned three register sizes (16, 32, and
64 bits) and multiple memory management models.  We choose to define
CHERI solely for the 64-bit x86 architecture for a variety of reasons
including its more mature virtual-memory model, as well as its larger
general-purpose integer register file.

\section{Capability Registers versus Segments}

The x86 architecture first added virtual memory support via
relocatable and variable-sized segments.  Each segment was assigned a
mask of permissions.  Memory references were resolved with respect to a
specific segment including relocation to a base address, bounds
checking, and access checks.  Special segment types permitted transitions
to and from different protection domains.

These features are similar to features in CHERI capabilities.
However, there are also some key differences.

First, x86 addresses are stored as a combination of an offset and a
segment spanning two different registers.  General-purpose registers
are used to hold offsets, and dedicated segment selector registers are
used to hold information about a single segment.  The x86 architecture
provides six segment selector registers -- three of which are reserved
for code, stack, and general data accesses.  A fourth register is
typically used to define the location of thread-local storage (TLS).
This leaves two segment registers to use for fine-grained segments
such as separate segments for individual stack variables.  These
registers do not load a segment descriptor from arbitrary locations in
memory.  Instead, each register selects a segment descriptor from a
descriptor table with a limited number of entries.  One could treat
the segment descriptor tables (or portions of these tables) as a cache
of active segments.

Second, more fine-grained segments are not derived from existing
segments.  Instead, each entry in a descriptor table is independent.
Write access to a descriptor table permits construction of arbitrary
segments (including special segments that permit privilege
transitions).  Restricting descriptor-table write access to kernel
mode does not protect against construction of arbitrary segments in
kernel mode due to bugs or vulnerabilities.  As a result, segment
descriptors are not able to provide the same provenance guarantees as
tagged capabilities.

Third, existing segment descriptors do not have available bits for
storing types or permissions more expressive than the existing
read, write, and execute.

Finally, x86 segmentation is typically not used in modern operating
systems.  On the 32-bit x86 architecture, systems generally create
segments with infinite bounds and use a non-zero base address only
for a single segment that provides TLS.  The 64-bit x86 architecture
codifies this by removing segment bounds entirely and supporting non-zero-base
addresses only for two segment registers.
Software for x86 systems stores only the offset portion of virtual
addresses in pointer variables.  Segment registers are set to fixed
values at program startup, never change, and are largely ignored.

One approach for providing a similar set of features to CHERI
capabilities on x86 would be to extend the existing segment primitives
to accommodate some of these differences.  For example, descriptor-table
entries could be tagged, whereby loading an untagged segment would trigger
an exception.  However, some other potential changes are broader in
scope (e.g., whether segment selectors should contain an index into a
table, versus a logical address of a segment descriptor).  Extending
segments would also result in a very different model compared to CHERI
capabilities on other architectures, limiting the ability to share code
and algorithms.  Instead, we propose to add CHERI capabilities to 64-bit
x86 by extending existing general-purpose integer registers.

\section{Tagged Capabilities and Memory}

As with CHERI-MIPS and CHERI-RISC-V, we recommend that both memory and
registers contain tagged capabilities.  Similar to CHERI-RISC-V, we also
recommend a single, 128-bit format for CHERI-x86-64 capabilities.

\section{Extending Existing Registers}

The x86 architecture has expanded its general-purpose integer registers multiple
times.  Thus, the 16-bit \AX{} register has been extended to 32-bit \EAX{}
and 64-bit \RAX{}.
We propose extending each general-purpose integer register to a tagged, 128-bit register
able to contain a single capability.
The capability-sized registers would be named with a `C' prefix in place
of the `R' prefix used for 64-bit registers
(\CAX{}, \CBX{}, etc.).
As with CHERI-RISC-V,
we recommend that the bottom 64 bits of capability registers contain
the integer value (virtual address) and the upper 64 bits contain
capability metadata.
Reads of capability registers as integers return the integer value.
Integer writes to capability registers
should clear the tag and upper 64 bits of capability metadata, storing the
desired integer value in the bottom 64 bits.

The \RIP{} register (which contains the address of the current instruction)
would also be extended into a \CIP{} capability.  This would function as
the equivalent of \PCC{} for CHERI-MIPS and CHERI-RISC-V.  As with
those architectures, the
value of \RIP{} should be the current offset of \CIP{} rather than the
integer value (virtual address).

\section{Additional Capability Registers}

Additional capability registers beyond those present in the general-purpose
integer
register set will also be required.

A new register will be required to hold \DDC{} for controlling
non-capability-aware memory accesses.

The x86 architecture currently uses the \FS{} and \GS{} segment selector registers
to provide thread-local storage (TLS).  In the 64-bit x86 architecture,
these selectors are mostly reduced to holding an alternate base address
that is added as an offset to the virtual address of existing instructions.
For CHERI-x86-64 we recommend replacing these segment registers with two
new capability registers: \CFS{} and \CGS{}.

In addition, new capability registers may be required to manage user
to kernel transitions as detailed below.

\section{Using Capabilities with Memory Address Operands}

As with other CHERI architectures, CHERI-x86-64 should support running existing
x86-64 code, capability-aware code, and hybrid code.  This
requires the architecture to support multiple addressing modes.
The x86 architecture has implemented this in the past when it was
extended to support 32-bit operation.  We propose to reuse some of the
same infrastructure to support a new capability-based addressing
mode.

When x86 was extended from 16 bits to 32 bits, the architecture
included the ability to run existing 16-bit code without modification
as well as execute individual 16-bit or 32-bit instructions within a
32-bit or 16-bit codebase.  The support for 16-bit versus 32-bit
operation was
split into two categories: operand size and addressing modes.  The
code segment descriptor contains a single-bit `D' flag, which sets the
default operand size and addressing mode.  These attributes can then
be toggled to the non-default setting via opcode prefixes.  The 0x66
prefix is used to toggle the operand size, and the 0x67 prefix is used
to toggle the addressing mode.

In 64-bit (``long'') mode, the `D' flag is currently always set to
0 to indicate 32-bit operands and 64-bit addressing.  A value of
1 for `D' is reserved.  The 0x67 opcode prefix is used to toggle
between 32-bit and 64-bit addresses, but a few other single-byte opcodes
are invalid in 64-bit mode and could be repurposed as a prefix.

We propose a new capability-aware addressing mode that can be
toggled via the `D' flag of the current code segment and a new 0x62
opcode prefix.  (In 32-bit x86, the 0x62 opcode is the
\insnnoref{BOUND} instruction, which is invalid in 64-bit mode.)
If the `D' flag of a 64-bit code segment is set to 1,
then the CPU would execute in ``capability mode'' -- which would include
using the capability-aware addressing mode by default.  Individual
instructions could toggle between capability-aware and ``plain''
64-bit addressing via the 0x62 opcode prefix.  Addresses using the
``plain'' 32-bit or 64-bit addressing would always be treated as offsets
relative to \DDC{}.  Instructions using capability-aware addressing
would always use 64-bit virtual addresses and ignore any 0x67 opcode
prefix.

Note that one can change the value of \CS{} in user mode (for example,
a user process in FreeBSD/amd64 can switch between 32 and 64-bit by
using a far call that loads a different value of \CS{}).  This would mean
that user code could swap into pure-capability mode without requiring
a system call.  However, this would not alter the contents of
capability registers or their enforcement, merely the decoding of
instructions.  If \DDC{} is invalid, then sandboxed code that switched to
a non-capability \CS{} would still require valid capability registers to
access memory.

\subsection{Capability-Aware Addressing}

For instructions with register-based memory operands, capability-aware
addressing would use the capability version of the register rather
than the virtual address relative to \DDC{}.

For example:

\begin{verbatim}
mov 0x8(%cbp),%rax
\end{verbatim}

would read the 64-bit value at offset 8 from the capability described
by the \CBP{} register.

On the other hand,

\begin{verbatim}
mov 0x8(%rbp),%rax
\end{verbatim}

would read the 64-bit value at an offset of RBP+8 from the \DDC{} capability.
Both instructions would use the same opcode aside from the addition of
an 0x62 opcode prefix.  In a code segment with `D' set to 1, the second
instruction would require the prefix.  In a code segment with `D' set to 0,
the first instruction would require the prefix.

\subsection{Scaled-Index Base Addressing}

x86 also supports an addressing mode that combines the values of two
registers to construct a virtual address known as scaled-index base
addressing.  These addresses use one register, the \emph{base}, and a
second register, the \emph{index}, multiplied by a scaling factor of 1, 2,
4, or 8.  For these addresses, capability-aware addresses would select
a capability for the base register, but the index register would use
the integer value of the register.  For example:

\begin{verbatim}
mov (%rax,%rbx,4),%rcx
\end{verbatim}

This computes an effective address of \RAX{} + \RBX{} * 4 and loads the value
at that address into \RCX{},  The capability-aware version would be:

\begin{verbatim}
mov (%cax,%rbx,4),%rcx
\end{verbatim}

That is, starting with the \CAX{} capability, \RBX{} * 4 would be added to the
offset, and the resulting address validated against the \CAX{} capability.

\subsection{RIP-Relative Addressing}

The 64-bit x86 architecture added a new addressing mode to support more
efficient Position-Independent Code (PIC) performance.
This addressing mode uses an immediate offset
relative to the current value of the instruction
pointer.  These addresses are known as \RIP{}-relative addresses.

To support existing code, \RIP{}-relative addresses should be resolved
relative to \DDC{} when using ``plain'' 64-bit addressing.
Specifically, the value of \RIP{} (offset of \CIP{}) would be added to
the immediate offset.  The resulting value would then be used as an
offset relative to \DDC{} for the load or store.

When capability-aware addressing is used, \RIP{}-relative addresses
should be resolved relative to \CIP{}.
The immediate offset is applied to \CIP{} and the load
or store is constrained by the bounds and permissions of \CIP{}.

\subsection{Using Additional Capability Registers}

The proposed capability-aware addressing mode proposed above allows
for the capability versions of existing general-purpose integer registers such
as \CAX{} or \CBP{} to be encoded in existing register instructions.
However, it does not permit the direct use of the additional
capability registers \DDC{}, \CFS{}, or \CGS{}.  \DDC{} is not expected to be
used as an explicit base address, but \CFS{} and \CGS{} must be usable in this
manner to support TLS with capability-aware addresses.

One option would be to repurpose the existing \FS{} and \GS{} segment
prefixes when used with instructions using capability-aware addresses
to select an implicit base register of \CFS{} or \CGS{}, respectively.
However, this approach is potentially confusing.  Would an instruction
using an existing address of ``(\%cax)'' and an instruction prefix of
``GS:'' simply use the integer value of \CAX{} (value of \RAX{}) as an offset
relative to \CGS{}?  In addition, instructions that manipulate
capabilities need a way to specify an additional capability register
as an operand.

To handle both of these cases, we propose to reuse the existing \FS{} and
\GS{} segment prefixes to extend the capability register selector field
in opcodes.  This is similar to the use of bits in \REX{} prefixes to
extend the general-purpose integer register selector fields in other
instructions.  Instructions with memory addresses will use at most one
capability-register, and the \FS{} prefix could be used to select
capability registers with an index of 32 or higher.  For instructions
operating on two capability registers, the \FS{} prefix would affect the
register selected for the first capability register operand, and the
\GS{} prefix would affect the register selected for the second capability
register operand.  Additional capability registers such as \DDC{}, \CFS{},
and \CGS{} would be assigned register indices starting at 32 and require
a suitable prefix.

\subsection{Instructions with Implicit Memory Operands}

Some x86 instructions have implicit memory operands addressed by a
register.  These instructions should support addressing memory with
capabilities.

The ``string''
instructions use \RSI{} as source address and \RDI{} as a destination address.
For example, the
\insnnoref{STOS} instruction stores the value in \AL{}/\AX{}/\EAX{}/\RAX{} to the address in
\RDI{}, and then either increments or decrements the destination
index register (depending on the Direction Flag).  When capability
addressing mode is enabled,
these string instructions should use \CSI{} instead of \RSI{} and \CDI{} instead of
\RDI{}.

Instructions that work with the stack such as \insnnoref{PUSH}
or \insnnoref{CALL} use the stack pointer as an implicit
operand.  In plain 64-bit mode, \RSP{} would be treated as an offset
into \DDC{} to compute the stack pointer.  In ``capability mode'',
these instructions would use \CSP{} as the stack pointer.  Hybrid code
executing in plain 64-bit mode that uses \CSP{} as the stack pointer
would simulate these instructions using \insnnoref{MOV}
instructions and adjusting the offset of \CSP{}.  Similarly,
``capability mode'' instructions using \RSP{} as the stack pointer
would use \insnnoref{MOV} combined with adjustments to \RSP{}.

\section{Capability-Aware Instructions}

\subsection{Control-Flow Instructions}

Existing control-flow operations such as \insnnoref{JMP},
\insnnoref{CALL}, and \insnnoref{RET} would modify the offset of the
\CIP{} capability as well as verify that the new offset is valid.

New instructions would be required when performing a control-flow
operation that loads a full \CIP{} capability.  For example, a new
\insnnoref{CJMP} instruction would accept either a capability register
or an in-memory capability as its sole argument and load the new capability
into \CIP{} similar to the CHERI-RISC-V \insnriscvref{CJR} instruction.
New \insnnoref{CCALL} and \insnnoref{CRET} instructions would be
used for function calls that push the full \CIP{} capability onto the stack
as the return address.

One option for these new instructions may be to use an opcode prefix
to toggle between loading a \RIP{} offset versus a full \CIP{}
capability.  In that case, the default mode of an instruction may be
governed by the `D' flag of the current code segment to permit compact
instruction encoding for capability-aware software.  Note that since
\insnnoref{CALL} and \insnnoref{JMP} may accept a memory
operand, this opcode prefix would need to be a different prefix than
the 0x62 used to toggle addressing modes.

Another option that avoids the use of a new opcode prefix would be to
offer new instructions that always operate on \CIP{} and choose a
subset of comonly-used but shorter opcodes that default to loading a
full \CIP{} instead of \RIP{} in ``capability mode''.

The implicit stack pointer operand in \insnnoref{CALL} and
\insnnoref{RET} also bears consideration.  To avoid an
explosion of new opcodes while also providing compact enoding for
common uses, these instructions would use \CSP{} as the stack pointer
in ``capability mode'' and \RSP{} as the stack pointer in plain 64-bit
mode.  Code that needed to use an alternate stack pointer would use
explicit stack manipulation along with \insnnoref{JMP}.

\subsection{Manipulating Capabilities}

New instructions will need to be defined to support capability
manipulations similar to other CHERI architectures.

New \insnnoref{MOV} variants could handle loading and storing of
capabilities similar to \insnref{CLC} and \insnref{CSC}.

NULL-derived capabilities with constant values can be loaded into
registers using existing \insnnoref{MOV} instructions which
write into general purpose registers since those instructions will
clear the upper 64-bits of metadata as a side effect.  Arbitrary
null-derived capabilities can be stored to memory as a two instruction
sequence where the first instruction uses an existing
\insnnoref{MOV} of an immediate value into a general purpose
register and the second stores the aliased capability register value
into the destination.

However, storing NULL pointers into memory is a frequent operation
which might warrant optimization.  A few alternatives include:

\begin{enumerate}
  \item Use a two instruction sequence where the first instruction is
    a well-known instruction for generating a zero value.  Modern
    x86 CPUs complete such instructions during decoding so only the
    second instruction would use execution pipeline resources.
    However, this requires using a general purpose register to hold
    the temporary zero value.

  \item Add a read-only \CNULL register in the additional bank of
    capability registers which can be used as the source of a
    capability-sized \insnnoref{MOV}.

  \item Add a new \insnnoref{MOVNULL} instruction which stores
    a NULL capability to the destination.
\end{enumerate}

A new variant of \insnnoref{CMPXCHG} will be required to support
atomic operations on capabilities.  (Note that \insnnoref{CMPXCHG16B}'s
existing semantics are not suitable for capabilities
as it divides the values into register pairs.)  It may also be
desirable to define a capability variant of \insnnoref{XADD}.

Variants of \insnnoref{PUSH}/\insnnoref{POP} could be used to
save and restore capability registers on the stack -- if those operations
are common enough to warrant new instructions rather than using
capability \insnnoref{MOV} instructions paired with adjustments to
\CSP{}.

For other capability operations such as \insnref{CIncOffset}, we
propose adding new CHERI-specific instructions rather than
repurposing existing instructions such as \insnnoref{ADD}.
Existing general-purpose x86 instructions support two operands rather
than three operands.  To avoid introducing a new operand encoding
format, we propose to use two-operand variants of CHERI
instructions when adapting instructions to x86.

The \insnnoref{LEA} instruction warrants additional consideration.
When capability mode addressing is enabled for this instruction, the
effective address should be a capability stored in a capability
destination register.  This would permit the use of \insnnoref{LEA}
to perform some operations such as \insnref{CIncOffset}, without
overwriting the source capability register.

Some x86 instructions such as \insnnoref{MOV} have variants
which accept operands of differing sizes.  Assemblers and
disassemblers can use suffixes such as \texttt{b}, \texttt{w},
\texttt{l}, and \texttt{q} to explicitly state the operand size.  We
recommand that the \texttt{c} suffix be used to explicitly state
operands which are capability-sized.

\subsection{Inspecting Tags}

CHERI-MIPS provides dedicated \insnref{CBTU} and
\insnref{CBTS} instructions for conditional branches.  For x86, a
more natural model may be to add a new ``tagged'' flag to the
\RFLAGS{} register.  Any instruction that stored a capability such as
capability-aware variants of \insnnoref{MOV} would set this
flag to the tag bit of the stored capability.  New conditional jump
instructions would be added that tested this flag.

\section{Capability Violation Faults}

For reporting capability violations, we propose reserving a new
exception vector.  This new exception would report an error code
pushed as part of the exception frame similar to GP\# and PF\# faults.
This error code would contain the capability exception code as
described in Section~\ref{sec:capability_exception_causes} to indicate
the specific violation.

CHERI-MIPS and CHERI-RISC-V include the name of the register which
triggers a capability violation.  It is not feasible to provide a
direct analog of this on x86.  Indirect jumps and calls may raise an
exception while loading a capability from memory that is not present
in any register at the start of the instruction.  However, unlike page
faults, capability violation faults are not generally restartable and
the register name's primary use is for debugging convenience rather than
correctness.  There are a few possible options for providing similar
information:

\begin{enumerate}
\item Provide a copy of the faulting capability via one of the
  currently-unused but reserved control registers, such as \CRFIVE{}
  or \CRTWELVE{} -- similar to the PF\# virtual address stored in
  \CRTWO{}.  This faulting capability would include the result of any
  offset adjustments from immediates or scaled indices.  If the result
  of offset adjustments made the capability unrepresentable, the
  faulting capability would be a null capability holding the computed
  virtual address.
\item Similar to the above, but ignore offset adjustments and only
  provide the base capability value.
\item Provide the virtual address from the faulting capability in
  \CRTWO{} similar to PF\#.  A debugger could examine the faulting
  instruction's operands to determine which capability triggered the fault.
\item Do nothing as the prior approaches may be too expensive to
  implement.
\end{enumerate}

\section{Interrupt and Exception Handling}

For interrupt and exception handling, we propose a new overall CPU
mode that enables the use of capabilities.  The availability of this
mode would be indicated by a new \insnnoref{CPUID} flag.  The
mode would be enabled by setting a new bit in \CRFOUR{}.  When this mode
is enabled, exceptions would push a new type of interrupt frame that
would replace \RIP{} with the full \CIP{} capability, and \RSP{} with the full
\CSP{} capability.  \insnnoref{IRET} would be modified to unwind this expanded stack
frame.

Interrupt and exception handlers require new capabilities for the
program counter (\CIP{}) and stack pointer (\CSP{}) registers.  We consider
two possible approaches.

\subsection{Kernel Code and Stack Capabilities}

The first approach would add two new control registers: the Kernel
Code Capability (\KCC{}) and Kernel Stack Capability (\KSC{}).  Access to
these registers would be restricted to supervisor mode.  These new registers
could be named as instruction operands, using the same approach decribed
earlier for \CFS{} and \CGS{}.

Transitions into
supervisor mode would load new offsets relative to \KCC{} and \KSC{} from
existing data structures and tables to construct the new \CIP{} and \CSP{}
register values.  For example, the current virtual address stored in
each Interrupt Descriptor Table (\IDT{}) entry would be used as an offset
relative to \KCC{} to build \CIP{}, and the address stored in the Interrupt
Stack Table (\IST{}) entry in the current Task-State Segment (\TSS{}) would
be used as an offset relative to \KSC{} to build \CSP{}.  Transitions via
the \insnnoref{SYSCALL} instruction would use the offset from \LSTAR{} to
construct the new \CIP{}.

This approach does require broad capabilities
for \KCC{} and \KSC{} that can accommodate any desired entry point or stack
location.  However, it will require minimal changes to existing systems
code such as operating-system kernels.

\subsection{Capabilities in Entry Points}

The second approach would be to replace virtual addresses stored in
existing entry points with complete capabilities.  This is a more
invasive change, requiring larger changes to existing systems code, but
it enables the use of more fine-grained capabilities for each entry
point.

Setting the desired kernel stack pointers \CSP{} would require a new
\TSS{} layout that expanded the existing \RSP{} and \IST{} entries to
capabilities.

For \insnnoref{SYSCALL}, a new control register \CSTAR{} could be
added to hold the target instruction pointer.  As with \KCC{},
this register would be a privileged register in the same bank as
\CFS{} and \CGS{}.

Entries in the \IDT{} would be expanded to 32-bytes, appending a capability
code pointer in the last 16 bytes.  This would double the size of the
\IDT{}, and most of the bytes would be unused.  However, it would
ensure that all of the information currently stored in an \IDT{} entry
(such as the segment selector, \IST{} index, and descriptor type) would
be configurable.

\subsection{\insnnoref{SWAPGS} and Capabilities}

The \insnnoref{SWAPGS} instruction is used in user-to-kernel
transitions for the 64-bit x86 architecture to permit separate TLS
pointers for user and kernel mode.  One option would be to provide a
capability version of \insnnoref{SWAPGS}, either by extending the
\KGSBASE{} MSR to a capability, or adding a new MSR.  However, this
instruction can be difficult to use.  Interrupt and exception handlers
must be careful not to invoke \insnnoref{SWAPGS} if the interrupt
or exception is taken while executing the kernel mode \GS{}.  We
recommend avoiding the use of \insnnoref{SWAPGS}, and instead defining
a new privileged control register \KGS{}.  Operating systems could
either choose to use \KGS{} to initialize \CGS{} in interrupt and
exception handlers, or else use \KGS{} directly as the kernel mode TLS
pointer.

\section{Page Tables}

Similar to CHERI on other architectures, additional page-table
permission bits governing loads and stores of capabilities are
desirable.  In addition, it may be beneficial to have a ``capability
dirty'' bit.  At present the 64-bit x86 architecture has reserved bits
in a range from bit 52 (\texttt{MAXPHYADDR}) to bit 62.  The Protection Keys
extension uses bits 59-62 from that range.  To avoid conflicting with
Protection Keys, CHERI-x86-64 could use bits starting at bit 58 as described in Table~\ref{table:x86:pte}.  Higher bits are
preferred, to permit maximal room for growth of the physical address
field that currently ends at bit 51.

\begin{table}
\begin{center}
\begin{tabular}{lll}
\toprule
Bit & Name & Description \\
\midrule
58 & CW & Permits writes of tagged capabilities \\
57 & CR & Permits reads of tagged capabilities \\
56 & CD & Set when a tagged capability is written to this page \\
\bottomrule
\end{tabular}
\end{center}
\caption{CHERI-x86-64 Page Table Bits}
\label{table:x86:pte}
\end{table}

If an instruction performs a memory access which violates a CHERI page
permission (such as a store of a tagged capability to a page where the
\texttt{CW} bit is clear), a page-fault (PF\#) exception should be
raised.  Bit 6 (currently reserved) should be set in the page-fault
error code provided by the processor indicating that the fault was
caused by a capabilty permission violation.  Other bits in the page
fault error code such as \texttt{P}, \texttt{W/R}, \texttt{U/S}, and
\texttt{I/D} should be set to indicate the type of memory access.  In
addition, the virtual address of the memory access should be provided
in the \CRTWO{} register similar to other page-fault exceptions.

Note that the \texttt{CR} and \texttt{CW} bits only fault if the
capability being read or written is tagged.  Untagged capability
values can be read from or written to memory regardless of the
\texttt{CR} and \texttt{CW} permissions.  In addition, if the
authorizing capability for a capability read does not hold \cappermLC,
then reading a tagged capability will always return a capability with
the tag cleared instead of faulting.

Instruction fetches always ignore tags and will never raise a
capability page-fault exception.

\section{Capabilities and Integer Instructions}

Throughout this chapter, we have proposed that CHERI-x86-64 should
assume that existing integer instructions writing to a capaiblity
register should always strip the tag with the exception of
\insnnoref{LEA} in capability mode.  This approach offers the
following benefits:

\begin{itemize}
\item
It avoids ``implicit'' operations on capabilities.
\item
It is consistent with 32-bit instructions zero-extending results when
stored in 64-bit registers in long mode.
\item
Micro-architecturally it permits integer instructions to always assume
an untagged output.
\end{itemize}

However, an alternative approach would be for integer instructions to
instead perform the requested integer operation against the virtual
address, and strip the tag only if the new virtual address results in
unrepresentable bounds.  In this model, existing instructions such as
\insnnoref{ADD} or \insnnoref{SUB} could be used in
place of \insnref{CIncOffset}.  This would offer the following
benefits:

\begin{itemize}
\item
Pointer manipulations may be able to use shorter instructions, because
new capability-specific instructions will all likely be using 2-
and 3-byte opcodes.
\item
Fewer new opcodes may be required, because some capability instructions
might be fully implemented via existing instructions.   
\item
Compilers may be able to reuse existing code sequences for function
prologues and epilogues.
\end{itemize}
