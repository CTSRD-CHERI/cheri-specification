This version of the \textit{CHERI Instruction-Set Architecture} is a full
release of the Version 8 specification:

\begin{itemize}
\item We have performed modest updates to discuss Arm's Morello processor,
  System-on-Chip (SoC), and board.
  The authoritative reference to Morello is Arm's Morello
  specification~\cite{arm-morello}.

\item We have added a new chapter, Chapter~\ref{chap:microarchitecture},
  describing the impact on CHERI on practical microarchitecture at a high
  level.
  It considers topics such as the impact of capabilities on the pipeline and
  register file, efficient implementation of bounds compression and
  decompression, fast bounds checking, tagged memory, and the potential
  interaction with speculative execution.
  This includes insights gained during the building of Arm's Morello
  processor and SoC.

\item Shift away from the idea that the fully precise 256-bit capabilities
  are the essential model for CHERI.
  Instead, describe capabilities as an architectural type made up of a set
  of architectural fields, which may be constrained in terms of precision, and
  that have an in-memory representation.
  The microarchitecture may hold capabilities in another format internally
  (e.g., when loaded in registers).

\item The CHERI Concentrate description has been improved, including adding
  information about the Fast Representability Check.
  A number of constants have been updated.

\item In several chapters, more care is taken when using the words
  ``capability,'' ``pointer,'' and ``address,'' which are not interchangeable.

\item Throughout, be more clear that CHERI applies to 32-bit architectures,
  not just 64-bit architectures.
  In Chapter~\ref{chap:architecture}, introduce \xlen{} and \clen{}
  terminology previously only used for CHERI-RISC-V, to better abstract away
  from specific address lengths.

\item We are now more clear that capabilities may describe physical as well as
  virtual addresses, and that virtual addressing may be implemented either
  using software-managed TLBs (as on MIPS) or via architectural page tables
  (as on RISC-V and ARMv8-A).

\item The \insnnoref{CCall} instruction has been replaced with
  \insnref{CInvoke}, which does not have a selector mechanism and supports
  only exception-free domain transition.
  The prior exception-based mechanism was introduced as scaffolding used
  during domain-transition research, and the exception-free mechanism is our
  preferred approach.
  We have removed capability exception cause codes previously used
  for that purpose, and will likely deprecate \insnnoref{CSetCause}, which was
  used only for exception-based domain transition.

\item The deprecated \insnnoref{CCheckPerm} and \insnnoref{CCheckType}
  instructions have been removed as they were intended to be used with the
  exception-based \insnnoref{CCall} mechanism.

\item We have removed the experimental 64-bit capability format based on our
  older CHERI-128 compression model.
  We now document a 64-bit capability format as part of CHERI Concentrate in
  Section~\ref{section:architectural-capabilities}.

\item We now advocate for a policy of specifying a minimum capability bounds
  precision, even with \insnref{CRRL} and \insnref{CRAM} instructions; see
  Section~\ref{sec:the-value-of-architectural-minimum-precision}.

\item We now more fully explore, and explain, exception throwing around
  capability load and store permissions on capabilities and on page-table /
  TLB entries.
  MMU-originated exceptions are now distinct from CHERI exceptions.
  Whereas a capability load through a capability without \cappermLC will always strip
  the tag, a capability load via a page without load capability permission
  will either strip the tag or throw an exception.
  Similarly, a store via a page without permission to store a capability will
  throw an exception if the tag bit is set on the capability being stored.

\item We now discuss the use of per-page capability load barriers to
  enable efficient capability revocation and garbage collection in
  Section~\ref{section:capability-load-barriers}.  Load
  barriers use additional MMU permissions to selectively trap on
  capability loads.

\item \insnnoref{CPtrCmp} has been changed to compare only the addresses of
  the two capabilities, and no longer consider the tag bit, for non-exact
  comparisons.
  The previous behavior could result in surprising run-time failures of C/C++
  programs.

\item Sentry capabilities are no longer considered experimental.

\item The instruction \insnref{CSealEntry} has been added to the ISA quick
  references, from which it was missing.

\item We now more completely elaborate how sentry capabilities interact with
  exception handling, including automatic unsealing of sentries when installed
  in exception program counters, not just when they are jumped to.

\item Two new instructions (\insnnoref{CGetPCCIncOffset} and
  \insnnoref{CGetPCCSetAddr}) have been added to improve code density when
  generating code to create \PCC{}-derived capabilities for code or data
  access.
  This is of particular use when utilizing sentry capabilities.

\item The \insnref{CRRL} and \insnref{CRAM} instructions, which assist with
  capability bounds alignment and padding, are no longer considered
  experimental.

\item The \insnref{CLoadTags} instruction is no longer considered
  experimental, and has now also been defined for CHERI-RISC-V.

\item We have removed a note on \insnnoref{CSeal}
  regarding representable unsealed capabilities being unrepresentable as
  sealed capabilities: all representable capabilities can now be sealed.

\item We have a removed a note on \insnnoref{CIncOffset} regarding a special
  case in which offset of 0 is permitted to operate on sealed capabilities,
  allowing \insnnoref{CIncOffset} to be used to implement
  \insnnoref{CMove} as a pseudo-instruction.
  This is no longer required, as \insnnoref{CMove} is now its own
  instruction to avoid this special casing.

\item \insnnoref{CIncOffset} is no longer specified to clear the base and
  length if the bounds become unrepresentable, as we guarantee that the cursor
  will hold the arithmetic result rather than the offset.

\item \insnref{CSetFlags} no longer incorrectly notes that an exception
  is thrown if the argument is untagged.

\item We have added a new chapter, Chapter~\ref{chap:isaref-riscv}, listing the
  semantics and encodings of CHERI-RISC-V instructions.

\item Capability flags, and their associated instructions,
  \insnref{CGetFlags} and \insnref{CSetFlags}, are no longer considered
  experimental.
  On CHERI-RISC-V, a capability flag is used to control what instruction
  endcoding mode is active.
  For details, see
  \cref{sec:model-flags,sec:arch-flags,sec:cheri-riscv-encmodes}.

\item Certain CHERI-RISC-V SCRs are now defined to exist only when their
  corresponding extensions (e.g., the N extension) is present.

\item In CHERI-RISC-V, when a special capability register is written using
  \insnriscvref{CSpecialRW}, and some bits in the register are defined as WARL
  (i.e., may be modified during the write), the tag bit will be cleared if the
  capability value is sealed.

\item A new Unaligned Base CHERI exception code has been defined, allowing
  CHERI-RISC-V to throw an exception if the installed \PCC{} value has an
  unaligned base.

\item CHERI-RISC-V encodings are now defined for \insnriscvref{CRRL} and
  \insnriscvref{CRAM}, and encoding space is reserved for a future implementation of
  \insnriscvref{CClear} when using a split register file.

\item CHERI-RISC-V encodings have been added for the
  \insnriscvref{CSetEqualExact}, \insnriscvref{CLoadTags}, and
  \insnriscvref{CClearTags} instructions.

\item The CHERI-RISC-V exception cause code has changed from 0x20 to 0x1c
  due to a collision with encoding space reserved for future use.

\item We now define a set of CHERI-RISC-V atomic instructions corresponding to
  the equivalent base RISC-V atomic instructions.

\item We now document which RISC-V CSRs and FCSRs are white listed to not
  require \cappermASR, such as the cycle counter.

\item Capability exception cause codes are now properly architecture neutral,
  but the mechanism for obtaining them is more accurately architecture
  specific.

\item CHERI-RISC-V now reports capability-related exception information via
  the existing RISC-V Trap Value CSRs, \xtval{}, rather than through a new
  capability cause register (a design inherited from CHERI-MIPS that is less
  consistent with the baseline RISC-V design).

\item We added two new exception codes for CHERI-related MMU faults to
  CHERI-RISC-V.  For these exceptions, \xtval{} holds the address of
  the faulting memory reference as with existing MMU faults on RISC-V.

\item We added additional PTE bits to CHERI-RISC-V to support
  capability revocation.  \texttt{CD} provides a capability dirty bit
  to track pages holding capabilities.  \texttt{CRM} and \texttt{CRG}
  enable per-page capability load barriers.

\item We document the CHERI-RISC-V \insnriscvref{CRET}, \insnriscvref{CJR}, and
  \insnriscvref{CJALR} assembly aliases.

\item We have added a CHERI-RISC-V assembly programming section to the
  CHERI-RISC-V ISA quick reference.

\item There have been numerous updates to our CHERI-x86-64 architectural
  sketch.  The page table permission bits have been adjusted to avoid
  conflicting with the Protection Keys extension.  Violations of the
  CHERI page table permissions now raise a page fault rather than a
  CHERI fault.  The read capability permission now faults if violated
  rather than stripping tags.  We have removed an ambiguous suggestion
  for handling RIP-relative addressing.

\item The interaction of CHERI with existing memory versioning schemes
  (e.g., Arm MTE) in \cref{app:exp:versioning} is now more fully articulated
  and includes support for integer-pointer versioning instructions as well as
  a new atomic instruction to manipulate version values in memory.

\item A new experimental instruction, \insnref{CLCNT}, has been proposed to
  perform a non-temporal (streaming) load of a capability via a capability.
  This instruction may prove useful when scanning memory for capabilities in
  order to implement revocation.
  We have not yet validated this approach through a full-stack implementation.

\item A new experimental instruction, \insnref{CClearTags}, has been proposed
  to perform fast zeroing of multiple tags in memory, and will not allocate
  cache lines if data is not already present in a cache.
  This instruction may prove useful when rapidly clearing capabilities for
  revocation purposes, in the absence of data confidentiality requirements.
  We have not yet validated this approach through a full-stack implementation.

\item A new experimental protection-domain transition mechanism,
  \textit{sealed indirect enter capabilities}, has been proposed to allow
  a sealed entry capability to carry not just a pointer to domain-specific
  code, but also to domain-specific data, by adding an additional level of
  indirection.
  A new instruction, \insnnoref{CInvokeInd}, would be used to invoke a
  sealed indirect enter capability.
  We have not yet validated this approach through a full-stack implementation.
  This is described in Section~\ref{app:exp:indsentry}.

\item A new \emph{ephemeral capability} type is proposed.
  Ephemeral capabilities can be held in registers but not stored to memory,
  and we consider the implications for fast cross-domain calls, and an
  explicitly maintained delegation tree to provide prompt revocation of
  delegation sub-trees.
  We have not yet validated this approach through a full-stack implementation.
  This is described in Section~\ref{app:exp:hierarchal-evocation}.

\item A new \emph{anti-tamper seal} mechanism is proposed, to allow
  validation that a delegated capability that has been returned has not been
  modified (other than its address) while in use.
  The suggested use case is around memory allocators, to ensure that a pointer
  passed to free is consistent with the pointer originally allocated.
  We have not yet validated this approach through a full-stack implementation.
  This is described in Section~\ref{sec:anti-tamper}.

\item We now describe potential capability-related prefetch instructions in
  Section~\ref{sec:caching-and-explicit-prefetch}, with specific
  consideration of side-channel attacks.
  We also explicitly specify \DDC{} bounds checking on the MIPS
  \insnnoref{PREF} prefetch instruction.

\item We now reference our papers \textit{CHERIvoke: Characterising Pointer
  Revocation using CHERI Capabilities for Temporal Memory Safety} (IEEE MICRO
  2019), \textit{Rigorous engineering for hardware security: Formal modelling
  and proof in the CHERI design and implementation process} (IEEE SSP 2020),
  and \textit{Cornucopia: Temporal Safety for CHERI Heaps} (IEEE SSP 2020).

\end{itemize}
