\chapter{The CHERI-RISC-V Instruction-Set Architecture}
\label{chap:cheri-riscv}


\newcommand{\riscvloadcappagefault}{0x1A}
\newcommand{\riscvstorecappagefault}{0x1B}
\newcommand{\riscvcheriexception}{0x1C}

Having considered the software-facing semantics and architecture-neutral
aspects of the CHERI protection model in previous chapters, we now turn to
elaborating CHERI capabilities within a specific architecture: 32-bit
and 64-bit RISC-V.
Wherever possible, CHERI-RISC-V implements the architecture-neutral concepts
described in Chapter~\ref{chap:architecture}.
We chose to design CHERI-RISC-V as a parameterizable instruction set that
includes several key design points that allow us to evaluate both
microarchitectural and architectural implications via side-by-side
experiments.
Detailed descriptions of specific capability-aware instructions can be found
in Chapter~\ref{chap:isaref-riscv}.

\section{The RISC-V Instruction-Set Architecture}

RISC-V is a contemporary open-source architecture developed at the University
of California at Berkeley.
RISC-V is intended to be used with a range of microprocessors spanning small
32-bit microcontrollers intended for embedded applications to larger 64-bit
superscalar processors intended for use in datacenter computing.
The RISC-V ISA is reminiscent of MIPS, with some important differences: a more
modular design allows the ISA to be more easily subsetted and extended; a
variable-length instruction encoding improves code density; the MMU has a
hardware page-table walker rather than relying on software TLB management;
the ISA avoids exposing pipelining behaviors to software (e.g., there is no
branch-delay slot); and it has a more contemporary approach to atomic memory
instructions.
Various drafts and standardized extensions add other more contemporary
features such as hypervisor support.  There is also ongoing work to define
broader platform behaviors beyond the architecture, including platform
self-description and peripheral-device enumeration.
At the time of writing, the RISC-V userspace ISA has been standardized
(v2.2)~\cite{RISCV:User:2.2}, but the privileged ISA remains under
development (v1.10)~\cite{RISCV:Privileged:1.10}\footnote{As v1.11 of the
privileged specification remains a work-in-progress, we define CHERI-RISC-V
relative to v1.10.}. \mmnote{1.11 is ratified now, so we should rewrite this chapter with v1.11 changes.}

%\textbf{XXXRW: It would be nice to propose an extension name; unfortunately,
%the `C' extension is already used for compressed instructions.
%`S' is taken for the supervisor extensions, and `P' for SIMD.
%Perhaps we should be `W' for Weally Secure, Doc?}

\section{CHERI-RISC-V Approach}

Our application of CHERI to the RISC-V architecture is motivated by several
opportunities:

\begin{itemize}
\item To gain access to a maturing open-source ISA, hardware, and software
  ecosystem, for the purposes of a stronger experimental baseline and
  methodology (such as more mature core variants).
  At the time of writing, the MIPS software ecosystem remains richer, but we
  see a substantial community effort at filling gaps for RISC-V.

\item To demonstrate the portability of the CHERI approach across multiple
  architectures, and in particular to illustrate how portable CHERI software
  stacks can be designed and maintained despite underlying architectural
  differences.

\item To apply lessons learned from CHERI-MIPS in an entirely fresh
  application of the protection model to a new architecture.
  Many of our MIPS design choices reflect pragmatic design choices made prior
  to the development of full compiler and operating-system stacks, and are
  difficult to change within those stacks.

\item To revisit and scientifically explore a design space around CHERI
  integration into a target architecture -- for example, around the use of
  register files and exceptions.

\item To support new CHERI experimentation in the space of microcontrollers,
  heterogenous cores and accelerators, and DMA, as well as in relation to
  microarchitectural side channels.

\item To lay groundwork for possible open-source transition of the CHERI
  protection model into the RISC-V architecture.
\end{itemize}

In the following subsections, we describe our high-level approach before
providing a more detailed specification of CHERI-RISC-V.

\subsection{Target RISC-V ISA Variants}

The RISC-V ISA defines both 32-bit (\texttt{XLEN}=32) and 64-bit
(\texttt{XLEN}=64) base integer instruction
sets (RV32I, RV64I). \mmnote{Maybe mention that we don't support RV128I,
because it has not been ratified yet.}
Our current proposal supports either mode with few differences beyond
capability width, although safe support for both modes in a single processor
is not specified at this time.
\pgnnote{This may be understated.  It may also be misinterpreted,
  as to whether what exists is not safe, or not done at all.
  That seems ambiguous.  The RISC-V privileged spec itself
  is mostly parameterized for XLEN (Prashanth), and I would presume
  perhaps the CHERI-RISC-V might also, as it seems is described below.}
\pdrnote{I think adding the word dynamically clears up this confusion?}
Our definition of CHERI-RISC-V should work with either 32-register or
16-register (RV32E) variants of RISC-V.
We specify CHERI as applied to RVG, which consists of the general-purpose
elements of the RISC-V ISA: integer, multiplication and division,
atomic, floating-point, and double floating-point instructions.
We also describe extensions to RVS, the supervisor extension defined in the
privileged portion of the ISA.

We view 64-bit CHERI-RISC-V as a mature specification suitable as a
starting point for an official RISC-V extension.  However, we feel
that 32-bit CHERI-RISC-V is less mature.  In particular, the current
encoding for 64-bit capabilities provides insufficient precision.
Further research is needed to determine if an alternate encoding,
perhaps using an alternate scheme for permissions, can provide better
precision.

\subsection{CHERI-RISC-V is an ISA Design Space}

A key aim in CHERI-RISC-V is to allow experiments to be run comparing various
CHERI-related parameters:  Are capabilities with respect to 32-bit or
64-bit virtual addresses?  What are the impacts of various instruction-set
variations or microarchitectural optimizations?  How does greater investment
of opcode space affect performance -- and what techniques, such as instruction
compression or different capability-aware modes, may impact this?  How can
CHERI interact with other architectural specializations such as DMA and
heterogenous compute?
To answer these and other questions, we have designed CHERI-RISC-V as an ISA
design space, in which several key design dimensions are parameterized:

\begin{itemize}
\item Both 32-bit and 64-bit RISC-V are extended, with 64-bit and 128-bit
  capabilities respectively.

% XXXRW: We now think that we are not doing this in CHERI-RISC-V.
%
%\item Both exception-throwing and tag-clearing variants of monotonic
%  capability manipulation instructions are supported.
%  \arnote{I thought we had concluded that we should not be throwing exceptions unless absolutely required.}

\item Optional instruction variations and an optional ``capability encoding
  mode'' that
  invest opcode space differently to reduce instruction count for common
  instruction sequences -- especially with respect to load/store instructions
  that occupy substantial quantities of opcode space.
\end{itemize}

With respect to all of these design dimensions, we intend that specific
instantiated microarchitectures, compiler targets, compiled operating systems,
and compiled software stacks support only one point in the space.
However, we hope that carefully parameterized hardware and software designs
will be able to target more than one point to allow side-by-side comparison
from the perspectives of hardware resource utilization, performance, security,
and compatibility.

\subsection{CHERI-RISC-V Strategy}

Wherever possible, we attempt to conform to the specific aesthetic of RISC-V,
such as with respect to opcode layout choices and aligning the semantics of
new Special Capability Register access instructions with existing RISC-V CSRs.

\pgnnote{This begs the question of whether we will remain fully compliant
  with the RISC-V privileged spec, or must necessarily deviate.}

\subsection{Common Architectural Features}

CHERI-RISC-V shares the following features with other CHERI architectures:

\begin{itemize}
\item Tagged memory with capability-width tag granularity and alignment.
\item Registers able to hold capabilities are tagged.
\item \PCC{} transforms and controls program-counter-relative fetches.
\item \DDC{} transforms and controls legacy RISC-V load-store instructions,
  including relocating access addresses using the capability base and offset.
\item Floating point is fully supported, including capability-relative
  floating-point load and store instructions.
\item General-purpose registers are extended to hold capabilities.
\item Capability-related violations (such as loads/stores/fetches via untagged
  capabilities, out-of-bound accesses, and so on) trigger immediate precise
  exceptions.
\item Requests for non-monotonic capability transformations result in
  the tag of the written back value being stripped.
\item It is never left ambiguous as to whether a register index operand to a
  load or store instruction, or the register target of a jump instruction,
  is a capability and therefore must have a tag set.
  This reinforces intentionality.
\item \cappermASR limits privileged ISA
  operations when within privileged rings.
  While RISC-V's specific privileged operations differ, the intent remains the
  same: to allow code compartmentalization within the privileged ring.
\end{itemize}

\subsection{Unique Architectural Features}

The following changes are specific to CHERI-RISC-V:

\begin{itemize}
\item RISC-V exception handling -- including register banking, scratch
  registers, and cause mechanism -- is used.
\item A new exception code, \riscvloadcappagefault{}, will be
  reported in the RISC-V \xcause{} CSRs when a load attempts to fetch a
  capability through a valid page table entry granting read permission but
  forbidding loads of capabilities.  This fault otherwise behaves like a RISC-V
  load page fault.
\item A new exception code, \riscvstorecappagefault{}, will be
  reported in the RISC-V \xcause{} CSRs when a store attempts to write a
  capability through a valid page table entry granting write permission but
  forbidding stores of capabilities.  This fault otherwise behaves like a
  RISC-V store/AMO page fault.
\item A new exception code, \riscvcheriexception{}, will be
  reported in the RISC-V \xcause{} CSRs when other
  capability-related exceptions (such as tag violations) occur.
  %
  Additional capability-specific exception cause information, such
  as more specific cause information and the identity of the faulting
  register is reported in the existing \xtval{} CSRs (see
  Section~\ref{subsection:riscv:cheri-exception-reporting}).
\item A new bit is defined in \menvcfg{} and \senvcfg{} to enable
  CHERI support.
\item New per-mode capability CSRs are added as \xccsr{} (see
  Section~\ref{subsubsec-ccsrs}).
\item CHERI-related page permissions are added to RISC-V architectural
  page-table formats.
\item The interpretation of addresses in memory capabilities
  depends on whether virtual addressing is enabled via the RISC-V
  \texttt{satp} CSR\footnote{This is not a substantially different design
  choice than in other architectures: memory
  capabilities are interpreted relative to the active address space, and
  control of that address space is delegated to suitably privileged code,
  whether configuring a simple direct map between virtual and physical memory,
  or managing multiple more complex address spaces.
  In all cases, care is required as physical-memory access authorized by a
  capability is determined by the addressing mode and current translation
  table contents.}.
  When \texttt{satp} is set to \texttt{Bare}, capabilities have a
  physical-address interpretation.
  When \texttt{satp} enables page-table translation, capabilities have a
  virtual-address interpretation.
\item Both XLEN=32 and XLEN=64 are supported (albeit not dynamically).
  In the future, it may be desirable to also support XLEN=128.
\item A rich set of atomic instructions is extended with capability
  support.
\item The \cflags{} field contains a single bit indicating the ``capability
  encoding mode'' to use when the capability is installed as \PCC{}.
\item In the non-compressed RISC-V encoding, the capability encoding mode
  allows existing opcodes, e.g.\ for loads, stores, \insnnoref{auipc},
  to be interpreted as expecting capability rather than integer operands
  (reducing opcode footprint while maintaining intentionality).
\item In the compressed RISC-V encoding, the capability encoding mode allows
  existing load, store, and jump opcodes to be interpreted as expecting
  capability rather than integer operands.

% XXXRW: We are no longer doing this in CHERI-RISC-V:
%
%\item Instructions performing operations that would violate monotonicity or
%  otherwise lead to an undereferenceable (e.g., unrepresentable) capability
%  clear the tag when writing the register back, rather than immediately
%  throwing an exception.
%  Later attempts to dereference those capabilities will throw a tag
%  violation exception.
%  \jrtcnote{In CHERI-RISC-V is an ISA Design Space we say both
%    exception-throwing and tag-clearing variants are supported}
\end{itemize}

\section{CHERI-RISC-V Specification}

In this section, we describe in greater detail the integration of CHERI into
the RISC-V instruction set.
Instruction opcode encodings can be found in
Appendix~\ref{app:isaquick-riscv}.

\pmnote{Perhaps mention something like this:

\subsection{CHERI as a non-standard RISC-V extension}

CHERI is integrated into the RISC-V ISA as a non-standard extension
named Xcheri, and follows the idioms for RISC-V extensions to the
extent possible.  In the extension terminology of the RISC-V
specification, CHERI is a \emph{greenfield} extension since it adds
new instructions by populating a new instruction encoding space.  The
prefix used for the encoding is currently ``1011011'', placing it in
the \emph{custom-2/rv128} opcode space that the specification allows
for use for custom instruction set extensions on RV64; this makes it a
standard-compatible global encoding. (See however the discussion in
Section~\ref{section:cheri-risc-v-rv128-lq-sq}.)

A CHERI-RISC-V processor has the X bit of the \texttt{misa} register
hardwired to 1 on boot to indicate the presence of a non-standard
extension.  Information tying this set X bit to the Xcheri extension
would be communicated to system software in a platform-specific manner.
}
\pdrnote{Agreed: this would be great, although pedantically, our adding of
load cap and store cap in non-extension space must make us brownfield?}

\subsection{Tagged Capabilities and Memory}

CHERI-RISC-V allows both registers and memory to hold tagged capabilities,
allowing capabilities and data to be intermingled.
This allows capabilities to be embedded within in-memory data structures,
supports the implementation of capability-oblivious memory copy operations,
and maintains strong C-language pointer compatibility for capabilities.
This implies the use of tagged memory consisting of 1-bit
tags protecting capability-aligned, capability-sized words of memory
implemented with suitable protection and atomicity properties.

While we currently do not define CHERI-RISC-V support for RV128, we anticipate
that we will wish to support RV128 in the future.
It seems plausible that 256-bit capabilities might incorporate 128-bit
addresses along with compressed bounds in a similar manner to our 128-bit
capabilities for 64-bit addresses.

\subsection{Capability Register File}

In CHERI-RISC-V,
general-purpose integer registers are extended to optionally hold
full
capabilities, along with a tag.

Extending general-purpose integer registers raises the
question of whether and how non-capability-aware instructions should
interact with capability values in registers -- a concern not dissimilar to
the behavior of instructions on 64-bit architectures offering legacy 32-bit
support.
We specify that individual instructions reading from, or writing to, a
register in the register file have fixed integer or capability interpretations
based on the opcode encoding -- i.e., that new instructions be introduced that
explicitly specify whether capability semantics are required for an input or
output register, or that the current architectural mode unambiguously specify
integer or capability operand interpretation.

A further design choice relates to the specific subset of general-purpose
integer registers that are extended to capability width, as it need not be the
case that all are.
In our baseline specification, we extend all registers, but allow software
ABIs to limit specific numbered registers to only integer use.
We hope to evaluate different points in this design space to determine whether
performance tradeoffs favour a complete set of capability registers, or simply
a partial set (which might reduce microarchitectural overhead).

The bottom \texttt{XLEN} bits of the register will contain the integer
interpretation (which, for a capability, will be its address\pdrnote{Does this also
cover capabilities which authorise type space rather than address space?}), and the
top \texttt{XLEN} bits (plus additional tag bit) will contain any capability
metadata.
When a register is read as an integer (i.e., using an opcode that dictates an
integer interpretation),
the register's bottom \texttt{XLEN} bits will be utilized, and any other bits ignored.
When a register is written as an integer, its bottom \texttt{XLEN} bits will
hold the new
integer value, and the top \texttt{XLEN} bits and tag bit will be cleared to match
those of the NULL capability. This both prevents in-register corruption of tagged
capabilities by implicitly clearing the tag, and also provides reasonable semantics
for integer access to capability values.

\subsubsection{Capability Length Architectural Constant (CLEN)}

One challenge in introducing CHERI support is that the architectural constant,
\texttt{XLEN}, the number of bits in a register, is used to define numerous
behaviors throughout the ISA, such as the size of CSRs, the operation of
integer operations, the size of addresses, and so on.
We choose to leave \texttt{XLEN} as constant as the majority of these operations
are intended to be of the natural integer size (e.g., for addition).
However, this does mean that in some cases we need to introduce new
instructions intended to operate on full capability-wide values.
We introduce a new architectural constant, \texttt{CLEN}, which we define as
$2\times$\texttt{XLEN}, which excludes the tag bit.
Operations such as capability-width CSR access, capability load, and capability
store will operate on \texttt{CLEN}$+1$ bits including the tag bit.

Specifically, for 32-bit CHERI-RISC-V, \texttt{CLEN} will be 64 bits, and for
64-bit CHERI-RISC-V, \texttt{CLEN} will be 128 bits, affecting a variety of
functions including the stride of tag bits in physical memory.
Opcode space is reserved in the RISC-V ISA for 64-bit load and store
instructions even when \texttt{XLEN} is 32, and we can reuse these opcode
reservations and encodings to load 64-bit \texttt{CLEN} words as well as
their tag bit.
Similarly, when \texttt{XLEN} is 64, we can use 128-bit \texttt{CLEN} load
and store opcodes.

We do not currently define support for 32-bit compatibility (with or without
capability support) when operating in a 64-bit RISC-V processor, but
anticipate that adding non-capability-aware 32-bit support would be
straightforward.
We also do not yet define an architecture supporting multiple capability
widths concurrently, but recognize that there are certain use cases -- such as
when interoperating between a 64-bit application core and a 32-bit
microcontroller within a single System-on-Chip (SoC) -- where this would be
valuable.

\subsection{Capability-Aware Instructions}

In CHERI-RISC-V, two general categories of instructions are added: those that
query or manipulate capability fields within registers, and those that
utilize registers for the purposes of load, store, or jump operations.

Register-to-register instructions querying and manipulating fields allow integer values to be moved in and
out of portions of an in-register capability, subject to guarded manipulation.
They are simply new instructions defined in CHERI-RISC-V and added to
the opcode space.

It is possible to imagine having memory-access and
control-flow instructions condition their behavior based on the presence of a
tag, selecting a compatible integer behavior if the tag is not set, and a
capability behavior if it is set.
However, this would violate the principle of intentional use: not only should
privilege be minimized, but it should not be unintentionally, implicitly, or
ambiguously exercised.
Allowing a corrupted capability (i.e., one with its tag stripped due to an
overlapping data write) to dereference \DDC{} implicitly would violate this
design goal.
We therefore specify strong \textit{type safety} for all capability-aware
instructions: all instructions explicitly encode whether an integer or
capability operand is being used, and attempts to use untagged values where
tagged ones are expected will lead to an exception.

\subsection{Control and Status Registers (CSRs)}
\label{subsection:cheri-riscv-csrs}

CHERI-RISC-V extends the behavior of the baseline RISC-V integer CSR set,
allowing capability control over access to some CSRs for compartmentalization
purposes, as well as adding several new CSRs to control capability-related
functionality.
These are accessed via existing RISC-V CSR instructions, and their encodings
are given in Table~\ref{tab:risc-v-control-and-status-registers}.
New Special Capability Registers (SCRs), accessed via new CSR-like
instructions, are described in Section~\ref{subsection:cheri-riscv-scrs}.

\begin{table}[h]
\centering
\begin{tabular}{c>{\raggedright\arraybackslash}p{2.7in}>{\raggedright\arraybackslash}p{2.5in}}
\toprule
\textbf{Encoding} & \textbf{Register} & Privilege notes \\
\midrule
\textbf{0x8C0} & User capability control and status register (\uccsr{}) & \PCC{}.\cperms{}.\emph{Access\_System\_Registers} \\
\textbf{0x9C0} & Supervisor capability control and status register (\sccsr{}) & \{S,M\}-mode \& \PCC{}.\cperms{}.\emph{Access\_System\_Registers} \\
\textbf{0xBC0} & Machine capability control and status register (\mccsr{}) & M-mode \& \PCC{}.\cperms{}.\emph{Access\_System\_Registers} \\
\bottomrule
\end{tabular}
\caption{Control and Status Registers (CSRs)}
\label{tab:risc-v-control-and-status-registers}
\end{table}

\subsubsection{Controlling Access to CSRs}

Accessing some RISC-V CSRs also requires the \PCC{}.\cperms{}.\emph{Access\_System\_Registers}
permission to be set for the currently executing code.
This allows privileged-level code to be constrained from interfering with key
system management functionality (such as exception handling).
We adopt a whitelist approach: reading or writing any CSR requires the permission, with the exceptions listed in Table~\ref{tab:risc-v-access-system-registers-whitelist}.

\pdrnote{Text describing current makeshift whitelist, pending updates. TODO we want three separate permissions: UASR, SASR, MASR, permitting access only to the
corresponding privilege mode's CSRs. Debate still open over whether these are encoded as 3 bits, or as a 2-bit counter that monotonicity can only decrease
(make less privileged). This could either be a separate instruction or implicitly enforced on CAndPerms. We likely still want some kind of whitelist after this change}
\ajnote{I am not sure I am fully happy with that, but probably just need to be convinced by someone that this is the way to go... It feels weird that a user task which used to just access, say, the instret or time csr suddenly needs Access\_System\_Registers on its PCC.}
\pmnote{User access to the instret and time csrs are already constrained by scounteren.{IR,TM}; i.e. the user task will suffer an exception if these are not enabled by the supervisor. Access\_System\_Registers on its PCC would be similar.}

\begin{table}[h!]
\centering
\begin{tabular}{cc}
\toprule
\textbf{CSR}           & \textbf{Read/Write} \\
\texttt{cycle(h)}      & Read-Only           \\
\texttt{time(h)}       & Read-Only           \\
\texttt{instret(h)}    & Read-Only           \\
\texttt{hmpcounter(h)} & Read-Only           \\
[1.5em]
\texttt{fflags}        & Read-Write          \\
\texttt{frm}           & Read-Write          \\
\texttt{fcsr}          & Read-Write          \\
\bottomrule
\end{tabular}
\caption{CSR Whitelist. The accesses shown are the only CSR accesses that are permitted when the installed PCC does not have \cappermASR{}.}
\label{tab:risc-v-access-system-registers-whitelist}
\end{table}

\subsubsection{CHERI Extension Control}

A new bit in the \menvcfg{} and \senvcfg{} CSRs is used to enable
CHERI for lower privilege levels.  When CHERI is disabled, attempting
to execute CHERI-specific instructions will raise an illegal
instruction fault including loads and stores which use a capability
register (excluding the implicit \DDC{} operand for legacy
loads/stores) as the memory operand.

Other CHERI extensions are always enabled regardless of the state of
this bit.  Specifically, bounds and permissions on \PCC{} and \DDC{}
are always honored.  Exceptions always copy \PCC{} to \xEPCC{} on
exception entry and restore the full \PCC{} on exception return.
Capability mode is always honored if enabled in \PCC{}.  Software
which disables CHERI in lower modes must take care to ensure that
\PCC{} and \DDC{} are set to suitable values while lower modes
execute.

Bit 28 (\texttt{0x1C}) in the \menvcfg{} and \senvcfg{} CSRs is
defined as the CHERI enable bit.  Its allocation within these CSRs may
change until CHERI is ratified as a RISC-V extension.

\subsubsection{Capability Control and Status Registers (CCSRs)}
\label{subsubsec-ccsrs}
New per HART \xccsr{} \texttt{XLEN}-bit RISC-V CSRs are defined as per
Figure~\ref{fig-ccsr} (shown for XLEN=32):

\begin{figure}[!h]
\begin{center}
\begin{bytefield}[bitwidth=\textwidth/34]{32}
  \bitheader[endianness=big]{0,1,2,31} \\
  \bitbox{32}{\textbf{WIRI}}
\end{bytefield}
\caption{\xccsr{} register format; WIRI bits are Write Ignore Read Ignore.}
\label{fig-ccsr}
\end{center}
\end{figure}

\pmnote{WIRI has been removed from the RISC-V specification.  The spec
  now typically uses WPRI (Write-Preserve, Read-Ignore) for such
  undefined bits (e.g. in \texttt{mstatus}).  Both WARL (Write-Any,
  Read-Legal) and WLRL (Write-Legal, Read-Legal) seem unsuitable here.
  The RISC-V spec unsatisfyingly does not specify what happens if
  software doesn't preserve values for WPRI, but then this spec
  doesn't need to either.}

\subsection{Special Capability Registers (SCRs)}
\label{subsection:cheri-riscv-scrs}

Special Capability Registers (SCRs) are similar to CSRs in that they affect
special functions such as exception delivery, rather than being
general-purpose registers, but have capability rather than integer types.
SCRs are therefore accessed via new capability-aware instructions.

The new \asm{CSpecialRW} instruction allows reading and writing special
capability registers. When the destination register is 0, the instruction shall
not read the special capability register and shall not cause any of the
side-effects that might occur on a special capability register read, similar to
the standard \asm{csrrw} RISC-V instruction. When the source register is 0, the
instruction will not write to the special capability register at all, and so
shall not cause any of the side effects that might otherwise occur on a special
capability register write, similarly to the standard \asm{csrrs/c} RISC-V
instruction.

Table~\ref{tab:risc-v-special-capability-registers} lists the SCRs
available via that instruction, as well as their values at CPU reset, which
will be set in a manner consistent with the description in
Section~\ref{sec:capability-state-on-cpu-reset}.
Whether a register is initialized to NULL or the omnipotent capability, its
flags field will be initialized to zero (specifying integer encoding mode).

\pdrnote{I have added MTDC, UTDC, STDC to match exception handling section.
Might be worth further discussion to decide if we need/want them.}

\begin{table}[h!]
\centering
\begin{tabular}{cllcccc@{}}
\toprule
& \textbf{Register} & \textbf{Modes} & \textbf{Access} & \textbf{Reset} & \textbf{Extends} \\ \midrule
\textbf{0} & Program counter capability (\PCC{})     & U, S, M & RO & $\infty$ & \PC{} \\
\textbf{1} & Default data capability (\DDC{})        & U, S, M & -  & $\infty$ & -     \\
[1.5em]
\textbf{4} & User trap code capability (\UTCC{})     & U, S, M & ASR & $\infty$    & \utvec{} \\
\textbf{5} & User trap data capability (\UTDC{})     & U, S, M & ASR & $\emptyset$ & -        \\
\textbf{6} & User scratch capability (\UScratchC{})  & U, S, M & ASR & $\emptyset$ & -        \\
\textbf{7} & User exception PC capability (\UEPCC{}) & U, S, M & ASR & $\infty$    & \uepc{} \\
[1.5em]

\textbf{12} & Supervisor trap code capability (\STCC{})     & S, M & ASR & $\infty$    & \stvec{} \\
\textbf{13} & Supervisor trap data capability (\STDC{})     & S, M & ASR & $\emptyset$ & -        \\
\textbf{14} & Supervisor scratch capability (\SScratchC{})  & S, M & ASR & $\emptyset$ & -        \\
\textbf{15} & Supervisor exception PC capability (\SEPCC{}) & S, M & ASR & $\infty$    & \sepc{}  \\
[1.5em]

\textbf{28} & Machine trap code capability (\MTCC{})     & M & ASR & $\infty$    & \mtvec{} \\
\textbf{29} & Machine trap data capability (\MTDC{})     & M & ASR & $\emptyset$ & -        \\
\textbf{30} & Machine scratch capability (\MScratchC{})  & M & ASR & $\emptyset$ & -        \\
\textbf{31} & Machine exception PC capability (\MEPCC{}) & M & ASR & $\infty$    & \mepc{}  \\
\bottomrule
\end{tabular}
\caption{Special Capability Registers (SCRs).
SCRs 4-7 are available only with the N extension, and 12-15 only with the S
extension.
\textbf{Modes} shows which RISC-V privilege modes are allowed to access the
registers.
\textbf{Access} indicates additional restrictions on accessing the registers:
\PCC{} is read-only via \insnriscvref{CSpecialRW}, but is set by
\insnriscvref{CJALR} and during exceptions; \textit{ASR} indicates
\PCC{}.\cperms{} must grant \cappermASR{} to permit access (in addition to
being in a permitted mode).
\textbf{Reset} indicates whether the register is initialised to the default
root capability ($\infty$) or NULL capability ($\emptyset$) on reset.
Some special capabilities registers are extensions of existing RISC-V
registers, with the capability offset being equal to the original register.
\note{We should describe this in more detail including behavior if they are
sealed or become unrepresentable and what to do about PC alignment. Note this
table shares quite a lot with \cref{subsection:riscv:exceptionhandling}}{rmn30}
}
\label{tab:risc-v-special-capability-registers}
\end{table}

Where an SCR extends a RISC-V CSR, e.g.\ \MTCC{} extending \mtvec{},
any read to the CSR shall return the offset of the corresponding SCR.
Similarly, any write to the CSR shall set the offset of the SCR to the value
written.
This shall be equivalent to a \insnriscvref{CSetOffset} instruction, but with
any exception condition instead just clearing the tag of the SCR.
This allows sealed capabilities to be held in SCRs without allowing them to
be modified in a tag-preserving way, while also preventing exceptions when
installing trap vectors: something that can be problematic where the task
is delegated to a higher privilege level.
Some RISC-V CSRs have write ignore bits, or otherwise implicitly modify
the written value to restrict the CSR to legal values.
These modifications must be applied to the SCR's new offset when writing a CSR
extended by an SCR, or to the offset of the newly written capability when
using \insnriscvref{CSpecialRW}.
\insnriscvref{CSpecialRW} of a sealed capability to an SCR which extends a CSR
with any non-preserved bits clears the tag on the capability, even if the
offset would not be changed.
As per the rest of the RISC-V specification, should the SCR become
unrepresentable as a result of the offset being set, the resulting address is
preserved, and the rest of the capability is cleared to match the null
capability.
Note that this means a read of the CSR (where the SCR has restricted bounds)
may not be consistent with the preceding write, but this approach is
consistent with the invalidating of other RISC-V capabilities, and allows a
meaningful address to be reported should the SCR be dereferenced.

\arnote{what about an unprivileged register for TLS?} \hmnote{Unlike ARM, RISC-V uses a general-purpose integer register \textit{tp}
for TLS in user-level. In the kernel, the thread's kernel's context pointer
is saved in the scratch register. We try to follow RISC-V design where possible.}

\subsection{Efficiently Encoding Capability-Relative Operations}
\label{subsec-encoding-cap-ops}

The RISC-V instructions that interpret arguments or results as addresses
(e.g.\ loads, stores, jumps, \insnnoref{auipc}) can either act on integer pointers
relative to \DDC{} or \PCC{}, or on explicit capabilities.
For example, capability-relative load and store instructions accept (and expect) capability
operands that relocate and constrain data accesses, performing tag, bounds,
permission, and other checks as required.
However, load and store instructions occupy large amounts of instruction
encoding space due to having multiple register operands and large immediate
values.
One consideration in adding CHERI support to RISC-V is the degree to which we
are willing to occupy large chunks of remaining encoding space by simply
supplementing each address-manipulating instruction with a
corresponding capability-relative version.
Other options that conserve opcode space include utilizing a more limited set
of addressing modes or using smaller immediate sizes for capability-relative
instructions -- with a potentially significant negative impact on performance
due to an increase in resulting code size.
We therefore consider several points in this design space:

\begin{itemize}
\item Introduce a full set of new load and store instructions occupying
  substantial opcode space, but providing more efficient capability-intensive
  generated code.

\item Introduce only a limited set of new load and store instructions,
  reducing new opcode utilization, and supportingly less efficient
  capability-intensive generated code.

\item Introduce a new \textit{capability encoding mode} in which
  existing RISC-V load-store opcode space is reused for capability-relative
  accesses, allowing a rich set of load-store instructions without
  substantially occupying available RISC-V opcode space.

  In the conventional (legacy) \textit{integer encoding mode}, a small set of
  capability-relative loads and stores are added, tuned to limit opcode
  space utilization -- e.g., by having small or no immediates -- at the cost
  of increased code footprint.

  To maintain intentionality, this approach is never ambiguous in either mode
  as to whether load and store opcodes are intended to access relative to
  integer or capability operand: address operands are always integer relative
  in integer encoding mode, and always capability relative in capability
  encoding mode.

  Pure-capability and hybrid code can be generated against either encoding,
  but will be most efficient (in terms of instruction footprint) when
  generated against the corresponding mode.
  We have specified that the encoding mode will change as a result of jumping
  to a \PCC{} with a different encoding-mode flag.
  Section~\ref{section:capability-encoding-mode} considers other options for
  encoding-mode switches.
\end{itemize}

\noindent
In the interest of experimentation, we plan to pursue all three approaches
in stages, considering tradeoffs around efficiency and instruction-set design
across a range of workloads:

\begin{enumerate}
\item Begin by adding a conservative set of capability-relative load and store
instructions without immediate offsets, which will consume a small amount of
opcode space, and be sufficient to allow initial compiler engineering to take
place.

\item Introduce an architectural ``encoding mode bit'' in which RISC-V
instruction encodings used for integer-relative \DDC{}-constrained loads and
stores are instead used for CHERI-RISC-V capability-relative loads and stores.
To continue to allow hybrid code when in capability mode, we would introduce a
further simple set of \DDC{}-indirected integer-relative loads and stores with
no immediate, similar to the capability-relative set described above.

\item Introduce a full set of capability-relative loads and stores
with immediate offsets to enable full flexibility without switching modes.
While this may not be acceptable to the upstream community without increasing
the size of instructions, we should understand how much performance is being
lost by reducing the flexibility of code generation.
\end{enumerate}

As register-relative jump instructions have relatively light opcode
utilization, and because there are many easy-to-imagine uses for protecting
control flow using capabilities even in hybrid code, we do not apply semantic
changes to those baseline non-compressed RISC-V instructions when in
capability encoding mode.
The implications for compressed instructions are described in
Section~\ref{subsection:compressed-instructions}.

\subsubsection{Encoding Modes}
\label{sec:cheri-riscv-encmodes}

We define two encoding modes, selected using the CHERI-RISC-V-specific
encoding-mode flag in the capability \cflags{} field of \PCC{}:

\begin{description}
\item[Integer encoding mode (0)] Conventional RISC-V execution mode, in which
  address operands to existing RISC-V load and store opcodes contain
  \textit{integer addresses}.
  The upper \texttt{XLEN} bits and tag bit of
  the operand register will be ignored.
  The dereference will implicitly occur relative to \DDC{}.
  The tag bit on \DDC{} must indicate that a valid capability is present, and
  all capability-related checks (such as bounds checks) must be performed in
  order for a successful load or store to take place.

\item[Capability encoding mode (1)] CHERI capability encoding mode, in which address operands to
  existing RISC-V load and store opcodes contain \textit{capabilities}.
  The tag bit must indicate a valid capability is present, and all
  capability-related checks (such as bounds checks) must be performed in order
  for a successful load or store to take place.
\end{description}

The operating system will automatically save and restore \PCC{} on context
switches, preserving an execution context's encoding mode.
It is essential that changes in encoding mode be properly observed when an
exception is processed, as the exception handler must execute with expected
semantics or risk insecure behavior.
When \xTCC{} is set by the operating system, it should contain an appropriate
encoding-mode flag to ensure that exception handlers utilize the correct
instruction encoding.

\subsubsection{Non-Compressed Instructions Affected by Capability Encoding
  Mode}

The following non-compressed RISC-V load and store instructions would be
affected by the capability encoding-mode bit (see the following section for
further details on compressed instructions):

\medskip

\begin{savenotes}
\begin{tabular}{llllll}
\textit{Integer load} & LB & LH & LW & LD & LQ \\
\textit{Integer load (unsigned)} & LBU & LHU & LWU & LDU & \\
\textit{Integer store} & SB & SH & SW & SD & SQ \\
\textit{Floating-point load} & FLW & FLD & FLQ & & \\
\textit{Floating-point store} & FSW & FSD & FSQ & & \\
\textit{Atomic} & LR & SC & AMOSWAP & AMOADD & AMOAND \\
\textit{Atomic (cont)} & AMOOR & AMOXOR & AMOMAX & AMOMIN & \\
\textit{Address calculation} & AUIPC\footnote{See Section~\ref{section:cheri-risc-v-auipc}.} & & & & \\
\end{tabular}
\end{savenotes}

\subsection{Compressed Instructions}
\label{subsection:compressed-instructions}

\rwnote{Ensure that this text is up-to-date after our revisions to the
  encoding-mode mechanism.}

The compressed instruction extension (extension \texttt{C}) is now
routinely used by the RISC-V gcc compiler to improve code density.  It seems
likely that the \texttt{C}ompressed extension will become mandatory for
the \texttt{G}eneral configuration (which is currently
\texttt{IMAFD}).
Two problems arise in adding compressed instruction support for capabilities:
the need for additional opcode space for load, store, and jump instructions;
and the need to add new instructions to load and store capabilities.

Given the tight encoding space for compressed instructions, some
registers (e.g.~the stack pointer - \texttt{sp}) are implicit for some
instructions.  Since there appears to be no free encoding space to
differentiate between a capability-\texttt{sp} and an
integer-\texttt{sp}, one potential design choice is to use our capability
encoding mode to also control the interpretation of compressed instructions.
Similarly, for compressed loads and
stores that can use only registers \texttt{x8}--\texttt{x15} as the
base address, the encoding mode would allow us to reuse opcode space.
As with the baseline compressed instruction set, this imposes ABI-related
constraints on the architecture, and would require the compiler to conform to
those constraints in order to accomplish the best code density.

In his thesis~\cite{WatermanThesis2016}, Waterman gives the following
function prologue and epilogue examples to illustrate how compressed
instructions improve code density:

\begin{small}
\begin{verbatim}
   prologue_legacy:       epilogue_legacy:
     c.addi sp, -16         c.ldsp ra, 8(sp)
     c.sdsp ra, 8(sp)       c.ldsp s0, 0(sp)
     c.sdsp s0, 0(sp)       c.addi sp, 16
     c.jr t0                c.jr ra
\end{verbatim}
\end{small}

For capability-aware code, saving and restoring the return address
(\texttt{ra}) requires capability store and load instructions.
Given the frequency of capability use for the pure-capability code targeted by
our capability encoding mode (all pointer loads and stores), one option might
be to relieve pressure on the compressed opcode space by removing the less
frequently used floating-point double load and store instructions.
For RV64, we could replace the compressed load floating-point double
(\texttt{C.FLD}) with compressed load capability (\texttt{C.LC}) with
the same encoding as compressed load quad (\texttt{C.LQ}) used in
RV128.  Similarly, replace: \texttt{C.FSD} with \texttt{C.SC}
(compressed store capability). For stack-relative memory access,
replace floating-point double operations with capability operations:
\texttt{C.FLDSP} with \texttt{C.LCSP} and \texttt{C.FSDSP} with
\texttt{C.SCSP}.

In the RISC-V I base instruction set (non-compressed instructions), we chose to make
capability jump instructions available in both integer and capability encoding
modes, as they use relatively little encoding space compared to the amount of
free space available.
In the RISC-V C extension (compressed instructions), the amount of free space is
far smaller, leading us to select a different design choice: when in capability
encoding mode, as with load-store instructions, we interpret existing compressed
instructions \insnnoref{C.J}, \insnnoref{C.JAL}, \insnnoref{C.JR}, and
\insnnoref{C.JALR} as the capability instructions \insnriscvref{CJAL},
\insnriscvref{CJR}, and \insnriscvref{CJALR}, accepting capability rather than
integer register operands for jump target registers and link registers.

There is one large gap in the compressed instruction encoding at
\verb#100X_XXXX_XXXX_XX00# (where \verb#X#$=$ don't care) that could
be used to support a \insnriscvref{CIncOffsetImm} (\texttt{c.cincoff})
instruction to allow the stack pointer to be adjusted.

This would result in capability-aware prologue and epilogues:

\begin{small}
\begin{verbatim}
   prologue_cap_aware:    epilogue_cap_aware:
     c.cincoff csp, -16     c.lcsp cra, 8(csp)
     c.scsp cra, 8(csp)     c.lcsp s0, 0(csp)
     c.scsp cs0, 0(csp)     c.cincoff csp, 16
     c.cjr ct0              c.cjr cra
\end{verbatim}
\end{small}

A further interaction relates to encoding-mode selection.
If we pursue a design choice using the lowest bit of a target jump address to
set the encoding used following a jump, then no additional opcode pressure is
introduced.
If we instead choose to use new instructions, such as to get or set the mode
explicitly, then additional space might be required.
The amount of space required would be modest, but this additional usage might
be a further consideration in the encoding-mode management strategy.

If the approach of using a mode bit for compressed instructions is
adopted, then it follows that the uncompressed versions of the
instructions should follow suit to preserve the design intent that
all compressed instructions can be expanded out to uncompressed
instructions by the processor's decoder, and that the assembler should
be permitted to optimize uncompressed instructions into their
compressed form where possible.
\pdrnote{These last two paragraphs do seem a bit stale, and do we need to
add a note about the prologue assuming it's been called in the correct
mode? (or doing a check)}

\subsubsection{Compressed Instructions Affected by Capability Encoding Mode}

The following compressed instructions are affected by capability encoding
mode:

\medskip

\begin{tabular}{llllll}
\textit{Control flow} & C.JALR & C.JR & & \\
\textit{Compressed integer load} & C.LW & C.LD & C.LWSP & C.LDSP & \\
\textit{Compressed integer store} & C.SW & C.SD & C.SWSP & C.SDSP & \\
\textit{Compressed floating-point load} & C.FLW & C.FLD & C.FLWSP & C.FLDSP & \\
\textit{Compressed floating-point store} & C.FSW & C.FSD & C.FSWSP & C.FSDSP & \\
\end{tabular}

\subsection{Floating Point}

The vast majority of floating-point instructions are not impacted by the
presence of CHERI-RISC-V.
Existing RISC-V floating-point load and store instructions, in the
integer encoding mode, are relocated and constrained by \DDC{}.
In CHERI-RISC-V, we define a new set of simple capability-relative load and
store instructions, as well as a more rich set via capability encoding mode.

The floating point control
registers (FCSR, FRM, and FFlags) are whitelisted in Table \ref{tab:risc-v-access-system-registers-whitelist}
so they can be accessed without needing \cappermASR{}.

\subsection{Exception Handling}
\label{subsection:riscv:exceptionhandling}
RISC-V defines several privilege modes, including machine mode, user mode, and
supervisor mode, with exceptions allowing controlled transition between those modes.
CHERI-RISC-V introduces several new exception-related Special Capability Registers
to supplement existing RISC-V exception CSRs with new capability-related functionality.
In addition, when a capability exception is raised, \xtval{} will provide
details about the exception as described in Section~\ref{subsection:riscv:cheri-exception-reporting}.

\subsubsection{Exceptions to Machine Mode}
We define the following new special capability registers that can be read and
written only from machine mode:

\begin{itemize}
\item \MEPCC{} - Machine Mode Exception Program Counter Capability (extends
  \mepc{})
\item \MTDC{} - Machine Mode Data Capability
\item \MTCC{} - Machine Mode Trap Code Capability (extends \mtvec{})
\item \MScratchC{} - Machine Mode Scratch Capability
\end{itemize}

\subsubsection{Exceptions to Supervisor Mode}

We define the following new special capability registers that can be read and
written only from supervisor mode and above:

\begin{itemize}
\item \SEPCC{} - Supervisor Mode Exception Program Counter Capability (extends
  \sepc{})
\item \STDC{} - Supervisor Mode Data Capability
\item \STCC{} - Supervisor Mode Trap Code Capability (extends
  \stvec{})
\item \SScratchC{} - Supervisor Mode Scratch Capability
\end{itemize}

\subsubsection{Exceptions to User Mode}

We employ the ``N'' extension (for ``User-Level Interrupts'') being developed in
the newer versions of the RISC-V specifications, and extend it with the following
new special capability registers that can be read and written from any mode:

\begin{itemize}
\item \UEPCC{} - User Mode Exception Program Counter Capability (extends
  \uepc{})
\item \UTDC{} - User Mode Data Capability
\item \UTCC{} - User Mode Trap Code Capability (extends \utvec{})
\item \UScratchC{} - User Mode Scratch Capability
\end{itemize}

The extension could be leveraged for user-space-only implementations
of \insnriscvref{CInvoke}, as well as routing specific interrupts from
suitable devices to user-level compartments for handling by sandboxed
device drivers.

Explicit vector and data capabilities give each ring its
own code and data capabilities to utilize during exception handling.
We define ``scratch capabilities'' to allow the exception handler to stash a
capability register for the purposes of having a working register that
corresponding data capabilities can be loaded to in order to begin a full
context save.
This is consistent with RISC-V's use of scratch registers in various modes to
avoid committing general-purpose integer registers to exception handling.

When exception behavior, e.g.\ a trapping instruction, \insnnoref{ecall},
or \xRET{}, causes \PCC{} to take a value stored in an SCR, it is possible that
the SCR contains a capability that would not be a valid \PCC{} (untagged,
sealed, not executable, or improperly aligned).
In these cases, the value is still installed in \PCC{}, and a check on the next
instruction fetch will trigger a further exception.

% XXXRW: We are no longer doing this in CHERI-RISC-V:
%
%\subsection{Exceptions vs. Tag Clearing on Capability Manipulation Failure}
%
%In CHERI-MIPS, in keeping with the MIPS ISA design, many instructions are able
%to throw exceptions -- not only the load/store/jump variants, but also
%operations to manipulate capability fields.
%In more complex microarchitectures, it may be desirable to limit the set of
%instructions that can throw exceptions.
%If that is desirable for CHERI-RISC-V, then this could be accomplished by
%shifting manipulation instructions from throwing exceptions on non-permitted
%access to clearing the tag bit in the target register.
%This maintains the security invariant that monotonicity (and other aspects)
%is enforced, but shifts the point of exception delivery from manipulation to
%dereference.
%
%One risk in adopting this approach is that debugging non-derefenceable
%pointers may become more difficult, due to greater asynchrony.
%An additional architectural status bit that can be checked to detect loss of a
%tag by an instruction would allow compiler-inserted instrumentation to check
%for tag loss at suitable moments (e.g., after potentially tag-stripping
%operations such as pointer manipulation).

\subsection{Capability Exception Reporting}
\label{subsection:riscv:cheri-exception-reporting}

CHERI-RISC-V extends the definition of the Trap Value CSRs, \xtval{}, to
report capability exception details as described in
Figure~\ref{fig-cheri-tval} (shown for XLEN=32):

\begin{figure}[!h]
\begin{center}
\begin{bytefield}[bitwidth=\textwidth/34]{32}
  \bitheader[endianness=big]{0,4,5,10,31} \\
  \bitbox{21}{\textbf{WPRI}}
  \bitbox{6}{\texttt{cap idx}}
  \bitbox{5}{\texttt{cause}}
\end{bytefield}
\caption{\xtval{} register format for Capability Exception}
\label{fig-cheri-tval}
\end{center}
\end{figure}

\begin{description}
\item [cause] The \texttt{cause} field reports the capability
  exception code from Table~\ref{tab:risc-v-capability-cause}.
\item [cap idx] The \texttt{cap idx} field reports the index of the capability register that caused the last exception.  When
the most significant bit is set, the 5 least significant bits are used to index
the special purpose capability register file described in
Table~\ref{tab:risc-v-special-capability-registers}, otherwise, they index the
general-purpose capability register file.
\end{description}

\begin{table}
\begin{center}
\begin{tabular}{ll}
\toprule
Value & Description \\
\midrule
0x00 & None \\
0x01 & Length Violation \\
0x02 & Tag Violation \\
0x03 & Seal Violation \\
0x04 & Type Violation \\
0x05-0x07 & \emph{reserved} \\
0x08 & Software-defined Permission Violation \\
0x09-0x0a & \emph{reserved} \\
0x0b & Unaligned Base \\
0x0c-0x0f & \emph{reserved} \\
0x10 & \cappermG Violation \\
0x11 & \cappermX Violation \\
0x12 & \cappermL Violation \\
0x13 & \cappermS Violation \\
0x14 & \cappermLC Violation \\
0x15 & \cappermSC Violation \\
0x16 & \cappermSLC Violation \\
0x17 & \emph{reserved} \\
0x18 & \cappermASR Violation \\
0x19 & \cappermInvoke Violation \\
0x1a-0x1b & \emph{reserved} \\
0x1c & \cappermCid Violation \\
0x1d-0x1f & \emph{reserved} \\
\bottomrule
\end{tabular}
\end{center}
\caption{CHERI-RISC-V Capability Exception Codes}
\label{tab:risc-v-capability-cause}
\end{table}

\jhbnote{The current exception code values are inherited from
  CHERI-MIPS.  They should probably be renumbered at some point.}

If an instruction could potentially throw more than one capability exception,
the capability exception code is set to the highest priority exception (numerically lowest
priority value) as shown in Table~\ref{table:risc-v-exception-priority}.

\begin{table}
\begin{center}
\begin{tabular}{ll}
\toprule
Priority & Description \\
\midrule
1  & \cappermASR Violation \\
2  & Tag Violation \\
3  & Seal Violation \\
4  & Type Violation \\
5  & \cappermInvoke Violation \\
   & \cappermCid Violation \\
6  & \cappermX Violation \\
7  & \cappermL Violation \\
   & \cappermS Violation \\
8  & \cappermLC Violation \\
   & \cappermSC Violation \\
9 & \cappermSLC Violation \\
10 & \cappermG Violation \\
11 & Length Violation \\
12 & Software-defined Permission Violation \\
\bottomrule
\end{tabular}
\end{center}
\caption{CHERI-RISC-V Capability Exception Priority}
\label{table:risc-v-exception-priority}
\end{table}

\jhbnote{Missing the unaligned base cause in the priority table}

\subsection{Virtual Memory and Page Tables}
\label{subsection:riscv:pagetables}

In CHERI-RISC-V, capability addresses are interpreted with respect to the
privilege level of the processor in line with RISC-V's handling of integer
addresses.
%
In Machine Mode, capability addresses are generally interpreted as physical
addresses; if the \texttt{mstatus} \texttt{MPRV} flag is asserted, then data
accesses (but not instruction accesses) will be interpreted as if performed by
lower-privileged modes.
%
In Supervisor and User Modes, capability addresses are interpreted as dictated
by the current \texttt{satp} configuration: addresses are virtual if paging is
enabled and physical if not.
%
% \hmnote{It is more accurate to say that addresses are interpreted as virtual
% addresses IIF SATP.mode != Bare. There could exist M/U processors with no
% virtual addresses, or even SW that, theoretically but not practically runs on
% M/S/U processors that still work with SATP.mode = Bare across all rings.}

In CHERI-RISC-V, we require \cappermASR{} to change
the page-table root (\texttt{satp}) and other virtual-memory parameters.
(In the future, it may be desirable to extend the page-table walking mechanism
to itself utilize capabilities, allowing the walker to be constrained;
see \cref{app:exp:physcap:ptw}.)

It is desirable to extend the Memory Management Unit
to constrain the loading and storing of valid capabilities via specific page
mappings by adding new permission bits to the current Page Table Entry
(PTE) format.
%
Unfortunately, there are no remaining spare bits in the RISC-V Sv32 (32-bit)
PTE format for additional hardware permissions.
(For the purposes of prototyping, we could utilize the two
available software-defined PTE permission bits -- but these are likely to be
used in current operating systems, requiring a longer-term solution.)
%
The Sv39 (39-bit) and Sv48 (48-bit) PTE formats include several reserved bits,
some of which we allocate for use by CHERI-RISC-V; see \cref{fig:riscv:sv39}.

\subsubsection{Capability Stores}

Capability stores are mediated with two bits per PTE, called CW and CD.  Their
effect on capability flow parallels the existing W and D bits and is described
by the following table:

\begin{center}
%
\begin{tabular}{ccl}

\textbf{CW} & \textbf{CD} & \textbf{Behavior} \\
0 & X & Trap on capability stores (exception code \riscvstorecappagefault{}) \\
1 & 0 & Capability stores atomically raise CD or fault (as above) \\
1 & 1 & Capability stores permitted
\end{tabular}
%
\end{center}

\noindent Currently, implementations must apply these behaviors to all
instructions which would store an asserted capability tag; that is, they are
dependent on the tag bit. This may be relaxed in future versions of this
specification to all instructions which \emph{could} store an asserted
capability tag, removing the dependence on the tag bit. Instructions which are
able to move only data (and so necessarily clear tags) will not interact with
these PTE flags. CHERI-aware Sv32 implementations, lacking room in their PTEs,
will act as though CW and CD are \emph{set}.

As with the existing D bit, there are two permitted approaches for hardware to
take in response to an attempted store with an asserted CHERI tag and through a
PTE with clear CD:
%
\begin{inenum}
%
  \item raise a store capability page fault (exception code
  \riscvstorecappagefault{}), or
%
  \item atomically update the PTE to set CD.  In this case, the existing rules
  regarding atomicity continue to apply: the PTW must check, atomically, that
  the PTE is valid and has W and CW both set, and the PTE update must become
  visible no later than the causal store.
%
\end{inenum}
%
Capability-store instructions are still stores and so are expected to check the
W permission, in addition to CW, and to set the existing D bit (or fault if it
is clear, using the existing RISC-V \xcause{} code) in addition to the CD bit
(or fault, using the new capability store/AMO page fault \xcause{} code).
%
The ordering of checks of, and updates to, the PTE follows the scheme of RISC-V
but interdigitates capability mediation: V, U, and W must be checked first,
followed by CW, before any of D and/or CD (and/or A) are atomically asserted or
are used as grounds for faulting.  In the latter case, D and/or A take
precedence over CD.

The PTE bits CW and CD have no necessary relationship to any of the CHERI tag
bits on the corresponding physical page.  In particular, CD does not reflect
the presence of capabilities on the page, much as D does not reflect anything
about the particular values of data on a page.  Software-enforced temporal
safety mechanisms, for example, are anticipated to regularly clear CD (and
even, occasionally, CW) on PTEs referencing pages that nevertheless contain
capabilities.

\subsubsection{Capability Loads}

Aside from experimental behavior,\nwfnote{to be fwd ref'd} there are three
behaviors we would like to elicit upon attempted load of a capability: succeed,
strip loaded tags, or raise a fault.  To leave sufficient room for
experimentation, we reserve three bits for mediation of capability loads, CR,
CRM, and CRG, but we reserve most configurations (which will raise page faults
by virtue of being invalid settings in PTEs).  These bits interact with
capability flow as follows:

\begin{center}
%
\begin{tabular}{cccl}

\textbf{CR} & \textbf{CRM} & \textbf{CRG} & \textbf{Behavior} \\
0 & 0 & 0 & Capability loads strip tags on loaded result \\
0 & 1 & 0 & Capability loads fault (exception code \riscvloadcappagefault) \\
0 & X & 1 & \textit{Reserved for future use} \\
1 & 0 & 0 & Capability loads are unaltered \\
1 & 0 & 1 & \textit{Reserved for future use} \\
1 & 1 & X & \textit{Reserved for generational load barriers}
\end{tabular}
%
\end{center}

\noindent As with CW/CD, implementations are required to take a data-dependent
disposition when a PTE is configured to fault on a capability load, raising
faults only when the resulting tag is set.  Future versions of this
specification may, similarly, relax this to any instruction which \emph{could}
load a capability with a set tag. CHERI-aware Sv32 implementations will act as
though CR is \emph{set} and both CRM and CRG are clear.

\nwfnote{Asking as an almost entirely software person... might it be
possible to use Sv39 PTE layouts on an otherwise standard RV32 machine?
Rather than using a 3-level translation table, one could maintain a 2-level
scheme by requiring the use of aligned, consecutive pairs of physical pages
for PDs and PTs, or even a kind of gross hybrid in which the PDs retain
their Sv32 structure and only the leaf PTs become increasingly aligned and
full of 64-bit structures.  Atomic operations in software against the page
structures would be impossible without some treachery, but these are
unlikely to be required by software.  This may also not be very attractive,
as it fundamentally makes CHERI-RISC-V an odd duck, but existing RISC-V code
is likely relatively easily adopted to such a Sv39-on-32 scheme.}

\begin{figure}

\begin{center}
\begin{bytefield}[endianness=big,bitwidth=2.4em]{16}
\bitheader[lsb=48]{48,53,54,58,59,60,61,62,63}\\
\bitbox{1}{CW}
\bitbox{1}{CR}
\bitbox{1}{CD}
\bitbox{1}{CRM}
\bitbox{1}{CRG}
\bitbox{5}{\textit{Reserved}}
\bitbox{6}{PPN[2] \dots}\\

\bitheader[lsb=32]{32,47}\\
\bitbox{16}{\dots PPN[2] \dots} \\

\bitheader[lsb=16]{16,18,19,27,28,31}\\
\bitbox{4}{\dots PPN[2]}
\bitbox{9}{PPN[1]}
\bitbox{3}{PPN[0] \dots}\\

\bitheader{0,1,2,3,4,5,6,7,8,9,10,15}\\
\bitbox{6}{\dots PPN[0]}
\bitbox{2}{RSW}
\bitbox{1}{D}
\bitbox{1}{A}
\bitbox{1}{G}
\bitbox{1}{U}
\bitbox{1}{X}
\bitbox{1}{W}
\bitbox{1}{R}
\bitbox{1}{V}
\end{bytefield}
\end{center}

\caption{A Sv39 PTE showing CHERI extensions in bits 59 through 63.}
\label{fig:riscv:sv39}
\end{figure}

\subsection{The RV128 LQ, SQ, and Atomic Instructions}
\label{section:cheri-risc-v-rv128-lq-sq}

The putative 128-bit RISC-V ISA (RV128) reserves additional quadword load and
store instructions, \insnnoref{LQ} and \insnnoref{SQ}, to be
used to load and store 128-bit quantities, as well as quad-word atomics.
In CHERI-RISC-V for RV64, we reuse these hypothesised opcode encodings for our
129-bit capability load and store instructions, \insnriscvref{LC} and
\insnriscvref{SC}, to avoid additional opcode commitment.
We also introduce corresponding atomics on capabilities reusing the
quad-word atomic opcodes.

Should the future RV128 standard utilize 128-bit addresses, then the most
natural course of action would be to utilize compressed 256-bit capabilities,
and add new capability load and store opcodes for the broader capability
width.
However, should an RV128 be defined that instead uses 64-bit virtual
addresses (i.e., one with 128-bit data registers but not a 128-bit address
space), our current opcode-space reuse would not be appropriate for a
corresponding CHERI-RISC-V variant.
Overloaded opcodes might reduce intentionality.
It remains to be seen how essential this
concern is with respect to security: tag-free copies could still be
implemented efficiently by stripping \cappermLC{} from a source
capability during a memory copy.
However, the alignment requirements imposed by our capability load, store,
and atomic instructions can be beneficial in debugging what is otherwise
potential tag loss.
Should RV128 be more fully specified in the future, we will need to revisit
whether capability load instructions can be combined with the
\insnnoref{LQ}, \insnnoref{SQ}, and atomic instructions.

\subsection{The AUIPC Instruction}
\label{section:cheri-risc-v-auipc}

The RISC-V \insnnoref{AUIPC} instruction generates an address derived
from \PC{} and a 20-bit immediate, typically intended to be used in generating
addresses for global variables.
Because this instruction occupies a significant amount of opcode space, we
choose to implement a capability-based version of the instruction only in the
capability encoding mode, where the instruction returns a capability derived
from \PCC{} rather than an integer virtual address.
When using \insnnoref{AUIPC} to generate an integer in the capability
encoding mode, or a capability in the integer encoding mode, an additional,
less efficient, instruction sequence must be used instead.
Depending on the code linkage model, it might also be desirable to have a
further version of the instruction, \insnnoref{AUICGP}, which returns
a capability derived from a global capability table register.

\subsection{\PCC{} alignment}
\label{section:cheri-risc-v-pcc-align}

Matching the RISC-V specification, we specify that installation of an
insufficiently aligned \PC{} triggers exception at the jump rather than at
the jump target.
However, CHERI-RISC-V poses additional potential alignment issues.
In particular, an unaligned base in \PCC{} would force either its address
or its offset to be similarly unaligned.
Since the architecturally visible \PC{} is the offset of \PCC{}, allowing
this to be unaligned would break assumptions made by RISC-V software, while
allowing the address to be unaligned may break assumptions made in
RISC-V SoCs.
We instead specify that the \PCC{} base must be 4-byte aligned when compressed
instructions are disabled.
This is enforced on the capability-based jump instruction, throwing an
Unaligned Base exception.
Whenever compressed instructions are enabled, the \PCC{} base may be aligned to
only 2 bytes, however writes to the WARL field of \texttt{misa} to re-enable
compressed instructions are ignored while the \PCC{} base is not 4-byte
aligned, as is the case if the \PC{} would be unaligned.
The following edge-case behaviors relating to CSRs/SCRs are also specified:
\begin{itemize}
\item \insnriscvref{CSpecialRW} instructions that attempt to write a capability
      into \xTCC{} with a base that is not 4-byte aligned (regardless of whether
      compressed instructions are enabled) shall leave \xTCC{} unmodified.
      This simplifies the treatment of the lowest two bits of \xtvec{},
      which are used in the RISC-V specification to determine the vectoring
      mode of an exception.
\item \xEPCC{} can be written with an arbitrarily aligned base.
      Reads of \xEPCC{} (via \xRET{}, \insnriscvref{CSpecialRW}, or CSR read
      from \xepc{}) implicitly mask the offset as per \xepc{} in the RISC-V
      specification, triggering a Representability Violation if the resulting
      capability is unrepresentable.
      An \xRET{} to a capability with an insufficiently aligned base causes an
      Unaligned Base exception on the following instruction fetch.
\end{itemize}

\pmnote{Perhaps add a sub-section on PCC alignment: The RISC-V
  architecture has the property that with the RVC compressed
  instruction-set extension, the processor cannot generate
  fetch-misaligned faults.  However, if PCC is improperly set, then
  CHERI-RISC-V could generate such faults.  Since compiler-generated
  code would normally use aligned offsets, we could just ensure that
  explicit assignments to PCC result in an aligned base.  One way to
  ensure this would be to zero the lowest bits needed for alignment:
  lowest bit for RVC, and lowest 2-bits for non-RVC systems.  It might
  be simplest to always zero the lowest 2-bits, even for RVC.}

% \subsection{Capability Encoding Mode}
% \label{subsec-cap-mode}
%
% \rwnote{This section not yet edited after planning session, relocation from
%   the CHERI-RISC-V appendix.}

% \newcommand{\capmode}{\textbf{capability encoding mode}}

% Instructions that alter the control flow and cause memory accesses for
% instruction fetches, such as branches and jumps, consider their target address
% as an offset in \PCC{}.

% In general, data memory accesses are checked against \DDC{} when not in
% \capmode{}, and are checked against specific capability operands otherwise.
% Instruction memory accesses are always checked against \PCC{}. On exception,
% depending on the current privilege mode, the appropriate trap code capability
% \UTCC{}, \STCC{} or \MTCC{} is loaded in \PCC{}, and \PCC{} is saved in the
% appropriate \UEPCC{}, \SEPCC{} or \MEPCC{}. If the exception is a capability
% exception, the appropriate RISC-V \ucause{}, \scause{} or \mcause{} is set to a
% new ``capability exception'' cause, and \xtval{} is updated appropriately.

% The following instruction produces a \PCC{}-derived capability when in
% \capmode{}, and an offset into \PCC{} otherwise:
% \asm{AUIPC}
% This allows the expected behavior both for integer-pointer code operating
% relative to \PCC{} or \DDC{}, and for capability-pointer code that expects to
% dereference a capability with its own bounds.

% When in \capmode{}, the following instructions (which perform ordered
% comparisons) read their operands as capabilities and must order tagged
% capabilities after \arnote{consider tagged capabilitie greater than?} untagged capabilities. These instructions are:
% \asm{BGEU},
% \asm{BGE},
% \asm{BLTU},
% \asm{BLT},
% \asm{SLTIU},
% \asm{SLTI},
% \asm{SLTU},
% \asm{SLT}.

% When in \capmode{}, the following instructions (which perform unordered
% comparisons) read their operands as capabilities and compare all of the bits in
% the register. These instructions are:
% \asm{BNE}
% \asm{BEQ}

% Shifts are not defined on capabilities because most shifts will quickly take an
% address out of the representable range.

% 64-bit instructions that are specifically designed to operate on only the low 32
% bits, similarly, do not make sense when applied to capabilities.

% With a merged register file, the following instructions will ignore the
% capability bits on source operands and clear them on the destination register:
% \asm{ADD},
% \asm{ADDI},
% \asm{ADDIW},
% \asm{ADDW},
% \asm{ANDI},
% \asm{AND},
% \asm{ORI},
% \asm{OR},
% \asm{SLLIW},
% \asm{SLLI},
% \asm{SLLI},
% \asm{SLLW},
% \asm{SLL},
% \asm{SRAIW},
% \asm{SRAI},
% \asm{SRAI},
% \asm{SRAW},
% \asm{SRA},
% \asm{SRLIW},
% \asm{SRLI},
% \asm{SRLI},
% \asm{SRLW},
% \asm{SRL},
% \asm{SUB},
% \asm{SUBW},
% \asm{XORI},
% \asm{XOR}.

% With a merged register file, the \asm{LUI} instruction clears all of the
% capablity bits in the destination register.

% When in \capmode{}, \asm{JALR} and \asm{JAL} write the capability link register
% with \PCC{} + 4. \asm{JALR} also uses a capability operand for the target of
% the jump.
% \dcnote{It would be nice if they could set a thing that's both a capability and a valid address, but that's possible only if we don't do PCC-offsetting.}
%
% When not in \capmode{}, all existing load and store instructions treat the
% memory address as a \DDC{}-relative address. With a merged register file, when
% not in \capmode{}, load instructions will zero all of the capability bits in
% their target register. When in \capmode{}, the address operand is a capability
% register.
% \asm{LBU},
% \asm{LB},
% \asm{LD},
% \asm{LHU},
% \asm{LH},
% \asm{LWU},
% \asm{LW},
% \asm{SB},
% \asm{SD},
% \asm{SH},
% \asm{SW}.

\pmnote{Need a section on capability state on reset.}

\section{Design Rationale}

The CHERI-RISC-V specification has evolved over time.  This section
describes some of the alternatives considered and other changes made.

\subsection{Capability Exception Reporting}

Initially CHERI-RISC-V reported additional information for capability
faults via \texttt{cause} and \texttt{cap idx} fields in a new
per-mode CSR \xccsr{}.  During CHERI-RISC-V's development, the RISC-V
privileged spec renamed the per-mode \xbadaddr{} CSR to \xtval{} and
reused it to store non-address information for some faults.  Following
this change, CHERI-RISC-V moved the \texttt{cause} and \texttt{cap
  idx} fields to \xtval{}.

\subsection{MMU Capability Exceptions}

Initially CHERI-RISC-V reported MMU tag permission violations via the
\riscvcheriexception{} exception code.  Unique values were assigned
for these violations to the \texttt{cause} field of \xtval{}.
However, storing the faulting address in \xtval{} similar to other MMU
faults was found to be more useful.  As a result, additional top-level
exception codes (\riscvloadcappagefault{} and
\riscvstorecappagefault{}) were added to report MMU tag permission
violations instead.

\subsection{Separate Capability Register File}

CHERI-RISC-V was originally specified with two different approaches
for adding general-purpose capability registers.  One approach
described earlier extended existing integer general-purpose registers
to hold capabilities (a ``merged'' register file).  The second
approach left the integer register file as-is and added a separate
capability register file to hold capabilities (a ``split'' register
file).  This second approach mirrored the approach used on CHERI-MIPS.

Most of the ISA is agnostic to this choice due to the principle of
intentional-use such as the explicit use of capability-relative
addresses versus integer pointers relative to \DDC{} or \PCC{}.
However, the ``split'' register file approach did require a few
additional changes:

\begin{itemize}
\item A separate \insnnoref{Clear} instruction similar to
  \insnriscvref{CClear} and \insnriscvref{FPClear} which cleared a set
  of general-purpose integer registers.

\item A separate \insnref{CGetAddr} instruction to fetch the address
  of a capability register.  In a ``merged'' register file this can be
  obtained by reading the aliased general-purpose integer register.
  Note that \insnref{CGetAddr} is currently still defined.

\item A separate \insnref{CSub} instruction to compute the difference
  between addresses in two capability registers.  In a ``merged''
  register file this can be computed via the existing \insnnoref{SUB}
  instruction on the aliased general-purpose integer registers.  Note
  that \insnref{CSub} is currently still defined.

\item Dirty bits were defined in a the per-mode CSR \xccsr{}. This bit
  was set anytime a general-purpose capability register was changed.
  It was intended to be used similar to \texttt{FS} and \texttt{XS}
  fields in \texttt{mstatus} to optimize context switching.
\end{itemize}

In practice, only variants of CHERI-RISC-V using a ``merged'' register
file were implemented in emulators, soft cores, toolchains, and
operating systems.
