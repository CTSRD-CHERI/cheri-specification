\clearpage
\phantomsection
\addcontentsline{toc}{subsection}{RET - Return from Procedure}
\insnxeslabel{ret}
\subsection*{RET - Return from Procedure}

\begin{x86opcodetable}
  \xopcode{C3}{RET}
  {ZO}{Valid}{Valid}
  {Near return.}
  \xopcode{CB}{RET}
  {ZO}{Valid}{Valid}
  {Far return.}
  \xopcode{C2 \emph{iw}}{RET \emph{imm16}}
  {I}{Valid}{Valid}
  {Near return and pop \emph{imm16} bytes from stack.}
  \xopcode{CA \emph{iw}}{RET \emph{imm16}}
  {I}{Valid}{Valid}
  {Far return and pop \emph{imm16} bytes from stack.}
\end{x86opcodetable}

\begin{x86opentable}
  \xopen{ZO}{NA}{NA}{NA}{NA}
  \xopen{I}{imm16}{NA}{NA}{NA}
\end{x86opentable}

\subsubsection*{Description}

Pops return address from the stack and transfers control to the popped
address.  In 64-bit mode the \insnnoref{CAP} prefix can be used with
near returns to select the capability operand size instead of 64-bit.
In capability mode, near returns always use the capability operand
size.

Near returns which use a capability operand size always pop a
capability off of the stack to load into \CIP{} rather than popping
off the new value of \RIP{}.

\subsubsection*{Flags Affected}

None
