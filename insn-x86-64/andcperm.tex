\clearpage
\phantomsection
\addcontentsline{toc}{subsection}{ANDCPERM - Mask Capability Permissions}
\insnxeslabel{andcperm}
\subsection*{ANDCPERM - Mask Capability Permissions}

\begin{x86opcodetable}
  \xopcode{NP 0F 0C \emph{/r}}{ANDCPERM \emph{r/mc, r64}}
  {MR}{Valid}{Valid}
  {Mask permissions of \emph{r/mc} by \emph{r64}.}
  \xopcode{37 /2 \emph{id}}{ANDCPERM \emph{r/mc, imm32}}
  {MI}{Valid}{Valid}
  {Mask permissions of \emph{r/mc} by sign-extended \emph{imm32}.}
\end{x86opcodetable}

\begin{x86opentable}
  \xopen{MR}{ModRM:r/m (r, w)}{ModRM:reg (r)}{NA}{NA}
  \xopen{MI}{ModRM:r/m (r,w)}{imm32}{NA}{NA}
\end{x86opentable}

\subsubsection*{Description}

Derives a new capability from the destination operand with the
\textbf{perms} and \textbf{uperms} field bitwise ANDed with the source
operand and stores the result in the destination operand.  The
destination operand can be a register or memory location; the source
operand can be a register or immediate.  If the destination operand is
sealed and tagged, set the destination operand to its original value
with the \textbf{tag} field cleared.

\subsubsection*{Flags Affected}

None
