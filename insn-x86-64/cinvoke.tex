\clearpage
\phantomsection
\addcontentsline{toc}{subsection}{CINVOKE - Invoke Sealed Capability Pair}
\insnxeslabel{cinvoke}
\subsection*{CINVOKE - Invoke Sealed Capability Pair}

\begin{x86opcodetable}
  \xopcode{EA \emph{/r}}{CINVOKE \emph{rc, r/mc}}
  {RM}{Valid}{Valid}
  {Set CAX to \emph{r/mc} and jump to \emph{rc}.}
\end{x86opcodetable}

\begin{x86opentable}
  \xopen{RM}{ModRM:reg (r)}{ModRM:r/m (r)}{NA}{NA}
\end{x86opentable}

\subsubsection*{Description}

Jumps to a pair of sealed capabilities.  The first source operand can
be a register; the second source operand can be a register or memory
location.

If both operands are sealed with the same \textbf{otype}, sets \CIP{}
to the unsealed first operand and sets \CAX{} to the unsealed second
operand.  Note that this control transfer is a jump and does not push
any values onto the stack.  If this instruction fails, it raises a
Capability Violation Fault (see
Section~\ref{sec:x86:capability-fault}).

\subsubsection*{Flags Affected}

None
