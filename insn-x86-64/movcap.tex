\clearpage
\phantomsection
\addcontentsline{toc}{subsection}{MOV -- Move to/from Additional Capability Registers}
\insnxeslabel{movcap}
\subsection*{MOV -- Move to/from Additional Capability Registers}

\begin{x86opcodetable}
  \xopcode{0F 24 \emph{/r}}{MOV \emph{rc,} CFS/CGS/DDC}
  {MR}{Valid}{Valid}
  {Move additional capability register to \emph{rc}.}
  \xopcode{0F 25 \emph{/r}}{MOV CFS/CGS/DDC\emph{, rc}}
  {RM}{Valid}{Valid}
  {Move \emph{rc} to additional capability register.}
\end{x86opcodetable}

\begin{x86opentable}
  \xopen{MR}{ModRM:r/m (w)}{ModRM:reg (r)}{NA}{NA}
  \xopen{RM}{ModRM:reg (w)}{ModRM:r/m (r)}{NA}{NA}
\end{x86opentable}

\subsubsection*{Description}

Moves the contents of an additional capability register to a
general-purpose capability register or vice versa.

Similar to the \insnnoref{MOV} opcodes for control and debug
registers, the \textbf{reg} field of the ModRM byte always identifies
the additional capability register to read or write.  The \textbf{mod}
field of ModRM is ignored, and the \textbf{r/m} field identifies the
general-purpose capability register.  Attempts to reference invalid
additional capability registers will raise a UD\# exception.

Attempts to access additional capability registers other than \CFS{},
\CGS{}, or \DDC{} from a privilege level other than 0 will raise a
GP\#(0) exception.

\subsubsection*{Flags Affected}

None
