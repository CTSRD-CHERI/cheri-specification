\clearpage
\phantomsection
\addcontentsline{toc}{subsection}{BUILDCAP - Construct Capability}
\insnxeslabel{buildcap}
\subsection*{BUILDCAP - Construct Capability}

\begin{x86opcodetable}
  \xopcode{VEX.LZ.0F.W0 0E \emph{/r}}{BUILDCAP \emph{rca, r/mc, rcb}}
  {RMV}{Valid}{Valid}
  {Construct capability from \emph{r/mc} and \emph{rcb} and store in
    \emph{rca}.}
\end{x86opcodetable}

\begin{x86opentable}
  \xopen{RMV}{ModRM:reg (w)}{ModRM:r/m (r)}{VEX.vvvv (r)}{NA}
\end{x86opentable}

\subsubsection*{Description}

Constructs a new capability equal to the second operand with the
\textbf{base}, \textbf{length}, \textbf{address}, \textbf{perms}, and
\textbf{uperms} fields replaced with the corresponding fields from the
third operand and stores the result in the first (destination) operand.
If the third operand is a sentry then the result is also sealed as a
sentry.  If the resulting capability is not a subset of the second
operand in bounds or permissions, or is not a legally-derivable
capability, or if the second operand did not have its \textbf{tag}
field set, or if the second operand was sealed as a non-sentry, the
resulting capability is set to the third operand with the \textbf{tag}
field cleared.

\subsubsection*{Flags Affected}

ZF is set to the \textbf{tag} field of the resulting capability.  The
CF, PF, AF, and OF flags are undefined.
