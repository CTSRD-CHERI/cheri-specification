\clearpage
\phantomsection
\addcontentsline{toc}{subsection}{SCBND - Set Capability Bounds}
\insnxeslabel{scbnd}
\subsection*{SCBND - Set Capability Bounds}

\begin{x86opcodetable}
  \xopcode{17 \emph{/r}}{SCBND \emph{r/mc, r64}}
  {MR}{Valid}{Valid}
  {Set bounds of \emph{r/mc} to \emph{r64}.}
  \xopcode{37 /0 \emph{id}}{SCBND \emph{r/mc, imm32}}
  {MI}{Valid}{Valid}
  {Set bounds of \emph{r/mc} to zero-extended \emph{imm32}.}
\end{x86opcodetable}

\begin{x86opentable}
  \xopen{MR}{ModRM:r/m (r, w)}{ModRM:reg (r)}{NA}{NA}
  \xopen{MI}{ModRM:r/m (r, w)}{imm32}{NA}{NA}
\end{x86opentable}

\subsubsection*{Description}

Derives a new capability from the destination operand and source
operand and then store the result in the destination operand.  The
destination operand can be a register or memory location; the source
operand can be an immediate or a register.  When an immediate value is
used as an operand, it is zero-extended.

The new capability's \textbf{base} field is set to the current
\textbf{address} field of the destination operand, and the
\textbf{length} field is set to the source operand.  If the resulting
capability cannot be represented exactly, the \textbf{base} will be
rounded down and the \textbf{length} will be rounded up by the
smallest amount needed to form a representable capability covering the
requested bounds.  If the original capability is sealed or the bounds
of the new capability exceed the original capability, the \textbf{tag}
field in the resulting capability is cleared.

\subsubsection*{Flags Affected}

The ZF flag is set to the value of the \textbf{tag} field in the
resulting capability.  The CF, PF, AF, SF, and OF flags are undefined.
