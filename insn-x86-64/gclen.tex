\clearpage
\phantomsection
\addcontentsline{toc}{subsection}{GCLEN - Get Capability Length}
\insnxeslabel{gclen}
\subsection*{GCLEN - Get Capability Length}

\begin{x86opcodetable}
  \xopcode{F3 0F 7A \emph{/r}}{GCLEN \emph{r64, r/mc}}
  {RM}{Valid}{Valid}
  {Store the bounds length of \emph{r/mc} in \emph{r64}.}
\end{x86opcodetable}

\begin{x86opentable}
  \xopen{RM}{ModRM:reg (w)}{ModRM:r/m (r)}{NA}{NA}
\end{x86opentable}

\subsubsection*{Description}

Sets the destination operand to the \textbf{length} field of the
source operand.  The source operand can be a register or memory
location.

\subsubsection*{Flags Affected}

ZF is set to 1 if the result is zero.  The CF, PF, AF, SF, and OF
flags are undefined.
