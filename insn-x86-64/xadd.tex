\clearpage
\phantomsection
\addcontentsline{toc}{subsection}{XADD -- Exchange and Add}
\insnxeslabel{xadd}
\subsection*{XADD -- Exchange and Add}

\begin{x86opcodetable}
  \xopcode{CAP + 0F C1 \emph{/r}}{XADD \emph{r/mc, rc, r64}}
  {MRR}{Valid}{Valid}
  {Load original value of \emph{r/mc} into \emph{rc}.  Add \emph{r64}
    to the address field of \emph{r/mc}.}
\end{x86opcodetable}

\begin{x86opentable}
  \xopen{MRR}{ModRM:r/m (r,w)}{ModRM:reg (w)}{ModRM:reg (r)}{NA}
\end{x86opentable}

\subsubsection*{Description}

Stores the original value of the destination (first) operand in the second
operand.  Derives a new capability value by adding the third operand
to the \textbf{address} field of the destination operand and stores
the result in the destination operand.  Note that the third operand
must be the 64-bit register which aliases the low 64-bits of the
second operand.

If the new value of the \textbf{address} field makes the resulting
capability unrepresentable, the \textbf{tag} field in the resulting
capability is cleared.

The instruction can be used with a \insnnoref{LOCK} prefix to execute
atomically.

\subsubsection*{Flags Affected}

The OF, SF, ZF, AF, CF, and PF flags are set according to the value of
the resulting \textbf{address} field.
