\clearpage
\phantomsection
\addcontentsline{toc}{subsection}{PUSH - Push Value Onto the Stack}
\insnxeslabel{push}
\subsection*{PUSH - Push Value Onto the Stack}

\begin{x86opcodetable}
  \xopcode{FF /6}{PUSH \emph{r/m16}}
  {M}{N.E.}{Valid}
  {Push \emph{r/m16}.}
  \xopcode{FF /6}{PUSH \emph{r/m32}}
  {M}{N.E.}{N.E.}
  {Push \emph{r/m32}.}
  \xopcode{FF /6}{PUSH \emph{r/m64}}
  {M}{Valid}{Valid}
  {Push \emph{r/m64}.}
  \xopcode{FF /6}{PUSH \emph{r/mc}}
  {M}{Valid}{Valid}
  {Push \emph{r/mc}.}
  \xopcode{50+\emph{rw}}{PUSH \emph{r16}}
  {O}{N.E.}{Valid}
  {Push \emph{r16}.}
  \xopcode{50+\emph{rd}}{PUSH \emph{r32}}
  {O}{N.E.}{N.E.}
  {Push \emph{r32}.}
  \xopcode{50+\emph{ro}}{PUSH \emph{r64}}
  {O}{Valid}{Valid}
  {Push \emph{r64}.}
  \xopcode{50+\emph{rc}}{PUSH \emph{rc}}
  {O}{Valid}{Valid}
  {Push \emph{rc}.}
  \xopcode{6A \emph{ib}}{PUSH \emph{imm8}}
  {I}{Valid}{Valid}
  {Push \emph{imm8}.}
  \xopcode{68 \emph{iw}}{PUSH \emph{imm16}}
  {I}{Valid}{Valid}
  {Push \emph{imm16}.}
  \xopcode{68 \emph{id}}{PUSH \emph{imm32}}
  {I}{Valid}{Valid}
  {Push \emph{imm32}.}
\end{x86opcodetable}

\begin{x86opentable}
  \xopen{M}{ModRM:r/m (r)}{NA}{NA}{NA}
  \xopen{O}{opcode + rd (r)}{NA}{NA}{NA}
  \xopen{I}{imm8/16/32}{NA}{NA}{NA}
\end{x86opentable}

\subsubsection*{Description}

Decrements the stack pointer and stores the source operand on the
stack.  The following extensions apply to these instructions in 64-bit
mode:

\begin{itemize}
  \item Address size: The 0x07 prefix selects a capability-aware
    address.

  \item Operand size: The \insnnoref{CAP} prefix selects a capability
    operand size.  When a capability operand size is used, immediate
    operands are sign-extended to 64-bits and the result used as the
    address of a null-derived capability.
\end{itemize}

In capability mode, the various sizes are:

\begin{itemize}
  \item Address size: The default addressing mode uses
    capability-aware addressing.  64-bit addresses can be used by
    specifying the 0x07 prefix.

  \item Operand size: The default operand size is a capability.  If
    the \insnnoref{CAP} prefix is specified, the operand size is
    64-bits.  Immediate operands are sign-extended to 64-bits.  When a
    capability operand size is used, sign-extended immediate operands
    are used as the address of a null-derived capability.

  \item Stack-address size: In capability mode the stack pointer is
    always \CSP{}.
\end{itemize}

\subsubsection*{Flags Affected}

None
