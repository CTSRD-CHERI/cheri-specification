\clearpage
\phantomsection
\addcontentsline{toc}{subsection}{CALL -- Call Procedure}
\insnxeslabel{call}
\subsection*{CALL -- Call Procedure}

\begin{x86opcodetable}
  \xopcode{E8 \emph{cw}}{CALL \emph{rel16}}
  {D}{Invalid}{N.S.}
  {Call near, relative displacement.}
  \xopcode{E8 \emph{cd}}{CALL \emph{rel32}}
  {D}{Valid}{Valid}
  {Call near, relative displacement, 32-bit displacement sign-extended
    to 64-bits.}
  \xopcode{FF /2}{CALL \emph{r/m16}}
  {M}{N.E.}{N.E.}
  {Call near, absolute indirect, address given in \emph{r/m16}.}
  \xopcode{FF /2}{CALL \emph{r/m32}}
  {M}{N.E.}{N.E.}
  {Call near, absolute indirect, address given in \emph{r/m32}.}
  \xopcode{FF /2}{CALL \emph{r/m64}}
  {M}{Invalid}{Valid}
  {Call near, absolute indirect, address given in \emph{r/m64}.}
  \xopcode{FF /2}{CALL \emph{r/mc}}
  {M}{Valid}{Valid}
  {Call near, absolute indirect, address given in \emph{r/mc}.}
  \xopcode{9A \emph{cd}}{CALL \emph{ptr16:16}}
  {D}{N.E.}{N.E.}
  {Call far, absolute, address in operand.}
  \xopcode{9A \emph{cp}}{CALL \emph{ptr16:32}}
  {D}{N.E.}{N.E.}
  {Call far, absolute, address in operand.}
  \xopcode{FF /3}{CALL \emph{m16:16}}
  {M}{Invalid}{Valid}
  {Call far, absolute indirect, address given in \emph{m16:16}.}
  \xopcode{FF /3}{CALL \emph{m16:32}}
  {M}{Invalid}{Valid}
  {Call far, absolute indirect, address given in \emph{m16:32}.}
  \xopcode{REX.W FF /3}{CALL \emph{m16:64}}
  {M}{Invalid}{Valid}
  {Call far, absolute indirect, address given in \emph{m16:64}.}
\end{x86opcodetable}

\begin{x86opentable}
  \xopen{D}{Offset}{NA}{NA}{NA}
  \xopen{M}{ModRM:r/m (r)}{NA}{NA}{NA}
\end{x86opentable}

\subsubsection*{Description}

Saves return address on the stack and branches to the location
specified by the first operand.  In 64-bit mode the \insnnoref{CAP}
prefix can be used with opcode \texttt{FF /2} to select the capability
operand size instead of 64-bit.  In capability mode, near calls always
use the capability operand size.

Relative calls always apply the relative displacement to the address
of the next instruction to compute a new value of the \textbf{address}
field of \CIP{}.

Near calls which use a capability operand size always push \CIP{} onto
the current stack before executing the first instruction at the target
rather than pushing the value of \RIP{}.

\subsubsection*{Flags Affected}

None
