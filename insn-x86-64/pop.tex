\clearpage
\phantomsection
\addcontentsline{toc}{subsection}{POP - Pop Value from the Stack}
\insnxeslabel{pop}
\subsection*{POP - Pop Value from the Stack}

\begin{x86opcodetable}
  \xopcode{8F /0}{POP \emph{r/m16}}
  {M}{N.E.}{Valid}
  {Pop \emph{r/m16}.}
  \xopcode{8F /0}{POP \emph{r/m32}}
  {M}{N.E.}{N.E.}
  {Pop \emph{r/m32}.}
  \xopcode{8F /0}{POP \emph{r/m64}}
  {M}{Valid}{Valid}
  {Pop \emph{r/m64}.}
  \xopcode{8F /0}{POP \emph{r/mc}}
  {M}{Valid}{Valid}
  {Pop \emph{r/mc}.}
  \xopcode{58+\emph{rw}}{POP \emph{r16}}
  {O}{N.E.}{Valid}
  {Pop \emph{r16}.}
  \xopcode{58+\emph{rd}}{POP \emph{r32}}
  {O}{N.E.}{N.E.}
  {Pop \emph{r32}.}
  \xopcode{58+\emph{ro}}{POP \emph{r64}}
  {O}{Valid}{Valid}
  {Pop \emph{r64}.}
  \xopcode{58+\emph{rc}}{POP \emph{rc}}
  {O}{Valid}{Valid}
  {Pop \emph{rc}.}
\end{x86opcodetable}

\begin{x86opentable}
  \xopen{M}{ModRM:r/m (w)}{NA}{NA}{NA}
  \xopen{O}{opcode + rd (w)}{NA}{NA}{NA}
\end{x86opentable}

\subsubsection*{Description}

Stores the value at the top of the stack in the destination operand
and increments the stack pointer.  The following extensions apply to
these instructions in 64-bit mode:

\begin{itemize}
  \item Address size: The 0x07 prefix selects a capability-aware
    address.

  \item Operand size: The \insnnoref{CAP} prefix selects a capability
    operand size.
\end{itemize}

In capability mode, the various sizes are:

\begin{itemize}
  \item Address size: The default addressing mode uses
    capability-aware addressing.  64-bit addresses can be used by
    specifying the 0x07 prefix.

  \item Operand size: The default operand size is a capability.  If
    the \insnnoref{CAP} prefix is specified, the operand size is
    64-bits.  16-bit and 32-bit operands cannot be used in capability
    mode.

  \item Stack-address size: In capability mode the stack pointer is
    always \CSP{}.
\end{itemize}

\subsubsection*{Flags Affected}

None
