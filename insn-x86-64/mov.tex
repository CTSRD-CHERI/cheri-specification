\clearpage
\phantomsection
\addcontentsline{toc}{subsection}{MOV - Move}
\insnxeslabel{mov}
\subsection*{MOV - Move}

\begin{x86opcodetable}
  \xopcode{CAP + 89 \emph{/r}}{MOV \emph{r/mc, rc}}
  {MR}{Valid}{Valid}
  {Move \emph{rc} to \emph{r/mc}.}
  \xopcode{CAP + 8B \emph{/r}}{MOV \emph{rc, r/mc}}
  {RM}{Valid}{Valid}
  {Move \emph{r/mc} to \emph{rc}.}
  \xopcode{CAP + C7 \emph{/0 id}}{MOV \emph{r/mc, imm32}}
  {MI}{Valid}{Valid}
  {Move \emph{imm32 sign-extended to 64-bits} to \emph{r/mc}.}
\end{x86opcodetable}

\begin{x86opentable}
  \xopen{MR}{ModRM:r/m (w)}{ModRM:reg (r)}{NA}{NA}
  \xopen{RM}{ModRM:Reg (w)}{ModRM:r/m (r)}{NA}{NA}
  \xopen{MI}{ModRM:r/m (w)}{imm32}{NA}{NA}
\end{x86opentable}

\subsubsection*{Description}

Copies the source operand to the destination operand. The destination
operand can be a register or a memory location. The source operand can
be an immediate, a register, or a memory location.  If the source
operand is an immediate, the value is sign-extended to 64-bits and
used as the address of a NULL-derived capability.

Note that some \insnnoref{MOV} opcodes such as \texttt{B8+ rw} and
\texttt{C6 /0} are not extended to support the \insnnoref{CAP} prefix
as the behavior would be identical.  The \texttt{C7 /0} opcode is
extended primarily to support storing constants such as NULL to
capabilities in memory without requiring an intermediate register.

The \texttt{A1} and \texttt{A3} opcodes are not extended to support
capabilities.

\subsubsection*{Flags Affected}

None
