\chapter{CHERI in High-Assurance Systems}
\label{chap:assurance}

This chapter considers the roles of formal methods relating to the assurance
of CHERI-MIPS hardware and software.  It gives an informal explanation of some
features of the CHERI mechanism that may of interest to developers of
high-assurance hardware, secure microkernels, and formal models of CHERI,
including an initial security argument for a reference monitor.  Further
work on proofs of properties of the CHERI ISA and were published in our
IEEE Symposium on Security and Privacy in 2020~\cite{cheri-formal-SP2020}.

\section{Unpredictable Behavior}

In the semantics for the CHERI instructions
we try
to avoid defining behavior as ``unpredictable''. There were several reasons
for avoiding unpredictable behavior, including the difficulty it creates for
formal verification.
Although CHERI is based on the MIPS ISA,
the MIPS ISA specification (e.g., for the
R4000) makes extensive use of ``unpredictable''. If ``unpredictable'' is
modeled as ``anything could happen'', then clearly the system is not secure.
As a
concrete example, imagine a hypothetical CHERI implementation that contains
a Trojan horse such that when a sandboxed program executes an arithmetic
instruction whose result is ``unpredictable'', it also changes the capability
registers so that a capability granting access to the entire virtual address
space is placed in a capability register. If ``unpredictable'' means that
anything could happen, then this is compliant with the MIPS ISA; it is also
obviously insecure. Later versions of the MIPS ISA (e.g., MIPS64 volume I) make
it clear that ``unpredictable'' is more restrictive than this, saying that
``\emph{unpredictable} operations must not read, write, or modify the contents of
memory or internal state that is inaccessible in the current processor mode''.
However, that is clearly not strong enough.

For the CHERI mechanism to be secure, we require that programs whose behavior
is ``unpredictable'' according to the MIPS ISA do not modify memory or
capability registers in a way that allows the capability mechanism to be bypassed.
One easy way to achieve this is that the ``unpredictable'' case requires that
neither memory nor capability registers are modified.

The test suite for our CHERI1 FPGA implementation checks that the CPU follows
known CHERI1-specific behavior in the ``unpredictable'' cases.

\section{Bypassing the Capability Mechanism Using the TLB}

If a program can modify the TLB (the status register has CU0 set, KSU not
equal to 2, EXL set or IRL set), then it can bypass the capability mechanism
by modifying the TLB. Although composition with the Memory Management Unit and
virtual-addressing mechanism in this manner is a critical and intentional part
of our design, it is worth considering the implications from the perspective
of high-assurance design.
The ``attack'' is as follows: Consider a location in memory
whose virtual address is not accessible using the capability mechanism; take
its physical address and change the TLB so that its new virtual address is one
to which you have a capability, and then access the data through the new virtual
address. There are several ways to prevent this attack:

\begin{itemize}
\item
In CheriBSD, user-space programs are unable to modify the TLB (except through
system calls such as \ccode{mmap}), and thus cannot carry out this attack.
This  security argument makes it explicit that the security of the capability
mechanism depends on the correctness of the underlying operating system.
However, this may not be adequate for high-assurance systems.
\item
Similarly, a high-assurance microkernel could run untrusted code in user
space, with KSU=2, CU0 false, EXL false, and IRL false. A security proof
for the combined hardware-software system could verify that untrusted code
cannot cause this condition to become false except by reentering the microkernel
via a system call or exception.
\item
A single-address-space microkernel that has no need for the TLB could run on
a CHERI-enabled CPU without a TLB. Our CHERI1 FPGA prototype can be
synthesized in a version without a TLB, and our formal model in the L3
specification language includes a TLB-less variant. Removing the TLB for
applications that don't need it saves chip area, and removes the risk that the
TLB could be used as part of an attack.
\item
We are considering future extensions to CHERI that would allow the capability
mechanism to be used for sandboxing in kernel mode; these would allow more control
over access to the TLB when in kernel mode. As well as enabling sandboxing of
device drivers in monolithic kernels such as that of CheriBSD, the same mechanism
could also be used by microkernels.
\end{itemize}

\section{Malformed Capabilities}

The encoding formats for capabilities can represent values that can never be
created using the capability instructions while taking the initial contents of
the capability registers as a starting point. For example, with compressed
128-bit capabilities, there are bit patterns corresponding to \cbase{} $+$
\clength{} $> 2^{64}$.
The capability registers are initialized on reset, so there will
never be malformed capabilities in the initial register contents, and
a CHERI instruction will never create malformed capabilities from
well-formed ones. However, DRAM is not cleared on system reset, so that it is
possible that the initial memory might contain malformed capabilities with the
tag bit set.

Operating systems or microkernels are expected to initialize memory before
passing references to it to untrusted code. (If you give untrusted code a capability that
has the \emph{Load\_Capability} permission and refers to uninitialized memory,
you don't know what rights you are delegating to it.)
This means that untrusted
code should not be in a position to make use of malformed capabilities.

There are (at least) two implementation choices. An implementation of the CHERI
instructions could perform access-control checks in a way that would
work on both well-formed and malformed capabilities.
Alternatively, the hardware could be
slightly simplified by performing the checks in a way that might behave
unexpectedly on malformed capabilities, and then rely on the capability
mechanism (plus the operating system initializing memory) to guarantee that
they will never become available to untrusted code.

If the hardware is designed to guard against malformed capabilities, this
presents special difficulties in testing. No program whose behavior
is defined by the ISA specification will ever trigger the case of encountering
a malformed capability. (Programs whose behavior is ``unpredictable'', because
they access uninitialized memory, may encounter them).
However, some approaches to
automatic test generation may have difficulty constructing such tests.

More generally, however, uninitialized memory might also contain highly
privileged and yet entirely well-formed capabilities, and hence references to
that memory should be given to less trustworthy code only after suitable
clearing.
This requirement is present today for current hardware, as uncleared memory on
boot might contain sensitive data from prior boots, but this requirement is
reinforced in a capability-oriented environment.

\section{Constants in the Formal Model}

The L3 language that we used to specify CHERI does not have a notion of
a named constant as distinct from a mutable variable. Fully machine-checked
security proofs may need to prove that some of these constants are in fact
constant. (For example, that it is not possible to bypass the capability
mechanism by changing the CPU's endianness and hence the effect of a capability
dereference, because there is no way to change the endianness).

\section{Outline of Security Argument for a Reference Monitor}

The CHERI ISA can be used to provide several different security properties (for
example, control-flow integrity or sandboxing). This section provides the
outline of a security argument for how the CHERI instructions can be used
to implement a reference monitor.

The Trusted Computer System Evaluation Criteria (``Orange Book'')~\cite{TCSEC}
expressed the requirement for a reference monitor as ``The TCB shall maintain
a domain for its own execution that protects it from external interference or
tampering''.

The Common Criteria~\cite{CC2012-3} contain a similar requirement:

\begin{quote}
``ADV\_ARC.1.1D The developer shall design and implement the [target of
evaluation] so that the security features of the [target of evaluation
security functionality] cannot be bypassed.''

``ADV\_ARC.1.2D The developer shall design and implement the [target of
evaluation security functionality] so that it is able to protect itself from
tampering by untrusted active entities.''
\end{quote}

In this section, we we explain how the CHERI mechanism can be used
to provide this requirement(s), and provides a semi-formal outline of a
proof of its correctness.

We are assuming that the system operates in an environment where
the attacker does not have physical access to the hardware, so that
hardware-level attacks such as introducing memory errors~\cite{Govinda+03}
are not applicable.
%%%% WE MIGHT ALSO MENTION the A2 best paper from IEEE SS&P 2016 on
%%%% injecting analog circuit malware into the hardware.

In this section, we do not consider covert channels. There are many
applications where protection against covert channels is not a requirement.
The CHERI1 FPGA implementation has memory caches, which probably could be
exploited as a covert channel.

The architecture we use to meet this requirement consists of
(a) some trusted code that initializes the CPU and then calls the untrusted
 code; and (b) some untrusted code. The CHERI capability mechanism is
used to restrict which memory locations can be accessed by the untrusted
code.
Here, ``trusted'' means
that, for the purpose of security analysis, we know what the code does. The
``untrusted'' code, on the other hand, might do anything.

The reference monitor consists of the trusted code and the CHERI hardware; and
the ``security domain'' provided for the reference monitor consists of a set of
memory addresses ($S_K$) for the data, code, and stack segments of the trusted
code, together with the CHERI reserved registers.

Our security requirement of the hardware is that the untrusted code will run
for a while, eventually returning control to the trusted code; and when the
trusted code is re-entered, (a) it will be reentered at one of a small number
of known entry points; (b) its code, data and stack will not have been modified
by the untrusted code; and (c) the reserved capability registers will not have
been modified by the untrusted code.

This security property provided by the hardware allows us to reason that the
trusted code is still trusted when it is reentered. If its code and data have
not been modified. we can still know what it will do
(to the extent that it is actually trustworthy -- not just ``trusted''),

The ``cannot be bypassed'' and ``tamperproof'' requirements  are here
interpreted as meaning that there is no way within the ISA to modify the
reference monitor's reserved memory or the reserved registers. That is, all
memory accesses are checked against a capability register, and do not succeed
unless the capability permits them. The untrusted code can access memory without
returning control to the trusted code;
however, all of its memory access are mediated
by the capability hardware, which is considered to be part of the reference
monitor. Tampering with the reference monitor by making physical modifications
to the hardware is considered to be out of scope; the attacker is assumed
not to have physical access.

The proof of this security property proceeds by induction on states. Let the
predicate \emph{SecureState} refer to the following set of conditions:

\begin{itemize}
\item
CP0.Status.KSU $\neq$ 0
\item
CP0.Status.CU0 = \algorithmicfalse{}
\item
CP0.Status.EXL = \algorithmicfalse{}
\item
CP0.Status.ERL = \algorithmicfalse{}
\item
The TLB is initialized such that every entry has been initialized; every entry has
a valid page mask; and there is no (ASID, virtual address) pair that matches multiple
entries.
\item
Let $S_U$ be a set of (virtual) memory addresses allocated for use by the untrusted code,
and $T_U$ a set of \cotype{} values allocated for use by the untrusted code.
\item
The set of virtual addresses $S_U$ does not contain an address that maps (under the
TLB state mentioned above) into any of the memory addresses reserved for use
by the trusted code's code, stack or data segments.
\item
The set of virtual addresses $S_U$ does not contain an address that maps (under the
TLB state mentioned above) into the physical address used by a memory-mapped I/O
device. (If this property is weakened to allow some I/O devices to be memory-mapped
by untrusted code, then the security proof has to show that the I/O device can't
be used to break the security property, e.g. by causing the I/O device to DMA
into a region of memory outside of $S_U$).
\item
The set of virtual addresses $S_U$ are all mapped to cached memory. (A load-linked
operation on uncached memory is defined as unpredictable in the MIPS ISA. While
this probably can't be used to attack a real system, any unpredictable behavior
has to prevent for provable security).
\item
All capability registers have \cbase{} $+$ \clength{} $\leq 2^{64}$
\algorithmicor{} \ctag{} $=$ \algorithmicfalse{}.
\item
The above is also true of all capabilities contained within the set of memory
addresses $S_U$.
\item
All capability registers are either (a) reserved registers; (b) have \ctag{} =
\algorithmicfalse{}; (c) are sealed with an \cotype{} not in $T_U$; or do not
grant \cappermASR* permission.
\item
The above is also true of all capabilities contained within the set of memory
addresses $S_U$.
\item
All capability registers are either (a) reserved registers; (b) have \ctag{} =
\algorithmicfalse{}; (c) are sealed with an \cotype{} not in $T_U$; or do not
grant access to a region of virtual addresses outside of $S_U$.
\item
The above is also true of all capabilities contained within the set of memory
addresses $S_U$.
\item
All capability registers are either (a) reserved registers; (b) have \ctag{} =
\algorithmicfalse{}; (c) are sealed with an \cotype{} not in $T_U$; or do not
grant access to a region of the \cotype{} space outside of $T_U$.
\item
The above is also true of all capabilities contained within the set of memory
addresses $S_U$.
\item
If the current instruction is in a branch delay slot, then the above restrictions
on capability registers also apply to the \PCC{} value that is the target of the
branch. That is, \emph{SecureState} is not true if the trusted code does a
\insnref{CJR} that grants privilege and then runs the first instruction of
the untrusted code in the branch delay slot.
\end{itemize}

Let the predicate \emph{TCBEntryState} refer to a state in which the trusted code
has been reentered at one of a small number of known entry points.

We assume that \emph{SecureState} is true initially (i.e.,
a requirement of
the trusted code is that it puts the CPU into this state before calling the
untrusted code).
We then wish to show that \emph{SecureState} $\Rightarrow$ \textbf{X}
 (\emph{SecureState} \algorithmicor{} \emph{TCBEntryState}) (where \textbf{X} is
the next operator in linear temporal logic). By induction on states,
\emph{SecureState} $\Rightarrow$ \emph{TCBEntryState} \textbf{R} \emph{SecureState}
(where \textbf{R} is the release operator in linear temporal logic).

The argument that \emph{SecureState} $\Rightarrow$ \textbf{X} (\emph{SecureState}
\algorithmicor{} \emph{TCBEntryState}) can be summarized as:

\begin{itemize}
\item
Given that CP0.Status.KSU $\neq$ 0, CP0.Status.CU0 = \algorithmicfalse{},
CP0.Status.EXL = \algorithmicfalse{} and CP0.Status.ERL = \algorithmicfalse{},
all instructions will either raise an exception (\textbf{X}
\emph{TCBEntryState}) or leave CP0 registers unchanged, leaving this
part of the \emph{SecureState} invariant unchanged.
\item
Given that CP0.Status.KSU $\neq$ 0 (etc.), all instructions will
either raise an exception or leave the TLB unchanged, preserving the parts of
\emph{SecureState} relating to the TLB.
\item
Given that the TLB is in the state given by \emph{SecureState}, load and store
operations will not result in ``undefined'' or ``unpredictable'' behavior due
to multiple matches in the TLB.
\item
Given that CP0.Status.KSU $\neq$ 0 (etc.), and the TLB is in the state
described above, no instruction can result in
behavior that is ``undefined'' according to the MIPS ISA. (The MIPS ISA
specification makes a distinction between ``undefined'' and ``unpredictable'',
but our model in the L3 language combines the two).
\item
However, instructions can still result in behavior that is ``unpredictable''
according to the MIPS ISA. These cases can be dealt with by providing a
CHERI-specific refinement of the MIPS ISA (i.e. describing what CHERI does in
these cases).
\item
The capability instructions preserve the part of \emph{SecureState} that
relates to the capability registers and to capabilities within $S_U$.
\item
Given that the capability registers (apart from reserved registers) do not grant
access to any memory addresses outside of $S_U$, store instructions might raise
an exception (\textbf{X} \emph{TCBEntryState}), but they will not modify locations
outside of $S_U$; thus, the trusted code's data, code and stack segments
 will be unmodified.
\item
Given that the capability registers (apart from the reserved registers) do
not grant \cappermASR* permission, the reserved
registers will not be modified.
\end{itemize}

The theorem \emph{SecureState} $\Rightarrow$ \emph{TCBEntryState} \textbf{R}
\emph{SecureState} uses the \textbf{R} operator, which is a weak form of
``until'': the system might continue in \emph{SecureState} indefinitely.
Sometimes it is desirable to have the stronger property that
\emph{TCBEntryState} is guaranteed to be reached eventually. This can be
ensured by having the trusted code enable timer interrupts, and use a
timer interrupt to force return to \emph{TCBEntryState} if the untrusted
code takes too long.

More formally, the following properties are added to \emph{SecureState}
to make a new predicate, \emph{SecureStateTimer}:

\begin{itemize}
\item
CP0.Status.IE = \algorithmictrue{}
\item
CP0.Status.IM(7) = \algorithmictrue{}
\end{itemize}

Given that CP0.Status.KSU $\neq$ 0 (etc.),  it follows that these properties
are also preserved, i.e.
\emph{SecureStateTimer} $\Rightarrow$ \emph{TCBEntryState} \textbf{R}
\emph{SecureStateTimer}.

As CP0.Count increases by at least one for every instruction, a timer interrupt
will eventually be triggered. (If Compare is 2, for example, and Count
increments from 1 to 3 without ever going through the intervening value of 2,
a timer interrupt is still triggered). As CP0.KSU $\neq$ 0, CP0.Status.EXL
= \algorithmicfalse{}, CP0.Status.ERL = \algorithmicfalse{}, CP0.Status.IE =
\algorithmictrue{} and CP0.Status.IM(7) = \algorithmictrue{}, the interrupt
will be enabled and return to \emph{TCBEntryState} will occur:

\emph{SecureStateTimer} $\Rightarrow$ \textbf{F} \emph{TCBEntryState}

It then follows that \emph{SecureStateTimer} $\Rightarrow$
\emph{SecureStateTimer} \textbf{U} \emph{TCBEntryState}, where
\textbf{U} is the until operator in linear temporal logic.

\section*{Illicit Information Flows}

Using an argument similar to the one in the preceding section, it ought
to be possible to formally prove confidentiality properties of the CHERI ISA.
However, proofs of confidentiality suffer from the ``refinement paradox'':
confidentiality properties are not preserved by refinement. If there is
any non-determinism in a specification, a refinement of it might leak
secret information via values that were originally left unspecified.

In more concrete terms, an implementation of the CHERI ISA might leak secret
information due to security problems at the microarchitectural level.

The Common Criteria~\cite{CC2017-1} uses the term ``covert channel'' (alternatively,
``illicit information flow'') for cases where it is possible to use
features of the implementation to signal information in a way that is
prohibited by the security policy.

The most obvious potential source of a covert channel in CHERI is using the
memory caches as a timing channel. Meltdown~\cite{Lipp2018meltdown} and
Spectre~\cite{Kocher2018spectre} are examples of realistic attacks against
a CPU's memory protection using the cache as a timing channel.
Subsequently, the related Foreshadow attacks have been reported~\cite{Foreshadow,Foreshadow-NG}.

To reduce the risk of an attack similar to Meltdown, implementations of CHERI
should perform MMU and capability permissions checks before a store or load,
rather than speculatively executing the load and store before all capability
checks have completed: tag violation, bounds check, permissions check, seal
check, and so on.
