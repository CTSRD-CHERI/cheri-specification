\chapter{Detailed Design Rationale}
\label{chap:rationale}

During the design of CHERI that began in 2010, we considered many
different capability architectures and design approaches. This chapter
describes various design choices; it briefly outlines some
possible alternatives, and provides rationales for the selected
choices.

\section{High-Level Design Approach: Capabilities as Pointers}

Our goals of providing fine-grained memory protection and compartmentalization
led to an early design choice to allow capabilities to be used as C- and
C++-language pointers.
This rapidly led to various conclusions:

\begin{itemize}
\item Capabilities exist within virtual address spaces, imposing an ordering in
  which capability protections are evaluated before virtual-memory
  protections; this in turn had implications for the hardware composition of
  the capability coprocessor and conventional interactions with the MMU.

\item Capability pointers can be treated by the compiler in much the same way
  as integer pointers, meaning that they will be loaded, manipulated,
  dereferenced, and stored via registers and to/from general-purpose memory
  only by explicit instructions.
  These instructions were modeled on similar conventional RISC instructions.

\item Incremental deployment within programs meant that not all pointers would
  immediately be converted from integers to capabilities, implying that both
  forms might coexist in the same virtual memory;
  also, there was a strong desire to embed capabilities
  within data structures, rather than store them in separate segments,
  which in turn required fine-granularity tagging.

\item Incremental deployment and compatibility with the UNIX model implied
  the need to retain
  the general-purpose memory management unit (MMU) more or less as
  it then existed, including support for variable page sizes, page table layout,
  and so on.
\end{itemize}

\section{Tagged Memory for Non-Probablistic Protection}
\label{sec:probablistic_capability_protection}

Introducing tagged memory has the potential to impose a substantial adoption
cost for CHERI, due to greater microarchitectural disruption.
We have demonstrated that there are efficient implementations of memory
tagging, even without integrated tag support within
DRAM~\cite{joannou2017:tagged-memory, UCAM-CL-TR-936}, but even so there is a significant
concern as to whether potential adopters will perceive the hurdle of adopting
tagged memory as outweighing the benefits that tagged memory brings.
In this section, we consider the benefits of tagging, as well as how
cryptographic non-tagged approaches might be used.
Tagging offers a number of significant potential benefits:

\begin{itemize}
\item Tags are a deterministic (non-probabilistic) means of protecting the
  integrity and provenance validity of pointers in memory.
  Probabilistic schemes, such as cryptographic hashes, are exposed both to
  direct brute forcing (especially due to limited bit investment within
  pointers) and also reinjection if leaked to attackers.

\item Tags offer strong atomicity properties that are also well-aligned with
  current microarchitecture (e.g., in caches), avoiding the need for
  substantial disruption close to the processor.

\item Tags have highly efficient microarchitectural implementations, including
  being directly embedded in tagged DRAM (an option likely to become increasingly
  available due to the widespread adoption of
error-correcting codes, and also via tag
  controllers and tag caches that are affine to the DRAM controller.
  These may be substantially more performance- and energy-efficient than
  cryptographic techniques that would require hashes to be calculated or checked.

\item Tags offer strong C-language compatibility, which has been demonstrated
  with significant software corpuses -- including operating-system kernels
  (FreeBSD), the complete UNIX userspace (FreeBSD), and significant C and
  C++-language applications (the Postgres database, OpenSSH client and server,
  and WebKit web-rendering framework).

  Key areas of incompatibility include the need to explicitly preserve tags
  during memory copies via capability-sized, capability-aligned loads and
  stores, and stronger alignment requirements for pointers.
  The operating system must also support maintaining tags in virtual memory,
  including across operations such as swapping, memory compression, and
  virtual-machine migration.
  In general, we have found that the modifications are modestly sized,
  although some impacts (such as the cost of tag preservation and
  restoration) are not yet fully quantified -- e.g., for memory compression.

\item Tags allow pointers to be deterministically identified in memory, a
  foundation for strong temporal memory-safety techniques such as revocation
  and garbage collection.

\item The choice between tag-preserving and tag-stripping memory copying
  allows software to impose policies on when it is appropriate and safe for
  pointers to move between protection domains.
  For example, a kernel can selectively preserve tags in system-call
  arguments,
  preventing data copied into the kernel from an untrustworthy process from
  being interpreted as a pointer within the kernel, or when received by
  another process.
\end{itemize}

As an alternative to tagging, one could imagine making use of probabilistic
cryptographic hashing techniques that protect capabilities from corruption,
not unlike Cryptographic Control-Flow Integrity
(CCFI)~\cite{Mashtizadeh_CCFICryptographicallyEnforced_2015} or Arm's ARM v8.3 Pointer
Authentication Codes (PAC).
Some number of bits would be co-opted from either the virtual address (as is
the case in CCFI or PAC), or from the metadata portion of a CHERI capability to
hold a keyed hash, protecting the contents from corruption in memory or due to
mis-manipulation in a register, rather than a tag.
With additional capability metadata bits available, consumption of
virtual-address bits could be reduced.

Wherever the CHERI architecture requires a tag check, a cryptographic hash
check could instead be required architecturally.
Wherever the CHERI architecture maintains a tag during pointer manipulation,
the cryptographic hash could be updated.
While architectural behavior might appear to require frequent checks of, and
updates to, the hash (e.g., during loop iteration as a register is
successively incremented and then used for loads or stores), it is conceivable
that microarchitectural techniques (such as speculation) might both reduce the
delay associated with those updates, and perhaps also elide them entirely,
updating the hash only during write back.
Tags appear to offer the following essential advantages over cryptographic
approaches:

\begin{itemize}
\item Tags offer deterministic rather than probabilistic protection,
  and require neither secrecy of a cryptographic key, nor brute-forcing resistance given a
  bounded number of hash bits.
  Depending on the OS model, cryptographic keys might also be shared by more
  than one address space -- e.g., if \ccode{fork()} is frequently used to
  generate multiple processes, or if there is a shared memory segment that
  includes linked pointers.

\item Tags do not rely on cryptographic hash generation during capability
  updates, nor checking during dereference.
  These could otherwise lead to a performance overhead (e.g., as a result of
  load-to-use or check-to-use delays), or energy-use overheads (due to
  frequent cryptographic hash operations).

\item Tags prevent reinjection of leaked pointer values, even though the
  bitwise pattern of the addressable memory contents remain identical.
  Potential vulnerabilities with hash-based protection include leaking a valid
  pointer value to a local or remote attacker via socket communications.
  The attacker could later reinject that value -- potentially into a different
  process if they share keying material (e.g., if they are forked from the
  same parent).

\item Tags ensure provenance validity of capabilities, such that the TCB can
  deterministically ensure that a pointer value is no longer in memory.
  As with the previous item, this protects against reinjection, but has the
  stronger inductive property that the TCB can reliably perform revocation or
  garbage collection.
  This is also essential to compartmentalization strength.
\end{itemize}

However, a hash-based approach also has several appealing properties when
compared to tags:

\begin{itemize}
\item Cryptographic hashes do not require the implementation of tagged memory,
  which could reduce memory-subsystem complexity and DRAM-traffic impact.

\item Cryptographic hashes do not impose alignment requirements on
  capabilities, which may improve compatibility.

\item Cryptographically protected capabilities can be copied in memory,
  swapped to disk, or migrated in virtual-machine images, without special
  support for tags.

  This could entirely avoid the need for special capability load and store
  instructions, although retaining them might assist with microarchitectural
  optimization of hash use.
\end{itemize}

If hashed-based protection were viewed as a stepping stone to a full CHERI
implementation, substituting hashing for tags in an initial implementation,
there are several steps that could be taken to reduce the further disruption
associated with later tag adoption:

\begin{itemize}
\item Explicit capability load and store instructions would be maintained and
  used in future capability-aware memory copying, etc.

\item Capability load and store instructions would require strong alignment
  for values that would later be used for load and store, even though this is
  not required with hashing.

\item Other non-tag-related capability properties, such as monotonicity, would
  continue to be enforced via guarded manipulation.
\end{itemize}

However, substantially smaller benefit would arise prior to the introduction
of tags: capabilities would be able to provide capability-like spatial memory
protection, and probabilistic pointer integrity protection, but not the
non-probabilistic protection or enforcement of provenance validity required
for stronger policies such as preventing pointer reinjection, supporting
temporal memory safety through deterministic pointer identification in memory,
or enabling in-address-space compartmentalization that depends on those
properties.

\section{Capability Register File}

The decision to separate the capability-register file from the general-purpose
integer
register file is somewhat arbitrary from a software-facing perspective: we
envision capabilities gradually displacing general-purpose integer registers as
pointers, but where management of the two register files will remain largely
the same, with stack spilling behaving the same way, and so on, as is already
the case for disjoint integer and floating-point register files.

We initially selected the separate representation on CHERI-MIPS for a
few pragmatic reasons:

\begin{itemize}
\item Coprocessor interfaces on MIPS assume additional
  register files (a la floating-point registers).
\item The initial 256-bit capability registers were quite large, and by giving the capability
  coprocessor its own pipeline for manipulations, we could avoid enforcing a
  256-bit-wide path through the main pipeline.
\item It is more obvious, given a coprocessor-based interface, how to provide
  compatibility support in which the capability coprocessor is ``disabled,''
  the default configuration in order to support unmodified MIPS compilers and
  operating systems.
\end{itemize}

\noindent
However, a design in
which (more similar to the manner in which the 32-bit x86 ISA was extended to
support 64-bit registers) the two files are \textit{merged} and able to hold
both general-purpose integer registers and capability registers is
also possible.
This became a more appealing choice in the presence of 128-bit compressed
capabilities, as register size doubled rather than quadrupled.

Early in our design cycle, capability registers were able to hold only true
capabilities (i.e., with tags); later, we weakened this requirement by adding
an explicit tag bit to each register, in order to improve support for
capability-oblivious code such as memory-copy routines able to copy data
structures consisting of both capabilities and ordinary data.
This shifted our approach somewhat more towards a merged approach; our view
is that efficiency of implementation and compatibility (rather than
maintaining a negligible
effect on the software model) would be the primary reasons to select one
approach or another for a particular starting-point ISA.

Another design variation might have specific capability registers
more tightly coupled with general-purpose integer registers -- an approach we discussed
extensively, especially when comparing with the bounds-checking literature,
which has explored techniques based on {\em sidecar registers} or associative
look-aside buffers.
Many of these approaches did not adopt tags as a means of strong integrity
protection (which we require for the compartmentalization model), which
would make associative techniques less suitable.
Further, we felt that the working-set properties of the two register files
might be quite different; effectively pinning the two to one another would
reduce the efficiency of both.

With a split register file, it may be useful for the architecture to add
instructions for copying non-capability data from a capability register
into a general-purpose integer register. A use case is when a function is called
with a parameter whose type is the union of a pointer and a non-pointer type,
such as an int. This parameter must be passed in a capability register, because
the tag needs to be preserved when it holds a capability. If the body of
the function accesses the non-capability branch of the union, it needs to
get the non-capability bits out of the capability register and into a general
purpose register. This can be done by spilling the capability register to the
stack and then reading it back into a general-purpose integer register, but a register
to register copy would be faster.

CHERI-RISC-V (Chapter~\ref{chap:cheri-riscv}),
Morello, and CHERI-x86-64 (Chapter~\ref{chap:cheri-x86-64}) use merged register files.

\section{The Compiler is Not Part of the TCB for Isolated Code}

CHERI is designed to support the isolation of arbitrary untrustworthy code,
including code compiled with an incorrect or compromised compiler.
The security argument outlined in
Chapter~\ref{chap:assurance} starts with the premise that the attacker is able to
run arbitrary machine-code. This approach has advantages for high-assurance systems:
compilers are often large and complex programs, and proving correctness of their
security mechanisms is easier if it does not depend on also proving the correctness
of the compiler. This approach also has the advantage that users are not restricted
by the security design to programming in just one programming language, and can use
any language for which a compiler has been written. In particular, it is a design
goal of CHERI that it be able to run legacy code written in C.

Some earlier capability machines, such as the Burroughs B5000, made the compiler
a privileged program. We have followed the alternative approach taken in capability machines
such as CAP, in which the compiler was not privileged.

\mrnote{We could expand on this, perhaps in the high-assurance section. We do depend
on the compiler being correct, in the sense that if the attacker has complete
control of the compiler, he can make the programs you've compiled with it do
whatever you want. The property we're looking for is more like: assuming the TCB
has been compiled with a correct compiler, we can allow untrusted users to compile
their code using whatever compiler they want, without fear that this will let them
break out of the sandbox. We probably do depend on the \emph{dynamic linker} being
correct -- this depends on how we load code into a sandbox.}

\section{Base and Length Versus Lower and Upper Bounds}

The CHERI architecture permits two different interpretations of capabilities:
as a virtual address paired with lower and upper bounds, and as a base,
length, and current offset.
These different interpretations support differing C-language models for
pointers.
The former, in which pointer-casts to integers return their virtual addresses,  is more compatible with current software, but risks leaking those virtual
addresses (or their implications) out of tagged values where they cannot be
found for the purposes of pointer-transformation techniques such as copying
garbage collection.
The latter, in which pointer-casts to integers return their offsets, is less
compatible (as comparisons between pointers into different buffers may give
surprising equality results), but avoids leakage of virtual address out of
tagged values, enabling techniques such as copying garbage collection.

Over time, our thinking on these two approaches has shifted from aiming to
support copying garbage collection in C to one focused on revocation and
greater compatibility.
While some C source code naturally is extremely careful to avoid integer
interpretations of pointers, significant amounts of historic code, especially
systems code, cannot avoid this idiomatic use.
For example, run-time linkers and memory allocators both naturally consider
integer virtual addresses as part of their operation.
More subtly, techniques such as ordering locks for objects based on object
address, or sorting trees based on object address, make copying garbage
collection a difficult prospect.
\pgnnote{copying???}
Compressed capabilities further complicate this story, as a precise lower
bound may not be possible without padding; this is easy to arrange within
memory allocators for new allocations, but when subsetting an existing
allocation (e.g., to describe the bounds of an array embedded within another
structure), the 0 offset from the bottom of the embedded structure may not
carry over to being a 0 offset relative to the base address of a capability.

In recent versions of the CHERI C compiler (with the CHERI-LLVM
back-end), we have shifted to preferring a virtual-address
interpretation of pointers in all cases except those where specific
built-in functions are used to query the offset.  We retain an
optional compiler mode utilizing an offset interpretation, which will
be suitable for future experimentation with copying garbage
collection.

\section{Signed and Unsigned Offsets}

In the CHERI instructions that take both a register offset and an immediate
offset, the register offset is treated as unsigned integer, whereas the
immediate offset is treated as a signed integer.

Register offsets are treated as unsigned, so that given a capability to
the entire address space (except for the very last byte, as
explained above), a register offset can be used to access any byte within it.
Signed register offsets would have the disadvantage that negative offsets
would fail the capability bounds check, and memory at offsets within the
capability greater than $2^{63}$ would not be accessible.

Immediate offsets, on the other hand, are signed, because the C compiler
often refers to items on the stack using the stack pointer as register
offset plus a negative immediate offset.
We have already encountered observable difficulty due to a reduced number of
bits available for immediate offsets in capability-relative memory operations
when dealing with larger stack-frame sizes; it is unclear what real
performance cost this might have (if any), but it does reemphasize the
importance of careful investment of how instruction bits are encoded.

\section{Address Computation Can Wrap Around}

If the target address of a load or store (base $+$ offset $+$ register offset
$+$ scaled immediate offset) is greater than \emph{max\_addr} or less than
zero, it wraps around modulo $2^{64}$. The load or store succeeds if this
modulo arithmetic address is within the bounds of the capability (and other
checks, such as for permissions, also succeed).

An alternative choice would have been for an overflow in the address computation
to cause the load or store to fail with a length-violation exception.

The approach of allowing the address to wrap around does not allow malicious
code to break out of a sandbox, because a bounds check is still performed on
the wrapped-around address.

However, there is a potential problem if a program uses an array offset that
comes from a potentially malicious source. For example, suppose that code for
parsing packet headers uses an offset within the packet to determine the
position of the next header. The threat is that an attacker can put in a
very large value for the offset, which will cause wrap-around, and result
in the program accessing memory that it is permitted to access, but was not
intended to be accessed at this point in the packet processing. This attack
is similar to the confused deputy attack. It can be defended against by
appropriate use of \insnref{CSetBounds}, or by using some explicit
range checks in application code in addition to the bounds checks that are
performed by the capability hardware.
\nwfnote{Maybe "Using \insnref{CSetBounds} to derive a capability
to just the array, and using this capability for offsetting, supplants any
explicit range checks in application code."  This might also be a good place
to say something about Meltdown and Spectre (variant 1)?  "By informing the
architecture of the intended bounds of access, even speculative use of a
capability can be precisely confined."}

The advantage of the approach that we have taken is that it fits more naturally
with C language semantics, and with optimizations that can occur inside compilers.
The following are equivalent in C:

\begin{itemize}
\item
a[x + y]
\item
*(a + x + y)
\item
(a + x)[y]
\item
(a + y)[x]
\end{itemize}

They would not be equivalent if they had different behavior on overflow, and
the C compiler would not be able to perform optimizations that relied on
this kind of reordering.

\section{Overwriting Capabilities in Memory}

In CHERI, if a valid in-memory capability is partly overwritten via an
untagged data store, then the tag associated with the in-memory capability
is cleared, making it an invalid capability that cannot be dereferenced.

Alternative designs would have been for the capability to be zeroed first
before being overwritten; or for the write to raise an exception (with
an explicit ``clear tag in memory'' operation for the case when a
program really intends to overwrite a capability with non-capability data).

The chosen approach is simpler to
implement in hardware. If store instructions needed to check the tag bit
of the memory location that was being written, then they would need a
read-modify-write cycle to memory, rather than just a write.
(However, once the memory system needs
to deal with cache coherence, a write is not that much simpler than a
read-modify-write.)

The CHERI behavior also has the advantage that programs can write to a
memory location (e.g., when spilling a register onto the stack) without
needing to worry about whether that location previously contained a
capability or non-capability data.

A potential disadvantage is that the contents of capabilities cannot be
kept secret from a program that uses them. A program can always discover
the contents of a capability by overwriting part of it, then reading the
result as non-capability data. In CHERI, there are
intentionally
other, more direct, ways
for a program to discover the contents of a capability it owns, and this
does not present a security vulnerability.

However, there are ABI concerns: we have tried to design the ISA in such a
way that software does not need to be aware of the in-memory layout of
capabilities.  As it is necessarily exposed, there is a risk that software
might become dependent on a specific layout.
One noteworthy case is in the operating-system paging code, which must
save and restore capabilities and their tags separately.
This can be
accomplished by using instructions such as \insnref{CGetBase} on untagged
values loaded from disk and then refining an in-hand capability using
\insnref{CSetBounds}; however,
this requires a complex series of instructions.
\insnref{CBuildCap} can add a
tag to an untagged value in a capability-register operand authorized by a
second operand holding a suitably authorized capability.
This avoids software
awareness of the in-memory layout and accelerates tag restoration
when implementing system services such as swap.
This instruction in effect implements rederivation, which is also possible
using a sequence of individual instructions refining the authorizing
capabilities bounds, permissions, object type, and so on.
\insnref{CBuildCap} is not intended to change the set of reachable
capabilities.

\section{Reading Capabilities as Bytes}

In CHERI, if a non-capability data load instruction such as \insnnoref{LD} is used
on a memory location containing a capability, the internal representation
of the capability is read. An alternative architecture would have
such loads return zero, or raise an exception.

As noted above,
because the contents of capabilities are not secret, allowing them to be
read as raw data is not a security vulnerability.

\section{OTypes Are Not Secret}

Another consequence of the decision not to make the contents of capabilities secret
is that the \cotype{} field is not secret. It is possible to determine the
\cotype{} of a capability by reading it with \insnref{CGetType}, or by
reading the capability as bytes. If a program has two pairs of code and data
capabilities, ($c_1$, $d_1$) and ($c_2$, $d_2$) it can check if $c_1$ and $c_2$
have the same \cotype{} by invoking \insnref{CInvoke} on ($c_1$, $d_2$).
\jrtcnote{This is a weird thing to say; yes you implicitly check by not
trapping, but, uh, don't use it that way?}

As a result, a program can tell whether it has been passed an object of
\cotype{} O or an interposing object of \cotype{} I that  forwards the
\insnref{CInvoke} on to an object of \cotype{} O (e.g. after having performed
some additional access control checks or auditing first).

\section{Capability Registers are Dynamically Tagged}

In CHERI, capability registers and memory locations have a tag bit
that indicates whether they hold a capability or non-capability data.
(An alternative architecture would give memory locations a tag bit,
where capability registers could contain only capabilities -- with
an exception raised if an attempt were made to load non-capability data into a
capability register with \insnref{CLC}.)

Giving capability registers and memory locations a tag bit
simplifies the implementation of \ccode{memcpy()}.
In CHERI, \ccode{memcpy()} copies
the tag bit as well as the data so that it can be used to copy structures
containing capabilities. As capability registers are dynamically tagged,
\ccode{memcpy()} can copy a structure by loading
its constituent words into capability
registers and storing them to memory, without needing to know at compile time
whether it is copying a capability or non-capability data.

Tag bits on capability registers may also be useful for dynamically typed
languages in which a parameter to a function can be (at run time) either a
capability or an integer. \ccode{memcpy()} can be regarded as
a function whose parameter (technically a \ccode{void *}) is
dynamically typed.

\section{Separate Permissions for Storing Capabilities and Data}

CHERI has separate permission bits for storing a capability versus storing
non-capability data (and similarly, for loading a capability versus loading
non-capability data).

(An alternative design would be just one \cappermL{} and just one
\cappermS{} permission that were used for both capabilities and non-capability data.)

The advantage of separate permission bits for capabilities is that
that there can be two protected subsystems that communicate via a memory
buffer to which they have \cappermL{} and \cappermS{} permissions, but
do not have \cappermLC{} or \cappermSC{}. Such
communicating subsystems cannot pass capabilities via the shared buffer, even
if they collude. (We realized that this was potentially a requirement when
trying to formally model the security guarantees provided by CHERI.)

\section{Capabilities Contain a Cursor}

In the C language, pointers can be both incremented and decremented.
C pointers are sometimes used as a cursor that points to the current working
element of an array, and is moved up and down as the computation progresses.

CHERI capabilities include an offset field, which gives the difference between
the base of the capability and the memory address that is currently of
interest. The offset can be both incremented and decremented without changing
\cbase{}, so that it can be used to implement C pointers.

In the ANSI C standard, the behavior is undefined if a pointer is incremented
more than {\it one} beyond the end of the object to which it points. However, we have found
that many existing C programs rely on being able to increment a pointer beyond
the end of an array, decrement it back within range, and then deference it.
In particular, network packet processing software often does this.
In order to support programs that do this, CHERI offsets are allowed to take
on any value.%
%
\footnote{CHERI Concentrate (\cref{subsec:cheri-concentrate}) exploits the
observation that, in practice, pointers do not wander ``far'' from their base
to reduce the number of bits used to store the base, cursor, and limit
addresses.  Attempts to move the cursor far out of bounds will, instead, yield
an un-tagged result.}
%
A range check is performed when the capability is
dereferenced, so buffer overflows are prevented; thus, the offset can take
on intermediate out-of-range values as long as it is not dereferenced.

An alternative architecture would have not included an offset within the
capability. This could have been supported by two different capability types
in C, one that could not be decremented (but was represented by just a
capability) and one that supported decrementing (but was represented by a pair of
a capability and a separate integer for the offset).  Programming languages
that did not have pointer arithmetic could have their pointers compiled as
just a capability.

The disadvantage of including offsets within capabilities is that it wastes
64 bits in each capability in cases where offsets are not needed (e.g.,
when compiling languages that don't have pointer arithmetic, or when
compiling C pointers that are statically known to never be decremented).

The alternative (no offset) architecture could have used those 64 bits
of the capability for other purposes, and stored an extra offset outside
the capability when it was known to be needed.  The disadvantage of the
no-offset architecture is that C pointers become either unable to support
decrementing or enlarging: because capabilities need to be aligned, a pair of a
capability and an integer will usually end up
being padded to the size of two capabilities, doubling the size of a C pointer,
and this is a serious performance consideration.

Another disadvantage of the no-offset alternative is that it makes the
seal/unseal mechanism considerably more complicated and hard to explain.
A program that has a capability for a range of types has to somehow select
which type within its permitted range of types it wishes to use when sealing a
particular data capability. The CHERI architecture uses the offset for this
purpose; not having an offset field leads to more complex encodings when
creating sealed capabilities.

By comparison, the CCured language includes both \ccode{FSEQ} and
\ccode{SEQ} pointers. CHERI capabilities are analogous to CCured's
\ccode{SEQ} pointers. The alternative (no offset) architecture
would have capabilities that acted like CCured's FSEQ, and used an extra
offset when implementing SEQ semantics.
\jhbnote{This section seems relevant to the initial 256-bit
  capabilities and no-longer relevant for compressed capabilities.
  Perhaps it just needs to be explained as such rather than outright removed.}

\section{NULL Does Not Have the Tag Bit Set}

In some programming languages, pointer variables must always point to
a valid object. In C, pointers can either point to an object or be NULL;
by convention, NULL is the integer value zero cast to a pointer type.

If hardware capabilities are used to implement a language that has NULL
pointers, how is the NULL pointer represented? CHERI capabilities have
a \ctag{} bit; if the \ctag{} bit is set, a valid capability follows, otherwise
the remaining data can be interpreted as (for example) bytes or integers.
The representation we have chosen for NULL is that the \ctag{} bit is not set
and the \cbase{} and \clength{} fields are zero; effectively, NULL is the
integer zero stored as a non-capability value in a capability register.

An alternative representation we have could have chosen for NULL would
have been with the \ctag{} bit set, and zero in the \cbase{} field and
\clength{} fields.  Effectively, NULL would have been a capability for
an array of length zero.

The advantages of NULL's \ctag{} bit being unset are:

\begin{itemize}
\item
Initializing a region of memory by writing zero bytes to it will initialize
all capability variables within the region to the NULL capability. Initializing
memory by writing zeros is, for example, done by the C \ccode{calloc()}
function, and by some operating systems.
\end{itemize}

\section{The length of NULL is MAXINT}

Given that we have chosen NULL to have its tag bit unset, it isn't semantically
meaningful to talk about its length, as NULL is not a reference to a region
of memory. But programs can still attempt to query the length of NULL, and
the questions arises as to which value is returned.

We have chosen the length of NULL to be $2^{64}-1$, as this simplifies the
implementation of compressed capabilities. To support the semantics of the
C language, the capability compression scheme must be able to represent
all $2^{64}$ possible values of \coffset{} when \ctag{} is set and \clength{}
is MAXINT. If we make the length of NULL be MAXINT, the compressed capability
format can use the same encoding regardless of whether \ctag{} is set or
not:  NULL becomes a value whose \coffset{} is currently zero, but that can
be changed (with \insnref{CIncOffset}) to any integer value without
becoming unrepresentable.

Alternative design choices included:

\begin{itemize}
\item
Use a capability compression algorithm that also has the property that all
values of \coffset{} are representable when \clength{} is zero, and make
the length of NULL be zero. Versions of the CHERI ISA prior to V7 allowed the
length of NULL to be implementation-defined, and used a compression algorithm
that had this property, so the length of NULL could be zero. To enable the
use of compression algorithms that don't have this property, the V7 ISA
defines the length of NULL to be MAXINT.
\item
Use a different compression algorithm depending on whether \ctag{} is set
or not. This might make the hardware more complex, but there is no reason in
principle why valid capabilities (\ctag{} set) and integers packed into
capability registers (\ctag{} unset) should have to use the same compression
algorithm.
\end{itemize}

\section{Permission Bits Determine the Type of a Capability}

In CHERI, a capability's permission bits together with the \cotype{} field
determine what kind of capability it is. A capability for a region of memory
is unsealed (a \cotype{} of $2^{64}-1$) and has \emph{\cappermL{}} and/or \emph{\cappermS{}} set;
a capability for an object is sealed and has \emph{\cappermX{}}
unset; a capability to call a protected subsystem (a ``call gate'') is
sealed and has \emph{\cappermX{}} set; a capability that allows
the owner to create objects whose type identifier (\cotype{}) falls within
a range is unsealed and has \emph{\cappermSeal{}} set.

An alternative architecture would have included a separate
\emph{capability type} field, as well as the \cperms{} field, within each
capability; the meaning of the rest of the bits in the capability would have
been dependent on the value of the \emph{capability type} field.

A potential disadvantage of not having a \emph{capability type} field is that
different kinds of capability cannot use the remaining bits of the capability
in different ways.

A consequence of the architecture we have chosen is that it is possible for
software receiving the primordial, omnipotent capability to create capabilities
with arbitrary permissions.  Some of these sets of permissions do not have a
clear use case; they just exist as a consequence of the representation chosen
for capabilities' permissions.  (Other choices are possible; see
\cref{app:exp:compressperm} for a less-orthogonal representation.)

\mrnote{TO DO: Explain that capabilities with the Permit\_Seal capability
are really a different type of capability from memory capabilities, and
could in principle have used a different encoding to save bits. We don't
have a use case for a capability with both Permit\_Seal and read/write
permissions. If they were different types, you would need some mechanism to
obtain the initial sealing capability.}

\section{Object Types Are Not Addresses}

In CHERI, we make a distinction between the unique identifier for an
object type (the \cotype{} field) and the address of the executable code
that implements a method on the type (the \cbase{} $+$ \coffset{} fields
in a sealed executable capability).

An alternative architecture would have been to use the same fields for
both, and take the entry address of an object's methods as a convenient
unique identifier for the type itself.

The architecture we have chosen is conceptually simpler and easier to
explain. It has the disadvantage that the type field is constrained to
a limited number of bits, as there is insufficient space inside the
capability for more.

The alternative of treating the set of object type identifiers as being the
same as the set of memory addresses enables the saving of some bits within
a capability by using the same field for both.
It also simplifies
assigning type identifiers to protected subsystems: each subsystem can
use its start address as the unique identifier for the type it implements.
Subsystems that need to implement multiple types, or create new types
dynamically can be given a capability with the permission
\emph{Permit\_Set\_Type} set for a
range of memory addresses, and they are then able to use types within that
range. (The current CHERI ISA does not include the
\emph{Permit\_Set\_Type} permission;
it would be needed only for this alternative approach). This avoids the need
for some sort of privileged type manager that
creates new type identifiers; such a type manager is potentially a source
of covert channels. (Suppose that the type manager and allocated
type identifiers in numerically ascending order. A subsystem that asks the
type manager twice for a new type id and gets back $n$ and $n+1$ knows that no
other subsystem has asked for a new type id in between the two calls; this
could in principle be used for covert communication between two subsystems
that were supposed to be kept isolated by the capability mechanism.)

\section{Unseal is an Explicit Operation}

In CHERI, it requires an explicit operation to
convert an undereferenceable  pointer to an object into a pointer that
allows the object's contents to be inspected or modified directly.
This can be done directly with the \insnref{CUnseal} operation,
or by using \insnref{CInvoke} to run the result of unsealing the first
argument on the result of unsealing the second argument.

An alternative architecture would have been one with ``implicit'' unsealing,
where a sealed capability could be dereferenced without
explicitly unsealing it first, provided that the subsystem attempting the
dereference had some kind of ambient authority that permitted it to deference
sealed capabilities of that type. This ambient authority could have taken
the form of a protection ring or the \cotype{} field of \PCC{}.

A disadvantage of an implicit unseal approach such as the one outlined above
is that it is potentially vulnerable to the ``confused deputy''
problem~\cite{Hardy1988}: the attacker calls a protected subsystem, passing
a sealed capability in a parameter that the called subsystem expects to be
unsealed. If unsealing is implicit, the protected subsystem can be tricked
by the attacker into using its privileges to read or write to memory to
which the attacker does not have access.

The disadvantage of the architecture we have chosen is that protected subsystems
need to be careful not to leak capabilities that they have unsealed, for example
by leaving them on the stack when they return to their caller. In an
architecture with ``implicit unseal'', protected subsystems would just need
to delete their ambient authority for the type before returning, and would
not need to explicitly clean up all the unsealed capabilities that they
had created.

\section{CMove is not Implemented as CIncOffset}

\insnref{CMove} is an independent instruction to move a capability value
from one register to another.
In conventional instruction-set design, integer \insnnoref{Move} is
frequently an assembler pseudo-operation that expands to an arithmetic
operation that does not modify the value (e.g., an add instruction with the
zero register as one operand).
In an earlier CHERI design, we similarly implemented \insnref{CMove} is an
assembler pseudo-operation that expanded to \insnref{CIncOffset} with an
offset of zero.
This required that the \insnref{CIncOffset} instruction treat a zero
offset as a special case, allowing it to be used to move sealed capabilities
and values with the tag bit unset.
Using a separate opcode for \insnref{CMove} has the disadvantage of
consuming another opcode, but avoids this special case in the definition of
\insnref{CIncOffset} in which an exception will not be thrown if a zero
operand is used.
We have therefore changed to specifying an explicit \insnref{CMove}
instruction, and removed special casing in \insnref{CIncOffset}.

\section{Instruction-Set Randomization}

CHERI does not include features for instruction set
randomization~\cite{Keromytis2003};
the unforgeability of capabilities in CHERI can be used as an alternative
method of providing control flow integrity.

However, instruction set randomization would be easy to add, as long as
there are enough spare bits available inside a capability (the 128 bit
representation of capabilities does not have many spare bits). Code
capabilities could contain a key to be used for instruction set
randomization, and capability branches such as \insnref{CJR} could
change the current ISR key to the value given in the capability that is
branched to.

\section{System Privilege Permission}

In the current version of the CHERI, one of the capability permission bits
authorizes access to privileged processor features that would allow
bypass of the capability model, if present on \PCC{}.
This is intended to be used by hybrid operating-system kernels to manage
virtual address spaces, exception handling, interrupts, and other necessary
architectural features that do not map cleanly into memory-oriented
capabilities.
It can also be used by stand-alone CHERI-based microkernels to control use
of the exception-handling and cache-management mechanisms, and of the MMU on
MMU-enabled hardware.
Although the permission limits use of features to control the virtual address
space (e.g., MMU special register manipulation), it does not prevent access to kernel-only
portions of the virtual address space.
This allows kernel code to operate without privileged permission using the
capability mechanism to limit which portions of kernel address space are
available for use in constrained compartments.

We employ a single permission bit to conserve space,
but also because it offers a coherent view on architectural
privilege: many of the privileged architectural instructions allow bypass of
in-address-space memory protection in different ways, and using subsets of
those operations safely would be quite difficult.
In earlier versions of the CHERI ISA, we employed multiple privileged bits,
but did not find the differentiation useful in practical software design.
In more feature-rich privileged instruction sets (e.g., those with
virtualization features), a more fine-grained decomposition might be of
greater utility, and could motivate a new capability format intended to
authorize use of privilege.

In earlier versions, the privileged permission(s) controlled use of only CHERI-specific
privileges (i.e., exception-handling capabilities); in the current version, the
bit controls all privileges available only in kernel mode including
MMU registers and exception return instructions.
This allows compartmentalization within the kernel address space (e.g., to
sandbox untrustworthy components), as well as more general mitigation by
limiting use of privileged features to only selected code components, jumped
to via code pointers carrying the privileged permission.
If virtual-memory and exception-handling features were not controlled by this
permission bit, use of those ISA features would allow bypass of in-kernel
compartmentalization.
Regardless of this bit, extreme care is required to safely compartmentalize
within an operating-system kernel.

In our design, absence of the privileged permission denies use of privileged
ISA features, but presence does not grant that right unless it is also
authorized by kernel mode.
Other compositions of the capability permission bit and existing
ring-based authorization are imaginable.
For example, the permission bit could grant privileged ISA use in userspace
regardless of ring.
While this composition might allow potentially interesting delegation of
privilege to user components, the lack of granularity of control appears to
offer little benefit when a similar effective delegation can be implemented
via the exception model and implied ring transition.
In a ring-free design (e.g., one without an MMU or kernel/supervisor/user
modes), however, the privileged permission would be the sole means of
authorizing privilege.

Another design choice is that we have not added new
capability-based privilege instructions; instead, we chose to limit use of
existing instructions (such as those used in MMU management).
This fails to extend the principle of intentional use to these privileged
features; in return we achieve reduced disruption to current software
stacks, and avoid introducing new instructions in the opcode space.
Despite that slight apparent shortcoming,
we observe that fine-grained privilege can still be accomplished --
due to use of a permission bit
%%%% NO buts, please
on \PCC{}: even within a highly privileged
kernel, most functions might operate without the ability to employ privileged
instructions, with an explicit use of \insnref{CJALR} to jump to a code
pointer with the \cappermASR{} permission enabled -- which
executes
only the necessary instructions and reduces the window of opportunity for
privilege misuse.

An alternative design would extend the privileged instruction set to
include versions that accept explicit capability operands authorizing use of
those instructions, in a manner similar to our extensions to our
capability-extended load and store instructions.
Another variation on this scheme would authorize setting of a privilege status
register, enabling specific instructions (or classes of instructions) based on
an offered capability, combining these two approaches to authorize selected
(but unmodified) privileged instructions.

Finally, it is conceivable that capabilities could be used to authorize
delegation of the right to use privileged instructions to userspace code,
rather than simply restricting the right to use privileged instructions in
kernel code.
We have opted to limit our approach to using capabilities to restrict features,
with a simple and deterministic composition of features.

\section{CInvoke: Jump-Based Domain Transition}
\label{sec:jump-based-domain-transition}

Earlier versions of the CHERI-MIPS ISA included an exception-based
mechanism for domain transition via a pair of \insnnoref{CCall}
and \insnnoref{CReturn} instructions.  The use of exceptions
introduced both runtime overhead and implementation complexity in the
kernel.  We replaced this mechanism with \insnref{CInvoke}
which provides jump-like semantics.
Non-monotonicity is accomplished by virtue of unsealing the sealed
operand capabilities to \insnref{CInvoke}.
%In both cases, destination code is controlled by the trusted computing base.
% What does this mean?  CCalls1 jumps into an arbitrary sealed domain.
% Will it always be controlled by the trusted computing base?
% Maybe we're saying that it often is?  Anyway, commented out for now.

It is possible to imagine more comprehensive jump-based instructions
including:
\begin{itemize}
\item A variation that has link-register semantics, saving the caller \PCC{}
  in a manner similar to \insnref{CJALR}.
  We choose not to implement this to avoid writing two general-purpose registers
  in one instruction, and because the
  caller can itself perform a move to a link destination based on
  \insnref{AUIPCC}.

\item A variation that seals caller \PCC{} and \IDC{} to construct a
  return-capability pair.
  We choose not to implement this to avoid multiple register writes in one instruction,
  and because the
  caller can itself perform any necessary sealing of its own return state, if
  required.
  Further, to provide strict call-return semantics, additional more complex
  behavior is required, which is not well captured by a single RISC
  instruction.
\end{itemize}

In general, we anticipate that \insnref{CInvoke} will be used
to invoke trusted software routines.  For situations involving
mutual distrust, \insnref{CInvoke} can be used to invoke
a trusted supervisor responsible for mediating messages and
requests between distrusting parties.  The supervisor would be
responsible for clearing non-argument capability and general-purpose
integer registers and performing any additional checks.
The \insnref{CInvoke} trusted
routine can jump out of trusted code without
any special handling in the ISA, as it will conform to monotonic
semantics -- i.e., the clearing of registers that should not be passed to the
callee, followed by a \insnref{CJR} to transfer control to the callee.

\section{Compressed Capabilities}
\label{sec:rational:comressed}
In prior CHERI ISA versions, we specified a 256-bit capability
representation able to fully represent byte-granularity protection.
This allowed arbitrary subsets of the address space to be described, as well as
providing substantial
space for object types, software-defined permissions, and so on.
However, they come at a significant performance overhead: the size of 64-bit
pointers is quadrupled, increasing cache footprint and utilization of memory
bandwidth.
Fat-pointer compression techniques exploit information redundancy between the
base, pointer, and bounds to reduce the in-memory footprint of fat pointers,
reducing the precision of bounds -- with substantial space savings.
We now specify only compressed capabilities, whether 64-bit capabilities for
32-bit architectural addresses, or 128-bit capabilities for 64-bit
architectural addresses.
Prior versions of our compression approaches, the CHERI-128 candidates, are
described in \cref{app:cheri-128}.

\subsection{Semantic Goals for Compressed Capabilities}

Our target for compressed capabilities was 128 bits: the next natural
power-of-two pointer size above 64-bit pointers, with an expected one-third of
the overhead of the full 256-bit scheme.
A key design goal was to allow both 128-bit and 256-bit capabilities to be
used with the same instruction set, permitting us to maintain and evaluate
both approaches side-by-side.
To this end, and in keeping with previously published schemes, the CHERI ISA
continues to access fields such as permissions, pointer, base, and bounds via
64-bit general-purpose integer registers.
The only visible semantic changes between 256-bit and 128-bit operation should
be these:
the in-memory footprint when a capability register is loaded or stored,
the density of tags (doubled when the size of a capability is halved),
potential imprecision effects when adjusting bounds, potential loss of tag if
a pointer goes (substantially) out of bounds, a reduced number of permission
bits, a reduced object type space, and (should software inspect it) a change
in the in-memory format.

The scheme described in our specification is the result of substantial
iteration through designs attempting to find a set of semantics that support
both off-the-shelf C-language use, as well as providing strong protection.
Existing pointer-compression schemes generally provided suitable monotonicity
(pointer manipulation cannot lead to an expansion of bounds) and a completely
accurate underlying pointer, allowing base and bounds to experience
imprecision only during bounds adjustment.
However, they did not, for example, allow pointers to go ``out of bounds'' --
a key C-language compatibility requirement identified in our analysis of
widely used C programs.
The described model is based on a floating-point representation of distances
between the pointer and base/bounds, and places a particular focus on fully
precise representation bounds for small memory allocations % ($<\frac{3}{4}MiB$)
-- e.g., as occur on the stack, or when performing string or image processing.

\subsection{Precision Effects for Compressed Capabilities}

Precision effects are primarily visible during the narrowing of bounds on an
existing capability.
In order to provide the implementation with maximum flexibility in selecting a
compression strategy for a particular set of bounds, we have removed the
\insnnoref{CIncBase} and \insnnoref{CSetLen} instructions in favor of
a single \insnref{CSetBounds} instruction that exposes adjustments to
both atomically.
This allows the implementation to select the best possible parameters with
full information about the required bounds, maximizing precision.
Precision effects occur in the form of increased alignment requirements for
base and bounds: if requested bounds are highly unaligned, then the resulting
capability returned by \insnref{CSetBounds} may have broader rights than
requested, following stronger alignment rules.
\insnref{CSetBounds} maintains full monotonicity; however, bounds on a
returned capability will never be broader than those of the capability passed in.
Further, narrowing bounds is itself monotonic: as allocations become smaller,
the potential for precision increases due to the narrower range described.
Precision effects will generally be visible in two software circumstances:
memory allocation and arbitrary subsetting, which have different
requirements.

Memory allocation subdivides larger chunks of memory into smaller ones, which
are then delegated to consumers -- which most frequently are heap and stack
allocation, but this
can also occur when the operating system inserts new memory
mappings into an address space, returning a pointer (now a capability) to that
memory.
Memory allocators already impose alignment requirements: certainly for word or
pointer alignment so that allocated data structures can be stored at natural
alignment, but also (for larger allocations) for page or superpage alignment to
encourage effective use of virtual memory.
Compressed capabilities strengthen these alignment requirements for large
allocations, which requires modest changes to heap, stack, and OS memory
allocators in order to avoid exposing undesired precision effects.
Bounds on memory allocations will be set using \insnref{CSetBoundsExact},
which will throw an exception if precise bounds are not possible due to
precision effects.

Arbitrary subsetting occurs when programmers explicitly request that a
capability to an existing allocation be narrowed, in order to enforce bounds
checks linked to software invariants.
For example, an MPEG decoder might subset a larger memory buffer containing
many frames into individual frames when processing them, in order to catch
misbehavior without permitting (for example) corruption of adjacent frames.
Similarly, packet-processing systems frequently embed packet data within other
data structures; bugs in protocol parsing or packet construction could affect
packet metadata, with security consequences.
128-bit CHERI can provide precise subsetting for smaller subsets, % ($<1MiB$),
but may experience precision effects for larger subsets.
These are accepted in our programmer model, and could permit buffer overflows
between subsets, which would be prevented in the 256-bit model.
Unless specifically annotated to require full precision, arbitrary subsetting
will utilize \insnref{CSetBounds}, which can return monotonically
non-increasing -- but with potentially imprecise bounds.

Two further cases required careful consideration: object capabilities, and the
default data capability, for quite different reasons.
Object capabilities require additional capability fields (software-defined
permission bits, and the fairly wide object type field).
The default data capability is an ordinary 128-bit capability, but has the
property that use of a full cursor (base plus offset) introduces a further
arithmetic addition in a critical path of legacy loads and stores.
In both cases, we have turned to reduced precision (i.e., increased alignment
requirements) to eliminate these problems, looking to minimum page-granularity
alignment of bounds while retaining fully precise pointers.
By requiring strong alignment for default data capabilities, the extra
addition becomes a \texttt{logical or} when constructing the final virtual address,
assisting with the critical path.
As object capabilities are used only by newly implemented software, and
provide coarser-grained protection, we accepted the stronger alignment
requirement for sealed capabilities, and have not encountered significant
problems as a result.

The final way in which imprecision may be visible to software is if the
pointer (offset) in a capability goes substantially out of bounds.
In this case, the compression scheme may not be able to represent the
distances from the pointer to its original bounds accurately.
In this scenario, the tag will be cleared on the capability to prevent
dereference, and then one of the resulting pointer value or bounds must be
cleared due to the unrepresentability of the resulting value.
To discourage this from happening in the more common software case of allowing
small divergence from the bounds, \insnref{CSetBounds} over-provisions
bits required to represent the distances during compression; however, that
over-provisioning comes at a slight cost to precision: i.e., we accept slightly
stronger alignment requirements in return for the ability to allow pointers to
be somewhat out of bounds.

\subsection{The Value of Architectural Minimum Precision}
\label{sec:the-value-of-architectural-minimum-precision}

The \insnref{CRepresentableAlignmentMask} and
\insnref{CRoundRepresentableLength} instructions avoid encoding the details of
the capability compression scheme into programs.
This is in turn allows different microarchitectures to choose different
tradeoffs in the precision of compressed capabilities.
With no constraints on implementations, this may lead to unnecessary work
when performing operations like loading programs into memory.
For example, if linkers allow the generation of static binary load addresses
which are insufficiently aligned to be representable, then program loaders
must potentially pad the beginning of the mapped region, and enter a loop
that adjusts the base and length of the region until they match.
This could be avoided if an architectural minimum precision as specified so
that an OS ABI could forbid under-aligned load addresses and thus
image activators could simply refuse to load such programs rather than having
to handle every edge case.
Microarchitectures could still implement more precision if they choose and
that should be handled by \insnref{CRepresentableAlignmentMask} and
\insnref{CRoundRepresentableLength}.

\section{Capability Encoding Mode}
\label{section:capability-encoding-mode}

CHERI-MIPS duplicated the full load-store encoding
space to provide capability-relative variations on load and store
instructions.
This approach ensures intentionality: the architecture is always able to
perform a \DDC{}-constrained access with legacy integer-relative load and store
instructions, and is always able to assert that the tag bit is set for
capability-relative load and store instructions.
However, this makes heavy use of remaining unused opcode space in many
instruction sets, and so finding alternative encoding models to make less
copious use of opcode space is desirable.

CHERI-RISC-V instead uses legacy vs.
capability encoding modes: in the legacy encoding mode, load and store opcodes
have their current interpretations, and a small selection of
capability-relative loads and stores are added.
To get access to the full range of load and store variations, the encoding
mode can be switched to one in which existing load and store opcodes are
instead interpreted as requiring capability operands, and \DDC{}-constrained
integer-based access is disabled.

There are a variety of mechanisms that could be used to switch between
encoding modes, but information on the mode must be available at the time
of instruction decode.
There are several essential considerations:

\begin{description}
\item[How frequently will mode switches take place?]
  There are a range of possibilities, from whole programs or systems operating
  within a single encoding, to inter-function or sub-function changes in mode
  depending on ABI and optimization requirements.
  Given our overall goal in CHERI of avoiding the need for additional
  exceptions to a privileged supervisor for capability manipulation, we
  similarly believe that a non-exception-based encoding transition mechanism
  is desirable to support more tight integrations of integer-relative and
  capability-relative generated code.
  As such, the mechanisms we consider will generally support granular
  transition, at least at library boundaries or individual function call and
  return.

\item[How will encoding mode be selected and preserved across function calls?]
  Assuming that a more granular approach to encoding is desired -- e.g., that
  there are direct calls between code generated in differing modes -- then it
  will be necessary to switch to the callee encoding during function entry,
  and restore the caller encoding on function return.
  This might be supported implicitly through contextual information, such as
  using page-table properties, or explicitly such as through extended or
  entirely new compiler- or linker-managed instructions saving and setting
  encoding modes.

\item[How will encoding mode be preserved across context switches?]
  As with function-call boundaries, this might be implicit (e.g., based on the
  address or metadata held in \PCC{}, or via page-table metadata for the
  target that \PCC{} points to) or explicit
  (e.g., the saving and restoring of a bit in \xccsr{} when an exception is taken).

\item[What will the performance implications be for microarchitectural
  optimizations?]
  For example, will the target encoding be accurately predicted alongside the
  target \PCC{,} so that speculative execution can utilize the correct
  encoding?

\item[How should encoding-mode selection work around protection-domain
  boundary crossings?]
  When control is transferred across a protection-domain boundary (e.g.,
  by virtue of an exception being thrown, or use of \insnref{CInvoke}), the
  destination code must be able to ensure that it is being safely executed
  with its intended interpretation.
  This might be implied by the mechanism (e.g., by virtue of properties of the
  virtual page holding the executing code) or explicit (e.g., using
  dedicated instructions in the callee to switch modes, or assert the mode,
  before any affected instructions are executed).

\item[Should encoding-mode switches require privilege?]
  One potential fear is that an additional encoding mode increases the gadget
  space available to control-flow attackers.
  As long as the effect is only for the current execution context,
  we currently take the view that changing encoding modes
  does not require privilege: the set of available capabilities remains the
  same; the increase in gadget space is small; and attacks on control flow to
  use gadgets rely on having bypassed control-flow robustness arising from
  fine-grained code capabilities.
  See Section~\ref{section-capability-encoding-mode-unprivileged} for further
  considerations.
\end{description}

\subsubsection{Potential encoding mode-switch mechanisms}

We are considering the following mechanisms:
\rwnote{Must add notes on the safety concern to each of these.}

\begin{description}
\item[New jump instruction sets mode flag in \xccsr{}]
  An architecture mode bit, held in \xccsr{}, would select between the two
  different instruction encodings.
  A new jump instruction would allow the target mode to be selected via an
  immediate operand (``enter integer encoding mode'' or ``enter capability
  encoding mode'').
  This is a simple mechanism allowing dynamic selection of encoding at a fine
  granularity -- e.g., per function.  It utilizes existing context
  switching, as \xccsr{} will already be saved and restored.

  This approach has a number of complications from a software perspective:
  on function call, the caller must be aware of the callee encoding; on
  function return, the callee must likewise be aware of the caller encoding,
  so as to ensure that the correct encoding is used when control flow moves
  between functions.
  In some usage scenarios, such as dynamically linked libraries, this might
  require the introduction of thin stubs -- already present thanks to PLTs
  during call, but not presently implemented in current software stacks.
  Certain more complex control flows, such as those relating to exception
  delivery, might similarly present obstacles.

\item[Flag in jump-target addresses, maintained in \xccsr{}]
  Because of a minimum code alignment of 16 bits in RISC-V, the lowest bit in
  a jump target address is ignored (and cleared when installed in \PC{}),
  leaving it available as a potential flag to instructions such as
  \insnnoref{JALR} and \insnref{CJALR}\footnote{The \insnnoref{JAL} instruction shifts its immediate
  operand, and could not be used to change mode.}.
  This bit could be used to select the target ISA encoding, in the style of
  ARMv7's instruction-set trigger to switch between 32-bit instructions and
  16-bit Thumb instructions.
  \xccsr{} would contain an architectural mode bit to select between the two
  instruction codings.
  A lowest bit of 0 in the target virtual address would select integer
  encoding mode; a
  lowest bit of 1 in the target virtual address would select capability
  encoding mode.
  JALR and CJALR would similarly adjust the virtual address of a generated
  return address or capability, so as to restore the correct encoding on
  function return or exception delivery.
  This approach would avoid the need for any new instructions being
  introduced, and would associate the encoding with the callee rather than
  caller.
  Branch-predictor targets could also reliably predict encoding to allow
  speculative fetch and decode.

  Software would be relatively easily modified to set the bit as needed
  during compile-time or run-time linking.
  However, there may already be software consumers making use of the same
  bit.

\item[Flag in jump-target addresses, maintained in \PCC{}]
  As with the prior option, the lowest bit in the target virtual address for
  JALR or CJALR would select the target encoding.
  However, rather than extending \xccsr{}, the lowest bit would persist in
  \PCC{} and be ignored as an address for fetching instructions, allowing it
  to continue to indicate the target encoding.
  This would require a modest change to the baseline RISC-V ISA to preserve
  but ignore the bit in \PC{}.
  This approach avoids the need for a new \xccsr{} bit, and differently
  addresses the goal of allowing encoding to track executing code, and be
  saved, set, and restored around function calls and returns.

  As the bit would not be cleared, debuggers and other address-aware code,
  such as code implementing PC-relative GOT access in hybrid mode, would have
  to be suitably adapted to ignore the bit.
  It might be desirable to have the bit also ignored for the purposes of
  \insnnoref{AUIPC} used for GOT access.

\item[New capability flag to select the target encoding of a jump]
  A new capability flag could be introduced to select the target encoding for
  capability-relative jump targets (i.e., capabilities authorizing instruction
  fetch).
  Changing the flag would not change the rights associated with a capability,
  allowing us to avoid a new permission bit to authorize changing the flag,
  and sealing would prevent modification.
  Target encodings could be saved with corresponding branch-predictor entries
  to allow speculative fetch and decode.
  The encoding state would be preserved with \PCC{} on call, return, and in
  exception handling.

\item[Explicit unprivileged instruction to switch modes]
  New instructions could be added to switch explicitly between the two opcode
  instructions, to be placed either in function prologues/epilogues, or in
  trampolines inserted by static or dynamic linkage.
  Standard RISC-V \insnnoref{RWI}, \insnnoref{RSI}, and \insnnoref{RCI}
  CSR manipulation instructions could be used.
  Dynamic changes of encoding might necessitate invalidating speculative
  decoding and execution, however.

\item[Page-table flag specifying encoding for executable code]
  On MMU-enabled systems, page-table mappings for pages could themselves
  contain information on the encoding of instructions stored in the page.
  As binary pages are typically mapped by a run-time linker that is aware of
  code properties, this would avoid changes to code generation itself, use of
  new instructions, flags, etc.
  However, this would be dependent on having an MMU present, software authors
  using the MMU, as well as code having page alignment by encoding type.
  When running without virtual addressing enabled, it would not be possible to
  switch modes, which would be undesirable for small embedded-class systems.
\end{description}

%Of these potential schemes, requesting a target encoding based on address
%lower bits, and using \xccsr{} to hold the current encoding state, seems the
%most appealing, as it minimizes disruption to the ISA, and allows
%encoding-mode switches to be directed by the compile-time or run-time linker,
%which will already be aware of code requirements.
%If sub-function control of encoding is required, then direct manipulation of
%the \xccsr{} bits would also be possible.

Of these potential schemes, requesting a target encoding based on a PCC flag
seems the most appealing.

\pdrnote{We changed the concluding paragraph to match the conclusion we made
but may need to justify why the advantages of this approach won out.}

\section{Capability Encoding Mode Switching Can Be Unprivileged}
\label{section-capability-encoding-mode-unprivileged}

In CHERI-RISC-V, we introduce the concept that existing integer-relative load
and store opcodes could be reused in a richer ``capability encoding mode'',
conserving opcode space.
We argue
above
\pgnnote{DUPLICATED TEXT COVERED ABOVE.  Check back whether duped.
   search on `fear' and `gadget space'}
that switching between encodings is a safe operation to be performed
without privilege -- i.e., by arbitrary untrustworthy code -- as long as safe
mechanisms exist to switch to a predetermined encoding state when
transitioning across trust boundaries.
For example, it must be the case that exception handlers can operate reliably
in their intended encoding regardless of the encoding mode being used by
unprivileged user code triggering an exception.
Similarly, a reliable encoding switch must be achieved when using
\insnref{CInvoke}.

Our argument for safe unprivileged use is grounded in the belief that the
primary concern is one of potential code-reuse attacks, as switching encodings
does not change the set of capabilities available to executing code.
Instead, the fear is that an attacker able to manipulate control flow now has
access to an increased number of gadgets, as executable memory may now be used
with multiple interpretations.
We agree that the gadget space does modestly increase, and consider the
problem from two perspectives:

\begin{description}
\item[When the attack is against hybrid code:] The attacker may have the
  ability to influence an integer-based \PC{} value, and will gain access to
  additional gadgets (possibly doubling the gadget space).
  However, in hybrid code making only limited use of capabilities, CHERI is
  not intended to provide additional control-flow robustness.

\item[When the attack is against pure-capability code:] The attacker must
  first gain influence over a capability-based \PCC{} value, which will not
  only be protected against a number of common attacks (e.g., by virtue of
  tagged memory detecting data overwrites), but also will have narrowed
  bounds significantly limiting available gadget space.

  Further, a successful mode switch will have the sole impact of converting
  capability-relative loads and stores to integer-relative loads and stores
  against \DDC{}, which will hopefully be set to NULL when executing in a
  pure-capability code environment -- meaning that while the interpretation of
  instructions has changed, the impact of the newly accessible instructions
  will, by default, be an exception being thrown.

  Neither of these arguments precludes potentially effective manipulations of
  the run-time environment by the attacker, but many tools currently available
  to attackers that might benefit from a mode switch are entirely eliminated
  or significantly mitigated.
\end{description}

Overall, this leads us to the conclusion that unprivileged transition between
encodings is permissible.
However, significant care must be taken to ensure that when a privilege change
does occur, there is a safe mechanism by which exception handlers or
domain-transition mechanisms can execute only in the desired mode.

\pgnnote{  [LOTS OF DUP????]  }

\section{Loading Multiple Tags Without Corresponding Data} % <<<
\label{sec:rationale:cloadtags}

Occasionally, one may wish to have access to tags without, or before,
loading capabilities to registers.  This would be potentially useful when
paging to disk, for example, where one may wish to use DMA to transfer
memory contents to the disk, but yet one must separately store the
corresponding tags.  In the absence of direct (i.e., read) access to the
tags, the only alternative would be to involve the CPU in the bulk data copy
and \insnref{CLC} all of the memory to be paged.  Separately, when
sweeping memory for revocation or garbage collection, being able to skip
contiguous spans of non-capabilities in memory could dramatically reduce the
DRAM traffic involved in sweeping.

Towards these ends, we have introduced the \insnref{CLoadTags}
instruction, which takes a capability to memory and loads several tag bits
into a target register.  The least-significant bit corresponds to the tag
for the memory at the capability cursor; more significant bits correspond to
tags of memory at larger addresses.  The design of our cache fabric allows
us to instantiate this instruction with an efficient load of the tag bits
from one cache-line worth of memory, or, for CHERI Concentrate, 8 tags at
once.

Full details of the \insnref{CLoadTags} instruction may be found
on \cpageref{\insnlabelname{cloadtags}}.

\section{Attempted Montonicity Violations Clear Tags}
\label{sec:rationale:tag-clear-vs-exception}

To ensure pointer provenance, attempts to violate non-monotonicity, for example to broaden
(rather than narrow) bounds, must be forbidden. This can be achieved in several ways.
The instruction could throw a hardware exception, or generate a non-deferenceable pointer
as its output, in effect deferring the exception until the time of an attempted load,
store, or instruction fetch.
Both of these implementations ensure monotonicity by preventing derived
pointers from improperly allowing increased access following guarded
manipulation, and are consistent with the CHERI model.

Initially, in our prototyping, we selected to deliver exceptions as early as
possible when such events occur.
However, all current CHERI ISA instantiations defer exceptions to the use
of a capability's authority, instead clearing the tag on operations that
would otherwise violate monotonicity.

The early exception approach offers slightly improved debuggability
by exposing the error earlier.
Clearing the capability tag may make debugging more expensive (if additional checks are
introduced) or more tricky (if loss of the tag is only discovered substantially later).

However, early exceptions limit compiler optimization as instructions that may
throw exceptions are restricted in how they can safely be reordered.
For example, this prevents a bounds restriction performed within a loop from
being hoisted outside the loop, unless that instruction is always executed.
If the loop is not always entered, this could turn a conditional execution
of a trapping instruction into an unconditional one.%
\footnote{This is not just a theoretical possibility -- we observed this
 happening in the FreeBSD kernel and had to modify the compiler to avoid
 hoisting any potentially-trapping CHERI instructions.}
In addition, code that manipulates untrusted capabilities is forced to branch
when the operation would be illegal, or risk being vulnerable to
denial-of-service attacks.
This may require it to recreate the hardware-performed checks in software.

With a deferred-exception approach, as well as avoiding these issues,
microarchitecture is simplified by reducing the set of instructions that can
throw exceptions.
While it is initially tempting to delay performing the required checks,
forwarding the common-case value and later flushing the pipeline if a check
fails, this leads to exploitable speculative side channel attacks.
As such, in either approach, microarchitecture must perform the checks
before forwarding the result.

Early exceptions can still be achieved if desired for debugging by
instrumenting potentially tag-clearing instructions with assertions about
the tag, either manually or in a compiler sanitization pass.
The CHERI ISA instantiation can ensure these checks are cheap, for example by
providing an instruction to throw an exception based on the tag.

\section{\DDC{} and \PCC{} Offsetting}

Originally, CHERI always treated integer pointers used for legacy
memory accesses as offsets.  For example, loads and stores using an
integer pointer treated integer address as an offset relative to the
base of \DDC{}\footnote{Some CHERI instantiations performed offsetting
with respect to the address of DDC, rather than the base.}.
Similarly, branch instructions which targeted an
absolute integer pointer set the offset of \PCC{} to the value of the
integer pointer.

Offsetting also impacted CHERI C in multiple ways.
Casts of a capability to an integer value returned the offset of the
capability rather than its address.  Similarly, casts between
capability pointers and integer pointers used special instructions
(\insnnoref{CFromPtr} and \insnnoref{CToPtr}) which took the offset of
\DDC{} into account.  Specifically, the compiler would use
\insnnoref{CFromPtr} to generate integer pointers which were not an
absolute virtual address of an object, but the offset of an object's
address relative to the base of \DDC{}.  Similarly, capability
pointers created via casts were derived from \DDC{} assuming that the
integer pointer was an offset.

To provide consistent semantics with pointer casts, arithmetic
operations performed on integer values of capabilities (uintcap\_t)
used the offset of the capability as the scalar value.  For example,
to mask off the low bits of a pointer, the compiler fetched the
offset, applied the requested mask, and saved the result as the new
offset.

Finally, offsetting affected sub-language integer pointers.  Integer
function pointers, such as those stored in GOT entries, had to store
offsets relative to the base of \PCC{} rather than absolute addresses.
Similarly, pointers to data objects were stored as offsets relative to
\DDC{}.

As CHERI matured and developers gained more experience, several
caveats of this approach arose:

\begin{itemize}
\item Using offset interpretation for arithmetic operations on
  uintcap\_t broke several common idioms in pure-capability CHERI C
  (where uintptr\_t is the same as uintcap\_t).  Aligning pointers did
  not work reliably since the offset of a misaligned capability was
  still zero.  Code using the integer value of pointers as a key for
  hash tables would see far more collisions.  Due to these types of
  issues CHERI LLVM switched the default interpretation for arithmetic
  operations to work with addresses.

  However, this did result in inconsistent semantics compared to
  pointer casts.  In particular, converting between capabilities and
  integers can have different results if intermediate uintcap\_t
  values are used compared to direct casts.

\item Adding the base of \DDC{} to the effective address of legacy
  loads and stores can have a prohibitive cost in microarchitecture.

\item Tight bounds for \DDC{} and \PCC{} for hybrid code required that
  the hybrid code be relocatable and position independent.  For simple
  support of legacy 64-bit processes for which \DDC{} and \PCC{}
  bounds covered the entire user portion of the address space with a
  base address of 0 this did not matter.  However, this was a hurdle
  for hybrid operating system kernels which tended to run in a higher
  range of virtual addresses and were not always relocatable.  In
  practice hybrid kernels ran with \DDC{} and \PCC{} whose bounds
  spanned the entire address space.

\item The Morello architecture shipped with knobs to toggle the
  offsetting behavior of \DDC{} and \PCC{}.  When offsetting was
  disabled, \DDC{} and \PCC{} still constrained legacy memory accesses
  via bounds and permissions, but legacy integer pointers were
  interpreted as addresses rather than offsets.
\end{itemize}

CHERI no longer mandates \DDC{} and \PCC{} offsetting by default.
CHERI architectures may provide it as an optional feature which can be
enabled at runtime or may omit it entirely.  CHERI compilers always
treat integer pointers as addresses using
\insnref{CSetAddr} to handle conversions between capabilities and
integers.  The \insnref{CFromPtr} and \insnref{CToPtr} instructions
may be provided on architectures supporting offsetting.

% >>>
